\section{Prime Numbers and Prime Factorizations} \label{S:primefactorizations}
%\markboth{Chapter~\ref{C:numbertheory}. Topics in Number Theory}{\ref{S:primefactorizations}. 
%Prime Factorizations}
\setcounter{previewactivity}{0}
%
\begin{previewactivity}[\textbf{Exploring Examples where $\boldsymbol{a}$ Divides 
$\boldsymbol{b \cdot c}$}]\label{PA:exporingexamples} \hfill
\begin {enumerate}
\item Find at least three different examples of nonzero integers  $a$, $b$, and  $c$  such that  
$a \mid \left( {bc} \right)$ but  $a$  does not divide  $b$  and  $a$  does not divide  $c$.  In  each case, compute  $\gcd( {a, b} )$  and  $\gcd( {a, c} )$.
\label{PA:exporingexamples1}%

\item Find at least three different examples of nonzero integers  $a$, $b$, and  $c$  such that %\linebreak   
$\gcd( {a, b} ) = 1$  and  $a \mid ( {bc} )$.  In each example, is there any relation between the integers  $a$  and  $c$? 
\label{PA:exporingexamples2}%

\item Formulate a conjecture based on your work in Parts~(\ref{PA:exporingexamples1}) and~(\ref{PA:exporingexamples2}).

\end{enumerate}
%
\end{previewactivity}
\hbreak
%


\endinput

\begin{previewactivity}[\textbf{Prime Factorizations}]\label{PA:primefactorization25} \hfill \\
Recall that a natural number  $p$  is  a \textbf{prime number}
\index{prime number}%
\label{def:prime}%
 provided that it is greater than 1 and the only natural numbers that divide  $p$  are  1  and  $p$.  A natural number other than 1 that is not a prime number is a \textbf{composite number}.
\index{composite number}%
  The number 1 is neither prime nor composite.  (See Exercise~\ref{exer:prime} from Section~\ref{S:quantifier} on page~\pageref{exer:prime}.)

\begin{enumerate}
\item Give examples of four natural numbers that are prime and four natural numbers that are composite.
\end{enumerate}

Theorem~\ref{T:primefactors} in Section~\ref{S:otherinduction} states that every natural number greater than  1  is either a prime number or a product of prime numbers.

When a composite number is written as a product of prime numbers, we say that we have obtained a \textbf{prime factorization}
\index{prime factorization}%
\label{def:primefactorization}%
 of that composite number.  For example, since $60 = 2^2 \cdot 3 \cdot 5$, we say that $2^2 \cdot 3 \cdot 5$ is a prime factorization of 60.

\begin{enumerate}
\setcounter{enumi}{1}
\item Write the number  40  as a product of prime numbers by first writing  $40 = 2 \cdot 20$
 and then factoring  20  into a product of primes.  Next, write the number  40  as a product of prime numbers by first writing  $40 = 5 \mspace{1mu}\cdot\mspace{1mu} 8$  and then factoring  8  into a product of primes.  
\label{PA:primefactorization25-2}%

\item In Part~(\ref{PA:primefactorization25-2}), we used two different methods to obtain a prime factorization of  40.  Did these methods produce the same prime factorization or different prime factorizations?  Explain.  
\label{PA:primefactorization25-3}%

\item Repeat Parts~(\ref{PA:primefactorization25-2}) and~(\ref{PA:primefactorization25-3}) with 150.  First, start with  
$150 = 3 \cdot 50$, and then start with  
$150 = 5 \cdot 30$.
\end{enumerate}
\end{previewactivity}

\hbreak
%


\endinput

%
\subsection*{Greatest Common Divisors and Linear Combinations}
In Section~\ref{S:gcd}, we introduced the concept of the greatest common divisor of two integers.  We showed how the Euclidean Algorithm can be used to find the greatest common divisor of two integers, $a$  and  $b$, and also showed how to use the results of the Euclidean Algorithm to write the greatest common divisor of  $a$  and  $b$  as a linear combination of  $a$ and  $b$.

In this section, we will use these results to help prove the so-called Fundamental Theorem of Arithmetic, which states that any natural number greater than 1 that is not prime can be written as  product of primes in ``essentially'' only one way.  This means that given two prime factorizations, the prime factors are exactly the same, and the only difference may be in the order in which the prime factors are written.  We start with more results concerning greatest common divisors.  We first prove Proposition~\ref{P:divlinearcomb}, which was part of Exercise~(\ref{exer52-choose}) on page~\pageref{exer52-choose} in Section~\ref{S:provingset} and Exercise~(\ref{exer81:lincomb}) on page~\pageref{exer81:lincomb} in Section~\ref{S:gcd}.

\noindent
\textbf{Proposition \ref{P:divlinearcomb}}  \emph{Let a, b, and  t  be integers with $t \ne 0$.  If  t  divides  a  and  t  divides  b, then for all integers  x  and  y,  t  divides  
\text{(}ax + by\text{)}.}

\begin{myproof} Let $a$, $b$, and  $t$  be integers with $t \ne 0$, and assume that $t$  divides  $a$  and  $t$  divides  $b$.  We will prove that for all integers  $x$  and  $y$,  $t$  divides  $(ax + by)$.

So let  $x \in \mathbb{Z}$ and let  $y \in \mathbb{Z}$.  Since  $t$  divides  $a$, there exists an integer  $m$  such that $a = mt$ and since $t$ divides $b$, there exists an integer $n$ such that $b = nt$.  Using substitution and algebra, we then see that
\begin{align*}
ax + by &= (mt)x + (nt)y \\
        &= t(mx + ny)
\end{align*}
Since $(mx + ny)$ is an integer, the last equation proves that $t$ divides $ax + by$ and this proves that for all integers  $x$  and  $y$,  $t$  divides  $(ax + by)$.
\end{myproof}

We now let  $a, b \in \mathbb{Z}$, not both 0, and let  $d = \gcd( {a, b} )$.  Theorem~\ref{T:gcdaslincomb} states that  $d$  can be written as a linear combination of  $a$  and  $b$.  Now, since  $d \mid a$  and  $d \mid b$, we can use the result of Proposition~\ref{P:divlinearcomb} to conclude that for all  $x, y \in \mathbb{Z}$,  
$d \mid \left( {ax + by} \right)$.  This means that  $d$  divides every linear combination of  $a$  and  $b$.  In addition, this means that $d$ must be the smallest positive number that is a linear combination of $a$ and $b$.  We summarize these results in Theorem~\ref{T:gcddivideslincombs}.

\begin{theorem} \label{T:gcddivideslincombs}
Let  $a, b \in \mathbb{Z}$, not both 0.  %The greatest common divisor of $a$ and $b$ divides every linear combination of $a$ and $b$ and is the least positive integer that is a linear combination of $a$ and $b$.
\begin{enumerate}
\item The greatest common divisor,  $d$, is a linear combination of  $a$  and  $b$.  That is, there exist integers  $m$  and  $n$  such that  $d = am + bn$.

\item The greatest common divisor,  $d$,  divides every linear combination of  $a$  and  $b$.  That is, for all  $x, y \in \mathbb{Z}$\,,  $d \mid \left( {ax + by} \right)$.

\item The greatest common divisor, $d$, is the smallest positive number that is a linear combination of $a$ and $b$.
\end{enumerate}
\end{theorem}
%

\endinput

\subsection*{Relatively Prime Integers}
In \typeu Activity~\ref*{PA:exporingexamples}, we constructed several examples of integers  $a$, $b$, and  $c$  such that  $a \mid \left( {bc} \right)$ but  $a$  does not divide  $b$  and  $a$  does not divide  $c$.  For each example, we observed that  $\gcd( {a, b} ) \ne 1$
  and  $\gcd( {a, c} ) \ne 1$.  

We also constructed several examples where  $a \mid \left( {bc} \right)$ and  
$\gcd( {a, b} ) = 1$.  In all of these cases, we noted that  $a$  divides  $c$.  Integers whose greatest common divisor is equal to 1 are given a special name.

\begin{defbox}{relativelyprime}{Two nonzero integers  $a$  and  $b$  are \textbf{relatively prime}
\index{relatively prime integers}%
 provided that  $\gcd( {a, b} ) = 1$.}
\end{defbox}
%
\begin{prog}[\textbf{Relatively Prime Integers}] \label{prog:relativelyprime} \hfill
\begin{enumerate}
\item Construct at least three different examples where  $p$  is a prime number, $a \in \mathbb{Z}$\,, and  $p \mid a$.  In each example, what is  $\gcd( {a, p} )$?  Based on these examples, formulate a conjecture about $\gcd( {a, p} )$ when $p \mid a$.

\item Construct at least three different examples where  $p$  is a prime number, $a \in \mathbb{Z}$\,, and  $p$  does not divide  $a$.  In each example, what is  
$\gcd( {a, p} )$?  Based on these examples, formulate a conjecture about $\gcd( {a, p} )$ when 
$p$ does not divide $a$.

\item Give at least three different examples of integers  $a$  and  $b$  where  $a$  is not prime,  $b$  is not prime, and  $\gcd( {a, b} ) = 1$, or explain why it is not possible to construct such examples.
\end{enumerate}
\end{prog}
\hbreak
%
\begin{theorem} \label{T:relativelyprime}
Let  $a$  and  $b$  be nonzero integers, and let  $p$  be a  prime number.
\begin{enumerate}
\item If  $a$  and  $b$  are relatively prime, then there exist integers  $m$  and  $n$  such that  $am + bn = 1$.  That is,  1  can be written as linear combination of  $a$  and  $b$. \label{T:relativelyprime1}

\item If  $p \mid a$, then  $\gcd( {a, p} ) = p$. \label{P:relativelyprime2}

\item If  $p$  does not divide  $a$, then  $\gcd( {a, p} ) = 1$. \label{T:relativelyprime3}
\end{enumerate}
\end{theorem}

Part~(\ref{T:relativelyprime1}) of Theorem~\ref{T:relativelyprime} is actually a corollary of Theorem~\ref{T:gcddivideslincombs}.  Parts~(\ref{P:relativelyprime2}) and~(\ref{T:relativelyprime3}) could have been the conjectures you formulated in 
Progress Check~\ref{prog:relativelyprime}.  The proofs are included in Exercise~(\ref{exer:sec82-1}).
\hbreak
%
Given nonzero integers  $a$  and  $b$, we have seen that it is possible to use the Euclidean Algorithm to write their greatest common divisor as a linear combination of  $a$  and  $b$.  We have also seen that this can sometimes be a tedious, time-consuming process, which is why people have programmed computers to do this.  Fortunately, in many proofs of number theory results, we do not actually have to construct this linear combination since simply knowing that it exists can be useful in proving results.  This will be illustrated in the proof of Theorem~\ref{T:relativelyprimeprop}, which is based on work in \typeu Activity~\ref*{PA:exporingexamples}.

\begin{theorem}\label{T:relativelyprimeprop}
Let  $a$, $b$ be nonzero integers and let $c$ be an integer.  If  $a$  and  $b$  are relatively prime  and  $a \mid \left( {bc} \right)$, then  $a \mid c$.
\end{theorem}

\setcounter{equation}{0}
The explorations in \typeu Activity~\ref*{PA:exporingexamples} were related to this theorem.  
We will first explore the forward-backward process for the proof.
The goal is to prove that  $a \mid c$.  A standard way to do this is to prove that there exists an integer  $q$  such that
%
\begin{equation}\label{eq:relprime1}
c = aq.
\end{equation}
%
Since we are given that  $a \mid \left( {bc} \right)$, there exists an integer  $k$  such that
\begin{equation}\label{eq:relprime2}
bc = ak.
\end{equation}
It may seem tempting to divide both sides of equation~(\ref{eq:relprime2})  by  $b$, but if we do so, we run into problems with the fact that the integers are not closed under division.  Instead, we look at the other part of the hypothesis, which is that  $a$  and  $b$  are relatively prime.  This means that   $\gcd( {a, b} ) = 1$.  How can we use this?  This means that  $a$  and  $b$  have no common factors except for  1.  In light of equation~(\ref{eq:relprime2}), it seems reasonable that any factor of  $a$ must also be a factor of  $c$.  But how do we formalize this?

One conclusion that we can use is that since  $\gcd( {a, b} ) = 1$, by Theorem~\ref{T:relativelyprime}, there exist integers  $m$  and  $n$  such that
\begin{equation} \label{eq:relprime3}
am + bn = 1.
\end{equation}

We may consider solving equation~(\ref{eq:relprime3}) for  $b$  and substituting this into equation~(\ref{eq:relprime2}).  The problem, again, is that in order to solve equation~(\ref{eq:relprime3}) for  $b$, we need to divide by  $n$.  

Before doing anything else, we should look at the goal in equation~(\ref{eq:relprime1}).  We need to introduce  $c$  into equation~(\ref{eq:relprime3}).  One way to do this is to multiply both sides of equation~(\ref{eq:relprime3}) by  $c$.  (This keeps us in the system of integers since the integers are closed under multiplication.)  This gives
\begin{align} \notag
  \left( {am + bn} \right)c &= 1 \cdot c \\ \label{eq:relprime4}
                  acm + bcn &= c. \\ \notag
\end{align} 
Notice that the left side of equation~(\ref{eq:relprime4}) contains a term,  $bcn$, that contains  $bc$.  This means that we can use equation~(\ref{eq:relprime2}) and substitute  
$bc = ak$  in equation~(\ref{eq:relprime4}).  After doing this, we can factor the left side of the equation to prove that  $a \mid c$.
%
\hbreak
%
\begin{prog}[\textbf{Completing the Proof of Theorem~\ref{T:relativelyprimeprop}}] 
\label{prog:relativelyprimeprop} \hfill \\
Write a complete proof of Theorem~\ref{T:relativelyprimeprop}.
\end{prog}
\hbreak
%
\begin{corollary}\label{C:primedivides} \hfill
\begin{enumerate}
\item Let  $a, b \in \mathbb{Z}$, and let  $p$  be a prime number.  If  
$p \mid \left( {ab} \right)$, then  $p \mid a$  or  $p \mid b$.  \label{C:primedivides1}

\item Let  $p$  be a prime number, let  $n \in \mathbb{N}$, and let  
$a_1 ,a_2 , \ldots , a_n  \in \mathbb{Z}$.  If  \linebreak
$p \mid \left( {a_1 a_2  \cdots a_n } \right)$, then there exists a natural number $k$ 
 with  $1 \leq k \leq n$ such that  $p \mid a_k $.  \label{C:primedivides2}
\end{enumerate}
\end{corollary}

Part~(\ref{C:primedivides1}) of Corollary~\ref{C:primedivides} is a corollary of Theorem~\ref{T:relativelyprimeprop}.  Part~(\ref{C:primedivides2}) is proved using mathematical induction.  The basis step is the case where  $n = 1$, and Part~(\ref{C:primedivides1}) is the case where  
$n = 2$.  The proofs of these two results are included in Exercises~(\ref{exer:sec82-2}) 
and~(\ref{exer:sec82-3}).
%
\hbreak
%
\subsection*{Historical Note}

Part~(\ref{C:primedivides1}) of Corollary~\ref{C:primedivides} is known as  \textbf{Euclid's Lemma}.
\index{Euclid's Lemma}%
  Most people associate geometry with \textit{Euclid's Elements}, but these books  also contain many basic results in number theory.  Many of the results that are contained in this section appeared in \textit{Euclid's Elements}.
\index{Euclid's Elements@\emph{Euclid's Elements}}%
\hbreak

\endinput

\subsection*{Prime Numbers and Prime Factorizations}
We are now ready to prove the Fundamental Theorem of Arithmetic.  The first part of this theorem was proved in Theorem~\ref{T:primefactors} in Section~\ref{S:otherinduction}.  This theorem states that each natural number greater than 1 is either a prime number or is a product of prime numbers.  Before we state the Fundamental Theorem of Arithmetic, we will discuss some notational conventions that will help us with the proof.  We start with an example.

We will use  $n = 120$.  Since  $5 \mid 120$, we can write  $120 = 5 \cdot 24$.  In addition, we can factor 24 as  $24 = 2 \cdot 2 \cdot 2 \cdot 3$.  So we can write
\[
\begin{aligned}
  120 &= 5 \cdot 24 \\ 
      &= 5\left( {2 \cdot 2 \cdot 2 \cdot 3} \right). \\ 
\end{aligned} 
\]
This is a prime factorization of 120, but it is not the way we usually write this factorization.  Most often, we will write the prime number factors in ascending order.  So we write
\[
120 = 2 \cdot 2 \cdot 2 \cdot 3 \cdot 5\text{  or  }120 = 2^3  \cdot 3 \cdot 5.
\]
Now, let  $n \in \mathbb{N}$.  To write the prime factorization of  $n$ with the prime factors in ascending order requires that if we write  
$n = p_1 p_2  \cdots p_r $, where  $p_1 , p_2 ,  \ldots, p_r $ are prime numbers, we will have 
$p_1  \leq p_2  \leq  \cdots  \leq p_r $.
\hbreak
%
\begin{theorem}[\textbf{The Fundamental Theorem of Arithmetic}] \label{T:fundtheorem}
\index{Fundamental Theorem! of Arithmetic}%
 \hfill
\begin{enumerate}
\item Each natural number greater than 1 is either a prime number or is a product of prime numbers.

\item Let   $n \in \mathbb{N}$ with  $n > 1$.  Assume that
\[
n = p_1 p_2  \cdots p_r \text{  and that  }n = q_1 q_2  \cdots q_s,
\]
where  $p_1 , p_2 ,  \ldots, p_r $ and  $q_1 , q_2 ,  \ldots, q_s $ are primes with  
$p_1  \leq p_2  \leq  \cdots  \leq p_r $ and  
$q_1  \leq q_2  \leq  \cdots  \leq q_s $.  Then  $r = s$, and for each  $j$  from  1  to  r,  $p_j  = q_j $.
\end{enumerate}
\end{theorem}
%
\begin{myproof}
The first part of this theorem was proved in Theorem~\ref{T:primefactors}.  We will prove the second part of the theorem by induction on  $n$  using the Second Principle of Mathematical Induction.  (See Section~\ref{S:otherinduction}.)  For each natural number  $n$  with  $n > 1$,  let  $P( n )$ be

\begin{list}{}
\item If  $n = p_1 p_2  \cdots p_r$ and $n = q_1 q_2  \cdots q_s $, where  
$p_1 , p_2 ,  \ldots, p_r $ and  
$q_1 , q_2 ,  \ldots, q_s $ are primes with  
$p_1  \leq p_2  \leq  \cdots  \leq p_r $  and  
$q_1  \leq q_2  \leq  \cdots  \leq q_s $, then  $r = s$, and for each  $j$  from  1  to  $r$,  $p_j  = q_j $.
\end{list}

%\vskip9pt
For the basis step, we notice that since  2  is a  prime number, its only factorization is  
$2 = 1 \cdot 2$.  This means that the only equation of the form  $2 = p_1 p_2  \cdots p_r $, where  $p_1 , p_2 ,  \ldots, p_r $ are prime numbers, is the case where  $r = 1$ and  $p_1  = 2$.  This proves that  $P( 2 )$  is true.

%\vskip9pt
For the inductive step, let  $k \in \mathbb{N}$ with  $k \geq 2$.  We will assume that \linebreak
$P( 2 ), P( 3 ),  \ldots , P( k )$ are true.  The goal now is to prove that  $P( {k + 1} )$ is true.  To prove this, we assume that  $(k + 1)$ has two prime factorizations and then prove that these prime factorizations are the same.  So we assume that

\begin{list}{}
\item $k + 1 = p_1 p_2  \cdots p_r$ and that $k + 1 = q_1 q_2  \cdots q_s $, where  
$p_1 , p_2 ,  \ldots, p_r $ and  $q_1 , q_2 ,  \ldots, q_s $ are primes with  
$p_1  \leq p_2  \leq  \cdots  \leq p_r $  and  
$q_1  \leq q_2  \leq  \cdots  \leq q_s $.
\end{list}

\vskip6pt
\noindent
We must now prove that  $r = s$, and for each  $j$  from  1  to  $r$,  $p_j  = q_j $.  We can break our proof into two cases:  (1)  $p_1  \leq q_1 $; and (2) $q_1  \leq p_1 $.  Since one of these must be true, and since the proofs will be similar, we can assume, without loss of generality,  that  $p_1  \leq q_1 $.

Since  $k + 1 = p_1 p_2  \cdots p_r $, we know that  $p_1 \mid \left( {k + 1} \right)$, and hence we may conclude that $p_1 \mid \left( {q_1 q_2  \cdots q_s } \right)$.  We now use Corollary~\ref{C:primedivides} to conclude that there exists a  $j$  with  
$1 \leq j \leq s$   such that  $p_1 \mid q_j $.  Since  $p_1 $  and  $q_j $ are primes, we conclude that
\[
p_1  = q_j.
\]
We have also assumed that  $q_1  \leq q_j$ for all $j$ and, hence, we know that  
$q_1  \leq p_1 $.  However, we have also assumed that  $p_1  \leq q_1 $.  Hence,
\[
p_1  = q_1.
\]
We now use this and the fact that  $k + 1 = p_1 p_2  \cdots p_r  = q_1 q_2  \cdots q_s $ to conclude that
\[
p_2  \cdots p_r  = q_2  \cdots q_s.
\]
The product in the previous equation is less that  $k + 1$. Hence, we can apply our induction hypothesis to these factorizations and conclude that  $r = s$, and for each  $j$  from  2  to  $r$,  $p_j  = q_j $.

This completes the proof that if  
$P( 2 ), P( 3 ),  \ldots , P( k )$ are true, then 
$P( {k + 1} )$ is true.  Hence, by the Second Principle of Mathematical Induction, we conclude that  $P( n )$ is true for all  $n \in \mathbb{N}$ with  $n \geq 2$.  This completes the proof of the theorem.
\end{myproof}
%
\hbreak
%
\noindent
\note  We often shorten the result of the Fundamental Theorem of Arithmetic by simply saying that each natural number greater than one that is not a prime has a \textbf{unique factorization}
\index{unique factorization}%
 as a product of primes.  This simply means that if  $n \in \mathbb{N}$, $n > 1$, and  $n$  is not prime, then no matter how we choose to factor  $n$  into a product of primes, we will always have the same prime factors.  The only difference may be in the order in which we write the prime factors.


\endinput

\subsection*{Further Results and Conjectures about Prime Numbers}
\begin{enumerate}
  \item \textbf{The Number of Prime Numbers} \\
Prime numbers have fascinated mathematicians for centuries.  For example, we can easily start writing a list of prime numbers in ascending order.  Following is a list of the prime numbers less than 100.

\begin{list}{}
\item 2, 3, 5, 7, 11, 13, 17, 19, 23, 29, 31, 37, 41, 43, 47, 53, 59, 61, 67, 71, 73, 79, 83, 89, 97
\end{list}
\newpar
This list contains the first 25 prime numbers.  Does this list ever stop?  The question was answered in \emph{Euclid's Elements}, and the result is stated in 
Theorem~\ref{T:infiniteprimes}.  The proof of this theorem is considered to be one of the classical proofs by contradiction.
%
%\hbreak

\begin{theorem}\label{T:infiniteprimes}
There are infinitely many prime numbers.
\end{theorem}

\setcounter{equation}{0}
\begin{myproof}
We will use a proof by contradiction.  We assume that there are only finitely many primes, and let
\[
p_1 , p_2 ,  \ldots , p_m
\]
be the list of all the primes.  Let
\begin{equation} \label{T:infiniteprimes1}
M = p_1 p_2  \cdots p_m  + 1.
\end{equation}
Notice that  $M \ne 1$. So  $M$  is either a prime number or, by the Fundamental Theorem of Arithmetic,  $M$  is a product of prime numbers.  In either case,  $M$  has a factor that is a prime number.  Since we have listed all the prime numbers, this means that there exists a natural number  $j$  with  
$1 \leq j \leq m$  such that  $p_j \mid  M$. Now, we can rewrite equation~(\ref{T:infiniteprimes1}) as follows:
\begin{equation} \label{T:infiniteprimes2}
1 = M - p_1 p_2  \cdots p_m. 
\end{equation}
We have proved $p_j \mid M$, and since  $p_j $ is one of the prime factors of  
$p_1 p_2  \cdots p_m $, we can also conclude that   
$p_j  \mid \left( {p_1 p_2  \cdots p_m } \right)$.  Since   $p_j $  divides both of the terms on the right side of equation~(\ref{T:infiniteprimes2}), we can use this equation to conclude that   $p_j$ divides 1.  This is a contradiction since a prime number is greater than 1 and cannot divide 1.  Hence, our assumption that there are only finitely many primes is false, and so there must be infinitely many primes.
\end{myproof}

%\begin{activity}[Proof of Theorem~\ref{T:infiniteprimes}]\label{A:infiniteprimes} \hfill \\
%Complete the following proof of Theorem~\ref{T:infiniteprimes}.
%\setcounter{equation}{0}
%\begin{myproof}
%We will use a proof by contradiction.  We assume that there are only finitely many primes, and let
%\[
%p_1 , p_2 ,  \ldots , p_m
%\]
%be the list of all the primes.  Let
%\begin{equation}
%M = p_1 p_2  \cdots p_m  + 1.\label{eq:infiniteprimes1}
%\end{equation}
%\begin{enumerate}
%\item Explain why $M \ne 1$ and why $M$ must have a factor that is a prime number.
%
%\item Since we have listed all the prime numbers, there exists a natural number $j$ with 
%$1 \leq j \leq m$ such that $p_j$ divides $M$. Use equation~(\ref{eq:infiniteprimes1}) to conclude that $p_j$ divides 1.
%
%\item Explain why this is a contradiction and why this proves that there are infinitely many prime numbers. \qedhere
%\end{enumerate}
%\end{myproof}
%\end{activity}
%\hbreak



\item \textbf{The Distribution of Prime Numbers} 
\index{prime numbers!distribution of}%

There are infinitely many primes, but when we write a list of the prime numbers, we can see some long sequences of consecutive natural numbers that contain no prime numbers.  For example, there are no prime numbers between 113 and 127.  The following theorem shows that there exist arbitrarily long sequences of consecutive natural numbers containing no prime numbers.  A guided proof of this theorem is included in Exercise~(\ref{exer:consecutivecomposites}).

\begin{theorem}\label{P:consecutivecomposites}
For any natural number  $n$, there exist at least  $n$  consecutive natural numbers that are composite numbers.
\end{theorem}
\end{enumerate}
%
There are many unanswered questions about prime numbers, two of which will now be discussed.

\begin{enumerate}
\setcounter{enumi}{2}
\item \textbf{The Twin Prime Conjecture}

By looking at the list of the first 25 prime numbers, we see several cases where consecutive prime numbers differ by 2.  Examples are:  3  and  5;  11 and  13;  17 and  19; 29  and  31.  Such pairs of prime numbers are said to be \textbf{twin primes}.
\index{twin primes}%
\index{prime numbers!twin}%
  How many twin primes exist?  The answer is not known.  The \textbf{Twin Prime Conjecture}
\index{Twin Prime Conjecture}%
 states that there are infinitely many twin primes.  As of June 25, 2010, this is still a conjecture as it has not been proved or disproved.


For some interesting information on prime numbers, visit the Web site \textit{The Prime Pages} 
% (http://www.utm.edu/research/primes/), 
 (http://primes.utm.edu/), 
where there is a link to The Largest Known Primes Web site. 
According to information at this site  
%(http://www.utm.edu/research/primes/largest.html\#twin),
%(http://primes.utm.edu/largest.html\#twin),
 as of June 25, 2010, the largest known twin primes are
\[ 
\left( 65516468355 \times 2^{333333} - 1 \right) \text{ and } 
\left( 65516468355 \times 2^{333333} + 1 \right).
\]
%$\left( 33218925 \times 2^{169690} - 1 \right)$ and \linebreak
%$\left( 33218925 \times 2^{169690} + 1 \right)$.  
Each of these prime numbers contains 100355 digits. 

\item \textbf{Goldbach's Conjecture} \index{Goldbach's Conjecture}

Given an even natural number, is it possible to write it as a sum of two prime numbers?  For example,
\begin{align*}
4 &= 2 + 2  &   6 &= 3 + 3  &  8 &= 5+3 \\ 
78 &= 37 + 41 &  90 &= 43 + 47  &  138 &= 67 + 71 
\end{align*}
One of the most famous unsolved problems in mathematics is a conjecture made by Christian Goldbach in a letter to Leonhard Euler in 1742.  The conjecture, now known as \textbf{Goldbach's Conjecture}, is as follows:
\begin{list}{}
\item Every even integer greater than 2 can be expressed as the sum of two (not necessarily distinct) prime numbers.
\end{list}
As of June 25, 2010, it is not known if this conjecture is true or false, although most mathematicians believe it to be true.
\hbreak
\end{enumerate}

\endinput



%\hbreak
%

%


%

%\begin{activity}[Square Roots Are Irrational]\label{A:squarerootsirrational} \hfill \\
%In Chapter~\ref{C:proofs}, we proved that some square roots (such as $\sqrt{2}$ and 
%$\sqrt{3}$) are irrational numbers.  In this activity, we will use the Fundamental Theorem of Arithmetic to prove that if a natural number is not a perfect square, then its square root is an irrational number.
%
%\setcounter{equation}{0}
%\begin{enumerate}
%\item Let $n$ be a natural number.  Use the Fundamental Theorem of Arithmetic to explain why if $n$ is composite, then there exist prime numbers $p_1, p_2, \ldots, p_r$ and natural numbers 
%$\alpha_1, \alpha_2, \ldots, \alpha_r$ such that
%\begin{equation}\label{A:squarerootsirrational1}
%n = p_1^{\alpha_1} p_2^{\alpha_2} \cdots p_r^{\alpha_r}.
%\end{equation}
%Then, if we use $r = 1$ and $\alpha_1 = 1$ for a prime number, explain why we can write any natural number in the form given in equation~(\ref{A:squarerootsirrational1}).
%
%\item A natural number $b$ is a \textbf{perfect square}
%\index{perfect square}%
% if and only if there exists a natural number $a$ such that $b = a^2$.  Let $n$ be a natural number written in the form given in 
%equation~(\ref{A:squarerootsirrational1}).  Prove that $n$ is a perfect square if and only for each natural number $k$ with $1 \leq k \leq r$, $\alpha_k$ is even.
%
%\item Prove that for all natural numbers $n$, if $n$ is not a perfect square, then 
%$\sqrt{n}$ is an irrational number. 
%\end{enumerate}
%\end{activity}
%\hbreak


\endinput
