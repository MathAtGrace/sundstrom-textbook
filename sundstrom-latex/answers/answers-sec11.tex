\section*{Section \ref{S:prop}}
\begin{list}{\bf{\ref{exer:sec11-1}.}}
\item Sentences (a), (c), (e), (f), (j) and (k) are statements.  Sentence (h) is a statement if we are assuming that $n$ is a prime number means that  $n$ is a natural number.
\end{list}

\begin{list}{\bf{\ref{exer:sec11-2}.}} \item 
\begin{tabular}[t]{| c | p{1.5in} | p{1.5in} |} \hline
%  &  Hypothesis  &  Conclusion \\ \hline
%a.  &  $n$ is a prime number.  &  $n^2$ has three positive divisors. \\ \hline
%b.  &  $a$ is an irrational number and $b$ is an irrational number.  &  $a \cdot b$ is an irrational number. \\ \hline
%c.  &  $p$ is a prime number.  &  $p = 2$ or $p$ is an odd number.  \\ \hline
%d.  &  $p$ is a prime number and $p \ne 2$. & $p$ is an odd number. \\ \hline
%e.  &  $p \ne 2$ and $p$ is an even number &  $p$ is not prime.  \\ \hline
(a) & $x$ is a positive real number. & $\sqrt{x}$ is a positive real number.  \\  \hline
(b) & $\sqrt{x}$ is not a real number. & $x$ is a negative real number. \\ \hline
(c) & the lengths of the diagonals of a parallelogram are equal. & the parallelogram is a rectangle. \\ \hline
\end{tabular}
\end{list}

\begin{list}{\bf{\ref{exer:sec11-3}.}}  
\item Statements (a), (c), and (d) are true.
\end{list}

\begin{list}{\bf{\ref{exer:sec11-4}.}}
\item \textbf{(a)} True when $a \ne 3$.  \textbf{(b)} True when $a= 3$.
\end{list}

%\begin{list}{\bf{\ref{exer:sec11-5}.}}
%\item In Part (c), the instructor would have lied.  Part (b) corresponds to the first row of the truth table for $P \to Q$ , Part (c) corresponds to the second row, and Part (d) corresponds to the last two rows of this truth table.
%\end{list}

\begin{list}{\bf{\ref{exer:sec11-6}.}}
\item \begin{list}{\bf{(a)}}
\item This function has a maximum value when $x = \dfrac{5}{16}$.
\end{list}
\end{list}

\begin{list}{}
\item \begin{list}{\bf{(b)}}
\item The function $h$ has a maximum value when $x = \dfrac{9}{2}$.
\end{list}
\end{list}


\begin{list}{}
\item \begin{list}{\bf{(c)}}
\item No conclusion can be made about this function from this theorem.
\end{list}
\end{list}




%\begin{list}{\bf{\ref{exer:sec11-7}.}}
%\item \begin{list}{\bf{(a)}}
%\item No conclusion can be made about the function $g$ from this theorem.
%\end{list}
%\end{list}
%
%\begin{list}{}
%\item \begin{list}{\bf{(b)}}
%\item No conclusion can be made about the function $h$ from this theorem.
%\end{list}
%\end{list}
%
%\begin{list}{}
%\item \begin{list}{\bf{(c)}}
%\item The function $k$ has two $x$-intercepts.
%\end{list}
%\end{list}
%
%\begin{list}{}
%\item \begin{list}{\bf{(d)}}
%\item The function $j$ has two $x$-intercepts.
%\end{list}
%\end{list}
%
%\begin{list}{}
%\item \begin{list}{\bf{(e)}}
%\item The function $f$ has two $x$-intercepts.
%\end{list}
%\end{list}
%
%\begin{list}{}
%\item \begin{list}{\bf{(f)}}
%\item No conclusion can be made about the function $F$ from this theorem.
%\end{list}
%\end{list}



\begin{list}{\bf{\ref{exer:sec11-8}.}}
\item \begin{list}{\bf{(a)}}
\item The set of natural numbers is not closed under division.
\end{list}
\end{list}

\begin{list}{}
\item \begin{list}{\bf{(b)}}
\item The set of rational numbers is not closed under division since division by zero is not defined.
\end{list}
\end{list}

\begin{list}{}
\item \begin{list}{\bf{(c)}}
\item The set of nonzero rational numbers is closed under division.
\end{list}
\end{list}

\begin{list}{}
\item \begin{list}{\bf{(d)}}
\item The set of positive rational numbers is closed under division.
\end{list}
\end{list}

\begin{list}{}
\item \begin{list}{\bf{(e)}}
\item The set of positive real numbers is not closed under subtraction.
\end{list}
\end{list}


\begin{list}{}
\item \begin{list}{\bf{(f)}}
\item The set of negative rational numbers is not closed under division.
\end{list}
\end{list}

\begin{list}{}
\item \begin{list}{\bf{(g)}}
\item The set of negative integers is closed under addition.
\end{list}
\end{list}

%\hang
%\ref{exer:sec11-5}.  In Part (c), I would have lied.  Part (b) corresponds to the first row of %the truth table for $P \to Q$ , Part (c) corresponds to the second row, and Part (d) %corresponds to the last two rows of this truth table.
\hbreak
