\subsection*{Section \ref{S:gcd}}

\begin{list}{\bf{\ref{exer:sec81-1}.}}
\item \begin{enumerate}%\begin{multicols}{2}
\item The set of positive common divisors of 21 and 28 is $\{1, 7\}$.  So \\$\gcd ( {21, 28} ) = 7$.

\item The set of positive common divisors of $-21$ and 28 is $\{1, 7\}$.  So \\$\gcd ( { - 21, 28} ) = 7$.

\item The set of positive common divisors of 58 and 63 is $\{1 \}$.  So \\$\gcd ( {58, 63} ) = 1$.

\item The set of positive common divisors of 0 and 12 is $\{1, 2, 3, 4, 6, 12 \}$.  So $\gcd ( {0, 12} ) = 12$.
%\end{multicols}
\end{enumerate}
\end{list}


\begin{list}{\bf{\ref{exer:sec81-2}.}}
\item \begin{list}{\bf{(a)}}
\item \hint  Prove that $k \mid \left[ ( {a+1} ) - a \right]$.
\end{list}
\end{list}


\begin{list}{\bf{\ref{exer:sec81-props}.}}
\item \begin{list}{\bf{(a)}}
\item $|b|$ is the largest natural number that divides 0 and $b$.
\end{list}
\end{list}

\begin{list}{}
\item \begin{list}{\bf{(b)}}
\item The integers $b$ and $-b$ have the same divisors.  Therefore, 
$\gcd ( a, -b ) = \gcd ( a, b )$.
\end{list}
\end{list}


\begin{list}{\bf{\ref{exer:sec81-4}.}}
\item \begin{tabular}[t] {l l}
\textbf{(a)} $\gcd ( {36, 60} ) = 12$  & $12 = 36 \cdot 2 + 60 \cdot ( { - 1} )$ \\
\textbf{(b)} $\gcd ( {901, 935} ) = 17$	& $17 = 901 \cdot 27 + 935 \cdot ( { - 26} )$ \\
\textbf{(e)} $\gcd ( {901, -935} ) = 17$	& $17 = 901 \cdot 27 + (-935) \cdot ( { 26} )$ \\
\end{tabular}
\end{list}


\begin{list}{\bf{\ref{exer81:solvingeqn}.}}
\item \begin{list}{\bf{(a)}}
\item One possibility is $u = -3$ and $v = 2$.  In this case, $9u + 14v = 1$.  We then multiply both sides of this equation by 10 to obtain
\[
9 \cdot (-30) + 14 \cdot 20 = 10.
\]
So we can use $x = -30$ and $y = 20$.
\end{list}
\end{list}



\begin{multicols}{2}
\begin{list}{\bf{\ref{exer:gcdandfractions}.}}
\item \begin{list}{\bf{(a)}}
\item $11 \cdot (-3) + 17 \cdot 2 = 1$
\end{list}
\end{list}

\begin{list}{}
\item \begin{list}{\bf{(b)}}
\item $\dfrac{m}{11} + \dfrac{n}{17} = \dfrac{17m +11n}{187}$
\end{list}
\end{list}
\end{multicols}
\hbreak
\endinput


