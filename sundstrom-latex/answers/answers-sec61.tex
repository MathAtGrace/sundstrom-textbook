\section*{Section \ref{S:introfunctions}}
\renewcommand{\labelenumi}{(\textbf{\alph{enumi}})}


\begin{list}{\bf{\ref{exer:sec61-2}.}}
\item \begin{list}{\bf{(a)}}
\item $f(-3) = 15$, $f(-1) = 3$, $f(1) = -1$, $f(3) = 3$.
\end{list}
\end{list}


\begin{list}{}
\item \begin{list}{\bf{(b)}}
\item The set of preimages of  0  is $\left\{ 0, 2 \right\}$. The set of preimages of 4  is  
$\left\{ \dfrac{{2 - \sqrt {20} }}{2}, \dfrac{{2 + \sqrt {20} }}{2} \right\}$.  (Use the quadratic formula.)
\end{list}
\end{list}
%
%\begin{list}{}
%\item \begin{list}{\bf{(c)}}
%\item There are no preimages of  $-2$.
%\end{list}
%\end{list}
%
\begin{list}{}
\item \begin{list}{\bf{(d)}}
\item $\text{range} ( f ) = \left\{ y \in \mathbb{R} \mid y \geq  - 1 \right\}$
\end{list}
\end{list}

%\begin{list}{\ref{exer:sec61-1}.}
%\item Only (a) can be used to represent a function from  $A$  to  $B$.  Why?
%\end{list}

\begin{list}{\bf{\ref{exer61:integerfunction}.}}
\item \begin{enumerate}
\item 
$f \left( -7 \right) = 10$, 
$f \left( -3 \right) = 6$, 
$f \left( 3 \right) = 0$, 
$f \left( 7 \right) = -4$.

\item The set of preimages of  5 is  $\left\{ -2 \right\}$.  There set of preimages of 4 is $\{-1\}$.

\item $\text{range} \left( f \right) = \Z$.  Notice that for all $y \in \Z$ (codomain), $f(3 - y) = y$ and $(3 - y) \in \Z$ (domain).
\end{enumerate}
\end{list}

\vskip9pt
\begin{list}{\bf{\ref{exer:sec61-5}.}}
\item \begin{list}{\bf{(b)}}
\item The set of preimages of  5 is  $\left\{ 2 \right\}$.  There set of preimages of 4 is 
$\emptyset$.
\end{list}
\end{list}
%
\begin{list}{}
\item \begin{list}{\bf{(c)}}
\item The range of the function  $f$  is the set of all odd integers.
\end{list}
\end{list}
%
\begin{list}{}
\item \begin{list}{\bf{(d)}}
\item The graph of the function  $f$  consists of an infinite set of discrete points.
\end{list}
\end{list}


\begin{list}{\bf{\ref{exer:sec61-9}.}}
\item \begin{list}{\bf{(b)}}
\item $\text{dom} ( F ) = \left\{ {x \in \mathbb{R} \mid x > \dfrac{1}{2}} \right\}$,  
$\text{range} ( F ) = \mathbb{R}$

\end{list}
\end{list}


\begin{list}{}
\item \begin{list}{\bf{(d)}}
\item $\text{dom} ( g ) = \left\{ {x \in \mathbb{R} \mid x \ne 2\text{  and  }x \ne  - 2} \right\}$, \\
$\text{range} ( g ) = \left\{ { {y \in \mathbb{R} } \mid y > 0} \right\} \cup 
\left\{ {y \in \mathbb{R} \mid y \leq  - 1} \right\}$
\end{list}
\end{list}




\begin{list}{\bf{\ref{exer:numberofdivisors}.}}
\item \begin{enumerate}
\item $d ( 1 ) = 1$, $d ( 2 ) = 2$, $d ( 3 ) = 2$, $d ( 4 ) = 3$, $d(5) = 2$, $d(6) = 4$, $d(7) = 2$, $d ( 8 ) = 4$, $d ( 9 ) = 3$, $d(10) = 4$, $d(11) = 2$, $d(12) = 6$.

\item There is no natural number $n$ other than 1 such that $d ( n ) = 1$ since every natural number greater than one has at least two divisors.  The set of preimages of 1 is $\{1 \}$.

\item The only natural numbers  $n$  such that $d( n ) = 2$   are the prime numbers. The set of preimages of the natural number  2 is the set of prime numbers.

\item The statement is false.  A counterexample is $m = 2$ and $n = 3$ since 
$d( 2 ) = 2$ and $d( 3 ) = 2$.

\item $d \left( 2^0 \right) = 1$, $d \left( 2^1 \right) = 2$, $d \left( 2^2 \right) = 3$, 
$d \left( 2^3 \right) = 4$, $d \left( 2^4 \right) = 5$, \\
$d \left( 2^5 \right) = 6$, and $d \left( 2^6 \right) = 7$.

\item For each nonnegative integer $n$, the divisors of $2^n$ are $2^0, 2^1, 2^2, \ldots, 2^{n-1}$, and $2^n$.  This is a list of $n+1$ natural numbers and so $d(2^n) = n+1$.

%Let $P \left( n \right)$ be, ``$d \left( 2^n \right) = n + 1$.''  $P \left( 0 \right)$ is true.  Let $k \in \mathbb{Z}$ with $k \geq 0$ and assume that $P \left( k \right)$ is true.  Then,
%\begin {center}
%$d \left( 2^k \right) = k + 1$.
%\end{center}
%This means that $2^k$ has $k + 1$ divisors.  Now, any divisor of $2^k$ is also a divisor of 
%$2^{k+1}$.  The only other divisor of $2^{k+1}$ is $2^{k+1}$.  Thus,
%\[
%\begin{aligned}
%d \left( 2^{k + 1} \right)&= ( k + 1 ) + 1 \\
%                      &= k + 2.
%\end{aligned}
%\]
%This proves that if $P( k )$ is true, then $P( k + 1 )$ is true.
\end{enumerate}
%\item \begin{list}{\bf{(a)}}
%\item $d ( 1 ) = 1$, $d ( 2 ) = 2$, $d ( 3 ) = 2$, 
%$d ( 4 ) = 3$, $d ( 8 ) = 4$, $d ( 9 ) = 3$
%\end{list}
%\end{list}
%%
%\begin{list}{}
%\item \begin{list}{\bf{(c)}}
%\item The only natural numbers  $n$  such that $d ( n ) = 2$   are the prime numbers. The set of preimages of the natural number  2 is the set of prime numbers.
%\end{list}
%\end{list}
%%
%\begin{list}{}
%\item \begin{list}{\bf{(e)}}
%\item $d ( 2^0 ) = 1$, $d ( 2^1 ) = 2$, $d ( 2^2 ) = 3$, 
%$d ( 2^3 ) = 4$
%\end{list}
\end{list}
%
\begin{list}{}
\item \begin{list}{\bf{(g)}}
%\item The idea for the inductive step is that the divisors of $2^{k+1}$   are  $2^{k+1}$  and the divisors of $2^k$.
\item The statement is true.  To prove this, let $n$ be a natural number.  Then $2^{n-1} \in \N$ and 
$d \left( 2^{n-1} \right) = (n - 1) + 1 = n$.
\end{list}
\end{list}


\begin{list}{\bf{\ref{exer:sec62-5}.}}
\item \begin{list}{\bf{(a)}}
\item The domain of $S$ is $\N$.  The power set of $\N$, $\mathcal{P} ( \N )$,  can be the codomain.  The rule for determining outputs is that for each $n \in \N$, 
$S(n)$ is the set of all distinct natural number factors of $n$.
\end{list}
\end{list}


\begin{list}{}
\item \begin{list}{\bf{(b)}}
\item For example,  $S( 8 ) = \left\{ {1, 2, 4, 8} \right\}$, 
$S( {15} ) = \left\{ {1, 3, 5, 15} \right\}$.
\end{list}
\end{list}
%
\begin{list}{}
\item \begin{list}{\bf{(c)}}
\item For example, $S( 2 ) = \left\{ {1, 2} \right\}$, 
$S( 3 ) = \left\{ {1, 3} \right\}$, 
$S( {31} ) = \left\{ {1, 31} \right\}$.
\end{list}
\end{list}

%

\hbreak
\renewcommand{\labelenumi}{\textbf{\arabic{enumi}.}}

\endinput


