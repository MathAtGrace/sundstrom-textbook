\section*{Section \ref{S:divalgo}}
\renewcommand{\labelenumi}{(\textbf{\alph{enumi}})}


\begin{list}{\bf{\ref{exer:sec35-2}.}}
\item \begin{enumerate}
\item The first case is when $\mod{n}{0}{3}$.  We can then use Theorem~\ref{T:propsofcong} to conclude that $\mod{n^3}{0^3}{3}$ or that $\mod{n^3}{0}{3}$.  So in this case, $\mod{n^3}{n}{3}$.

\noindent
For the second case, $\mod{n}{1}{3}$.  We can then use Theorem~\ref{T:propsofcong} to conclude that $\mod{n^3}{1^3}{3}$ or that $\mod{n^3}{1}{3}$.  So in this case, $\mod{n^3}{n}{3}$.

\noindent
The last case is when $\mod{n}{2}{3}$.  We then get $\mod{n^3}{2^3}{3}$ or $\mod{n^3}{8}{3}$.  Since $\mod{8}{2}{3}$, we can use the transitive property to conclude that $\mod{n^3}{2}{3}$, and so $\mod{n^3}{n}{3}$.

\noindent
Since we have proved it in all three cases, we conclude that for each integer $n$, $\mod{n^3}{n}{3}$.

\item Since $\mod{n^3}{n}{3}$, we use the definition of congruence to conclude that 3 divides $\left( n^3 - n \right)$.
\end{enumerate}
\end{list}



\begin{list}{\bf{\ref{exer:cong-symm}.}}
\item Let  $n \in \mathbb{N}$.  For  $a, b \in \mathbb{Z}$, you need to prove that if  
$a \equiv b \pmod n$,  then  $b \equiv a \pmod n$.  So let  $a, b \in \mathbb{Z}$ and assume that 
$a \equiv b \pmod n$.  So $n \mid \left( a - b \right)$ and there exists an integer $k$ such that 
$a - b = nk$.  Then, $b - a = n ( -k )$ and $b \equiv a \pmod n$.
\end{list}


\begin{list}{\bf{\ref{exer:sec35-a2mod3}.}}
\item \begin{enumerate}
\item The contrapositive is:  For each integer $a$, if 3 does not divide $a$, then 3 divides $a^2$.
\item To prove the contrapositive, let $a \in \Z$ and assume that 3 does not divide $a$.  So using the Division Algorithm, we can consider two cases:  (1) There exists a unique integer $q$ such that $a = 3q + 1$.  (2) There exists a unique integer $q$ such that $a = 3q + 2$.

For the first case, show that $a^2 = 3 \left( 3q^2 + 2q \right) + 1$.  For the second case, show that $a^2 = 3 \left( 3q^2 + 4q + 1 \right) + 1$.  Since the Division Algorithm states that the remainder is unique, this shows that in both cases, the remainder is 1 and so 3 does not divide $a^2$.
\end{enumerate}
\end{list}






\begin{list}{\bf{\ref{exer:sec34-4}.}}
\item \begin{enumerate}
\item $a \equiv 0 \pmod n$ if and only if $n \mid \left( a - 0 \right)$.

\item Let  $a \in \mathbb{Z}$.  Corollary~\ref{C:congtorem} tell us that  if  $\notmod{a}{0}{3}$, then  \\
$a \equiv 1 \pmod 3$ or  $a \equiv 2 \pmod 3$.

\item Part (b) tells us we can use a proof by cases using the following two cases:  
(1) $a \equiv 1 \pmod 3$;  (2) $a \equiv 2 \pmod 3$.

So, if $a \equiv 1 \pmod 3$, then by Theorem~\ref{T:propsofcong}, 
$a \cdot a \equiv 1 \cdot 1 \pmod 3$, and hence, $a^2 \equiv 1 \pmod 3$.

If $a \equiv 2 \pmod 3$, then by Theorem~\ref{T:propsofcong}, 
$a \cdot a \equiv 2 \cdot 2 \pmod 3$, and hence, $a^2 \equiv 4 \pmod 3$.  Since 
$4 \equiv 1 \pmod 3$, this implies that $a^2 \equiv 1 \pmod 3$.
\end{enumerate}

%\item \begin{list}{\bf{(a)}}
%\item Use the definition of congruence.
%\end{list}
%\end{list}
%%
%\begin{list}{}
%\item \begin{list}{\bf{(b)}}
%\item Let  $a \in \mathbb{Z}$.  Corollary~\ref{C:congtorem} tell us that  if  
%$a \not \equiv 0 \pmod 3$, then  $a \equiv 1 \pmod 3$ or  
%$a \equiv 2 \pmod 3$.
%\end{list}
%\end{list}
%%
%\begin{list}{}
%\item \begin{list}{\bf{(c)}}
%\item For one of the conditional statements, Part (b) tells us we can use a proof by cases using the following two cases:  
%(1) $a \equiv 1 \pmod 3$;  (2) $a \equiv 2 \pmod 3$.
%\end{list}
\end{list}


\begin{list}{\bf{\ref{exer:congto3}.}}
\item The contrapositive is:  Let $a$ and $b$ be integers.  If $a \not\equiv 0 \pmod3$ and 
$b \not\equiv 0 \pmod 3$, then $ab \not\equiv 0 \pmod 3$.

Using Exercise~\ref{exer:sec34-4}(b), we can use the following four cases: \\ 
(1) $a \equiv 1 \pmod 3$ and $b \equiv 1 \pmod 3$;  \\ 
(2) $a \equiv 1 \pmod 3$ and $b \equiv 2 \pmod 3$; \\
(3) $a \equiv 2 \pmod 3$ and $b \equiv 1 \pmod 3$;  \\ 
(4) $a \equiv 2 \pmod 3$ and $b \equiv 2 \pmod 3$. \\
In all four cases, we use Theorem~\ref{T:propsofcong} to conclude that 
$ab \not\equiv 0 \pmod 3$.  For example, for the third case, we see that $\mod{ab}{2 \cdot 1}{3}$.  That is, $\mod{ab}{2}{3}$.
\end{list}


\begin{list}{\bf{\ref{exer:3divprod}.}}
\item \begin{enumerate}
\item This follows from Exercise~(5) and the fact that  $3 \mid k$ if and only if  
$k \equiv 0 \pmod 3$.
\item This follows directly from Part~(a) using $a = b$.
\end{enumerate}
%\item \begin{list}{\bf{(a)}}
%\item Remember that  $3 \mid k$ if and only if  $k \equiv 0 \pmod 3$.  
%%This proposition is a special case of the proposition in Exercise~(\ref{exer:dividesproductc}) with  $d = 3$.  The general result in Exercise~(\ref{exer:dividesproductc}) is false, but this special case is true.
%\end{list}
\end{list}


\begin{list}{\bf{\ref{exer:sqrt3}.}}
\item \begin{list}{\bf{(a)}}
\item Use a proof similar to the proof of Theorem~\ref{T:squareroot2}.  The result of Exercise~(\ref{exer:3divprod}) may be helpful.
\end{list}
\end{list}



\begin{list}{\bf{\ref{exer:notperfectsquare}.}}
\item The result in Part (c) of Exercise~(\ref{exer:sec34-4}) may be helpful in a proof by contradiction.
\end{list}






\begin{list}{\bf{\ref{exer:sec34-2}.}}
\item \begin{list}{\bf{(b)}}
\item Factor $n^3 - n$.
\end{list}
\end{list}


\begin{list}{}
\item \begin{list}{\bf{(c)}}
\item Consider using cases based on congruence modulo 6.
\end{list}
\end{list}

%\begin{list}{\bf{\ref{exer:remainderbycong}.}}
%\item \begin{list}{\bf{(a)}}
%\item Use the results in Theorem~\ref{T:propsofcong} to prove that the remainder must be 1.
%\end{list}
%\end{list} 


%\begin{list}{\bf{\ref{exer:sec34-8}.}}
%\item \begin{list}{\bf{(a)}}
%\item 21 \qquad \qquad \textbf{(b)} 43
%\end{list}
%\end{list}

\hbreak
\renewcommand{\labelenumi}{\textbf{\arabic{enumi}.}}
\endinput

