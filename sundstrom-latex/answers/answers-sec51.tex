\section*{Section \ref{S:setoperations}}
\renewcommand{\labelenumi}{(\textbf{\alph{enumi}})}

\begin{list}{\bf{\ref{exer:sec41-1}.}}
\item \begin{tabular}[t]{p{1.2in} p{1.2in} p{1.2in}}
\textbf{(a)} $A = B$  &  \textbf{(c)} $C \ne D$  &  \textbf{(e)} $A \not \subseteq D$ \\
\textbf{(b)} $A \subseteq B$  &  \textbf{(d)} $C \subseteq D$ &  \\
\end{tabular}
\end{list}


\begin{list}{\bf{\ref{exer41-equalsets}.}}
\item In both cases, the two sets have preceisely the same elements.
\end{list}


\begin{list}{\bf{\ref{exer:sec41-3}.}}
\item \begin{tabular}[t]{r c l c r c l }
$A$ & $\subset, \subseteq , \ne$ & $B$ &  &  $\emptyset$ & $\subset , \subseteq , \ne$ & $A$  \\
5   & $\in$ & $C$ & & $\left\{ 5 \right\}$ & $\subset , \subseteq , \ne$ & $C$ \\
$A$ & $\subset, \subseteq , \ne$ & $C$ &  &  $\left\{ 1,2 \right\}$ & $\subset, \subseteq , \ne$ & $B$ \\
$\left\{ 1,2 \right\}$ & $\not\subseteq$, $\ne$ & $A$ &  &  $\left\{ 3,2,1 \right\}$ & $\subset, \subseteq , \ne$ & $D$ \\
4  &  $\notin$ &  $B$ &  &  $D$ &  $\not\subseteq$, $\ne$ &  $\emptyset$ \\
$\card(A)$ &  $=$ & $\card(D)$ & & $\card(A)$ & $\ne$ & $\card(B)$ \\
$A$ & $\in$ &  $\mathcal{P} \left( A \right)$ & & $A$ & $\in$ &  $\mathcal{P} \left( B \right)$ \\
\end{tabular}
\end{list}



\begin{list}{\bf{\ref{exer:sec51-numbers}.}}
\item \begin{tabular}[t]{l l l}
$\mathbb{N} \subset \mathbb{Z}$  &  $\mathbb{N} \subset \mathbb{Q}$ & 
$\mathbb{N} \subset \mathbb{R}$   \\
$\mathbb{Z} \subset \mathbb{Q}$  &  $\mathbb{Z} \subset \mathbb{R}$   & $\mathbb{Q} \subset \mathbb{R}$ \\
\end{tabular}
\end{list}


\begin{list}{\bf{\ref{exer:sec41-5}.}}
\item \begin{enumerate}
\item The set $\left\{ {a, b} \right\}$ is a not a subset of  $\left\{ {a, c, d, e} \right\}$
 since  $b \in \left\{ {a, b} \right\}$ and  $b \notin \left\{ {a, c, d, e} \right\}$.

\item $\left\{ { - 2,0,2} \right\} = \left\{ {x \in \mathbb{Z} \mid  
x\text{ is even  and  }x^2  < 5} \right\}$ since both sets have precisely the same elements.

\item $\emptyset \subseteq \left\{ 1 \right\}$ since the following statement is true:
For every $x \in U$, if $x \in \emptyset$, then $x \in \left\{ 1 \right\}$.

\item The statement is false.  The set $\left\{ a \right\}$ is an element of 
$\mathcal{P} \left( A \right)$.
\end{enumerate}
\end{list}

\begin{list}{\bf{\ref{exer:sec41-8}.}}
\item \begin{enumerate}
\item $x \notin A \cap B$ if and only if $x \notin A$ or $x \notin B$.
\end{enumerate}
\end{list}


\begin{list}{\bf{\ref{exer:sec41-6}.}}
\item 
\begin{enumerate}
  \begin{multicols}{2}
  \item $A \cap B = \left\{ 5, 7 \right\}$
  \item $A \cup B = \left\{ 1, 3, 4, 5, 6, 7, 9 \right\}$
  \item $\left( {A \cup B} \right)^c = \left\{ 2, 8, 10 \right\}$
  \item $A^c  \cap B^c = \left\{ 2, 8, 10 \right\}$
  \item $( {A \cup B} ) \cap C = \left\{ {3, 6, 9} \right\}$
  \item $A \cap C = \left\{ 3, 6 \right\}$
  \item $B \cap C = \left\{ 9 \right\}$
  \item $( {A \cap C} ) \cup ( {B \cap C} ) = \left\{ 3, 6, 9 \right\}$
  \item $B \cap D = \emptyset$
  \item $\left( {B \cap D} \right)^c = U$
  \item $A - D = \left\{ 3, 5, 7 \right\}$
  \item $B - D = \left\{ 1, 5, 7, 9 \right\}$
\end{multicols}
\end{enumerate}
\end{list}

\begin{list}{}
\item \begin{list}{\textbf{(m)}}
\item $\left( {A - D} \right) \cup \left( {B - D} \right) = \left\{ 1, 3, 5, 7, 9 \right\}$
\end{list}
\end{list}

\begin{list}{}
\item \begin{list}{\textbf{(n)}}
\item $\left( {A \cup B} \right) - D = \left\{ 1, 3, 5, 7, 9 \right\}$
\end{list}
\end{list}



\begin{list}{\bf{\ref{exer:sec41-4sets}.}}
\item \begin{list}{\bf{(b)}}
\item There exists an $x \in U$ such that $x \in (P - Q)$ and 
$x \notin (R \cap S)$.  This can be written as,  There exists an $x \in U$ such that 
$x \in P$, $x \notin Q$, and $x \notin R$ or $x \notin S$.
\end{list}
\end{list}


\begin{list}{\bf{\ref{exer:sec41-10}.}}
\item \begin{list}{\bf{(a)}}
\item The given statement is a conditional statement.  We can rewrite the subset relations in terms of conditional sentences:  $A \subseteq B$ means, ``For all $x \in U$, if $x \in A$, then 
$x \in B$,'' and $B^c \subseteq A^c$ means, ``For all $x \in U$, if $x \in B^c$, then 
$x \in A^c$.''
\end{list}
\end{list}

%\begin{list}{\ref{exer:sec41-7}.}
%\item \begin{list}{(a)}
%\item The interval $( {a, b} $  is a proper subset of   $( {a, b} \right]$.  
%[$( b \in ( a, b \right]$ and $b \notin ( a, b $.]
%\end{list}
%\end{list}
%
%\begin{list}{}
%\item \begin{list}{(e)}
%\item $\left\{  x \in \mathbb{R} \mid \left| x \right| > 2 \right\} = ( { - \infty , 2}  \cup ( {2, \infty } $.
%\end{list}
%\end{list}
\hbreak
\renewcommand{\labelenumi}{\textbf{\arabic{enumi}.}}

\endinput

