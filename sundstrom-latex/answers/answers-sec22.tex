\section*{Section \ref{S:logequiv}}

\begin{list}{\bf{\ref{exer:sec23-1}.}}
\item \begin{list}{\bf{(a)}}
\item Converse:  If $a^2 = 25$, then $a = 5$.  Contrapositive:  If $a^2 \ne 25$, then $a \ne 5$.
\end{list}
\end{list}

\begin{list}{}
\item \begin{list}{\bf{(b)}}
\item Converse:  If Laura is playing golf, then it is not raining.  Contrapositive:  If Laura is not playing golf, then it is raining.
\end{list}
\end{list}

\begin{list}{}
\item \begin{list}{\bf{(c)}}
\item Converse:  If $a^4 \ne b^4$, then $a \ne b$.   Contrapositive:  If $a^4 = b^4$, then $a = b$.
\end{list}
\end{list}

\begin{list}{}
\item \begin{list}{\bf{(d)}}
\item Converse:  If $3a$ is an odd integer, then $a$ is an odd integer.   Contrapositive:  If $3a$ is an even integer, then $a$ is an even integer.
\end{list}
\end{list}

%\begin{list}{}
%\item \begin{list}{\bf{(c)}}
%\item $a = b$ or $a^4 \ne b^4$. \qquad \textbf{(d)} $a$ is an even integer or $3a$ is an odd integer.
%\end{list}
%\end{list}


\begin{list}{\bf{\ref{exer:sec23-2}.}}
\item \begin{list}{\bf{(a)}}
\item Disjunction:  $a \ne 5$ or $a^2 = 25$.  Negation:  $a = 5$ and $a^2 \ne 25$.
\end{list}
\end{list}

\begin{list}{}
\item \begin{list}{\bf{(b)}}
\item Disjunction:  It is raining or Laura is playing golf.  \\Negation:  It is not raining and Laura is not playing golf.
\end{list}
\end{list}

\begin{list}{}
\item \begin{list}{\bf{(c)}}
\item Disjunction:  $a = b$ or $a^4 \ne b^4$.  Negation:  $a \ne b$ and $a^4 = b^4$.
\end{list}
\end{list}

\begin{list}{}
\item \begin{list}{\bf{(d)}}
\item Disjunction:  $a$ is an even integer or $3a$ is an odd integer.  \\Negation:  $a$ is an odd integer and $3a$ is an even integer.
\end{list}
\end{list}



\begin{list}{\bf{\ref{exer:sec23-3}.}}
\item \begin{list}{\bf{(a)}}
\item We will not win the first game or we will not win the second game.
\end{list}
\end{list}

\begin{list}{}
\item \begin{list}{\bf{(b)}}
\item They will not lose the first game and they will not lose the second game.
\end{list}
\end{list}

\begin{list}{}
\item \begin{list}{\bf{(c)}}
\item You mow the lawn and I will not pay you \$20.
\end{list}
\end{list}

\begin{list}{}
\item \begin{list}{\bf{(d)}}
\item We do not win the first game and we will play a second game.
\end{list}
\end{list}

\begin{list}{}
\item \begin{list}{\bf{(e)}}
\item I will not wash the car and I will not mow the lawn.
\end{list}
\end{list}

%\begin{list}{}
%\item \begin{list}{\bf{(f)}}
%\item You graduate from college, and you will not get a job and you will not go to graduate school.
%\end{list}
%\end{list}

%\begin{list}{}
%\item \begin{list}{\bf{(g)}}
%\item I play tennis, and I will  not wash the car and I will not do the dishes.
%\end{list}
%\end{list}
%
%\begin{list}{}
%\item \begin{list}{\bf{(h)}}
%\item Clean your room or do the dishes, and you cannot go to see a movie.
%\end{list}
%\end{list}
%
%\begin{list}{}
%\item \begin{list}{\bf{(i)}}
%\item It is not warm outside, or it does not rain and I will not play golf.
%\end{list}
%\end{list}


\begin{list}{\bf{\ref{exer:sec23-7}.}}
\item \begin{list}{\bf{(a)}}
\item In this case, it may be better to work with the right side first.
\begin{align*}
\left( {P \to R} \right) \vee \left( {Q \to R} \right) &\equiv 
\left( \mynot P \vee R \right) \vee \left( \mynot Q \vee R \right) \\
&\equiv 
\left( \mynot P \vee \mynot Q \right) \vee \left( R \vee R \right) \\
&\equiv 
\left( \mynot P \vee \mynot Q \right) \vee R \\
&\equiv 
\mynot \left( P \wedge Q \right) \vee R \\
&\equiv \left( P \wedge Q \right) \to R.
\end{align*}
\end{list}
\end{list}

\begin{list}{}
\item \begin{list}{\bf{(b)}}
\item In this case, we start with the left side.
\begin{align*}
\left[ P \to \left( Q \wedge R \right) \right] &\equiv \mynot P \vee \left( Q \wedge R \right) \\
              &\equiv \left( \mynot P \vee Q \right) \wedge \left( \mynot P \vee R \right) \\
              &\equiv \left( P \to Q \right) \wedge \left( P \to R \right)
\end{align*}
\end{list}
\end{list}



\begin{list}{\bf{\ref{exer:diffimpliescont}.}}
\item Statements (c) and (d) are logically equivalent to the given conditional statement.  
Statement~(f) is the negation of the given conditional statement.
%\item \begin{list}{(a)}
%\item This is the converse of the given conditional statement.  It is not logically equivalent to the given statement and is not the negation of the given
%\end{list}
\end{list}


\begin{list}{\bf{\ref{exer:sec23-10}.}}
\item \begin{list}{\bf{(d)}}
\item This is the contrapositive of the given statement and hence, it is logically equivalent to the given statement.
\end{list}
\end{list}
\hbreak
