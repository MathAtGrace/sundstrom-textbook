\section*{Section \ref{S:provingset}}
\renewcommand{\labelenumi}{(\textbf{\alph{enumi}})}

%\begin{list}{\ref{exer:sec42-1}.}
%\item \begin{list}{(a)}
%\item If $x \in A$, then there exists an integer $k$ such that $x = 9k$.  This means that 
%$x = 3 ( 3k )$ and since $3k \in \mathbb{Z}$, we see that $x \in B$.
%\end{list}
%\end{list}
%
%\begin{list}{}
%\item \begin{list}{(b)}
%\item The set $B$ is not a subset of $A$ since there exist elements (such as 3 and 6) that are in 
%$B$ but not in $A$.
%
%\end{list}
%\end{list}

\begin{list}{\bf{\ref{exer:sec42-2}.}}
\item \begin{list}{\bf{(a)}}
\item The set  $A$  is  a subset of  $B$.  To prove this, we let  $x \in A$.  Then  
$ - 2 < x < 2$.  Since  $x < 2$, we conclude that  $x \in B$ and hence, we have proved that $A$ is a subset of $B$.
\end{list}
\end{list}

\begin{list}{}
\item \begin{list}{\bf{(b)}}
\item The set  $B$  is not a subset of  $A$.  There are many examples of a real number that is in  $B$  but not in  $A$.  For example, $-3$ is in $A$, but $-3$ is not in $B$.
\end{list}
\end{list}



\begin{list}{\bf{\ref{exer:modsubset}.}}
\item \begin{enumerate}
  \item $A = \{ \ldots, -9, -1, 7, 15, 23, \ldots \}$ and $B = \{ \ldots, -9, -5, -1, 3, 7, 11, 15, \ldots \}$.
  \item To prove that $A \subseteq B$, let $x \in A$.  Then, $x \equiv 7 \pmod 8$ and so, 
$8 \mid \left( x - 7 \right)$.  This means that there exists an integer $m$ such that $x - 7 = 8m$.  
By adding 4 to both sides of this equation, we see that $x - 3 = 8m + 4$, or 
$x - 3 = 4 \left( 2m + 1 \right)$.  From this, we conclude that $4 \mid \left( x - 3 \right)$ and that $x \equiv 3 \pmod 4$.  Hence, $x \in B$.
\item $B \not \subseteq A$.  For example, $3 \in B$ and $3 \notin A$.
\end{enumerate}
\end{list}

\begin{list}{\bf{\ref{exer51:modsets}.}}
\item \begin{enumerate}
\item We will prove that $A = B$.  Notice that if $x \in A$, then there exists an integer $m$ such that $x - 2 = 3m$.  We can use this equation to see that $2x - 4 = 6m$ and so 6 divides $(2x - 4)$.  Therefore, $x \in B$ and hence, $A \subseteq B$.

Conversely, if $y \in B$, then there exists an integer $m$ such that $2y - 4 = 6m$.  Hence, $y - 2 = 3m$, which implies that $\mod{y}{2}{3}$ and $y \in A$.  Therefore, $B \subseteq A$.

\addtocounter{enumi}{1}
\item $A \cap B = \emptyset$.  To prove this, we will use a proof by contradiction and assume that $A \cap B \ne \emptyset$.  So there exists an $x$ in $A \cap B$.  We can then conclude that there exist integers $m$ and $n$ such that $x - 1 = 5m$ and $x - y = 10n$.  So $x = 5m + 1$ and $x = 10n + 7$.  We then see that
\begin{align*}
5m + 1 &= 10n + 7 \\
5(m - 2n) &= 6
\end{align*}
The last equation implies that 5 divides 6, and this is a contradiction.
\end{enumerate}
\end{list}



\begin{list}{\bf{\ref{exer:intersectandunion}.}}
\item \begin{enumerate}
\item Let $x \in A \cap B$.  Then, $x \in A$ and $x \in B$.  This proves that if $x \in A \cap B$, then $x \in A$, and hence, 
$A \cap B \subseteq A$.
\item Let $x \in A$.  Then, the statement ``$x \in A$ or $x \in B$'' is true.  Hence, 
$A \subseteq A \cup B$.
\setcounter{enumi}{4}
\item By Theorem~\ref{T:subsets}, $\emptyset \subseteq A \cap \emptyset$.  By Part~(a), 
$A \cap \emptyset \subseteq \emptyset$.  Therefore, $A \cap \emptyset = \emptyset$.
\end{enumerate}

%\item \begin{list}{\bf{(a)}}
%\item Start by letting $x$ be an element of $A \cap B$. \quad \textbf{(b)} Start by letting $x$ be an element of $A$.
%\end{list}
%\end{list}
%
%\begin{list}{}
%\item \begin{list}{\bf{(e)}}
%\item By Theorem~\ref{T:subsets}, $\emptyset \subseteq A \cap \emptyset$.  By Part~(a), 
%$A \cap \emptyset \subseteq \emptyset$.  Therefore, $A \cap \emptyset = \emptyset$.
%\end{list}
\end{list}

\begin{list}{\bf{\ref{exer:subsetprop}.}}
\item Start with, ``Let $x \in A$.''  Then use the assumption that $A \cap B^c = \emptyset$ to prove that $x$ must be in $B$.
\end{list}

\begin{list}{\bf{\ref{exer:unionandintersect}.}}
\item \begin{list}{\bf{(a)}}
\item Let $x \in A \cap C$.  Then $x \in A$ and $x \in C$.  Since we are assuming that 
$A \subseteq B$, we see that $x \in B$ and $x \in C$.  This proves that \linebreak
$A \cap C \subseteq B \cap C$.
\end{list}
\end{list}



\begin{list}{\bf{\ref{exer42:setstruefalse}.}}
\item \begin{list}{\bf{(a)}}
\item This is Proposition~\ref{P:subsetprop}.  (See Exercise~\ref{exer:subsetprop} .)
%``If $A \subseteq B$, then $A \cap B^c = \emptyset$'' is 
%Proposition~\ref{P:subsetprop}.  To prove the other conditional statement, start with, 
%``Let $x \in A$.''  Then use the assumption that $A \cap B^c = \emptyset$ to prove that $x$ must be in $B$.
\end{list}
\end{list}


\begin{list}{}
\item \begin{list}{\bf{(b)}}
\item To prove ``If $A \subseteq B$, then $A \cup B = B$,'' first note that if $x \in B$, then 
$x \in A \cup B$ and, hence, $B \subseteq A \cup B$.  Now let $x \in A \cup B$ and note that since $A \subseteq B$, if $x \in A$, then $x \in B$.  Use this to argue that under the assumption that $A \subseteq B$, $A \cup B \subseteq B$.

To prove ``If $A \cup B = B$, then $A \subseteq B$,'' start with,  Let $x \in A$ and use this assumption to prove that $x$ must be an element of $B$.
\end{list}
\end{list}

\hbreak
\renewcommand{\labelenumi}{\textbf{\arabic{enumi}.}}

\endinput

