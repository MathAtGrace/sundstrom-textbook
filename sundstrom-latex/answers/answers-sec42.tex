\section*{Section \ref{S:otherinduction}}
\renewcommand{\labelenumi}{(\textbf{\alph{enumi}})}

\begin{list}{\bf{\ref{exer:sec52-1}.}}
\item \begin{list}{\bf{(a)}}
\item Let $P \left( n \right)$ be, ``$3^n > 1 + 2^n$.''  $P \left( 2 \right)$ is true since $3^2 = 9$, $1 + 2^2 = 5$, and $9 > 5$.

For the inductive step, we assume that $P \left( k \right)$ is true and so 
\begin{equation}
3^k > 1 + 2^k.
\end{equation}
To prove that $P(k+1)$ is true, we must prove that $3^{k+1} > 1 + 2^{k+1}$.  Multiplying both sides inequality~(1) by 3 gives
\[
3^{k + 1} > 3 + 3 \cdot 2^k.
\]
Now, since $3 > 1$ and $3 \cdot 2^k > 2^{k + 1}$, we see that $3 + 3 \cdot 2^k > 1 + 2^{k+1}$ and hence, $3^{k + 1} > 1 + 2^{k+1}$.  Thus, if $P \left( k \right)$ is true, then $P \left( k +1  \right)$ is true.  This proves the inductive step.
\end{list}
\end{list}

\setcounter{equation}{0}
\begin{list}{\bf{\ref{exer:sec52-6}.}} 
\item If $n \geq 5$, then $n^2 < 2^n$.  To prove this, we let $P(n)$ be $n^2 < 2^n$.  For the basis step, when $n = 5$, $n^2 = 25$, $2^n = 32$, and $25 < 32$.  For the inductive step, we assume that $k \geq 5$  and that $P(k)$ is true or that $k^2 < 2^k$.  With these assumptions, we need to prove that $P(k+1)$ is true or that $(k+1)^2 < 2^{k+1}$.  We first note that 
\begin{equation}
( k + 1 )^2 = k^2 + 2k + 1 < 2^k + 2k + 1.
\end{equation}
Since $k \geq 5$, we see that $5k < k^2$ and so $2k + 3k < k^2$.  However, $3k > 1$ and so $2k + 1 < 2k + 3k < k^2$.  Combinining this with inequality~(1), we obtain $(k+1)^2 < 2^k+ k^2$.  Using the assumption that $P(k)$ is true $\left(k^2 < 2^k\right)$, we obtain
\begin{align*}
(k+1)^2 &< 2^k + 2^k = 2 \cdot 2^k \\
(k+1)^2 &< 2^{k+1}
\end{align*}
This proves that if $P(k)$ is true, then $P(k+1)$ is true.
\end{list}



\begin{list}{\bf{\ref{exer:sec52-3}.}}
\item Let  $P( n )$  be the predicate, ``$8^n \mid ( 4n )!$.''  Verify that  
$P( 0 ), P( 1 ), P( 2 )$, and $P( 3 )$ are true. 
For the inductive step, the following fact about factorials may be useful:
\begin{align*}
[ 4 (k + 1 ) ]! &= ( 4k + 4 )! \\
                           &= (4k + 4)(4k + 3)(4k + 2)(4k + 1) ( 4k )!.
\end{align*}
\end{list}

\begin{list}{\bf{\ref{exer:sec52-4}.}}
\item Let  $P( n )$  be, ``The natural number  $n$  can be written as a sum of natural numbers, each of which is a 2  or a  3.''  Verify that  
$P( 4 ), P( 5 ), P( 6 )$, and $P( 7 )$ are true.

To use the Second Principle of Mathematical Induction, assume that  $k \in \mathbb{N}, k \geq 5$ and that  
$P( 4 ), P( 5 ),  \ldots , P( k )$ are true.  Then notice that
\[
k + 1 = ( {k - 1} ) + 2.
\]
Since  $k - 1 \geq 4$, we have assumed that  $P( {k - 1} )$ is true.  This means that $(k-1)$ can be written as a sum of natural numbers, each of which is a 2 or a 3.  Since $k + 1 = (k - 1) + 2$, we can conclude that $(k+1)$ can be written as a sum of natural numbers, each of which is a 2 or a 3.  This completes the proof of the inductive step.
\end{list}



\begin{list}{\bf{\ref{exer:2elementsubsets}.}}
\item Let $P ( n )$ be, ``Any set with $n$ elements has 
$\dfrac{n ( n - 1 )}{2}$ 2-element subsets.''  $P ( 1 )$ is true since any set with only one element has no 2-element subsets.

Let $k \in \mathbb{N}$ and assume that $P ( k )$ is true.  This means that any set with $k$ elements has $\dfrac{k ( k - 1 )}{2}$ 2-element subsets.  Let $A$ be a set with $k + 1$ elements, and let $x \in A$. Now use the inductive hypothesis on the set 
$A - \left\{ x \right\}$, and determine how the 2-element subsets of $A$ are related to the set 
$A - \left\{ x \right\}$. 

%Then, the set $A - \left\{ x \right\}$ is a set with $k$ elements.  So, $A - \left\{ x \right\}$ has $\dfrac{k ( k - 1 )}{2}$ 2-element subsets.  These are also 2-element subsets of $A$.  The other 2-element subsets of $A$ are of the form 
%$\left\{ x, y \right\}$ where $y \in A - \left\{ x \right\}$.  There are $k$ such 2-element subsets of $A$.  So, the total number of 2-element subsets of $A$ is
%\[
%\begin{aligned}
%\frac{k ( k - 1 )}{2} + k &= \frac{k ( k - 1 ) +2k}{2} \\
%                                     &= \frac{ ( k + 1 ) k}{2}. \\
%\end{aligned}
%\]
%This proves that if $P ( k )$ is true, then $P ( k + 1 )$ is true.
\end{list}



%\begin{list}{\bf{\ref{exer:sec51-11}.}}
%\item For the inductive step, the following trigonometric identities are useful:
%\begin{list}{}
%\item $\cos ( \alpha + \beta ) = \cos \alpha \cos \beta - \sin \alpha \sin \beta$.
%\item $\sin ( \alpha + \beta ) = \sin \alpha \cos \beta + \cos \alpha \sin \beta$.
%\end{list}
%\end{list}


\begin{list}{\bf{\ref{exer:specialFTA}.}}
\item \begin{list}{\bf{(a)}}
\item Use Theorem~\ref{T:primefactors}.
\end{list}
\end{list}

\begin{list}{}
\item \begin{list}{\bf{(b)}}
\item  Assume $k \ne q$ and consider two cases:  (i) $k < q$; (ii) $k > q$.
\end{list}
\end{list}




\hbreak
\renewcommand{\labelenumi}{\textbf{\arabic{enumi}.}}


\endinput


