\section*{Section \ref{S:recursion}}
\begin{list}{\bf{\ref{exer:sec51-factorial}.}}
\item  Let $P(n)$ be $a_n = n!$.  Since $a_0 = 1$ and $0 ! = 1$, we see that $P(0)$ is true.  For the inductive step, we assume that $k \in \N \cup \{0\}$ and that $P(k)$ is true or that  $a_k = k!$.
\begin{align*}
a_{k+1} &= \left( k + 1 \right)a_k \\
        &= \left( k + 1 \right) k! \\
        &= \left( k + 1 \right)!. 
\end{align*}
This proves the inductive step that  if $P(k)$ is true, then $P(k+1)$ is true.
\end{list}




\begin{list}{\bf{\ref{exer:sec53-fib}.}}
\item \begin{list}{\bf{(a)}}
\item Let $P( n )$ be, ``$f_{4n}$ is a multiple of 3.''  Since $f_4 = 3$, 
$P( 1 )$ is true.  If $P( k )$ is true, then there exists an integer $m$ such that $f_{4k} = 3m$.  We now need to prove that $P(k+1)$ is true or that $f_{4(k+1)}$ is a multiple of 3.  We use the following:
\begin{align*}
f_{4 \left( k + 1 \right)} &= f_{4k + 4} \\
                           &= f_{4k+3} + f_{4k+2} \\
                 &= \left( f_{4k+2} + f_{4k+1} \right) + \left( f_{4k+1} + f_{4k} \right) \\
                 &= f_{4k+2} + 2 f_{4k+1} + f_{4k} \\
                 &= \left( f_{4k+1} + f_{4k} \right) + 2 f_{4k+1} + f_{4k} \\
                 &= 3f_{4k + 1} + 2f_{4k} \\
\end{align*}
We now use the assumption that $f_{4k} = 3m$ and the last equation to obtain $f_{4(k+1)} = 3f_{4k + 1} + 2\cdot 3m$ and hence,
$f_{4(k+1)} = 3 \left( f_{4k +1} + 2m \right)$.  Therefore, $f_{4(k+1)}$ is a multiple of 3 and this completes the proof of the inductive step.
\end{list}
\end{list}


\begin{list}{}
\item \begin{list}{\bf{(c)}}
\item Let $P( n )$ be, ``$f_1  + f_2  +  \cdots  + f_{n - 1}  = f_{n + 1}  - 1$.''  Since $f_1 = f_3 - 1$, $P ( 2 )$ is true.  For $k \geq 2$, if $P ( k )$ is true, then 
$f_1  + f_2  +  \cdots  + f_{k - 1}  = f_{k + 1}  - 1$.  Then
\[
\begin{aligned}
\left( f_1  + f_2  +  \cdots  + f_{k - 1} \right) + f_k  &= \left( f_{k + 1}  - 1 \right) + f_k \\
            &= \left( f_{k+1} + f_k \right) - 1 \\
            &= f_{k+2} - 1.
\end{aligned}
\]
This proves that if $P \left( k \right)$ is true, then $P \left( k + 1\right)$ is true.
\end{list}
\end{list}


\begin{list}{}
\item \begin{list}{\bf{(f)}}
\item Let $P( n )$ be, ``$f_1^2  + f_2^2  +  \cdots  + f_n^2  = f_n f_{n + 1} $.''  For the basis step, we notice that $f_1^2 = 1$ and $f_1 \cdot f_2 = 1$ and hence, $P(1)$ is true.  For the inductive step, we need to prove that if $P(k)$ is true, then $P(k+1)$ is true.  That is, we need to prove that if $f_1^2  + f_2^2  +  \cdots  + f_k^2  = f_k f_{k + 1} $, then 
$f_1^2  + f_2^2  +  \cdots  + f_k^2 + f_{k+1}^2  = f_{k+1} f_{k + 2} $.  To do this, we can use
\[
\begin{aligned}
\left( f_1^2  + f_2^2  +  \cdots  + f_k^2  \right) + f_{k+1}^2  &= 
f_k f_{k + 1} + f_{k+1}^2 \\
f_1^2  + f_2^2  +  \cdots  + f_k^2 + f_{k+1}^2  &= f_{k+1} \left( f_k + f_{k+1} \right) \\
                                                &= f_{k+1} f_{k+2}.         
\end{aligned}
\]
\end{list}
\end{list}



\begin{list}{\bf{\ref{exer:geomseq}.}}
\item For the inductive step, if $a_k = a \cdot r^{k - 1}$, then
\[
\begin{aligned}
a_{k+1} &= r \cdot a_k \\
        &= r \left( a \cdot r^{k-1} \right) \\
        &= a \cdot r^k. 
\end{aligned}
\]
\end{list}




\begin{list}{\bf{\ref{exer:geometricseries2}.}}
\item For the inductive step, use the assumption that 
$S_k  = a\left( {\dfrac{{1 - r^k }}{{1 - r}}} \right)$ and the recursive definiton to write 
$S_{k + 1}  = a + r \cdot S_k$.
\end{list}


\begin{list}{\bf{\ref{exer:sec53-arithexample}.}}
\item \begin{list}{\bf{(a)}}
\item $a_2 = 7$, $a_3 = 12$, $a_4 = 17$, $a_5 = 22$, $a_6 = 27$.
\end{list}
\end{list}

\begin{list}{}
\item \begin{list}{\bf{(b)}}
\item One possibility is:  For each $n \in \N$, $a_n = 2 + 5(n - 1)$.
\end{list}
\end{list}



\begin{list}{\bf{\ref{exer:sec53-5}.}}
\item \begin{list}{\bf{(a)}}
\item $a_2 = \sqrt{6}$, $a_3 = \sqrt{\sqrt{6} + 5} \approx 2.729$, $a_4 \approx 2.780$, 
$a_5 \approx 2.789$, $a_6 \approx 2.791$
\end{list}
\end{list}


\begin{list}{}
\item \begin{list}{\bf{(b)}}
\item Let $P( n )$ be, ``$a_n < 3$.''  Since $a_1 = 1$, $P( 1 )$ is true.  For $k \in \mathbb{N}$, if $P( k )$ is true, then $a_k < 3$.  Now
\[
a_{k+1} = \sqrt{5 + a_k}.
\] 
Since $a_k < 3$, this implies that $a_{k+1} < \sqrt{8}$ and hence, $a_{k+1} < 3$.

This proves that if $P \left( k \right)$ is true, then $P \left( k + 1\right)$ is true.
\end{list}
\end{list}


\begin{list}{\bf{\ref{exer:sec53-6}.}}
\item \begin{list}{\bf{(a)}}
\item $a_3 = 7$, $a_4 =15$, $a_5 =31$, $a_6 =63$
\end{list}
\end{list}

\begin{list}{}
\item \begin{list}{\bf{(b)}}
\item Think in terms of powers of 2.
\end{list}
\end{list}

\begin{list}{\bf{\ref{exer:sec53-7}.}}
\item \begin{list}{\bf{(a)}}
\item $a_3 = \dfrac{3}{2}$, $a_4 =\dfrac{7}{4}$, $a_5 =\dfrac{37}{24}$, $a_6 =\dfrac{451}{336}$
\end{list}
\end{list}

%\begin{list}{\ref{exer:sec53-8}.}
%\item \begin{list}{(a)}
%\item $a_4 = 3$, $a_5 =5$, $a_6 =9$, $a_7 =17$
%\end{list}
%\end{list}


\pagebreak
\begin{list}{\bf{\ref{exer:sec53-9}.}}
\item \begin{list}{\bf{(b)}}
\item \begin{multicols}{3}
$a_2 = 5$

$a_3 = 23$

$a_4 = 119$

$a_5 = 719$

$a_6 = 5039$

$a_7 = 40319$

$a_8 = 362879$

$a_9 = 3628799$

$a_{10} = 39916799$
\end{multicols}
\end{list}
\end{list}


\begin{list}{\bf{\ref{exer:lucasnumbers}.}}
\item \begin{list}{\bf{(a)}}
\item Let $P(n)$ be, ``$L_n = 2f_{n+1} - f_n$.''  First, verify that $P(1)$ and $P(2)$ are true.  Now let $k$ be a natural number with $k \geq 2$ and assume that $P(1)$, $P(2)$, \ldots, 
$P(k)$ are all true. Since $P(k)$ and $P(k-1)$ are both assumed to be true, we can use them to help prove that $P(k+1)$ must then be true as follows:
\begin{align*}
L_{k+1} &= L_k + L_{k-1} \\
        &= \left( 2f_{k+1} - f_k \right) + \left( 2f_k - f_{k-1} \right) \\
        &= 2 \left( f_{k+1} + f_k \right) - \left( f_k + f_{k-1} \right) \\
        &= 2f_{k+2} - f_{k+1}.
\end{align*}
\end{list}
\end{list}


%\begin{list}{}
%\item \begin{list}{\bf{(b)}}
%\item Let $P(n)$ be, ``$5f_n = L_{n-1} + L_{n+1}$.''  First, verify that $P(2)$ and $P(3)$ are true.  Now let $k$ be a natural number with $k \geq 3$ and assume that $P(2)$, $P(3)$, \ldots, 
%$P(k)$ are all true. Since $P(k)$ and $P(k-1)$ are both assumed to be true, we can use them to help prove that $P(k+1)$ must then be true as follows:
%\begin{align*}
%5f_{k+1} &= 5f_k + 5f_{k-1} \\
%        &= \left( L_{k-1} + L_{k+1} \right) + \left( L_{k-2} + L_k \right) \\
%        &= \left( L_{k-1} + L_{k-2} \right) - \left( L_k + L_{k+1} \right) \\
%        &= L_k + L_{k+2}.
%\end{align*}
%\end{list}
%\end{list}


\hbreak
\endinput


