\section*{Section \ref{S:relations}}

\begin{list}{\bf{\ref{exer:sec71-1}.}}
\item \begin{list}{\bf{(a)}}
\item The set  $A \times B$  contains  nine  ordered pairs.  The set  $A \times B$  is a relation from  $A$  to  $B$  since 
$A \times B$  is a subset of  $A \times B$.
\end{list}
\end{list}

\begin{list}{}
\item \begin{list}{\bf{(b)}}
\item The set  $R$  is a relation from  $A$  to  $B$  since  $R \subseteq A \times B$.
\end{list}
\end{list}

\begin{list}{}
\item \begin{list}{\bf{(c)}}
\item $\text{dom}( R ) = A$, $\text{range}( R ) = \left\{ {p, q} \right\}$
\end{list}
\end{list}

%\begin{list}{}
%\item \begin{list}{\bf{(d)}}
%\item $R^{ - 1}  = \left\{ {( {p, a} ), ( {q, b} ), ( {p, c} ), ( {q, a} )} \right\}$
%\end{list}
%\end{list}



\begin{list}{\bf{\ref{exer:sec71-2}.}}
\item \begin{enumerate}
\item The statement is false since $\left( c, c \right) \notin R$, which can be written as $c \notrel{R} d$.
\item The statement is true since whenever $\left( x, y \right) \in R$, $\left( y, x \right)$ is also in $R$.  That is, whenever $x \mathrel{R} y$, 
$y\mathrel{R} x$.
\item The statement is false since $\left( a, c \right) \in R$, $\left( c, b \right) \in R$, but 
$\left( a, b \right) \notin R$.  That is, $a\mathrel{R} c$, $c \mathrel{R} b$, but $a \notrel{R} b$.
\item The statement is false since $\left( a, a \right) \in R$ and $\left( a, c \right) \in R$.
\end{enumerate}

\end{list}


\begin{list}{\bf{\ref{exer:sec71-3}.}}
\item \begin{list}{\bf{(a)}}
\item The domain of  $D$  consists of the female citizens of the United States whose mother is a female citizen of the United States.
\end{list}
\end{list}

\begin{list}{}
\item \begin{list}{\bf{(b)}}
\item The range of  $D$  consists of those female citizens of the United States who have a daughter that is a female citizen of the United States.
\end{list}
\end{list}

\begin{list}{\bf{\ref{exer:sec71-4}.}}
\item \begin{list}{\bf{(a)}}
\item $( {S, T} ) \in R$ means that $S \subseteq T$.
\end{list}
\end{list}

\begin{list}{}
\item \begin{list}{\bf{(b)}}
\item The domain of the subset relation is $\mathcal{P} ( U )$.
\end{list}
\end{list}

\begin{list}{}
\item \begin{list}{\bf{(c)}}
\item The range of the subset relation is $\mathcal{P} ( U )$.
\end{list}
\end{list}

%\begin{list}{}
%\item \begin{list}{\bf{(d)}}
%\item $R^{-1} = \left\{ (T, S ) \in \mathcal{P} ( U ) \times 
%\mathcal{P} ( U ) \mid S \subseteq T \right\}$.
%\end{list}
%\end{list}

\begin{list}{}
\item \begin{list}{\bf{(d)}}
\item The relation $R$ is not a function from $\mathcal{P} ( U )$ to 
$\mathcal{P} ( U )$ since any proper subset of $U$ is a subset of more than one subset of $U$.
\end{list}
\end{list}


\begin{list}{\bf{\ref{exer:circle-main}.}}
\item \begin{enumerate}
\item $\left\{ {\left. {x \in \mathbb{R}\,} \right| \left( {x, 6} \right) \in S} \right\} = \left\{ { - 8, 8} \right\}$. \\ $\left\{ {\left. {x \in \mathbb{R}\,} \right| \left( {x, 9} \right) \in S} \right\} = \left\{ { - \sqrt {19} , \sqrt {19} } \right\}$.

\item The domain of the relation $S$ is the closed interval $\left[ -10, 10 \right]$.

The range of the relation $S$ is the closed interval $\left[ -10, 10 \right]$.

%\item $S^{-1} = S$.

\item The relation $S$ is not a function from $\mathbb{R}$ to $\mathbb{R}$.

\item The graph of the relation $S$ is the circle of radius 10 whose center is at the origin.
\end{enumerate}
\end{list}


\begin{list}{\bf{\ref{exer:absvalueless2}.}}
\item \begin{list}{\bf{(a)}}
\item $R = \left\{ (a, b) \in \Z \times \Z \left| \left| a - b \right| \leq 2 \right. \right\}$
\end{list}
\end{list}


\begin{list}{}
\item \begin{list}{\bf{(b)}}
\item $\text{dom}(R) = \Z$ and $\text{range}(R) = \Z$
\end{list}
\end{list}
\hbreak

\endinput


