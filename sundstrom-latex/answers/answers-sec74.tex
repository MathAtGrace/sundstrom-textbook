\section*{Section \ref{S:modulararithmetic}}

\begin{list}{\bf{\ref{exer:sec74-modtables}.}}
\item \begin{list}{\bf{(a)}}
\item
\begin{tabular}{ c | c  c  c  c  p{0.5in} c | c  c  c  c}
$\oplus$ & $[ 0 ]$ & $[ 1 ]$ & $[ 2 ]$ & $[ 3 ]$ & 
  & $\odot$ & $[ 0 ]$ & $[ 1 ]$ & $[ 2 ]$ & $[ 3 ]$   \\ \cline{1-5} \cline{7-11}

$[ 0 ]$ & $[ 0 ]$ & $[ 1 ]$ & $[ 2 ]$ & 
$[ 3 ]$ & & $[ 0 ]$ & $[ 0 ]$ & 
$[ 0 ]$ & $[ 0 ]$ & $[ 0 ]$  
\\ 

$[ 1 ]$ & $[ 1 ]$ & $[ 2 ]$ & $[ 3 ]$ & 
$[ 0 ]$ & & $[ 1 ]$ & $[ 0 ]$ & 
$[ 1 ]$ & $[ 2 ]$ & $[ 3 ]$ \\ 

$[ 2 ]$ & $[ 2 ]$ & $[ 3 ]$ & $[ 0 ]$ & 
$[ 1 ]$ & & $[ 2 ]$ & $[ 0 ]$ & 
$[ 2 ]$ & $[ 0 ]$ & $[ 2 ]$ \\ 

$[ 3 ]$ & $[ 3 ]$ & $[ 0 ]$ & $[ 1 ]$ & 
$[ 2 ]$ & & $[ 3 ]$ & $[ 0 ]$ & 
$[ 3 ]$ & $[ 2 ]$ & $[ 1 ]$ \\ 
\end{tabular}
\end{list}
\end{list}
\vskip6pt

\begin{list}{}
\item \begin{list}{\bf{(b)}}
\item \begin{tabular}{ c | c  c  c  c  c  c  c}
$\oplus$ & $[ 0 ]$ & $[ 1 ]$ & $[ 2 ]$ & $[ 3 ]$ & 
$[ 4 ]$ & $[ 5 ]$ & $[ 6 ]$  \\ \hline

$[ 0 ]$ & $[ 0 ]$ & $[ 1 ]$ & $[ 2 ]$ & 
$[ 3 ]$ & $[ 4 ]$ & $[ 5 ]$ & $[ 6 ]$  \\ 

$[ 1 ]$ & $[ 1 ]$ & $[ 2 ]$ & $[ 3 ]$ & 
$[ 4 ]$ & $[ 5 ]$ & $[ 6 ]$ & $[ 0 ]$  \\ 

$[ 2 ]$ & $[ 2 ]$ & $[ 3 ]$ & $[ 4 ]$ & 
$[ 5 ]$ & $[ 6 ]$ & $[ 0 ]$ & $[ 1 ]$  \\ 

$[ 3 ]$ & $[ 3 ]$ & $[ 4 ]$ & $[ 5 ]$ & 
$[ 6 ]$ & $[ 0 ]$ & $[ 1 ]$ & $[ 2 ]$ \\ 

$[ 4 ]$ & $[ 4 ]$ & $[ 5 ]$ & $[ 6 ]$ & 
$[ 0 ]$ & $[ 1 ]$ & $[ 2 ]$ & $[ 3 ]$  \\ 

$[ 5 ]$ & $[ 5 ]$ & $[ 6 ]$ & $[ 0 ]$ & 
$[ 1 ]$ & $[ 2 ]$ & $[ 3 ]$ & $[ 4 ]$  \\ 

$[ 6 ]$ & $[ 6 ]$ & $[ 0 ]$ & $[ 1 ]$ & 
$[ 2 ]$ & $[ 3 ]$ & $[ 4 ]$ & $[ 5 ]$  \\ 
\end{tabular}

\vskip9pt

\begin{tabular}{ c | c  c  c  c  c  c  c}
$\odot$ & $[ 0 ]$ & $[ 1 ]$ & $[ 2 ]$ & $[ 3 ]$ & 
$[ 4 ]$ & $[ 5 ]$ & $[ 6 ]$  \\ \hline

$[ 0 ]$ & $[ 0 ]$ & $[ 0 ]$ & $[ 0 ]$ & 
$[ 0 ]$ & $[ 0 ]$ & $[ 0 ]$ & $[ 0 ]$  \\ 

$[ 1 ]$ & $[ 0 ]$ & $[ 1 ]$ & $[ 2 ]$ & 
$[ 3 ]$ & $[ 4]$ & $[ 5 ]$ & $[ 6 ]$  \\ 

$[ 2 ]$ & $[ 0 ]$ & $[ 2 ]$ & $[ 4 ]$ & 
$[ 6 ]$ & $[ 1 ]$ & $[ 3 ]$ & $[ 5 ]$  \\ 

$[ 3 ]$ & $[ 0 ]$ & $[ 3 ]$ & $[ 6 ]$ & 
$[ 2 ]$ & $[ 5 ]$ & $[ 1 ]$ & $[ 4 ]$  \\ 

$[ 4 ]$ & $[ 0 ]$ & $[ 4 ]$ & $[ 1 ]$ & 
$[ 5 ]$ & $[ 2 ]$ & $[ 6 ]$ & $[ 3 ]$  \\ 

$[ 5 ]$ & $[ 0 ]$ & $[ 5 ]$ & $[ 3 ]$ & 
$[ 1 ]$ & $[ 6 ]$ & $[ 4 ]$ & $[ 2 ]$  \\ 

$[ 6 ]$ & $[ 0 ]$ & $[ 6 ]$ & $[ 5 ]$ & 
$[ 4 ]$ & $[ 3 ]$ & $[ 2 ]$ & $[ 1 ]$  \\ 
\end{tabular}
\end{list}
\end{list}

%\begin{list}{\ref{exer:sec74-modtables}.}
%\item \begin{list}{(a)}
%\item \begin{tabular}[t]{ c | c  c  p{0.5in} c | c  c}
%$\oplus$ & $[ 0 ]$ & $[ 1 ]$ &  & $\odot$ & $[ 0 ]$ & $[ 1 ]$ \\ \cline{1-3} \cline{5-7}
%
%$[ 0 ]$ & $[ 0 ]$ & $[ 1 ]$ &  & $[ 0 ]$ & $[ 0 ]$ & $[ 0 ]$ \\ 
%
%$[ 1 ]$ & $[ 1 ]$ & $[ 0 ]$ &  & $[ 1 ]$ & $[ 0 ]$ & $[ 1 ]$ \\ 
%\end{tabular}
%
%\end{list}
%\end{list}
%\vskip6pt
%
%\begin{list}{}
%\item \begin{list}{(b)}
%\item \begin{tabular}[t]{ c | c  c  c  p{0.5in} c | c  c  c}
%$\oplus$ & $[ 0 ]$ & $[ 1 ]$ & $[ 2 ]$ & &   $\odot$ & $[ 0 ]$ & $[ 1 ]$ & $[ 2 ]$   \\ \cline{1-4} \cline{6-9}
%
%$[ 0 ]$ & $[ 0 ]$ & $[ 1 ]$ & $[ 2 ]$ &  & 
%$[ 0 ]$ & $[ 0 ]$ & $[ 0 ]$ & $[ 0 ]$  
%\\ 
%
%$[ 1 ]$ & $[ 1 ]$ & $[ 2 ]$ & $[ 0 ]$ &  & 
%$[ 1 ]$ & $[ 0 ]$ & $[ 1 ]$ & $[ 2 ]$ 
%\\ 
%
%$[ 2 ]$ & $[ 2 ]$ & $[ 0 ]$ & $[ 1 ]$ &  & 
%$[ 2 ]$ & $[ 0 ]$ & $[ 2 ]$ & $[ 1 ]$  
%\\ 
%\end{tabular}
%\end{list}
%\end{list}
%\vskip6pt


\begin{multicols}{2}
\begin{list}{\bf{\ref{exer:sec74-3}.}}
\item \begin{list}{\bf{(a)}}
\item $[ x ] = [ 1 ]$ or $[ x ] = [ 3 ]$ \qquad 
%(e) $[ x ] = [ 2 ]$ or $[ x ] = [ 3 ]$
\end{list}
\end{list}

\begin{list}{}
\item \begin{list}{\bf{(e)}}
\item $[ x ] = [ 2 ]$ or $[ x ] = [ 3 ]$
\end{list}
\end{list}
\end{multicols}

\begin{list}{}
\item \begin{list}{\bf{(g)}}
\item The equation has no solution.
\end{list}
\end{list}

\begin{list}{\bf{\ref{exer:sec74-4}.}}
\item \begin{enumerate}
\item The statement is false.  By using the multiplication table for $\mathbb{Z}_6$, we see that a counterexample is $\left[ a \right] = \left[ 2 \right]$.

\item The statement is true.  By using the multiplication table for $\mathbb{Z}_5$, we see that:

\begin{multicols}{2}
\begin{list}{}
\item $\left[ 1 \right] \odot \left[ 1 \right] = \left[ 1 \right]$.

\item $\left[ 2 \right] \odot \left[ 3 \right] = \left[ 1 \right]$.

\item $\left[ 3 \right] \odot \left[ 2 \right] = \left[ 1 \right]$.

\item $\left[ 4 \right] \odot \left[ 4 \right] = \left[ 1 \right]$.
\end{list}
\end{multicols}
\end{enumerate}
\end{list}





\begin{list}{\bf{\ref{exer:squaresinZ5}.}}
\item \begin{list}{\bf{(a)}}
\item The proof consists of the following computations:
\begin{multicols}{2}
\begin{list}{}
\item $[ 1 ]^2 = [ 1 ]$
\item $[ 2 ]^2 = [ 4 ]$
\item $[ 3 ]^2 = [ 9 ] = [ 4 ]$
\item $[ 4 ]^2 = [ 16 ] = [ 1 ]$.
\end{list}
\end{multicols}
\end{list}
\end{list}


\begin{list}{\bf{\ref{exer:sec74cong3}.}}
\item \begin{list}{\bf{(a)}}
\item Prove the contrapositive by calculating $[ a ]^2 + [ b ]^2$ for all nonzero $[ a ]$ and $[ b ]$ in $\Z_3$.
\end{list}
\end{list}
\hbreak
%\pagebreak
\endinput


