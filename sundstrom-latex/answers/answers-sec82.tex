\subsection*{Section \ref{S:primefactorizations}}

\begin{list}{\bf{\ref{exer:sec82-1}.}}
\item For both parts, use the fact that the only natural number divisors of a prime number $p$ are 1 and $p$.
\end{list}

\begin{list}{\bf{\ref{exer:sec82-2}.}}
\item Use cases: (1) $p$ divides $a$; (2) $p$ does not divide $a$.  In the first case, the conclusion is automatically true.  For the second case, use the fact that $\gcd ( p, a ) = 1$ and so we can use Theorem~\ref{T:relativelyprimeprop} to conclude that $p$ divides $b$.  Another option is to write the number 1 as a linear combination of $a$ and $p$ and then multiply both sides of the equation by $b$.
\end{list}

\begin{list}{\bf{\ref{exer:sec82-3}.}}
\item A hint for the inductive step: Write  $p \mid ( {a_1 a_2  \cdots a_m } )a_{m + 1} $.  Then look at two cases:  (1) $p \mid a_{m + 1} $; (2) $p$  does not divide  $a_{m + 1} $.
\end{list}

\begin{list}{\bf{\ref{exer:lincombequalone}.}}
\item \begin{list}{\bf{(a)}}
\item $\gcd ( a, b ) = 1$.  Why?
\end{list}
\end{list}

\begin{list}{}
\item \begin{list}{(\bf{b)}}
\item $\gcd ( a, b ) = 1$ or $\gcd ( a, b ) = 2$.  Why?
\end{list}
\end{list}

\begin{list}{\bf{\ref{exer:dividebygcd}.}}
\item \begin{enumerate}
\item $\gcd \left( 16, 28 \right) = 4$.  Also, $\dfrac{16}{4} = 4$, $\dfrac{28}{4} = 7$, and 
$\gcd \left( 4, 7 \right) = 1$.

\item $\gcd \left( 10, 45 \right) = 5$.  Also, $\dfrac{10}{5} = 2$, $\dfrac{45}{5} = 9$, and 
$\gcd \left( 2, 9 \right) = 1$.
\end{enumerate}
\end{list}


\begin{list}{\bf{\ref{exer:24divides-nsquaredminus1}.}}
\item Part (b) of Exercise~(\ref{exer:sec82-truefalse}) can be helpful.
\end{list}


\begin{list}{\bf{\ref{exer:cdividessum}.}}
\item The statement is true.  Start of a proof:  If   $\gcd ( {a, b} ) = 1$  and  
$c \mid ( {a + b} )$, then there exist integers $x$ and $y$ such that 
$ax + by = 1$ and there exists an integer $m$ such that $a + b = cm$.  
%So, we can write 
%$b = cm - a$ and substitute this into the other equation.  This gives
%\[
%\begin{aligned}
%              ax + ( cm - a ) y &= 1 \\
%a ( x - y ) + c ( my ) &= 1. \\
%\end{aligned}
%\]
%This proves that $\gcd ( a, c ) = 1$.  A similar proof shows that 
%$\gcd ( b, c ) = 1$.
\end{list}
\hbreak
\pagebreak
\endinput


