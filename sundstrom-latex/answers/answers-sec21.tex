\section*{Section \ref{S:logop}}

\begin{list}{\bf{\ref{exer:sec22-1}.}} 
\item The statement was true. When the hypothesis is false, the conditional statement is true.
\end{list}

\begin{list}{\bf{\ref{exer:sec22-2}.}} 
\item \begin{multicols}{3}
\textbf{(a)} $P$ is false.  

\textbf{(b)} $P \wedge Q$ is false.

\textbf{(c)} $P \vee Q$ is false.
\end{multicols}
\end{list}


%\begin{list}{\bf{\ref{exer:sec22-3}.}} 
%\item \begin{multicols}{3}
%\textbf{(a)} $\mynot P \to Q$ is true.
%\textbf{(b)} $Q \to P$ is true.
%\textbf{(c)} $P \vee Q$ is true.
%\end{multicols}
%\end{list}

\begin{list}{\bf{\ref{exer:statements4}.}}
%\item \begin{list}{\bf{(a)}}
%\item $\mynot Q \to P$ is true.
%\end{list}
%\end{list}
%
%\begin{list}{}
%\item \begin{list}{\bf{(b)}}
%\item $P$ is true.
%\end{list}
%\end{list}
%
%
%\begin{list}{}
\item \begin{list}{\bf{(c)}}
\item Statement $P$ is true but since we do not know if $R$ is true or false, we cannot tell if $P \wedge R$ is true or false.
\end{list}
\end{list}

%\begin{list}{}
%\item \begin{list}{\bf{(d)}}
%\item Cannot tell if $R \to \mynot P$ is true or false.
%\end{list}
%\end{list}


\begin{list}{\bf{\ref{exer:sec22-5}.}}  
\item Statements (a) and (d) have the same truth table.  Statements (b) and (c) have the same truth table.
$$
\BeginTable
    \BeginFormat
    | c | c | c | c |
    \EndFormat
     \_6
      | $P$ | $Q$ \|6 $P \to Q$ | $Q \to P$ |\\+22 \_6
      | T   |  T  \|6  T | T | \\ 
      | T   |  F  \|6  F | T | \\ 
      | F   |  T  \|6  T | F | \\ 
      | F   |  F  \|6  T | T | \\ \_6
 \EndTable
 $$
\end{list}

\begin{list}{\bf{\ref{exer:sec22-6}.}}  
\item $$
\BeginTable
\BeginFormat
| c | c | c | c | c |
\EndFormat
\_6
       | $P$  |  $Q$  |  $R$  \|6  $P \wedge (Q \vee R)$  |  $(P \wedge Q) \vee (P \wedge R)$  | \\+22 \_6
          | T | T | T \|6 T | T | \\ 
          | T | T | F \|6 T | T |  \\ 
          | T | F | T \|6 T | T | \\ 
          | T | F | F \|6 F | F | \\ 
          | F | T | T \|6 F | F | \\ 
          | F | T | F \|6 F | F | \\ 
          | F | F | T \|6 F | F |  \\ 
          | F | F | F \|6 F | F |  \\ \_6
\EndTable
$$
The two statements have the same truth table.
\end{list}


%\begin{list}{\bf{\ref{exer:sec22-7}.}}
%\item Statements (a) and (c) are true.  Statement (b) is false.
%\end{list}

\begin{list}{\bf{\ref{exer:sec22-8}.}} 
%\item \begin{list}{\bf{(a)}} 
%\item If the integer $x$ is even, then $x^2$ is even.
%\end{list}
%\end{list}
%
%\begin{list}{} 
%\item \begin{list}{\bf{(b)}} 
%\item The integer $x$ is even implies that $x^2$ is even.
%\end{list}
%\end{list}
%
%
%\begin{list}{} 
\item \begin{list}{\bf{(c)}} 
\item The integer  $x$  is even only if $x^2$   is even.
\end{list}
\end{list}


\begin{list}{} 
\item \begin{list}{\bf{(d)}} 
\item For the integer $x$ to be even, it is necessary that $x^2$ be even.  %The integer $x^2$ is even is necessary for $x$ to be even.
\end{list}
\end{list}


%\begin{list}{} 
%\item \begin{list}{\bf{(e)}} 
%\item The integer $x$ is even is sufficient for $x^2$ to be even.
%\end{list}
%\end{list}

%\begin{list}{\bf{\ref{exer:sec22-9}.}} 
%\item \begin{list}{\bf{(a)}} 
%\item If $x^2$ is even, then the integer $x$ is even.
%\end{list}
%\end{list}
%
%\begin{list}{} 
%\item \begin{list}{\bf{(b)}} 
%\item The integer $x^2$ is even implies that the integer $x$ is even.
%\end{list}
%\end{list}
%
%
%\begin{list}{} 
%\item \begin{list}{\bf{(c)}} 
%\item The integer  $x^2$  is even only if the integer $x$ is even.
%\end{list}
%\end{list}
%
%
%\begin{list}{} 
%\item \begin{list}{\bf{(d)}} 
%\item The integer $x$ is even is necessary for $x^2$ to be even.
%\end{list}
%\end{list}
%
%
%\begin{list}{} 
%\item \begin{list}{\bf{(e)}} 
%\item The integer $x^2$ is even is sufficient for the integer $x$ to be even.
%\end{list}
%\end{list}

\begin{list}{\bf{\ref{exer:tautology-contra}.}}
\item \begin{description} 
\item [(a)] $\mynot Q \vee (P \to Q)$ is a tautology.
\item [(b)] $Q \wedge (P \wedge \mynot Q)$ is a contradiction.
\item [(c)] $(Q \wedge P) \wedge (P \to \mynot Q)$ is a contradiction.
\item [(d)] $\mynot Q \to (P \wedge \mynot P)$ is neither a tautology nor a contradiction. 
\end{description}
\end{list}

\hbreak
%
