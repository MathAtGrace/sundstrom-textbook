\subsection*{Section \ref{S:diophantine}}

\begin{multicols}{2}
\begin{list}{\bf{\ref{exer:sec83-2}.}}
\item \begin{list}{\bf{(a)}}
\item $x = -3 + 14k$, $y = 2 - 9k$
\end{list}
\end{list}

\begin{list}{}
\item \begin{list}{\bf{(b)}}
\item $x = -1 + 11k$, $y = 1 - 9k$
\end{list}
\end{list}

\begin{list}{}
\item \begin{list}{\bf{(c)}}
\item No solution
\end{list}
\end{list}

\begin{list}{}
\item \begin{list}{\bf{(d)}}
\item $x = 2+3k$, $y = -2-4k$
\end{list}
\end{list}
\end{multicols}


\begin{list}{\bf{\ref{exer:balancing}.}}
\item There are several possible solutions to this problem, each of which can be generated from the solutions of the Diophantine equation \linebreak
$27x + 50y = 25$.
\end{list}

\begin{list}{\bf{\ref{exer:sec83-5}.}}
\item This problem can be solved by finding all solutions of a linear Diophantine equation $25x + 16y = 1461$, where both $x$ and $y$ are positive.  The mininum number of people attending the banquet is 66.
\end{list}

\begin{list}{\bf{\ref{exer:lindioph3}.}}
\item \begin{enumerate}
\item $y = 12 + 16k, x_3 = -1 - 3k$

\item If $3y = 12x_1 + 9x_2$ and $3y + 16x_3 = 20$, we can substitute for $3y$ and obtain 
$12x_1 + 9x_2 + 16x_3 = 20$.

\item Rewrite the equation $12x_1 + 9x_2 = 3y$ as $4x_1 + 3x_2 = y$.  A general solution for this linear Diophantine equation is
\begin{align*}
x_1 &= y + 3n \\
x_2 &= -y - 4n. \\
\end{align*}
\end{enumerate}
\end{list}
\hbreak
\endinput


