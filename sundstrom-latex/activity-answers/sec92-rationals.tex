\documentclass[11pt]{article}
\usepackage{c://pctex/activity}

\lhead{}
\chead{\textbf{\large{Exercise 14 - Section 9.2\\Another Proof that $\boldsymbol{\Q^+}$  Is Countable}}}
\rhead{}
\lfoot{\emph{Mathematical Reasoning: Writing and Proof, Third Ed.} \\Ted Sundstrom}
\cfoot{}
\rfoot{\copyright \the\year\, by Pearson Education, Inc.\\}


\begin{document}
\begin{enumerate}
\item \begin{multicols}{2}
$f \left( \dfrac{2}{3} \right) = {2^2} \cdot {3^1} = 12$ 

$f \left( \dfrac{5}{6} \right) = f \left( \dfrac{5}{2 \cdot 3} \right) = {5^2}\cdot {2 \cdot 3} = 150$ 

$f \left( 6 \right) = f(2 \cdot 3) = 2^2 \cdot 3^2 = 36$ 

$f \left( \dfrac{12}{25} \right) = f \left( \dfrac{2^2 \cdot 3}{5^2} \right) = {2^4 \cdot 3^2}\cdot {5^3}$ 

$f \left( \dfrac{375}{392} \right) = f \left( \dfrac{3 \cdot 5^3}{2^3 \cdot 7^2} \right) = {3^2 \cdot 5^6} \cdot {2^5 \cdot 7^3}$

$f \left( \dfrac{2^3 \cdot 11^3}{3 \cdot 5^4} \right) = 2^6 \cdot 11^6 \cdot 3 \cdot 5^7$.
\end{multicols}

\begin{multicols}{2}
\item $f \left( 10 \right) = 100$

\item $f \left( \dfrac{2}{3} \right) = {2^2} \cdot {3^1} = 12$

\item $f \left( \dfrac{2^4 \cdot 17}{3^3 \cdot 13} \right) = 2^8 \cdot 3^5 \cdot 13 \cdot 17^2$
\end{multicols}

\item To prove that the function $f$ is an injection, we let $x, y \in \Q^+$ and assume that $f(x) = f(y)$.  We will write
\begin{equation}
x = \dfrac{p_1^{\alpha_1} p_2^{\alpha_2} \cdots p_r^{\alpha_r}}{q_1^{\beta_1} q_2^{\beta_2} \cdots q_s^{\beta_s}}
\end{equation}
where $p_1$, $p_2$, \ldots, $p_r$ are distinct prime numbers, $q_1$, $q_2$, \ldots, $q_s$ are distinct prime numbers, and \\
$\alpha_1$, $\alpha_2$, \ldots, $\alpha_r$ and $\beta_1$, $\beta_2$, \ldots, $\beta_s$ are natural numbers.   We will also write
\begin{equation}
y = \dfrac{u_1^{\gamma_1} u_2^{\gamma_2} \cdots u_m^{\gamma_m}}{v_1^{\delta_1} v_2^{\delta_2} \cdots v_n^{\delta_n}}
\end{equation}
where $u_1$, $u_2$, \ldots, $u_m$ are distinct prime numbers, $v_1$, $v_2$, \ldots, $v_n$ are distinct prime numbers, and \\
$\gamma_1$, $\gamma_2$, \ldots, $\gamma_m$ and $\delta_1$, $\delta_2$, \ldots, $\delta_n$ are natural numbers.  From the assumption that $f(x) = f(y)$, we conclude that
\[
\left( p_1^{2 \alpha_1} p_2^{2 \alpha_2} \cdots p_r^{2 \alpha_r} \right) \left( q_1^{2 \beta_1-1} q_2^{ \beta_2-1} \cdots q_s^{2 \beta_s-1} \right) = 
\left( u_1^{2 \gamma_1} u_2^{2 \gamma_2} \cdots u_m^{2 \gamma_m} \right) \left( v_1^{2 \delta_1-1} v_2^{ \delta_2-1} \cdots v_n^{2 \delta_n-1} \right).
\]
Now, the Fundamental Theorem of Arithmetic tells us that the prime factoriztion of a natural number is unique.  Using this and the previous equation, we can conclude that 
$r = m$, $s = n$, and 
\begin{itemize}
\item For each natural number $k$ with $1 \leq k \leq r$, $p_k = u_k$ and 
$\alpha_k = \gamma_k$;
\item For each natural number $j$ with $1 \leq j \leq s$, $q_j = v_j$ and 
$\beta_j = \delta_j$.
\end{itemize}
Using the expressions for $x$ and $y$ in~(1) and~(2), we can then conclude that $x = y$.  This proves the function $f$ is an injection.


\pagebreak
\item To prove that $f$ is a surjection, we let $n \in \N$.  We then write the prime factorization of $n$ as follows:
\[
n = \left( p_1^{\gamma_1} p_2^{\gamma_2} \cdots p_r^{\gamma_r} \right) \left( q_1^{\delta_1} q_2^{\delta_2} \cdots q_s^{\delta_s} \right),
\]
where for each natural number $k$ with $1 \leq k \leq r$, $\gamma_k$ is even, and for each $j$ with $1 \leq j \leq s$, $\delta_j$ is even.  So:
\begin{itemize}
\item For each natural number $k$ with 
$1 \leq k \leq r$, there exists $\alpha_k \in \N$ such that $\gamma_k = 2 \alpha_k$.
\item For each natural number $j$ with 
$1 \leq j \leq s$, there exists $\beta_j \in \N$ such that $\delta_j = 2 \beta_j - 1$.
\end{itemize}

We now let $x = \dfrac{p_1^{\alpha_1} p_2^{\alpha_2} \cdots p_r^{\alpha_r}}{q_1^{\beta_1} q_2^{\beta_2} \cdots q_s^{\beta_s}}$.  Then, $x \in \Q^+$ and
\begin{align*}
f \left( x \right) &= f \left( \frac{p_1^{\alpha_1} p_2^{\alpha_2} \cdots p_r^{\alpha_r}}{q_1^{\beta_1} q_2^{\beta_2} \cdots q_s^{\beta_s}} \right) \\
                   &= \left( p_1^{2 \alpha_1} p_2^{2 \alpha_2} \cdots p_r^{2 \alpha_r} \right) \left( q_1^{2 \beta_1-1} q_2^{ \beta_2-1} \cdots q_s^{2 \beta_s-1} \right) \\
                   &= \left( p_1^{\gamma_1} p_2^{\gamma_2} \cdots p_r^{\gamma_r} \right) \left( q_1^{\delta_1} q_2^{\delta_2} \cdots q_s^{\delta_s} \right) \\
                   &= n
\end{align*}
This proves that the function $f$ is a surjection.


\item The proofs in Parts~(5) and~(6), prove that the function $f$ is a bijection.  From this, we conclude that $\Q^+ \approx \N$.

\end{enumerate}
\end{document}
