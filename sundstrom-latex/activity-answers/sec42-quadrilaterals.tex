\documentclass[11pt]{article}
\usepackage{c://pctex/activity}

\lhead{}
\chead{\textbf{\large{Exercise 18 -- Section 4.2\\The Sum of the Angles of a Convex Quadrilateral}}}
\rhead{}
\lfoot{\emph{Mathematical Reasoning: Writing and Proof, Third Ed.} \\Ted Sundstrom}
\cfoot{}
\rfoot{\copyright \the\year\, by Pearson Education, Inc.\\}


\begin{document}
\begin{enumerate}
\item Draw a diagonal to divide the quadrilateral into two triangles.  Conclude that the sum of the angles of a convex quadrilateral is $360^\circ$.

\item Draw a line between two vertices of the pentagon to divide it into a triangle and a quadrilateral.  Use this to conclude that the sum of the angles of a convex pentagon is 
$540^\circ$.

\item Draw a line between two vertices of the hexagon to divide it into a triangle and a pentagon.  Use this to conclude that the sum of the angles of a convex hexagon is 
$720^\circ$.

\item Let $P(n)$ be, ``The sum of the angles of a convex polygon with $n$ sides is 
$(n - 2)180^\circ$.  The basis step is the theorem in Euclidean geometry for the sum of the angles of a triangle.  We have also established that $P(4)$, $P(5)$, and $P(6)$ are true.  For the inductive step, let $k \in \N$ with $k \geq 3$ and assume that $P(k)$ is true.

Now consider a convex polygon with $(k + 1)$ sides.  Pick one of the vertices of this polygon and call its adjacent vertices $A$ and $B$.  If we draw a line segment between $A$ and $B$, we will divide this polygon into a triangle and a convex polygon with $k$ sides.  This means that the sum of the angles of this convex polygon with $(k + 1)$ sides is
\[
(k - 2)180^\circ + 180^\circ = (k - 1) 180^\circ,
\]
and this shows that if $P(k)$ is true, then $P(k + 1)$ is true.
\end{enumerate}

\end{document}
