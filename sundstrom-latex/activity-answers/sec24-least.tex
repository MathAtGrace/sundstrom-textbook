\documentclass[11pt]{article}
\usepackage{c://pctex/activity}

\lhead{}
\chead{\textbf{\large{Exercise 15 - Section 2.4\\Least Upper Bound for a Subset of $\boldsymbol{\R}$}}}
\rhead{}
\lfoot{\emph{Mathematical Reasoning: Writing and Proof, Third Ed.} \\Ted Sundstrom}
\cfoot{}
\rfoot{\copyright \the\year\, by Pearson Education, Inc.\\}


\begin{document}
\begin{enumerate}
\item The universal quantifier was used for the real number  $\beta $
  since  $\beta $  represents an upper bound for  $A$  and the least upper bound  $\alpha $
 must be less than or equal to every other upper bound for  $A$.

%\item We use one of De Morgan's Laws to negate a conjunction: 
%\[
%\mynot \left( {P \wedge Q} \right) \equiv \mynot P \vee \mynot Q.
%\]

\item A real number  $\alpha $ is not the least upper bound for  $A$  provided that  
\[
\mynot P\left( \alpha  \right) \vee \left[ {\left( {\exists \beta  \in \mathbb{R}} \right)\left( {P\left( \beta  \right) \wedge \left( {\alpha  > \beta } \right)} \right)} \right].
\]

\item A real number  $\alpha $ is not the least upper bound for  $A$  provided that  $\alpha $ is not an upper bound for  $A$  or there exists an upper bound  $\beta $  for  $A$  such that  $\alpha  > \beta $.
\end{enumerate}

\end{document}
