\documentclass[11pt]{article}
\usepackage{c://pctex/activity}

\lhead{}
\chead{\textbf{\large{Exercise 22 -- Section 4.3\\Compound Interest}}}
\rhead{}
\lfoot{\emph{Mathematical Reasoning: Writing and Proof, Third Ed.} \\Ted Sundstrom}
\cfoot{}
\rfoot{\copyright \the\year\, by Pearson Education, Inc.\\}


\begin{document}
\begin{enumerate}
\item The amount of money in the account at the end of the third period equals the amount of money in the account at the end of the second period plus the interest earned during the third period.  The amount of money in the account at the end of the second period is  $V_2 $, and the interest earned during the third period is  $i \cdot V_2 $.  Hence,  $V_3  = V_2  + i \cdot V_2 $.So, 
\[
\begin{aligned}
V_3  &= V_2  + i \cdot V_2  \\ 
     &= \left( {1 + i} \right)V_2  \\ 
     &= \left( {1 + i} \right)\left[ {\left( {1 + i} \right)^2 R} \right] \\ 
     &= \left( {1 + i} \right)^3 R \\ 
\end{aligned}
\]
\item The amount of money in the account at the end of the $\left( {n + 1} \right)^{st} $
 period equals the amount of money in the account at the end of the $n^{th} $ period plus the interest earned during the $n^{th} $ period.  The amount of money in the account at the end of the $n^{th} $ period is  $V_n $, and the interest earned during the $\left( {n + 1} \right)^{st}$ period is  $i \cdot V_n $.  Hence,  $V_{n + 1}  = V_n  + i \cdot V_n $.

\item
\begin{tabular}[t]{p{2in} p{3in}}
 $V_{n + 1}  = V_n  + i \cdot V_n$
   
 $V_{n + 1}  = \left( {1 + i} \right)V_n$ 
&
This is a geometric sequence with intial term  $\left( {1 + i} \right)$ and a common ratio of  
$R$. \\
\end{tabular}

\item For each  $n \in \mathbb{N}$,  $V_n  = \left( {1 + i} \right)^n R$.


\end{enumerate}


\end{document}
