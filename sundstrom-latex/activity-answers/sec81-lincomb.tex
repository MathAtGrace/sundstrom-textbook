\documentclass[11pt]{article}
\usepackage{c://pctex/activity}

\lhead{}
\chead{\textbf{\large{Exercise 9 -- Section 8.1\\Linear Combinations and the Greatest Common Divisor}}}
\rhead{}
\lfoot{\emph{Mathematical Reasoning: Writing and Proof, Third Ed.} \\Ted Sundstrom}
\cfoot{}
\rfoot{\copyright \the\year\, by Pearson Education, Inc.\\}


\begin{document}
\begin{enumerate}
\item The greatest common divisor of 12 and 20 is 4.

\item $\gcd(20, 12) = 4 = 20\cdot (-1) + 12 \cdot 2$.

\item Following are some liner combinations of 20 and 12. Notice that $\gcd(20, 12)$ divides each of these linear combinations.
\begin{align*}
20 \cdot 1 + 12 \cdot 1 &= 32  & 20 \cdot 2 + 12 \cdot 1 &= 52 \\
20 \cdot 1 + 12 \cdot (-1) &= 8 & 20 \cdot 2 + 12 \cdot (-2) &= 16 \\
20 \cdot (-2) + 12 \cdot 1 &= -28 & 20 \cdot (-2) + 12 \cdot 3 &= 4 \\
20 \cdot 3 + 12 \cdot 2 &= 84  &  20 \cdot 3 + 12 \cdot (-5) &= 0
\end{align*}

\item The greatest common divisor of 21 and $-6$ is 3.  Following are some linear combinations of 21 and $-6$.  Notice that $\gcd(21, -6)$ divides each of these linear combinations.
\begin{align*}
21 \cdot 1 + (-6) \cdot 1 &= 15  & 21 \cdot 2 + (-6) \cdot 1 &= 36 \\
21 \cdot 1 + (-6) \cdot (-1) &= 27 & 21 \cdot 2 + (-6) \cdot (-2) &= 54 \\
21 \cdot (-2) + (-6) \cdot 1 &= -48 & 21 \cdot (-2) + (-6) \cdot 3 &= -60 \\
21 \cdot 3 + (-6) \cdot 2 &= 51  &  21 \cdot 3 + (-6) \cdot (-5) &= 93
\end{align*}

\item \textbf{Proposition 4.15} \emph{Let $a$, $b$, and $t$ be integers with $t \ne 0$.  If $t$ divides $a$ and $t$ divides $b$, then for all integers $x$ and $y$, $t$ divides 
$(ax + by)$}.

\begin{myproof}
Let $a$, $b$, and  $d$  be integers, and assume that $d$  divides  $a$  and  $d$  divides  $b$.  We will prove that for all integers  $x$  and  $y$,  $d$  divides  $ax + by$.

So, let  $x \in \mathbb{Z}$ and let  $y \in Z$.  Since  $d$  divides  $a$ and $d$ divides $b$, there exist an integers  $m$ and $n$  such that
\[
a = md \qquad \text{ and } \qquad b = nd.
\]
We substitute the expressions for  $a$  and  $b$  given in these two equations into  $ax + by$.  This gives
\[
\begin{aligned}
  ax + by &= \left( {md} \right)x + \left( {nd} \right)y \\ 
          &= d\left( {mx + ny} \right). \\ 
\end{aligned} 
\]
By the closure properties of the integers,  $mx + ny$ is an integer, and hence we may conclude that  $d$  divides  $ax + by$.  Since  $x$  and  $y$  were chosen as arbitrary integers, we have proven that if  $d$  divides  $a$  and  $d$  divides  $b$, then for all integers  $x$  and  $y$,  $d$  divides  $ax + by$.
\end{myproof}

\item \textbf{Proposition} \emph{Let $a$ and $b$ be integers, not both zero and let 
$d = \gcd \left(a, b \right)$.  In addition, let $S$ and $T$ be the following sets:
\[
S = \left\{ ax + by \mid x, y \in \Z \right\} \qquad \text{and} \qquad 
T = \left\{ kd \mid k \in \Z \right\}.
\]
That is, $S$ is the set of all linear combinations of $a$ and $b$, and $T$ is the set of all multiples of the greatest common divisor of $a$ and $b$.  Then $S = T$}.

\begin{myproof}
Let $a$ and $b$ be integers, not both zero and let 
$d = \gcd \left(a, b \right)$.  In addition, let $S$ and $T$ be the following sets:
\[
S = \left\{ ax + by \mid x, y \in \Z \right\} \qquad \text{and} \qquad 
T = \left\{ kd \mid k \in \Z \right\}.
\]
We will prove that $S = T$ by proving that each set is a subset of the other set. We first let $c \in S$ so that $c = ax + by$ for some integers $x$ and $y$.  By Proposition~4.15, we know that $d$ divides $c$.  So there exists an integer $q$ such that $c = qd$.  This proves that $c \in T$ and hence, $S$ is a subset of $T$. 


We now note that by Theorem~8.8, $d$ is a linear combination of $a$ and $b$.  So, there exist integers $u$ and $v$ such that
\[
d = au + bv.
\]
Let $t \in T$.  So, there exists an integer $k$ such that $t = kd$.  We now use the fact that $d$ is a linear combination of $a$ and $b$ and write
\begin{align*}
t &= k \left( au + bv \right) \\
  &= kau + kbv \\
  &= a(ku) + b(kv)
\end{align*}
Since $ku$ and $kv$ are integers, this proves that $t \in S$ and hence, we have proven that $T$ is a subset of $S$.  Since we have proven that each set is a subset of the other, we have proven that $S = T$.
\end{myproof}


\end{enumerate}

\end{document}
