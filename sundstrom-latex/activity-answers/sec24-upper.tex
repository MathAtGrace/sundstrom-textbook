\documentclass[11pt]{article}
\usepackage{c://pctex/activity}

\lhead{}
\chead{\textbf{\large{Exercise 14 - Section 2.4\\Upper Bounds for a Subset of $\boldsymbol{\R}$}}}
\rhead{}
\lfoot{\emph{Mathematical Reasoning: Writing and Proof, Third Ed.} \\Ted Sundstrom}
\cfoot{}
\rfoot{\copyright \the\year\, by Pearson Education, Inc.\\}


\begin{document}
\begin{enumerate}
\item Let  $A$  be a subset of the real numbers.  A number  $b$  is called an upper bound for the set  $A$ provided that $\left( \forall x \in A\right) \left( x \leq b \right)$.

\item Three different upper bounds for the set  
$A = \left\{ {\left. {x \in \mathbb{R} } \right| 1 \leq x \leq 3} \right\}$
 are  3, $\pi $, and 20.

\item The set  $A = \left\{ {x \in \mathbb{R}\left.   \right| x > 0} \right\}$
 does not have an upper bound since for all real numbers  $b$, there exists an element  $a \in A$
 such that  $a > b$.

\item Three different real numbers that are not upper bounds for the set  
$A = \left\{ {\left. {x \in \mathbb{R} } \right| 1 \leq x \leq 3} \right\}$
 are 2.99, 1, and $-5$.

\item Let  $A$  be a subset of the real numbers.  A number  $b$  is not an upper bound for the set  $A$   provided that  $\left( \exists x \in A\right) \left( x > b \right)$.

\item Let  $A$  be a subset of the real numbers.  A number  $b$  is not an upper bound for the set  $A$  provided that there exist an element  $x$  in  $A$  such that  $x > b$.

\item The examples in Part (4) are consistent with  Part (6) as the following shows:

\begin{itemize}
\item The number 2.99  is not an upper bound for  $A$  since  $3 \in A$ and  $3 > 2.99$.

\item The number 1  is not an upper bound for  $A$  since  $3 \in A$ and  $3 > 1$.

\item The number $-5$  is not an upper bound for  $A$  since  $3 \in A$ and  $3 >  - 5$.
\end{itemize}

\end{enumerate}



\end{document}
