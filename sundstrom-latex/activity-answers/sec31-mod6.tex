\documentclass[11pt]{article}
\usepackage{c://pctex/activity}

\lhead{}
\chead{\textbf{\large{Exercise 20 - Section 3.1\\Congruence Modulo 6}}}
\rhead{}
\lfoot{\emph{Mathematical Reasoning: Writing and Proof, Third Ed.} \\Ted Sundstrom}
\cfoot{}
\rfoot{\copyright \the\year\, by Pearson Education, Inc.\\}


\begin{document}
\begin{enumerate}
\item The integers that are congruent to 5 modulo 6 are in the set  
$\left\{ { \ldots ,  - 13,  - 7,  - 1, 5, 11, 17,  \ldots } \right\}$.

\item For each integer  $m$  from Part (1), $m^2  \equiv 1 \pmod 6$.  

\item \textbf{Proposition}:  For all  $m \in \mathbb{Z}$, if  $m \equiv 5 \pmod 6$, then  
$m^2  \equiv 1 \pmod 6$.

\begin{myproof}
Let $m$ be an integer and assume that $m \equiv 5 \pmod 6$.  We will prove that 
$m^2  \equiv 1 \pmod 6$.  From the assumption, we know that 6 divides $(m - 5)$ and hence, there exists an integer $k$ such that $m - 5 = 6k$.  We can then conclude that 
$m = 5 + 6k$ and use algebra as follows:
\begin{align*}
m^2 - 1 &= \left( 5 + 6k \right)^2 - 1 \\
        &= \left( 25 + 60k + 36k^2 \right) - 1 \\
        &= 24 + 60k + 36k^2 \\
        &= 6 \left( 4 + 10k + 6k^2 \right)
\end{align*}
From the closure properties of the integers, we know that 
$\left( {4 + 10k + 6k^2 } \right)$ is an integer, and hence, the last equation proves that 6 divides $\left( m^2 - 1 \right)$, which in turn means that $m^2  \equiv 1 \pmod 6$.  This proves that if $m \equiv 5 \pmod 6$, then  $m^2  \equiv 1 \pmod 6$.
\end{myproof}



\vskip12pt
\noindent
Following is a know-show table for the proof:

\begin{center}
\begin{tabular}[h]{|p{0.4in}|p{2in}|p{2.5in}|}
  \hline
  \textbf{Step}  &  \textbf{Know}  &  \textbf{Reason}     \\ \hline
  $P$     &  $m \equiv 5 \pmod 6$     &  Hypothesis \\ \hline
  $P1$    &  $6 \mid \left( m - 5 \right)$  &  Definition of congruence modulo 6           \\ \hline
  $P2$    &  $\left( {\exists k \in \mathbb{Z}} \right)\left( {m - 5 = 6k} \right)$     & Definition of divides.   \\  \hline
  $P3$    &     $m = 5 + 6k$     & Algebra.   \\  \hline  
 % $P4$    &     $m^2  = 25 + 60k + 36k^2 $     & Algebra.   \\  \hline  
  $P4$    &     $m^2  - 1 = 24 + 60k + 36k^2$  &  Algebra  \\
          &     $m^2 - 1 = 6\left( {4 + 10k + 6k^2 } \right)$ &  \\ \hline
  $P5$    &     $\left( {4 + 10k + 6k^2 } \right) \in \mathbb{Z}$    & Closure properties of the integers.   \\  \hline  

  $Q1$    &     $6 \mid \left( {m^2 - 1} \right)$     & Definition of divides.   \\  \hline  
  
  $Q$     &  $m^2 \equiv 1 \pmod 6$  &  Definition of congruence modulo 6.    \\ \hline
\end{tabular}
\end{center}

\end{enumerate}
\end{document}
