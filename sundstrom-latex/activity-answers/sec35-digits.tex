\documentclass[11pt]{article}
\usepackage{c://pctex/activity}

\lhead{}
\chead{\textbf{\large{Exercise 23 -- Section 3.5\\The Last Two Digits of a Large Number}}}
\rhead{}
\lfoot{\emph{Mathematical Reasoning: Writing and Proof, Third Ed.} \\Ted Sundstrom}
\cfoot{}
\rfoot{\copyright \the\year\, by Pearson Education, Inc.\\}


\begin{document}
\begin{enumerate}
\item Since  $3^4  \equiv 81 \pmod{100}$, we can use Theorem 3.24 to conclude that
\[
\begin{aligned}
  3^4  \cdot 3^4  &\equiv 81 \cdot 81 \pmod{100} \\ 
  3^8  &\equiv 6561 \pmod{100} \\ 
  3^8  &\equiv 61 \pmod{100} \\ 
\end{aligned}
\]
We then square both sides of this congruence to obtain  
$3^{16}  \equiv 3721 \pmod{100}$ and hence that  $3^{16}  \equiv 21 \pmod{100}$.  This tells us that the last two digits in the decimal representation of  $3^{16} $are  21.

\item Since  $3^4  \equiv 81 \pmod{100}$ and  $3^{16}  \equiv 21 \pmod{100}$, we conclude that  
$3^4  \cdot 3^{16}  \equiv 81 \cdot 21 \pmod{100}$. Since  $81 \cdot 21 = 1701$, we see that this implies that  $3^{20}  \equiv 1 \pmod{100}$. This tells us that the last two digits in the decimal representation of  $3^{20} $ are  01.

\item $3^{20}  \equiv 1 \pmod{100}$ and so  
$\left( {3^{20} } \right)^{20}  \equiv 1^{20} \pmod{100}$  or  
$3^{400}  \equiv 1 \pmod{100}$. This tells us that the last two digits in the decimal representation of  $3^{400} $ are  01.

\item \begin{multicols}{3}
$4^2  \equiv 16 \pmod{100}$

$4^4  \equiv 56 \pmod{100}$

$4^8  \equiv 36 \pmod{100}$

$4^{16}  \equiv 96 \pmod{100}$

$4^{32}  \equiv 16 \pmod{100}$

$4^{64}  \equiv 56 \pmod{100}$

$4^{128}  \equiv 36 \pmod{100}$

$4^{256}  \equiv 96 \pmod{100}$

$4^{512}  \equiv 16 \pmod{100}$

\end{multicols}
Now,  $804 = 512 + 256 + 32 + 4$.  So,  
$4^{804}  = 4^{512}  \cdot 4^{256}  \cdot 4^{32}  \cdot 4^4 $.  This means that
\begin{align*}
  4^{804}  &\equiv 4^{512}  \cdot 4^{256}  \cdot 4^{32}  \cdot 4^4  \pmod{100} \hfill \\
  4^{804}  &\equiv 16 \cdot 96 \cdot 16 \cdot 56 \pmod{100} \hfill \\
  4^{804}  &\equiv 56 \pmod{100}. 
\end{align*}
This tells us that the last two digits in the decimal representation of  $4^{804} $
 are  56.

\item \begin{multicols}{3}
$3^2  \equiv 9 \pmod{100}$

$3^4  \equiv 81 \pmod{100}$

$3^8  \equiv 61 \pmod{100}$

$3^{16}  \equiv 21 \pmod{100}$

$3^{32}  \equiv 41 \pmod{100}$

$3^{64}  \equiv 81 \pmod{100}$

$3^{128}  \equiv 61 \pmod{100}$

$3^{256}  \equiv 21 \pmod{100}$

$3^{512}  \equiv 41 \pmod{100}$

$3^{1024} \equiv 81 \pmod{100}$

$3^{2048} \equiv 61 \pmod{100}$
\end{multicols}
Now,  $3356 = 2048 + 1024 + 256 + 16 + 8 + 4$.  So,  
$3^{3356}  = 3^{2048}  \cdot 3^{1024}  \cdot 3^{256}  \cdot 3^{16} \cdot 3^8 \cdot 3^4$.  This means that
\begin{align*}
  3^{3356}  &\equiv 3^{2048}  \cdot 3^{1024}  \cdot 3^{256}  \cdot 3^{16} \cdot 3^8 \cdot 3^4  \pmod{100} \hfill \\
  3^{3356}  &\equiv 61 \cdot 81 \cdot 21 \cdot 21 \cdot 61 \cdot 81 \pmod{100} \hfill \\
  3^{3356}  &\equiv 21 \pmod{100}. 
\end{align*}
This tells us that the last two digits in the decimal representation of  $3^{3356} $
 are  21.


\newpage
\item \begin{multicols}{3}
$7^2  \equiv 49 \pmod{100}$

$7^4  \equiv 1 \pmod{100}$

$7^8  \equiv 1 \pmod{100}$

$7^{16}  \equiv 1 \pmod{100}$

$7^{32}  \equiv 1 \pmod{100}$

$7^{64}  \equiv 1 \pmod{100}$

$7^{128}  \equiv 1 \pmod{100}$

$7^{256}  \equiv 1 \pmod{100}$
\end{multicols}
Now,  $403 = 256 + 128 + 16 + 2 + 1$.  So,  
$7^{403}  = 7^{256}  \cdot 7^{128}  \cdot 7^{16}  \cdot 7^2 \cdot 7^1$.  This means that
\begin{align*}
  7^{403}  &\equiv 7^{256}  \cdot 7^{128}  \cdot 7^{16}  \cdot 7^2 \cdot 7^1  \pmod{100} \hfill \\
  7^{403}  &\equiv 1 \cdot 1 \cdot 1 \cdot 49 \cdot 7 \pmod{100} \hfill \\
  7^{403}  &\equiv 43 \pmod{100}. 
\end{align*}
This tells us that the last two digits in the decimal representation of  $7^{403} $
 are  43.



\end{enumerate}

\end{document}
