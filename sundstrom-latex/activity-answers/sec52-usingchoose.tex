\documentclass[11pt]{article}
\usepackage{c://pctex/activity}

\lhead{}
\chead{\textbf{\large{Exercise 20 -- Section 5.2\\Using the Choose-an-Element Method in a Different Setting}}}
\rhead{}
\lfoot{\emph{Mathematical Reasoning: Writing and Proof, Third Ed.} \\Ted Sundstrom}
\cfoot{}
\rfoot{\copyright \the\year\, by Pearson Education, Inc.\\}


\begin{document}
\begin{enumerate}
\item Let  $a = 20$, $b = 12$, and  $d = 4$.  In this case,  $d \mid a$ and $d \mid b$. 
\begin{center}
\begin{tabular}[t]{| c | c | c | c |} \hline
$x$  &  $y$  &  $ax + by$  &	Does  $d$  divide  $ax + by$? \\ \hline
1  &	1  &	32  &	Yes \\ \hline
1  &	$-1$ &	8   &	Yes \\ \hline
2  &	2  &	64  &	Yes \\ \hline
2  &	$-3$ &	4   &	Yes \\ \hline
$-2$ &	3  &	$-4$  &	Yes \\ \hline
$-2$ &	$-5$ &	$-100$ & Yes \\ \hline
\end{tabular}
\end{center}

\item Let  $a = 21$, $b =  - 6$, and  $d = 3$. In this case,  $d \mid a$ and $d \mid b$.
\begin{center}
\begin{tabular}[t]{| c | c | c | c |} \hline
$x$  &  $y$  &  $ax + by$  &	Does  $d$  divide  $ax + by$? \\ \hline
1  &	1  &	15  &	Yes \\ \hline
1  &	$-1$ &	27  &	Yes \\ \hline
2  &	2  &	30  &	Yes \\ \hline
2  &	$-3$ &	60   &	Yes \\ \hline
$-2$ &	3  &	$-60$  &	Yes \\ \hline
$-2$ &	$-5$ &	$-72$ & Yes \\ \hline
\end{tabular}
\end{center}

\item \textbf{Proposition 5.16.}  Let $a$, $b$, and  $d$  be integers.  If  $d$  divides  $a$  and  $d$  divides  $b$, then for all integers  $x$  and  $y$,  $d$  divides  $ax + by$.

\begin{myproof}
Let $a$, $b$, and  $d$  be integers, and assume that $d$  divides  $a$  and  $d$  divides  $b$.  We will prove that for all integers  $x$  and  $y$,  $d$  divides  $ax + by$.

So, let  $x \in \mathbb{Z}$ and let  $y \in Z$.  Since  $d$  divides  $a$ and $d$ divides $b$, there exist an integers  $m$ and $n$  such that
\[
a = md \qquad \text{ and } \qquad b = nd.
\]
We substitute the expressions for  $a$  and  $b$  given in these two equations into  $ax + by$.  This gives
\[
\begin{aligned}
  ax + by &= \left( {md} \right)x + \left( {nd} \right)y \\ 
          &= d\left( {mx + ny} \right). \\ 
\end{aligned} 
\]
By the closure properties of the integers,  $mx + ny$ is an integer, and hence we may conclude that  $d$  divides  $ax + by$.  Since  $x$  and  $y$  were chosen as arbitrary integers, we have proven that if  $d$  divides  $a$  and  $d$  divides  $b$, then for all integers  $x$  and  $y$,  $d$  divides  $ax + by$.
\end{myproof}

\end{enumerate}




\end{document}
