\documentclass[11pt]{article}
\usepackage{c://pctex/activity}

\lhead{}
\chead{\textbf{\large{Exercise 14 -- Section 7.3\\Equivalence Relations on a Set of Matrices}}}
\rhead{}
\lfoot{\emph{Mathematical Reasoning: Writing and Proof, Third Ed.} \\Ted Sundstrom}
\cfoot{}
\rfoot{\copyright \the\year\, by Pearson Education, Inc.\\}
%\renewcommand{\labelenumi}{\textbf{\arabic{enumi}.}}
%\renewcommand{\labelenumii}{(\textbf{\alph{enumii}})}


\begin{document}
\noindent
Let $I$ be the identity matrix in $\mathcal{M}_{n, n}\left( \R \right)$.
\begin{enumerate}
\item For $A \in \mathcal{M}_{n, n}\left( \R \right)$, $A = IAI^{-1}$ and so $A \sim A$.  Hence, the relation $\sim$ is reflexive.  

Now let $A, B, C \in \mathcal{M}_{n, n}\left( \R \right)$.  First assume that $A \sim B$.  Then there exists an invertible matrix $P$ in $\mathcal{M}_{n, n}\left( \R \right)$ such that $B = PAP^{-1}$.  From linear algebra, we know that $P^{-1}$ is invertible and so
\begin{align*}
P^{-1}BP &= P^{-1} \left( PAP^{-1} \right) P \\
         &= \left( P^{-1}P \right) A \left( P^{-1} P \right) \\
         &= A.
\end{align*}
Hence, $B \sim A$ and the relation $\sim$ is symmetric.  Now assume that $A \sim B$ and 
$B \sim C$.  Then there exist invertible matrices $P$ and $Q$ in 
$\mathcal{M}_{n, n}\left( \R \right)$ such that $B = PAP^{-1}$ and $C = QBQ^{-1}$.  We then see that
\begin{align*}
C &= Q \left( PAP^{-1} \right) Q^{-1} \\
  &= \left(QP \right) A \left( P^{-1} Q^{-1} \right) \\
  &= \left( QP \right) A \left( QP \right)^{-1}.
\end{align*}
This proves that the relation $\sim$ is transitive and hence, $\sim$ is an equivalence relation.

\item Let $A, B, C \in \mathcal{M}_{n, n}\left( \R \right)$.  Since $\det(A) = \det(A)$, we see that $A \mathrel{R} A$ and $R$ is reflexive.  In addition, if $\det(A) = \det(B)$, then $\det(B) = \det(A)$.  This can be used to prove that $R$ is symmetric.  Finally, if 
$\det(A) = \det(B)$ and $\det(B) = \det(C)$, then $\det(A) = \det(C)$.  This can be used to prove that $R$ is transitive.

\item For this problem, $\sim$ is an equivalence relation on $\R$. Let $A, B, C \in \mathcal{M}_{n, n}\left( \R \right)$.  Since $\det(A) \in \R$ and $\sim$ is reflexive, 
$\det(A) \sim \det(A)$.  This proves that $\approx$ is reflexive on 
$\mathcal{M}_{n, n}\left( \R \right)$.  Now assume that $A \approx B$.  Then $\det(A) \sim \det(B)$.  Since $\sim$ is symmetric, we conclude that $\det(B) \sim \det(A)$ and hence, 
$B \approx A$.  This proves that $\approx$ is symmetric.

Finally, assume that $A \approx B$ and $B \approx C$.  Then $\det(A) \sim \det(B)$ and 
$\det(B) \sim \det(C)$.  Since $\sim$ is transitive, we can conclude that 
$\det(A) \sim \det(C)$.  Hence, $A \approx C$ and so $\approx$ is transitive.  This proves that $\approx$ is an equivalence relation on $\mathcal{M}_{n, n}\left( \R \right)$.
\end{enumerate}
\end{document}
