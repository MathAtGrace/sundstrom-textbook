\documentclass[11pt]{article}
\usepackage{c://pctex/activity}

\lhead{}
\chead{\textbf{\large{Exercise 11 -- Section 6.2\\Integration as a Function}}}
\rhead{}
\lfoot{\emph{Mathematical Reasoning: Writing and Proof, Third Ed.} \\Ted Sundstrom}
\cfoot{}
\rfoot{\copyright \the\year\, by Pearson Education, Inc.\\}


\begin{document}
\begin{enumerate}
\item \begin{enumerate}
\item For each  $f \in C_{\left[ {a, b} \right]} $, we can associate one real number given by 
$ dx$.  So we can define  $I:C_{\left[ {a, b} \right]}  \to \mathbb{R}$
by
\[
I\left( f \right) =  \int_a^b {f\left( x \right) dx}, \text{ for each }  
f \in C_{\left[ {a, b} \right]}.
\]

\item $I \left( f \right) =\int_0^2 {\left( x^2+1 \right)}  dx = \left. {\left( {\dfrac{1}{3}x^3  + x} \right)} \right|_0^2  = \dfrac{{14}}{3}$.

\item $I\left( g \right) = \int_0^2 \text{sin} \left( \pi x \right) dx = 
\left. {\left( {\dfrac{{ - 1}}{\pi }\cos \left( {\pi x} \right)} \right)} \right|_0^2  = 0$
\end{enumerate}

\item We determine that
\begin{align*}
\int {\left( {x^2  + 1} \right) dx} &= \dfrac{1}{3}x^3  + x + c  \text{ where } c 
\text{ is a real number and that} \\
\int {\cos \left( 2x \right) dx} &= \dfrac{1}{2}\sin \left( 2x \right) + c 
\text{ where } c  \text{ is a real number}.
\end{align*}
 

\item This does not define a function from  $C_{\left[ {0, 1} \right]} $ to  $T$  since  
$A\left( f \right)$ represents infinitely many functions that are antiderivatives of  $f$.

\item If  $f$  is continuous on the interval  $\left[ {a, b} \right]$, then by the Fundamental Theorem of Calculus,  the function  $g$  (where  
$g\left( x \right) =  \int_a^x {f\left( t \right) dt}$ is one particular antiderivative of  $f$.  That is, each input  $f \in C_{\left[ {a, b} \right]} $  has one output  $g$.  In this case,  $g'\left( x \right) = f\left( x \right)$  and  $g\left( a \right) = 0$.




\end{enumerate}



\end{document}
