\documentclass[11pt]{article}
\usepackage{c://pctex/activity}

\lhead{}
\chead{\textbf{\large{Exercise 19 -- Section 3.2\\Some Propositions about Congruence}}}
\rhead{}
\lfoot{\emph{Mathematical Reasoning: Writing and Proof, Third Ed.} \\Ted Sundstrom}
\cfoot{}
\rfoot{\copyright \the\year\, by Pearson Education, Inc.\\}


\begin{document}
\begin{enumerate}
\item The following biconditional statement is false.
\begin{list}{}
\item For each integer $a$, $a \equiv 3 \pmod 7$ if and only if $(a^2 + 5a) \equiv 3 \pmod 7$.
\end{list}

A counterexample is $a = 6$.  When $a = 6$, $a \not \equiv 3 \pmod 7$, but $a^2 + 5a = 66$ and 
$66 \equiv 3 \pmod 7$.

The given biconditional statement is logically equivalent to the following conjunction of two conditional statements:
\begin{list}{}
\item For each integer $a$, if $a \equiv 3 \pmod 7$, then $(a^2 + 5a) \equiv 3 \pmod 7$, and if $(a^2 + 5a) \equiv 3 \pmod 7$, then $a \equiv 3 \pmod 7$.
\end{list}
The counterexample $a = 6$ is also a counterexample for the conditional statement, ``For each integer $a$, if 
$(a^2 + 5a) \equiv 3 \pmod 7$, then $a \equiv 3 \pmod 7$.''


\newpar
However, the following conditional statement is true.
\begin{list}{}
\item For each integer $a$, if $a \equiv 3 \pmod 7$, then $(a^2 + 5a) \equiv 3 \pmod 7$.  
\end{list}
To prove this, if $a \equiv 3 \pmod 7$, then there exists an integer $k$ such that $a = 3 + 7k$.  We can then prove that
\begin{align*}
a^2 + 5a &= 24 + 77k + 49k^2\\
         & = 3 + 7(3 + 11k + 7k^2).
\end{align*}
Hence, $\left( a^2 + 5a \right) - 3 = 7 \left( 3 + 11k + 7k^2 \right)$ and since 
$\left( 3 + 11k + 7k^2 \right)$ is an integer, we conclude that 7 divides $\left( a^2 + 5a \right) - 3$.  This proves that $(a^2 + 5a) \equiv 3 \pmod 7$.

%  When $a = 6$, $a^2 + 5a = 66$ and 
%$66 \equiv 3 \pmod 7$.
%
%Since one of the two conditional statements in Part~(b) is false, the given proposition is false.


\newpage
\item The following biconditional statement is false.
\begin{list}{}
\item For each integer $a$, $a \equiv 2 \pmod 8$ if and only if $(a^2 + 4a) \equiv 4 \pmod 8$.
\end{list}

A counterexample is $a = 6$.  When $a = 6$, $a \not \equiv 2 \pmod 8$, but $a^2 + 4a = 60$ and 
$60 \equiv 4 \pmod 8$.

The given biconditional statement is logically equivalent to the following conjunction of two conditional statements:
\begin{list}{}
\item For each integer $a$, if $a \equiv 2 \pmod 8$, then $(a^2 + 4a) \equiv 4 \pmod 8$, and if $(a^2 + 4a) \equiv 4 \pmod 8$, then $a \equiv 2 \pmod 8$.
\end{list}
The counterexample $a = 6$ is also a counterexample for the conditional statement, ``For each integer $a$, if 
$(a^2 + 4a) \equiv 4 \pmod 8$, then $a \equiv 2 \pmod 8$.''


\newpar
However, the following conditional statement is true.
\begin{list}{}
\item For each integer $a$, if $a \equiv 2 \pmod 8$, then $(a^2 + 4a) \equiv 4 \pmod 8$.  
\end{list}
To prove this, if $a \equiv 2 \pmod 8$, then there exists an integer $k$ such that $a = 2 + 8k$.  We can then prove that
\begin{align*}
a^2 + 4a &= 12 + 48k + 64k^2\\
         & = 4 + 8(1 + 6k + 8k^2).
\end{align*}
Hence, $\left( a^2 + 4a \right) - 4 = 8 \left( 1 + 6k + 8k^2 \right)$ and since 
$\left( 1 + 6k + 8k^2 \right)$ is an integer, we conclude that 8 divides $\left( a^2 + 4a \right) - 4$.  This proves that $(a^2 + 4a) \equiv 4 \pmod 8$.
\end{enumerate}
\end{document}
