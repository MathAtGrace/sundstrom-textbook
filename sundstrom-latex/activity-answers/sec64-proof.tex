\documentclass[11pt]{article}
\usepackage{c://pctex/activity}

\lhead{}
\chead{\textbf{\large{Exercise 10 -- Section 6.4\\The Proof of Theorem 6.21}}}
\rhead{}
\lfoot{\emph{Mathematical Reasoning: Writing and Proof, Third Ed.} \\Ted Sundstrom}
\cfoot{}
\rfoot{\copyright \the\year\, by Pearson Education, Inc.\\}


\begin{document}
\begin{enumerate}
\item Let  $A$, $B$, and  $C$  be nonempty sets and let  $f:A \to B$  and  $g:B \to C$.  If  
$g \circ f:A \to C$  is an injection, then  $f:A \to B$  is an injection.

\textbf{\emph{Proof:}}  Let  $A$, $B$, and  $C$  be nonempty sets and let  $f:A \to B$  and  
$g:B \to C$.  Assume that $g \circ f$ is an injection.  To prove that $f$ is an injection, we let $x$ and $y$ be in $A$ and assume that $f \left( x \right) = f \left( y \right)$.  Since $f \left( x \right)$ is in $B$, we can use the function $g$ to conclude that
\[
\begin{aligned}
g \left( f \left( x \right) \right) &= g \left( f \left( y \right) \right) \\
\left( g \circ f \right) \left( x \right) &= \left( g \circ f \right) \left( y \right). \\
\end{aligned}
\]
Since $g \circ f$ is an injection, the last equation implies that $x = y$. Hence, $f$ is an injection.

\item Let  $A$, $B$, and  $C$  be nonempty sets and let  $f:A \to B$  and  $g:B \to C$.  If  
$g \circ f:A \to C$  is a surjection, then  $g:B \to C$  is a surjection.

\textbf{\emph{Proof:}}  Let  $A$, $B$, and  $C$  be nonempty sets and let  $f:A \to B$  and  
$g:B \to C$, and assume that $g \circ f$ is a surjection.  To prove that $g$ is a surjection, we let $z \in C$.  Now, since $g \circ f$ is a surjection, there exists an $x \in A$ such that
\[
\left( g \circ f \right) \left( x \right) = z,
\]
and this implies that $g \left( f \left( x \right) \right) = z$.  In addition, 
$f \left( x \right) \in B$ and hence using $y = f \left( x \right)$, we have proven that there exists an element $y \in B$ such that $g \left( y \right) = z$.  Therefore, $g$ is a surjection.
\end{enumerate}

\end{document}
