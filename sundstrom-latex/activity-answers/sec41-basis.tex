\documentclass[11pt]{article}
\usepackage{c://pctex/activity}

\lhead{}
\chead{\textbf{\large{Exercise 19 -- Section 4.1\\The Importance of the Basis Step}}}
\rhead{}
\lfoot{\emph{Mathematical Reasoning: Writing and Proof, Third Ed.} \\Ted Sundstrom}
\cfoot{}
\rfoot{\copyright \the\year\, by Pearson Education, Inc.\\}


\begin{document}
\setcounter{equation}{0}
Let  $P\left( n \right)$ be, $1 + 2 +  \cdots  + n = \dfrac{{n^2  + n + 1}}
{2}$.

\begin{enumerate}
\item Let  $k \in \mathbb{N}$.  Complete the following proof that if  $P\left( k \right)$ is true, then  $P\left( {k + 1} \right)$ is true.

Let  $k \in \mathbb{N}$.  Assume that  $P\left( k \right)$ is true.  That is, assume that
\begin{equation} \label{eq:act52a}
1 + 2 +  \cdots  + k = \frac{{k^2  + k + 1}}{2}.
\end{equation}

The goal is to prove that  $P\left( {k + 1} \right)$ is true.  That is, we need to prove that
\begin{equation} \label{eq:act52b}
1 + 2 +  \cdots  + k + \left( {k + 1} \right) = \frac{{\left( {k + 1} \right)^2  + \left( {k + 1} \right) + 1}}{2}.
\end{equation}

To do this, we add  $\left( {k + 1} \right)$ to both sides of Equation~(\ref{eq:act52a}).  This gives
\[
\begin{aligned}
  1 + 2 +  \cdots  + k + \left( {k + 1} \right) &= \frac{{k^2  + k + 1}}{2} + \left( {k + 1} \right) \\ 
               &= \frac{{k^2  + 3k + 3}}{2} \\ 
               &= \frac{{\left( {k^2  + 2k + 1} \right) + \left( {k + 2} \right)}}{2} \\ 
               &= \frac{{\left( {k + 1} \right)^2  + \left( {k + 1} \right) + 1}}{2} \\ 
\end{aligned}
\]
This proves that if  $P\left( k \right)$ is true, then  $P\left( {k + 1} \right)$ is true.

\item The propositions  $P\left( 1 \right)$, $P\left( 2 \right)$,  $P\left( 3 \right)$, and 
$P\left( 4 \right)$  are all false.  We have proved that the conditional statement for the inductive step is true, but since we have not proved a basis step, we cannot conclude that any of the open sentences $P(1)$, $P(2)$, $P(3)$, \ldots, $P(n)$, \ldots  are true.
\end{enumerate}



\end{document}
