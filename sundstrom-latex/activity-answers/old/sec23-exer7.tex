\documentclass[11pt]{article}
\usepackage{c://pctex/activity}

\lhead{}
\chead{\textbf{\large{Exercise 7 - Section 2.3\\Working with Truth Values of Statements}}}
\rhead{}
\lfoot{\emph{Mathematical Reasoning: Writing and Proof, Third Ed.} \\Ted Sundstrom}
\cfoot{}
\rfoot{\copyright 2009 by Pearson Education, Inc.\\}


\begin{document}
\begin{enumerate}
  \item The set of all odd integers is:
\begin{itemize}
\item not closed under addition.  For example, 3 and 5 are odd integers, but 
$3 + 5 = 8$, and 8 is not an odd integer.  
\item closed under multiplication.  In Theorem 1.7 in Section 1.2, we proved that the set of all odd integers is closed under multiplication.
\item not closed under subtraction.  For example, 3 and 5 are odd integers, but 
$5 - 3 = 2$, and 2 is not an odd integer.
\end{itemize}

  \item The set of all even integers is:
\begin{itemize}
\item closed under addition.  See Exercise~(2), Part~(b) from Section~1.2.
\item closed under multiplication.  This is a consequence of Exercise~(3), Part~(a) from Section~1.2.
\item closed under subtraction.  Although we have not proved this, the proof would be very similar to the proof that the set of even integers is closed under addition.  See Exercise~(2), Part~(b) from Section~1.2.
\end{itemize}

  \item The set $A = \{1, 4, 7, 10, 13, \ldots \}$ is:
\begin{itemize}
\item not closed under addition.  For example, 1 and 4 are in $A$, but $1 + 4 = 5$ and 5 is not in $A$.
\item appears to be closed under multiplication.  Any example of the product of two elements of $A$ will be in the set $A$.  We will formally prove such results in Chapter~3.  To see how to do this, it is a good idea to first write the set $A$ using set builder notation.
\[
A = \{ 3n + 1 \mid n \text{ is a nonnegative integer} \}.
\]
Then, if $a, b \in A$, then there exist non-negative integers $m$ and $n$ such that $a = 3m + 1$ and $b = 3n + 1$.  We then see that
\begin{align*}
ab &= (3m + 1)(3n + 1) \\
   &= 9mn + 3m + 3n + 1 \\
   &= 3(3mn + m + n) + 1
\end{align*}
Since $m$ and $n$ are nonnegative integers and the nonnegative integers are closed under addition and multiplication, we see that $(3mn + m + n)$ is a nonnegative integer and this proves that $ab \in A$.  Hence, the set $A$ is closed under multiplication.
\item not closed under subtraction.  For example, 1 and 4 are in $A$, but $4 -1 = 3$ and 3 is not in $A$.
\end{itemize}

  \item Any examples that are tried indicate that the set $B$ is closed under addition, multiplication, and subtraction.  To formally prove this, it is a good idea to first write the set $B$ using set builder notation.
\[
B = \{ 3n \mid n \in \Z \}.
\]
So if $a, b \in B$, then there exist integers $m$ and $n$ such that $a = 3m$ and $b = 3n$.  We then see that
\begin{align*}
a + b &= 3m + 3n = 3(m + n), \\
 a b  &= (3m)(3n) = 3(3mn), \\
a - b &= 3m - 3n = 3(m - n).
\end{align*}
In the first equation, $(m + n)$ is an integer and this proves the set $B$ is closed under addition.  In the second equation, $(3mn)$ is an integer and this proves the set $B$ is closed under multiplication. In the third equation, $(m - n)$ is an integer and this proves the set $B$ is closed under subtraction.

\item The set $C = \{ 3n + 1 \mid n \in \Z \}$ is not closed under addition, is closed under multiplication, and is not closed under subtraction.  The proofs of these results is similar to the proofs of the results for the set $A$ in Part~(c).  Notice that the only difference between the set $C$ and the set $A$ in Part~(c) is that the variable $n$ in the set builder notation can be any integer for $C$ but has to be a nonnegative integer for $A$.

\item The set $D = \left\{ \left. \dfrac{1}{2^n} \right| n \in \N \right\}$:
\begin{itemize}
\item not closed under addition.  For example, $\dfrac{1}{2}$ and $\dfrac{1}{4}$ are in $D$ but $\dfrac{1}{2} + \dfrac{1}{4} = \dfrac{3}{4}$ and $\dfrac{3}{4}$ is not in $D$.
\item is closed under multiplication.  If $a, b \in D$, then there exist natural numbers $m$ and $n$ such that 
$a = \dfrac{1}{2^m}$ and $b = \dfrac{1}{2^n}$. Then $ab = \dfrac{1}{2^m} \cdot \dfrac{1}{2^n} = \dfrac{1}{2^{m+n}}$.  Since $(m + n)$ is a natural number, this proves that $ab \in \N$.
\item not closed under addition.  For example, $\dfrac{1}{2}$ and $\dfrac{1}{8}$ are in $D$ but $\dfrac{1}{2} - \dfrac{1}{8} = \dfrac{3}{8}$ and $\dfrac{3}{8}$ is not in $D$.

\end{itemize}

\end{enumerate}



\end{document}
