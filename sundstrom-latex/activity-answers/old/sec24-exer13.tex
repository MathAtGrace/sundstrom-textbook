\documentclass[11pt]{article}
\usepackage{c://pctex/activity}

\lhead{}
\chead{\textbf{\large{Exercise 13 - Section 2.4\\Prime Numbers}}}
\rhead{}
\lfoot{\emph{Mathematical Reasoning: Writing and Proof, Third Ed.} \\Ted Sundstrom}
\cfoot{}
\rfoot{\copyright 2009 by Pearson Education, Inc.\\}


\begin{document}
\begin{enumerate}
\item Examples of natural numbers that are prime numbers are:  11, 13, 17, 19, 23, 29, 31, 37, 41, 43.

\item The conditional statement,``For all  $d \in \mathbb{N}$, if  $d$  is a factor of  $p$, then  $d = 1$ or  $d = p$'', means that the only natural number factors of  $p$  are  1  and  $p$.  Hence, this means that  $p$  is a prime number.

\item Examples of composite numbers are:  4, 6, 8, 9, 14, 15, 16, 18, 20, 21 since each of these numbers can be factored into a product of two natural numbers, neither of which is 1.

\item An integer  $n$ is a composite number provided that it is greater than one and that there exists a $d \in \mathbb{N}$  such that  $d$  is a factor of  $n$  and  $d \ne 1$ and  $d \ne n$.
\end{enumerate}


\end{document}
