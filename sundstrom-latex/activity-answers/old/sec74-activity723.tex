\documentclass[11pt]{article}
\usepackage{c://pctex/activity}

\lhead{}
\chead{\textbf{\large{Activity 7.23 -- Section 7.4\\Using Congruence Modulo 4}}}
\rhead{}
\lfoot{\emph{Mathematical Reasoning: Writing and Proof, Third Ed.} \\Ted Sundstrom}
\cfoot{}
\rfoot{\copyright 2009 by Pearson Education, Inc.\\}
\renewcommand{\labelenumi}{\textbf{\arabic{enumi}.}}
\renewcommand{\labelenumii}{(\textbf{\alph{enumii}})}


\begin{document}
\begin{enumerate}
\item If  $n \in \mathbb{Z}$, then in  $\mathbb{Z}_4 $, $\left[ n \right] = \left[ 0 \right]$,
$\left[ n \right] = \left[ 1 \right]$, $\left[ n \right] = \left[ 2 \right]$, or 
$\left[ n \right] = \left[ 3 \right]$.  So, using the multiplication table for  $\mathbb{Z}_4 $, we see that  $\left[ n \right]^2  = \left[ 0 \right]$  or  
$\left[ n \right]^2  = \left[ 1 \right]$.  Since  $\left[ n \right]^2  = \left[ {n^2 } \right]$, we conclude that  $\left[ {n^2 } \right] = \left[ 0 \right]$  or  
$\left[ {n^2 } \right] = \left[ 1 \right]$.

\item For each  $n \in \mathbb{Z}$,  $n^2  \equiv 0 \pmod 4$  or  $n^2  \equiv 1 \pmod 4$.

\item Since  $59 \equiv 3 \pmod 4$, we use the results in Part~(2) of Activity 7.23 to conclude that  \\
$104 257 833 259 \equiv 3 \pmod 4$.

\item The number  104,257,833,259 is not a perfect square since for each  $n \in \mathbb{Z}$,  
$n^2  \equiv 0 \pmod 4$  or  $n^2  \equiv 1 \pmod 4$, and  
$104,257,833,259 \equiv 3 \pmod 4$.
\end{enumerate}


\end{document}
