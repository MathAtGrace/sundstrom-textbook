\documentclass[11pt]{article}
\usepackage{c://pctex/activity}

\lhead{}
\chead{\textbf{\large{Exercise 13 -- Section 6.5\\Exploring Composite Functions}}}
\rhead{}
\lfoot{\emph{Mathematical Reasoning: Writing and Proof, Third Ed.} \\Ted Sundstrom}
\cfoot{}
\rfoot{\copyright 2009 by Pearson Education, Inc.\\}


\begin{document}
\begin{enumerate}
\item Draw the graph of $f \left( x \right) = 2x^3 - 7$ using $-3 \leq x \leq 3$ and 
$-60 \leq y \leq 40$.  Since each horizontal line hits the graph of  $f$  exactly once, the graph indicates that the function  $f$  is both an injection and surjection.

\newpar
To prove that $f$ is an injection, we let $a, b \in \R$ and assume that $f(a) = f(b)$.  This implies that 
$2a^3 - 7 = 2b^3 - 7$ and from this, we can conclude that $a^3 = b^3$.  We now take the cube root of both sides of this equation and obtain $a = b$.  This proves that $f$ is an injection.

\newpar
Now let $b \in \R$ (codomain).  To prove that $f$ is a surjection, we will prove that there exists an $a \in \R$ such that $f(a) = b$.  For this to be true, we need $2a^3 - 7 = b$.  If we solve this for $a$, we obtain 
$a = \sqrt[3]{\dfrac{b + 7}{2}}$.  Since $b \in \R$, we conclude that $a = \sqrt[3]{\dfrac{b + 7}{2}} \in \R$, and furthermore,
\begin{align*}
f(a) &= 2a^3 - 7 \\
     &= 2 \left( \sqrt[3]{\dfrac{b + 7}{2}} \right)^3 - 7 \\
     &= 2 \left( \dfrac{b + 7}{2} \right) - 7 \\
     &= (b + 7) - 7 \\
     &= b
\end{align*}
This proves that for each $b \in \R$, there exists an $a \in \R$ such that $f(a) = b)$ and hence, that $f$ is a surjection.


\item We can solve the equation $y = 2x^3 - 7$ for $y$ as follows:
\begin{align}
  y &= 2x^3  - 7  &  x^3  &= \frac{{y + 7}}{2} \notag \\
2x^3 &= y + 7     &    x &= \sqrt[3]{{\frac{{y + 7}}{2}}} \notag
\end{align}

So,  $f^{ - 1} :\mathbb{R} \to \mathbb{R}$ where  
$f^{ - 1} \left( y \right) = \sqrt[3]{{\dfrac{{y + 7}}{2}}}$.

\item Using the formulas for $f$ and $f^{-1}$, we see that:
\begin{align}
f^{ - 1} \left( {f\left( x \right)} \right) &= f^{ - 1} \left( {2x^3  - 7} \right)  & 
f\left( {f^{ - 1} \left( y \right)} \right) &= f\left( {\sqrt[3]{{\frac{{y + 7}}
{2}}}} \right) \notag\\
   &= \sqrt[3]{{\frac{{\left( {2x^3  - 7} \right) + 7}}{2}}} &
   &= 2\left( {\sqrt[3]{{\frac{{y + 7}}{2}}}} \right)^3  - 7 \notag \\
   &= \sqrt[3]{{x^3 }}  &     &= 2\left( {\frac{{y + 7}}{2}} \right) - 7 \notag \\
   &= x                 &     &= y \notag
\end{align}
\noindent
\textbf{Note}:  If we write  $y = f( x )$ for the function  $f$, we are using  $x$  as the independent variable (input) and  $y$  as the dependent variable (output).  In the case where  $f^{ - 1} $ is a function, we write  $x = f^{ - 1} ( y )$, and  $y$  is the independent variable for  $f^{ - 1} $.  However, when we are using real functions, we frequently want to graph them.  So we traditionally write each function as a function of  $x$.  Even if we have 
\[
f^{ - 1} ( y ) = \sqrt[3]{{\frac{{y + 7}}{2}}}, 
\]
we will frequently write 
\[
f^{ - 1} ( x ) = \sqrt[3]{{\frac{{x + 7}}{2}}}.
\]


\item If we solve the equation $y = e^{2x - 1}$ for $x$, we obtain $x = \dfrac{\ln(y) + 1}{2}$.

\item We can define $h: \R^+ \to \R$ by $h(y) = \dfrac{\ln(y) + 1}{2}$ for each $y \in \R^+$.

\item Let $x \in \R$ and let $y \in \R^+$.  We then see that
\begin{align*}
(h \circ g)(x) &= h \left( g(x) \right)  &  (g \circ h)(y) &= g \left( h(y) \right) \\
               &= h \left( e^{2x - 1} \right) &  &= g \left( \frac{\ln(y) + 1}{2} \right) \\
               &= \frac{\ln \left( e^{2x - 1} \right) + 1}{2} & &= e^{2 \left( \frac{\ln(y) + 1}{2} \right) - 1} \\
               &= \frac{(2x - 1) + 1)}{2} & &= e^{\left( \ln(y) + 1 \right) - 1} \\
               &= x   &  &= e^{\ln(y)} \\
               &      &  &= y
\end{align*}


\item By Part~(c) of Exercise~(6), we can conclude that $g$ and $h$ are bijections and that $h = g^{-1}$
\end{enumerate}

\end{document}
