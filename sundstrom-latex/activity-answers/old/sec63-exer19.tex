\documentclass[11pt]{article}
\usepackage{c://pctex/activity}

\lhead{}
\chead{\textbf{\large{Exercise 19 -- Section 6.3\\Functions Whose Domain is $\boldsymbol{\mathcal{M}_2(\R)}$}}}
\rhead{}
\lfoot{\emph{Mathematical Reasoning: Writing and Proof, Third Ed.} \\Ted Sundstrom}
\cfoot{}
\rfoot{\copyright 2009 by Pearson Education, Inc.\\}


\begin{document}
\begin{enumerate}
\item The determinant function is not an injection.  For example,
\[
\det \left[ {\begin{array}{*{20}c}
   1 & 0  \\
   0 & 1  \\
\end{array} } \right] =  \det \left[ {\begin{array}{*{20}c}
   1 & 2  \\
   0 & 1  \\
 \end{array} } \right].
\]
The determinant function is a surjection.  To prove this, let $a \in \mathbb{R}$.  Then
\[
\det \left[ {\begin{array}{*{20}c}
   a & 0  \\
   0 & 1  \\
\end{array} } \right] =  a.
\]

\item The transpose function is a bijection.  To prove it is an injection, let 
$\left[ {\begin{array}{*{20}c}
   a & b  \\
   c & d  \\
\end{array} } \right], \left[ {\begin{array}{*{20}c}
   p & q  \\
   r & s  \\
\end{array} } \right] \in \mathcal{M}_2(\R)$ and assume that
\[
\text{tran}\left[ {\begin{array}{*{20}c}
   a & b  \\
   c & d  \\
 \end{array} } \right] = \text{tran}\left[ {\begin{array}{*{20}c}
   p & q  \\
   r & s  \\
 \end{array} } \right].
\]
Then, $\left[ {\begin{array}{*{20}c}
   a & c  \\
   b & d  \\
\end{array} } \right] = \left[ {\begin{array}{*{20}c}
   p & r  \\
   q & s  \\
\end{array} } \right]$.  Therefore, $a = p$, $b = q$, $c = r$, and $d = s$ and hence, 
$\left[ {\begin{array}{*{20}c}
   a & b  \\
   c & d  \\
\end{array} } \right] = \left[ {\begin{array}{*{20}c}
   p & q  \\
   r & s  \\
\end{array} } \right]$.  To prove that the transpose function is a surjection, let 
$\left[ {\begin{array}{*{20}c}
   a & b  \\
   c & d  \\
\end{array} } \right] \in \mathcal{M}_2(\R)$.  Then,
\[
\text{tran}\left[ {\begin{array}{*{20}c}
   a & c  \\
   b & d  \\
 \end{array} } \right] = \left[ {\begin{array}{*{20}c}
   a & b  \\
   c & d  \\
 \end{array} } \right].
\]

\item The function $F$ is not an injection.  For example
\[
F \left[ {\begin{array}{*{20}c}
   0 & 0  \\
   0 & 0  \\
\end{array} } \right] =  0 \qquad \text{and} \qquad
F \left[ {\begin{array}{*{20}c}
   1 & 1  \\
   1 & 1  \\
\end{array} } \right] =  0.
\]
The function $F$ is a surjection.  To prove this, let $y \in \R$.  Consider three cases.
\begin{itemize}
\item If $y = 0$, then $F \left[ {\begin{array}{*{20}c}
   0 & 0  \\
   0 & 0  \\
\end{array} } \right] =  0 = y$.

\item If $y > 0$, then $\sqrt{y} \in \R$ and $F \left[ {\begin{array}{*{20}c}
   \sqrt{y} & 0  \\
   0 & 0  \\
\end{array} } \right] =  \left( \sqrt{y} \right)^2 = y$.

\item If $y < 0$, then $\sqrt{-y} \in \R$ and $F \left[ {\begin{array}{*{20}c}
   0 & \sqrt{-y}  \\
   0 & 0  \\
\end{array} } \right] =  -\left( \sqrt{-y} \right)^2 = y$.

\end{itemize}

\end{enumerate}


\end{document}
