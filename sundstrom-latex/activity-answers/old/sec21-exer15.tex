\documentclass[11pt]{article}
\usepackage{c://pctex/activity}

\lhead{}
\chead{\textbf{\large{Exercise 15 - Section 2.1\\Working with Truth Values of Statements}}}
\rhead{}
\lfoot{\emph{Mathematical Reasoning: Writing and Proof, Third Ed.} \\Ted Sundstrom}
\cfoot{}
\rfoot{\copyright 2009 by Pearson Education, Inc.\\}




\begin{document}

\vskip20pt
\noindent
Suppose that $P$ and $Q$ are true statements, that $U$ and $V$ are false statements, and that $W$ is a statement and it is not known if $W$ is true or false.

\begin{enumerate}
  \item $(P \vee Q) \vee (U \wedge W)$ is true.

Since $P$ and $Q$ are true, $(P \vee Q)$ is true, and hence, $(P \vee Q) \vee (U \wedge W)$ is true

  \item It is not possible to determine if $P \wedge (Q \to W)$ is true or false.

Since $Q$ is true, we cannot tell if $Q \to W$ is true or false.  Hence, we cannot tell if $P \wedge (Q \to W)$ is true or false.
  \item $P \wedge (W \to Q)$ is true.

Since $W$ is false and $Q$ is true, $(W \to Q)$ is true.  Since $P$ is also true, $P \wedge (W \to Q)$ is true.
  \item It is not possible to determine if $W \to (P \wedge U)$ is true or false.

Since $P$ is true and $U$ is false, $(P \wedge U)$ is false.  However, since we do not know if $W$ is true or false, we cannot determine if $W \to (P \wedge U)$ is true or false.
  \item $W \to (P \wedge \mynot U)$ is true.

Since $P$ is true and $U$ is false, $(P \wedge \mynot U)$ is true.  Hence, even though we do not know if $W$ is true or false, we can conclude that $W \to (P \wedge \mynot U)$ is true.

  \item $(\mynot P \vee \mynot U) \wedge (Q \vee \mynot V)$ is true.

Since $P$ is true and $U$ is false, $(\mynot P \vee \mynot U)$ is true.  Since $Q$ is true and $V$ is false, 
$(Q \vee \mynot V)$ is true.  Therefore, $(\mynot P \vee \mynot U) \wedge (Q \vee \mynot V)$ is true.

  \item $(P \wedge \mynot V) \wedge (U \vee W)$
  \item $(P \vee \mynot Q) \to (U \wedge W)$
  \item $(P \vee W) \to (U \wedge W)$
  \item $(U \wedge \mynot V) \to (P \wedge W)$

\end{enumerate}





\end{document}
