\documentclass[11pt]{article}
\usepackage{c://pctex/activity}

\lhead{}
\chead{\textbf{\large{Exercise 12 -- Section 7.1\\The Proof of Theorem 7.5}}}
\rhead{}
\lfoot{\emph{Mathematical Reasoning: Writing and Proof, Third Ed.} \\Ted Sundstrom}
\cfoot{}
\rfoot{\copyright \the\year\, by Pearson Education, Inc.\\}


\begin{document}
\noindent
\textbf{Theorem 7.5}.  Let  $R$  be a relation from the set  $A$  to the set  $B$.  Then,
\begin{enumerate}
\item The domain of  $R^{ - 1} $ is the range of  $R$.  That is, 
$\text{dom}\left( {R^{ - 1} } \right) = \text{range}\left( R \right)$.

\item The range of  $R^{ - 1} $  is the domain of  $R$.   That is, 
$\text{range}\left( {R^{ - 1} } \right) = \text{dom}\left( R \right)$.

\item The inverse of  $R^{ - 1} $  is  R.  That is, $\left( {R^{ - 1} } \right)^{ - 1}  = R$.
\end{enumerate}

\begin{myproof}
Let  $R$  be a relation from the set  $A$  to the set  $B$.  For Part (1), we will first prove that  $\text{dom}\left( {R^{ - 1} } \right) \subseteq \text{range}\left( R \right)$.  So, let  
$y \in \text{dom}\left( {R^{ - 1} } \right)$.  This means that there exists an  $x \in A$ such that
\[
\left( {y, x} \right) \in R^{ - 1}.
\]
Using the definition of the inverse relation, this means that  $\left( {x, y} \right) \in R$, and hence,  $y \in \text{range}\left( R \right)$.  This proves that  
$\text{dom}\left( {R^{ - 1} } \right) \subseteq \text{range}\left( R \right)$.

Now let  $y \in \text{range}\left( R \right)$.  Then, there exists an  $x \in A$ such that  
$\left( {x, y} \right) \in R$.  But this means that  
\[
\left( {y, x} \right) \in R^{ - 1} 
\]
and hence,  $y \in \text{dom}\left( {R^{ - 1} } \right)$.  Consequently,  
$\text{range}\left( R \right) \subseteq \text{dom}\left( {R^{ - 1} } \right)$ and hence, 
$\text{dom}\left( {R^{ - 1} } \right) = \text{range}\left( R \right)$.
\vskip6pt

We will now prove Part (2) that  
$\text{range}\left( {R^{ - 1} } \right) = \text{dom}\left( R \right)$.  This will also be done by proving that each set is a subset of the other.  So, let  
$x \in \text{range}\left( {R^{ - 1} } \right)$. This means that there exists a  $y \in B$ such that  
\[
\left( {y, x} \right) \in R^{ - 1}.
\]
This in turn means that
\[
\left( {x, y} \right) \in R,
\]
and hence that  $x \in \text{dom}\left( R \right)$.  This proves that  
$\text{range}\left( {R^{ - 1} } \right) \subseteq \text{dom}\left( R \right)$. 

Now let $x \in \text{dom}\left( R \right)$.  Then, there exists a  $y \in B$ such that  
$\left( {x, y} \right) \in R$.  Consequently,  
\[
\left( {y, x} \right) \in R^{ - 1} 
\]
and so, $x \in \text{range}\left( {R^{ - 1} } \right)$.  This proves that  
$\text{dom}\left( R \right) \subseteq \text{range}\left( {R^{ - 1} } \right)$ and hence that \\ $\text{range}\left( {R^{ - 1} } \right) = \text{dom}\left( R \right)$.
\vskip6pt

Part (3) is proven using the following set equalities:

\[
\begin{aligned}
\left( {R^{ - 1} } \right)^{ - 1}  &= \left\{ {\left. {\left( {x, y} \right) \in A \times B} \right| \left( {y, x} \right) \in R^{ - 1} } \right\} \\ 
                                   &= \left\{ {\left. {\left( {x, y} \right) \in A \times B} \right| \left( {x, y} \right) \in R} \right\} \\ 
                                   &= R  \\
\end{aligned} 
\]
\end{myproof}

\end{document}
