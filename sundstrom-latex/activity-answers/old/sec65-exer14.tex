\documentclass[11pt]{article}
\usepackage{c://pctex/activity}

\lhead{}
\chead{\textbf{\large{Exercise 14 -- Section 6.5\\The Inverse Sine Function}}}
\rhead{}
\lfoot{\emph{Mathematical Reasoning: Writing and Proof, Third Ed.} \\Ted Sundstrom}
\cfoot{}
\rfoot{\copyright 2009 by Pearson Education, Inc.\\}


\begin{document}
\begin{enumerate}
\item Let $f: \R \to \R$ be defined by $f(x) = \sin x$.  Since the range of the sine function is the closed interval $[-1, 1]$, the function $f$ is not a surjection.  Since the sine function is periodic, the function $f$ is not an injection.  Consequently, the inverse of $f$ is not a function.

\item Let $F:\left[ {- \dfrac{{\pi }}{2}, \dfrac{\pi }{2}} \right] \to \left[ { - 1, 1} \right]$
be defined by $F\left( x \right) = \sin x$.  For all 
$x, t \in \left[ {- \dfrac{{\pi }}{2}, \dfrac{\pi }{2}} \right]$, if $x \ne t$, then 
$\sin x \ne \sin t$, and so, the function $F$ is an injection.  Also, for each 
$y \in [-1, 1]$, there exists an $x \in \left[ {- \dfrac{{\pi }}{2}, \dfrac{\pi }{2}} \right]$ such that $\sin x = y$.  Hence, $F$ is a surjection and so, $F$ is a bijection.

\item The domain of the inverse sine function is the closed interval $[-1, 1]$, and the range of the inverse sine function is the closed interval 
$\left[ {- \dfrac{{\pi }}{2}, \dfrac{\pi }{2}} \right]$.

\item We know that if $F$ is a bijection (and so $F^{-1}$ is a function), then
\begin{equation}
F^{-1}(y) = x \text{ if and only if } F(x) = y.
\end{equation}
For this particular function
\begin{center}
$F(x) = y$ means that $y = \sin x$ and $-\dfrac{\pi}{2} \leq x \leq \dfrac{\pi}{2}$,
\end{center}
and that
\begin{center}
$F^{-1}(y) = x$ means that $\text{Sin}^{-1}y = x$.
\end{center}
Using this information in equation~(1), we see that
\begin{center}
$\text{Sin}^{-1}y = x$ if and only if $y = \sin x$ and 
$-\dfrac{\pi}{2} \leq x \leq \dfrac{\pi}{2}$.
\end{center}
\end{enumerate}

\end{document}
