\section*{Exercises 6.3}
%
\begin{enumerate}
%\item Let  $f\xA \to B$.  \label{exer:sec63-injection}
%\begin{enumerate}
%  \item Carefully explain what it means to say that the function  $f$  is an injection.
%
%  \item Carefully explain what it means to say that the function  $f$  is not an injection.
%\end{enumerate}
%
%\item Let  $f\xA \to B$.  \label{exer:sec63-surjection}
%\begin{enumerate}
%  \item The function  $f$  is a surjection provided that 
%$\text{range}( f ) = \text{codom}( f )$.  In terms of the elements of the sets  $A$  and  $B$, carefully explain what it means to say that the function  $f$  is a surjection.
%
%  \item The function  $f$  is not a surjection provided that 
%$\text{range}( f ) \ne \text{codom}( f )$.  In terms of the elements of the sets  $A$  and  $B$, carefully explain what it means to say that the function  $f$  is not a surjection.
%\end{enumerate}

\item \begin{enumerate} \item Draw an arrow diagram that represents a function that is an injection but is not a surjection.

  \item Draw an arrow diagram that represents a function that is an injection and is a surjection.

  \item Draw an arrow diagram that represents a function that is not an injection and is not a surjection.

  \item Draw an arrow diagram that represents a function that is not an injection but is a surjection.

  \item Draw an arrow diagram that represents a function that is not a bijection.
\end{enumerate}

\item \label{exer:sec63-mod5function} We know  $R_5  = \left\{ {0, 1, 2, 3, 4} \right\}$ and $R_6  = \left\{ {0, 1, 2, 3, 4, 5} \right\}$.  For each of the following functions, determine if the function is an injection and determine if the function is a surjection.  Justify all conclusions.

\begin{enumerate}
\yitem $f\x R_5  \to R_5 $  by $f( x ) = x^2 + 4 \pmod 5$, for all 
$x \in R_5$

\item $g\x R_6  \to R_6$  by $g( x ) = x^2 + 4 \pmod 6$, for all 
$x \in R_6$

\yitem $F\x R_5  \to R_5 $  by $F( x ) = x^3 + 4 \pmod 5$, for all 
$x \in R_5$
\end{enumerate}

\item For each of the following functions, determine if the function is an injection and determine if the function is a surjection.  Justify all conclusions.\label{exer:sec63-5}

\begin{enumerate}
  \yitem $f\x\mathbb{Z} \to \mathbb{Z}$ defined by $f( x ) = 3x + 1$, for all $x \in \Z$.  

  \yitem $F\x\mathbb{Q} \to \mathbb{Q}$ defined by  $F( x ) = 3x + 1$, for all $x \in \Q$.  

  \item $g\x\mathbb{R} \to \mathbb{R}$ defined by  $g( x ) = x^3 $, for all $x \in \R$.  

  \item $G \x\mathbb{Q} \to \mathbb{Q}$ defined by  $G( x ) = x^3 $, for all $x \in \Q$.

  \item $k \x \R \to \R$ defined by $k(x) = e^{-x^2}$, for all $x \in \R$.

  \item $K\x \R^* \to \R$ defined by $K(x) = e^{-x^2}$, for all $x \in \R^*$.  \\Note:  $\R^* = \left\{ x \in \R \mid x \geq 0 \right\}$.
  
  \item $K_1 \x \R^* \to T$ defined by $K_1(x) = e^{-x^2}$, for all $x \in \R^*$, where $T = \left\{ y \in \R \mid 0 < y \leq 1 \right\}$.

  \yitem $h \x\R \to \R$ defined by $h ( x ) = \dfrac{2x}{x^2 + 4}$, for all $x \in \R$. 

  \item $H \x \{x \in \R \mid x \geq 0 \} \to \left\{ y \in \R \left| 0 \leq y \leq \dfrac{1}{2} \right. \right\}$ defined by \linebreak  
$H ( x ) = \dfrac{2x}{x^2 + 4}$, for all $x \in \{x \in \R \mid x \geq 0 \}$.
\end{enumerate} 

\item \label{exer:forexample} For each of the following functions, determine if the function is a bijection.  Justify all conclusions.
\begin{enumerate}
\yitem $F\x\mathbb{R} \to \mathbb{R}$  defined by  $F( x ) = 5x + 3$, for all  $x \in \mathbb{R}$.  

\yitem $G\x\Z \to \Z$  defined by  $G( x ) = 5x + 3$, for all  $x \in \Z$.

\item $f\x ( \R -\left\{ 4 \right\} ) \to \R$ defined  by $f ( x ) = \dfrac{3x}{x - 4}$, for all 
          $x \in \left( \R - \{ 4 \} \right)$.  

\item $g\x ( \R -\left\{ 4 \right\} ) \to ( \R - \left\{ 3 \right\} )$ defined by $g ( x ) = \dfrac{3x}{x - 4}$,  for all \linebreak$x \in \left( \R - \{ 4 \} \right)$.   
\end{enumerate}

%\item Let $f\x\R \to \R$ be defined by $f ( x ) = \dfrac{2x - 1}{x^2 + 4}$.  Is the function $f$ an injection?  Is the function $f$ a surjection?  Justify your conclusions.

%\item \begin{enumerate}
%\item Let $f\x \R \to \R$ be defined by $f(x) = e^{-x^2}$.  Is the function $f$ and injection?  Is the function $f$ a surjection?  Justify your conclusions.
%
%\item Let $\R^* = \left\{ x \in \R \mid x \geq 0 \right\}$.  Let $g\x \R^* \to \R$ be defined by $g(x) = e^{-x^2}$.  Is the function $g$ and injection?  Is the function $g$ a surjection?  Justify your conclusions.
%
%\item Let $T = \left\{ y \in \R \mid 0 < y \leq 1 \right\}$.  Let $h\x \R^* \to T$ be defined by $h(x) = e^{-x^2}$.  Is the function $h$ and injection?  Is the function $h$ a surjection?  Justify your conclusions.
%\end{enumerate}


%\item We often say that a function is a \textbf{piecewise defined function}
%\index{function!piecewise defined}%
%\index{piecewise defined function}%
%\label{exer:piecewisefunction}%
% if it has different rules for determining the output for different parts of its domain.  For example, we can define a function $f\x\R \to \R$ by giving a rule for calculating $f(x)$ when $x \geq 0$ and giving a rule for calculating $f(x)$ when $x < 0$ as follows:
%\begin{equation}
%f(x) = 
%\begin{cases}
%x^2 + 1,    &  \text{if $x \geq 0$;} \\
%x - 1      &  \text{if $x < 0$.}
%\end{cases} \notag
%\end{equation}
%\begin{enumerate}
%\item Sketch a graph of the function $f$\!.
%\item Is the function $f$ and injection?  Is the function $f$ a surjection?  Justify your conclusions.
%\end{enumerate}
%
%\item For each of the following functions, determine if the function is an injection and determine if the function is a surjection.  Justify all conclusions.
%\label{exer:piecewisefunction2}%
%
%$$
%\BeginTable
%\BeginFormat
%| p(2in) | p(2in) |
%\EndFormat
%" \textbf{(a) } $g\x[0, 1] \to (0, 1)$ by
%\begin{equation}
%g(x) = 
%\begin{cases}
%0.8,      &  \text{if $x = 0$;} \\
%0.5 x,    &  \text{if $0 < x < 1$;} \\
%0.6       &  \text{if $x = 1$.}
%\end{cases} \notag
%\end{equation} " 
%\textbf{(b) } $h\x\Z \to \left\{ 0, 1 \right\}$ by
%\begin{equation}
%h(x) = 
%\begin{cases}
%0,        & \text{if $x$ is even;} \\
%1,        & \text{if $x$ is odd.}
%\end{cases} \notag
%\end{equation} " \\
%\EndTable
%$$

%\begin{enumerate}
%\begin{multicols}{2}
%\item $g\x[0, 1] \to (0, 1)$ by
%\begin{equation}
%g(x) = 
%\begin{cases}
%0.8,      &  \text{if $x = 0$;} \\
%0.5 x,    &  \text{if $0 < x < 1$;} \\
%0.6       &  \text{if $x = 1$.}
%\end{cases} \notag
%\end{equation}
%
%\item $h\x\Z \to \left\{ 0, 1 \right\}$ by
%\begin{equation}
%h(x) = 
%\begin{cases}
%0,        & \text{if $x$ is even;} \\
%1,        & \text{if $x$ is odd.}
%\end{cases} \notag
%\end{equation}
%\end{multicols}
%\end{enumerate}



\item Let $s\x\mathbb{N} \to \mathbb{N}$, where for each  $n \in \mathbb{N}$,  
$s( n )$ is the sum of the distinct natural number divisors of  $n$.  This is the \textbf{sum of the divisors function}
\index{sum of divisors function}%
 that was introduced in \typeu Activity~\ref*{PA:otherfunctions} from Section~\ref{S:introfunctions}.  Is  $s$  an injection?  Is  $s$  a surjection?  Justify your conclusions.
\label{exer:sec63-sumdivisors}

\item Let  $d\x\mathbb{N} \to \mathbb{N}$,  where  $d( n )$ is the number of natural number divisors of  $n$.  This is the \textbf{number of divisors function}
\index{number of divisors function}%
 introduced in Exercise~(\ref{exer:numberofdivisors}) from Section~\ref{S:introfunctions}.  Is the function  $d$  an injection?  Is the function  $d$  a surjection?  Justify your conclusions.
\label{exer:sec63-numdivisors}

\xitem In \typeu Activity~\ref*{PA:otherfunctions} from Section~\ref{S:introfunctions} , we introduced the \textbf{birthday function}.
\index{birthday function}%
%  Its definition is repeated here.
%
%\noindent
%Let  $b$  be the function that assigns to each person his or her birthday (month and day).  The domain of the function  $b$  is the set of all people and the codomain of  $b$  is the set of all days in a leap year (i.e., January 1 through December 31, including February 29).
Is the birthday function an injection?  Is it a surjection?  Justify your conclusions. \label{exer:sec63-9}

\item \label{exer:sec63-10}
\begin{enumerate}
\item Let $f\x\mathbb{Z} \times \mathbb{Z} \to \mathbb{Z}$ be defined by  
$f( {m, n} ) = 2m + n$.  Is the function  $f$  an injection?  Is the function  $f$  a surjection?  Justify your conclusions.

\item Let $g\x\mathbb{Z} \times \mathbb{Z} \to \mathbb{Z}$ be defined by  
$g( {m, n} ) = 6m + 3n$.  Is the function  $g$  an injection?  Is the function  $g$  a surjection?  Justify your conclusions.
\end{enumerate}


\xitem \label{exer:sec63-11} \begin{enumerate} \item Let 
$f\x\mathbb{R} \times \mathbb{R} \to \mathbb{R} \times \mathbb{R}$ be defined by  
$f( {x, y} ) = ( {2x, x + y} )$. Is the function  $f$  an injection?  Is the function  $f$  a surjection?  Justify your conclusions.

  \item Let $g\x\mathbb{Z} \times \mathbb{Z} \to \mathbb{Z} \times \mathbb{Z}$ be defined by  $g( {x, y} ) = ( {2x, x + y} )$. Is the function  $g$  an injection?  Is the function  $g$  a surjection?  Justify your conclusions.
\end{enumerate}

\item Let $f\x\R \times \R \to \R$ be the function defined by $f (x, y ) = -x^2y+3y$, for all $(x, y ) \in \R \times \R$.  Is the function $f$ an injection?  Is the function $f$ a surjection?  Justify your conclusions.


\item Let $g\x\R \times \R \to \R$ be the function defined by 
$g (x, y ) = ( x^3 + 2) \sin y$, for all 
$(x, y ) \in \R \times \R$.  Is the function $g$ an injection?  Is the function $g$ a surjection?  Justify your conclusions.


\item Let  $A$  be a nonempty set.  The \textbf{identity function on the set}
\index{identity function}%
  $\boldsymbol{A}$, denoted by  $I_A $, is the function  $I_A \x A \to A$ defined by  $I_A ( x ) = x$
for every  $x$  in  $A$.  Is  $I_A $ an injection?  Is  $I_A $ a surjection?  Justify your conclusions.

\item Let  $A$  and  $B$  be two nonempty sets.  Define  
\[
p_1 \x A \times B \to A \text{ by }  p_1 ( {a, b} ) = a
\]
for every  $( {a, b} ) \in A \times B$.  This is the \textbf{first projection function}
\index{projection function}%
 introduced in Exercise~(\ref{sym:projfunc}) in Section~\ref{S:moreaboutfunctions}.

\begin{enumerate}
  \item Is the function  $p_1 $ a surjection? Justify your conclusion.

  \item If  $B = \left\{ b \right\}$, is the function  $p_1 $ an injection?  Justify your conclusion.

  \item Under what condition(s) is the function  $p_1 $ not an injection?  Make a conjecture and prove it.
\end{enumerate}




\item Define $f:\N \to \Z$ as follows: For each $n \in \N$,
\[
f(n) = \dfrac{1 + (-1)^n (2n - 1)}{4}.
\]
Is the function $f$ an injection?  Is the function $f$ a surjection?  Justify your conclusions.

\newpar
\textbf{Suggestions}.  Start by calculating several outputs for the function before you attempt to write a proof.  In exploring whether or not the function is an injection, it might be a good idea to use cases based on whether the inputs are even or odd.  In exploring whether $f$ is a surjection, consider using cases based on whether the output is positive or less than or equal to zero.


\item Let $C$ be the set of all real functions that are continuous on the closed interval 
$\left[ 0, 1 \right]$.  Define the function $A\x C \to \mathbb{R}$ as follows:
For each $f \in C$, 
\[
A ( f ) = \int_0^1 {f( x ) \, dx}.
\]
Is the function $A$ an injection?  Is it a surjection?  Justify your conclusions.


\item Let $A = \left\{ ( m, n ) \mid m \in \mathbb{Z}, n \in \mathbb{Z}, \text{ and } n \ne 0 \right\}$.  Define $f\x A \to \mathbb{Q}$ as follows:

\begin{list}{}
\item For each $( m, n ) \in A$, $f ( m, n ) = \dfrac{m+n}{n}$.
\end{list}

\begin{enumerate}
\item Is the function $f$ an injection?  Justify your conclusion.
\item Is the function $f$ a surjection?  Justify your conclusion.
\end{enumerate}


\item \textbf{Evaluation of proofs}  \hfill \\
See the instructions for Exercise~(\ref{exer:proofeval}) on 
page~\pageref{exer:proofeval} from Section~\ref{S:directproof}.

\begin{enumerate}
\item \textbf{Proposition}. The function $f\x\R \times \R \to \R \times \R$ defined by  
$f( {x, y} ) = ( {2x + y, x - y} )$ is an injection. 
\
\begin{myproof}
For each $(a, b)$ and $(c, d)$ in $\R \times \R$, if $f(a, b) = f(c, d)$, then
\[
(2a + b, a - b) = (2c + d, c - d).
\]
We will use systems of equations to prove that $a = c$ and $b = d$.
\begin{align*}
2a + b &= 2c + d \\
 a - b &=  c - d \\
   3a  &= 3c \\
    a  &= c
\end{align*}
Since $a = c$, we see that
\[
(2c + b, c - b) = (2c + d, c - d).
\]
So $b = d$.  Therefore, we have proved that the function $f$ is an injection.
\end{myproof}


\item \textbf{Proposition}. The function $f\x\R \times \R \to \R \times \R$ defined by  
$f( {x, y} ) = ( {2x + y, x - y} )$ is a surjection. 

\begin{myproof}
We need to find an ordered pair such that $f(x, y) = (a, b)$ for each $(a, b)$ in 
$\R \times \R$.  That is, we need
$(2x + y, x - y) = (a, b)$, or
\[
2x + y = a \qquad \text{and} \qquad x - y = b.
\]
Treating these two equations as a system of equations and solving for $x$ and $y$, we find that
\[
x = \frac{a + b}{3} \qquad \text{and} \qquad y = \frac{a - 2b}{3}.
\]
Hence, $x$ and $y$ are real numbers, $(x, y) \in \R \times \R$, and
\begin{align*}
f(x, y) &= f \left( \frac{a + b}{3}, \frac{a - 2b}{3} \right) \\
        &= \left(2 \left( \frac{a + b}{3} \right) + \frac{a - 2b}{3}, \frac{a + b}{3} - \frac{a - 2b}{3} \right) \\
        &= \left( \frac{2a + 2b + a - 2b}{3}, \frac{a + b - a + 2b}{3} \right) \\
        &= \left( \frac{3a}{3}, \frac{3b}{3} \right) \\
        &= (a, b).
\end{align*}
Therefore, we have proved that for every $(a, b) \in \R \times \R$, there exists an 
$(x, y) \in \R \times \R$ such that $f(x, y) = (a, b)$.  This proves that the function $f$ is a surjection.
\end{myproof}
\end{enumerate}
\end{enumerate}
%\hbreak


\subsection*{Explorations and Activities}
\setcounter{oldenumi}{\theenumi}
\begin{enumerate} \setcounter{enumi}{\theoldenumi}
%  \item \textbf{Injections, Surjections, and Bijections}. \label{A:injsurjbi} 
%For each of the following functions, use a graphing calculator to construct a table of values (using at least five different elements of the domain) and if appropriate, draw the graph of the function.  Based on this information, make a conjecture as to whether the function is an injection, surjection, or bijection, and then prove your conjecture.
%
%\begin{multicols}{2}
%\begin{enumerate}
%%\item $f\x \mathbb{R} \to \mathbb{R}$ by  $f( x ) = e^{ - x}$
%
%\item $f\x \mathbb{R} \to ( {0, + \infty } )$ by  \\$f( x ) = e^{ - x}$
%
%\item $f\x \mathbb{R} \to \mathbb{R}$  by  $f( x ) = e^{ - x^2 } $
%
%\item $s\x \mathbb{Z} \to \mathbb{Z}$  by  $s( x ) = 2x + 1$
%
%\item $f\x \mathbb{R} \to \left[ {0, + \infty } \right)$ by  \\$f( x ) = x^2$
%
%\item $f\x \left[ {0, + \infty } \right) \to \left[ {0, + \infty } \right)$  by \\ 
%$f( x ) = x^2 $
%
%\item $g\x \mathbb{R} \to \mathbb{R}$  by  \\$g( x ) = \sqrt[3]{{2x - 3}}$
%
%
%\item $f\! \x \! \left[ {\dfrac{{ - \pi }}{2}, \dfrac{\pi }{2}} \right] \to \left[ { - 1,\;1} \right]$
% by  $f( x ) = \sin x$
%
%%\item $g:\mathbb{Z} \to \left\{ {0, 1} \right\}$ by  \\
%%$g( x ) = \left\{ \begin{gathered}
%%  0\text{ if }x\text{ is even} \hfill \\
%%  1\text{ if }x\text{ is odd} \hfill \\ 
%%\end{gathered}  \right.$
%\end{enumerate}
%\end{multicols}
%%
%For example, if $f\x \mathbb{R} \to \mathbb{R}$ by  $f( x ) = e^{ - x}$, then $f$ is an injection but is not a surjection.  Since this is a real function, a graph can be used to suggest these facts.  To prove that it is an injection, we assume that   $s, t \in \mathbb{R}$ with  $f( s ) = f( t )$.  Then
%\[
%e^{ - s}  = e^{ - t}.
%\]
%Taking the natural logarithm of both sides of the equation yields
%\[
%\begin{aligned}
%  \ln \!\left( {e^{ - s} } \right) &= \ln \!\left( {e^{ - t} } \right) \\ 
%                                - s&=  - t \\ 
%                                 s &= t. \\ 
%\end{aligned}
%\]
%This proves that  $f$  is an injection.  Also, since  $e^{ - x}  > 0$ for all  
%$x \in \mathbb{R}$, we can conclude that  $f( x ) > 0$ for all  
%$x \in \mathbb{R}$.  This means that the range of  $f$  is not equal to  $\mathbb{R}$
%and hence  $f$  is not a surjection.


\item \textbf{Piecewise Defined Functions}.  We often say that a function is a \textbf{piecewise defined function}
\index{function!piecewise defined}%
\index{piecewise defined function}%
\label{exer:piecewisefunction}%
 if it has different rules for determining the output for different parts of its domain.  For example, we can define a function $f\x\R \to \R$ by giving a rule for calculating $f(x)$ when $x \geq 0$ and giving a rule for calculating $f(x)$ when $x < 0$ as follows:
\begin{equation}
f(x) = 
\begin{cases}
x^2 + 1,    &  \text{if $x \geq 0$;} \\
x - 1      &  \text{if $x < 0$.}
\end{cases} \notag
\end{equation}
\begin{enumerate}
\item Sketch a graph of the function $f$\!.  Is the function $f$ an injection?  Is the function $f$ a surjection?  Justify your conclusions. 
\end{enumerate}

For each of the following functions, determine if the function is an injection and determine if the function is a surjection.  Justify all conclusions.
\label{exer:piecewisefunction2}%

$$
\BeginTable
\BeginFormat
| p(2in) | p(2in) |
\EndFormat
" \textbf{(b) } $g\x[0, 1] \to (0, 1)$ by
\begin{equation}
g(x) = 
\begin{cases}
0.8,      &  \text{if $x = 0$;} \\
0.5 x,    &  \text{if $0 < x < 1$;} \\
0.6       &  \text{if $x = 1$.}
\end{cases} \notag
\end{equation} " 
\textbf{(c) } $h\x\Z \to \left\{ 0, 1 \right\}$ by
\begin{equation}
h(x) = 
\begin{cases}
0,        & \text{if $x$ is even;} \\
1,        & \text{if $x$ is odd.}
\end{cases} \notag
\end{equation} " \\
\EndTable
$$

  

\item \textbf{Functions Whose Domain is $\boldsymbol{\mathcal{M}_2(\R)}$}.  Let  $\mathcal{M}_2 ( \R )$ represent the set of all  2 by 2  matrices over  $\mathbb{R}$.  
\begin{enumerate}
\item Define  $\det \x\mathcal{M}_2( \R ) \to \mathbb{R}$  \label{sym:detfunc} by  
\[
\det \left[ {\begin{array}{*{20}c}
   a & b  \\
   c & d  \\
 \end{array} } \right] = ad - bc.
\]
This is the \textbf{determinant function}
\index{determinant}%
 introduced in Exercise~(\ref{exer:determinant}) from Section~\ref{S:moreaboutfunctions}.  
\label{exer:sec63-det}%
Is the determinant function an injection?  Is the determinant function a surjection? Justify your conclusions. 

\item Define  
$\text{tran}\x\mathcal{M}_2( \R ) \to \mathcal{M}_2 ( \R )$  
\label{sym:tranfunc} by  
\[
\text{tran}\left[ {\begin{array}{*{20}c}
   a & b  \\
   c & d  \\
 \end{array} } \right] = A^T  = \left[ {\begin{array}{*{20}c}
   a & c  \\
   b & d  \\

 \end{array} } \right].
\]
This is the \textbf{transpose function}
\index{transpose}%
 introduced in Exercise~(\ref{exer:transpose}) from Section~\ref{S:moreaboutfunctions}. 
\label{exer:sec63-trans}
Is the transpose function an injection?  Is the transpose function a surjection? Justify your conclusions.

\item Define $F \x\mathcal{M}_2( \R ) \to \mathbb{R}$  \label{sym:detfunc} by  
\[
F \left[ {\begin{array}{*{20}c}
   a & b  \\
   c & d  \\
 \end{array} } \right] = a^2 + d^2 - b^2 - c^2.
\]
Is the function $F$ an injection?  Is the function $F$ a surjection? Justify your conclusions.
\end{enumerate} 

\end{enumerate}
\hbreak

\endinput

\item Let $S$ be the set of all real functions that are differentiable and let $T$ be the set of all real functions.  Let  $D$ be the derivative function [i.e., $D\x S \to T$, where for each 
$f \in S$, $D ( f ) = f'$].  Is the function $D$ an injection?  Is it a surjection?  Justify your conclusions.
\hbreak
