\section*{Exercises 7.2}
%
\begin{enumerate}
\xitem Let  $A = \left\{ {a, b} \right\}$ and let  
$R = \left\{ {\left( {a, b} \right)} \right\}$. \label{exer:sec72-1}
Is $R$ an equivalence relation on $A$?  If not, is $R$ reflexive, symmetric, or transitive?  Justify all conclusions.

%\begin{enumerate}
%  \item Is  $R$  a reflexive relation on  $A$?  
%
%  \item Is  $R$  a symmetric relation?  
%
%  \item Is  $R$  a transitive relation?  
%
%  \item Is  $R$  an equivalence relation on  $A$?  
%\end{enumerate}
\item Let  $A = \left\{ {a, b, c} \right\}$.  For each of the following, draw a directed graph that represents a relation with the specified properties. \label{exer:sec72examples}

\begin{enumerate}
  \item A relation on  $A$  that is symmetric but not transitive

  \item A relation on  $A$  that is transitive but not symmetric

  \item A relation on  $A$  that is symmetric and transitive but not reflexive \linebreak on  $A$

  \item A relation on  $A$  that is not reflexive on  $A$, is not symmetric, and is not         transitive

  \item A relation on  $A$, other than the identity relation, that is an equivalence relation on          $A$
\end{enumerate}


\xitem Let  $A = \left\{ {1, 2, 3, 4, 5} \right\}$.  The identity relation on  $A$  is  
\[
I_A  = \left\{ {( {1, 1} ), ( {2, 2} ), ( {3, 3} ), ( {4, 4} ), ( {5, 5} )} \right\}.
\]
Determine an equivalence relation on  $A$  that is different from  $I_A $ or explain why this is not possible. \label{exer71:notidentity}


\xitem Let  
$R = \left\{ { {\left( {x, y} \right) \in \mathbb{R} \times \mathbb{R} } \mid \left| x \right| + \left| y \right| = 4} \right\}$.  Then  $R$  is a relation on  $\mathbb{R}$. 
\label{exer:sec72-2}
Is $R$ an equivalence relation on $\R$?  If not, is $R$ reflexive, symmetric, or transitive?  Justify all conclusions.

\item A relation $R$ is defined on $\Z$ as follows:  For all $a, b \in \Z$, $a \mathrel{R} b$ if and only if $\left| a - b \right| \leq 3$.  Is $R$ an equivalence relation on $\R$?  If not, is $R$ reflexive, symmetric, or transitive?  Justify all conclusions.
\label{exer:sec72-absvalue}%

%\begin{enumerate}
%  \item Is  $R$  a reflexive relation on  $\mathbb{R}$?  
%
%  \item Is  $R$  a symmetric relation?  
%
%  \item Is  $R$  a transitive relation?  
%
%  \item Is  $R$  an equivalence relation on  $\mathbb{R}$?  
%\end{enumerate}




\xitem Let  $f\x \mathbb{R} \to \mathbb{R}$ be defined by  $f( x ) = x^2  - 4$
for each  $x \in \mathbb{R}$.  Define a relation  $\sim$  on  $\mathbb{R}$ as follows:
\label{exer:equivrelwithfunction}%
\[
\text{For }  a, b \in \R,  a \sim b \text{ if and only if }  
f( a ) = f( b ).
\]
%\begin{center}
%For  $a, b \in \mathbb{R}$,  $a \sim b$ if and only if  
%$f\left( a \right) = f\left( b \right)$.
%\end{center}
\begin{enumerate}
  \item Is the relation  $\sim$  an equivalence relation on  $\mathbb{R}$?  Justify your         conclusion.  \label{exer:equivrelwithfunctiona}

  \item Determine all real numbers in the set $C = \left\{ { {x \in \R } \mid x \sim 5} \right\}$.
\end{enumerate}

\item Repeat Exercise~(\ref{exer:equivrelwithfunction}) using the function  
$f\x \mathbb{R} \to \mathbb{R}$ that is defined by 
$f( x ) = x^2  - 3x - 7$ for each  
$x \in \mathbb{R}$.  \label{exer:equivrelwithfunction2}

\item \begin{enumerate}
\item Repeat Exercise~(\ref{exer:equivrelwithfunctiona}) using the function 
$f\x \mathbb{R} \to \mathbb{R}$ that is defined by  $f( x ) = \sin x$ for each  
$x \in \mathbb{R}$.

\item Determine all real numbers in the set $C = \left\{ { {x \in \R } \mid x \sim \pi } \right\}$.
\end{enumerate}

\item Define the relation $\sim$ on $\Q$ as follows:  For $a, b \in \Q$, $a \sim b$ if and only if $a - b \in \Z$. \label{exer:sec72-diffQ}  In Progress Check~\ref{prog:example-equiv}, we showed that the relation $\sim$ is an equivalence relation on $\Q$.

\begin{enumerate}
  \item List  four  different elements of the set  
        $C = \left\{ { {x \in \Q } \mid x \sim \dfrac{5}{7} } \right\}$.

  \item Use set builder notation (without using the symbol $\sim$) to specify the set $C$.

  \item Use the roster method to specify the set $C$.
\end{enumerate}

\item \label{exer:oneequivonenot} Let $\sim$ and $\approx$ be relations on $\Z$ defined as follows:

\begin{itemize}
\yitem For $a, b \in \Z$, $a \sim b$ if and only if 2 divides $a + b$.
\item For $a, b \in \Z$, $a \approx b$ if and only if 3 divides $a + b$.
\end{itemize}

\begin{enumerate}
\item Is $\sim$ an equivalence relation on $\Z$?  If not, is this relation reflexive, symmetric, or transitive?
\item Is $\approx$ an equivalence relation on $\Z$?  If not, is this relation reflexive, symmetric, or transitive?
\end{enumerate}


\item Let  $U$  be a finite, nonempty set and let  $\mathcal{P}( U )$ be the power set of  $U$.  That is, $\mathcal{P} ( U )$ is the set of all subsets of $U$.  Define the relation  $\sim$  on $\mathcal{P}( U )$  as follows:  For  $A, B \in \mathcal{P}( U )$,
 $A \sim B$  if and only if  $A \cap B = \emptyset $.	That is,  the ordered pair  
$\left( {A, B} \right)$  is in the relation  $\sim$  if and only if  $A$  and  $B$  are disjoint.

\noindent
Is the relation  $\sim$  an equivalence relation on  $\mathcal{P}( U )$?  If not, is it reflexive, symmetric, or transitive?  Justify all conclusions.



\item Let $U$ be a nonempty set and let $\mathcal{P} ( U )$ be the power set of $U$.  That is, 
$\mathcal{P} \left( U \right)$ is the set of all subsets of $U$. 
\label{exer:sec72-powerset}

For $A$ and $B$ in $\mathcal{P} ( U )$, define $A \sim B$ to mean that there exists a bijection 
$f:A \to B$.  Prove that $\sim$ is an equivalence relation on 
$\mathcal{P} ( U )$.

\hint  Use results from Sections~\ref{S:compositionoffunctions} 
and~\ref{S:inversefunctions}.


\item Let $\sim$ and $\approx$ be relations on $\Z$ defined as follows: 
\label{exer:modequivrel}

\begin{itemize}
\item For $a, b \in \Z$, $a \sim b$ if and only if $2a + 3b \equiv 0 \pmod 5$.
\item For $a, b \in \Z$, $a \approx b$ if and only if $a + 3b \equiv 0 \pmod 5$.
\end{itemize}

\begin{enumerate}
\item Is $\sim$ an equivalence relation on $\Z$?  If not, is this relation reflexive, symmetric, or transitive?
\item Is $\approx$ an equivalence relation on $\Z$?  If not, is this relation reflexive, symmetric, or transitive?
\end{enumerate}

\item Let $\sim$ and $\approx$ be relations on $\R$ defined as follows: \label{exer72-signed}

\begin{itemize}
\item For $x, y \in \R$, $x \sim y$ if and only if $xy \geq 0$.
\item For $x, y \in \R$, $x \approx y$ if and only if $xy \leq 0$.
\end{itemize}

\begin{enumerate}
\item Is $\sim$ an equivalence relation on $\R$?  If not, is this relation reflexive, symmetric, or transitive?
\item Is $\approx$ an equivalence relation on $\R$?  If not, is this relation reflexive, symmetric, or transitive?
\end{enumerate}

\item Define the relation  $ \approx $ on  $\mathbb{R} \times \mathbb{R}$ as follows:  
\label{exer:sec72-circles}%
For  $\left( {a, b} \right), \left( {c, d} \right) \in \mathbb{R} \times \mathbb{R}$,  $\left( {a, b} \right) \approx \left( {c, d} \right)$ if and only if 
$a^2  + b^2  = c^2  + d^2 $.
%\vskip10pt

\begin{enumerate}
  \item Prove that   $ \approx $ is an equivalence relation on  $\mathbb{R} \times \mathbb{R}$.

  \item List  four  different elements of the set 
  \[
C = \left\{ { {\left( {x, y} \right) \in \mathbb{R} \times \mathbb{R} } \mid                 \left( {x, y} \right) \approx \left( {4, 3} \right)} \right\}.
\]
  \yitem Give a geometric description of the set  $C$.
\end{enumerate}


\item \textbf{Evaluation of proofs}  \hfill \\
See the instructions for Exercise~(\ref{exer:proofeval}) on 
page~\pageref{exer:proofeval} from Section~\ref{S:directproof}.

\begin{enumerate}
\item \textbf{Proposition}. Let $R$ be a relation on a set $A$.  If $R$ is symmetric and transitive, then $R$ is reflexive.

\begin{myproof}
Let $x, y \in A$.  If $x \mathrel{R} y$, then $y \mathrel{R} x$ since $R$ is symmetric.  Now, 
$x \mathrel{R} y$ and $y \mathrel{R} x$, and since $R$ is transitive, we can conclude that 
$x \mathrel{R} x$.  Therefore, $R$ is reflexive. 
\end{myproof}

\item \textbf{Proposition}. Let $\sim$ be a relation on $\Z$ where for all $a, b \in \Z$,  
$a \sim b$ if and only if $\left( a + 2b \right) \equiv 0 \pmod 3$.  The relation 
$\sim$ is an equivalence relation on $\Z$.

\begin{myproof}
Assume $a \sim a$.  Then $\left( a + 2a \right) \equiv 0 \pmod 3$ since \linebreak
$\left( 3a \right) \equiv 0 \pmod 3$. Therefore, $\sim$ is reflexive on $\Z$.  In addition, if 
$a \sim b$, then $\left( a + 2b \right) \equiv 0 \pmod 3$, and if we multiply both sides of this congruence by 2, we get
\begin{align*}
2\left( a + 2b \right) &\equiv 2 \cdot 0 \pmod 3 \\
\left( 2a + 4b \right) &\equiv 0 \pmod 3 \\
\left( 2a + b \right) &\equiv 0 \pmod 3 \\
\left( b + 2a \right) &\equiv 0 \pmod 3. 
\end{align*}
This means that $b \sim a$ and hence, $\sim$ is symmetric.

We now assume that $\left( a + 2b \right) \equiv 0\pmod 3$ and 
$\left( b + 2c \right) \equiv 0\pmod 3$.  By adding the corresponding sides of these two congruences, we obtain
\begin{align*}
\left( a + 2b \right) + \left( b + 2c \right) &\equiv 0 + 0 \pmod 3 \\
\left( a + 3b + 2c \right) &\equiv 0 \pmod 3 \\
\left( a + 2c \right) &\equiv 0 \pmod 3.
\end{align*}
Hence, the relation $\sim$ is transitive and we have proved that $\sim$ is an equivalence relation on $\Z$.
\end{myproof}
\end{enumerate}
\end{enumerate}


\subsection*{Explorations and Activities}
\setcounter{oldenumi}{\theenumi}
\begin{enumerate} \setcounter{enumi}{\theoldenumi} 
\item \textbf{Other Types of Relations}.  In this section, we focused on the properties of a relation that are part of the definition of an equivalence relation.  However, there are other properties of relations that are of importance.  We will study two of these properties in this activity.

A relation $R$ on a set $A$ is a \textbf{circular relation}
\index{circular relation} %
\index{relation!circular} % 
 provided that for all $x$, $y$, and $z$ in $A$, if $x \mathrel{R} y$ and $y \mathrel{R} z$, then $z \mathrel{R} x$.  
\begin{enumerate}
  \item Carefully explain what it means to say that a relation $R$ on a set $A$ is not circular.
  \item Let $A = \{ 1, 2, 3 \}$.  Draw a directed graph of a relation on $A$ that is circular and draw a directed graph of a relation on $A$ that is not circular.
  \item Let $A = \{ 1, 2, 3 \}$.  Draw a directed graph of a relation on $A$ that is circular and not transitive and draw a directed graph of a relation on $A$ that is transitive and not circular.
  \item Prove the following proposition:
\begin{list}{}
\item A relation $R$ on a set $A$ is an equivalence relation if and only if it is reflexive and circular.
\end{list}
\end{enumerate}


A relation $R$ on a set $A$ is an \textbf{antisymmetric relation}
\index{antisymmetric relation} %
\index{relation!antisymmetric} %
 provided that for all $x, y \in A$, if $x \mathrel{R} y$ and $y \mathrel{R} x$, then $x = y$.  
\begin{enumerate} \setcounter{enumii}{4} 
  \item Carefully explain what it means to say that a relation on a set $A$ is not antisymmetric.
  \item Let $A = \{ 1, 2, 3 \}$.  Draw a directed graph of a relation on $A$ that is antisymmetric and draw a directed graph of a relation on $A$ that is not antisymmetric.
\item Are the following propositions true or false?  Justify all conclusions.
\begin{itemize}
\item If a relation $\mathrel{R}$ on a set $A$ is both symmetric and antisymmetric, then 
$\mathrel{R}$ is transitive.
\item If a relation $\mathrel{R}$ on a set $A$ is both symmetric and antisymmetric, then 
$\mathrel{R}$ is reflexive.
\end{itemize}
\end{enumerate}

\end{enumerate}


\hbreak

%\markboth{Chapter~\ref{C:equivrelations}. Equivalence Relations}{\ref{S:equivclasses}. Equivalence Classes}

\endinput
