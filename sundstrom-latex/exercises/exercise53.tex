\section*{Exercises for Section~\ref{S:setproperties}}
%
\begin{enumerate}
\item Let  $A$ be a subset of some universal set  $U$\!.  Prove each of the following (from Theorem~\ref{T:propsofcomplements}):\label{exer:sec43-1}
\begin{multicols}{2}
  \begin{enumerate}
    \yitem $\left( A^c \right)^c = A$
    \item $A - \emptyset = A$
    \yitem $\emptyset^c = U$
    \item $U^c = \emptyset$
  \end{enumerate}
\end{multicols}

%\item Proposition~\ref{P:subsetandcomp} states the following:
%
%\begin{list}{}
%\item Let  $A$  and  $B$  be subsets of some universal set  $U$.  If  $A \subseteq B$, then  $B^c  \subseteq A^c $.
%\end{list}
%\vskip10pt
%This was proven in Activity~\ref{A:usingchoose}.  Now, prove the following proposition: \label{exer:sec43-2}
%
%\begin{list}{}
%\item Let  $A$  and  $B$  be subsets of some universal set  $U$.   Then  $A \subseteq B$ if and only if   $B^c  \subseteq A^c $.
%\end{list}

\xitem Let  $A$, $B$,  and  $C$  be subsets of some universal set  $U$\!.	As part of 
Theorem~\ref{T:algebraofsets}, we proved one of the distributive laws.  Prove the other one.  That is, prove that
\label{exer:distributive}%
\[
A \cap \left( {B \cup C} \right) = \left( {A \cap B} \right) \cup \left( {A \cap C} \right)\!.
\]

\item Let  $A$, $B$,  and  $C$  be subsets of some universal set  $U$.	As part of 
Theorem~\ref{T:propsofcomplements}, we proved one of De Morgan's Laws.  Prove the other one.  That is, prove that
\label{exer:demorgan}%
\[
\left( {A \cap B} \right)^c  = A^c  \cup B^c.
\]

\item Let  $A$, $B$,  and  $C$  be subsets of some universal set  $U$\!.
\label{exer:sec43-5}%
\begin{enumerate}
  \yitem Draw two general Venn diagrams for the sets  $A$, $B$, and  $C$.  On one, shade the region that represents  $A - \left( {B \cup C} \right)$, and on the other, shade the region that represents  $\left( {A - B} \right) \cap \left( {A - C} \right)$.  Based on the Venn diagrams, make a conjecture about the relationship between the sets  $A - \left( {B \cup C} \right)$  and  $\left( {A - B} \right) \cap \left( {A - C} \right)$.
\label{exer:sec43-setdiff}%

  \item Use the choose-an-element method to prove the conjecture from Exercise~(\ref{exer:sec43-setdiff}).

  \yitem Use the algebra of sets to prove the conjecture from Exercise~(\ref{exer:sec43-setdiff}).
\end{enumerate}




\item Let  $A$, $B$,  and  $C$  be subsets of some universal set  $U$\!.
\label{exer:sec43-6}%
\begin{enumerate}
  \item Draw two general Venn diagrams for the sets  $A$, $B$, and  $C$.  On one, shade the region that represents  $A - \left( {B \cap C} \right)$, and on the other, shade the region that represents  $\left( {A - B} \right) \cup \left( {A - C} \right)$.  Based on the Venn diagrams, make a conjecture about the relationship between the sets  $A - \left( {B \cap C} \right)$  and  $\left( {A - B} \right) \cup \left( {A - C} \right)$.
\label{exer:sec43-setdiff2}%

  \item Use the choose-an-element method to prove the conjecture from 
Exercise~(\ref{exer:sec43-setdiff2}).

  \item Use the algebra of sets to prove the conjecture from 
Exercise~(\ref{exer:sec43-setdiff2}).
\end{enumerate}


%\item Prove or disprove each of the following propositions:
%\begin{enumerate}
%\item If $A$, $B$,  and  $C$  are subsets of some universal set  $U$\!, then 
%\[
%\left( A \cap B \right) - C = \left( A - C \right) \cap \left( B - C \right).
%\]
%
%\item If $A$ and  $B$  are subsets of some universal set  $U$\!, then 
%\[
%\left( A \cup B \right) - \left( A \cap B \right) = \left( A - B \right) \cup 
%\left(B - A \right).
%\]
%\end{enumerate}

\item Let $A$, $B$,  and  $C$  be subsets of some universal set  $U$\!.  \label{exer53:exer6} Prove or disprove each of the following:
\begin{enumerate}
\yitem $\left( A \cap B \right) - C = \left( A - C \right) \cap \left( B - C \right)$ 

\item $\left( A \cup B \right) - \left( A \cap B \right) = \left( A - B \right) \cup 
\left(B - A \right)$
\end{enumerate}

\item Let  $A$, $B$,  and  $C$  be subsets of some universal set  $U$\!. 
\label{exer:sec43-setdiff3x}%
\begin{enumerate}
  \item Draw two general Venn diagrams for the sets  $A$, $B$, and  $C$.  On one, shade the region that represents  $A - \left( {B - C} \right)$, and on the other, shade the region that represents  $\left( {A - B} \right) - C$.  Based on the Venn diagrams, make a conjecture about the relationship between the sets  $A - \left( {B - C} \right)$  and  $\left( {A - B} \right)  - C$.  (Are the two sets equal?  If not, is one of the sets a subset of the other set?)
\label{exer:sec43-setdiff3}%

  \item Prove the conjecture from Exercise~(\ref{exer:sec43-setdiff3}).
\end{enumerate}


\item Let  $A$, $B$,  and  $C$  be subsets of some universal set  $U$\!. 
\label{exer:sec43-setdiff3x}%
\begin{enumerate}
  \item Draw two general Venn diagrams for the sets  $A$, $B$, and  $C$.  On one, shade the region that represents  $A - \left( {B - C} \right)$, and on the other, shade the region that represents  
$\left( {A - B} \right) \cup \left( A - C^c \right)$.  Based on the Venn diagrams, make a conjecture about the relationship between the sets  $A - \left( {B - C} \right)$  and  
$\left( {A - B} \right) \cup \left( A - C^c \right)$.  (Are the two sets equal?  If not, is one of the sets a subset of the other set?)
\label{exer:sec43-setdiff3a}%

  \item Prove the conjecture from Exercise~(\ref{exer:sec43-setdiff3a}).
\end{enumerate}

%\item Prove or disprove (with a counterexample) the following proposition:
%\begin{list}{}
%\item If $A$ and  $B$  are subsets of some universal set  $U$, then 
%\[
%\left( A \cup B \right) - \left( A \cap B \right) = \left( A - B \right) \cup 
%\left(B - A \right).
%\]
%\end{list}


\item Let  $A$  and  $B$  be subsets of some universal set  $U$\!. 
\label{exer:sec43-7}%
\begin{enumerate}
  \yitem Prove that  $A$  and  $B - A$ are disjoint sets.
  \item Prove that  $A \cup B = A \cup \left( {B - A} \right)$.
\end{enumerate}

\item Let  $A$  and  $B$  be subsets of some universal set  $U$. \label{exer:sec43-8}
\begin{enumerate}
  \item Prove that  $A - B$ and  $A \cap B$ are disjoint sets.
  \item Prove that  $A = \left( {A - B} \right) \cup \left( {A \cap B} \right)$.
\end{enumerate}

\item Let $A$ and $B$ be subsets of some universal set $U$\!.  Prove or disprove each of the following:

%\BeginTable
%\def\L{\JustLeft}
%\BeginFormat
%| p(10pt) | p(1.8in) | p(2.2in) |
%\EndFormat
%" " (\textbf{a}) \; $A - \left(A \cap B^c \right) = A \cap B$ " (\textbf{c}) \;$\left( A \cup B \right) - A = B - A$ " \\+22
%"  " (\textbf{b}) \; $\left( A^c \cup B \right)^c \cap A = A - B$ " (\textbf{d}) \; 
%$\left( A \cup B \right) - B = A - \left( A \cap B \right)$ " \\+22
%"  " \use2 \L (\textbf{e}) \; $\left( A \cup B \right) - \left( A \cap B \right) = \left( A - B \right) \cup \left( B - A \right)$ " \\+20
%\EndTable

%\begin{multicols}{2}
\begin{enumerate}
\item $A - \left(A \cap B^c \right) = A \cap B$
\item $\left( A^c \cup B \right)^c \cap A = A - B$
\item $\left( A \cup B \right) - A = B - A$
\item $\left( A \cup B \right) - B = A - \left( A \cap B \right)$
%\end{enumerate}
%\end{multicols}
%\begin{enumerate} \setcounter{enumii}{4}
%\item $\left( A \cup B \right) - \left( A \cap B \right) = \left( A - B \right) \cup 
%       \left( B - A \right)$
\end{enumerate}


\item \textbf{Evaluation of proofs}  \hfill \\
See the instructions for Exercise~(\ref{exer:proofeval}) on 
page~\pageref{exer:proofeval} from Section~\ref{S:directproof}.

\begin{enumerate}
\item If $A$, $B$, and $C$ are subsets of some universal set $U$, then 
$A - \left( B - C \right) = A - \left(B \cup C \right)$.

\begin{myproof}
\[
\begin{aligned}
A - \left( B - C \right) &= \left(A - B \right) - \left( A - C \right) \\
                         &= \left( A \cap B^c \right) \cap \left(A \cap C^c \right) \\
                         &= A \cap \left( B^c \cap C^c\right) \\
                         &= A \cap \left( B \cup C \right)^c \\
                         &= A - \left( B \cup C \right) \hfill 
\end{aligned}
\] \qedhere
\end{myproof}

\item If $A$, $B$, and $C$ are subsets of some universal set $U$, then 
$A - \left(B \cup C \right) = \left(A - B \right) \cap \left(A - C \right)$.

\begin{myproof}
We first write $A - \left(B \cup C \right) = A \cap \left(B \cup C \right)^c$ and then use one of De Morgan's Laws to obtain
\[ 
A - \left(B \cup C \right)= A \cap \left(B^c \cap C^c \right).
\]
We now use the fact that $A = A \cap A$ and obtain

$A - \left(B \cup C \right)= A \cap A \cap B^c \cap C^c 
= \left(A \cap B^c \right) \cap \left(A \cap C^c \right) 
= \left(A - B \right) \cap \left(A - C \right)$. 
\end{myproof}
\end{enumerate}
\end{enumerate}


\subsection*{Explorations and Activities}
\setcounter{oldenumi}{\theenumi}
\begin{enumerate} \setcounter{enumi}{\theoldenumi}
  \item (\textbf{Comparison to Properties of the Real Numbers}).  
%\index{real numbers!properties}%
%\label{A:comparisontoreals}%
The following are some of the basic properties of addition and multiplication of real numbers.

\begin{flushleft}
%\textbf{Properties of the Real Numbers}
\begin{tabular}{p{1.6in} p{2.65in}}
\textbf{Commutative Laws}:
\index{commutative laws!for real numbers}%
  &  $a+b=b+a$, for all $a,b \in \mathbb{R}$. \\
                   &  $a \cdot b=b \cdot a$, for all $a,b \in \mathbb{R}$. \\
                   &                \\
\end{tabular}
\begin{tabular}{p{1.6in} p{2.75in}}
\textbf{Associative Laws}:
\index{associative laws!for real numbers}%
     &  $\left( {a + b} \right) + c = a + \left( {b + c} \right)$, for all  $a,b,c \in \mathbb{R}$.  \\
                   &  $\left( {a \cdot b} \right) \cdot c = a \cdot \left( {b \cdot c} \right)$
, for all  $a,b,c \in \mathbb{R}$.  \\
                   &  \\
\end{tabular}
\begin{tabular}{p{1.6in} p{2.65in}}
\textbf{Distributive Law}:
\index{distributive laws!for real numbers}%
  &  $a \cdot \left( {b + c} \right) = a \cdot b + a \cdot c$, for all  $a,b,c \in \mathbb{R}$.  \\
                   & \\
\end{tabular}
\begin{tabular}{p{1.6in} p{2.65in}}
\textbf{Additive Identity}:
\index{additive identity}%
 &  For all  $a \in \mathbb{R}$, $a + 0 = a = 0 + a$.  \\
                  &  \\
\textbf{Multiplicative Identity}:
\index{multiplicative identity}%
  &  For all  $a \in \mathbb{R}$, $a \cdot 1 = a = 1 \cdot a$.  \\
                 &  \\
\end{tabular}
\begin{tabular}{p{1.6in} p{2.65in}}
\textbf{Additive Inverses}:
\index{additive inverse}%
  &  For all  $a \in \mathbb{R}$, $a + ( - a) = 0 = ( - a) + a$.  \\
                 &  \\
\end{tabular}
\begin{tabular}{p{1.55in} p{2.65in}}
\textbf{Multiplicative Inverses}:
\index{multiplicative inverse}%
  &  For all  $a \in \mathbb{R}$ with  $a \ne 0$, $a \cdot a^{ - 1}  = 1 = a^{ - 1}  \cdot a$.\\
%                          &  $a \cdot a^{ - 1}  = 1 = a^{ - 1}  \cdot a$. \\
\end{tabular}
\end{flushleft}

Discuss the similarities and differences among the properties of addition and multiplication of real numbers and the properties of union and intersection of sets.
\end{enumerate}

%\markboth{Chapter~\ref{C:settheory}. Set Theory}{\ref{S:cartesian}. Cartesian Products}
\hbreak
\endinput
