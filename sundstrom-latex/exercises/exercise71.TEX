\section*{Exercises 7.1}
%
\begin{enumerate}
\xitem Let  $A = \left\{ {a, b, c} \right\}$, $B = \left\{ {p, q, r} \right\}$, and let $R$ be the set of ordered pairs defined by 
$R = \left\{ {\left( {a, p} \right), \left( {b, q} \right), \left( {c, p} \right), \left( {a, q} \right)} \right\}$. 
\label{exer:sec71-1}%
\begin{enumerate}
  \item Use the roster method to list all the elements of  $A \times B$.  Explain why  
        $A \times B$  can be  considered to be a relation from  $A$  to  $B$.

  \item Explain why  $R$  is a relation from  $A$  to  $B$.  

  \item What is the domain of  $R$?  What is the range of  $R$?

%  \item What is  $R^{ - 1} $, the inverse relation of  $R$?
\end{enumerate}


\xitem Let  $A = \left\{ {a, b, c} \right\}$ and let  
$R = \left\{ {\left( {a, a} \right), \left( {a, c} \right), \left( {b, b} \right), \left( {b, c} \right), \left( {c, a} \right), \left( {c, b} \right)} \right\}$ (so $R$ is a relation on $A$). \label{exer:sec71-2}
Are the following statements true or false?  Explain. 
\begin{enumerate}
  \item For each  $x \in A$, $x \mathrel{R} x$.

  \item For every  $x, y \in A$,  if  $x \mathrel{R} y$, then  $y \mathrel{R} x$.

  \item For every  $x, y, z \in A$, if  $x \mathrel{R} y$  and  $y \mathrel{R} z$, then  
        $x \mathrel{R} z$.

  \item $R$ is a function from  $A$  to  $A$.
\end{enumerate}

%\item Repeat Exercise~(\ref{exer:sec71-2}) using $A = \left\{ a, b, c, d \right\}$ and 
%$R = \left\{(a, a), (a, b), (a, d), (b, c), (b, d) \right\}$.

\item Let  $A$  be the set of all female citizens of the United States.  Let  $D$  be the relation on  $A$  defined by
\[
D = \left\{ { {\left( {x, y} \right) \in A \times A } \mid x\text{  is a daughter of  }y} \right\}\!.
\]
That is,  $x \mathrel{D} y$ means that  $x$  is a daughter of  $y$. \label{exer:sec71-3}

\begin{enumerate}
  \yitem Describe those elements of  $A$  that are in the domain of  $D$.

  \yitem Describe those elements of  $A$  that are in the range of  $D$.

  \item Is the relation  $D$  a function from  $A$  to  $A$?  Explain.

%  \item Use set builder notation to define the inverse relation of  $D$.
\end{enumerate}

\xitem \label{exer:sec71-4}
Let  $U$  be a nonempty set, and let  $R$  be the ``subset relation'' on  
$\mathcal{P}( U )$.  That is, 
\[
R = \left\{ { {\left( {S, T} \right) \in \mathcal{P}( U ) \times \mathcal{P}( U ) } \mid S \subseteq T} \right\}\!.
\]

\begin{enumerate}
  \item Write the open sentence  $\left( {S, T} \right) \in R$ using standard subset notation.

  \item What is the domain of this subset relation, $R$?

  \item What is the range of this subset relation, $R$?

 % \item Use set builder notation to define the inverse relation, $R^{ - 1} $.

  \item Is  $R$  a function from  $\mathcal{P}( U )$  to  
        $\mathcal{P}( U )$?  Explain.
\end{enumerate} 

\item Let  $U$  be a nonempty set, and let  $R$  be the ``element of''  relation from  $U$  to   $\mathcal{P}\left( U \right)$.  That is, \label{exer:sec71-5}
\[
R = \left\{ { {\left( {x, S} \right) \in U \times \mathcal{P}( U ) } \mid x \in S} \right\}\!.
\]

\begin{enumerate} 

  \item What is the domain of this ``element of'' relation, $R$?

  \item What is the range of this ``element of'' relation, $R$?

%  \item Use set builder notation to define the inverse relation, $R^{ - 1} $.

  \item Is  $R$  a function from  $U$  to  $\mathcal{P}( U )$?  Explain.
\end{enumerate}

\xitem Let  
$S = \left\{ {\left( {x, y} \right) \in \mathbb{R} \times \mathbb{R} \mid x^2  + y^2  = 100} \right\}$. \label{exer:circle-main} %Note that  $S$  is a relation on $\mathbb{R}$. 

\begin{enumerate}
  \item Determine the set of all values of  $x$  such that  $\left( {x, 6} \right) \in S$, and determine the set of all values of $x$  such that  $\left( {x, 9} \right) \in S$\!. 
\label{exer:circle-a}

  \item Determine the domain and range of the relation  $S$ and write each set using set builder notation.  

%  \item What is the inverse of the relation  $S$?

  \item Is the relation  $S$  a function from  $\mathbb{R}$ to  $\mathbb{R}$?  Explain.                  \label{exer:circle-e}

  \item Since  $S$ is a relation on  $\mathbb{R}$, its elements can be graphed in the coordinate         plane.  Describe the graph of the relation  $S$. Is the graph consistent with your answers in Exercises~(\ref{exer:circle-a}) through~(\ref{exer:circle-e})?  Explain.
\end{enumerate}

\item Repeat Exercise~(\ref{exer:circle-main}) using the relation on $\R$ defined by
\[
S = \left\{ {\left( {x, y} \right) \in \mathbb{R} \times \mathbb{R}   \mid y = \sqrt {100 - x^2 } } \right\}\!.\]
What is the connection between this relation and the relation in Exercise~(\ref{exer:circle-main})? \label{exer:semicircle}

\item Determine the domain and range of each of the following relations on $\R$ and sketch the graph of each relation.
\begin{enumerate}
\item $R = \left\{ {\left( {x, y} \right) \in \mathbb{R} \times \mathbb{R} \mid x^2  + y^2  = 10} \right\}$
\item $S = \left\{ {\left( {x, y} \right) \in \mathbb{R} \times \mathbb{R} \mid y^2  = x+10} \right\}$
\item $T = \left\{ {\left( {x, y} \right) \in \mathbb{R} \times \mathbb{R} \mid |x| + |y| = 10} \right\}$
\item $R = \left\{ {\left( {x, y} \right) \in \mathbb{R} \times \mathbb{R} \mid x^2  =y^2} \right\}$
\end{enumerate}


\item Let $\mathrel{R}$ be the relation on $\Z$ where for all $a, b \in \Z$, $a \mathrel{R} b$ if and only if $\left| a - b \right| \leq 2$. \label{exer:absvalueless2}

\begin{enumerate}
\yitem Use set builder notation to describe the relation $\mathrel{R}$ as a set of ordered pairs.

\yitem Determine the domain and range of the relation $\mathrel{R}$.

\item Use the roster method to specify the set of all integers $x$ such that $x \mathrel{R} 5$
and the set of all integers $x$ such that $5 \mathrel{R} x$.

\item If possible, find integers $x$ and $y$ such that $x \mathrel{R} 8$, $8 \mathrel{R} y$, but 
$x \mathrel{\not \negthickspace R} y$.

\item If $b \in \Z$, use the roster method to specify the set of all $x \in \Z$ such that 
$x \mathrel{R} b$.

\end{enumerate}

\item Let $R_{  < }  = \left\{ { {\left( {x, y} \right) \in \mathbb{R} \times \mathbb{R} } \mid x < y} \right\}$.  This means that $R_{  < } $ is the ``less than'' relation on  $\R$.
\label{exer:sec71-lessthan}%
\begin{enumerate}
  \item What is the domain of the relation  $R_{  < } $?  

  \item What is the range of the relation  $R_{  < } $?

%  \item What is the inverse of the relation  $R_{  < } $?

  \item Is the relation  $R_{  < } $ a function from  $\mathbb{R}$ to  $\mathbb{R}$?  Explain.
\end{enumerate}

%\item Prove the following proposition: \label{exer:sec71-inverses}
%\begin{list}{}
%\item Let  $A$  and  $B$  be nonempty sets, and let  $R$  and  $S$  be relations from  $A$  to  $B$. If  $R \subseteq S$, then  $R^{ - 1}  \subseteq S^{ - 1} $.
%\end{list}

\noindent
\note  Remember that a relation is a set.  Consequently, we can talk about one relation being a subset of another relation.  Another thing to remember is that the elements of a relation are ordered pairs.

\end{enumerate}



\subsection*{Explorations and Activities}
\setcounter{oldenumi}{\theenumi}
\begin{enumerate} \setcounter{enumi}{\theoldenumi} 
\item \textbf{The Inverse of a Relation}.  
In Section~\ref{S:inversefunctions}, we introduced the \textbf{inverse of a function}.
\index{inverse of a function}%
\index{function!inverse of}%
 If  $A$ and $B$ are nonempty sets and if $f:A \to B$ is a function, then the inverse of  $f$, denoted by  $f^{ - 1} $, is defined as
\begin{align*}
f^{ - 1}  &= \left\{ { {\left( {b, a} \right) \in B \times A } \mid f\left( a \right) = b} \right\} \\
          &= \left\{ { {\left( {b, a} \right) \in B \times A } \mid \left( {a, b} \right) \in f} \right\}\!.
\end{align*}
%If we use the ordered pair representation for  $f$, we could also write
%\[
%f^{ - 1} = \left\{ { {\left( {b, a} \right) \in B \times A } \mid \left( {a, b} \right) \in f} \right\}.
%\]
Now that we know about relations, we see that  $f^{ - 1} $  is always a relation from  $B$  to 
$A$.  The concept of the inverse of a function is actually a special case of the more general concept of the inverse of a relation, which we now define.
\begin{defbox}{inverseofrelation}{Let  $R$  be a relation from the set  $A$  to the set  $B$.  The \textbf{inverse of}  $\boldsymbol{R}$,
\index{inverse of a relation}%
\index{relation!inverse of}%
 written  $R^{ - 1} $ 
\label{sym:Rinverse} and read  ``$R$  inverse,'' is the relation from  $B$  to  $A$  defined by
\[
\begin{aligned}
  R^{ - 1}  &= \left\{ { {\left( {y, x} \right) \in B \times A } \mid \left( {x, y} \right) \in R} \right\}\text{, or} \hfill \\
  R^{ - 1}  &= \left\{ { {\left( {y, x} \right) \in B \times A } \mid x \mathrel{R} y} \right\}\!. \\ 
\end{aligned} 
\]
That is, $R^{ - 1} $ is the subset of  $B \times A$ consisting of all ordered pairs  
$\left( {y, x} \right)$  such that  $x \mathrel{R} y$.}
\end{defbox}
For example, let $D$ be the ``divides'' relation on  $\mathbb{Z}$.  See Progress 
Check~\ref{prog:dividesrelation}.  So
\[
D = \left\{ { {\left( {m, n} \right) \in \mathbb{Z} \times \mathbb{Z} } \mid m\text{  divides  }n} \right\}\!.
\]
This means that we can write
%\begin{center}
$m \mid n$ if and only if $\left( {m, n} \right) \in D$.
%\end{center}
So, in this case,
\[
\begin{aligned}
D^{ - 1}  &= \left\{ { {\left( {n, m} \right) \in \mathbb{Z} \times \mathbb{Z} } \mid \left( {m, n} \right) \in D} \right\} \\ 
          &= \left\{ { {\left( {n, m} \right) \in \mathbb{Z} \times \mathbb{Z} } \mid m\text{  divides  }n} \right\}\!. \\ 
\end{aligned}
\]
Now, if we would like to focus on the first coordinate instead of the second coordinate in  
$D^{ - 1} $, we know that  ``$m$  divides  $n$''  means the same thing as  ``$n$  is a multiple of  $m$.''  Hence,
\[
D^{ - 1}  = \left\{ { {\left( {n, m} \right) \in \mathbb{Z} \times \mathbb{Z} } \mid n\text{  is a multiple of  }m} \right\}\!.
\]
We can say that the inverse of the ``divides'' relation on  $\mathbb{Z}$  is the ``is a multiple of'' relation on  $\mathbb{Z}$.

Theorem~\ref{T:inverserelations}, which follows, contains some elementary facts about inverse relations.  %The proofs of these results are included in Exercise~(\ref{exer:provinginverse}).
%
\begin{theorem} \label{T:inverserelations}
Let  $R$  be a relation from the set  $A$  to the set  $B$.  Then

\begin{itemize}
\item The domain of  $R^{ - 1} $ is the range of  $R$.  That is, 
$\text{dom}\!\left( {R^{ - 1} } \right) = \text{range}( R )$.  \label{T:inverserelations1}

\item The range of  $R^{ - 1} $  is the domain of  $R$.   That is, 
$\text{range}\!\left( {R^{ - 1} } \right) = \text{dom}( R )$.  \label{T:inverserelations2}

\item The inverse of  $R^{ - 1} $  is  $R$.  That is, $\left( {R^{ - 1} } \right)^{ - 1}  = R$.  \label{T:inverserelations3}
\end{itemize}
\end{theorem}
To prove the first part of \label{exer:provinginverse}
Theorem~\ref{T:inverserelations}, observe that the goal is to prove that two sets are equal, 
\[
\text{dom}\!\left( {R^{ - 1} } \right) = \text{range}( R ).
\]
One way to do this is to prove that each is a subset of the other.  %Another way is to use a sequence of  ``if and only if'' statements.
To prove that  $\text{dom}\!\left( {R^{ - 1} } \right) \subseteq \text{range}( R )$, we can start by choosing an arbitrary element of  $\text{dom}\!\left( {R^{ - 1} } \right)$.  So let  
$y \in \text{dom}\!\left( {R^{ - 1} } \right)$.  The goal now is to prove that 
$y~\in~\text{range}( R )$.  What does it mean to say that  
$y \in \text{dom}\!\left( {R^{ - 1} } \right)$?  It means that there exists an  $x \in A$  such that
\[
\left( {y, x} \right) \in R^{ - 1}. 
\]
Now what does it mean to say that  $( {y, x} ) \in R^{ - 1} $?  It means that  
$( {x, y} ) \in R$.  What does this tell us about  $y$?

Complete the proof of the first part of Theorem~\ref{T:inverserelations}.  Then, complete the proofs of the other two parts of Theorem~\ref{T:inverserelations}.

\end{enumerate}


\hbreak

%\markboth{Chapter~\ref{C:equivrelations}. Equivalence Relations}{\ref{S:equivrelations}. Equivalence Relations}

\endinput

