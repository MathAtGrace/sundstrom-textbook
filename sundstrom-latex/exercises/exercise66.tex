\section*{Exercises \ref{S:functionsonsets}}

\begin{enumerate}
\item Let $f\x S \to T$, let $A$ and $B$ be subsets of $S$, and let $C$ and $D$ be subsets of $T$.  For $x \in S$ and $y \in T$, carefully explain what it means to say that \label{exer:sec91-1}

\begin{multicols}{2}
\begin{enumerate}
\yitem $y \in f ( A \cap B )$
\item $y \in f ( A \cup B )$
\item $y \in f ( A ) \cap f ( B )$
\yitem $y \in f ( A ) \cup f ( B )$
\item $x \in f^{-1} ( C \cap D )$
\yitem $x \in f^{-1} ( C \cup D )$
\item $x \in f^{-1} ( C ) \cap f^{-1} ( D )$
\yitem $x \in f^{-1} ( C ) \cup f^{-1} ( D )$
\end{enumerate}
\end{multicols}

\item Let $f\x \mathbb{R} \to \mathbb{R}$ by $f ( x ) = -2x + 1$.  Let 
\begin{multicols}{4}
$A = \left[ 2, 5 \right]$

$B = \left[ -1, 3 \right]$

$C = \left[ -2, 3 \right]$

$D = \left[ 1, 4 \right]$.
\end{multicols}

Find each of the following:  \label{exer:sec91-2}

\begin{multicols}{2}
\begin{enumerate}
\item $f ( A )$

\yitem $f^{-1} ( f ( A ) )$

\item $f^{-1} ( C )$

\yitem $f ( f^{-1} ( C ) )$

\yitem $f ( A \cap B )$

\yitem $f ( A ) \cap f ( B )$

\item $f^{-1} ( C \cap D )$

\item $f^{-1} ( C ) \cap f^{-1} ( D )$

\end{enumerate}
\end{multicols}

\item Let $g\x \mathbb{N} \times \mathbb{N} \to \mathbb{N}$ by $g ( m, n ) = 2^m 3^n$, let $A = \left\{ 1, 2, 3 \right\}$, and let $C = \left\{ 1, 4, 6, 9, 12, 16, 18 \right\}$.  Find \label{exer:sec91-3}
\begin{multicols}{2}
\begin{enumerate}
\yitem $g ( A \times A )$
\yitem $g^{-1} ( C )$
\item $g^{-1} \!\left( g ( A \times A ) \right)$
\item $g \!\left( g^{-1} ( C ) \right)$
\end{enumerate}
\end{multicols}

\item \begin{enumerate}
\yitem Let $S = \left\{1, 2, 3, 4 \right\}$.  Define $F\x S \to \mathbb{N}$ by 
$F ( x ) = x^2$ for each $x \in S$.  What is the range of the function $F$ and what is $F ( S )$?  How do these two sets compare?
\end{enumerate} \label{exer:sec91-4}
Now let $A$ and $B$ be sets and let $f\x A \to B$ be an arbitrary function from $A$ to $B$.
\begin{enumerate} \addtocounter{enumii}{1}
\item Explain why $f ( A ) = \text{range} ( f )$.

\item Define a function $g\x A \to f ( A )$ by 
$g ( x  ) = f ( x )$ for all $x$ in $A$.  Prove that the function $g$ is a surjection.
\end{enumerate}

\xitem Prove Part~(\ref{T:imageofoperations2}) of Theorem~\ref{T:imageofoperations}.
\label{exer:sec91-5}

Let $f\x S \to T$ be a function and let $A$ and $B$ be subsets of $S$.  Then 
$f ( A \cup B ) = f ( A ) \cup f ( B )$.

\xitem Prove Part~(\ref{T:invimageofoperations1}) of Theorem~\ref{T:invimageofoperations}.
\label{exer:sec91-6}

Let $f\x S \to T$ be a function and let $C$ and $D$ be subsets of $T$.  Then
$f^{-1} ( C \cap D ) = f^{-1} ( C ) \cap f^{-1} ( D )$.

\item Prove Part~(\ref{T:imageofinvimage2}) of Theorem~\ref{T:imageofinvimage}.
\label{exer:sec91-7}

Let $f\x S \to T$ be a function and let $C \subseteq T$.  Then 
$f \!\left( f^{-1} ( C ) \right) \subseteq C$.


\item Let $f\x S \to T$ and let $A$ and $B$ be subsets of $S$.  Prove or disprove each of the following:
\begin{enumerate}
\item If $A \subseteq B$, then $f( A ) \subseteq f( B )$.

\item If $f ( A ) \subseteq f (B )$, then $A \subseteq B$.
\end{enumerate}


\xitem \label{exer:invimagetruefalse} Let $f\x S \to T$ and let $C$ and $D$ be subsets of $T$.  Prove or disprove each of the following:
\begin{enumerate}
\item If $C \subseteq D$, then $f^{-1} ( C ) \subseteq f^{-1} (D )$.

\item If $f^{-1} ( C ) \subseteq f^{-1} (D )$, then $C \subseteq D$.
\end{enumerate}

\item Prove or disprove:

If $f\x S \to T$ is a function and  $A$ and $B$ are subsets of $S$,  then \\
$f ( A ) \cap f ( B ) \subseteq f ( A \cap B )$.

\underline{Note}:  Part~(\ref{T:imageofoperations1}) of Theorem~\ref{T:imageofoperations} states that 
$f ( A \cap B ) \subseteq f ( A ) \cap f ( B )$.

\item Let $f\x S \to T$ be a function, let $A \subseteq S$, and let $C \subseteq T$.
\label{exer:sec91-11}
\begin{enumerate}
\item Part~(\ref{T:imageofinvimage1}) of Theorem~\ref{T:imageofinvimage} states that 
$A \subseteq f^{-1} \!\left( f ( A ) \right)$.  Give an example where 
$f^{-1} \!\left( f ( A ) \right) \not \subseteq A$.

\item Part~(\ref{T:imageofinvimage2}) of Theorem~\ref{T:imageofinvimage} states that 
$f \!\left( f^{-1} ( C ) \right) \subseteq C$.  Give an example where 
$C \not \subseteq f \!\left( f^{-1} ( C ) \right)$.
\end{enumerate}

\item Is the following proposition true or false?  Justify your conclusion with a proof or a counterexample. \label{exer:sec91-12}

If $f\x  S \to T$ is an injection and $A \subseteq S$, then 
$f^{-1} \!\left( f ( A ) \right) = A$.

\item Is the following proposition true or false?  Justify your conclusion with a proof or a counterexample. \label{exer:sec91-13}

If $f\x  S \to T$ is a surjection and $C \subseteq T$, then 
$f \!\left( f^{-1} ( C ) \right) = C$.

\item Let $f\x S \to T$.  Prove that 
$f ( A \cap B ) = f ( A ) \cap f ( B )$ for all subsets $A$ and $B$ of $S$ if and only if $f$ is an injection.

%\item Let $f\x S \to T$ and let $A$ and $B$ be subsets of $S$.  Investigate the relationship between the sets $f ( A - B )$ and $f ( A ) - f ( B )$.  Are the two sets equal?  If not, is one a subset of the other?  Are there any conditions on the function $f$ that will ensure that the two sets are equal?  Justify your conclusions.
\hbreak
\end{enumerate}


\endinput
