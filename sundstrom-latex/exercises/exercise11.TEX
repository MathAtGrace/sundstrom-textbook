\section*{Exercises for Section~\ref{S:prop}}
\begin{enumerate}
  \xitem Which of the following sentences are statements? \label{exer:sec11-1}%
  \begin{enumerate}
  \item $3^2  + 4^2  = 5^2$.

  \item $a^2  + b^2  = c^2$. 
  \item There exist integers $a$, $b$, and $c$ such that $a^2 = b^2 + c^2$.

  \item If  $x^2  = 4$, then  $x = 2$.
  \item For each real number $x$, if $x^2 = 4$, then $x = 2$.

  \item For each real number  $t$,  $\sin ^2 t + \cos ^2 t = 1$.

  \item $\sin x < \sin \! \left( {\dfrac{\pi }{4}} \right)$.

  \item If  $n$  is a prime number, then  $n^2$ has three positive factors.

  \item $1 + \tan ^2 \theta  = \sec ^2 \theta $.

  \item Every rectangle is a parallelogram.
  \item Every even natural number greater than or equal to 4 is the sum of two prime numbers.
  \end{enumerate}

\xitem \label{exer:sec11-2}%
Identify the hypothesis and the conclusion for each of the following conditional statements. 
  \begin{enumerate}
  \item If  $n$  is a prime number, then  $n^2$ has three positive factors.
  \item If  $a$  is an irrational number and  $b$  is an irrational number, then  $a \cdot b$ is an irrational number.
  \item If  $p$  is a prime number, then  $p = 2$ or   $p$ is an odd number.
  \item If $p$ is a prime number and $p \ne 2$, then $p$ is an odd number.
  \item If $p \ne 2$ and $p$ is an even number, then $p$ is not prime.
%\item If $x$ is a positive real number, then $\sqrt{x}$ is a positive real number.
%\item If $\sqrt{x}$ is not a real number, then $x$ is a negative real number.
%\item If the lengths of the diagonals of a parallelogram are equal, then the parallelogram is a rectangle.
  \end{enumerate}

\xitem \label{exer:sec11-3}%
Determine whether each of the following conditional statements is true or false. 
\begin{multicols}{2}
  \begin{enumerate}
  \item If $ 10 < 7$, then $3 = 4$.

  \item If  $7 < 10$, then  $3 = 4$.

  \item If $10 < 7$, then $3 + 5 = 8$.

  \item If $7 < 10$, then $3 + 5 =8$.

\end{enumerate}
\end{multicols}

\xitem \label{exer:sec11-4}%
Determine the conditions under which each of the following conditional sentences will be a true statement. 
\begin{multicols}{2}
  \begin{enumerate}
  \item If $a+2=5$, then $8<5$.

  \item If $5<8$, then $a+2=5$.
\end{enumerate}
\end{multicols}

\item Let  $P$  be the statement ``Student X passed every assignment in Calculus I,'' and let  $Q$  be the statement ``Student X received a grade of  C  or better in Calculus I.''
\label{exer:sec11-5}%
  \begin{enumerate}
  \item What does it mean for  $P$  to be true?  What does it mean for  $Q$  to be true?
  
  \item \label{exer:conditionalb}%
Suppose that  Student X passed every assignment in Calculus I and received a grade of 
B$-$, and  that  the instructor made the statement  $P \to Q$.  Would you say that the instructor lied or told the truth? 

  \item \label{exer:conditionalc}%
Suppose that Student X passed every assignment in Calculus I and received a grade of  C$-$,  and that the instructor made the statement  $P \to Q$.  Would you say that the instructor lied or told the truth? 

  \item \label{exer:conditionald}%
Now suppose that Student X did not pass two assignments in Calculus I and received a grade of  D, and that the instructor made the statement  $P \to Q$.  Would you say that the instructor lied or told the truth? 

  \item How are Parts~(\ref{exer:conditionalb}), (\ref{exer:conditionalc}), and~(\ref{exer:conditionald}) related to the truth table for  \mbox{$P \to Q$}?
  \end{enumerate}

\item \label{exer:sec11-6}%
Following is a statement of a theorem which can be proven using calculus or precalculus mathematics.  For this theorem, $a$, $b$, and $c$ are real numbers. 

\begin{list}{}
\item 
\parbox{4in}{\textbf{Theorem}.  If  $f$  is a quadratic function of the form  \\
$f\left( x \right) = ax^2  + bx + c$   and  $a < 0$, then the function  $f$  has a maximum value  when  
$x = \dfrac{{ - b}}{{2a}}$.}
\end{list}

Using \textbf{only} this theorem, what can be concluded about the functions given by the following formulas?
\begin{multicols}{2}
\begin{enumerate}
\yitem $g\left( x \right) =  - 8x^2  + 5x - 2$


\yitem $h\left( x \right) =  - \dfrac{1}{3}x^2  + 3x$


\yitem $k\left( x \right) = 8x^2  - 5x - 7$


\item $j\left( x \right) =  - \dfrac{{71}}{{99}}x^2  + 210$


\item $f\left( x \right) =  - 4x^2  - 3x + 7$


\item $F\left( x \right) =  - x^4  + x^3  + 9$
\end{enumerate}
\end{multicols}

\item \label{exer:sec11-7}%
Following is a statement of a theorem which can be proven using the quadratic formula.  For this theorem, $a$, $b$, and $c$ are real numbers. 

\begin{list}{}
\item 
\parbox{4in}{\textbf{Theorem}  If  $f$  is a quadratic function of the form  \\
$f\left( x \right) = ax^2  + bx + c$   and  $ac < 0$, then the function  $f$  has two $x$-intercepts.}
\end{list}

Using \textbf{only} this theorem, what can be concluded about the functions given by the following formulas?
\begin{multicols}{2}
\begin{enumerate}
\item $g\left( x \right) =  - 8x^2  + 5x - 2$


\item $h\left( x \right) =  - \dfrac{1}{3}x^2  + 3x$


\item $k\left( x \right) = 8x^2  - 5x - 7$


\item $j\left( x \right) =  - \dfrac{{71}}{{99}}x^2  + 210$


\item $f\left( x \right) =  - 4x^2  - 3x + 7$


\item $F\left( x \right) =  - x^4  + x^3  + 9$
\end{enumerate}
\end{multicols}


\item Following is a statement of a theorem about certain cubic equations.  For this theorem, 
$b$ represents a real number.\label{exer:sec1-1-8}

\textbf{Theorem A}.  If $f$ is a cubic function of the form $f(x) = x^3 - x + b$ and $b > 1$, then the function $f$ has exactly one $x$-intercept.

Following is another theorem about $x$-intercepts of functions:

\textbf{Theorem B}.  If $f$ and $g$ are functions with $g(x) = k \cdot f(x)$, where $k$ is a nonzero real number, then $f$ and $g$ have exactly the same $x$-intercepts.

Using only these two theorems and some simple algebraic manipulations, what can be concluded about the functions given by the following formulas?

\begin{multicols}{2}
\begin{enumerate}
\item $f(x) = x^3 - x + 7$
\item $g(x) = x^3 + x + 7$
\item $h(x) = -x^3 + x - 5$
\item $k(x) = 2x^3 + 2x + 3$
\item $r(x) = x^4 - x + 11$
\item $F(x) = 2x^3 - 2x + 7$
\end{enumerate}
\end{multicols}


\xitem \label{exer:sec11-8}% 
\begin{enumerate}
\item Is the set of natural numbers closed under division?

\item Is the set of rational numbers closed under division?

\item Is the set of nonzero rational numbers closed under division?

\item Is the set of positive rational numbers closed under division?
\item Is the set of positive real numbers closed under subtraction?

\item Is the set of negative rational numbers closed under division?
\item Is the set of negative integers closed under addition?

%\item Is the set of irrational numbers closed under multiplication?
\end{enumerate} \label{exer:closure}
\end{enumerate}


\subsection*{Explorations and Activities}
%\newcounter{oldenumi}
\setcounter{oldenumi}{\theenumi}
\begin{enumerate} \setcounter{enumi}{\theoldenumi}
  \item \textbf{Exploring Propositions}.  \label{exer11:explore}  In Progress Check~\ref{pr:explores}, we used exploration to show that certain statements were false and to make conjectures that certain statements were true.  We can also use exploration to formulate a conjecture that we believe to be true.  For example, if we calculate successive powers of 2 $\left( 2^1, 2^2, 2^3, 2^4, 2^5, \ldots \: \right)$ and examine the units digits of these numbers, we could make the following conjectures (among others):

\begin{itemize}
\item If $n$ is a natural number, then the units digit of $2^n$ must be 2, 4, 6, or 8.

\item The units digits of the successive powers of 2 repeat according to the pattern 
``2, 4, 8, 6.''
\end{itemize}

\begin{enumerate}
\item Is it possible to formulate a conjecture about the units digits of successive powers of 4 
$\left( 4^1, 4^2, 4^3, 4^4, 4^5, \ldots \: \right)$?  If so, formulate at least one conjecture.

\item Is it possible to formulate a conjecture about the units digit of numbers of the form 
$7^n - 2^n$, where $n$ is a natural number?  If so, formulate a conjecture in the form of a conditional statement in the form ``If $n$ is a natural number, then $\ldots$ .''

\item Let  $f \left( x \right) = e^{2x}$.  Determine the first eight derivatives of this function.  What do you observe?  Formulate a conjecture that appears to be true.  The conjecture should be written as a conditional statement in the form, ``If $n$ is a natural number, then 
$\ldots$ .''
\end{enumerate}
\end{enumerate}



\hbreak
%\markboth{Chapter \ref{C:intro}. Introduction to Writing}{\ref{S:direct}. Constructing Direct Proofs}
\endinput
