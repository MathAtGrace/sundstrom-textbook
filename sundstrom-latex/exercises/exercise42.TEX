\section*{Exercises for Section~\ref{S:otherinduction}}
%
\begin{enumerate}

\item Use mathematical  induction to prove each of the following: \label{exer:sec52-1}

\begin{enumerate}
  \yitem For each natural  number  $n$  with  $n \geq 2$,  $3^n  > 1 + 2^n $.

  \item For each natural  number  $n$  with  $n \geq 6$,  $2^n  > \left( n + 1 \right)^2$.

  \item For each natural  number  $n$  with  $n \geq 3$,  $\left( {1 + \dfrac{1}{n}} \right)^n  < n$.
\end{enumerate}


\xitem For  which natural numbers  $n$  is  $n^2  < 2^n$?  Justify your conclusion.
\label{exer:sec52-6}%

\item For which natural numbers $n$ is $n! > 3^n$?  Justify your conclusion.



\item \label{exer:sec52-2} \begin{enumerate} \item Verify that  $\left( {1 - \dfrac{1}{4}} \right) = \dfrac{3}
{4}$ and that  $\left( {1 - \dfrac{1}{4}} \right)\left( {1 - \dfrac{1}{9}} \right) = \dfrac{4}
{6}$. \label{exer:522a}

\item Verify that  $\left( {1 - \dfrac{1}{4}} \right)\left( {1 - \dfrac{1}{9}} \right)\left( {1 - \dfrac{1}{{16}}} \right) = \dfrac{5}{8}$ and that  
\\$\left( {1 - \dfrac{1}{4}} \right)\left( {1 - \dfrac{1}{9}} \right)\left( {1 - \dfrac{1}{{16}}} \right)\left( {1 - \dfrac{1}{{25}}} \right) = \dfrac{6}{{10}}$. \label{exer:522b}

\item For  $n \in \mathbb{N}$ with  $n \geq 2$, make a conjecture about a formula for the product  $\left( {1 - \dfrac{1}{4}} \right)\left( {1 - \dfrac{1}{9}} \right)\left( {1 - \dfrac{1}{{16}}} \right) \cdots \left( {1 - \dfrac{1}{{n^2 }}} \right)$.

\item Based on your work in Parts~(\ref{exer:522a}) and~(\ref{exer:522b}), state a proposition and then use the Extended Principle of Mathematical Induction to prove your proposition.

\end{enumerate}

\xitem Is the following proposition true or false?  Justify your conclusion. \label{exer:sec52-3}

\begin{list}{}
\item For each nonnegative integer $n$, $8^n \mid \left( {4n} \right)!.$
\end{list}

\item Let $y = \ln x$. \label{exer:sec51-12}

\begin{enumerate}
\item Determine $\dfrac{{dy}}{{dx}}$, $\dfrac{{d^2 y}}{{dx^2 }}$, 
$\dfrac{{d^3 y}}{{dx^3 }}$, and $\dfrac{{d^4 y}}{{dx^4 }}$.

\item Let $n$ be a natural number.  Formulate a conjecture for a formula for 
$\dfrac{{d^n y}}{{dx^n }}$.  Then use mathematical induction to prove your conjecture.
\end{enumerate}

\item For which natural numbers  $n$  do there exist nonnegative integers  $x$  and  $y$  such that  $n = 4x + 5y$?  Justify your conclusion.
\label{exer:sec52-7}%

\xitem Can each natural number greater than or equal to 4 be written as the sum of at least two natural numbers, each of which is a 2 or a 3?  Justify your conclusion. 
\label{exer:sec52-4}%
For example,  $7 = 2 + 2 + 3$, and  $17 = 2 + 2 + 2 + 2 + 3 + 3 + 3$.

\item Can each natural number greater than or equal to 6 be written as the sum of at least two natural numbers, each of which is a 2 or a 5?  Justify your conclusion.
\label{exer:sec52-5}%
For example,  $6 = 2 + 2 + 2$, $9 = 2 + 2 + 5$, and  $17 = 2 + 5 + 5 + 5$.



\item Use mathematical induction to prove the following proposition:
\begin{list}{}
\item Let $x$ be a real number with $x > 0$.  Then for each natural number $n$ with $n \geq 2$, 
      $\left( {1 + x} \right)^n  > 1 + nx$.
\end{list}
Explain where the assumption that $x > 0$ was used in the proof.

\item Prove that for each odd natural number $n$ with $n \geq 3$,
\[
\left( {1 + \frac{1}{2}} \right)
\left( {1 - \frac{1}{3}} \right)
\left( {1 + \frac{1}{4}} \right) 
\cdots 
\left( {1 + \frac{{\left( { - 1} \right)^n }}{n}} \right) = 1.
\]

\xitem Prove that for each natural number $n$, \label{exer:2elementsubsets}
\begin{center}
any set with $n$ elements has 
$\dfrac{n \left( {n-1} \right)}{2}$ two-element subsets.
\end{center}



\item Prove or disprove each of the following propositions:
\begin{enumerate}
\item For each $n \in \N$, $\dfrac{1}{1 \cdot 2} + \dfrac{1}{2 \cdot 3} + \cdots + \dfrac{1}{n (n + 1 )} = \dfrac{n}{n+1}$.

\item For each natural number $n$ with $n \geq 3$, 
\[
\frac{1}{3 \cdot 4} + \frac{1}{4 \cdot 5} + \cdots + \frac{1}{n (n + 1 )} = \frac{n-2}{3n + 3}.
\]

\item For each $n \in \N$, $1 \cdot 2 + 2 \cdot 3 + 3 \cdot 4 + \cdots + n (n + 1 ) = \dfrac{n (n + 1 ) (n + 2 )}{3}$.
\end{enumerate}

%\item Use mathematical induction to prove that the sum of the cubes of any three consecutive natural numbers is a multiple of 9. \label{exer:sec51-13}


\item Is the following proposition true or false?  Justify your conclusion.
\begin{list}{}
\item For each natural number $n$, 
$\left( \dfrac{n^3}{3} + \dfrac{n^2}{2} + \dfrac{7n}{6} \right)$ is a natural number.
\end{list}


\item Is the following proposition true or false?  Justify your conclusion.
\begin{list}{}
\item For each natural number $n$, 
$\left( \dfrac{n^5}{5} + \dfrac{n^4}{2} +  \dfrac{n^3}{3} - \dfrac{n}{30} \right)$ is an integer.
\end{list}




\xitem \label{exer:specialFTA} \begin{enumerate}
\item Prove that if $n \in \N$, then there exists an odd natural number $m$ and a nonnegative integer $k$ such that $n = 2^k m$.

\item For each $n \in \N$, prove that there is only one way to write $n$ in the form described in Part~(a).  To do this, assume that $n = 2^k m$ and $n = 2^q p$ where $m$ and 
$p$ are odd natural numbers and $k$ and $q$ are nonnegative integers.  Then prove that 
$k = q$ and $m = p$. 
\end{enumerate}



\item \textbf{Evaluation of proofs}  \hfill \\
See the instructions for Exercise~(\ref{exer:proofeval}) on 
page~\pageref{exer:proofeval} from Section~\ref{S:directproof}.
\begin{enumerate}
\item For each natural number $n$ with $n \geq 2$, $2^n > 1 + n$.
\begin{myproof}
We let $k$ be a natural number and assume that $2^k > 1 + k$.  Multiplying both sides of this inequality by 2, we see that $2^{k+1} > 2 + 2k$.  However, $2 + 2k > 2 + k$ and, hence,
\[
2^{k+1} > 1 + \left(k + 1 \right)\!.
\]
By mathematical induction, we conclude that $2^n > 1 + n$.
\end{myproof}

\item Each natural number greater than or equal to 6 can be written as the sum of natural numbers, each of which is a 2 or a 5.

\begin{myproof}
We will use a proof by induction.  For each natural number $n$, we let $P( n )$ be, 
``There exist nonnegative integers $x$ and $y$ such that $n = 2x + 5y$.''
Since
\begin{align} \notag
6 &= 3 \cdot 2 + 0 \cdot 5  &  7 &= 2 + 5 \\ \notag
8 &= 4 \cdot 2 + 0 \cdot 5  &  9 &= 2 \cdot 2 + 1 \cdot 5 \notag 
\end{align}
we see that $P(6 ), P(7 ), P(8 )$, and $P( 9 )$ are true.

We now suppose that for some natural number $k$ with $k \geq 10$ that 
$P(6 ), P (7 ), \ldots, P (k )$ are true.  Now
\[
k + 1 = \left(k - 4 \right) + 5.
\]
Since $k \geq 10$, we see that $k - 4 \geq 6$ and, hence, $P(k - 4 )$ is true.  So
$k - 4 = 2x + 5y$ and, hence,
\[
\begin{aligned}
k + 1 &= \left(2x + 5y \right) + 5 \\
      &= 2x + 5 \left(y + 1 \right)\!.
\end{aligned}
\]
This proves that $P(k + 1 )$ is true, and hence, by the Second Principle of Mathematical Induction, we have proved that for each natural number $n$ with $n \geq 6$, there exist nonnegative integers $x$ and $y$ such that $n = 2x + 5y$.     
\end{myproof}
\end{enumerate}
\end{enumerate}


\subsection*{Explorations and Activities}
\setcounter{oldenumi}{\theenumi}
\begin{enumerate} \setcounter{enumi}{\theoldenumi}
\item \textbf{The Sum of the Angles of a Convex Quadrilateral}.  There is a famous theorem in Euclidean geometry that states that the sum of the interior angles of a triangle is $180^\circ$.
\label{exer:convexquads}%  
%(May need a definition of a convex quadrilateral.)

\begin{enumerate}
\item Use the theorem about triangles to determine the sum of the angles of a convex quadrilateral. \hint Draw a convex quadrilateral and draw a diagonal.

\item Use the result in Part~(a) to determine the sum of the angles of a convex pentagon.

\item Use the result in Part~(b) to determine the sum of the angles of a convex hexagon.

\item Let $n$ be a natural number with $n \geq 3$.  Make a conjecture about the sum of the angles of a convex polygon with $n$ sides and use mathematical induction to prove your conjecture.
\end{enumerate}


\item \textbf{De Moivre's Theorem}.  
\index{De Moivre's Theorem}%
One of the most interesting results in trigonometry is De Moivre's Theorem, which relates the complex number $i$ to the trigonometric functions.  Recall that the number $i$ is a complex number whose square is $-1$, that is, $i^2 = -1$.  One version of the theorem can be stated as follows:
\begin{list}{}
\item If  $x$  is a real number, then for each nonnegative integer $n$,
\[
\left[ {\cos x + i(\sin x)} \right]^n  = \cos (nx) + i( {\sin (nx)} )\!.
\]
\end{list}
This theorem is named after Abraham de Moivre
\index{de Moivre, Abraham}%
 (1667 -- 1754), a French mathematician.

\begin{enumerate}
  \item The proof of De Moivre's Theorem requires the use of the trigonometric identities for the sine and cosine of the sum of two angles.  Use the Internet or a book to find identities for $\sin( \alpha + \beta)$ and 
$\cos (\alpha + \beta)$.
  \item To get a sense of how things work, expand $\left[ {\cos x + i(\sin x)} \right]^2$ and write the result in the form $a + bi$.  Then use the identities from part~(1) to prove that 
$\left[ {\cos x + i(\sin x)} \right]^2  = \cos (2x) + i( {\sin (2x)} )$.
  \item Use mathematical induction to prove De Moivre's Theorem.
\end{enumerate}

\end{enumerate}


\hbreak



\endinput
