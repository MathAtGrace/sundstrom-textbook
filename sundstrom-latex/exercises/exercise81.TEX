\section*{Exercises 8.1}
%
\begin{enumerate}
\item Find each of the following greatest common divisors by listing all of the positive common  divisors of each pair of integers. \label{exer:sec81-1}

\begin{multicols}{3} 
\begin{enumerate}
\yitem $\gcd( {21, 28} )$
\yitem $\gcd( { - 21, 28} )$
\yitem $\gcd( {58, 63} )$
\yitem $\gcd( {0, 12} )$
\item $\gcd( {110, 215} )$
\item $\gcd( {110,  - 215} )$
\end{enumerate}
\end{multicols}

\item \label{exer:sec81-2} \begin{enumerate} \yitem Let  $a \in \Z$ and let  $k \in \Z$ with 
$k \ne 0$.  Prove that if  $k \mid a$ and  $k \mid \left( {a + 1} \right)$, then  $k \mid 1$, and hence  $k = \pm 1$. 

  \item Let  $a \in \mathbb{Z}$.  Find the greatest common divisor of the consecutive integers  $a$  and  $a + 1$.  That is, determine  $\gcd( {a, a + 1} )$.
\end{enumerate}

\item \begin{enumerate} \item Let  $a \in \Z$ and let  $k \in \Z$ with $k \ne 0$.  Prove that if  $k \mid a$ and  $k \mid \left( {a + 2} \right)$, then  $k \mid 2$.

  \item Let  $a \in \mathbb{Z}$.  What conclusions can be made about the greatest common divisor of  $a$  and  $a + 2$?  
\end{enumerate}

\xitem Let  $a, b \in \mathbb{Z}$ with  $b \ne 0$.  Prove each of the following:
\label{exer:sec81-props}

\begin{enumerate}
\item $\gcd( {0, b} ) = \left| b \right|$

\item If  $\gcd( {a, b} ) = d$, then  $\gcd( {a,  - b} ) = d$.  
That is, \\
$\gcd \left( {a,  - b} \right) = \gcd( {a, b} )$.
\end{enumerate}


\item For each of the following pairs of integers, use the Euclidean Algorithm to find  
$\gcd( {a, b} )$ and to write  $\gcd( {a, b} )$ as a linear combination of  $a$  and  $b$.  That is, find integers  $m$  and  $n$  such that  $d = am + bn$. \label{exer:sec81-4}

\begin{multicols}{2}
\begin{enumerate}
\yitem $a = 36, b = 60$
\yitem $a = 901, b = 935$
\item $a = 72, b = 714$
\item $a = 12628, b = 21361$
\yitem $a = 901$, $b = -935$
\item $a = -36$, $b = -60$
\end{enumerate}
\end{multicols}

\item \label{exer81:solvingeqn} \begin{enumerate}
\yitem Find integers $u$ and $v$ such that $9u + 14v = 1$ or explain why it is not possible to do so.  Then find integers $x$ and $y$ such that $9x + 14y = 10$ or explain why it is not possible to do so.

\item Find integers $x$ and $y$ such that $9x + 15y = 10$ or explain why it is not possible to do so.

\item Find integers $x$ and $y$ such that $9x + 15y = 3162$ or explain why it is not possible to do so.
\end{enumerate}


\item \label{exer:gcdandfractions}
\begin{enumerate} \yitem Notice that  $\gcd( {11, 17} ) = 1$. Find integers  $x$  and  $y$  such that  
\linebreak
$11x + 17y = 1$. \label{exer:gcdanddena}

  \yitem Let  $m, n \in \mathbb{Z}$.  Write the sum  $\dfrac{m}{{11}} + \dfrac{n}{{17}}$
as a single fraction. \label{exer:gcdanddenb}

  \item Find two rational numbers with denominators of  11  and  17, respectively , whose sum is equal to $\dfrac{{10}}{{187}}$.  \hint  Write the rational numbers in the form  
$\dfrac{m}{{11}}$  and  $\dfrac{n}{{17}}$,  where  $m, n \in \mathbb{Z}$.  Then write  
\[
\frac{m}{{11}} + \frac{n}{{17}} = \frac{{10}}{{187}}.
\]
Use Exercises~(\ref{exer:gcdanddena}) and~(\ref{exer:gcdanddenb}) to determine  $m$  and  $n$.

\item Find two rational numbers with denominators 17 and 21, respectively, whose sum is equal to 
$\dfrac{326}{357}$ or explain why it is not possible to do so.

\item Find two rational numbers with denominators 9 and 15, respectively, whose sum is equal to 
$\dfrac{10}{225}$ or explain why it is not possible to do so.

\end{enumerate}


%\item Visit the Euclidean Algorithm Web site at \label{exer:sec81-web}
%
%\begin{list}{}
%\item http://bigcheese.math.sc.edu/$\sim$ sumner/numbertheory/ \\
%euclidean/euclidean.html.
%\end{list}
%
%\begin{enumerate} 
%\item Use this Web site to determine  $\gcd( {40608, 151280} )$
% and to write this greatest common divisor as a linear combination of  40608  and  151280.
%
%\item Use this Web site to determine  $\gcd( {380488, 6251740} )$
% and to write this greatest common divisor as a linear combination of  
%$380488$ and $6251740$.
%\end{enumerate}
\end{enumerate}


\subsection*{Explorations and Activities}
\setcounter{oldenumi}{\theenumi}
\begin{enumerate} \setcounter{enumi}{\theoldenumi}
%\setcounter{oldenumi}{8}
%\begin{enumerate} \setcounter{enumi}{\theoldenumi} 
\item  \textbf{Linear Combinations and the Greatest Common Divisor} \label{exer81:lincomb} 
\begin{enumerate}
\item Determine the greatest common divisor of 20 and 12.

\item Let  $d = \gcd( {20, 12} )$.  Write  $d$  as a linear combination of  20  and  12.

\item Generate at least six different linear combinations of  20  and  12.  Are these linear combinations of 20 and 12 multiples of $\gcd( {20, 12} )$?  

\item Determine the greatest common divisor of 21 and $-6$ and then generate at least six different linear combinations of  21  and  $-6$.  Are these linear combinations of 21 and $-6$ multiples of 
$\gcd( {21, -6} )$?

\item The following proposition was first introduced in Exercise~(\ref{exer52-choose}) on page~\pageref{exer52-choose} in Section~\ref{S:provingset}.  Complete the proof of this proposition if you have not already done so.

\noindent
\textbf{Proposition \ref{P:divlinearcomb}}  \emph{Let a, b, and  t  be integers with $t \ne 0$.  If  t  divides  a  and  t  divides  b, then for all integers  x  and  y,  t  divides  
\text{(}ax + by\text{)}.}

\noindent
\textbf{\textit{Proof}}. Let $a$, $b$, and  $t$  be integers with $t \ne 0$, and assume that $t$  divides  $a$  and  $t$  divides  $b$.  We will prove that for all integers  $x$  and  $y$,  $t$  divides  $(ax + by)$.

So let  $x \in \mathbb{Z}$ and let  $y \in \mathbb{Z}$.  Since  $t$  divides  $a$, there exists an integer  $m$  such that $ \ldots .$

\item Now let $a$ and $b$ be integers, not both zero, and let $d = \gcd(a, b )$.  Theorem~\ref{T:gcdaslincomb} states that $d$ is a linear combination of $a$ and $b$.  In addition, let $S$ and $T$ be the following sets:
\[
S = \left\{ ax + by \mid x, y \in \Z \right\} \qquad \text{and} \qquad 
T = \left\{ kd \mid k \in \Z \right\}\!.
\]
That is, $S$ is the set of all linear combinations of $a$ and $b$, and $T$ is the set of all multiples of the greatest common divisor of $a$ and $b$.  Does the set $S$ equal the set $T$?  If not, is one of these sets a subset of the other set?  Justify your conclusions.

\note In Parts (c) and (d), we were exploring special cases for these two sets.
\end{enumerate}

\end{enumerate}


\hbreak
%
%\markboth{Chapter~\ref{C:numbertheory}. Topics in Number Theory}{\ref{S:primefactorizations}. 
%Prime Factorizations}


\endinput
