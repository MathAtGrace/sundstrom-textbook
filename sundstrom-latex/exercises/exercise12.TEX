\section*{Exercises for Section~\ref{S:direct}}
\begin{enumerate}

\item Construct a know-show table for each of the following statements and then write a formal proof for one of the statements. 
\label{exer:evenodd}%
  \begin{enumerate}
    \yitem If  $m$  is an even integer, then  $m + 1$  is an odd integer.
    \item If  $m$  is an odd integer, then  $m + 1$ is an even integer.
  \end{enumerate}
  \label{exer:nextint}%

%\item Write a complete formal proof of one of the statements in Exercise~(\ref{exer:evenodd}).
%\label{exer:evenodd2}

\item \label{exer:evenoddadd}%
Construct a know-show table for each of the following statements and then write a formal proof for one of the statements.  

  \begin{enumerate}
    \item If  $x$  is an even integer and  $y$  is an even integer, then  $x + y$ is an even integer.
    \item If  $x$  is an even integer and  $y$  is an odd integer, then  $x + y$ is an odd integer.
    \yitem If  $x$  is an odd integer and  $y$  is an odd integer, then  $x + y$ is an even integer.
  \end{enumerate}
  \label{exer:integeradd}%

%\item Write a complete formal proof of one of the statements in Exercise~(\ref{exer:evenoddadd}).
%\label{exer:evenoddadd2}

\item \label{exer:evenoddmult}%
Construct a know-show table for each of the following statements and then write a formal proof for one of the statements.  

  \begin{enumerate}
    \yitem If  $m$  is an even integer and  $n$  is an integer, then  $m \cdot n$ is an even integer.
    \yitem If  $n$  is an even integer, then  $n^2$ is an even integer.
    \item \label{exer:x2odd}%
       If  $n$  is an odd integer, then  $n^2$ is an odd integer. 
  \end{enumerate}
  \label{exer:integermult}%

%\item Write a complete formal proof of one of the statements in Exercise~(\ref%{exer:evenoddmult}).
%\label{exer:evenoddmult2}

\item Construct a know-show table and write a complete proof for each of the following statements:

\begin{enumerate}
\yitem If $m$ is an even integer, then $5m + 7$ is an odd integer.

\item If $m$ is an odd integer, then $5m + 7$ is an even integer.

\item If $m$ and $n$ are odd integers, then $mn + 7$ is an even integer.
\end{enumerate}\label{exer:5m+7}%

\item \label{exer:3m2}%
Construct a know-show table and write a complete proof for each of the following statements: 

\begin{enumerate}
\item If $m$ is an even integer, then $3m^2 + 2m + 3$ is an odd integer.
\yitem If $m$ is an odd integer, then $3m^2 + 7m + 12$ is an even integer.
\end{enumerate}

\item \label{exer:sec12-7}%
In this section, it was noted that there is often more than one way to answer a backward question.  For example, if the backward question is, ``How can we prove that two real numbers are equal?'', one possible answer is to prove that their difference equals 0.  Another possible answer is to prove that the first is less than or equal to the second and that the second is less than or equal to the first. 

  \begin{enumerate}
    \yitem Give at least one more answer to the backward question, ``How can we prove that two real numbers are equal?''
    \item List as many answers as you can for the backward question, ``How can we prove that a real number is equal to zero?''
    \item List as many answers as you can for the backward question, ``How can we prove that two lines are parallel?''

    \yitem List as many answers as you can for the backward question, ``How can we prove that a triangle is isosceles?''

  \end{enumerate}


%\item The \textbf{Pythagorean Theorem}\label{exer:sec12-pythag}
%\index{Pythagorean Theorem}%
%\label{exer:pythag}%
%for right triangles states that if $a$ and $b$ are the lengths of the legs of a right triangle and $c$ is the length of the hypotenuse, then $a^2 + b^2 = c^2$.
%
%Prove that if $m$ is a real number and $m$, $m + 1$, and $m + 2$ are the lengths of the three sides of a right triangle, then $m = 3$. 


\item Are the following statements true or false?  Justify your conclusions. \label{exer:sec12-new7}
\begin{enumerate}
  \item If $a$, $b$, and $c$ are integers, then $ab + ac$ is an even integer.
  \item If $b$ and $c$ are odd integers and $a$ is an integer, then $ab + ac$ is an even integer.
\end{enumerate}





\item \label{exer:sec12-8}%
Is the following statement true or false?  Justify your conclusion.
\begin{list}{}
\item If  $a$  and  $b$  are nonnegative real numbers and  $a + b = 0$, then  $a = 0$.
\end{list}
 
Either give a counterexample to show that it is false or outline a proof by completing a know-show table.

%\item An integer $a$ is said to be \textbf{congruent to 1 modulo 3} if there exists an integer $n$ such that $a=3n+1$. An integer $a$ is said to be \textbf{congruent to 2 modulo 3} if there exists an integer $m$ such that $a=3m+2$. \label{exer:sec12-9}
%
%  \begin{enumerate}
%    \item Give examples of at least four different integers that are congruent to 1 modulo 3. \label{exer:mod3a}
%    \item Give examples of at least four different integers that are congruent to 2 modulo 3.
%    \item By multiplying pairs of integers from the list in Exercise~(\ref{exer:mod3a}), does it appear that the following statement is true or false?  
%
%\begin{list}{}
%\item If $a$ is congruent to 1 modulo 3 and $b$ is congruent to 1 modulo 3, then $ a \cdot b$ is congruent to 1 modulo 3.
%\end{list}
%
%  \end{enumerate}

\item \label{exer:sec12-type}%
An integer $a$ is said to be a \textbf{type 0 integer} if there exists an integer $n$ such that 
$a=3n$. An integer $a$ is said to be a \textbf{type 1 integer} if there exists an integer $n$ such that $a=3n+1$. An integer $a$ is said to be a \textbf{type 2 integer} if there exists an integer $m$ such that $a=3m+2$. 

\begin{enumerate}
    \yitem \label{exer:mod3a}%
         Give examples of at least four different integers that are type 1 integers. 

    \item Give examples of at least four different integers that are type 2 integers.
    \yitem By multiplying pairs of integers from the list in Exercise~(\ref{exer:mod3a}), does it appear that the following statement is true or false?  

\begin{list}{}
\item If $a$ and $b$ are both type 1 integers, then $ a \cdot b$ is a type 1 integer.
\end{list}
\end{enumerate}

%\item Write a proof for each of the following statements: \label{exer:sec12-10}
%  \begin{enumerate}
%    \item If $a$ is congruent to 1 modulo 3 and $b$ is congruent to 1 modulo 3, then $ a + b$ is congruent to 2 modulo 3.
%    \item If $a$ is congruent to 2 modulo 3 and $b$ is congruent to 2 modulo 3, then $ a + b$ is congruent to 1 modulo 3.
%    \item If $a$ is congruent to 1 modulo 3 and $b$ is congruent to 2 modulo 3, then $ a \cdot b$ is congruent to 2 modulo 3.
%    \item If $a$ is congruent to 2 modulo 3 and $b$ is congruent to 2 modulo 3, then $ a \cdot b$ is congruent to 1 modulo 3.
%  \end{enumerate}

\item \label{exer:sec12-typeproof}%
Use the definitions in Exercise~(\ref{exer:sec12-type}) to help write a proof for each of the following statements: 
  \begin{enumerate}
    \yitem If $a$ and $b$ are both type 1 integers, then $ a + b$ is a type 2 integer.
    \item If $a$ and $b$ are both type 2 integers, then $ a + b$ is a type 1 integer.
    \item If $a$ is a type 1 integer and $b$ is a type 2 integer, then $ a \cdot b$ is a type 2 integer.
    \item If $a$ and $b$ are both type 2 integers, then $ a \cdot b$ is type 1 integer.
  \end{enumerate}

%\item For a right triangle, suppose that the hypotenuse has length $c$ feet and the lengths of %the sides are $a$ feet and $b$ feet.
%  \begin{enumerate}
%   \item What is a formula for the area of this right triangle?
%    \item What can be concluded from the Pythagorean Theorem for right triangles?
%    \item What is an isosceles triangle?
%    \item Prove that if the right triangle described above is an isosceles triangle, then the %area of the right triangle is $\frac{1}{4} c^2$.
%  \end{enumerate}

\item  \label{exer:sec12-11}%
Let $a$, $b$, and $c$ be real numbers with $a \ne 0$.  The solutions of the \textbf{quadratic equation} 
$ax^2 + bx + c = 0$ 
\index{quadratic equation}%
are given by the \textbf{quadratic formula}, 
\index{quadratic formula}%
 which states that the solutions are $x_1$ and $x_2$, where
\[
x_1 = \frac{-b + \sqrt{b^2 - 4ac}}{2a} \quad \text{and} \quad x_2 = \frac{-b - \sqrt{b^2 - 4ac}}{2a}.
\]
\begin{enumerate}
\item Prove that the sum of the two solutions of the quadratic equation \linebreak $ax^2 + bx + c = 0$ is equal to 
$- \dfrac{b}{a}$.
\item Prove that the product of the two solutions of the quadratic equation $ax^2 + bx + c = 0$ is equal to 
$\dfrac{c}{a}$.
\end{enumerate}
%\hbreak


\item \label{exer:sec12-quadratic2}
\begin{enumerate}
\item See Exercise~(\ref{exer:sec12-11}) for the quadratic formula, which gives the solutions to a quadratic equation.  Let $a$, $b$, and $c$ be real numbers with $a \ne 0$.  The discriminant of the quadratic equation $ax^2 + bx + c = 0$ is defined to be $b^2 - 4ac$.  Explain how to use this discriminant to determine if the quadratic equation has two real number solutions, one real number solution, or no real number solutions.

\item Prove that if $a$, $b$, and $c$ are real numbers with $a > 0$ and $c < 0$, then one solutions of the quadratic equation $ax^2 + bx + c = 0$ is a positive real number.

\item Prove that if $a$, $b$, and $c$ are real numbers with $a \ne 0$, $b > 0$, and $b < 2 \sqrt{ac}$, then the quadratic equation $ax^2 + bx + c = 0$ has no real number solutions.

\end{enumerate}

%\item A \textbf{2 by 2 matrix over}
%\index{matrix}%
%  $\mathbb{R}$ is a rectangular array of four real numbers arranged in two rows and two columns.  We usually write this array inside brackets (or parentheses) as follows:
%\[
%A = \twotwo{a}{b}{c}{d},
%\]
%where  $a$, $b$, $c$, and  $d$  are real numbers.  The \textbf{determinant}
%\index{determinant}%
%\index{matrix!determinant}%
% of the 2 by 2 matrix  $A$, denoted by  $\det( A )$ 
%\label{sym:determinant},  is defined as
%\[
%\det( A ) = ad - bc .
%\]
%\label{exer:determinant}
%\begin{enumerate} 
%\yitem Calculate the determinant of each of the following matrices: 
%\[
%\twotwo{3}{5}{4}{1} \quad \twotwo{1}{0}{0}{7} \quad \twotwo{3}{-2}{5}{0}
%\]
%\end{enumerate}
%Let $\mathscr{M}_{2}( \R )$ be the set of all  2 by 2  matrices over $\R$ and let \linebreak 
%$S = \left\{ A \in \mathscr{M}_2 \mid \det(A) = 1 \right\}$.
%\begin{enumerate} \setcounter{enumii}{1}
%  \item Is the set $S$ closed under matrix addition?  Justify your conclusion.
%  \item Is the set $S$ closed under matrix multiplication?  Justify your conclusion.
%\end{enumerate}
%\note Recall that the sum and product of two matrices in $\mathscr{M}_{2}( \R )$ is defined as follows:
%\begin{align*}
%\twotwo{a}{b}{c}{d} + \twotwo{w}{x}{y}{z} &= \twotwo{a+w}{b+x}{c+y}{d+z}. \\
%\twotwo{a}{b}{c}{d} \twotwo{w}{x}{y}{z} &= \twotwo{aw+by}{ax+bz}{cw+dy}{cx+dz}.
%\end{align*}
%


%\item Is the following proposition true or false?  Justify your conclusion.
%\begin{list}{}
%\item If $m$ is a real number and $m$, $m + 7$, and $m + 8$ are the lengths of the three sides of a right triangle, then $m = 5$.
%\end{list}


\end{enumerate}


\subsection*{Explorations and Activities}
\setcounter{oldenumi}{\theenumi}
\begin{enumerate} \setcounter{enumi}{\theoldenumi}
  \item \textbf{Pythagorean Triples}.  Three natural numbers $a$, $b$, and $c$ with $a < b < c$ are said to form a 
\textbf{Pythagorean triple} \label{exer:pythag}
\index{Pythagorean triple}%
\label{exer31:pythagorean}
 provided that $a^2 + b^2 = c^2$.  For example, 3, 4, and 5 form a Pythagorean triple since 
$3^2 + 4^2 = 5^2$.  The study of Pythagorean triples began with the development of the \textbf{Pythagorean Theorem} for right triangles, which states that if $a$ and $b$ are the lengths of the legs of a right triangle and $c$ is the length of the hypotenuse, then 
$a^2 + b^2 = c^2$.  For example, if the lengths of the legs of a right triangle are 4 and 7 units, then 
$c^2 = 4^2 + 7^2 = 63$, and the length of the hypotenuse must be $\sqrt{63}$ units (since the length must be a positive real number).  Notice that $4$, $7$, and $\sqrt{63}$ are not a Pythagorean triple since 
$\sqrt{63}$ is not a natural number.
\begin{enumerate}
  \item Verify that each of the following triples of natural numbers forms a Pythagorean triple.
\begin{multicols}{3}
\begin{itemize}
\item 3, 4, and 5
\item 6, 8, and 10
\item 8, 15, and 17
\item 10, 24, and 26
\item 12, 35, and 37
\item 14, 48, and 50
\end{itemize}
\end{multicols}

  \item Does there exist a Pythagorean triple of the form $m$, $m + 7$, and $m + 8$, where $m$ is a natural number?  If the answer is yes, determine all such Pythagorean triples.  If the answer is no, prove that no such Pythagorean triple exists.
  \item Does there exist a Pythagorean triple of the form $m$, $m + 11$, and $m + 12$, where $m$ is a natural number?  If the answer is yes, determine all such Pythagorean triples.  If the answer is no, prove that no such Pythagorean triple exists.
\end{enumerate}



\item \textbf{More Work with Pythagorean Triples}. \label{exer:morepythag}
In Exercise~({\ref{exer:pythag}}), we verified that all of the following triples of natural numbers are Pythagorean triples:  
\begin{multicols}{3}
\begin{itemize}
\item 3, 4, and 5
\item 6, 8, and 10
\item 8, 15, and 17
\item 10, 24, and 26
\item 12, 35, and 37
\item 14, 48, and 50
\end{itemize}
\end{multicols}

\begin{enumerate} \label{exer:sec12-morepythag}
  \item Focus on the least even natural number in each of these Pythagorean triples.  Let  $n$ be this even number and find $m$ so that $n = 2m$.  Now try to write formulas for the other two numbers in the Pythagorean triple in terms of $m$.  For example, for 3, 4, and 5, $n = 4$ and $m = 2$. %, and for 8, 15, and 17, $n = 8$ and $m = 4$.  
Once you think you have formulas, test your results with $m = 10$.  That is, check to see that you have a Pythagorean triple whose smallest even number is 20.
\item Write a proposition and then write a proof of the proposition.  The proposition should be in the form:  If $m$ is a natural number and $m \geq 2$, then \ldots \ldots
\end{enumerate}

\end{enumerate}
\hbreak
%%\hrule


%\subsection*{Proofs to Evaluate}
%These type of exercises will appear at the end of many sections of the book1 x.  In each case, there is a proposed proof of a proposition.  However, the proposition may be true or may be false.  
%
%\begin{itemize}
%\item If a proposition is false, the proposed proof is, of course, incorrect.  In this situation, you are to find the error in the proof and then provide a counterexample showing that the proposition is false.
%
%\item If a proposition is true, the proposed proof may still be incorrect.  In this case, you are to determine why the proof is incorrect and then write a correct proof.
%
%\item If a proposition is true and the proof is correct, you are to give the proof a grade of $A$ if it is well-written but give it a grade of $C$ if it is not well-written.  You must then revised the proof so that it is well-written.
%\end{itemize}

\section{Chapter \ref{C:intro} Summary}
\subsection*{Important Definitions}
\begin{multicols}{2}
\begin{itemize}
\item Statement, page~\pageref*{D:prop}
\item Conditional statement, pages~\pageref*{PA:prop}, ~\pageref*{D:conditional}
\item Even integer, page~\pageref*{D:even}
\item Odd integer, page~\pageref*{D:even}
\item Pythagorean triple, page~\pageref*{exer:pythag}
\end{itemize}
\end{multicols}

\endinput



