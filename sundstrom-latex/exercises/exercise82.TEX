\section*{Exercises 8.2}
%
\begin{enumerate}
\xitem Prove the second and third parts of Theorem~\ref{T:relativelyprime}. \label{exer:sec82-1} 

\begin{enumerate}
  \item Let  $a$ be a nonzero integer, and let  $p$  be a  prime number.  If  $p \mid a$, then  $\gcd( {a, p} ) = p$.

  \item Let  $a$ be a nonzero integer, and let  $p$  be a  prime number.  If  $p$ does not divide $a$, then  $\gcd( {a, p} ) = 1$.
\end{enumerate}

\xitem Prove the first part of Corollary~\ref{C:primedivides}. 
\label{exer:sec82-2}%

Let  $a, b \in \mathbb{Z}$, and let  $p$  be a prime number.  If  
$p \mid \left( {ab} \right)$, then  $p \mid a$  or  $p \mid b$.  
\hint  Consider two cases: 
(1) $p \mid a$; and (2) $p$ does not divide $a$.

\xitem Use mathematical induction to prove the second part of Corollary~\ref{C:primedivides}. \label{exer:sec82-3}%

Let  $p$  be a prime number, let  $n \in \mathbb{N}$, and let  
$a_1 ,a_2 , \ldots , a_n  \in \mathbb{Z}$.  If  \linebreak
$p \mid \left( {a_1 a_2  \cdots a_n } \right)$, then there exists a  $k \in \mathbb{N}$
 with  $1 \leq k \leq n$ such that  $p \mid a_k $.

\xitem 
\label{exer:lincombequalone}% 
\begin{enumerate} \item Let  $a$  and  $b$  be nonzero integers.  If there exist integers  $x$  and  $y$  such that  $ax + by = 1$, what conclusion can be made about  
$\gcd( {a, b} )$?  Explain.

  \item Let  $a$  and  $b$  be nonzero integers.  If there exist integers  $x$  and  $y$  such that  $ax + by = 2$, what conclusion can be made about  $\gcd( {a, b} )$?  Explain.
\end{enumerate}

\item \label{exer:sec82-gcdconsec}
\begin{enumerate} \item Let  $a \in \mathbb{Z}$.  What is  
$\gcd( {a, a + 1} )$?  That is, what is the greatest common divisor of two consecutive integers? Justify your conclusion. 

\hint  Exercise~(\ref{exer:lincombequalone}) might be helpful.

  \item Let  $a \in \mathbb{Z}$.  What conclusion can be made about  
$\gcd( {a, a + 2} )$?  That is, what conclusion can be made about the greatest common divisor of two integers that differ by 2? Justify your conclusion.
\end{enumerate}

\item \label{exer:sec82-moregcd}
\begin{enumerate}
\item Let  $a \in \mathbb{Z}$.  What conclusion can be made about  
$\gcd( {a, a + 3} )$?  That is, what conclusion can be made about the greatest common divisor of two integers that differ by 3? Justify your conclusion.

\item Let  $a \in \mathbb{Z}$.  What conclusion can be made about  
$\gcd( {a, a + 4} )$?  That is, what conclusion can be made about the greatest common divisor of two integers that differ by 4? Justify your conclusion.
\end{enumerate}

\item \begin{enumerate} \label{exer:dividebygcd}
\yitem Let $a = 16$ and $b = 28$.  Determine the value of $d = \gcd( a, b )$, and then determine the value of $\gcd \!\left( {\dfrac{a}{d}, \dfrac{b}{d}} \right )$.
\label{exer:gcd-divbyd}

\yitem Repeat Exercise~(\ref{exer:gcd-divbyd}) with $a = 10$ and $b = 45$.

\item Let $a, b \in \mathbb{Z}$, not both equal to 0, and let $d = \gcd( a, b )$.
Explain why $\dfrac{a}{d}$ and $\dfrac{b}{d}$ are integers.  Then prove that 
$\gcd \!\left( {\dfrac{a}{d}, \dfrac{b}{d}} \right ) = 1$.  \hint  Start by writing $d$ as a linear combination of $a$ and $b$.

\end{enumerate}

This says that if you divide both  $a$  and  $b$  by their greatest common divisor, the result will be two relatively prime integers.

\item Are the following propositions true or false? Justify your conclusions.
\label{exer:sec82-truefalse}

\begin{enumerate}
  \item For all integers $a, b, \text{ and }c$, if $a \mid c$ and  $b \mid c$, then  
        $\left( {ab} \right)  \mid  c$.

  \item For all integers $a, b, \text{ and }c$, if  $a \mid c$,  $b \mid c$ and  
        $\gcd( {a, b} ) = 1$, then  $( {ab} )  \mid  c$.
\end{enumerate}

\xitem In Exercise~(\ref{exer:sec34-new9}) in Section~\ref{S:divalgo}, it was proved that if $n$ is an odd integer, then $8 \mid \left( n^2 - 1 \right)$.  (This result was also proved in 
Exercise~(\ref{exer:sec74-15}) in Section~\ref{S:modulararithmetic}.)  Now, prove the following proposition: \label{exer:24divides-nsquaredminus1}

\begin{list}{}
\item If $n$ is an odd integer and 3 does not divide $n$, then $24 \mid \left( n^2 - 1 \right)$.
\end{list}

\item \begin{enumerate} \item Prove the following proposition:

For all  $a, b, c \in \mathbb{Z}$, $\gcd(a, bc) = 1$ if and only if $\gcd( {a, b} ) = 1$ and  
$\gcd( {a, c} ) = 1$.

  \item Use mathematical induction to prove the following proposition:

Let  $n \in \mathbb{N}$ and let  $a, b_1 , b_2 ,  \ldots , b_n  \in \mathbb{Z}$.  If  
$\gcd \!\left( {a, b_i } \right) = 1$ for all  $i \in \mathbb{N}$ with  
$1 \leq i \leq n$, then  $\gcd \!\left( {a, b_1 b_2  \cdots b_n } \right) = 1$.
\end{enumerate}

\xitem Is the following proposition true or false? Justify your conclusion.
\label{exer:cdividessum}
\begin{list}{}
\item For all integers $a, b, \text{ and }c$, if   $\gcd( {a, b} ) = 1$  and  
$c \mid \left( {a + b} \right)$, then  $\gcd( {a, c} ) = 1$  and  
$\gcd( {b, c} ) = 1$.
\end{list}

\item Is the following proposition true or false? Justify your conclusion.
\begin{list}{}
\item If  $n \in \mathbb{N}$, then  $\gcd( {5n + 2, 12n + 5} ) = 1$.
\end{list}

\item Let $y \in \mathbb{N}$.  Use the Fundamental Theorem of Arithmetic to prove that there exists an odd natural number $x$ and a nonnegative integer $k$ such that \linebreak 
$y = 2^k x$. 
\label{exer:fundtheoremcons}

\item 
\label{exer:sec82-primescong3}%
\begin{enumerate} \item Determine five different primes that are congruent to  3  modulo  4.

  \item Prove that there are infinitely many primes that are congruent to  3  modulo  4.
\end{enumerate}

\item \label{exer:consecutivecomposites} \begin{enumerate} \item Let  $n \in \mathbb{N}$.  Prove that  2 divides $\left[ {\left( {n + 1} \right)!  + 2} \right]$.

  \item Let  $n \in \mathbb{N}$ with  $n \geq 2$.  Prove that  
        3 divides $\left[ {\left( {n + 1} \right)!  + 3} \right]$. 

  \item Let  $n \in \mathbb{N}$.  Prove that for each  $k \in \mathbb{N}$ with  
        $2 \leq k \leq \left( {n + 1} \right)$,  $k$ divides  
        $\left[ {\left( {n + 1} \right)!  + k} \right]$.                  
  \label{exer:consecutivecompositesc}

  \item Use the result of Exercise~(\ref{exer:consecutivecompositesc}) to prove that for each          $n \in \mathbb{N}$, there exist at least  $n$  consecutive composite natural numbers.
\end{enumerate}

\item The Twin Prime Conjecture states that there are infinitely many twin primes, but it is not known if this conjecture is true or false.  The answers to the following questions, however, can be determined.

\begin{enumerate}
\item How many pairs of primes  $p$ and $q$ exist where  $q - p = 3$?  That is, how many pairs of primes are there that differ by 3?  Prove that your answer is correct.  (One such pair is 2 and 5.)

\item How many triplets of primes of the form $p$, $p+2$, and $p+4$ are there?  That is, how many triplets of primes exist where each prime is 2 more than the preceding prime?  Prove that your answer is correct.  Notice that one such triplet is 3, 5, and 7.

\hint  Try setting up cases using congruence modulo 3.

\end{enumerate}

\item Prove the following proposition:
\begin{list}{}
\item Let  $n \in \mathbb{N}$.  For each  $a \in \mathbb{Z}$, if  
$\gcd( {a, n} ) = 1$, then for every  $b \in \mathbb{Z}$, there exists an  
$x \in \mathbb{Z}$ such that  $ax \equiv b \pmod n$.
\end{list}

\hint  One way is to start by writing 1 as a linear combination of $a$ and $n$.

\item  Prove the following proposition: \label{exer:sec82-twinprimes}
\begin{list}{}
\item For all natural numbers $m$ and $n$, if $m$ and $n$ are twin primes other than the pair 3 and 5, then 36 divides $mn + 1$ and $mn + 1$ is a perfect square.
\end{list}

\hint  Look at several examples of twin primes.  What do you  notice about the number that is between the two twin primes?  Set up cases based on this observation.

\end{enumerate}




\hbreak
%\markboth{Chapter~\ref{C:numbertheory}. Topics in Number Theory}{\ref{S:diophantine}. Linear 
%Diophantine Equations}

\subsection*{Explorations and Activities}
\setcounter{oldenumi}{\theenumi}
\begin{enumerate} \setcounter{enumi}{\theoldenumi}
\item \textbf{Square Roots and Irrational Numbers}.  \label{exer82:squareroot}  In Chapter~\ref{C:proofs}, we proved that some square roots (such as $\sqrt{2}$ and 
$\sqrt{3}$) are irrational numbers.  In this activity, we will use the Fundamental Theorem of Arithmetic to prove that if a natural number is not a perfect square, then its square root is an irrational number.

\setcounter{equation}{0}
\begin{enumerate}
\item \label{exer82:squareroot-a}Let $n$ be a natural number.  Use the Fundamental Theorem of Arithmetic to explain why if $n$ is composite, then there exist distinct prime numbers $p_1, p_2, \ldots, p_r$ and natural numbers 
$\alpha_1, \alpha_2, \ldots, \alpha_r$ such that
\begin{equation}\label{A:squarerootsirrational1}
n = p_1^{\alpha_1} p_2^{\alpha_2} \cdots p_r^{\alpha_r}.
\end{equation}
Using $r = 1$ and $\alpha_1 = 1$ for a prime number, explain why we can write any natural number greater than one in the form given in equation~(\ref{A:squarerootsirrational1}).

\item A natural number $b$ is a \textbf{perfect square}
\index{perfect square}%
 if and only if there exists a natural number $a$ such that $b = a^2$.  Explain why 36, 400, and 15876 are perfect squares.  Then determine the prime factorization of these perfect squares.  What do you notice about these prime factorizations?

\item Let $n$ be a natural number written in the form given in 
equation~(\ref{A:squarerootsirrational1}) in part~(a).  Prove that $n$ is a perfect square if and only if for each natural number $k$ with $1 \leq k \leq r$, $\alpha_k$ is even.

\item Prove that for all natural numbers $n$, if $n$ is not a perfect square, then 
$\sqrt{n}$ is an irrational number. \hint  Use a proof by contradiction.
\end{enumerate}
\hbreak

\end{enumerate}

\endinput
