\section*{Exercises 6.2}
%
\begin{enumerate}
\xitem Let  $R_5  = \left\{ {0, 1, 2, 3, 4} \right\}$.  Define  
$f\x R_5  \to R_5 $  by $f( x ) = x^2 + 4 \pmod 5$, 
\label{exer:sec62-1}
and define $g\x R_5  \to R_5 $  by 
$g( x ) = (x + 1)(x + 4) \pmod 5$.

\begin{enumerate}
  \item Calculate  $f( 0 )$, $f( 1 ), f( 2 )$, 
$f( 3 )$, and $f( 4 )$. 

  \item Calculate  $g( 0 )$, $g( 1 ), g( 2 )$, 
$g( 3 )$, and $g( 4 )$.

  \item Is the function  $f$  equal to the function $g$?  Explain.
\end{enumerate}

\item Let  $R_6  = \left\{ {0, 1, 2, 3, 4, 5} \right\}$.  Define  
$f\x R_6  \to R_6$  by $f( x ) = x^2 + 4 \pmod 6$, 
\label{exer:sec62-2}
and define $g\x R_6  \to R_6 $  by 
$g( x ) = (x + 1)(x + 4) \pmod 6$.

\begin{enumerate}
  \item Calculate  $f( 0 )$, $f( 1 ), f( 2 )$, 
$f( 3 )$, $f( 4 )$, and $f( 5 )$. 

  \item Calculate  $g( 0 )$, $g( 1 ), g( 2 )$, 
$g( 3 )$, $g( 4 )$, and $g( 5 )$.

  \item Is the function  $f$  equal to the function $g$?  Explain.
\end{enumerate}

%\item Let  $R_5  = \left\{ {0, 1, 2, 3, 4} \right\}$.  Define  
%$g:R_5  \to R_5 $  by $g\left( x \right) = x^5 \pmod 5$.
%
%\begin{enumerate}
%  \item Calculate  $g\left( 0 \right)$, $g\left( 1 \right)$, $g\left( 2 \right)$, 
%$g\left( 3 \right)$, and $g\left( 4 \right)$.
%
%  \item Represent the function  $g$  with an arrow diagram.
%
%  \item Represent the function  $g$   as a set of ordered pairs.
%
%  \item The function  $g$  is equal to what standard function on  $R_5 $?  Explain.
%\end{enumerate}

\xitem Let $f\x \left(\R - \{0 \} \right) \to \R$ by $f(x) = \dfrac{x^3 + 5x}{x}$ and let 
$g\x \R \to \R$ by \linebreak $g(x) = x^2 + 5$. \label{exer62:realfunction}

\begin{enumerate}
  \item Calculate  $f( 2 ), f( -2 )$, 
$f( 3 )$, and $f( \sqrt{2} )$.


  \item Calculate  $g( 0 )$, $g( 2 ), g( -2 )$, 
$g( 3 )$, and $g( \sqrt{2} )$.

  \item Is the function  $f$  equal to the function $g$?  Explain.

  \item Now let $h\x \left(\R - \{0 \} \right) \to \R$ by $h(x) = x^2 + 5$.  Is the function  
$f$  equal to the function $h$?  Explain.
\end{enumerate}

\item Represent each of the following sequences as functions.  In each case, state a domain, codomain, and rule for determining the outputs of the function.  Also, determine if any of the sequences are equal. \label{exer:sec62-3}

\begin{enumerate}
  \yitem $1, \dfrac{1}{4}, \dfrac{1}{9}, \dfrac{1}{{16}},  \ldots $

  \item $\dfrac{1}{3}, \dfrac{1}{9}, \dfrac{1}{{27}}, \dfrac{1}{{81}},  \ldots $

  \item $1,  - 1, 1,  - 1, 1,  - 1,  \ldots $

  \yitem $\cos( 0 ), \cos( \pi  ), \cos( {2\pi } ), \cos( {3\pi } ), \cos( {4\pi } ),  \ldots $
\end{enumerate}

\item Let  $A$  and  $B$  be two nonempty sets.  There are two \textbf{projection functions}
\index{projection function}%
\index{function!projection}%
 with domain  $A \times B$, the Cartesian product of  $A$  and  $B$.  One projection function will map an ordered pair to its first coordinate, and the other projection function will map the ordered pair to its second coordinate.   So we define  \label{sym:projfunc}
\begin{list}{}
\item $p_1\x A \times B \to A$ by $p_1( {a, b} ) = a$ for every  
$( {a, b} ) \in A \times B$; and 

\item $p_2\x A \times B \to B$  by  $p_2( {a, b} ) = b$  for every  
$( {a, b} ) \in A \times B$. 
\end{list}

Let  $A = \left\{ {1, 2} \right\}$  and let  $B = \left\{ {x, y, z} \right\}$.  

\begin{enumerate}
  \yitem Determine the outputs for all possible inputs for the projection function  
$p_1\x A \times B \to A$.

  \item Determine the outputs for all possible inputs for the projection function  
$p_2\x A \times B \to B$.

  \yitem What is the range of these projection functions?  

  \item Is the following statement true or false?  Explain.

  \begin{list}{}
  \item For all  $( {m, n} ), ( {u, v} ) \in A \times B$, if  
    $( {m, n} ) \ne ( {u, v} )$, then  \\
    $p_1( {m, n} ) \ne p_1( {u, v} )$.
  \end{list}
\end{enumerate}

%\item In Exercise~(\ref{exer:numberofdivisors}) from Section~\ref{S:introfunctions}, we introduced the \textbf{number of divisors function}
%\index{number of divisors function}%
%  $d$.  For this function, $d\x \mathbb{N} \to \mathbb{N}$,
%where  $d( n )$ is the number of natural number divisors of  $n$.  
%
%A function that is related to this function is the so-called \textbf{set of divisors function}.
%\index{set of divisors function}%
%  This can be defined as a function  $S$  that associates with each natural number the set of its distinct natural number factors.  For example,  
%$S( 6 ) = \left\{ {1, 2, 3, 6} \right\}$. \label{exer:sec62-5}
%
%\begin{enumerate}
%\yitem Discuss the function  $S$  by carefully stating its domain, codomain, and its rule for determining outputs.
%
%\yitem Determine  $S( n )$  for at least five different values of  $n$.
%
%\yitem Determine  $S( n )$  for at least three different prime number values of  $n$.
%
%\item Does there exist a natural number  $n$  such that  
%$\left| {S( n )} \right| = 1$?  Explain.  [Recall that  
%$\left| {S( n )} \right|$ represents the number of elements in the set  
%$S( n )$.]
%
%\item Does there exist a natural number  $n$  such that  
%$\left| {S( n )} \right| = 2$?  Explain.
%
%\item Write the output for the function  $d$  in terms of the output for the function  $S$.  That is, write  $d( n )$  in terms of  $S( n )$.
%
%\item Is the following statement true or false?  Justify your conclusion.
%\begin{list}{}
%\item For all natural numbers $m$ and $n$, if $m \ne n$, then 
%$S ( m ) \ne S ( n )$.
%\end{list}
%
%\item Is the following statement true or false?  Justify your conclusion.
%\begin{list}{}
%\item For all sets $T$ that are subsets of $\mathbb{N}$, there exists a natural number $n$ such that $S( n ) = T$.
%\end{list}
%
%\end{enumerate}

\xitem Let  $D = \mathbb{N} - \left\{ {1, 2} \right\}$ and define  
$d\x D \to \mathbb{N} \cup \left\{ 0 \right\}$  by  $d( n ) = $ the number of diagonals of a convex polygon with  $n$  sides.  In \typeu Activity~\ref*{PA:diagonals}, we showed that for values of  $n$  from  3 through 8,  
\[
d( n ) = \frac{{n\left( {n - 3} \right)}}{2}.
\]

Use mathematical induction to prove that for all  $n \in D$,
\begin{center}
$d( n ) = \dfrac{{n\left( {n - 3} \right)}}{2}$.
\end{center}

\hint  To get an idea of how to handle the inductive step, use a pentagon.  First, form all the diagonals that can be made from  four  of the vertices.  Then consider how to make new diagonals when the fifth vertex is used.  This may generate an idea of how to proceed from a polygon with  $k$  sides  to a polygon with  $k + 1$  sides.  \label{exer:sec62-6}


\xitem Let $f\x \Z \times \Z \to \mathbb{Z}$ be defined by  
$f( {m, n} ) = m + 3n$. \label{exer:sec61-7}

\begin{enumerate}
  \item Calculate  $f( { - 3, 4} )$ and  $f( { - 2, - 7} )$.

  \item Determine the set of all the preimages of  4 by using set builder notation to describe the set of all $\left( {m, n} \right) \in \mathbb{Z} \times \mathbb{Z}$ such that  
$f( {m, n} ) = 4$.
\end{enumerate}

\item Let $g\x \mathbb{Z} \times \mathbb{Z} \to \mathbb{Z} \times \mathbb{Z}$ be defined by  
$g( {m, n} ) = \left( {2m, m - n} \right)$.  
\label{exer:sec61-8}%

\begin{enumerate}
  \yitem Calculate  $g( {3, 5} )$ and  $g( { - 1, 4} )$.

  \item Determine all the preimages of  $\left( {0, 0} \right)$.  That is, find all  
$( {m, n} ) \in \mathbb{Z} \times \mathbb{Z}$ such that  $g( {m, n} ) = ( {0, 0} )$.

  \yitem Determine the set of all the preimages of  $( {8,  - 3} )$.

  \item Determine the set of all the preimages of  $( {1, 1} )$.

  \item Is the following proposition true or false?  Justify your conclusion.
  \begin{list}{}
  \item For each  $( {s, t} ) \in \mathbb{Z} \times \mathbb{Z}$, there exists an $( {m, n}) \in \mathbb{Z} \times \mathbb{Z}$ such that  $g( {m, n} ) = ( {s, t} )$.
  \end{list}
\end{enumerate}


\item A \textbf{2 by 2 matrix over}
\index{matrix}%
  $\mathbb{R}$ is a rectangular array of four real numbers arranged in two rows and two columns.  We usually write this array inside brackets (or parentheses) as follows:
\[
A = \left[ {\begin{array}{*{20}c}
   a & b  \\
   c & d  \\

 \end{array} } \right],
\]
where  $a$, $b$, $c$, and  $d$  are real numbers.  The \textbf{determinant}
\index{determinant}%
\index{matrix!determinant}%
 of the 2 by 2 matrix  $A$, denoted by  $\det( A )$ 
\label{sym:determinant},  is defined as
\[
\det( A ) = ad - bc .
\]
\label{exer:determinant}
\begin{enumerate} 
\yitem Calculate the determinant of each of the following matrices: 
\[
\left[ {\begin{array}{*{20}c}
   3 & 5  \\
   4 & 1  \\

 \end{array} } \right],\;\left[ {\begin{array}{*{20}c}
   1 & 0  \\
   0 & 7  \\

 \end{array} } \right],\;\text{and }\left[ {\begin{array}{*{20}c}
   3 & { - 2}  \\
   5 & 0  \\

 \end{array} } \right].
\]

\item Let  $\mathcal{M}_2( \R )$ be the set of all  2 by 2  matrices over  
$\R$.  The mathematical process of finding the determinant of a 2 by 2 matrix over  $\R$ can be thought of as a function.  Explain carefully how to do so, including a clear statement of the domain and codomain of this function.
\end{enumerate}

\item Using the notation from Exercise~(\ref{exer:determinant}), let  
\[
A = \left[ {\begin{array}{*{20}c}
   a & b  \\
   c & d  \\

 \end{array} } \right]
\]
be a  2 by 2  matrix over  $\R$.  The \textbf{transpose of the matrix}
\index{transpose}%
\index{matrix!transpose}%
  $\boldsymbol{A}$, denoted by  $A^T $,
\label{sym:transpose} is the  2 by 2  matrix over  $\mathbb{R}$ defined by
\[
A^T  = \left[ {\begin{array}{*{20}c}
   a & c  \\
   b & d  \\

 \end{array} } \right].
\]
\label{exer:transpose}
\begin{enumerate}
  \item Calculate the transpose  of each of the following matrices:  

\[
\left[ {\begin{array}{*{20}c}
   3 & 5  \\
   4 & 1  \\

 \end{array} } \right],\;\left[ {\begin{array}{*{20}c}
   1 & 0  \\
   0 & 7  \\

 \end{array} } \right],\;\text{and }\left[ {\begin{array}{*{20}c}
   3 & { - 2}  \\
   5 & 0  \\

 \end{array} } \right].
\]

  \item Let  $\mathcal{M}_2( \R )$ be the set of all  2 by 2  matrices over  
$\mathbb{R}$.  The mathematical process of finding the transpose of a 2 by 2 matrix over  
$\mathbb{R}$ can be thought of as a function.  Carefully explain how to do so, including a clear statement of the domain and codomain of this function.

\end{enumerate}

%\item We often say that a function is a \textbf{piecewise defined function}
%\index{function!piecewise defined}%
%\index{piecewise defined function}%
%\label{exer:piecewisefunction}%
% if it has different rules for determining the output for different parts of its domain.  For example, we can define a function $f:\R \to \R$ by giving a rule for calculating $f(x)$ when $x \geq 0$ and giving a rule for calculating $f(x)$ when $x < 0$ as follows:
%\begin{equation}
%f(x) = 
%\begin{cases}
%x^2 + 1,    &  \text{if $x \geq 0$;} \\
%x - 1      &  \text{if $x < 0$.}
%\end{cases} \notag
%\end{equation}
%\begin{enumerate}
%\item Sketch a graph of the function $f$.
%\item Is the following statement true or false?  Justify your conclusion.
%\begin{list}{}
%\item For all $a, b \in \R$, if $a \ne b$, then $f(a) \ne f(b)$.
%\end{list}
%\item Is the following statement true or false?  Justify your conclusion.
%\begin{list}{}
%\item For all $y \in \R$, there exists an $x \in \R$ such that $f(x) = y$.
%\end{list}
%\end{enumerate}
%
%\item Define $g:[0, 1] \to (0, 1)$ by
%\begin{equation}
%g(x) = 
%\begin{cases}
%0.8,      &  \text{if $x = 0$;} \\
%0.5 x,    &  \text{if $0 < x < 1$;} \\
%0.6       &  \text{if $x = 1$.}
%\end{cases} \notag
%\end{equation}
%\begin{enumerate}
%\item Sketch a graph of the function $g$.
%\item Is the following statement true or false?  Justify your conclusion.
%\begin{list}{}
%\item For all $a, b \in [0, 1]$, if $a \ne b$, then $g(a) \ne g(b)$.
%\end{list}
%\item Is the following statement true or false?  Justify your conclusion.
%\begin{list}{}
%\item For all $y \in (0, 1)$, there exists an $x \in [0, 1]$ such that $g(x) = y$.
%\end{list}
%\end{enumerate}

\end{enumerate}

%\markboth{Chapter~\ref{C:functions}. Functions}{\ref{S:typesoffunctions}. Types of Functions}

%
\subsection*{Explorations and Activities}
%\newcounter{oldenumi}
\setcounter{oldenumi}{\theenumi}
\begin{enumerate} \setcounter{enumi}{\theoldenumi}
 \item \textbf{Integration as a Function}.  In calculus, we learned that if  $f$  is real function that is continuous on the closed interval  $\left[ {a, b} \right]$, then the definite integral  $\int_a^b {f( x ) \, dx} $ is a real number.  In fact, one form of the \textbf{Fundamental Theorem of Calculus}
\index{Fundamental Theorem!of Calculus}%
 states that
\[
\int_a^b {f( x ) \, dx} = F( b ) - F( a ),
\]
where  $F$  is any antiderivative of  $f$, that is, where  $F \,' = f$\!.
%
\begin{enumerate}
\item Let  $\left[ {a, b} \right]$ be a closed interval of real numbers and let  
$C {\left[ {a, b} \right]} $ be the set of all real functions that are continuous on  
$\left[ {a, b} \right]$.  That is,
\[
C{\left[ {a, b} \right]}  = \left\{ {f\x \left[ {a, b} \right] \to \mathbb{R}  \mid f\text{ is continuous on }\left[ {a, b} \right]} \right\}.
\]
%
\begin{enumerate}
\item Explain how the definite integral   $\int_a^b {f( x ) \, dx} $  can be used to define a function  $I$  from  $C{\left[ {a, b} \right]} $ to  $\mathbb{R}$.

\item Let  $\left[ {a, b} \right] = \left[ {0, 2} \right]$.  Calculate  $I( f )$,
  where  $f( x ) = x^2  + 1$.

\item Let  $\left[ {a, b} \right] = \left[ {0, 2} \right]$.  Calculate  $I( g )$,
  where  $g( x ) = \sin ( {\pi x} )$.
\end{enumerate}
%
\end{enumerate}
%
In calculus, we also learned how to determine the indefinite integral  
$\int {f( x )} \;dx$ of a continuous function  $f$.

\begin{enumerate}
\setcounter{enumii}{1}
\item Let  $f( x ) = x^2  + 1$ and $g(x) = \cos( 2x )$. Determine  
$\int {f( x )} \;dx$ and $\int {g( x )} \;dx$.

\item Let  $f$  be a continuous function on the closed interval  $\left[ {0, 1} \right]$
  and let  $T$  be the set of all real functions.  Can the process of determining the indefinite integral of a continuous function be used to define a function from $C{\left[ {0, 1} \right]}$ to  $T$?  Explain.

\item Another form of the Fundamental Theorem of Calculus states that if  $f$  is continuous on the interval  $\left[ {a, b} \right]$ and if  
\[
g( x ) = \int_a^x {f( t ) \, dt}
\]
for each  $x$  in  $\left[ {a, b} \right]$, then  $g'\left( x \right) = f\left( x \right)$.  That is,  $g$  is an antiderivative of  $f$.  Explain how this theorem can be used to define a function from $C{\left[ {a,\;b} \right]} $ to  $T$\!,   where the output of the function is an antiderivative of the input.  (Recall that  $T$  is the set of all real functions.)
\end{enumerate}
\end{enumerate}
\hbreak

\endinput
