\section*{Exercises \ref{S:finitesets}}

\begin{enumerate}
\item Prove that the function $g: A \cup \left\{ x \right\} \to \N_{k+1}$ in 
Lemma~\ref{L:addone} is a surjection. 
\label{exer:addonesurjection}%

\xitem Let $A$ be a subset of some universal set $U$.  Prove that if $x \in U$\!, then 
\linebreak $A \times \left\{ x \right\} \approx A$. 
\label{exer:sec92-1}%

\xitem Let $E^+$ be the set of all even natural numbers.  Prove that $\mathbb{N} \approx E^+$. \label{exer91:evennaturals}

\xitem  Prove Corollary~\ref{C:removeone}. 
\label{exer:sec92corollary}%

If $A$ is a finite set and $x \in A$, then $A - \left\{ x \right\}$ is a finite set and  \\
$\text{card} ( A - \left\{ x \right\} ) = \text{card} ( A ) - 1$.

\hint  One approach is to use the fact that $A = \left( A - \left\{ x \right\} \right) \cup \left\{x \right\}$.

\item Let $A$ and $B$ be sets.  Prove that 
\label{exer:sec92-finitesets}%
\begin{enumerate}
\yitem If $A$ is a finite set, then $A \cap B$ is a finite set.

\yitem If $A \cup B$ is a finite set, then $A$ and $B$ are finite sets.

\item If $A \cap B$ is an infinite set, then $A$ is an infinite set.

\item If $A$ is an infinite set or $B$ is an infinite set, then $A \cup B$ is an infinite set.
\end{enumerate}

\item There are over 7 million people living in New York City.  It is also known that the maximum number of hairs on a human head is less than 200,000.  Use the Pigeonhole Principle to prove that there are at least two people in the city of New York with the same number of hairs on their heads. 
\label{exer:sec92-pigeon}%


\item Prove the following propositions: 
\label{exer:sec92-7}%
\begin{enumerate}
\yitem If $A$, $B$, $C$, and $D$ are sets with $A \approx B$ and $C \approx D$,  then \linebreak
$A \times C \approx B \times D$. 
\label{exer:equivcartesian}%

\item If $A$, $B$, $C$, and $D$ are sets with $A \approx B$ and $C \approx D$ and if $A$ and $C$ are disjoint and $B$ and $D$ are disjoint, then $A \cup C \approx B \cup D$. 
\label{exer:equivunion}%
\end{enumerate}

\hint Since $A \approx B$ and $C \approx D$, there exist bijections $f:A \to B$ and $g:C \to D$.  To prove that $A \times C \approx B \times D$, prove that 
$h:A \times C \to B \times D$ is a bijection, where 
$h ( a, c ) = \left( f ( a ), g ( c ) \right)$, 
for all $( a, c ) \in A \times C$.

%If $A \cap C = \emptyset$ and $B \cap D = \emptyset$, then to prove that 
%$A \cup C \approx B \cup D$, prove that the following function is a bijection:  
%$k:A \cup C \to B \cup D$, where
%\[
%k ( x ) = \left\{ \begin{gathered}
%  f ( x ) \text{  if  }x \in A \hfill \\
%  g ( x ) \text{  if  }x \in C. \hfill \\ 
%\end{gathered}  \right.
%\]

\item Let  $A = \left\{ a, b, c \right\}$. 
\label{exer:sec927}%
\begin{enumerate}
\yitem Construct a function $f:\mathbb{N}_5 \to A$ such that $f$ is a surjection.

\item Use the function $f$ to construct a function $g:A \to \mathbb{N}_5$ so that \linebreak
$f \circ g = I_A$, where $I_A$ is the identity function on the set $A$.  Is the function $g$ an injection?  Explain.
\end{enumerate}

\item This exercise is a generalization of Exercise~(\ref{exer:sec927}). Let $m$ be a natural number, let $A$ be a set, and assume that $f:\mathbb{N}_m \to A$ is a surjection.  Define \linebreak
$g:A \to \mathbb{N}_m$ as follows: 
\label{exer:sec928}%

\begin{list}{}
\item For each $x \in A$, $g ( x ) = j$, where $j$ is the least natural number in 
$f^{-1} ( \left\{ x \right\} ) $.
\end{list}

Prove that $f \circ g = I_A$, where $I_A$ is the identity function on the set $A$, and prove that $g$ is an injection.

\item Let $B$ be a finite, nonempty set and assume that $f:B \to A$ is a surjection.  Prove that there exists a function $h:A \to B$ such that $f \circ h = I_A$ and $h$ is an injection.

\hint  Since $B$ is finite, there exists a natural number $m$ such that $\mathbb{N}_m \approx B$.  This means there exists a bijection $k:\mathbb{N}_m \to B$.  Now let $h = k \circ g$, where $g$ is the function constructed in Exercise~(\ref{exer:sec928}).

\end{enumerate}



\subsection*{Explorations and Activities}
\setcounter{oldenumi}{\theenumi}
\begin{enumerate} \setcounter{enumi}{\theoldenumi}
\item \textbf{Using the Pigeonhole Principle}.  \label{A:usingpigeon} 
\index{Pigeonhole Principle}%
For this activity, we will consider subsets of $\mathbb{N}_{30}$ that contain eight elements.  

\begin{enumerate}
\item One such set is $A = \left\{ 3, 5, 11, 17, 21, 24, 26, 29 \right\}$.  Notice that
\begin{align}
\left\{3, 21, 24, 26 \right\} &\subseteq A \quad \text{ and }   &3 + 21 + 24 + 26 &= 74 \notag \\
\left\{3, 5, 11, 26, 29 \right\} &\subseteq A \quad \text{ and }   &3 + 5 + 11 + 26 + 29 &= 74. \notag
\end{align}
Use this information to find two disjoint subsets of $A$ whose elements have the same sum.
\label{A:usingpigeon1}

\item Let $B = \left\{ 3, 6, 9, 12, 15, 18, 21, 24 \right\}$.  Find two disjoint subsets of $B$ whose elements have the same sum.  \note  By convention, if $T = \left\{ a \right\}$, where $a \in \mathbb{N}$, then the sum of the elements in $T$ is equal to $a$.

\item Now let $C$ be any subset of $\mathbb{N}_{30}$ that contains eight elements.
\begin{enumerate}
\item How many subsets does $C$ have?
\item The sum of the elements of the empty set is 0.  What is the maximum sum for any subset of 
$\mathbb{N}_{30}$ that contains eight elements?  Let $M$ be this maximum sum.

%\hint  Make the elements of a subset of $\mathbb{N}_{30}$ with 8 elements as large as possible.

\item Now define a function $f\x \mathcal{P} ( C ) \to \mathbb{N}_M$ so that for each 
$X \in \mathcal{P} ( C )$, $f ( X )$ is equal to the sum of the elements in $X$.

Use the Pigeonhole Principle to prove that there exist two subsets of $C$ whose elements have the same sum. \label{A:usingpigeon3c}
\end{enumerate}

\item If the two subsets in part~(\ref{A:usingpigeon3c}) are not disjoint, use the idea presented in part~(\ref{A:usingpigeon1}) to prove that there exist two disjoint subsets of $C$ whose elements have the same sum.


\item Let $S$ be a subset of $\mathbb{N}_{99}$ that contains 10 elements.  Use the Pigeonhole Principle to prove that there exist two disjoint subsets of $S$ whose elements have the same sum.
\label{exer:sec92-pigeon2}%
\end{enumerate}

\end{enumerate}


\hbreak

\endinput
