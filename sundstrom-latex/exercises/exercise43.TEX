\section*{Exercises for Section~\ref{S:recursion}}
%
\begin{enumerate}
\xitem For the sequence   $a_{0}, a_1 ,a_2 , \ldots , a_n , \ldots $ , assume that  $a_0  = 1$
and that for each  $n \in \mathbb{N} \cup \left\{ 0 \right\}$,  $a_{n + 1}  = \left( {n + 1} \right)a_n $.  Use mathematical induction to prove that  for each  $n \in \mathbb{N} \cup \left\{ 0 \right\}$,  $a_n  = n!$.
\label{exer:sec51-factorial}%



\item Assume that   $f_1 ,f_2 , \ldots ,f_n , \ldots $ are the Fibonacci numbers.  Prove each of the following: \label{exer:sec53-fib}

\begin{enumerate}
  \yitem For each  $n \in \mathbb{N}$,  $f_{4n} $  is a multiple of  3.

  \item For each  $n \in \mathbb{N}$,  $f_{5n} $  is a multiple of  5.

  \yitem For each  $n \in \mathbb{N}$ with  $n \geq 2$,  $f_1  + f_2  +  \cdots  + f_{n - 1}  = 
                   f_{n + 1}  - 1$.

  \item For each  $n \in \mathbb{N}$,  $f_1  + f_3  +  \cdots  + f_{2n - 1}  = f_{2n} $.

  \item For each  $n \in \mathbb{N}$,  $f_2  + f_4  +  \cdots  + f_{2n}  = f_{2n + 1}  - 1$.

  \yitem For each  $n \in \mathbb{N}$,  $f_1^2  + f_2^2  +  \cdots  + f_n^2  = f_n f_{n + 1} $.

  \item For each  $n \in \mathbb{N}$ such that  $n\not  \equiv 0 \pmod 3$,  $f_n $ is an odd integer.
\end{enumerate}


\item Use the result in Part~(f) of Exercise~(\ref{exer:sec53-fib}) to prove that
\[
\frac{f_1^2  + f_2^2  +  \cdots  + f_n^2 + f_{n+1}^2}{f_1^2  + f_2^2  +  \cdots  + f_n^2} 
     = 1 + \frac{f_{n+1}}{f_n}.
\]


\item \label{exer:binet} The quadratic formula can be used to show that  $\alpha = \dfrac{1 + \sqrt{5}}{2}$ and 
\mbox{$\beta = \dfrac{1 - \sqrt{5}}{2}$} are the two real number solutions of the quadratic equation 
$x^2 - x - 1 = 0$.  Notice that this implies that
\begin{align*}
\alpha^2 &= \alpha + 1, \text{ and} \\
\beta^2 &= \beta + 1.
\end{align*}
It may be surprising to find out that these two irrational numbers are closely related to the Fibonacci numbers.  
\begin{enumerate}
  \item Verify that $f_1 = \dfrac{\alpha^1 - \beta^1}{\alpha - \beta}$ and that 
$f_2 = \dfrac{\alpha^2 - \beta^2}{\alpha - \beta}$.

  \item (This part is optional, but it may help with the induction proof in part~(c).)  Work with the relation $f_3 = f_2 + f_1$ and substitute the expressions for $f_1$ and $f_2$ from part~(a).  Rewrite the expression as a single fraction and then in the numerator use $\alpha^2 + \alpha = \alpha ( \alpha + 1 )$ and a similar equation involving $\beta$.  Now prove that $f_3 = \dfrac{\alpha^3 - \beta^3}{\alpha - \beta}$.

  \item Use induction to prove that for each natural number $n$, if $\alpha = \dfrac{1 + \sqrt{5}}{2}$ and 
$\beta = \dfrac{1 - \sqrt{5}}{2}$, then $f_n = \dfrac{\alpha^n - \beta^n}{\alpha - \beta}$.  \note This formula for the 
$n^{th}$ Fibonacci number is known as Binet's formula, 
\index{Binet's formula}%
named after the French mathematician Jacques Binet (1786 -- 1856).
\end{enumerate}

\item Is the following conjecture true or false?  Justify your conclusion.

\newpar
\textbf{Conjecture}.  Let $f_1, f_2, \ldots, f_m, \ldots$ be the sequence of the Fibonacci numbers.  For each natural number $n$, the numbers $f_nf_{n+3}$, $2f_{n+1}f_{n+2}$, and 
$\left( f_{n+1}^2 + f_{n+2}^2 \right)$ form a Pythagorean triple.


\xitem Prove Proposition~\ref{P:geometricsequence}.
\label{exer:geomseq}%
Let  $a,r \in \mathbb{R}$.  If a geometric sequence is defined by  $a_1  = a$ and for each  $n \in \mathbb{N}$,  $a_{n + 1}  = r \cdot a_n $, then for each  $n \in \mathbb{N}$,   $a_n  = a \cdot r^{n - 1} $. 

\item Prove Proposition~\ref{P:geometricseries}. 
\label{exer:geomser}%
Let  $a,r \in \mathbb{R}$.  If the sequence  $S_1 ,S_2 , \ldots ,S_n , \ldots $ is defined by  $S_1  = a$ and for each  $n \in \mathbb{N}$,  $S_{n + 1}  = a + r \cdot S_n $, then for each  $n \in \mathbb{N}$,   $S_n  = a + a \cdot r + a \cdot r^2  +  \cdots  + a \cdot r^{n - 1} $.  That is, the geometric series  $S_n $ is the sum of the first  $n$  terms of the corresponding geometric sequence.

\xitem Prove Proposition~\ref{P:geometricseries2}.
\label{exer:geometricseries2}%
Let  $a, r \in \mathbb{R}$ and  $r \ne 1$.  If the sequence  $S_1 ,S_2 , \ldots ,S_n , \ldots $ is defined by  $S_1  = a$ and for each  $n \in \mathbb{N}$,  $S_{n + 1}  = a + r \cdot S_n $, then for each  $n \in \mathbb{N}$,   $S_n  = a\left( {\dfrac{{1 - r^n }}{{1 - r}}} \right)$.


\item For the sequence $a_1 ,a_2 , \ldots ,a_n , \ldots $ , assume that  $a_1  = 2$ and that for each  $n \in \mathbb{N}$,  $a_{n + 1}  = a_n + 5$. \label{exer:sec53-arithexample}
\begin{enumerate}
  \yitem Calculate  $a_2 $  through  $a_6$.

  \yitem Make a conjecture for a formula for  $a_n $ for each  $n \in \mathbb{N}$. 
\label{exer:sec53-arithexampleb}

  \item Prove that your conjecture in Exercise~(\ref{exer:sec53-arithexampleb}) is correct.
\end{enumerate}



\item \label{exer:sec53-arith}The sequence in Exercise~(\ref{exer:sec53-arithexample}) is an example of an 
\textbf{arithmetic sequence}.  
\index{arithmetic sequence}%
\index{sequence!arithmetic}%
An arithmetic sequence is defined recursively as follows:

Let $c$ and $d$ be real numbers.  Define the sequence $a_1 ,a_2 , \ldots ,a_n , \ldots $ by  $a_1  = c$ and for each  $n \in \mathbb{N}$,  $a_{n + 1}  = a_n + d$.
\begin{enumerate}
  \item Determine formulas for  $a_3 $  through  $a_8 $.

  \item Make a conjecture for a formula for  $a_n$ for each  $n \in \mathbb{N}$. \label{exer:sec53-arithb}

  \item Prove that your conjecture in Exercise~(\ref{exer:sec53-arithb}) is correct.
\end{enumerate}


\item For the sequence   $a_1 ,a_2 , \ldots ,a_n , \ldots $ , assume that  $a_1  = 1$, $a_2 = 5$, and that for each  $n \in \mathbb{N}$ with $n\geq 2$,  $a_{n + 1}  = a_n + 2a_{n-1}$.  Prove that for each natural number $n$, 
$a_n = 2^n + (-1)^n$.


\xitem For the sequence   $a_1 ,a_2 , \ldots ,a_n , \ldots $ , assume that  $a_1  = 1$ and that for each  $n \in \mathbb{N}$,  $a_{n + 1}  = \sqrt {5 + a_n } $. \label{exer:sec53-5}

\begin{enumerate}
  \item Calculate, or approximate,  $a_2 $  through  $a_6 $.

  \item Prove that for each  $n \in \mathbb{N}$,  $a_n  < 3$.
\end{enumerate}

\item For the sequence   $a_1 ,a_2 , \ldots ,a_n , \ldots $ , assume that  $a_1  = 1$,  $a_2  = 3$, and that for each  $n \in \mathbb{N}$,  $a_{n + 2}  = 3a_{n + 1}  - 2a_n $. \label{exer:sec53-6}

\begin{enumerate}
  \yitem Calculate  $a_3 $  through  $a_6 $.

  \yitem Make a conjecture for a formula for  $a_n $ for each  $n \in \mathbb{N}$. \label{exer:536b}

  \item Prove that your conjecture in Exercise~(\ref{exer:536b}) is correct.
\end{enumerate}

\item For the sequence   $a_1 ,a_2 , \ldots ,a_n , \ldots $ , assume that  $a_1  = 1$,  $a_2  = 1$, and that for each  $n \in \mathbb{N}$,  $a_{n + 2}  = \dfrac{1}{2}\left( {a_{n + 1}  + \dfrac{2}{{a_n }}} \right)$. \label{exer:sec53-7}

\begin{enumerate}
  \yitem Calculate  $a_3 $ through  $a_6 $.

  \item Prove that for each  $n \in \mathbb{N}$,  $1 \leq a_n  \leq 2$.
\end{enumerate}

\item For the sequence   $a_1 ,a_2 , \ldots ,a_n , \ldots $, assume that $a_1=1$, $a_2=1$, $a_3=1$, and for that each natural number $n$, 
\label{exer:sec53-8}%
\[
a_{n+3} = a_{n+2} + a_{n+1} + a_n.
\]
\begin{enumerate}
\item Compute $a_4$, $a_5$, $a_6$, and $a_7$.

\item Prove that for each natural number $n$ with $n>1$, $a_n \leq 2^{n-2}$.
\end{enumerate}

\item For the sequence $a_1, a_2, \ldots, a_n, \ldots$, assume that $a_1 = 1$, and that for each natural number $n$,
\[
a_{n + 1} = a_n + n \cdot n!.
\]
\begin{enumerate}\label{exer:sec53-9}
\item Compute $n!$ for the first 10 natural numbers.

\yitem Compute $a_n$ for the first 10 natural numbers.

\item Make a conjecture about a formula for $a_n$ in terms of $n$ that does not involve a summation or a recursion.

\item Prove your conjecture in Part~(c).
\end{enumerate}


\item For the sequence $a_1, a_2, \ldots, a_n, \ldots$ , assume that $a_1 = 1$, $a_2 = 1$, and for each $n \in \N$, $a_{n+2} = a_{n+1} + 3a_n$.  Determine which terms in this sequence are divisible by 4 and prove that your answer is correct.


\item The \textbf{Lucas numbers} \label{exer:lucasnumbers}
\index{Lucas numbers}%
 are a sequence of natural numbers 
$L_1, L_2, L_3, \ldots, L_n, \ldots ,$ which are defined recursively as follows:

\begin{itemize}
\item $L_1 = 1$ and $L_2 = 3$, and
\item For each natural number $n$, $L_{n+2} = L_{n+1} + L_n$.
\end{itemize}

List the first 10 Lucas numbers and the first ten Fibonacci numbers and then prove each of the following propositions.  The Second Principle of Mathematical Induction may be needed to prove some of these propositions.

\begin{enumerate}
\yitem For each natural number $n$, $L_n = 2f_{n+1} - f_n$.

\item For each  $n \in \mathbb{N}$ with  $n \geq 2$,  $5f_n  = L_{n - 1}  + L_{n + 1} $.

\item For each  $n \in \mathbb{N}$ with  $n \geq 3$,  $L_n  = f_{n + 2}  - f_{n - 2} $.
\end{enumerate}



\item \label{exer:lucasnumbers2} There is a formula for the Lucas numbers similar to the formula for the Fibonacci numbers in 
Exercise~(\ref{exer:binet}).  Let $\alpha = \dfrac{1 + \sqrt{5}}{2}$ and 
\mbox{$\beta = \dfrac{1 - \sqrt{5}}{2}$}.  Prove that for each $n \in \N$, $L_n = \alpha^n + \beta^n$.



\item Use the result in Exercise~(\ref{exer:lucasnumbers2}), previously proven results from Exercise~(\ref{exer:lucasnumbers}), or mathematical induction to prove each of the following results about Lucas numbers and Fibonacci numbers.
\begin{enumerate}
\item For each  $n \in \mathbb{N}$,  $L_n  = \dfrac{{f_{2n} }}{{f_n }}$.

%\item For each  $n \in \mathbb{N}$,  $f_{2n}  = f_n L_n$.

\item For each  $n \in \mathbb{N}$,  $f_{n+1}  = \dfrac{f_n + L_n}{2}$.

\item For each  $n \in \mathbb{N}$,  $L_{n+1} = \dfrac{L_n +5f_n}{2}$.

\item For each  $n \in \mathbb{N}$ with $n \geq 2$,  $L_n = f_{n+1} + f_{n-1}$.

%\item For each  $n \in \mathbb{N}$ with $n \geq 2$, $f_n = \dfrac{L_{n+1} +L_{n-1}}{5}$.  
\end{enumerate}


\item \textbf{Evaluation of proofs}  \hfill \\
See the instructions for Exercise~(\ref{exer:proofeval}) on 
page~\pageref{exer:proofeval} from Section~\ref{S:directproof}.

%\begin{enumerate} 
\noindent
\textbf{Proposition}. Let $f_n$ be the $n^\text{th}$ Fibonacci number, and let $\alpha$ be the positive solution of the equation $x^2 = x + 1$.  So $\alpha = \dfrac{1 + \sqrt{5}}{2}$.  For each natural number 
$n$, $f_n \leq \alpha^{n-1}$.

\begin{myproof}
We will use a proof by mathematical induction.  For each natural number $n$, we let 
$P( n )$ be, ``$f_n \leq \alpha^{n-1}$.''

We first note that $P( 1 )$ is true since $f_1 = 1$ and $\alpha^0 = 1$.  We also notice that 
$P( 2 )$ is true since $f_2 = 1$ and, hence, $f_2 \leq \alpha^1$.

We now let $k$ be a natural number with $k \geq 2$ and assume that $P ( 1 )$, 
$P( 2 )$, \ldots , $P( k )$ are all true.  We now need to prove that 
$P \left( k + 1 \right)$ is true or that $f_{k+1} \leq \alpha^k$.

Since $P( k -1 )$ and $P ( k )$ are true, we know that 
$f_{k-1} \leq \alpha^{k-2}$ and $f_k \leq \alpha^{k-1}$.  Therefore,
\[
\begin{aligned}
f_{k + 1} &= f_k + f_{k-1} \\
f_{k + 1} &\leq \alpha^{k-1} + \alpha^{k-2} \\
f_{k + 1} &\leq \alpha^{k-2} \left(\alpha + 1 \right)\!.
\end{aligned}
\]
We now use the fact that $\alpha + 1 = \alpha^2$ and the preceding inequality to obtain
\[
\begin{aligned}
f_{k+1} &\leq \alpha^{k-2} \alpha^2 \\
f_{k+1} &\leq \alpha^k.
\end{aligned}
\]
This proves that if $P( 1 )$, $P( 2 )$, \ldots , $P( k )$ are true, then $P(k + 1 )$ is true.  Hence, by the Second Principle of Mathematical Induction, we conclude that for each natural number $n$, $f_n \leq \alpha^{n-1}$\!.
\end{myproof}
%\end{enumerate}
\end{enumerate}



\subsection*{Explorations and Activities}
\setcounter{oldenumi}{\theenumi}
\begin{enumerate} \setcounter{enumi}{\theoldenumi}
  \item \textbf{Compound Interest}.    \label{exer:compoundinterest}
\index{compound interest}%
Assume that  $R$  dollars is deposited in an account that has an interest rate of  $i$  for each compounding period.  A compounding period is some specified time period such as a month or a year.

For each integer  $n$  with  $n \geq 0$, let  $V_n $ be the amount of money in an account at the end of the  $n${th}  compounding period.  Then  
\[
\begin{aligned}[t]
V_1  &= R + i \cdot R \\ 
     &= R\left( {1 + i} \right) \\ 
\end{aligned} 
\text{\qquad \qquad}
\begin{aligned}[t]
V_2  &= V_1  + i \cdot V_1  \\ 
     &= \left( {1 + i} \right)V_1  \\ 
     &= \left( {1 + i} \right)^2 R. \\ 
\end{aligned} 
\]
\begin{enumerate}
\item Explain why  $V_3  = V_2  + i \cdot V_2 $.  Then use the formula for  $V_2 $ to determine a formula for  $V_3 $ in terms of  $i$  and   $R$.

\item Determine a recurrence relation for  $V_{n + 1} $ in terms of  $i$  and  $V_n $. \label{A:compoundinterest2}

\item Write the recurrence relation in Part~(\ref{A:compoundinterest2}) so that it is in the form of a recurrence relation for a geometric sequence.  What is the initial term of the geometric sequence and what is the common ratio?

\item Use Proposition~\ref{P:geometricsequence} to determine a formula for  $V_n $ in terms of  $I$, $R$, and  $n$.
\end{enumerate}



\item \textbf{The Future Value of an Ordinary Annuity}.   \label{exer:annuity} 
For an \textbf{ordinary annuity},
\index{ordinary annuity}%
  $R$  dollars is deposited in an account at the end of each compounding period.  It is assumed that the interest rate, $i$, per compounding period for the account remains constant.

Let  $S_t $ represent the amount in the account at the end of the  $t${th} compounding period. $S_t $ is frequently called the \textbf{future value}
\index{future value}%
 of the ordinary annuity.

So  $S_1  = R$.  To determine the amount after two months, we first note that the amount after one month will gain interest and grow to  $( {1 + i} )S_1 $.  In addition, a new deposit of  $R$  dollars will be made at the end of the second month.  So
\[
S_2  = R + ( {1 + i} )S_1.
\]
\begin{enumerate}
\item For each  $n \in \mathbb{N}$, use a similar argument to determine a recurrence relation for  $S_{n + 1} $ in terms of  $R$, $i$, and  $S_n $.

\item By recognizing this as a recursion formula for a geometric series, use Proposition~\ref{P:geometricseries2} to determine a formula for  $S_n $ in terms of  $R$, $i$, and  $n$  that does not use a summation.  Then show that this formula can be written as  
\[
S_n  = R\left( {\frac{{\left( {1 + i} \right)^n  - 1}}{i}} \right).
\]
\item What is the future value of an ordinary annuity in 20 years if \$200 dollars is deposited in an account at the end of each month where the interest rate for the account is 6\%  per year compounded monthly?  What is the amount of interest that has accumulated in this account during the 20 years?

\end{enumerate}
\end{enumerate}
\hbreak
\endinput
