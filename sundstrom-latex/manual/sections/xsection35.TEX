\section*{Section~\ref{S:divalgo} The Division Algorithm and Congruence}
Plan at least one and one-half class periods for this section.  


\subsection*{Main Topics}
The Division Algorithm, using the Division Algorithm to define cases, congruence and the relation between congruence and the Division Algorithm.  

Many of the exercises in this section can take students a long time to complete, especially if they work with the Division Algorithm rather than congruence arithmetic.  I try make sure that students undersand the importance of Theorem~\ref{T:propsofcong} and how it can be used in proofs.  Proposition~\ref{P:3dividesver2} is meant to illustrate this but I also another example in class.  This example could be done as a guided activty with something like Exercises~(\ref{exer:squaremod5}) or~(\ref{exer:remainderbycong}).  For example, the following question could be posed to the class:

\begin{list}{}
\item If an integer has a remainder of 7 when divided by 12, is it possible to make any conclusion about the remainder of the cube of that integer when it is divided by 12?
\end{list}

\vskip6pt
Some students will need guidance to start with something like, ``Let $n$ be an integer and assume that $n \equiv 7 \pmod {12}$.''  Quite a bit of algebra is needed if a student uses the fact that there exists an integer $k$ such that $n = 7 + 12k$ and then tries to find $n^3$.  Instead, try to guide the students to conclude that $n^3 \equiv 7^3 \pmod {12}$.  Then they can calculate $7^3 = 343$ and determine that $343 \equiv 1 \pmod {12}$ and conclude that 
$n^2 \equiv 7 \pmod {12}$.  This means that $n^3$ has a remainder of 7 when divided by 12.

\subsection*{The Preview Activities}
\subsubsection*{Preview Activity~\ref{PA:quotients} (Quotients and Remainders)} 
This preview activity serves as an introduction to the Division Algorithm.  It is intended to make sure that students will be careful when using the terms ``quotient'' and ``remainder.''  This will be especially true when negative integers are involved.

\subsubsection*{Preview Activity~\ref{PA:congruencereview} (A Review of Congruence)} 
Since congruence is still a relatively new concept to most students, I think this review is necesary before beginning this section.
\hbreak


\subsection*{Activity~\ref{A:lasttwo} (The Last Two Digits of a Large Integer)}
Although not really mathematically significant, students seem to enjoy this activity.  It does provide them with good practice with congruences.
\hbreak

\subsection*{The Exercises}
Exercises~(\ref{exer:congto3}), 
(\ref{exer:3divprod}), and~(\ref{exer:sqrt3}) are good exercises to assign together, as are 
Exercises~(\ref{exer:squaremod5}) and~(\ref{exer:sqrt5-irrational}). 
Exercise~(\ref{exer:falsecongruence}) can frustrate some students since all parts contain a false statement.  The one in Part~(d) takes some exploration to find a counterexample.  The student has to take effort to get to $a = 5$ and $b = 11$ or be willing to use negative integers such as $a = -1$ and $b = -7$.  Exercise~(\ref{exer:sec34-9}) is a good exercise that combines the use of cases with the method of proof by contradiction.

\vskip6pt
\noindent
Typical Assignment:  Exercises 2, 3, 5, 6, 7, 11, 12, 15 or 16, two parts of 17
\hbreak

\endinput


Plan at most one class periods for this section.  Be careful not to have the students get too ``bogged down'' in the algebra in this section.  Some may find the algebra overwhelming.  The important idea is the distinction between constructive proofs and non-constructive proofs.  I have at times omitted this section and discussed the distinction between constructive and non-constructive proofs at appropriate times when studying other parts of the text.

\subsection*{Main Topics}
Constructive proofs of existence theorems and non-constructive proofs of existence theorems.  Be careful not to have the students get too ``bogged down'' in the algebra in this section.  Some may find the algebra overwhelming.  The important idea is the distinction between constructive proofs and non-constructive proofs.

\subsection*{The Preview Activities}
\subsubsection*{Preview Activity~\ref{PA:linearsystems}} 
This preview activity reviews the methods of finding the solution of a system of two linear equations in two unknowns.  One system has a unique solution, one has infinitely many solutions, and one is inconsistent.

\subsubsection*{Preview Activity~\ref{PA:circles}}  
This preview activity reviews the standard equatin for a circle and asks them to solve a system of two equations in two unknowns to find the points of intersection of two circles.
\hbreak

\subsection*{The Activities}
There are two activities in this section.

\subsubsection*{Activity~\ref{A:linearsystem}}
This activity is actually part of the proof of Theorem~\ref{T:linearsystem}.  Students at this level should be able to handle the algebra involved in this proof, but do not let them get too frustrated.  If necessary, give them a short time to get started and then complete the proof in class.

\subsubsection*{Activity~\ref{A:circles}}
This activity is quite likely too involved for an in-class activity.  It will take the students most (if not all) of a class period to complete.  It might be best to skip this activity or assign it as a outside of class activity.
\hbreak

\subsection*{The Exercises}
Four or five of these exercises should be sufficient.  Exercise~(\ref{exer:sec35-4}) provides another opportunity to use cases in a proof.  Exercise~(\ref{exer:sec35-solution}) provides a method to use the concept of divides to actually prove there is no such integer in Part~(b).  Exercise~(\ref{exer:sec35-10}) is fairly difficult.  One method to prove this is to use an idea similar to the one used in Exercise~(\ref{exer:sec35-solution}).  I often use Exercise~(\ref{exer:sec35-10}) as part of an assignment.

\vskip6pt
\noindent
Typical Assignment:  Exercises 1, 4, 5, 6, 8

\hbreak
