\section*{Section~\ref{S:directproof} Direct Proofs}
Plan about one and one-half class periods for this section.  It may be a good idea to plan on three class days for this section and Section~\ref{S:moremethods}.  Two important relations are introduced in this section:  divides and congruence.  These relations are used throughout the text to help develop students abilities to construct and write proofs.


\subsection*{Main Topics}
Mathematical terminology -- proof, undefined terms, axiom, definition, conjecture, theorem, corollary, and lemma; divides, divisor, ``know-show table'', congruence and properties of congruence.

When dealing with congruence, I try to make sure students understand that the following are equivalent:
\begin{itemize}
\item $a \equiv b \pmod n$

\item $n$ divides $(a - b)$.

\item There exists an integer $k$ such that $a - b = nk$.

\item There exists an integer $k$ such that $a = b + nk$.
\end{itemize}
The last item in the list can be used by students to quickly generate examples of integers that are congruent modulo $n$ to a given integer by starting with that integer repeatedly adding a subtracting $n$.  For example, some integers that are congruent to 7 modulo 12 are
\[
\ldots, -17, -5, 7, 19, 31, 43, \ldots .
\]

\subsection*{The Preview Activities}
\subsubsection*{Preview Activity~\ref{PA:divisor} (Definition of Divides, Divisor, Multiple)} 
This is an important preview activity.  It introduces students to the formal definitions of divides and divisor and provides them with an opportuntity to work with and understand these definitions before they start trying to construct proofs.  It may be a good idea to tell them to make sure they answer the questions using the definitions.  Some will not do this and answer questions with something like, ``You cannot divide by zero.''  This is why some of the questions begin with the phrase, ``According to the definition, \ldots.''

It is also important to discuss the conjecture between questions (6) and (7) and how this conjecture was created based on the work in question (6).  It is important to formulate conjectures precisely, and students have had little or no practice at doing this.  It must emphasized that they should formulate their conjecture in the form of a statement, and you should discuss with the class the importance of using the universal quantifier with this conditional statement.

An important point to make is that the conjecture needs to be a proposition or statement that can be proven to be true or false.  So if the conjecture is stated as ``If $a$ divides $b$ and $b$ divides $c$, then $a$ divides $c$,'' then it is not stated in the form of a statement.  A quantier needs to be used. So, this conjecture should be stated something like the following:

\begin{center}
For all integers $a$, $b$, and  $c$, if $a$ divides $b$ and $b$ divides $c$, then $a$ divides $c$.
\end{center}

The answers for the preview activities indicate that sometimes, to vary the way we write things, we do not explicitly use the quantifiers.  It is also possible to write the following:

\vskip6pt
Let $a$, $b$, and  $c$  be integers.  If $a$ divides $b$ and $b$ divides $c$, then $a$ divides 
$c$.
\vskip6pt

Point out that using the word ``let'' is a method that mathematicians accept as equivalent to saying it is true for all integers.  The idea is the we can let $a$, $b$, and  $c$  be any integers.

Students will be asked to formulate conjectures in this and other courses.  The message here is to be very careful in how they formulate conjectures, and that they should be formulated as a statement or proposition that can be proven true or false.



%\subsubsection*{Preview Activity~\ref{PA:evenprop} (A Proposition about Multiples of 3)}  
%In this preview activity, students will review some of the ideas presented in Section~\ref{S:direct}.  In particular, they will need to recall how to construct a know-show table.  They will construct a know-show table about a proposition dealing with multiples of 3.  It might be a good idea to point out that the ideas used in the proof are similar to the ideas used to prove propositions about even integers since an even integer is simply a multiple of 2.

 

\subsubsection*{Preview Activity~\ref{PA:calender} (Calendars and Clocks)}
This preview activity is an introduction to the concept of congruence.  The idea is to have students work with things that repeat on a regular basis (time and the day of the week).  In addition, students are asked to generate a list of numbers and observe that the difference of any two numbers in the list will be a multiple of a given number.  Hopefully, this will help make the definition of congruence make sense.
\hbreak


%\subsection*{Activity~\ref{A:mod6} (Congruence Modulo 6)}
%A good deal of the work is done for the students in this activity.  The reason is that this will be the first time they will have worked with the concept of congruence.  If you prefer to give them less direction, use a different but similar activity.  One suggestion is the following:
%\begin{enumerate}
%\item Find several integers that are congruent to 3 modulo 7 and then square each of these integers.
%\item For each integer $m$, determine an integer $k$ such that $0 \leq k < 7$ and 
%$m^2 \equiv k \pmod 7$.  What do you observe?
%\item Based on your work in Parts~(1) and~(2), formulate a conjecture about the value of $m^2$ modulo 7 when $m \equiv 3 \pmod 7$.  Your conjecture should start with the sentence, ``Let $m$ be an integer.'' and should be completed with a conditional statement.
%\item Construct a Know-Show-Table for a proof of the conjecture in Part~(3).
%\end{enumerate}
%\hbreak
\endinput

\subsection*{The Exercises}

Assign at least four of the exercises in Exercise~(1).    Exercise~(1e) should be included since it is false and many students will think it is true without taking the time to explore some examples.  In addition, make sure that at least two exercises dealing with congruence are included.  You may choose to present Exercise~(\ref{exer:cong-symm}) in class.  At least one part of Exercise~(\ref{exer:sec31-11}) should be included.  Part~(a) is fairly straightforward, but Part~(b) will cause some difficulties.  Exercise~(\ref{exer:sec31-11}) is included in the text as Theorem~\ref{T:propsofcong} in Section~\ref{S:cases}.  Exercises~(\ref{exer:circle-31}) through~(\ref{exer:sec31-pythag}) provide practice with direct proofs using concepts of precalculus and calculus.  Exercise~(\ref{exer:sec31-pythag}) makes a good use of the Pythagorean Theorem.

\vskip6pt
\noindent
Typical Assignment:  Exericises 1(b, d, e, g, h), 4, 5, 6, 9(a), one of 10 through 14.
\hbreak


\subsection*{Explorations and Activities}
