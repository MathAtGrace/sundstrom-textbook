\section*{Section~\ref{S:predicates} Predicates, Sets, and Quantifiers}
Plan about one class period for this section.

\subsection*{Main Topics}
Basic set notation including set-builder notation, variables and predicates, truth set of a predicate, and statements involving quantifiers.

\subsection*{The Preview Activities}
\subsubsection*{Preview Activity~\ref{PA:sentences} (Sentences that Are Not Statements)} 
The purpose of this preview activity is simply to introduce students to the fact that not every mathematical sentence is a statement.  They may have some difficulty answering Problem~(2), but that is to be expected here.  It can provide some incentive for classroom discussion.

\subsubsection*{Preview Activity~\ref{PA:variables} (Variables)}  
The concept of a \textbf{universal set} is introduced.  The purpose of the five problems is to provide a ``lead-in'' to the concept of a \textbf{truth set} introduced in the section.  This is also done in Progress Check~\ref{pr:predicates}.
\hbreak

\subsection*{Activity~\ref{A:closure-explore} (Closure Explorations)}
This activiy is intended to give students a better understanding of closure for a set with respect to an operation.  It is provides an opportunity to work with a universally quatified conditional statement and counterexamples for such statements.
\hbreak
%\subsubsection*{Activity~\ref{A:predicates}} This comes before the definition of a truth set. If students have read this section before class or if Preview Activity~\ref{PA:variables} seems to provide a sufficient introduction, then this activity can be skipped.

%\subsubsection*{Activity~\ref{A:truthset}}  This comes right after the definition of truth set and provides practice with working with this definition.  If students do not work on this activity, then it should be used to provide examples in class.  (Or the instructor can use other examples.
%\hbreak
%
\subsection*{The Exercises}

It is a good idea to assign most of the exercises in this section.  Exercise~(\ref{exer:sec21-3}) and~(\ref{exer:sec23-sets}) are needed to provide practice in using set-builder notation.  Exercise~(\ref{exer:sec21-4}) is a good exercise since it makes the students distinguish between statements and predicates.  Exercise~(\ref{Exer:quantifier}) is also good for making the distinction between statements and predicates.

\vskip6pt
\noindent
Typical Assignment:  Exercises 1(a, b, d, e), 2(a, b), 3, 4, 5, 6, 7
\hbreak
\endinput
