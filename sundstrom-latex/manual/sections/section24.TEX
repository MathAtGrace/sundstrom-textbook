\section*{Section~\ref{S:quantifier} Quantifiers and Negations}
Plan at least one class period for this section.  Some students may have difficulty with forming appropriate negations of quantified statement, and so it may be necessary to spend about one and one-half periods on this section.


\subsection*{Main Topics}
Quantifiers, negations of quantified statements, counterexamples, statements with more than one quantifier.

\subsection*{The Preview Activities}
\subsubsection*{Preview Activity~\ref{PA:quantifier} (Quantifiers)} 
The purpose of this preview activity is to introduce students to the use of quantifiers in mathematics.  The problems in this preview activity are intended to help students recognize the difference between an open sentence and a statement in which the variables are quantified.

%

\subsubsection*{Preview Activity~\ref{PA:negatequantifier} (Attempting to Negate Quantified Statements)}  
The purpose of this preview activity is to have the students try to form negations of statements that contain a quantifier.  One statement uses a universal quantifier and the other uses an existential quantifier.  Perfect answers are not expected on this preview activity, but having the students make the attempt will foster a good classroom discussion.    This is an important preview activity and section as students will need the ability to quickly and accurately write negations in their study of mathematics.
\hbreak




\subsection*{The Exercises}

Assign the first five exercises or at least substantial parts of each of these exercises.  Exercises~(\ref{exer:24-increasing}) and~(\ref{exer:24-continuous}) deal with the formal definitions of some concepts from calculus.  If calculus is a prerequisite for your course, it is a good idea to assign at least one of these exercises.  
Exercise~(\ref{exer:advanced}) has been added to this edition to show students that quantifiers are frequently used in some of the courses they will take in the future.

\vskip6pt
\noindent
Typical Assignment:  Exercises 1, 2, 3, 4, 5, 6 or 7, 10, two parts of 12.
\hbreak


\subsection*{Explorations and Activities}
The activity in Exercise~(\ref{exer:prime}) should be done in class or assigned as it deals with the formal definition of a prime number.  Students may be able to give examples of prime numbers but they will have difficulty working with the formal defintion.

\newpar
The activity in Exercise~(\ref{A:upper}) must be assigned if you plan to do the activity in  Exercise~(\ref{exer:leastupper}).  As with the the activity in Exercise~(\ref{exer:prime}), this is an opportunity for students to work with a formal definition in mathematics.  These activities should be considered only if you plan to spend more than one day on this section.
\hbreak

\endinput
