\section*{Section~\ref{S:quantifier} Quantifiers and Negations}
Plan at least one class period for this section.  Some students may have difficulty with forming appropriate negations of quantified statement, and so it may be necessary to spend about one and one-half periods on this section.


\subsection*{Main Topics}
Quantifiers, negations of quantified statements, counterexamples, statements with more than one quantifier.

\subsection*{The Preview Activities}
\subsubsection*{Preview Activity~\ref{PA:quantifier} (Quantifiers)} 
The purpose of this preview activity is to have the students try to form negations of statements that contain a quanfifier.  One statement uses a universal quantifier and the other uses an existential quantifier.  Perfect answers are not expected on this preview activity, but having the students make the attempt will foster a good classroom discussion.

\subsubsection*{Preview Activity~\ref{PA:morethan} (Statements with Two Quantifiers)}  
In this preview activity, students will be forced to make a distinction between a predicate and a statement.  The sentence 
$\left( {\exists x \in \mathbb{R}} \right)\left( {x \cdot y = 100} \right)$ is not a statement since the variable $y$ is not quantified.  This is the point of the first five exercises in this activity.  In the last two exercises, students will work with statements that contain more than one quantifier.  Notice that the predicate in both exercises is the same.  The point to make is that the order of the quantifiers is important.  Some students may have difficulty with this preview activity.
\hbreak


%\subsection*{The Activities}
%There are five activities in this section.  If you plan to do two or more in class, you will need to use more than one class day for this section.  Have the students complete at least one of Activities~\ref{A:square} and~\ref{A:primes} as they provide practice with reading and understanding a definition.  This is valuable practice.  A good option is to do one of the activities in class and assign the other along with the exericses.
%
%Activities~\ref{A:upper} and~\ref{A:least} are optional, but they are valuable activities for students who will be studying the structure of the real number system in future courses.



%\subsubsection*{Activity~\ref{A:negating}}  
%This activity is similar to Preview Activity~\ref{PA:quantifier}.  It is useful to do this activity after students have done the preview activity and there has been discussion of negating quantified statements in class.  
%
%\subsubsection*{Activity~\ref{A:square}}
%Most students can give examples of numbers that are perfect squares but few will have worked with a formal defintion of this concept.  Stress that the first two exercises in this activity are a good way to deal with any new definition in mathematics.  It is important to point out that many definitions will involve the uese of a quantifier.  The last two exercises give students a chance to work with the existential quantifier in this definition.

\subsection*{Activity~\ref{A:primes} (Prime Numbers and Composite Numbers)}
This activity should be done in class or assigned as it deals with the formal definition of a prime number.  Students may be able to give examples of prime numbers but they will have difficulty working with the formal defintion.

\subsection*{Activity~\ref{A:upper} (Upper Bounds for Subsets of $\boldsymbol{\R}$)}
This activity must be assigned if you plan to assign Exercise~(\ref{exer:leastupper}).  As with the previous activity, this is an opportunity for students to work with a formal defintion in mathematics.  This activity should be considered only if you plan to spend more than one day on this section.

%\subsubsection*{Activity~\ref{A:least}}
%The idea of a least upper bound is difficult for many students.  The notation will also be difficult to handle for many students.
\hbreak


\subsection*{The Exercises}

Assign the first four exercises or at least substantial parts of each of these exercises.  Exericses~(\ref{exer:24-increasing}) and~(\ref{exer:24-continuous}) deal with the formal definitions of some concepts from calculus.  At least one of these exercises should be assigned.

\vskip6pt
\noindent
Typical Assignment:  Exercises 1, 2, 3, 4, 5, 6
\hbreak
\endinput
