\section*{Section~\ref{S:direct} Constructing Direct Proofs}
Plan about one class period for this section.

\subsection*{Main Topics}
Constructing and writing proofs in mathematics, properties of number systems, even and odd integers.  

\subsection*{The Preview Activities}
\subsubsection*{Preview Activity~\ref{PA:even} (Definition of Even and Odd Integers)}  
The first proof discussed in the text is, ``If $x$ and $y$ are odd integers, then $x \cdot y$ is an odd integer.''  So, the definitions of even and odd integers will be used right away.  Students, however, are not yet use to working with precise, formal definitions.  This is why the instructions in the preview activity say to ``Use the definition \ldots''

\subsubsection*{Preview Activity~\ref{PA:thinking} (Thinking about a Proof)}  In the text, we will use the proposition, ``If $x$ and $y$ are odd integers, then $x \cdot y$ is an odd integer'' to illustrate the process of constructing a proof and illustrate the use of a know-show table.  The purpose of this activity is to get students thinking about this proposition before it is discussed in the text or in class.  Following are some suggestions for instructors who would like to use a different example to illustrate this process in class.
\begin{itemize}
\item If $x$ and $y$ are odd integers, then $x + y$ is an even integer.
\item If $x$ is an even integer and $y$ is an integer, then $x \cdot y$ is an even integer.
\end{itemize}
The idea is to keep the proposition simple so that the process can be discussed.
\hbreak

\subsection*{Activity~\ref{A:kstable2} (Exploring a Proposition)}
This activity is intended to get students use to the idea of exploring whether a proposition is true or false and to provide them with practice at using a know-show table.  Remember that the tables themselves are not as important as the fact that they provide a convenient way for students to organize their work and as a way of thinking about the problem.    Instead of immediately trying to write a complete proof, the know-show table forces students to stop, think, and ask questions about how to prove the result and how to use the assumptions of the problem.
\hbreak

\subsection*{The Exercises}

In the interest of time, do not assign all the exercises.  Students will not become experts at writing proofs at this time.  The important thing is to get them started with the process.


I usually assign one part from each of Exercises~(\ref{exer:nextint}), (\ref{exer:integeradd}), (\ref{exer:integermult}), (\ref{exer:5m+7}), and~(\ref{exer:3m2}).  In addition, I usually assign at most four of the remaining problems.  
Exercises~(\ref{exer:sec12-type}) and~(\ref{exer:sec12-typeproof}) provide an opportunity for students to work with another definition.  I usually assign one or two of the proofs in Exercise~(\ref{exer:sec12-typeproof}).  Do not worry about the general definition of congruence modulo $n$.  This will be discussed in Section~\ref{S:directproof} and will be used throughout the text afterwards.

\vskip6pt
\noindent
Typical Assignment:  Exercises 1(a), 2(b), 3(c), 4(b), 5(a), 6, 7, 9, 10(a)

\hbreak










\endinput
