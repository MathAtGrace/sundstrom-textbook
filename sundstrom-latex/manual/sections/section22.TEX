\section*{Section~\ref{S:logequiv} Logically Equivalent Statements}
Plan about one class period for this section.

\subsection*{Main Topics}
Logically equivalent statements, converse and contrapositive, De Morgan's Laws, logical equivalencies related to conditional statements, the negation of a conditional statement.


\subsection*{The Preview Activities}
\subsubsection{Preview Activity~\ref{PA:logequiv} (Logically Equivalent Statements)}  This Preview Activity contains a new definition (logically equivalent statements), but students should be able to understand this.  In Part~(2) of the Preview Activity, students will verify one of De Morgan's Laws $\left( \mynot \left( P \wedge Q \right) \equiv \mynot P \vee \mynot Q \right)$.  This is part of Theorem~\ref{T:demorgan} in the section.  

\subsubsection*{Preview Activity~\ref{PA:converse} (Converse and Contrapositive)}
The converse and contrapositive of a conditional statement are defined.  Students are asked to complete truth tables to show that the contrapositive is logically equivalent to the original conditional statement but the converse is not.  Students must understand this before they go on to the next section.  The logical equivalence of a conditional statement and its contrapositive is an important idea that will be used when proof methods are studied in Chapter~\ref{C:proofs}.

%\subsubsection*{Preview Activity~\ref{PA:conditional2} (Conditional Statements)}
%The purpose of this Preview Activity is to show that a conditional statement is logically equivalent to a disjunction, and then to begin work with the negation of a conditional statement.  Students may have a difficult time with this.  However, this is very important as it forms the logical basis for a proof by contradiction.  Examples~\ref{E:conditionalasor} and~\ref{E:negationofcond} are related to this Preview Activity.
\hbreak


\subsection*{The Exercises}

Each of the exercises is a straightforward application of the material in the section.  The first three exercises should be assigned so that students get plenty of practice forming negations of statements.  Exercise~(\ref{exer:sec23-biconda}) should be assigned as it will be used to justify a proof technique in Chapter~\ref{C:proofs}.  Assign at least one of the parts in 
Exercise~(\ref{exer:sec22-distrib}).  Exercise~(\ref{exer:sec23-6}), 
%(\ref{exer:sec23-7}), and~(\ref{exer:sec23-9}) 
is strongly recommended as it forms the basis for a proof technique in Chapter~\ref{C:proofs} (Proof using cases).  Exercises~(\ref{exer:diffimpliescont}) and~(\ref{exer:sec23-10}) are good exercises as they ask students to work with the logical equivalencies with actual conditional statements from mathematics.   Parts of exercise~(\ref{exer:sec23-8}) should be assigned if you are interested in having students be able to establish logical equivalencies without using truth tables.

\vskip6pt
\noindent
Typical Assignment:  Exercises 1, 2, 3, 4(a), 5(a), 6, 7(a), 8, 9(a, b, c), 10 or 11
\hbreak


\subsection*{Explorations and Activities}

The short activity in exercise~(\ref{A:workingeq}) is a useful activity.  It provides an opportunity for the students to rewrite a conditional statement into an equivalent conditional statement using some of the standard logical equivalencies.  The students will actually use this method to prove such statements in Chapter~\ref{C:proofs}.

%\subsubsection*{Activity~\ref{A:workingeq2}}
%This activity provides practice at constructing a truth table and then using truth tables to conlude that two statements are logically equivalent.  The activity ends by using this logical equivalency in a context that will be seen later in the text.
\hbreak


\endinput
