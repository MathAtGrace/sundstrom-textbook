\section*{Section~\ref{S:prop} Statements}  
Plan about one class period for this section.

\subsection*{Main Topics}
Statements in mathematics, conditional statements in mathematics, closure properties of the standard number systems.  

\subsection*{The Preview Activities}
There are no preview activities for this section.  The preview activities from the second edition have been incorportated into the section.  This was done so that it would be possible to have students read the entire section before the second class period.

%\subsubsection*{Preview Activity~\ref{PA:prop} (Statements)}  
%The purpose of this preview activity is to introduce students to the definition of a \textbf{statement} as it is used in mathematics.  Students may not at first realize that an equation is a sentence that is not a statement.  In addition, some of the sentences in this preview activity involve quantifiers even though these have not yet been defined.  The purpose here is not to be mathematically complete but simply to get them to be able to determine which sentences are statements and which are not.  It is hoped that this preview activity will generate some discussion in class.
%
%\subsubsection*{Preview Activity~\ref{PA:conditional} (Conditional Statements)}  
%This is a very important preview activity.  \textbf{Conditional statements} are extremely important in mathematics and students must have a thorough understanding of conditional statements.  Students may have a difficult time because their use of conditional statements may not be consistent with the mathematical use of conditional statements.  Plan to spend some time discussing this in class.  The statement in 
%Part~(\ref{PA:conditional3}) (If $n$ is a positive integer, then $n^2 -n +41$ is a prime number) is discussed at the end of the section.
\hbreak


\subsection*{The Exercises}

Exercise~(\ref{exer:sec11-5}) is a good exercise to help students understand the truth values of conditional statements.  Assign at least one of Exercises~(\ref{exer:sec11-6}), ~(\ref{exer:sec11-7}), and~(\ref{exer:sec1-1-8}).  These are exercises intended to help students understand how conditional statements are used in mathematics.

\vskip6pt
\noindent
Typical Assignment:  Exercises 1, 2, 4, 5, (6, 7, or 8), 9.
\hbreak

\subsection*{The Activities and Explorations}
Exercise~(\ref{exer11:explore}) can be used as an in-class group activity.  Each one of the statements in this activity can be stated in the form of a conditional statement or can use quantifiers.  Students generally do not worry about this at this time and a very informal use of quantifiers works well here.  If students ask, I will discuss quantifiers briefly and indicate we will study them further in Section~\ref{S:quantifier}.  After students work on this activity, I usually return to the statements in Progress Check~\ref{prog:condition}.
\hbreak

\endinput










\endinput
