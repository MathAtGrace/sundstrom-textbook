\section*{Section~\ref{S:introfunctions} Introduction to Functions}

\subsection*{Preview Activity 1 (Functions from Previous Courses)}

Equations (1), (3), (4), (6), and (7) can be used to define a function with  $x$  as the input and  $y$  as the output.  In Equation (7), the domain must be restricted to all real numbers not equal to 1.
\hbreak

\noindent
\subsection*{Preview Activity 2 (Some Other Types of Functions)}
\begin{enumerate}
\item \begin{enumerate}
\item The birthday function  $b$  is  a function since each person has exactly one birthday.

\item We can write the fact that Andrew Wiles'  birthday is April 11 as 
$b( {\text{Andrew Wiles}} ) = \text{ April 11}$.

\item The statement is true since there has been at least one person born on each day of the year.

\item The statement is false since there do exist different people who have the same birthday.
\end{enumerate}


\item \begin{enumerate}
%\item The domain of the function  $s$  is the set of natural numbers $\mathbb{N}$  .  We can also use   $\mathbb{N}$   as the codomain of   $s$.

\item \begin{multicols}{4}
$s( 1 ) = 1$	

$s( 2 ) = 3$	

$s( 3 ) = 4$	

$s( 4 ) = 7$

$s( 5 ) = 6$	

$s( 6 ) = 12$	

$s( 7 ) = 8$	

$s( 8 ) = 15$

$s( 9 ) = 13$	

$s( {10} ) = 18$	

$s( {11} ) = 12$	

$s( {12} ) = 28$

$s( {13} ) = 14$	

$s( {14} ) = 24$	

$s( 15 ) = 24$	

$s( {16} ) = 31$
\end{multicols}

%\item The numbers  $\sqrt 5$ , $\pi$ , and  $- 6$  are not natural numbers and hence, are not in the domain of the function  $s$. The domain of the function $s$ is the set of natural numbers $\N$.
%This means that  $s( {\sqrt 5 } )$, 
%$s( \pi  )$, and $s( { - 6} )$  are not defined.

\item There does not exist a natural number  $n$  such that  $s( n ) = 5$.  This can be seen by the values of  $s( 1 )$, $s( 2 )$, $s( 3 )$, and 
$s( 4 )$ in Part (2) and by observing that if  $n \geq 5$, then  1  and  $n$  are factors of  $n$, and so  $s( n ) > 5$.

\item Yes.  For example,  $s( 6 ) = s( {11} ) = 12$ and  
$s( {14} ) = s( {15} ) = 24$.

\item Part (b) shows that Statement (i) is false, and Part~(c) shows that Statement~(ii) is false.
\end{enumerate}
\end{enumerate}
\hbreak

\endinput
