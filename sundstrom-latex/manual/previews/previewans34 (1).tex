\section*{Section~\ref{S:cases} Using Cases in Proofs}

\subsection*{Preview Activity 1 (A Logical Equivalency)}

\begin{enumerate}
\item 
$$
\BeginTable
\BeginFormat
| c | c | c | c | c | c | c | c |
\EndFormat
\_6
       | $P$  |  $Q$  |  $R$  \|6  $P \vee Q$  |  $(P \vee Q) \to R$  |  $P \to R$ | $Q \to R$ | $(P \to R) \wedge (Q \to R)$ | \\+22 \_6
          | T | T | T \|6 T | T | T  | T | T | \\ 
          | T | T | F \|6 T | F | F  | F | F | \\ 
          | T | F | T \|6 T | T | T  | T | T | \\ 
          | T | F | F \|6 T | F | F  | T | F | \\ 
          | F | T | T \|6 T | T | T  | T | T | \\ 
          | F | T | F \|6 T | F | T  | F | F | \\ 
          | F | F | T \|6 F | T | T  | T | T | \\ 
          | F | F | F \|6 F | T | T  | T | T | \\ \_6
\EndTable
$$


\item If we prove both  $P \to R$  and  $Q \to R$, then we have proven  
$( {P \to R} ) \wedge ( {Q \to R} )$.  Since the statements   
$( {P \vee Q} ) \to R$  and   
$( {P \to R} ) \wedge ( {Q \to R} )$  are logically equivalent, this means we have also proven  $( {P \vee Q} ) \to R$.

\item The contrapositive is:  For all integers $x$ and $y$, if $x$ is even or $y$ is even, then $xy$ is even.

\item If $x$ is an even integer, then there exists an integer $k$ such that $x = 2k$.  So if $y$ is an integer, then
\[
xy = (2k)y = 2(ky).
\]
Since $ky$ is an integer, this proves that if $x$ is even, then $xy$ is even.

%\newpar
\eighth
\noindent
If $y$ is an even integer, then there exists an integer $m$ such that $y = 2m$.  So if $x$ is an integer, then
\[
xy = x(2m) = 2(xm).
\]
Since $xm$ is an integer, this proves that if $y$ is even, then $xy$ is even.

\item The proposition in part~(3) is of the form $(P \vee Q) \to R$, which is logically equivalent to 
$( {P \to R} ) \wedge ( {Q \to R} )$.  In part~(4), we proved $P \to R$ and $Q \to R$ and so we proved 
$( {P \to R} ) \wedge ( {Q \to R} )$.  Because of the logical equivalency, we have proved the proposition in part~(3).



%\item Since an integer is either even or odd, the hypothesis, ``$n$  is an integer'', can be written as, ``$n$ is an even integer or  $n$  is an odd  integer.''  This means that the given proposition can be written in the form  $( {P \vee Q} ) \to R$.  Hence, it is logically equivalent to  $( {P \to R} ) \wedge ( {Q \to R} )$.
\end{enumerate}
\hbreak


\subsection*{Preview Activity 2 (Using Cases in a Proof)}

\begin{enumerate}
\item Let  $n$  be an even integer.  We will show that  $n^2  + n$  is an even integer.  By the definition of an even integer, there exists an integer  $m$  such that
\[
n = 2m.
\]
Substituting this into the expression  $n^2  + n$  yields
\[
\begin{aligned}
  n^2  + n &= ( {2m} )^2  + 2m \\ 
           &= 4m^2  + 2m \\ 
           &= 2( {2m^2  + m} ). \\ 
\end{aligned} 
\]
By the closure properties of the integers,  $2m^2  + m$  is an integer, and hence   
$n^2  + n$  is even.  So this proves that  when  $n$  is an even integer, $n^2  + n$  is an even integer. \qedsymbol

\item Let  $n$  be an odd integer.  We will show that  $n^2  + n$  is an even integer.  By the definition of an odd integer, there exists an integer  $m$  such that
\[
n = 2m+1.
\]
Substituting this into the expression  $n^2  + n$  yields
\[
\begin{aligned}
  n^2  + n &= ( {2m + 1} )^2  + ( 2m+1 ) \\ 
           &= 4m^2  + 6m + 2 \\ 
           &= 2( {2m^2  + 3m + 1} ). \\ 
\end{aligned} 
\]
By the closure properties of the integers,  $2m^2  + 3m + 1$  is an integer, and hence   
$n^2  + n$  is even.  So this proves that  when  $n$  is an odd integer, $n^2  + n$  is an even integer. \qedsymbol

\item The proofs of Propositions 2 and 3 constitute a proof of Proposition 1 since the integer $n$ must be even or must be odd.
\end{enumerate}
\hbreak




\newpage

\endinput
