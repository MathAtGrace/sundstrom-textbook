\section*{Section~\ref{S:directproof} Direct Proofs}

\subsection*{Preview Activity 1 (Definition of Divides, Divisor, Multiple)}

\begin{enumerate} 
  \item The integer 4 divides 32 since $32 = 4 \cdot 8$.  The integer 8 divides $-96$ since 
          $-96 = 8 \cdot (-12)$.
  \item For example:  3 does not divide 4; 5 does not divide 17; 2 does not divide $-7$.
  \item The integer 10 divides 0 since $0 = 10 \cdot 0$.

  \item The integer  $m$  does not divide the integer  $n$  means:  
          $\left( {\forall q \in \mathbb{Z}} \right)\left( {n \ne m \cdot q} \right)$.

\item The integer  $m$  does not divide the integer  $n$  means that for all  
$q \in \mathbb{Z}$, $n \ne m \cdot q$.
\addtocounter{enumi}{1}
\item Each example in part (6) should be an example where the hypothesis of the conjecture is true and the conclusion of the conjecture is true, that is, the examples should indicate that the conjecture is true.
\end{enumerate}

\noindent
\textbf{A Note about the Conjecture}
This conjecture could have been be stated as follows:

\vskip6pt
For all integers $a$, $b$, and  $c$, if $a$ divides $b$ and $b$ divides $c$, then $a$ divides $c$.
\vskip6pt

\noindent
Sometimes, to vary the way we write things, we do not explicitly use the quantifiers.  The conjecture written in the text is an acceptable way to write the conjecture.  Using the word ``let'' is a method that mathematicians accept as equivalent to saying it is true for all integers.  
The idea is the we can let $a$, $b$, and  $c$  be any integers.

You will be asked to formulate conjectures in this and other courses.  The message here is to be very careful in how you formulate your conjecture, and that you should do so by formulating the conjecture as a statement or proposition that can be proven true or false.

By doing this, we can also easily formulate the negation of the conjecture in case the conjecture happens to be false.


%\setcounter{oldenumi}{7}
\begin{enumerate} \setcounter{enumi}{7} %\setcounter{enumi}{\theoldenumi}
  \item We would assume that $a, b$, and $c$ are integers, $a \ne 0$, $b \ne 0$, and that $a$ divides $b$ and 
        $b$ divides $c$.
  \item There exist integers $s$ and $t$ such that $b = a \cdot s$ and $c = b \cdot t$.
  \item We would be trying to prove that $a$ divides $c$.
  \item We could prove that there exists an integer $q$ such that  $c = a \cdot q$.
\end{enumerate}

\hbreak




\subsection*{Preview Activity 2 (Calendars and Clocks)}
\begin{enumerate}
\item \begin{multicols}{3} 
\begin{enumerate}
\item Friday

\item Friday

\item Friday
\end{enumerate}
\end{multicols}

\begin{enumerate}
\setcounter{enumii}{3}
\item Some examples are: 31, 38, 45, 52, 59, 66, 73.

\addtocounter{enumii}{1}
\item All the numbers from Part (e) are multiples of 7.
\end{enumerate}

%\newpage
\item \begin{multicols}{2} 
\begin{enumerate}

\item 9:00

\item 9:00
\end{enumerate}
\end{multicols}

\begin{enumerate}
\setcounter{enumii}{2}
\item Some examples are: 28, 40, 52, 64, 76, 88.

\addtocounter{enumii}{1}
\item All the numbers from Part (d) are multiples of 12.
\end{enumerate}

\item \begin{multicols}{2} 
\begin{enumerate}

\item Friday

\item Friday
\end{enumerate}
\end{multicols}

\begin{enumerate}
\setcounter{enumii}{2}
\item Some examples are: -32, -25, -18, -11, -4, 3, 10, 17, 24, 31, 38.

\addtocounter{enumii}{1}
\item All the numbers from Part (d) are multiples of 7.
\end{enumerate}

\end{enumerate}
\hbreak



\newpage

\endinput
