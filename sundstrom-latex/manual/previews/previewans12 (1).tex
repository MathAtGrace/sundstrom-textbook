\section*{Section~\ref{S:direct} Constructing Direct Proofs}

\subsection*{Preview Activity 1 (Definition of Even and Odd Integers)}
\begin{enumerate}
\item \begin{enumerate}
\item The integer 28 is an even integer since $28 = 2 \cdot 14$ and 14 is an integer. \\
The integer $-42$ is an even integer since $-42 = 2 \cdot (-21)$ and $-21$ is an integer. \\
The integer 24 is an even integer since $24 = 2 \cdot 12$ and 12 is an integer. \\
The integer 0 is an even integer since $0 = 2 \cdot 0$ and 0 is an integer. \\

\item The integer 51 is an odd integer since $51 = 2 \cdot 25 + 1$ and 25 is an integer. \\
The integer $-11$ is an odd integer since $-11 = 2 \cdot (-6) + 1$ and $-6$ is an integer. \\
The integer 51 is an odd integer since $51 = 2 \cdot 25 + 1$ and 25 is an integer. \\
The integer 1 is an odd integer since $1 = 2 \cdot 0 + 1$ and 0 is an integer. \\
The integer $-1$ is an odd integer since $-1 = 2 \cdot (-1) + 1$ and $-1$ is an integer. \\
\end{enumerate}
\end{enumerate}

%\begin{enumerate}
%\item $8 = 2 \cdot 4$   and 4 is an integer. \qquad $-12 = 2 \cdot \left( -6 \right)$  and -6 is an integer.\\
%$24 = 2 \cdot 12$ and 12 is an integer. \qquad $0 = 2 \cdot 0$ and 0 is an integer.
%
%
%\item $7 = 2 \cdot 3 + 1$  and 3 is an integer. \quad  $-11 = 2 \cdot \left( -6 \right) + 1$ and  -6 is an integer. \\
%$51 = 2 \cdot 25 + 1$ and 51 is an integer. \quad $1 = 2 \cdot 0 + 1$ and 0 is an integer. \\
%$-1 = 2 \cdot \left( -1 \right) + 1$ and -1 is an integer.
%\end{enumerate}
\hbreak



%\newpage
%\section*{Section~\ref{S:direct} Constructing Direct Proofs}
\subsection*{Preview Activity 2 (Thinking about a Proof)}
\begin{enumerate}
\item The hypothesis of the conditional statement is ``$x$  and  $y$  are odd integers'', and the conclusion of the conditional statement is ``$x \cdot y$ is an odd integer.''

\item It does not prove the conditional statement is false since it provides an example where the hypothesis of the conditional statement is false.  In this situation, the conditional statement is true.

\item This does not prove the conditional statement is true.  It is only one example.  We must be able to prove that no matter what odd integers we choose for  $x$  and  $y$, the product  
$x \cdot y$  will always be odd.

\item Since $y$ is odd, there exists an integer $n$ such that $y = 2n + 1$.

\item We can prove that $x \cdot y$ is an odd integer by proving that there exists an integer $q$ such that 
$x \cdot y = 2q + 1$.

%\item To prove the conditional statement is true, we must prove that the conclusion of the statement, $x \cdot y$ is an odd integer, is true whenever the hypothesis, $x$  and  $y$  are odd integers,  is true.  We do not have to worry about the situation when the hypothesis is false.
%
%\item We start the proof by assuming that  $x$  and  $y$  are odd integers.
%
%\item We need to prove that the product  $x \cdot y$  is an odd integer.
%
%\item In order to prove that  $x \cdot y$ is an odd integer, we need to prove that there exists an integer  $q$  such that  $x \cdot y = 2q + 1$.
\end{enumerate}

\hbreak
\newpage

\endinput
