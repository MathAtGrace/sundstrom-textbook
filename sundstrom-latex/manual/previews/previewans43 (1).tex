\section*{Section~\ref{S:recursion} Induction and Recursion}

\subsection*{Preview Activity 1 (Recursively Defined Sequences)}
\begin{enumerate} 
  \item Based on the following, it appears that as $n$ gets larger, $b_n$ gets closer to zero.
\begin{align*}
b_4 &= 2   & b_5 &= 1 &  b_6 &= \frac{1}{2} &  b_7 &= \frac{1}{4} \\
b_8 &= \frac{1}{8} &  b_9 &= \frac{1}{16}  &  b_{10} &= \frac{1}{32} 
\end{align*}
  \item Based on the following, it appears that as $n$ gets larger, $T_n$ gets closer to 32.
\begin{align*}
T_3 &= 28  &  T_4 &= 30   & T_5 &= 31 &  T_6 &= 31  \\
T_7 &= 31.75 & T_8 &= 31.875 &  T_9 &= 31.9375  &  T_{10} &= 31.96875 
\end{align*}

\item Based on the following, a conjecture could be:  $a_n = r^{n-1} \cdot a$.
\begin{align*}
a_2 &= r \cdot a  &  a_3 &= r^2 \cdot & a_4 &= r^3 \cdot a \\
a_5 &= r^4 \cdot a & a_6 &= r^5 \cdot a
\end{align*}

\item Based on the following, a conjecture could be:  $S_n = a + ra + r^2 + \cdots r^{n-1} a$.
\begin{align*}
S_2 &= a + ra  &  S_3 &= a + ra + r^2 a  &  S_4 &= a + ra + r^2 a + r^3 a \\
S_5 &= a + ra + r^2 a + r^3 a r^4 a & S_6 &= a + ra + r^2 a + r^3 a r^4 a + r^5 a
\end{align*}

\item
\begin{tabular}[t]{| c | c | c | c | c | c | c | c |} \hline
$n$  &  1  &  2  &  3  &  4  &  5  &  6  &  7  \\ \hline
$a_n$ &  1  &  2  &  6  &  24  &  120  &  720  &  5040  \\ \hline
$n!$  &  1  &  2  &  6  &  24  &  120  &  720  &  5040  \\ \hline
\end{tabular}

\item In order to calculate  $a_{100} $, we first need to calculate  
$a_0 , a_1 , a_2 ,  \ldots , a_{99} $.

\item For each  $n$,  in order to calculate  $a_n $, we first need to calculate  
$a_0 , a_1 , a_2 , \ldots , a_{n - 1} $.

\item It seems that for each nonnegative integer  $n$,  $a_n  = n!$.

\end{enumerate}
\hbreak


\subsection*{Preview Activity 2 (The Fibonacci Numbers)}
\begin{multicols}{4}
$f_5  = 5$

$f_6  = 8$

$f_7  = 13$

$f_8  = 21$

$f_9  = 34$

$f_{10}  = 55$

$f_{11}  = 89$

$f_{12}  = 144$

$f_{13}  = 233$

$f_{14}  = 377$

$f_{15}  = 610$

$f_{16}  = 987$

$f_{17}  = 1597$

$f_{18}  = 2584$

$f_{19}  = 4181$

$f_{20}  = 6765$

%$f_{21}  = 10946$
%
%$f_{22}  = 17711$
%
%$f_{23}  = 28657$
%
%$f_{24}  = 46368$
\end{multicols}

\begin{enumerate} \setcounter{enumi}{1}
\item  The Fibonacci numbers $f_3$, $f_6$, $f_9$, $f_{12}$, $f_{15}$, and $f_{18}$ are even.  \\
The Fibonacci numbers $f_4$, $f_8$, $f_{12}$, $f_{16}$, and $f_{20}$ are multiples of three.

\item In each case, the sum of the first $(n - 1)$ Fibonacci numbers is equal to $f_{n+1} - 1$.
\item One other observation is that it appears that for each  $n \in \mathbb{N}$,  $f_{5n} $  is a multiple of 5.
\end{enumerate}
\hbreak


\newpage

\endinput
