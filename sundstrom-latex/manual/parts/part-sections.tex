\part{The Chapters and Sections}
\markboth{Chapters and Sections}{Chapters and Sections}
\chapter*{Chapter~\ref{C:intro} \\Introduction to Writing in Mathematics}

\section*{Main Objectives}
After this chapter, students should:
\begin{itemize}
\item Understand the definition of a statement in mathematics.
\item Understand how mathematicians use conditional statements.
\item Understand the closure properties of the standard number systems.
\item Have begun the process of constructing and writing simple, direct proofs.
\item Understand and be able to use some of the conventions used in writing mathematics.

\end{itemize}
%\begin{itemize}
%\item Get the students started in writing mathematics.
%\item Introduce students to the definition of a statement in mathematics.
%\item Begin work on understanding conditional statements.
%\item Start the process of constructing and writing simple, direct proofs.
%\end{itemize}

It must be emphasized that the goal is to get students started with writing proofs and to understand how conditional statements are used in mathematics.  It is not expected that they will perfect their writing skills and their understanding of conditional statements at this time.  However, it is extremely important that students have a thorough understanding of conditional statements before starting Chapter~\ref{C:proofs}.  More work on conditional statements will occur in Chapter~\ref{C:logic}.
\hbreak

\section*{Section~\ref{S:prop} Statements}  
Plan about one class period for this section.

\subsection*{Main Topics}
Statements in mathematics, conditional statements in mathematics, closure properties of the standard number systems.  

\subsection*{The Preview Activities}
There are no preview activities for this section.  The preview activities from the second edition have been incorportated into the section.  This was done so that it would be possible to have students read the entire section before the second class period.

%\subsubsection*{Preview Activity~\ref{PA:prop} (Statements)}  
%The purpose of this preview activity is to introduce students to the definition of a \textbf{statement} as it is used in mathematics.  Students may not at first realize that an equation is a sentence that is not a statement.  In addition, some of the sentences in this preview activity involve quantifiers even though these have not yet been defined.  The purpose here is not to be mathematically complete but simply to get them to be able to determine which sentences are statements and which are not.  It is hoped that this preview activity will generate some discussion in class.
%
%\subsubsection*{Preview Activity~\ref{PA:conditional} (Conditional Statements)}  
%This is a very important preview activity.  \textbf{Conditional statements} are extremely important in mathematics and students must have a thorough understanding of conditional statements.  Students may have a difficult time because their use of conditional statements may not be consistent with the mathematical use of conditional statements.  Plan to spend some time discussing this in class.  The statement in 
%Part~(\ref{PA:conditional3}) (If $n$ is a positive integer, then $n^2 -n +41$ is a prime number) is discussed at the end of the section.
\hbreak


\subsection*{The Exercises}

Exercise~(\ref{exer:sec11-5}) is a good exercise to help students understand the truth values of conditional statements.  Assign at least one of Exercises~(\ref{exer:sec11-6}), ~(\ref{exer:sec11-7}), and~(\ref{exer:sec1-1-8}).  These are exercises intended to help students understand how conditional statements are used in mathematics.

\vskip6pt
\noindent
Typical Assignment:  Exercises 1, 2, 4, 5, (6, 7, or 8), 9.
\hbreak

\subsection*{The Activities and Explorations}
Exercise~(\ref{exer11:explore}) can be used as an in-class group activity.  Each one of the statements in this activity can be stated in the form of a conditional statement or can use quantifiers.  Students generally do not worry about this at this time and a very informal use of quantifiers works well here.  If students ask, I will discuss quantifiers briefly and indicate we will study them further in Section~\ref{S:quantifier}.  After students work on this activity, I usually return to the statements in Progress Check~\ref{prog:condition}.
\hbreak

\endinput










\endinput

\section{Constructing Direct Proofs} \label{S:direct}
%\markboth{Chapter \ref{C:intro}. Introduction to Writing}{\ref{S:direct}. Constructing Direct Proofs}
\setcounter{previewactivity}{0}
%\input{focus/focus12}
\begin{previewactivity}[\textbf{Definition of Even and Odd Integers}]\label{PA:even} \hfill \\
Definitions play a very important role in mathematics.  A direct proof of a proposition in mathematics is often a demonstration that the proposition follows logically from certain definitions and previously proven propositions.  A \textbf{definition}
\index{definition}%
 is an agreement that a particular word or phrase will stand for some object, property, or other concept that we expect to refer to often.  In many elementary proofs, the answer to the question, ``How do we prove a certain proposition?'', is often answered by means of a definition.  For example, in Progress Check~\ref{pr:explores} on page~\pageref{pr:explores}, all of the examples should have indicated that the following conditional statement is true:
\begin{center}
If  $x$  and  $y$  are odd integers, then $x \cdot y$ is an odd integer.
\end{center}
In order to construct a mathematical proof of this conditional statement, we need a precise definition of what it means to say that an integer is an even integer and what it means to say that an integer is an odd integer.
%
\begin{defbox}{D:even}{An integer  $a$  is an \textbf{even integer} 
\index{even integer}%
 provided that there exists an integer  $n$  such that  $a = 2n$. An integer  $a$  is an 
\textbf{odd integer}
\index{odd integer}%
 provided there exists an integer  $n$  such that  $a = 2n + 1$.}
\end{defbox}
\newpar
\label{def:even}
Using this definition, we can conclude that the integer 16 is an even integer since $16 = 2 \cdot 8$ and 8 is an integer.  By answering the following questions, you should obtain a better understanding of these definitions.  These questions are not here just to have questions in the textbook.  Constructing and answering such questions is a way in which many mathematicians will try to gain a better understanding of a definition.
%
%\begin{flushleft}
%\fbox{\parbox{5in}{\begin{definition}
%An integer  $a$  is an \textbf{even integer} if there exists an integer  $n$  such that  $a = 2n$. An integer  $a$  is an \textbf{odd integer} if there exists an integer  $n$  such that  $a = 2n + 1$.
%\end{definition}}}
%\end{flushleft}
\begin{enumerate}
\item Use the definitions given above to
\begin{enumerate}
\item Explain why  28, $-42$, 24, and 0 are even integers.

\item Explain why 51, $-11$, 1, and $-1$  are odd integers.
\end{enumerate}
\end{enumerate}

\noindent
It is important to realize that mathematical definitions are not made randomly.  In most cases, they are motivated by a mathematical concept that occurs frequently.
%
\begin{enumerate}
\setcounter{enumi}{1}
\item Are the definitions of even integers and odd integers consistent with your previous ideas about even and odd integers?	
\end{enumerate}
\hbreak
\end{previewactivity}

\endinput

\begin{previewactivity}[\textbf{Thinking about a Proof}]\label{PA:thinking} \hfill \\
Consider the following proposition:  
\begin{flushleft}
\textbf{Proposition.}  If  $x$  and  $y$  are odd integers, then $ x \cdot y$  is an odd integer.
\end{flushleft}
Think about how you might go about proving this proposition. A \textbf{direct proof}
\index{direct proof}%
\index{proof!direct}%
 of a conditional statement is a demonstration that the conclusion of the conditional statement follows logically from the hypothesis of the conditional statement.  Definitions and previously proven propositions are used to justify each step in the proof.    To help get started in proving this proposition, answer the following questions:
\begin{enumerate}
  \item The proposition is a conditional statement.  What is the hypothesis of this conditional statement?  What is the conclusion of this conditional statement?
  \item If  $x = 2$ and  $y = 3$, then  $x \cdot y = 6$, and 6 is an even integer.  Does this example prove that the proposition is false?  Explain.
  \item If  $x = 5$ and  $y = 3$, then  $x \cdot y = 15$.  Does this example prove that the proposition is true?  Explain.
\end{enumerate}
In order to prove this proposition, we need to prove that whenever both $x$ and $y$ are odd integers, 
$x \cdot y$ is an odd integer.  Since we cannot explore all possible pairs of integer values for $x$ and $y$, we will use the definition of an odd integer to help us construct a proof.  

\begin{enumerate} \setcounter{enumi}{3}
  \item To start a proof of this proposition, we will assume that the hypothesis of the conditional statement is true.  So in this case, we assume that both $x$ and $y$ are odd integers.  We can then use the definition of an odd integer to conclude that there exists an integer $m$ such that $x = 2m + 1$.  Now use the definition of an odd integer to make a conclusion about the integer $y$.
\label{PA:prev12-Q4}

\note The definition of an odd integer says that a certain other integer exists.  This definition may be applied to both $x$ and $y$.  However, do not use the same letter in both cases.  To do so would imply that 
$x = y$ and we have not made that assumption.  To be more specific, if $x = 2m + 1$ and $y = 2m +1$, then 
$x = y$.
  \item We need to prove that if the hypothesis is true, then the conclusion is true.  So, in this case, we need to prove that $x \cdot y$ is an odd integer.  At this point, we usually ask ourselves a so-called 
\textbf{backward question}.  In this case, we ask, ``Under what conditions can we conclude that $x \cdot y$ is an odd integer?''  Use the definition of an odd integer to answer this question.% and be careful to use a different letter for the new integer than was used in Part~(\ref{PA:prev12-Q4}).
\end{enumerate}
\hbreak
\end{previewactivity}


\endinput



\subsection*{Properties of Number Systems}\label{SS:properties}
At the end of Section~\ref{S:prop}, we introduced notations for the standard number systems we use in mathematics and discussed their closure properties.  For this text, it is assumed that the reader is familiar with these closure properties and the basic rules of algebra that apply to all real numbers that are given in %That is, it is assumed the reader is familiar with the properties of the real numbers shown in 
Table~\ref{Ta:propertiesofreals}. 
\begin{table}[h]
$$
\BeginTable
\BeginFormat 
| p(1.5in)|p(3.0in)| 
\EndFormat
"   " For all real numbers $x$, $y$, and $z$ "  \\+02  \_
| Identity Properties |  $x+0=x$ and $x \cdot 1=x$ |   \\ \_1 
| Inverse Properties  |  $x + \left( { - x} \right) = 0$ and if $x \ne 0$, then 
$x \cdot \dfrac{1}{x} = 1$.  |  \\+55 \_1
|Commutative Properties  |  \Lower{$x + y = y + x$ and  $x  y = y  x$} | \\ \_1
|Associative Properties  |   \Lower{$\left( {x + y} \right) + z = x + \left( {y + z} \right)$ and 
                        $\left( {x  y} \right)  z = x  \left( {y  z} \right)$} | \\ \_1
|Distributive Properties |  \Lower{$x\left( {y + z} \right) = x  y + x  z$ and  
                       $\left( {y + z} \right)x = y  x + z  x$}  | \\ \_ 
\EndTable
$$
    \caption{Properties of the Real Numbers}
    \label{Ta:propertiesofreals}
\end{table}
\endinput

%


\subsection*{Constructing a Proof of a Conditional Statement}
In order to prove that a conditional statement $P \to Q$ is true, we only need to prove that  $Q$  is true whenever  $P$  is true.  This is because the conditional statement is true whenever the hypothesis is false.  So in a direct proof of  $P \to Q$, we assume that  $P$  is true, and using this assumption, we proceed through a logical sequence of steps to arrive at the conclusion that  $Q$  is true.  

Unfortunately, it is often not easy to discover how to start this logical sequence of steps. 
%or how to get to the conclusion that $Q$  is true.  
We will describe a method of exploration that often can help in discovering the steps of a proof.  This method will involve working forward from the hypothesis, $P$, and backward from the conclusion, $Q$.  We will use a device called the \textbf{``know-show table''}
\index{know-show table|(}%
 to help organize our thoughts and the steps of the proof.  This will be illustrated with the proposition from \typeu Activity~\ref*{PA:thinking}.
\begin{flushleft}
\textbf{Proposition.}  \emph{If  $x$  and  $y$  are odd integers, then $ x \cdot y$  is an odd integer.}
\end{flushleft}
The first step is to identify the hypothesis,  $ P$,  and the conclusion, $Q$,  of the conditional statement.  
%The hypothesis consists of everything you are assuming, and the conclusion consists of everything you are trying to prove.  
In this case,  we have the following:
\begin{center}
$P$: $x$ and $y$ are odd integers. \qquad $Q$: $x \cdot y$ is an odd integer.
\end{center}
%\vskip11pt
%\begin{multicols}{2}
%$P$:	$x$  and  $y$  are odd integers.
%
%$Q$:	$x \cdot y$ is an odd integer.
%\end{multicols}
%\noindent
We now treat  $P$  as what we know (we have assumed it to be true) and treat $Q$ as what we want to show (that is, the goal).  So we organize this by using  $P$  as the first step in the know portion of the table and  $Q$  as the last step in the show portion of the table.  We will put the know portion of the table at the top and the show portion of the table at the bottom.
$$
\BeginTable
\def\C{\JustCenter}
\BeginFormat
|p(0.4in)|p(2in)|p(1.8in)|
\EndFormat
  \_
  | \textbf{Step}  |  \textbf{Know}  |  \textbf{Reason}  |    \\+02 \_
|  $P$     |  $x$ and $y$ are odd integers.  |  Hypothesis | \\ \_1
|  $P1$    |                                 |             | \\ \_1
|  \C $\vdots$  |  \C $\vdots$                         | \C $\vdots$      | \\ \_1
|  $Q1$    |                                 |             | \\  \_1 
|  $Q$     |  $x \cdot y$ is an odd integer. |  ?          | \\ \_
|  \textbf{Step}  |  \textbf{Show}  |  \textbf{Reason}     | \\+20 \_
\EndTable
$$
We have not yet filled in the reason for the last step because we do not yet know how we will reach the goal.  The idea now is to ask ourselves questions about what we know and what we are trying to prove.  We usually start with the conclusion that we are trying to prove by asking a so-called \textbf{backward question.}
\index{know-show table!backward question}%
  The basic form of the question is, ``Under what conditions can we conclude that  $Q$  is true?''  How we ask the question is crucial since we must be able to answer it.  We should first try to ask and answer the question in an abstract manner and then apply it to the particular form of statement  $Q$.  

In this case, we are trying to prove that some integer is an odd integer.  So our backward question could be, ``How do we prove that an integer is odd?''  At this time, the only way we have of answering this question is to use the definition of an odd integer.  So our answer could be, ``We need to prove that there exists an integer  $q$  such that the integer equals  $2q + 1$.''  We apply this answer to statement  $Q$  and insert it as the next to last line in the know-show table.
$$
\BeginTable
\def\C{\JustCenter}
\BeginFormat
|p(0.4in)|p(2in)|p(1.8in)|
\EndFormat
  \_
  | \textbf{Step}  |  \textbf{Know}  |  \textbf{Reason}  |    \\+02 \_
|  $P$     |  $x$ and $y$ are odd integers.  |  Hypothesis | \\ \_1
|  $P1$    |                                 |             | \\ \_1
|  \C $\vdots$  |  \C $\vdots$                         | \C $\vdots$      | \\ \_1
|  $Q1$    |  There exists an integer $q$ such that $xy = 2q + 1$.                               |             | \\  \_1 
|  $Q$     |  $x \cdot y$ is an odd integer. |  Definition of an odd integer          | \\ \_
|  \textbf{Step}  |  \textbf{Show}  |  \textbf{Reason}     | \\+20 \_
\EndTable
$$
%The idea is to write the first step for the beginning of the proof ($P$) and the steps for the end of the proof ($Q$ and $Q1$).  We then try to fill in the steps for the middle of the proof, working backward from $Q1$ and working forward from $P$.
We now focus our effort on proving statement $Q1$ since we know that if we can prove $Q1$, then we can conclude that  $Q$  is true.  We ask a backward question about  $Q1$ such as, ``How can  we prove that there exists an integer $q$  such that  %\linebreak
$x \cdot y = 2q + 1$?''  We may not have a ready answer for this question, and so we look at the know portion of the table and try to connect the know portion to the show portion.  To do this, we work forward from step $P$, and this involves asking a \textbf{forward question.}
\index{know-show table!forward question}%
  The basic form of this type of question is, ``What can we conclude from the fact that  $P$  is true?''  In this case, we can use the definition of an odd integer to conclude that there exist integers  $m$  and $ n$  such that  $x = 2m + 1$  and  $y = 2n + 1$.  We will call this Step $P1$ in the know-show table.  It is important to notice that we were careful not to use the letter $q$ to denote these integers.  If we had used  $q$  again, we would be claiming that the same integer that gives  $x \cdot y = 2q + 1$  also gives  $x = 2q + 1$.  This is why we used  $m$  and  $n$  for the integers  $x$  and $y$  since there is no guarantee that  $x$ equals $y$.  The basic rule of thumb is to use a different symbol for each new object we introduce in a proof.  So at this point, we have:
\begin{itemize}
  \item Step $P1$.  We know that there exist integers $m$ and $n$ such that $x = 2m + 1$ and $y = 2n + 1$.
  \item Step $Q1$.  We need to prove that there exists an integer $q$ such that \\$x \cdot y = 2q + 1$.
\end{itemize}
%\end{flushleft}
%
%$$
%\BeginTable
%\def\C{\JustCenter}
%\BeginFormat
%|p(0.4in)|p(2in)|p(1.8in)|
%\EndFormat
%  \_
%  | \textbf{Step}  |  \textbf{Know}  |  \textbf{Reason}  |    \\+02 \_
%|  $P$     |  $x$ and $y$ are odd integers.  |  Hypothesis | \\ \_1
%|  $P1$    |  There exist integers $m$ and $n$ such that $x = 2m + 1$ and $y = 2n + 1$.                               | Definition of an odd integer |  \\ \_1
%|  \C $\vdots$  |  \C $\vdots$                         | \C $\vdots$      | \\ \_1
%|  $Q1$  |  There exists an integer $q$ such that $xy = 2q + 1$.                               |             | \\  \_1 
%|  $Q$     |  $x \cdot y$ is an odd integer. |  Definition of an odd integer          | \\ \_
%|  \textbf{Step}  |  \textbf{Show}  |  \textbf{Reason}     | \\+20 \_
%\EndTable
%$$
We must always be looking for a way to link the ``know part'' to the ``show part''.  There are conclusions we can make from $P1$, but as we proceed, we must always keep in mind the form of statement in $Q1$.  The next forward question is, ``What can we conclude about  $x \cdot y$  from what we know?''  One way to answer this is to use our prior knowledge of algebra.  That is, we can first use substitution to write  $x \cdot y = \left( {2m + 1} \right)\left( {2n + 1} \right)$.  Although this equation does not prove that $x \cdot y$ is odd, we can use algebra to try to rewrite the right side of this equation 
$\left( {2m + 1} \right)\left( {2n + 1} \right)$ in the form of an odd integer so that we can arrive at 
step $Q1$.  We first expand the right side of the equation to obtain
\begin{align*}
x \cdot y &= (2m + 1)(2n + 1) \\
          &= 4mn + 2m + 2n + 1
\end{align*}
%$$
%\BeginTable
%\def\C{\JustCenter}
%\BeginFormat
%|p(0.4in)|p(2in)|p(1.8in)|
%\EndFormat
%  \_
%  | \textbf{Step}  |  \textbf{Know}  |  \textbf{Reason}  |    \\+02 \_
%|  $P$     |  $x$ and $y$ are odd integers.  |  Hypothesis | \\ \_1
%|  $P1$    |  There exist integers $m$ and $n$ such that $x = 2m + 1$ and $y = 2n + 1$.                               | Definition of an odd integer |  \\ \_1
%| $P2$   | $xy = \left(2m + 1\right)\left(2n + 1 \right)$ | Substitution | \\ \_1
%| $P3$   | $xy = 4mn + 2m + 2n + 1$                       |  Algebra      | \\ \_1
%|  \C $\vdots$  |  \C $\vdots$                         | \C $\vdots$      | \\ \_1
%|  $Q1$  |  There exists an integer $q$ such that $xy = 2q + 1$.                               |             | \\  \_1 
%|  $Q$     |  $x \cdot y$ is an odd integer. |  Definition of an odd integer          | \\ \_
%|  \textbf{Step}  |  \textbf{Show}  |  \textbf{Reason}     | \\+20 \_
%\EndTable
%$$
Now compare the right side of the last equation to the right side of the equation in step $Q1$.  Sometimes the difficult part at this point is the realization that  $q$  stands for  some integer and that we only have to show that $x \cdot y$ equals two times some integer plus one.  Can we now make that conclusion?  The answer is yes because we can factor a 2 from the first three terms on the right side of the equation and obtain
\begin{align*}
x \cdot y &= 4mn + 2m + 2n + 1 \\
          &= 2 (2mn + m + n) + 1
\end{align*}
We can now complete the table showing the outline of the proof as follows:

$$
\BeginTable
\def\C{\JustCenter}
\BeginFormat
|p(0.4in)|p(2in)|p(1.8in)|
\EndFormat
  \_
  | \textbf{Step}  |  \textbf{Know}  |  \textbf{Reason}  |    \\+02 \_
|  $P$     |  $x$ and $y$ are odd integers.  |  Hypothesis | \\ \_1
|  $P1$    |  There exist integers $m$ and $n$ such that $x = 2m + 1$ and $y = 2n + 1$.                               | Definition of an odd integer. |  \\ \_1
| $P2$   | $xy = \left(2m + 1\right)\left(2n + 1 \right)$ | Substitution | \\ \_1
| $P3$   | $xy = 4mn + 2m + 2n + 1$                       |  Algebra      | \\ \_1
| $P4$   | $xy = 2 \left( 2mn + m + n \right) + 1$        |  Algebra      | \\ \_1
| $P5$   | $\left( 2mn + m + n \right)$ is an integer. | Closure properties of the integers | \\ \_1
|  $Q1$  |  There exists an integer $q$ such that $xy = 2q + 1$.                               | Use $q = \left( 2mn + m + n \right)$            | \\  \_1 
|  $Q$     |  $x \cdot y$ is an odd integer. |  Definition of an odd integer          | \\ \_
%|  \textbf{Step}  |  \textbf{Show}  |  \textbf{Reason}     | \\+20 \_
\EndTable
$$

%
%\begin{center}
%\begin{tabular}[h]{|p{0.4in}|p{2in}|p{1.8in}|}
%  \hline
%  \textbf{Step}  &  \textbf{Know}  &  \textbf{Reason} \\ \hline
%  $P$  &  $x$ and $y$ are odd integers.  &  Hypothesis \\ \hline
%  $P1$ &  There exist integers $m$ and $n$ such that $x = 2m+1$ and   &  Definition of an odd integer \\ 
%       &  $y = 2n+1$.                    &  \\  \hline
%  $P2$  &  $x \cdot y = \left( 2m+1 \right) \left( 2n+1 \right)$  &  Substitution \\ \hline
%  $P3$  &  $x \cdot y = 4mn+2m+2n+1$  &  Algebra  \\ \hline
%  $P4$  &  $x \cdot y = 2 \left( 2mn+m+n \right)+1$  &  Algebra  \\ \hline
%  $P5$  &  $\left( 2mn+m+n \right)$ is an integer.  &  Closure properties of the integers \\ \hline
%  $Q1$  &  There exists an integer $q$ such that $x \cdot y = 2q+1$.  &   \\ \hline
%  $Q$  &  $x \cdot y$ is an odd integer. &  Definition of an odd integer \\ \hline
%\end{tabular}
%\end{center}
%
It is very important to realize that we have only constructed an outline of a proof.  Mathematical proofs are not written in table form.  They are written in narrative form using complete sentences and correct paragraph structure, and they follow certain conventions used in writing mathematics.  In addition, most proofs are written only from the forward perspective.  That is, although the use of the backward process was essential in discovering the proof, when we write the proof in narrative form, we use the forward process described in the preceding table.
\index{know-show table|)}%
  A completed proof follows.

\hbreak

\begin{theorem}\label{T:xyodd}
If  $x$  and  $y$  are odd integers, then  $x \cdot y$  is an odd integer.
\end{theorem}
\begin{myproof}
We assume that  $x$  and  $y$  are odd integers and will prove that  $x \cdot y$ is an odd integer.  Since $x$ and $y$ are odd, there exist integers  $m$  and  $n$  such that
\[
x = 2m + 1 \text{ and } y = 2n + 1.
\]
Using algebra, we obtain
\begin{align}
  x \cdot y &= \left( {2m + 1} \right)\left( {2n + 1} \right) \notag \\
   &= 4mn + 2m + 2n + 1 \notag \\
   &= 2\left( {2mn + m + n} \right) + 1. \notag
\end{align}
 Since  $m$  and  $n$  are integers and the integers are closed under addition and multiplication, we conclude that  $\left( {2mn + m + n} \right)$ is an integer.  This means that  $x \cdot y$ has been written in the form  $\left( {2q + 1} \right)$ for some integer  $q$, and hence, $x \cdot y$ is an odd integer.  Consequently, it has been proven that if  $x$  and  $y$  are odd integers, then  $x \cdot y$ is an odd integer.
\end{myproof}
\hbreak

\endinput




\subsection*{Writing Guidelines for Mathematics Proofs}
\index{writing guidelines|(}%
%In this section, the emphasis is on constructing an outline of a proof using a know-show table.  However, some proof writing will be done, and 
At the risk of oversimplification, doing mathematics can be considered to have two distinct stages.  The first stage is to convince yourself that you have solved the problem or proved a conjecture.  This stage is a creative one and is quite often how mathematics is actually done.  The second equally important stage is to convince other people that you have solved the problem or proved the conjecture.  This second stage often has little in common with the first stage in the sense that it does not really communicate the process by which you solved the problem or proved the conjecture.   However, it is an important part of the process of communicating mathematical results to a wider audience.

A \textbf{mathematical proof} is a convincing argument (within the accepted standards of the mathematical community) that a certain \index{proof}%
mathematical statement is necessarily true.  A proof generally uses deductive reasoning and logic but also contains some amount of ordinary language (such as English).  A mathematical proof that you write should convince an appropriate audience that the result you are proving is in fact true. So we do not consider a proof complete until there is a well-written proof.  So it is important to introduce some writing guidelines.  The preceding proof was written according to the following basic guidelines for writing proofs.  More writing guidelines will be given in Chapter~\ref{C:proofs}.
\begin{enumerate}
%\item \label{writing:know}%
%\textbf{Know Your Audience}. 
%
%Every writer should have a clear idea of the intended audience for a piece of writing.  In that way, the writer can give the right amount of information at the proper level of sophistication to communicate effectively.  This is especially true for mathematical writing.  For example, if a mathematician is writing a solution to a textbook problem for a solutions manual for instructors, the writing would be brief with many details omitted.  However, if the writing was for a students' solution manual, more details would be included.  %This is why the instructions for Beginning Activity~\ref{PA:equation} stated that your descriptions should be written for someone who already knows basic algebra and how to solve quadratic equations.


\item \textbf{Begin with a carefully worded statement of the theorem or result to be proven.}
This should be a simple declarative statement of the theorem or result.  Do not simply rewrite the problem as stated in the textbook or given on a handout.  Problems often begin with phrases such as ``Show that'' or ``Prove that.''  This should be reworded as a simple declarative statement of the theorem.  Then skip a line and write ``Proof''  in italics or boldface font (when using a word processor).  Begin the proof on the same line.  Make sure that all paragraphs can be easily identified.  Skipping a line between paragraphs or indenting each paragraph can accomplish this.

As an example, an exercise in a text might read, ``Prove that if $x$  is an odd integer, then $x^2$ is an odd integer.''  This could be started as follows:

\textbf{Theorem.} 
If  $x$  is an odd integer, then $x^2$ is an odd integer.

\textbf{\emph{Proof}}:  We assume that  $x$  is an odd integer  $\ldots$

\item \textbf{Begin the proof with a statement of your assumptions.}
Follow the statement of your assumptions with a statement of what you will prove.

\noindent
\textbf{Theorem.} 
If  $x$  is an odd integer, then $x^2$ is an odd integer.

%\begin{flushleft}
\noindent
\emph{\textbf{Proof}}.  We assume that  $x$  is an odd integer and will prove that $x^2$   is an odd integer.
%\end{flushleft}

\item \textbf{Use the pronoun ``we.''}
If a pronoun is used in a proof, the usual convention is to use ``we'' instead of ``I.''  The idea is to stress that you and the reader are doing the mathematics together.  It will help encourage the reader to continue working through the mathematics.  Notice that we started the proof of Theorem~\ref{T:xyodd} with ``We assume that $\ldots$ .''

%If a pronoun is used in a proof, the usual convention is to use ``we'' instead of ``I.''  The idea is that the author and the reader are proving the theorem together.



\item \textbf{Use italics for variables when using a word processor.}
When using a word processor to write mathematics, the word processor needs to be capable of producing the appropriate mathematical symbols and equations.  The mathematics that is written with a word processor should look like typeset mathematics.  This means that italics is used for variables, boldface font is used for vectors, and regular font is used for mathematical terms such as the names of the trigonometric and logarithmic functions.  

For example, we do not write sin (x) or \emph{sin (x)}.  The proper way to typeset this is $\sin (x)$.



%\item \textbf{Do not use $*$ for multiplication or \^{} for exponents.} \\
%Leave this type of notation for writing computer code.  The use of this notation makes it difficult for humans to read.  In addition, avoid using $/$ for division when using a complex fraction.  
%
%For example, it is very difficult to read 
%$\left(x^3 -3x^2 + 1/2 \right)/\left(2x/3 - 7\right)$; the fraction
%\[
%\frac{x^3 - 3x^2 +\dfrac{1}{2}}{\dfrac{2x}{3} - 7}
%\]
%is much easier to read.


%\item \textbf{Use complete sentences and proper paragraph structure.}
%
%Good grammar is an important part of any writing.  Therefore, conform to the accepted rules of grammar.  Pay careful attention to the structure of sentences.  Write proofs using \textbf{complete sentences} but avoid run-on sentences.  Also, do not forget punctuation, and always use a spell checker when using a word processor.


\item \textbf{Display important equations and mathematical expressions.}
Equations and manipulations are often an integral part of mathematical exposition.  Do not write equations, algebraic manipulations, or formulas in one column with reasons given in another column. 
%(as is often done in geometry texts).
   Important equations and manipulations should be displayed.  This means that they should be centered with blank lines before and after the equation or manipulations, and if the left side of the equations does not change, it should not be repeated.  For example,

Using algebra, we obtain	
\begin{align}
  x \cdot y &= \left( {2m + 1} \right)\left( {2n + 1} \right)  \notag \\ 
            &= 4mn + 2m + 2n + 1  \notag \\ 
            &= 2\left( {2mn + m + n} \right) + 1.  \notag  
\end{align} 
Since  $m$  and  $n$  are integers, we conclude that $ \ldots $ .

%\item \textbf{Do not use a mathematical symbol at the beginning of a sentence.}
%For example, we should not write, ``Let $n$ be an integer.  $n$ is an odd integer provided that \ldots''  Many people find this hard to read and often have to re-read it to understand it.  It would be better to write, ``An integer $n$ is an odd integer provided that \ldots''


\item \textbf{Tell the reader when the proof has been completed.}
Perhaps the best way to do this is to simply write, ``This completes the proof.''  Although it may seem repetitive, a good alternative is to finish a proof with a sentence that states precisely what has been proven.  In any case, it is usually good practice to use  some ``end of proof symbol'' such as  $\blacksquare$.
\index{writing guidelines|)}%


%\item \textbf{Keep it simple}.
%
%It is often difficult to understand a mathematical argument no matter how well it is written.  Do not let your writing help make it more difficult for the reader.  Use simple, declarative sentences and short paragraphs, each with a simple point.
\end{enumerate}
\hbreak

\begin{prog}[\textbf{Proving Propositions}] \label{prog:proving} \hfill \\
Construct a know-show table for each of the following propositions and then write a formal proof for one of the propositions.
\begin{enumerate}
  \item If $x$ is an even integer and $y$ is an even integer, then $x + y$ is an even integer.
  \item If $x$ is an even integer and $y$ is an odd integer, then $x + y$ is an odd integer.
  \item If $x$ is an odd integer and $y$ is an odd integer, then $x + y$ is an even integer.
\end{enumerate}
\end{prog}
\hbreak



\endinput




\subsection*{Some Comments about Constructing Direct Proofs}
\index{direct proof|(}%
\index{proof!direct|(}%
\begin{enumerate}
  \item When we constructed the know-show table prior to writing a proof for Theorem~\ref{T:xyodd}, we had only one answer for the backward question and one answer for the forward question.  Often, there can be more than one answer for these questions.  For example, consider the following statement:
\begin{center}
If  $x$  is an odd integer, then  $x^2$ is an odd integer.
\end{center}
The backward question for this could be, ``How do I prove that an integer is an odd integer?''  One way to answer this is to use the definition of an odd integer, but another way is to use the result of 
Theorem~\ref{T:xyodd}.  That is, we can prove an integer is odd by proving that it is a product of two odd integers.

The difficulty then is deciding which answer to use.  Sometimes we can tell by carefully watching the interplay between the forward process and the backward process.  Other times, we may have to work with more than one possible answer.  
\label{proofcomment1}%
%\hrulefill

%\begin{prog}[Constructing a Know-Show Table]\label{pr:kstable1} \hfill \\
%Construct a know-show table for the following statement that uses the result of 
%Theorem~\ref{T:xyodd}:
%
%\begin{list}{}
%\item If  $x$  is an odd integer, then  $x^2$ is an odd integer.
%\end{list}
%%\hrulefill
%\end{prog}
%
\item Sometimes we can use previously proven results to answer a forward question or a backward question.  This was the case in the example given in 
Comment~(\ref{proofcomment1}), where Theorem~\ref{T:xyodd} was used to answer a backward question.

\item Although we start with two separate processes (forward and backward), the key to constructing a proof is to find a way to link these two processes.  This can be difficult.  One way to proceed is to use the know portion of the table to motivate answers to backward questions and to use the show portion of the table to motivate answers to forward questions.

\item Answering a backward question can sometimes be tricky.  If the goal is the statement  $Q$, we must construct the know-show table so that if we know that  $Q$1 is true, then we can conclude that $Q$ is true.  It is sometimes easy to answer this in a way that if it is known that  $Q$ is true, then we can conclude that $Q$1 is true.  For example, suppose the goal is to prove 
\[
y^2  = 4,
\]
where  $y$  is a real number.  A backward question could be, ``How do we prove the square of a real number equals four?''  One possible answer is to prove that the real number equals 2.  Another way is to prove that the real number equals $-2$.  This is an appropriate backward question, and these are appropriate answers.

However, if the goal is to prove
\[
y = 2,
\]
where  $y$  is a real number, we could ask, ``How do we prove a real number equals 2?''  It is not appropriate to answer this question with ``prove that the square of the real number equals 4.''  
%That is, we should not have the show portion of the table as follows:
%$$
%\BeginTable
%\BeginFormat
%|p(0.4in)|p(1.6in)|p(1.6in)|
%\EndFormat
%\_
%  | $Q1$  |   $y^2=4$             |           |  \\ \_1
%  | $Q$   |  $y=2$                |  Square root of both sides | \\ \_
%  |\textbf{Step}  |  \textbf{Show}  |  \textbf{Reason} | \\+20 \_
%\EndTable
%$$
%\begin{center}
%\begin{tabular}[h]{|p{0.4in}|p{1.6in}|p{1.6in}|}
%  \hline
%  $Q1$  &   $y^2=4$             &             \\ \hline
%  $Q$  &  $y=2$  &  Square root of both sides \\ \hline
%  \textbf{Step}  &  \textbf{Show}  &  \textbf{Reason} \\ \hline
%\end{tabular}
%\end{center}
This is because if $y^2=4$, then it is not necessarily true that $y=2$.

\item Finally, it is very important to realize that not every proof can be constructed by the use of a simple know-show table.  Proofs will get more complicated than the ones that are in this section.  The main point of this section is not the know-show table itself, but the way of thinking about a proof that is indicated by a know-show table.  In most proofs, it is very important to specify carefully what it is that is being assumed and what it is that we are trying to prove.  The process of asking the ``backward questions'' and the ``forward questions'' is the important part of the know-show table.  It is very important to get into the ``habit of mind'' of working backward from what it is we are trying to prove and working forward from what it is we are assuming.  Instead of immediately trying to write a complete proof, we need to stop, think, and ask questions such as

\begin{itemize}
\item Just exactly what is it that I am trying to prove?
\item How can I prove this?
\item What methods do I have that may allow me to prove this?
\item What are the assumptions?
\item How can I use these assumptions to prove the result?
\end{itemize}
\index{direct proof|)}%
\index{proof!direct|)}%

\end{enumerate}
%\hrule




\begin{prog}[\textbf{Exploring a Proposition}]\label{A:kstable2} \hfill \\
Construct a table of values for  $\left( {3m^2  + 4m + 6} \right)$
 using at least six different integers for  $m$.  Make one-half of the values for  $m$  even integers and the other half odd integers.  Is the following proposition true or false?  

\begin{center}
If $m$ is an odd integer, then $\left(3m^2 + 4m + 6 \right)$ is an odd integer.
\end{center}
Justify your conclusion.  This means that if the proposition is true, then you should write a proof of the proposition.  If the proposition is false, you need to provide an example of an odd integer for which $\left(3m^2 + 4m + 6 \right)$ is an even integer.
\end{prog}
\hbreak


\begin{prog}[\textbf{Constructing and Writing a Proof}] \label{pr:pythag} \hfill \\
The \textbf{Pythagorean Theorem}
\index{Pythagorean Theorem}%
for right triangles states that if $a$ and $b$ are the lengths of the legs of a right triangle and $c$ is the length of the hypotenuse, then $a^2 + b^2 = c^2$.  For example, if $a = 5$ and $b = 12$ are the lengths of the two sides of a right triangle and if $c$ is the length of the hypotenuse, then the $c^2 = 5^2 + 12^2$ and so 
$c^2 = 169$.  Since $c$ is a length and must be positive, we conclude that $c = 13$.

\newpar
Construct and provide a well-written proof for the following proposition.

\newpar
\textbf{Proposition}.  If $m$ is a real number and $m$, $m + 1$, and $m + 2$ are the lengths of the three sides of a right triangle, then $m = 3$.

\newpar
Although this proposition uses different mathematical concepts than the one used in this section, the process of constructing a proof for this proposition is the same forward-backward method that was used to construct a proof for Theorem~\ref{T:xyodd}.  However, the backward question, ``How do we prove that $m = 3$?'' is simple but may be difficult to answer.  The basic idea is to develop an equation from the forward process and show that $m = 3$ is a solution of that equation.
\end{prog}
\hbreak


\endinput


















\chapter*{Chapter~\ref{C:logic} \\Logical Reasoning}

\section*{Main Objectives}
After this chapter, students should:
\begin{itemize}
\item Have a thorough understanding of the logical operations of negation, conjunction, and disjunction.
\item Have a thorough understanding of conditional statements in mathematics.
\item Have a thorough understanding of the use of quantifiers in mathematics.
\item Know some basic logical equivalencies, especially those used later in the text for methods of proof.
\item Be able to establish logical equivalencies.
\item Be able to write negations of statements with quantifiers.
\end{itemize}
%\begin{itemize}
%\item Provide students with a thorough understanding of the logical operations of negation, conjunction, and disjunction.
%\item Provide students with a thorough understanding of conditional statements in mathematics.
%\item Provide students with a thorough understanding of the use of quantifiers in mathematics.
%\item Provide students with an understanding of logical equivalencies, especially those used later in the text for methods of proof.
%\item Provide students with the ability to establish logical equivalencies.
%\end{itemize}
\hbreak
\section{Statements and Logical Operators}\label{S:logop}
\setcounter{previewactivity}{0}
%\input{focus/focus21}
%\textbf{Preview Activity 2.2 A - Truth Values of Statements} \\
\begin{previewactivity}[\textbf{Compound Statements}]\label{PA:compound} \hfill \\
Mathematicians often develop ways to construct new mathematical objects from existing mathematical objects.  It is possible to form new statements from existing statements by connecting the statements with words such as ``and'' and ``or'' or by negating the statement.  A \textbf{logical operator}
\index{logical operator}%
 (or \textbf{connective})
\index{connective}%
 on mathematical statements is a word or combination of words that combines one or more mathematical statements to make a new mathematical statement.  A \textbf{compound statement}
\index{compound statement}%
\index{statement!compound}%
 is a statement that contains one or more operators.  Because some operators are used so frequently in logic and mathematics, we give them names and use special symbols to represent them.
\begin{itemize}
  \item The \textbf{conjunction}
\index{conjunction}%
 of the statements $P$ and $Q$ is the statement ``$P$ \textbf{and} $Q$'' and its denoted by $P \wedge Q$ \label{sym:and} .  The statement $P \wedge Q$ is true only when both $P$ and $Q$ are true.
  \item The \textbf{disjunction}
\index{disjunction}%
 of the statements $P$ and $Q$ is the statement ``$P$ \textbf{or} $Q$'' and its denoted by $P \vee Q$ \label{sym:or}.  The statement $P \vee Q$ is true only when at least one of $P$ or $Q$ is true.
  \item The \textbf{negation} 
\index{negation}%
of the statement $P$ is the statement ``\textbf{not} $P$'' and is denoted by $\mynot P$ \label{sym:not}.  The negation of $P$ is true only when $P$ is false, and $\mynot P$ is false only when $P$ is true.  
  \item The \textbf{implication} 
\index{implication}%
\index{conditional}%
or \textbf{conditional} is the statement ``\textbf{If } $P$ \textbf{then} $Q$'' and is denoted by $P \to Q$ \label{sym:cond2}.  The statement $P \to Q$ is often read as ``$P$ \textbf{implies} $Q$,'' and we have seen in Section~\ref{S:prop} that $P \to Q$ is false only when $P$ is true and $Q$ is false.
\end{itemize}
\newpar
\textbf{Some comments about the disjunction}.  \\It is important to understand the use of the operator ``or.''  In mathematics, we use the \textbf{``inclusive or''}
\index{inclusive or}%
 unless stated otherwise.  This means that  $P \vee Q$ is true when both  $P$  and  $Q$  are true and also when only one of them is true.  That is, $P \vee Q$  is true when at least one of  $P$  or  $Q$  is true, or $P \vee Q$  is false only when both $P$  and  $Q$  are false.

A different use of the word ``or'' is the \textbf{``exclusive or.''}
\index{exclusive or}%
  For the exclusive or, the resulting statement is false when both statements are true. That is, ``$P$ exclusive or $Q$'' is true only when exactly one of  $P$  or  $Q$  is true.  In everyday life, we often use the exclusive or.  When someone says, ``At the intersection, turn left or go straight,'' this person is using the exclusive or.

\newpar
\textbf{Some comments about the negation}.  Although the statement, $\mynot P$, can be read as ``It is not the case that $P$,'' there are often better ways to say or write this in English.  For example, we would usually say (or write): 
\begin{itemize}
  \item The negation of the statement, ``391 is prime'' is  ``391 is not prime.''
  \item The negation of the statement, ``$12 < 9$'' is  ``$12 \geq 9$.''
\end{itemize}


\begin{enumerate}
  \item For the statements
\begin{center}
$P$: 15 is odd \qquad \qquad $Q$: 15 is prime
\end{center}
write each of the following statements as English sentences and determine whether they are true or false.  Notice that $P$ is true and $Q$ is false.
\begin{multicols}{4}
\begin{enumerate}
  \item $P \wedge Q$.
  \item $P \vee Q$.
  \item $P \wedge \mynot Q$.
  \item $\mynot P \vee \mynot Q$.
\end{enumerate}
\end{multicols}

  \item For the statements
\begin{center}
$P$:  15 is odd \qquad \qquad $R$: $15 < 17$
\end{center}
write each of the following statements in symbolic form using the operators $\wedge$, $\vee$, and 
$\mynot$.
\begin{multicols}{2}
\begin{enumerate}
  \item $15 \geq 17$.
  \item 15 is odd or $15 \geq 17$.
  \item 15 is even or $15 < 17$.
  \item 15 is odd and $15 \geq 17$.
\end{enumerate}
\end{multicols}
\end{enumerate}
\end{previewactivity}
\hbreak




\endinput
We will now learn how mathematicians and logicians create new statements from existing statements.  One common way to do this is to insert the word ``and'' or ``or'' between two existing statements $P$ and $Q$.  The \textbf{conjunction} of the statements $P$ and $Q$ is the statement
\[
P \textbf{ and } Q.
\]
Another common mathematical practice is to have a notation for new objects.  The conjunction of $P$ and $Q$ is denoted by $P \wedge Q$.
 tr

\begin{previewactivity}[\textbf{Truth Values of Statements}]\label{PA:truth} \hfill \\
We will use the following two statements for all of this activity:
\begin{itemize}
\item $P$  is the statement ``It is raining.'' 
\item $Q$  is the statement ``Daisy is playing golf.''
\end{itemize}
In each of the following four parts, a truth value will be assigned to statements $P$ and 
$Q$.  For example, in Question~(1), we will assume that each statement is true.  In Question~(2), we will assume that $P$ is true and $Q$ is false.  In each part, determine the truth value of each of the following statements:
\renewcommand{\labelenumi}{(\textbf{\alph{enumi}})}
  \begin{enumerate}
    \item \makebox[1.2in][l]{($P \wedge Q$)} It is raining and Daisy is playing golf.
    \item \makebox[1.2in][l]{($P \vee Q$)}  It is raining or Daisy is playing golf.
    \item \makebox[1.2in][l]{($P \to Q$)}	If it is raining, then Daisy is playing golf.
    \item \makebox[1.2in][l]{($\mynot P$)}	It is not raining.
    %\item \makebox[1.2in][l]{(not $P$ or $Q$)}	It is not raining or Daisy is playing golf.
    %\item \makebox[1.2in][l]{($P$ and not $Q$)}	It is raining and Daisy is not playing golf.
  \end{enumerate}
\renewcommand{\labelenumi}{\textbf{\arabic{enumi}.}}

\newpar
Which of the four statements [(a) through (d)] are true and which are false in each of the following four situations?
\begin{enumerate}
\item When $P$ is true (it is raining) and  $Q$ is true (Daisy is playing golf).  

\item When  $P$ is true (it is raining) and  $Q$ is false (Daisy is not playing golf).   

\item When  $P$ is false (it is not raining) and  $Q$ is true (Daisy is playing golf).    

\item When  $P$ is false (it is not raining) and  $Q$ is false (Daisy is not playing golf).    
\end{enumerate}
\hbreak
\end{previewactivity}




In the \typel activities for this section, we learned about compound statements and their truth values.  This information can be summarized with the following truth tables: 

\begin{center}
\begin{tabular}[t]{ c p{1in}  c }
 $$
\BeginTable
    \BeginFormat
    | c | c |
    \EndFormat
     \_6
      | $P$ \|6 $\mynot P$ |  \\+22 \_6
      | T   \|6   F   | \\ 
      | F   \|6   T   | \\ \_6 
 \EndTable
 $$
&   &  
 $$
\BeginTable
    \BeginFormat
    | c | c | c |
    \EndFormat
     \_6
      | $P$ | $Q$ \|6 $P \wedge Q$ | \\+22 \_6
      | T   |  T  \|6  T | \\ 
      | T   |  F  \|6  F | \\ 
      | F   |  T  \|6  F | \\ 
      | F   |  F  \|6  F | \\ \_6
 \EndTable
 $$
\end{tabular}
\end{center}

%\begin{center}
%\begin{tabular}{|c||c| p{1in} |c|c||c|}
%     \cline{1-2} \cline{4-6}
%      $P$ & $\mynot P$ & & $P$ & $Q$ & $P \wedge Q$\\ \cline{1-2} \cline{4-6}
%       T  &  F  &   &  T  & T &  T \\ \cline{1-2} \cline{4-6}
%       F  &  T  &   &  T  & F  & F  \\ \cline{1-2} \cline{4-6}
%      \multicolumn{3}{c |}{ } &  F  & T  & F  \\ \cline{4-6}
%      \multicolumn{3}{c |}{ } &  F  &  F  & F \\ \cline{4-6}
%\end{tabular}
%\end{center}

\begin{center}
\begin{tabular}[t]{ c p{0.5in}  c }
 $$
\BeginTable
    \BeginFormat
    | c | c | c |
    \EndFormat
     \_6
      | $P$ | $Q$ \|6 $P \vee Q$ | \\+22 \_6
      | T   |  T  \|6  T | \\ 
      | T   |  F  \|6  T | \\ 
      | F   |  T  \|6  T | \\ 
      | F   |  F  \|6  F | \\ \_6
 \EndTable
 $$
&   &  
 $$
\BeginTable
    \BeginFormat
    | c | c | c |
    \EndFormat
     \_6
      | $P$ | $Q$ \|6 $P \to Q$ | \\+22 \_6
      | T   |  T  \|6  T | \\ 
      | T   |  F  \|6  F | \\ 
      | F   |  T  \|6  T | \\ 
      | F   |  F  \|6  T | \\ \_6
 \EndTable
 $$
\end{tabular}
\end{center}


Rather than memorizing the truth tables, for many people it is easier to remember the rules summarized in Table~\ref{T:truthvalues}.



\begin{table}[h]
$$\BeginTable
\def\C{\JustCenter}
\BeginFormat
| l | c | p(2.4in) |
\EndFormat
  \hline
  | \textbf{Operator}  |   \textbf{Symbolic Form}  |  \textbf{Summary of Truth Values} | \\+22 \hline
  | Conjunction  |     $P \wedge Q$  | True only when both $P$ and $Q$ are true   | \\ \hline
  | Disjunction  |     $P \vee Q$    | False only when both $P$ and $Q$ are false |\\ \hline
  | Negation     |    $\neg  P$      | Opposite truth value of $P$                | \\  \hline
  | Conditional  |     $P \to Q$     | False only when $P$ is true and $Q$ is false | \\ \hline
\EndTable$$
\caption{Truth Values for Common Connectives}
\label{T:truthvalues}%
\index{conjunction}%
\index{disjunction}%
\index{negation}%
\end{table}


\endinput

\subsection*{Other Forms of Conditional Statements}
Conditional statements
\index{conditional statement}%
\index{conditional statement!forms of}%
 are extremely important in mathematics because almost all mathematical theorems are (or can be) stated as a conditional statement in the following form:
\begin{center}
If ``certain conditions are met,'' then ``something happens.''
\end{center}
It is imperative that all students studying mathematics thoroughly understand the meaning of a conditional statement and the truth table for a conditional statement.

We also need to be aware that in the English language, there are other ways for expressing the conditional statement $P \to Q$ other than  ``If  $P$, then  $Q$.''   
Following are some common ways to express the conditional statement  $P \to Q$ in the English language:

\begin{multicols}{2}
\begin{itemize}
  \item If  $P$, then  $Q$.
  \item $P$  implies  $Q$.
  \item $P$  only if  $Q$.
\index{only if}%
  \item $Q$  if  $P$.
  \item Whenever  $P$  is true,  $Q$  is true.
  \item $Q$  is true whenever  $P$  is true.
\end{itemize}
\end{multicols}

\begin{itemize}
  \item $Q$  is necessary
\index{necessary condition}%
 for  $P$.  
(This means that if  $P$  is true, then  $Q$  is necessarily true.)
  \item $P$  is sufficient for  $Q$.  
(This means that if you want  $Q$  to be true, it is sufficient
\index{sufficient condition}%
 to show that  $P$  is true.)
  %\item If  $P$  is true, then  $Q$  is also true.
\end{itemize}

In all of these cases,  $P$  is the \textbf{hypothesis}
\index{conditional statement!hypothesis}%
 of the conditional statement and  $Q$ is the \textbf{conclusion}
\index{conditional statement!conclusion}%
 of the conditional statement.
\hbreak

\begin{prog}[\textbf{The ``Only If'' Statement}]\label{pr:onlyif} \hfill \\
Recall that a quadrilateral is a four-sided polygon.  Let  $S$  represent the following true conditional statement:
\begin{center}
If a quadrilateral is a square, then it is a rectangle.
\end{center}

\noindent
Write this conditional statement in English using

\begin{multicols}{2}
\begin{enumerate}
\item the word ``whenever''

\item the phrase ``only if''

\item the phrase ``is necessary for''

\item the phrase ``is sufficient for''
\end{enumerate}
\end{multicols}

%Now let  $T$  represent the following statement:
%\begin{center}
%A quadrilateral is a square only if it is a rectangle.
%\end{center}
%\begin{enumerate}
%  \item Is  $T$  a true statement?  Explain your reasoning.
%  \item Let  $P$  represent ``The quadrilateral is a square,'' and let  $Q$  represent ``The quadrilateral is a rectangle.''  Using  $P$  and  $Q$  and logical operators, write symbolic expressions for statements  $S$  and $T$.
%\end{enumerate}
\end{prog}
\hbreak


\endinput

\subsection*{Constructing Truth Tables}
Truth tables for compound statements can be constructed by using the truth tables for the basic connectives.  To illustrate this, we will construct a truth table for  
$\left( {P \wedge \mynot  Q} \right) \to R$.  The first step is to determine the number of rows needed.
\begin{itemize}
  \item For a truth table with two different simple statements, four rows are needed  since there are four different combinations of truth values for the two statements.  We should be consistent with how we set up the rows.  The way we will do it in this text is to label the rows for the first statement with (T, T, F, F) and the rows for the second statement with (T, F, T, F).  All truth tables in the text have this scheme.

  \item For a truth table with three different simple statements, eight rows are needed  since there are eight different combinations of truth values for the three statements.  Our standard scheme for this type of truth table is shown in 
Table~\ref{Ta:compoundtruthtable}.
\end{itemize}


The next step is to determine the columns to be used.  One way to do this is to work backward from the form of the given statement.  For $\left( {P \wedge \mynot  Q} \right) \to R$, the last step is to deal with the conditional operator $\left(  \to  \right)$.  To do this, we need to know the truth values of  $\left( {P \wedge \mynot  Q} \right)$ and  $R$.  To determine the truth values for  $\left( {P \wedge \mynot  Q} \right)$, we need to apply the rules for the conjunction operator $\left(  \wedge  \right)$ and we need to know the truth values for  $P$  and  $\mynot  Q$.

Table~\ref{Ta:compoundtruthtable} is a  completed truth table for  
$\left( {P \wedge \mynot  Q} \right) \to R$ with the step numbers indicated at the bottom of each column.  The step numbers correspond to the order in which the columns were completed.

\begin{table}[h]
$$
\BeginTable
\BeginFormat
| c | c | c | c | c | c |
\EndFormat
\_6
       | $P$  |  $Q$  |  $R$  \|6  $\mynot  Q$  |  $P \wedge \mynot  Q$  |  $\left( P \wedge \mynot  Q \right) \to R$ | \\+22 \_6
          | T | T | T \|6 F | F | T  | \\ 
         | T | T | F \|6 F | F | T  | \\ 
          | T | F | T \|6 T | T | T  | \\ 
           | T | F | F \|6 T | T | F  | \\ 
          | F | T | T \|6 F | F | T  | \\ 
          | F | T | F \|6 F | F | T  | \\ 
          | F | F | T \|6 T | F | T  | \\ 
          | F | F | F \|6 T | F | T |  \\ \_6
 | 1 | 1 | 1 \|6 2 | 3 | 4 |   \\ \_6
\EndTable
$$
\caption{Truth Table for $\left( P \wedge \mynot  Q \right) \to R$}
\label{Ta:compoundtruthtable}%
\end{table}
%\begin{table}[h]
%$$
%\BeginTable
%\BeginFormat
%| l | c | c | c | c | c | c |
%\EndFormat
%"  & \use6 \=6 & \\0
%    "    | $P$  |  $Q$  |  $R$  |  $\mynot  Q$  |  $P \wedge \mynot  Q$  |  $\left( P \wedge \mynot  Q \right) \to R$ | \\+22 
%"  & \use6 \=6 &  \\0
%   "        | T | T | T | F | F | T  | \\ 
%   "        | T | T | F | F | F | T  | \\ 
%   "        | T | F | T | T | T | T  | \\ 
%   "        | T | F | F | T | T | F  | \\ 
%   "        | F | T | T | F | F | T  | \\ 
%   "        | F | T | F | F | F | T  | \\ 
%   "        | F | F | T | T | F | T  | \\ 
%   "        | F | F | F | T | F | T |  \\ \_6
% "Step No. | 1 | 1 | 1 | 2 | 3 | 4 |   \\ \_6
%\EndTable
%$$
%\caption{Truth Table for $\left( P \wedge \mynot  Q \right) \to R$}
%\label{Ta:compoundtruthtable}%
%\end{table}

\begin{itemize}
  \item When completing the column for  $P \wedge \mynot  Q$, remember that the only time the conjunction is true is when both  $P$  and  $\mynot  Q$ are true.  
  \item When completing the column for  $\left( {P \wedge \mynot  Q} \right) \to R$, remember that the only time the conditional statement is false is when the hypothesis $\left( {P \wedge \mynot  Q} \right)$ is true and the conclusion, $R$, is false.  
\end{itemize}
The last column entered is the truth table for the statement  $\left( {P \wedge \mynot  Q} \right) \to R$ using the setup in the first three columns.
%hbreak
%
\begin{prog}[\textbf{Constructing Truth Tables}]\label{pr:truthtables} \hfill \\
Construct a truth table for each of the following statements: 
\label{exer:sec22-4}%
  \begin{multicols}{2}
  \begin{enumerate}
    \item $P \wedge \mynot  Q$
    \item $\mynot  \left( {P \wedge Q} \right)$
    \item $\mynot  P \wedge \mynot  Q$
    \item $\mynot  P \vee \mynot  Q$
  \end{enumerate}
  \end{multicols}
\noindent
Do any of these statements have the same truth table?
\end{prog}
\hbreak


\endinput



\subsection*{The Biconditional Statement}
\index{biconditional statement}%
Some mathematical results are stated in the form  ``$P$  if and only if  $Q$'' or ``$P$  is necessary and sufficient for  $Q$.''  An example would be, ``A triangle is equilateral if and only if its three interior angles are congruent.''
%\begin{center}
%\parbox{4in}{A triangle is equilateral if and only if its three interior angles are congruent.}
%\end{center}
The symbolic form for the biconditional statement  ``$P$  if and only if  $Q$''   is  $P \leftrightarrow Q$. 
\label{sym:bicond}%
  In order to determine a truth table for a biconditional statement, it is instructive to look carefully at the form of the phrase  ``$P$  if and only if  $Q$.''  The word ``and'' suggests that this statement is a conjunction.  Actually it is a conjunction of the statements 
``$P$ if $Q$'' and ``$P$ only if $Q$.''  
The symbolic form of this conjunction is  $\left[ {\left( {Q \to P} \right) \wedge \left( {P \to Q} \right)} \right]$.
%\hbreak
%\enlargethispage{\baselineskip}
\begin{prog}[\textbf{The Truth Table for the Biconditional Statement}]\label{pr:biconditional} \hfill \\
Complete a truth table for $\left[ {\left( {Q \to P} \right) \wedge \left( {P \to Q} \right)} \right]$.  Use the following columns:  $P$, $Q$, $Q \to P$, $P \to Q$, and $\left[ {\left( {Q \to P} \right) \wedge \left( {P \to Q} \right)} \right]$.  The last column of this table will be the truth table for $P \leftrightarrow Q$.
\end{prog}
\vskip6pt
%\hrule

\subsection*{Other Forms of the Biconditional Statement}
\index{biconditional statement!forms of}%
As with the conditional statement, there are some common ways to express the biconditional statement,  $P \leftrightarrow Q$, in the English language. For example,
\begin{multicols}{2}
\begin{itemize}
  \item $P$  if and only if  $Q$.
  \item $P$ implies  $Q$  and  $Q$  implies  $P$.
  \item $P$ is necessary and sufficient for  $Q$.
  %\item $P$  is equivalent to  $Q$.
\end{itemize}
\end{multicols}
\hbreak
%\index{biconditional statement|)}%
%\pagebreak


\endinput

\subsection*{Tautologies and Contradictions}
\begin{defbox}{D:tautology}{A \textbf{tautology}
\index{tautology}%
 is a compound statement $S$ that is true for all possible combinations of truth values of the component statements that are part of $S$.  A \textbf{contradiction}
\index{contradiction}%
 is a compound statement that is false for all possible combinations of truth values of the component statements that are part of $S$.}
\end{defbox}
That is, a tautology is necessarily true in all circumstances, and a contradiction is necessarily false in all circumstances.

\begin{prog}[\textbf{Tautologies and Contradictions}] \label{pr:tautology} \hfill 
For statements $P$ and $Q$:
\begin{enumerate}
\item Use  a truth table to show that  $\left( {P \vee \mynot  P} \right)$ is a tautology.
\item Use a truth table to show that  $\left( {P \wedge \mynot  P} \right)$ is a contradiction.
\item Use a truth table to determine if $P \to (P \vee Q)$ is a tautology, a contradiction, or neither.
\end{enumerate}
\end{prog}
\hbreak

\endinput



\endinput






One way to formally define the biconditional statement is to simply define its truth table to be the following:

\begin{table}[h]
 \begin{center}
    \begin{tabular}{|c|c ||c|}
     \hline
      $P$ & $Q$ & $P \leftrightarrow Q$\\ \hline
       T  &  T  & T \\ \hline
       T  &  F  & F \\ \hline
       F  &  T  & F \\ \hline
       F  &  F  & T \\ \hline
     \end{tabular}
     \caption{Truth Table for $P \leftrightarrow $Q}
     \label{Ta:bicond}
  \end{center}
\end{table}


\begin{activity}[Working with Conditional Statements]\label{A:working}
Complete \\the following table:
\begin{center}
\begin{tabular}{|l|c|c|c|}
  \hline
  \textbf{English Form}  &  \textbf{Hypothesis}  &  \textbf{Conclusion} &  \textbf{Symbolic Form} \\ \hline
  If $P$, then $Q$.          &  $P$  &  $Q$  &  $P \to Q$  \\ \hline
  $Q$ only if $P$.           &  $Q$  &  $P$  &  $Q \to P$  \\ \hline
  $P$ is necessary for $Q$.  &       &       &             \\ \hline
  $P$ is sufficient for $Q$. &       &       &             \\ \hline
  $Q$ is necessary for $P$.  &       &       &             \\ \hline
  $P$ implies $Q$.           &       &       &             \\ \hline
  $P$ only if $Q$.           &       &       &             \\ \hline
  $P$ if $Q$.                &       &       &             \\ \hline
  If $Q$ then $P$.           &       &       &             \\ \hline
  If  $\mynot  Q$, then $\mynot  P$. &  &   &             \\ \hline
  If $P$, then $Q \wedge R$. &       &       &             \\ \hline
  If $P \vee Q$, then $R$.   &       &       &             \\ \hline
\end{tabular}
\end{center}
\end{activity}
\hbreak

\section*{Section~\ref{S:logequiv} Logically Equivalent Statements}
Plan about one class period for this section.

\subsection*{Main Topics}
Logically equivalent statements, converse and contrapositive, De Morgan's Laws, logical equivalencies related to conditional statements, the negation of a conditional statement.


\subsection*{The Preview Activities}
\subsubsection{Preview Activity~\ref{PA:logequiv} (Logically Equivalent Statements)}  This Preview Activity contains a new definition (logically equivalent statements), but students should be able to understand this.  In Part~(2) of the Preview Activity, students will verify one of De Morgan's Laws $\left( \mynot \left( P \wedge Q \right) \equiv \mynot P \vee \mynot Q \right)$.  This is part of Theorem~\ref{T:demorgan} in the section.  

\subsubsection*{Preview Activity~\ref{PA:converse} (Converse and Contrapositive)}
The converse and contrapositive of a conditional statement are defined.  Students are asked to complete truth tables to show that the contrapositive is logically equivalent to the original conditional statement but the converse is not.  Students must understand this before they go on to the next section.  The logical equivalence of a conditional statement and its contrapositive is an important idea that will be used when proof methods are studied in Chapter~\ref{C:proofs}.

%\subsubsection*{Preview Activity~\ref{PA:conditional2} (Conditional Statements)}
%The purpose of this Preview Activity is to show that a conditional statement is logically equivalent to a disjunction, and then to begin work with the negation of a conditional statement.  Students may have a difficult time with this.  However, this is very important as it forms the logical basis for a proof by contradiction.  Examples~\ref{E:conditionalasor} and~\ref{E:negationofcond} are related to this Preview Activity.
\hbreak


\subsection*{The Exercises}

Each of the exercises is a straightforward application of the material in the section.  The first three exercises should be assigned so that students get plenty of practice forming negations of statements.  Exercise~(\ref{exer:sec23-biconda}) should be assigned as it will be used to justify a proof technique in Chapter~\ref{C:proofs}.  Assign at least one of the parts in 
Exercise~(\ref{exer:sec22-distrib}).  Exercise~(\ref{exer:sec23-6}), 
%(\ref{exer:sec23-7}), and~(\ref{exer:sec23-9}) 
is strongly recommended as it forms the basis for a proof technique in Chapter~\ref{C:proofs} (Proof using cases).  Exercises~(\ref{exer:diffimpliescont}) and~(\ref{exer:sec23-10}) are good exercises as they ask students to work with the logical equivalencies with actual conditional statements from mathematics.   Parts of exercise~(\ref{exer:sec23-8}) should be assigned if you are interested in having students be able to establish logical equivalencies without using truth tables.

\vskip6pt
\noindent
Typical Assignment:  Exercises 1, 2, 3, 4(a), 5(a), 6, 7(a), 8, 9(a, b, c), 10 or 11
\hbreak


\subsection*{Explorations and Activities}

The short activity in exercise~(\ref{A:workingeq}) is a useful activity.  It provides an opportunity for the students to rewrite a conditional statement into an equivalent conditional statement using some of the standard logical equivalencies.  The students will actually use this method to prove such statements in Chapter~\ref{C:proofs}.

%\subsubsection*{Activity~\ref{A:workingeq2}}
%This activity provides practice at constructing a truth table and then using truth tables to conlude that two statements are logically equivalent.  The activity ends by using this logical equivalency in a context that will be seen later in the text.
\hbreak


\endinput

\section*{Section~\ref{S:predicates} Open Sentences and Sets}
Plan about one class period for this section.

\subsection*{Main Topics}
Basic set notation including set-builder notation, variables and open sentences, the truth set of an open sentence, and set builder notation.

\subsection*{The Preview Activities}
\subsubsection*{Preview Activity~\ref{PA:sets} (Sets and Set Notation)} 
The purpose of this preview activity is simply to introduce students to an intuitive idea of how mathematicians use sets and the use of the roster method to designate the elements of a set.

\subsubsection*{Preview Activity~\ref{PA:variable} (Variables)}  
The concept of a \textbf{universal set} is introduced.  The purpose of problems two through seven is to provide a ``lead-in'' to the concept of a \textbf{truth set} introduced in the section.  This is also done in Progress Check~\ref{pr:predicates}.
\hbreak

%\subsubsection*{Activity~\ref{A:predicates}} This comes before the definition of a truth set. If students have read this section before class or if Preview Activity~\ref{PA:variables} seems to provide a sufficient introduction, then this activity can be skipped.

%\subsubsection*{Activity~\ref{A:truthset}}  This comes right after the definition of truth set and provides practice with working with this definition.  If students do not work on this activity, then it should be used to provide examples in class.  (Or the instructor can use other examples.
%\hbreak
%
\subsection*{The Exercises}

It is a good idea to assign most of the exercises in this section.  Exercise~(\ref{exer:sec21-3}) and~(\ref{exer:sec23-sets}) are needed to provide practice in using set-builder notation.  %Exercise~(\ref{exer:sec21-4}) is a good exercise since it makes the students distinguish between statements and open sentences.  Exercise~(\ref{Exer:quantifier}) is also good for making the distinction between statements and predicates.

\vskip6pt
\noindent
Typical Assignment:  Exercises 1(a, b, d, e), 2, 3, 4, 5, 6
\hbreak


\subsection*{Explorations and Activities}
This activity is intended to give students a better understanding of closure for a set with respect to an operation.  It is provides an opportunity to work with a universally quantified conditional statement and counterexamples for such statements.
\hbreak


\endinput

\section{Quantifiers and Negations}\label{S:quantifier}
%\markboth{Chapter \ref{C:logic}. Logical Reasoning}{\ref{S:quantifier}. Quantifiers and Negations}
%
\setcounter{previewactivity}{0}
%\hbreak
\begin{previewactivity}[\textbf{An Introduction to Quantifiers}]\label{PA:quantifier} \hfill \\
\index{quantifier}%
We have seen that one way to create a statement from an open sentence is to substitute a specific element from the universal set for each variable in the open sentence.  Another way is to make some claim about the truth set of the open sentence.  This is often done by using a quantifier.    For example, if the universal set is  $\mathbb{R}$, then the following sentence is a statement.
\begin{center}
For each real number  $x$,  $x^2 > 0$.
\end{center}
The phrase ``For each real number  $x$'' is said to \textbf{quantify the variable} that follows it in the sense that the sentence is claiming that something is true for all real numbers.  So this sentence is a statement (which happens to be false).
%
\begin{defbox}{D:every}{The phrase ``for every'' (or its equivalents) is called a \textbf{universal quantifier}.
\index{universal quantifier}%
\index{quantifier!universal}%
  The phrase ``there exists'' (or its equivalents) is called an \textbf{existential quantifier}.
\index{existential quantifier}%
\index{quantifier!existential}%
  The symbol $\forall$ 
\label{sym:forall}%
 is used to denote a universal quantifier, and the symbol  $\exists $ 
\label{sym:exist}%
 is used to denote an existential quantifier.}
\end{defbox}
Using this notation, the statement ``For each real number  $x$,  $x^2 > 0$'' could be written in symbolic form as: $\left( {\forall x \in \mathbb{R}} \right)\left( {x^2 > 0} \right)$.
%\[
%\left( {\forall x \in \mathbb{R}} \right)\left( {x^2 > 0} \right).
%\]
The following is an example of a statement involving an existential quantifier.
\begin{center}
There exists an integer $x$ such that  $3x - 2 = 0$.
\end{center}
This could be written in symbolic form as
\[
\left( {\exists x \in \Z} \right) \left( 3x - 2 = 0 \right).
\]
This statement is false because there are no integers that are solutions of the linear equation $3x - 2 = 0$.
Table~\ref{T:quantifiers} summarizes the facts about the two types of quantifiers.

\begin{table}[!h]
$$
\BeginTable
\BeginFormat
|p(1.2in)|p(1.5in)|p(1.5in)|
\EndFormat
\_
 | \textbf{A statement involving }  |  \textbf{Often has the form}  |  \textbf{The statement is true provided that} | \\+22 \_
 | A universal quantifier: $\left( \forall x, P(x) \right)$  |  ``For every $x$, $P(x)$,'' where $P(x)$ is a predicate.  |  Every value of $x$ in the universal set makes $P(x)$ true. | \\ \_
 | An existential quantifier: $\left( \exists x, P(x) \right)$     |   ``There exists an $x$ such that $P(x)$,'' where $P(x)$ is a predicate.                              |   There is at least one value of $x$ in the universal set that makes $P(x)$ true.   |       \\ \_
\EndTable
$$
\caption{Properties of Quantifiers}
\label{T:quantifiers}
\end{table}

In effect, the table indicates that the universally quantified statement is true provided that the truth set of the predicate equals the universal set, and the existentially quantified statement is true provided that the truth set of the predicate contains at least one element.  
%We will study quantifiers more extensively in Section~\ref{S:quantifier}.
  \item Each of the following sentences is a statement or an open sentence.  Assume that the universal set for each variable in these sentences is the set of all real numbers.  If a sentence is an open sentence (predicate), determine its truth set.  If a sentence is a statement, determine whether it is true or false. \label{exer:sec21-4}
  \begin{enumerate}
    \item $\left( \forall a \in \mathbb{R}\right) \left(a + 0 = a\right)$.
    \item $3x - 5 = 9$.
    \item $\sqrt x  \in \mathbb{R}$.
    %\item $\left( \forall x \in \mathbb{R}\right) \left( \sin( {2x}) = 2( {\sin x})( {\cos x})$\right).
    \item $\sin( {2x} ) = 2( {\sin x} )( {\cos x})$.
    \item $\left( \forall x \in \R\right) \left(\sin( {2x} ) = 2( {\sin x})( {\cos x}) \right)$.
    \item $\left( \exists x \in \R \right)\left( x^2  + 1 = 0 \right)$.
    \item $\left( \forall x \in \R \right) \left( x^3  \geq x^2 \right)$.
    \item $x^2  + 1 = 0$. 
    \item If  $x^2 \geq 1$, then  $x  \geq 1$.
    \item $\left( \forall x \in \R \right)\left( \text{If } x^2 \geq 1, \text{ then } x \geq 1 \right)$.
%    \item $\forall x \in \mathbb{R}, \exists y \in \mathbb{R}\text{ such that } x + y = 0$.
%    \item $\exists y \in \mathbb{R}\text{ such that }\forall x \in \mathbb{R}, x + y = 0$.
%    \item $\sqrt x  \in \mathbb{Z}$.
  \end{enumerate}




%\begin{enumerate}
%\item Consider the following statement written in symbolic form:\\  $\left( {\forall x \in \mathbb{Z}} \right)\left( {x\text{ is a multiple of 2}} \right)$.
%  \begin{enumerate}
%    \item Write this statement as an English sentence.
%    \item Is the statement true or false?  Why?
%    \item How would you write the negation of this statement as an English sentence?
%    \item Is it possible to write your negation of this statement from part~(2) symbolically (using a quantifier)?
%  \end{enumerate}
%%
%
%
%\item Consider the following statement written in symbolic form:\\  $\left( {\exists x \in \mathbb{Z}} \right)\left( {x^3 > 0} \right)$.
%  \begin{enumerate}
%    \item Write this statement as an English sentence.
%    \item Is the statement true or false?  Why?
%    \item How would you write the negation of this statement as an English sentence?
%    \item Is it possible to write your negation of this statement from part~(2) symbolically (using a quantifier)?
%  \end{enumerate}
%\end{enumerate}
\end{previewactivity}
\hbreak
%
\endinput

\begin{previewactivity}[\textbf{Attempting to Negate Quantified Statements}]\label{PA:negatequantifier} \hfill 
\begin{enumerate}
\item Consider the following statement written in symbolic form:\\  $\left( {\forall x \in \mathbb{Z}} \right)\left( {x\text{ is a multiple of 2}} \right)$.
  \begin{enumerate}
    \item Write this statement as an English sentence.
    \item Is the statement true or false?  Why?
    \item How would you write the negation of this statement as an English sentence?
    \item If possible, write your negation of this statement from part~(2) symbolically (using a quantifier).
  \end{enumerate}
%


\item Consider the following statement written in symbolic form:\\  $\left( {\exists x \in \mathbb{Z}} \right)\left( {x^3 > 0} \right)$.
  \begin{enumerate}
    \item Write this statement as an English sentence.
    \item Is the statement true or false?  Why?
    \item How would you write the negation of this statement as an English sentence?
    \item If possible, write your negation of this statement from part~(2) symbolically (using a quantifier).
  \end{enumerate}
\end{enumerate}
\end{previewactivity}
\hbreak


\endinput


%
We introduced the concepts of open sentences and quantifiers in Section~\ref{S:predicates}. Review the definitions given on pages~\pageref{D:universal}, \pageref{D:truthset}, 
and~\pageref{D:every}.

\subsection*{Forms of Quantified Statements in English}
There are many ways to write statements involving quantifiers in English.  In some cases, the quantifiers are not apparent, and this often happens with conditional statements.  The following examples illustrate these points.  Each example contains a quantified statement written in symbolic form followed by several ways to write the statement in English.
\begin{enumerate}
  \item $\left( {\forall x \in \mathbb{R}} \right)\left( {x^2  > 0} \right)$.
  \begin{itemize}
    %\item For any real number  $x$,  $x^2  > 0$.
    \item For each real number  $x$, $x^2  > 0$.
    \item The square of every real number is greater than 0.
    \item The square of a real number is greater than 0.
    \item If  $x \in \mathbb{R}$, then  $x^2  > 0$.
  \end{itemize}
In the second to the last example, the quantifier is not stated explicitly.  Care must be taken when reading this because it really does say the same thing as the previous examples.
The last example illustrates the fact that conditional statements often contain a ``hidden'' universal quantifier.  

If the universal set is  $\R$, then the truth set of the open sentence  $x^2  > 0$ is the set of all nonzero real numbers.  That is, the truth set is
\[
\left\{ {x \in \mathbb{R}} \mid x \ne 0 \right\}.
\]
So the preceding statements are false.  For the conditional statement, the example using  
$x = 0$ produces a true hypothesis and a false conclusion.  This is a \textbf{counterexample}
\index{counterexample}%
\label{D:counterexample}%
 that shows that the statement with a universal quantifier is false.

%\pagebreak
\item $\left( {\exists x \in \mathbb{R}} \right)\left( {x^2  = 5} \right)$.
  \begin{itemize}
    \item There exists a real number  $x$  such that  $x^2  = 5$.
    \item $x^2  = 5$ for some real number $x$.
    \item There is a real number whose square equals 5.
  \end{itemize}

The second example is usually not used since it is not considered good writing practice to start a sentence with a mathematical symbol. 

If the universal set is  $\R$, then the truth set of the predicate  ``$x^2  = 5$''  is  
$\left\{ { - \sqrt 5 ,\;\sqrt 5 } \right\}$.  So these are all true statements.
\end{enumerate}
\hbreak

\endinput


\subsection*{Negations of Quantified Statements}
\index{negation!of a quantified statement|(}%
In \typeu Activity~\ref*{PA:quantifier}, we wrote negations of some quantified statements.  This is a very important mathematical activity.  As we will see in future sections, it is sometimes just as important to be able to describe when some object does not satisfy a certain property as it is to describe when the object satisfies the property.  Our next task is to learn how to write negations of quantified statements in a useful English form.

We first look at the negation of a statement involving a universal quantifier.  The general form for such a statement can be written as
$\left( {\forall x} \in U \right)\left( {P( x )} \right)$,
where  $P( x )$ is an open sentence and $U$ is the universal set for the variable $x$.  When we write
\[
\mynot  \left( {\forall x} \in U \right)\left[ {P\left( x \right)} \right],
\]
we are asserting that the statement  $\left( {\forall x} \in U \right)\left[ {P( x )} \right]$ is false.  This is equivalent to saying that the truth set of the open sentence   
$P( x )$ is not the universal set.  That is, there exists an element  $x$  in the universal set  $U$  such that  $P( x )$ is false.  This in turn means that there exists an element  $x$  in  $U$  such that  $\mynot  P( x )$ is true,   which is equivalent to saying that  $\left( {\exists x} \in U \right)\left[ {\mynot  P( x )} \right]$ is true.  This explains why the following result is true:
%
\[
\mynot  \left( {\forall x} \in U \right)\left[ {P( x )} \right] \equiv \left( {\exists x} \in U \right)\left[ {\mynot  P( x )} \right].
\]
Similarly, when we write
\[
\mynot  \left( {\exists x} \in U \right)\left[ {P( x )} \right],
\]
we are asserting that the statement  $\left( {\exists x} \in U \right)\left[ {P( x )} \right]$ is false.  This is equivalent to saying that the truth set of the open sentence  $P( x )$ is the empty set.  That is, there is no element  $x$  in the universal set  $U$  such that  
$P( x )$ is true.  This in turn means that for each element  $x$  in  $U$, 
$\mynot  P( x )$ is true, and this is equivalent to saying that  
$\left( {\forall x} \in U \right)\left[ {\mynot  P( x )} \right]$ is true.  This explains why the following result is true: 
\[
\mynot  \left( {\exists x} \in U \right)\left[ {P( x )} \right] \equiv \left( {\forall x} \in U \right)\left[ {\mynot  P( x )} \right].
\]
We summarize these results in the following theorem.
%\hbreak
\begin{theorem}\label{T:negations}
For any open sentence  $P( x )$,
\[
\begin{aligned}
  \mynot  \left( {\forall x} \in U \right)\left[ {P( x )} \right] &\equiv \left( {\exists x} \in U \right)\left[ {\mynot  P( x )} \right]\text{, and} \\ 
  \mynot  \left( {\exists x} \in U \right)\left[ {P( x )} \right] &\equiv \left( {\forall x} \in U \right)\left[ {\mynot  P( x )} \right]. \\ 
\end{aligned}
\] 
\end{theorem}
\hbreak
%
\begin{example}[\textbf{Negations of Quantified Statements}]\label{E:negations} \hfill \\
Consider the following statement:  $\left( {\forall x \in \mathbb{R}} \right)\left( {x^3  \geq x^2 } \right)$.

We can write this statement as an English sentence in several ways.  Following are two different ways to do so.
\begin{itemize}
  \item For each real number $x$, $x^3  \geq x^2 $.
  \item If  $x$  is a real number, then  $x^3 $ is greater than or equal to  $x^2 $.
\end{itemize}
The second statement shows that in a conditional statement, there is often a hidden universal quantifier.  This statement is false since there are real numbers  $x$  for which  $x^3 $ is not greater than or equal to  $x^2 $. For example, we could use  $x =  - 1$ or  $x = \frac{1}{2}$.

%Since the phrase ``is not greater than or equal to'' means the same thing as ``is less than,'' we usually say that there are real numbers  $x$  for which  $x^3  < x^2 $. 
This means that the negation must be true.  We can form the negation as follows:
\[
\mynot  \left( {\forall x \in \mathbb{R}} \right)\left( {x^3  \geq x^2 } \right) \equiv \left( {\exists x \in \mathbb{R}} \right)\mynot  \left( {x^3  \geq x^2 } \right).
\]
In most cases, we want to write this negation in a way that does not use the negation symbol.  In this case, we can now write the open sentence $\mynot  \left( {x^3  \geq x^2 } \right)$ as  $\left( {x^3  < x^2 } \right)$.  (That is, the negation of ``is greater than or equal to'' is ``is less than.'')  So we obtain the following:
\[
\mynot  \left( {\forall x \in \mathbb{R}} \right)\left( {x^3  \geq x^2 } \right) \equiv \left( {\exists x \in \mathbb{R}} \right)\left( {x^3  < x^2 } \right).
\]
The statement $\left( {\exists x \in \mathbb{R}} \right)\left( {x^3  < x^2 } \right)$
could be written in English as follows:
\begin{itemize}
  \item There exists a real number  $x$  such that  $x^3  < x^2 $.
  \item There exists an  $x$  such that  $x$  is a real number and  $x^3  < x^2 $.
\end{itemize}
%
\end{example}
\hbreak
%
\begin{prog}[\textbf{Negating Quantified Statements}]\label{pr:negating} \hfill \\
For each of the following statements
\renewcommand{\theenumi}{\alph{enumi}}
\begin{itemize}
  \item Write the statement in the form of an English sentence that does not use the symbols for quantifiers.
  \item Write the negation of the statement in a symbolic form that does not use the negation symbol.
  \item Write the negation of the statement in the form of an English sentence that does not use the symbols for quantifiers.
\end{itemize}

%\renewcommand{\theenumi}{\arabic{enumi}}
\begin{enumerate}
  \item $\left( \forall a \in \mathbb{R}\right) \left( a + 0 = a \right)$.
  \item $\left( \forall x \in \mathbb{R} \right) \left[ \sin ( {2x} ) = 2 ( {\sin x} )( {\cos x} ) \right]$.
  \item $\left( \forall x \in \mathbb{R} \right) \left( \tan ^2 x + 1 = \sec ^2 x \right)$.
  \item $\left( \exists x \in \mathbb{Q} \right) \left( x^2  - 3x - 7 = 0 \right)$.
  \item $\left( \exists x \in \mathbb{R} \right) \left( x^2  + 1 = 0 \right)$.
\end{enumerate}
\end{prog}
\index{negation!of a quantified statement|)}%
\index{counterexample}%
\hbreak


\endinput

\subsection*{Counterexamples and Negations of Conditional Statements}
\index{negation!of a conditional statement}%
The real number  $x =  - 1$ in the previous example was used to show that the statement  
$\left( {\forall x \in \mathbb{R}} \right)\left( {x^3  \geq x^2 } \right)$ is false.  This is called a \textbf{counterexample} to the statement.  In general, a \textbf{counterexample} 
\label{D:counterexample2}% 
to a statement of the form  $\left( {\forall x} \right)\left[ {P( x )} \right]$ is an object  $a$  in the universal set  $U$  for which  $P( a )$ is false.  It is an example that proves that  $\left( {\forall x} \right)\left[ {P( x )} \right]$ is a false statement, and hence its negation, 
$\left( {\exists x} \right)\left[ {\mynot  P( x )} \right]$,  is a  true statement.

In the preceding example, we also wrote the universally quantified statement as a conditional statement.  The number  $x =  - 1$ is a counterexample for the statement 
%
\begin{center}
If  $x$  is a real number, then  $x^3 $ is greater than or equal to  $x^2 $.
\end{center}
%
So the number $-1$  is an example that makes the hypothesis of the conditional statement true and the conclusion false.  Remember that a conditional statement often contains a ``hidden'' universal quantifier.  Also, recall that in Section~\ref{S:logequiv} we saw that the negation of the conditional statement ``If $P$ then $Q$'' is the statement ``$P$ and not $Q$.''  Symbolically, this can be written as follows:
\[
\mynot  \left( {P \to Q} \right) \equiv \;P \wedge \mynot  Q.
\]
So when we specifically include the universal quantifier, the symbolic form of the negation of a conditional statement is
%
\[
\begin{aligned}
  \mynot  \left( {\forall x} \in U \right)\left[ {P( x ) \to Q( x )} \right] &\equiv \left( {\exists x} \in U \right)\mynot  \left[ {P( x ) \to Q( x )} \right] \\ 
&\equiv \left( {\exists x} \in U \right)\left[ {P( x ) \wedge \mynot  Q( x )} \right]. \\ 
\end{aligned} 
\]
%
That is,
%
\[
\mynot  \left( {\forall x} \in U \right)\left[ {P( x ) \to Q( x )} \right] \equiv \left( {\exists x} \in U \right)\left[ {P( x ) \wedge \mynot  Q( x )} \right].
\]
%
\hbreak
\begin{prog}[\textbf{Using Counterexamples}]\label{pr:counterexamples} \hfill \\
Use counterexamples to explain why each of the following statements is false.
\begin{enumerate}
\item For each integer $n$, $\left( n^2 + n + 1 \right)$ is a prime number.

\item For each real number $x$, if $x$ is positive, then $2x^2 > x$.
\end{enumerate}
\end{prog}
\hbreak

%\begin{prog}[Negating Quantified Statements] \label{pr:negating} \hfill \\
%For each of the following statements:
%\renewcommand{\theenumi}{\alph{enumi}}
%\begin{itemize}
%  \item Write the statement in the form of an English sentence that does not use the symbols for quantifiers.
%  \item Write the negation of the statement in a symbolic form that does not use the negation symbol.
%  \item Write the negation of the statement in the form of an English sentence that does not use the symbols for quantifiers.
%\end{itemize}
%
%%\renewcommand{\theenumi}{\arabic{enumi}}
%\begin{enumerate}
%  \item $\forall a \in \mathbb{R},\;a + 0 = a$.
%  \item $\forall x \in \mathbb{R},\;\sin \left( {2x} \right) = 2\left( {\sin x} \right)\left( {\cos x} \right)$.
%  \item $\forall x \in \mathbb{R},\;\tan ^2 x + 1 = \sec ^2 x$.
%  \item $\exists x \in \mathbb{Q}\mathbf{ }\text{ such that }x^2  - 3x - 7 = 0$.
%  \item $\exists x \in \mathbb{R}\mathbf{ }\text{ such that }x^2  + 1 = 0$.
%\end{enumerate}
%\end{prog}
%\hbreak


\endinput

\subsection*{Quantifiers in Definitions}
\index{quantifier}%
Definitions of terms in mathematics often involve quantifiers.  These definitions are often given in a form that does not use the symbols for quantifiers.  Not only is it important to  know a definition, it is also important to be able to write a negation of the definition.  This will be illustrated with the definition of what it means to say that a natural number is a perfect square.

%Recall that the natural numbers, denoted by  $\mathbb{N}$, consist of the positive whole numbers.  That is, $\mathbb{N} = \left\{ {1,\;2,\;3,\; \ldots } \right\}$. 
%
\begin{defbox}{D:square}{A natural number  $n$  is a \textbf{perfect square}
\index{perfect square}%
 provided that there exists a natural number  $k$  such that  $n = k^2$.}  
\end{defbox}
%
This definition can be written in symbolic form using appropriate quantifiers as follows:
\begin{center}
A natural number  $n$  is a \textbf{perfect square} provided  $\left( {\exists k \in \mathbb{N}} \right) \! \left( {n = k^2 } \right)$.
\end{center}

We frequently use the following steps to gain a better understanding of a definition.

\begin{enumerate}
  \item Examples of natural numbers that are perfect squares are 1, 4, 9, and 81 since 
$1 = 1^2$, $4 = 2^2$, $9 = 3^2$, and $81 = 9^2$.

  \item Examples of natural numbers that are not perfect squares are 2, 5, 10, and 50.

  \item This definition gives two ``conditions.''  One is that the natural number $n$ is a perfect square and the other is that there exists a natural number $k$ such that $n = k^2$.  The definition states that these mean the same thing.  So when we say that a natural number $n$ is not a perfect square, we need to negate the condition that  there exists a natural number $k$ such that $n = k^2$.  We can use the symbolic form to do this.

\[
\mynot \left( {\exists k \in \mathbb{N}} \right)\left( {n = k^2 } \right) \equiv 
\left( \forall k \in \N \right) \left( n \ne k^2 \right)
\]

Notice that instead of writing $\mynot \left(n = k^2 \right)$, we used the equivalent form of 
$\left(n \ne k^2 \right)$.  This will be easier to translate into an English sentence.  So we can write,

\begin{list}{}
\item A natural number $n$  is not a perfect square provided that for every natural number $k$, $n \ne k^2$.
\end{list}
\end{enumerate}


The preceding method illustrates a good method for trying to understand a new definition.  Most textbooks will simply define a concept and leave it to the reader to do the preceding steps.  Frequently, it is not sufficient just to read a definition and expect to understand the new term.  We must provide examples that satisfy the definition, as well as examples that do not satisfy the definition, and we must be able to write a coherent negation of the definition.
\hbreak

%\pagebreak
\begin{prog}[\textbf{Multiples of Three}]\label{pr:mutliple3} \hfill 
\begin{defbox}{D:multiple3}{An integer  $n$  is a \textbf{multiple of 3} provided that there exists an integer  $k$  such that  $n = 3k$.}  
\end{defbox}

\begin{enumerate}
  \item Write this definition in symbolic form using quantifiers by completing the following:

\begin{list}{}
\item An integer $n$ is a multiple of 3 provided that \ldots .
\end{list}
  \item Give several examples of integers (including negative integers) that are multiples of 3.
  \item Give several examples of integers (including negative integers) that are not multiples of 3.
  \item Use the symbolic form of the definition of a multiple of 3 to complete the following sentence: ``An integer $n$  is not a multiple of 3 provided that \ldots .''

  \item Without using the symbols for quantifiers, complete the following sentence:  ``An integer  $n$ is not a multiple of 3 provided that  \ldots .''
\end{enumerate}
\end{prog}
\hbreak


\endinput


\subsection*{Statements with More than One Quantifier}
When a predicate contains more than one variable, each variable must be quantified to create a statement.  For example, assume the universal set is the set of integers, $\mathbb{Z}$, and let  $P\left( {x, y} \right)$ be the predicate, ``$x + y = 0$.''  We can create a statement from this predicate in several ways.
\begin{enumerate}
  \item $\left( {\forall x \in \mathbb{Z}} \right)\left( {\forall y \in \mathbb{Z}} \right)\left( {x + y = 0} \right)$. \label{twoquantifiers1}%

We could read this as, ``For all integers  $x$  and  $y$, $x + y = 0$.''  This is a false statement since it is possible to find two integers whose sum is not zero $\left( {2 + 3 \ne 0} \right)$.

  \item $\left( {\forall x \in \mathbb{Z}} \right)\left( {\exists y \in \mathbb{Z}} \right)\left( {x + y = 0} \right)$. \label{twoquantifiers2}%

We could read this as, ``For every integer  $x$, there exists an integer  $y$  such that 
$x + y = 0$.''  This is a true statement.

  \item $\left( {\exists x \in \mathbb{Z}} \right)\left( {\forall y \in \mathbb{Z}} \right)\left( {x + y = 0} \right)$. \label{twoquantifiers3}%

We could read this as, ``There exists an integer  $x$  such that for each integer   $y$, $x + y = 0$.''  This is a false statement since there is no integer  whose sum with each integer is zero.

  \item $\left( {\exists x \in \mathbb{Z}} \right)\left( {\exists y \in \mathbb{Z}} \right)\left( {x + y = 0} \right)$.

We could read this as, ``There exist integers  $x$  and  $y$  such that \\
$x + y = 0$.''  This is a true statement.  For example, $2 + \left( { - 2} \right) = 0$.  
\end{enumerate}
%
When we negate a statement with more than one quantifier, we consider each quantifier in turn and apply the appropriate part of Theorem~\ref{T:negations}.  As an example, we will negate Statement~(\ref{twoquantifiers3}) from the preceding list.  The statement is
\[
\left( {\exists x \in \mathbb{Z}} \right)\left( {\forall y \in \mathbb{Z}} \right)\left( {x + y = 0} \right).
\]
We first treat this as a statement in the following form:  
$\left( {\exists x \in \mathbb{Z}} \right)\left( {P( x )} \right)$  where  $P( x )$ is the predicate  $\left( {\forall y \in \mathbb{Z}} \right)\left( {x + y = 0} \right)$.  Using Theorem~\ref{T:negations}, we have
\[
\mynot  \left( {\exists x \in \mathbb{Z}} \right)\left( {P( x )} \right) \equiv \left( {\forall x \in \mathbb{Z}} \right)\left( {\mynot  P( x )} \right).
\]
%
Using Theorem~\ref{T:negations} again, we obtain the following:
\[
\begin{aligned}
  \mynot  P( x ) &\equiv \mynot  \left( {\forall y \in \mathbb{Z}} \right)\left( {x + y = 0} \right) \\ 
   &\equiv \left( {\exists y \in \mathbb{Z}} \right)\mynot  \left( {x + y = 0} \right) \\ 
   &\equiv \left( {\exists y \in \mathbb{Z}} \right)\left( {x + y \ne 0} \right). \\ 
\end{aligned} 
\]
%
Combining these two results, we obtain
\[
\mynot  \left( {\exists x \in \mathbb{Z}} \right)\left( {\forall y \in \mathbb{Z}} \right)\left( {x + y = 0} \right) \equiv \left( {\forall x \in \mathbb{Z}} \right)\left( {\exists y \in \mathbb{Z}} \right)\left( {x + y \ne 0} \right).
\]
%
%This process can be written as follows:
%\[
%\begin{aligned}
%  \mynot  \left( {\exists x \in \mathbb{Z}} \right)\left( {\forall y \in \mathbb{Z}} \right)\left( {x + y = 0} \right) &\equiv \left( {\forall x \in \mathbb{Z}} \right)\left[ {\mynot  \left( {\forall y \in \mathbb{Z}} \right)\left( {x + y = 0} \right)} \right] \\ 
%   &\equiv \left( {\forall x \in \mathbb{Z}} \right)\left[ {\left( {\exists y \in \mathbb{Z}} \right)\mynot  \left( {x + y = 0} \right)} \right] \\ 
%   &\equiv \left( {\forall x \in \mathbb{Z}} \right)\left( {\exists y \in \mathbb{Z}} \right)\left( {x + y \ne 0} \right). \\ 
%\end{aligned}
%\]
%
The results are summarized in the following table.

$$
\BeginTable
\BeginFormat
|p(0.75in)|p(2in)|p(1.5in)|
\EndFormat
\_
|             |  \textbf{Symbolic Form}  |  \textbf{English Form} | \\+22 \_
|  \Lower{Statement}  |  \Lower{$\left( {\exists x \in \mathbb{Z}} \right)\left( {\forall y \in \mathbb{Z}} \right)\left( {x + y = 0} \right)$}  |  There exists an integer  $x$  such that for each integer   $y$, $x + y = 0$. | \\  \_1
|  \Lower{Negation}   |  \Lower{$\left( {\forall x \in \mathbb{Z}} \right)\left( {\exists y \in \mathbb{Z}} \right)\left( {x + y \ne 0} \right)$}  |  For each integer  $x$, there exists an integer  $y$  such that  $x + y \ne 0$. | \\  \_
\EndTable
$$

%
%\begin{center}
%\begin{tabular}[h]{|p{0.75in}|p{2in}|p{1.5in}|}
%  \hline
%             &  \textbf{Symbolic Form}  &  \textbf{English Form} \\ \hline
%  Statement  &  $\left( {\exists x \in \mathbb{Z}} \right)\left( {\forall y \in \mathbb{Z}} \right)\left( {x + y = 0} \right)$  &  There exists an integer  $x$  such that for each integer   $y$, $x + y = 0$.  \\  \hline
%  Negation   &  $\left( {\forall x \in \mathbb{Z}} \right)\left( {\exists y \in \mathbb{Z}} \right)\left( {x + y \ne 0} \right)$  &  For each integer  $x$, there exists an integer  $y$  such that  $x + y \ne 0$.  \\  \hline
%\end{tabular}
%\end{center}
%
\noindent
Since the given statement is false, its negation is true.

We can construct a similar table for each of the four statements.  The next table shows Statement~(\ref{twoquantifiers2}), which is true, and its negation, which is false.

$$
\BeginTable
\BeginFormat
|p(0.75in)|p(2in)|p(1.5in)|
\EndFormat
\_
 |            |  \textbf{Symbolic Form}  |  \textbf{English Form}  | \\+22 \_
 | \Lower{Statement}  |  \Lower{$\left( {\forall x \in \mathbb{Z}} \right)\left( {\exists y \in \mathbb{Z}} \right)\left( {x + y = 0} \right)$}  |  For every integer  $x$, there exists an integer  $y$  such that $x + y = 0$. | \\  \_1
 | \Lower{Negation}   |  \Lower{$\left( {\exists x \in \mathbb{Z}} \right)\left( {\forall y \in \mathbb{Z}} \right)\left( {x + y \ne 0} \right)$}  |  There exists an integer  $x$  such that for every integer  $y$,  $x + y \ne 0$. | \\  \_
\EndTable
$$
%Since the given statement is true, its negation is false.
%\hbreak

\begin{prog}[\textbf{Negating a Statement with Two Quantifiers}]\label{pr:twoquant} \hfill \\ 
Write the negation of the statement
\[
\left( {\forall x \in \mathbb{Z}} \right)\left( {\forall y \in \mathbb{Z}} \right)\left( {x + y = 0} \right)
\]
\noindent
in symbolic form and as a sentence written in English.
\end{prog}
\hbreak

\endinput

\subsection*{Writing Guideline}
\index{writing guidelines}%
Try to use English and minimize the use of cumbersome notation.  Do not use the special symbols for quantifiers $\forall$ (for all), 
$\exists$ (there exists), $\mathrel\backepsilon$ (such that), or $\therefore $ (therefore) in formal mathematical writing.  It is often easier to write and usually easier to read, if the English words are used instead of the symbols.  For example, why make the reader interpret
\[
\left( \forall x \in \R \right) \left( \exists y \in \R \right)\left( x + y = 0 \right)
\]
when it is possible to write
\begin{center}
For each real number $x$, there exists a real number $y$ such that $x + y = 0$,
\end{center}
or, more succinctly (if appropriate),
\begin{center}
Every real number has an additive inverse.
\end{center}
\hbreak

\endinput









%The following definition of a prime number is very important in many areas of mathematics.  We will use this definition at various places in the text.  It is introduced now as an example of how to work with a definition in mathematics in Activity~\ref{A:primes}.
%
%\begin{defbox}{D:prime}{A natural number  $p$  is  a \textbf{prime number}
%\index{prime number}%
% provided that it is greater than 1 and the only natural numbers that are factors of  $p$  are  1  and  $p$.  A natural number other than 1 that is not a prime number is a \textbf{composite number}.
%\index{composite number}%
%  The number 1 is neither prime nor composite.}
%\end{defbox}
%
%
%\begin{activity}[\textbf{Prime Numbers and Composite Numbers}]\label{A:primes} \hfill \\
%Using the definition of a prime number, we see that  2, 3, 5, and  7  are prime numbers.  Also, 4  is a composite number since  $4 = 2 \cdot 2$;  10 is a composite number since  $10 = 2 \cdot 5$; and 60 is a composite number since $60 = 4 \cdot 15$.
%\begin{enumerate}
%  \item Give examples of four natural numbers other than 2, 3, 5, and 7 that are prime numbers.
%  \item Explain why a natural number  $p$  that is greater than 1 is a prime number provided that
%\begin{center} For all  $d \in \mathbb{N}$, if  $d$ is a factor of $p$, then  $d = 1$  or  
%$d = p$.
%\end{center} 
%  \item Give examples of four natural numbers that are composite numbers and explain why they are composite numbers.
%  \item Write a useful description of what it means to say that a natural number is a composite number (other than saying that it is not prime).
%\end{enumerate}
%\end{activity}
%\hbreak

%\begin{activity}[\textbf{Upper Bounds for Subsets of $\mathbb{R}$}]\label{A:upper} \hfill \\
%Let  $A$  be a subset of the real numbers.  A number  $b$  is called an \textbf{upper bound}
%\index{upper bound}%
% for the set  $A$ provided that for each element  $x$  in $A$, $x \leq b$.
%
%\begin{enumerate}
%  \item Write this definition in symbolic form by completing the following:
%
%Let  $A$  be a subset of the real numbers.  A number  $b$  is called an upper bound for the set  $A$ provided that $ \ldots .$
%
%  \item Give examples of three different upper bounds for the set \\ 
%$A = \left\{ x \in \mathbb{R} \mid 1 \leq x \leq 3 \right\}$.
%
%  \item Does the set  $B = \left\{ x \in \mathbb{R} \mid x > 0 \right\}$ have an upper bound?  Explain.
%
%  \item Give examples of three different real numbers that are not upper bounds for the set  
%$A = \left\{ x \in \mathbb{R} \mid 1 \leq x \leq 3 \right\}$. \label{A:upper4}%
%
%  \item Complete the following in symbolic form:  ``Let  $A$  be a subset of $\R$.  A number  $b$  is not an upper bound for the set  $A$   provided that $ \ldots .$''
%
%  \item Without using the symbols for quantifiers, complete the following sentence:  ``Let  $A$  be a subset of $\R$.  A number  $b$  is not an upper bound for the set  $A$ provided that $ \ldots .$''  \label{A:upper6}%
%
%  \item Are your examples in Part~(\ref{A:upper4}) consistent with your work in 
%Part~(\ref{A:upper6})?  Explain.
%\end{enumerate}
%\hbreak
%\end{activity}
%
%\begin{activity}[Least Upper Bound for a Subset of $\mathbb{R}$] \label{A:least}
%In Activity~\ref{A:upper}, we introduced the definition of an upper bound for a subset of the real numbers.  Assume that we know this definition and that we know what it means to say that a number is not an upper bound for a subset of the real numbers.
%
%Let  $A$  be a subset of  $\mathbb{R}$.  A real number  $\alpha $ is the \textbf{least upper bound} for  $A$  provided that  $\alpha $  is an upper bound for  $A$, and if $\beta $ is an upper bound for  $A$, then  $\alpha  \leq \beta $.
%
%\noindent
%\textbf{Note:}  The symbol  $\alpha $ is  the lowercase Greek letter alpha,  and the symbol  $\beta $ is  the lowercase Greek letter beta.
%
%If we define  $P\left( x \right)$ to be ``$x$  is an upper bound for  $A$,'' then we can write the definition for least upper bound as follows:
%
%A real number  $\alpha $ is the \textbf{least upper bound} for  $A$  provided that \\ $P\left( \alpha  \right) \wedge \left[ {\left( {\forall \beta  \in \mathbb{R}} \right)\left( {P\left( \beta  \right) \to \left( {\alpha  \leq \beta } \right)} \right)} \right]$.
%%
%\begin{enumerate}
%  \item Why is a universal quantifier used for the real number  $\beta $?
%  \item How do we negate a conjunction?
%  \item Complete the following sentence in symbolic form:  ``A real number  $\alpha $ is not the least upper bound for  $A$  provided that $ \ldots $''.
%  \item Complete the following sentence as an English sentence:  ``A real number  
%$\alpha $ is not the least upper bound for  $A$  provided that $ \ldots $''.
%\end{enumerate}
%
%\end{activity}
%\hbreak



\endinput

%\section*{Section~\ref{S:logequiv} Logically Equivalent Statements}
Plan about one class period for this section.

\subsection*{Main Topics}
Logically equivalent statements, converse and contrapositive, De Morgan's Laws, logical equivalencies related to conditional statements, the negation of a conditional statement.


\subsection*{The Preview Activities}
\subsubsection{Preview Activity~\ref{PA:logequiv} (Logically Equivalent Statements)}  This Preview Activity contains a new definition (logically equivalent statements), but students should be able to understand this.  In Part~(2) of the Preview Activity, students will verify one of De Morgan's Laws $\left( \mynot \left( P \wedge Q \right) \equiv \mynot P \vee \mynot Q \right)$.  This is part of Theorem~\ref{T:demorgan} in the section.  

\subsubsection*{Preview Activity~\ref{PA:converse} (Converse and Contrapositive)}
The converse and contrapositive of a conditional statement are defined.  Students are asked to complete truth tables to show that the contrapositive is logically equivalent to the original conditional statement but the converse is not.  Students must understand this before they go on to the next section.  The logical equivalence of a conditional statement and its contrapositive is an important idea that will be used when proof methods are studied in Chapter~\ref{C:proofs}.

\subsubsection*{Preview Activity~\ref{PA:conditional2} (Conditional Statements)}
The purpose of this Preview Activity is to show that a conditional statement is logically equivalent to a disjunction, and then to begin work with the negation of a conditional statement.  Students may have a difficult time with this.  However, this is very important as it forms the logical basis for a proof by contradiction.  Examples~\ref{E:conditionalasor} and~\ref{E:negationofcond} are related to this Preview Activity.
\hbreak

\subsection*{Activity~\ref{A:workingeq} (Working with a Logical Equivalency)}
This short activity is a useful activity.  It provides an opportunity for the students to rewrite a conditional statement into an equivalent conditional statement using some of the standard logical equivalencies.  The students will actually use this method to prove such statements in Chapter~\ref{C:proofs}.

%\subsubsection*{Activity~\ref{A:workingeq2}}
%This activity provides practice at constructing a truth table and then using truth tables to conlude that two statements are logically equivalent.  The activity ends by using this logical equivalency in a context that will be seen later in the text.
\hbreak


\subsection*{The Exercises}

Each of the exercises is a straightforward application of the material in the section.  The first three exercises should be assigned so that students get plenty of practice forming negations of statements.  Exercise~(\ref{exer:sec23-biconda}) should be assigned as it will be used to justify a proof technique in Chapter~\ref{C:proofs}.  Assign at least one of the parts in 
Exercise~\ref{exer:sec23-distrib}, and Exercise~(\ref{exer:sec23-6}), 
%(\ref{exer:sec23-7}), and~(\ref{exer:sec23-9}) 
is strongly recommended as it forms the basis for a proof technique in Chapter~\ref{C:proofs} (Proof using cases).  Exercises~(\ref{exer:diffimpliescont}) and~(\ref{exer:sec23-10}) are good exercises as they ask students to work with the logical equivalencies with actual conditional statements from mathematics.  
Exercise~(\ref{exer:sec23-8}) is new to the second edition.  Parts of it should be assigned if you are interested in having students be able to establish logical equivalencies without using truth tables.

\vskip6pt
\noindent
Typical Assignment:  Exercises 1, 2, 3, 4(a), 5(a), 6, 7(a), 8, 9(a, b, c), 10 or 11
\hbreak
\endinput

%\section*{Section~\ref{S:predicates} Predicates, Sets, and Quantifiers}
Plan about one class period for this section.

\subsection*{Main Topics}
Basic set notation including set-builder notation, variables and predicates, truth set of a predicate, and statements involving quantifiers.

\subsection*{The Preview Activities}
\subsubsection*{Preview Activity~\ref{PA:sentences} (Sentences that Are Not Statements)} 
The purpose of this preview activity is simply to introduce students to the fact that not every mathematical sentence is a statement.  They may have some difficulty answering Problem~(2), but that is to be expected here.  It can provide some incentive for classroom discussion.

\subsubsection*{Preview Activity~\ref{PA:variables} (Variables)}  
The concept of a \textbf{universal set} is introduced.  The purpose of the five problems is to provide a ``lead-in'' to the concept of a \textbf{truth set} introduced in the section.  This is also done in Progress Check~\ref{pr:predicates}.
\hbreak

\subsection*{Activity~\ref{A:closure-explore} (Closure Explorations)}
This activiy is intended to give students a better understanding of closure for a set with respect to an operation.  It is provides an opportunity to work with a universally quatified conditional statement and counterexamples for such statements.
\hbreak
%\subsubsection*{Activity~\ref{A:predicates}} This comes before the definition of a truth set. If students have read this section before class or if Preview Activity~\ref{PA:variables} seems to provide a sufficient introduction, then this activity can be skipped.

%\subsubsection*{Activity~\ref{A:truthset}}  This comes right after the definition of truth set and provides practice with working with this definition.  If students do not work on this activity, then it should be used to provide examples in class.  (Or the instructor can use other examples.
%\hbreak
%
\subsection*{The Exercises}

It is a good idea to assign most of the exercises in this section.  Exercise~(\ref{exer:sec21-3}) and~(\ref{exer:sec23-sets}) are needed to provide practice in using set-builder notation.  Exercise~(\ref{exer:sec21-4}) is a good exercise since it makes the students distinguish between statements and predicates.  Exercise~(\ref{Exer:quantifier}) is also good for making the distinction between statements and predicates.

\vskip6pt
\noindent
Typical Assignment:  Exercises 1(a, b, d, e), 2(a, b), 3, 4, 5, 6, 7
\hbreak
\endinput

%\section*{Section~\ref{S:quantifier} Quantifiers and Negations}
Plan at least one class period for this section.  Some students may have difficulty with forming appropriate negations of quantified statement, and so it may be necessary to spend about one and one-half periods on this section.


\subsection*{Main Topics}
Quantifiers, negations of quantified statements, counterexamples, statements with more than one quantifier.

\subsection*{The Preview Activities}
\subsubsection*{Preview Activity~\ref{PA:quantifier} (Quantifiers)} 
The purpose of this preview activity is to have the students try to form negations of statements that contain a quanfifier.  One statement uses a universal quantifier and the other uses an existential quantifier.  Perfect answers are not expected on this preview activity, but having the students make the attempt will foster a good classroom discussion.

\subsubsection*{Preview Activity~\ref{PA:morethan} (Statements with Two Quantifiers)}  
In this preview activity, students will be forced to make a distinction between a predicate and a statement.  The sentence 
$\left( {\exists x \in \mathbb{R}} \right)\left( {x \cdot y = 100} \right)$ is not a statement since the variable $y$ is not quantified.  This is the point of the first five exercises in this activity.  In the last two exercises, students will work with statements that contain more than one quantifier.  Notice that the predicate in both exercises is the same.  The point to make is that the order of the quantifiers is important.  Some students may have difficulty with this preview activity.
\hbreak


%\subsection*{The Activities}
%There are five activities in this section.  If you plan to do two or more in class, you will need to use more than one class day for this section.  Have the students complete at least one of Activities~\ref{A:square} and~\ref{A:primes} as they provide practice with reading and understanding a definition.  This is valuable practice.  A good option is to do one of the activities in class and assign the other along with the exericses.
%
%Activities~\ref{A:upper} and~\ref{A:least} are optional, but they are valuable activities for students who will be studying the structure of the real number system in future courses.



%\subsubsection*{Activity~\ref{A:negating}}  
%This activity is similar to Preview Activity~\ref{PA:quantifier}.  It is useful to do this activity after students have done the preview activity and there has been discussion of negating quantified statements in class.  
%
%\subsubsection*{Activity~\ref{A:square}}
%Most students can give examples of numbers that are perfect squares but few will have worked with a formal defintion of this concept.  Stress that the first two exercises in this activity are a good way to deal with any new definition in mathematics.  It is important to point out that many definitions will involve the uese of a quantifier.  The last two exercises give students a chance to work with the existential quantifier in this definition.

\subsection*{Activity~\ref{A:primes} (Prime Numbers and Composite Numbers)}
This activity should be done in class or assigned as it deals with the formal definition of a prime number.  Students may be able to give examples of prime numbers but they will have difficulty working with the formal defintion.

\subsection*{Activity~\ref{A:upper} (Upper Bounds for Subsets of $\boldsymbol{\R}$)}
This activity must be assigned if you plan to assign Exercise~(\ref{exer:leastupper}).  As with the previous activity, this is an opportunity for students to work with a formal defintion in mathematics.  This activity should be considered only if you plan to spend more than one day on this section.

%\subsubsection*{Activity~\ref{A:least}}
%The idea of a least upper bound is difficult for many students.  The notation will also be difficult to handle for many students.
\hbreak


\subsection*{The Exercises}

Assign the first four exercises or at least substantial parts of each of these exercises.  Exericses~(\ref{exer:24-increasing}) and~(\ref{exer:24-continuous}) deal with the formal definitions of some concepts from calculus.  At least one of these exercises should be assigned.

\vskip6pt
\noindent
Typical Assignment:  Exercises 1, 2, 3, 4, 5, 6
\hbreak
\endinput



\chapter*{Chapter~\ref{C:proofs} \\Constructing and Writing Proofs in Mathematics}

\section*{Main Objectives}
\begin{itemize}
\item Provide students with a thorough understanding of the logical foundation of the following methods of proof:
\begin{multicols}{2}
\begin{list}{}
\item Direct proof;
\item Proof using the contrapositive;
\item Proof by contradiction;
\item Proof using cases.
\end{list}
\end{multicols}
\item Develop students' abilities to construct and write proofs using these and other proof methods.
\item Introduce students to the relations of ``divides'' and congruence.
\item Develop students' abilities to construct and write proofs dealing with divisors and congruence.
\item Develop students' understanding of the Division Algorithm and their ability to use the Division Algorithm in proofs using cases.
\item Develop students' understanding of the distinction between a constructive proof and a non-constructive proof.
\end{itemize}
\hbreak
\section{Direct Proofs}\label{S:directproof}
%\markboth{Chapter \ref{C:proofs}. Constructing Proofs}{\ref{S:directproof}. Direct Proofs}
\setcounter{previewactivity}{0}
%\hbreak
\begin{previewactivity}[\textbf{Definition of Divides, Divisor, Multiple}]\label{PA:divisor} \hfill \\
%
In Section~\ref{S:direct}, we studied the concepts of even integers
\index{even integer}%
 and odd integers.  The definition of an even integer was a formalization of our concept of an even integer as being one that is ``divisible by 2,'' or a ``multiple of 2.''  We could also say that if ``2 divides an integer,'' then that integer is an even integer.  We will now extend this idea to integers other than 2.  Following is a formal definition of what it means to say that a nonzero integer $m$ divides an integer $n$.

\begin{defbox}{divides}{A nonzero integer $m$  \textbf{divides}
\index{divides}%
 an integer  $n$  provided that there is an integer  $q$  such that  $n = m \cdot q$.  We also say that  $m$  is a \textbf{divisor}
\index{divisor}%
 of  $n$, $m$ is a \textbf{factor}
\index{factor}%
 of $n$, and $n$  is a \textbf{multiple}
\index{multiple}%
 of  $m$.  The integer 0 is not a divisor of any integer.  If $a$ and $b$ are integers and $a \ne 0$, we frequently use the notation $a \mid b$ as a shorthand for ``$a$ divides $b$.''}
\label{sym:divides}%}  
\end{defbox}
%
\noindent
\textbf{A Note about Notation}:  Be careful with the notation $a \mid b$.  This does not represent the rational number  $\dfrac{a}{b}$.  The notation  $a \mid b$  represents a relationship between the integers  
$a$  and  $b$  and is simply a shorthand for ``$a$  divides  $b$.''

\newpar
\textbf{A Note about Definitions}:  Technically, a definition in mathematics should almost always be written using ``if and only if.''  It is not clear why, but the convention in mathematics is to replace the phrase ``if and only if'' with ``if'' or an equivalent.  Perhaps this is a bit of laziness or the ``if and only if'' phrase can be a bit cumbersome.  In this text, we will often use the phrase ``provided that'' instead.
%\vskip6pt

The definition for ``divides'' can be written in symbolic form using appropriate quantifiers as follows:
A nonzero integer  $m$  \textbf{divides} an integer  $n$  provided that $\left( {\exists q \in \mathbb{Z}} \right)\left( {n = m \cdot q} \right)$.
%
\begin{enumerate}
  \item Use the definition of divides to explain why 4 divides 32 and to explain why 8 divides $-96$.
%  \item Give three different examples of three integers where the first integer divides the second integer and the second integer divides the third integer.  
%\label{PA:divisor1}%
%
%  \item In your examples in Part~(\ref{PA:divisor1}), is there any relationship between the first and the third integer?  Explain, and formulate a conjecture.  \textbf{Write your conjecture in the form of a conditional statement with appropriate quantifiers}.

  \item Give several examples of two integers where the first integer does not divide the second integer.

  %\item According to the definition of ``divides,'' does the integer  0  divide the integer 10?  That is, is  0  a divisor of 10?  Explain.

  \item According to the definition of ``divides,'' does the integer  10  divide the integer  0?  That is, is  10  a divisor of  0?  Explain.

  \item Use the definition of ``divides'' to complete the following sentence in symbolic form:  ``The nonzero integer  $m$ does not divide the integer $n$ means that \ldots .''

  \item Use the definition of ``divides'' to complete the following sentence without using the symbols for quantifiers:  ``The nonzero integer  $m$  does not divide the integer $n \ldots .$''
  \item Give three different examples of three integers where the first integer divides the second integer and the second integer divides the third integer.  
\label{PA:divisor1}%
\end{enumerate}
As we have seen in Section~\ref{S:direct}, a definition is frequently used when constructing and writing mathematical proofs.  Consider the following conjecture:

\eighth
\setlength{\hangindent}{60pt}
\noindent
\textbf{Conjecture:} \emph{Let $a$, $b$, and $c$ be integers with $a \ne 0$ and $b \ne 0$.  If $a$ divides $b$ and $b$ divides $c$, then $a$ divides $c$.}

\setcounter{oldenumi}{\theenumi}
\begin{enumerate} \setcounter{enumi}{\theoldenumi}
\item Explain why the examples you generated in part~(\ref{PA:divisor1}) provide evidence that this conjecture is true.
\end{enumerate}
\setlength{\hangindent}{0pt}
In Section~\ref{S:direct}, we also learned how to use a \textbf{know-show table} to help organize our thoughts when trying to construct a proof of a statement.  If necessary, review the appropriate material in Section~\ref{S:direct}.
\setcounter{oldenumi}{\theenumi}
\begin{enumerate} \setcounter{enumi}{\theoldenumi}
  \item  State precisely what we would assume if we were trying to write a proof of the preceding conjecture. \label{PA:divisor2}
  \item Use the definition of ``divides'' to make some conclusions based on your assumptions in part~(\ref{PA:divisor2}).
  \item State precisely what we would be trying to prove if we were trying to write a proof of the conjecture. \label{PA:divisor3}
  \item Use the definition of divides to write an answer to the question, ``How can we prove what we stated in part~(\ref{PA:divisor3})?''
\end{enumerate}

\hbreak
\end{previewactivity}


\endinput

\begin{previewactivity}[\textbf{Calendars and Clocks}]\label{PA:calender} \hfill \\
This activity is intended to help with understanding the concept of congruence, which will be studied at the end of this section.
\begin{enumerate}
  \item Suppose that it is currently Tuesday.
\label{PA:calender1}%
  \begin{enumerate}
    \item What day will it be 3 days from now?
    \item What day will it be 10 days from now?
    \item What day will it be 17 days from now?  What day will it be 24 days from now?
    \item Find several other natural numbers  $x$  such that it will be Friday  $x$  days from now.
\label{PA:calender1d}%
    \item Create a list (in increasing order) of the numbers $3, 10, 17, 24$, and the numbers you generated in Part~(\ref{PA:calender1d}).  Pick any two numbers from this list and subtract one from the other. Repeat this several times.
\label{PA:calender1e}%
    \item What do the numbers you obtained in Part~(\ref{PA:calender1e}) have in common?
  \end{enumerate}

  \item Suppose that we are using a twelve-hour clock with no distinction between {\smallc a.m.} and {\smallc p.m.}  Also, suppose that the current time is 5:00.
  \begin{enumerate}
    \item What time will it be  4 hours from now?
    \item What time will it be 16 hours from now?  What time will it be 28 hours from now?
    \item Find several other natural numbers  $x$  such that it will be 9:00  $x$  hours from now.
\label{PA:calender2c}%
    \item Create a list (in increasing order) of the numbers $4, 16, 28$, and the numbers you generated in Part~(\ref{PA:calender2c}).  Pick any two numbers from this list and subtract one from the other. Repeat this several times.
\label{PA:calender2e}%
    \item What do the numbers you obtained in Part~(\ref{PA:calender2e}) have in common? 
  \end{enumerate}

  \item This is a continuation of Part~(\ref{PA:calender1}).  Suppose that it is currently Tuesday.
  \begin{enumerate}
    \item What day was it 4 days ago?
    \item What day was it 11 days ago?  What day was it 18 days ago?
    \item Find several other natural numbers  $x$  such that it was Friday  $x$  days ago.  \label{PA:calender3c}%
    \item Create a list (in increasing order) consisting of the numbers 
\linebreak
$-18, -11, -4$, the opposites of the numbers you generated in Part~(\ref{PA:calender3c}) and the positive numbers in the list from Part~(\ref{PA:calender1e}).  Pick any two numbers from this list and subtract one from the other.  Repeat this several times.
\label{PA:calender3d}%
    \item What do the numbers you obtained in Part~(\ref{PA:calender3d}) have in common?
  \end{enumerate}
%
\end{enumerate}
\end{previewactivity}
\hbreak


\endinput

%
\subsection*{Some Mathematical Terminology}
In Section~\ref{S:direct}, we introduced the idea of a direct proof.  Since then, we have used some common terminology in mathematics without much explanation.  Before we proceed further, we will discuss some frequently used mathematical terms.

A \textbf{proof}
\label{proof}%
\index{proof}%
 in mathematics is a convincing argument that some mathematical statement is true.  A proof should contain enough mathematical detail to be convincing to the person(s) to whom the proof is addressed.  In essence, a proof is an argument that communicates a mathematical truth to another person (who has the appropriate mathematical background).  A proof must use correct, logical reasoning and be based on previously established results.  These previous results can be axioms, definitions, or previously proven theorems.  These terms are discussed below.

Surprising to some is the fact that in mathematics, there are always \textbf{undefined terms}.
\label{undefined}%
\index{undefined term}%
  This is because if we tried to define everything, we would end up going in circles.  Simply put, we must start somewhere.  For example, in Euclidean geometry, the terms ``point,'' ``line,'' and ``contains'' are undefined terms.  In this text, we are using our number systems such as the natural numbers and integers as undefined terms.  We often assume that these undefined objects satisfy certain properties.  These assumed relationships are accepted as true without proof and are called axioms (or postulates).  An \textbf{axiom} 
\label{axiom}%
\index{axiom}%
 is a mathematical statement that is accepted without proof.  Euclidean geometry starts with undefined terms and a set of postulates and axioms.  For example, the following statement is an axiom of Euclidean geometry:

\newpar
\setlength{\hangindent}{20pt}
\indent
\emph{Given any two distinct points, there is exactly one line that contains these two points.}


\begin{center}
\fbox{\parbox{4.68in}{The closure properties of the number systems discussed in Section~\ref{S:prop} and the properties of the number systems in Table~\ref{Ta:propertiesofreals} on page~\pageref{Ta:propertiesofreals} are being used as axioms in this text.}}
\end{center}

%The closure properties of the number systems discussed in Section~\ref{S:prop} are being used as axioms in this text.

A \textbf{definition}
\label{definition}%
\index{definition}%
 is simply an agreement as to the meaning of a particular term.  For example, in this text, we have defined the terms ``even integer'' and ``odd integer.''  Definitions are not made at random, but rather, a definition is usually made because a certain property is observed to occur frequently.  As a result, it becomes convenient to give this property its own special name.  Definitions that have been made can be used in developing mathematical proofs.  In fact, most proofs require the use of some definitions.

In dealing with mathematical statements, we frequently use the terms ``conjecture,'' ``theorem,'' ``proposition,'' ``lemma,'' and ``corollary.''  A \textbf{conjecture}
\label{conjecture}%
\index{conjecture}%
 is a statement that we believe is plausible.  That is, we think it is true, but we have not yet developed a proof that it is true.  A \textbf{theorem} 
\label{theorem}%
\index{theorem}%
 is a mathematical statement for which we have a proof.  A term that is often considered to be synonymous with ``theorem'' is \textbf{proposition}. 
\label{proposition}%.  %One difference, however, is that a proposition can be false.  In this case, if the proposition involves a universal quantifier, we often show it is false by giving a counterexample.  

Often the proof of a theorem can be quite long.  In this case, it is often easier to communicate the proof in smaller ``pieces.''    These supporting pieces are often called lemmas.  A 
\textbf{lemma} 
\label{lemma}%
\index{lemma}%
 is a true mathematical statement that was proven mainly to help in the proof of some theorem.  Once a given theorem has been proven, it is often the case that other propositions follow immediately from the fact that the theorem is true.  These are called corollaries of the theorem.  The term \textbf{corollary} 
\label{corollary}%
\index{corollary}%
 is used to refer to a theorem that is easily proven once some other theorem has been proven.
\hbreak

\endinput

\subsection*{Constructing Mathematical Proofs}
To create a proof of a theorem, we must use correct logical reasoning and mathematical statements that we already accept as true.  These statements include axioms, definitions, theorems, lemmas, and corollaries.

In Section~\ref{S:direct}, we introduced the use of a \textbf{know-show table}
\index{know-show table}%
 to help us organize our work when we are attempting to prove a statement.  We also introduced some guidelines for writing mathematical proofs once we have created the proof.  These guidelines should be reviewed before proceeding.

Please remember that when we start the process of writing a proof, we are essentially ``reporting the news.''  That is, we have already discovered the proof, and now we need to report it.  This reporting often does not describe the process of discovering the news (the investigative portion of the process).

Quite often, the first step is to develop a conjecture.  This is often done after working within certain objects for some time.  This is what we did in \typeu Activity~\ref*{PA:divisor} when we used examples to provide evidence that the following conjecture is true:
%\hbreak

\newpar
\setlength{\hangindent}{60pt}
\noindent
\textbf{Conjecture:} \emph{Let $a$, $b$, and $c$ be integers with $a \ne 0$ and $b \ne 0$.  If $a$ divides $b$ and $b$ divides $c$, then $a$ divides $c$.}
%\hbreak

%\eighth
%To try to prove this conjecture, we will, of course, have to use the definitions in Beginning 
%Activity~\ref{PA:divisor} on page~\pageref{PA:divisor}.  

%\begin{defbox}{divides2}{A nonzero integer $m$  \textbf{divides}
%\index{divides}%
% an integer  $n$  provided that there is an integer  $q$  such that  $n = m \cdot q$.  We also say that  $m$  is a \textbf{divisor}
%\index{divisor}%
% of  $n$, $m$ is a \textbf{factor}
%\index{factor}%
% of $n$, and $n$  is a \textbf{multiple}
%\index{multiple}%
% of  $m$.  The integer 0 is not a divisor of any integer.}
%\end{defbox}
%

%\noindent
%\textbf{Important Comment about Notation}: When a nonzero integer  $m$  divides an integer $n$, we frequently use the notation  $m \mid n$. 
%\label{sym:divides}%
%  Be careful with this notation.  It does not represent the rational number  $\dfrac{m}{n}$.  The notation  $m \mid n$  represents a relationship between the integers  $m$  and  $n$  and is simply a shorthand for ``$m$  divides  $n$.''
%\hbreak
%
Before we try to prove a conjecture, we should make sure we have explored some examples.  This simply means to construct some specific examples where the integers  $a$, $b$, and  $c$  satisfy the hypothesis of the conjecture in order to see if they also satisfy the conclusion.   We did this for this conjecture in \typeu Activity~\ref*{PA:divisor}.  %If we happen to find an example of three integers that satisfy the hypothesis but make the conclusion false, then we would have found a counterexample for the conjecture.  We could then conclude the conjecture is false.  This will not happen for the current conjecture.

%One example for this conjecture is  $a = 3, b = 12, c = 48$.  Notice that  $3 \mid 12$ and  
%$12 \mid 48$, and we observe that  $3 \mid 48$.  In particular, if we use the definition of divides, we see that
%\[
%12 = 3 \cdot 4\text{ and that  }48 = 12 \cdot 4.
%\]
%Now, substitute the right side of the first equation for 12 in the second equation.  This gives
%\[
%\begin{aligned}
%  48 &= \left( {3 \cdot 4} \right) \cdot 4 \\ 
%  48 &= 3 \cdot \left( {4 \cdot 4} \right). \\ 
%\end{aligned} 
%\]
%This last equation shows that  3  divides  48.  While the examples for this conjecture may seem trivial, this is not always the case.  Exploring examples can sometimes lead to a counterexample for a conjecture, and other times examples can suggest a method of proof.  For this example, the main step was to substitute the expression  $3 \cdot 4$ for  12  from one equation into the other equation.  

We will now start a know-show~table for this conjecture.
$$
\BeginTable
\def\C{\JustCenter}
\BeginFormat
|p(0.4in)|p(2in)|p(1.8in)|
\EndFormat
\_
 | \textbf{Step}  |  \textbf{Know}  |  \textbf{Reason} |    \\+02 \_
 | $P$     |  $a, b, c \in \Z$, $a \ne 0$, $b \ne 0$, $a \mid b$ and  $b \mid c$     |  Hypothesis | \\ \_1
 | $P1$    |                                 |           |  \\ \_1
 | \C $\vdots$  |  \C $\vdots$                         | \C $\vdots$   |   \\ \_1
 | $Q1$    |                                 |            | \\  \_1 
 | $Q$     |  $a \mid c$                     |            | \\  \_
 | \textbf{Step}  |  \textbf{Show}  |  \textbf{Reason}    | \\+20  \_
\EndTable
$$


%
%\begin{center}
%\begin{tabular}[h]{|p{0.4in}|p{2in}|p{1.8in}|}
%  \hline
%  \textbf{Step}  &  \textbf{Know}  &  \textbf{Reason}     \\ \hline
%  $P$     &  $a, b, c \in \Z$, $a \ne 0$, $b \ne 0$, $a \mid b$ and  $b \mid c$     &  Hypothesis \\ \hline
%  $P1$    &                                 &             \\ \hline
%  $\vdots$  &  $\vdots$                         & $\vdots$      \\ \hline
%  $Q1$    &                                 &             \\  \hline  
%  $Q$     &  $a \mid c$                     &             \\ \hline
%  \textbf{Step}  &  \textbf{Show}  &  \textbf{Reason}     \\ \hline
%\end{tabular}
%\end{center}
%
The backward question we ask is, ``How can we prove that  $a$  divides  $c$?''  One answer is to use the definition and show that there exists an integer  $q$  such that  $c = a \cdot q$.  This could be step $Q1$ in the know-show table.

We now have to prove that a certain integer $q$ exists, so we ask the question, ``How do we prove that this integer exists?''  When we are at such a stage in the backward process of a proof, we usually turn to what is known in order to prove that the object exists or to find or construct the object we are trying to prove exists.  We often say that we try to ``construct'' the object or at least prove it exists from the known information.  So at this point, we go to the forward part of the proof to try to prove that there exists an integer $q$ such that $c = a \cdot q$.

The forward question we ask is, ``What can we conclude from the facts that  $a \mid b$  and  $b \mid c$?''  Again, using the definition, we know that there exist integers  $s$  and  $t$  such that  $b = a \cdot s$ and  $c = b \cdot t$.   This could be step $P1$ in the know-show table.

The key now is to determine how to get from $P1$ to $Q1$.  That is, can we use the conclusions that the integers $s$ and $t$ exist in order to prove that the integer $q$ (from the backward process) exists.  
%We might get some motivation from the numerical example we explored.  
Using the equation  $b = a \cdot s$, we can substitute  $a \cdot s$ for  $b$  in the second equation, $c = b \cdot t$.  This gives
\[
\begin{aligned}
  c &= b \cdot t \\ 
    &= ( {a \cdot s} ) \cdot t \\ 
    &= a( {s \cdot t}). \\ 
\end{aligned}
\]
The last step used the associative property of multiplication. (See 
Table~\ref{Ta:propertiesofreals} on page~\pageref{Ta:propertiesofreals}.) This shows that  $c$  is equal to  
$a$  times some integer.  (This is because  $s \cdot t$ is an integer by the closure property for integers.)  So although we did not use the letter  $q$, we have arrived at step $Q1$.  The completed know-show~table follows.
$$
\BeginTable
\def\C{\JustCenter}
\BeginFormat
|p(0.4in)|p(2in)|p(1.8in)|
\EndFormat
\_
 | \textbf{Step}  |  \textbf{Know}  |  \textbf{Reason} |    \\+02 \_
 | $P$     |  $a, b, c \in \Z$, $a \ne 0$, $b \ne 0$, $a \mid b$ and  $b \mid c$     |  Hypothesis | \\ \_1
 | $P1$    | $\left( {\exists s \in \mathbb{Z}} \right)\left( {b = a \cdot s} \right)$\\ $\left( {\exists t \in \mathbb{Z}} \right)\left( {c = b \cdot t} \right)$   | Definition of ``divides''          |  \\ \_1
 | $P2$  |  $c = \left( {a \cdot s} \right) \cdot t$                         | Substitution for $b$   |   \\ \_1
 | $P3$  | $c = a \cdot \left( {s \cdot t} \right)$ | Associative property of multiplication | \\ \_1
 | $Q1$    |  $\left( {\exists q \in \mathbb{Z}} \right)\left( {c = a \cdot q} \right)$                               | Step $P3$ and the closure properties of the integers            | \\  \_1 
 | $Q$     |  $a \mid c$                     | Definition of ``divides''            | \\  \_
% | \textbf{Step}  |  \textbf{Show}  |  \textbf{Reason}    | \\+20  \_
\EndTable
$$

\newpar
Notice the similarities between what we did for this proof and many of the proofs about even and odd integers we constructed in Section~\ref{S:direct}.  When we try to prove that a certain object exists, we often use what is called the  \textbf{construction method for a proof}.
\index{construction method}%
  The appearance of an existential quantifier in the show (or backward) portion of the proof is usually the indicator to go to what is known in order to prove the object exists.

\newpar
We can now report the news by writing a formal proof.
%
\setcounter{equation}{0}
\begin{theorem}\label{T:transdivide}
Let  $a$, $b$, and  $c$  be integers with $a \ne 0$ and $b \ne 0$.  If  $a$  divides  $b$  and  $b$  divides  $c$, then  $a$  divides  $c$.
\end{theorem}
%
\begin{myproof}
We assume that $a$, $b$, and  $c$  are integers with $a \ne 0$ and $b \ne 0$.  We further assume that  $a$  divides  $b$  and that  $b$  divides  $c$.  We  will prove that  $a$  divides  $c$.

Since  $a$  divides  $b$ and $b$ divides $c$, there exist integers  $s$ and $t$  such that
\begin{align}\label{eq:3a}
b &= a \cdot s, \text{ and } \\
c &= b \cdot t.\label{eq:3b}
\end{align}
We can now substitute the expression for  $b$  from equation~(\ref{eq:3a}) into equation~(\ref{eq:3b}).  This gives
\begin{equation}\label{eq:3c} \notag
c = ( {a \cdot s}) \cdot t.
\end{equation}
Using the associate property for multiplication, we can rearrange the right side of the last equation to obtain
\begin{equation}\label{eq:3d} \notag
c = a \cdot ( {s \cdot t} ).
\end{equation}
Because both  $s$  and  $t$  are integers, and since the integers are closed under multiplication, we know that  $s \cdot t \in \mathbb{Z}$.  Therefore, the previous equation 
proves that  $a$  divides  $c$.  Consequently, we have proven that whenever  $a$, $b$, and  $c$  are integers with $a \ne 0$ and $b \ne 0$ such that  $a$  divides  $b$ and  $b$ divides  $c$, then  $a$  divides  $c$.
\end{myproof}
\hbreak

\endinput

\subsection*{Writing Guidelines for Equation Numbers}
\index{writing guidelines}%

We wrote the proof for Theorem~\ref{T:transdivide} according to the guidelines introduced in Section~\ref{S:direct}, but a new element that appeared in this proof was the use of equation numbers.  Following are some guidelines that can be used for \textbf{equation numbers}.
\index{equation numbers}%

If it is necessary to refer to an equation later in a proof, that equation should be centered and displayed.  It should then be given a number.  The number for the equation should be written in parentheses on the same line as the equation at the right-hand margin as in shown in the following example.

%\begin{example} \label{E:eqnum} \hfill
\setcounter{equation}{0}

\eighth
Since  $x$  is an odd integer, there exists an integer  $n$  such that
\begin{equation}\label{eq:3e}
x = 2n + 1.
\end{equation}

\begin{flushleft}
Later in the proof, there may be a line such as
\begin{center}
Then, using the result in equation~(\ref{eq:3e}), we obtain  \ldots .
\end{center}
Notice that we did not number every equation in Theorem~\ref{T:transdivide}.  We should only number those equations we will be referring to later in the proof, and we should only number equations when it is necessary.  For example, instead of numbering an equation, it is often better to use a phrase such as, ``the previous equation proves that \ldots" or ``we can rearrange the terms on the right side of the previous equation.'' 
Also, note that the word ``equation'' is not capitalized when we are referring to an equation by number.  Although it may be appropriate to use a capital ``E,'' the usual convention in mathematics is not to capitalize.
%Also, note that the word ``Equation'' begins with a capital ``E'' when we are referring to an equation by number.
\end{flushleft}
%\end{example}
\hbreak

\begin{prog}[\textbf{A Property of Divisors}]\label{pr:divisors} \hfill
%In this activity, the universal set for each variable is the set of integers.
\begin{enumerate}
\item Give at least four different examples of integers  $a$, $b$, and  $c$  with $a \ne 0$ such that  $a$  divides  $b$  and  $a$  divides  $c$.
\label{pr:divisors1}%

\item For each example in Part~(\ref{pr:divisors1}), calculate the sum  $b + c$.  Does the integer  $a$  divide the sum  $b + c$?
\label{pr:divisors2}%

\item Construct a know-show table for the following proposition:  For all integers $a$, $b$, and  $c$  with 
$a \ne 0$, if $a$ divides $b$ and $a$ divides $c$, then $a$ divides $(b + c)$.
\label{pr:divisors3}%
 
%\item Construct a Know-show table for a proof of the conjecture in Part~(\ref{pr:divisors3}).

\end{enumerate}
\end{prog}
\hbreak

\endinput

\subsection*{Using Counterexamples}
\index{counterexample|(}%
In Section~\ref{S:direct} and so far in this section, our focus has been on proving statements that involve universal quantifiers.  However, another important skill for mathematicians is to be able to recognize when a statement is false and then to be able to prove that it is false.  For example, suppose we want to know if the following proposition is true or false.

\begin{list}{}
\item For each integer $n$, if 5 divides $\left(n^2 - 1 \right)$, then 5 divides $\left( n - 1 \right)$.
\end{list}  

\newpar
Suppose we start trying to prove this proposition.  In the backward process, we would say that in order to prove that 5 divides $\left( n - 1 \right)$, we can show that there exists an integer $k$ such that
\[
Q_1:  n - 1 = 5k \quad \text{or} \quad n = 5k + 1.
\]
For the forward process, we could say that since 5 divides $\left(n^2 - 1 \right)$, we know that there exists an integer $m$ such that
\[
P_1:  n^2 - 1 = 5m \quad \text{or} \quad n^2 = 5m + 1.
\]
The problem is that there is no straightforward way to use $P_1$ to prove $Q_1$.  At this point, it would be a good idea to try some examples for $n$ and try to find situations in which the hypothesis of the proposition is true.  (In fact, this should have been done before we started trying to prove the proposition.)  The following table summarizes the results of some of these explorations with values for $n$.
$$
\BeginTable
\BeginFormat
| c | c | c | c | c |
\EndFormat
" $n$ | $n^2 - 1$ | Does 5 divide $\left( n^2 - 1 \right)$ | $n - 1$ | Does 5 divide $(n - 1)$ " \\ \_
" 1   |   0  |  yes  |  0  |  yes " \\ \_
" 2   |   3  |  no   |  1  | no " \\ \_
" 3   |  8  |  no  |  2 | no " \\ \_
" 4  |  15  | yes |  3  | no " \\ \_
\EndTable
$$
We can stop exploring examples now since the last row in the table provides an example where the hypothesis is true and the conclusion is false.  Recall from Section~\ref{S:quantifier} 
(see page~\pageref{D:counterexample2}) that a \textbf{counterexample} for a statement of the form 
$\left( \forall x \in U \right) \left( P(x) \right)$ is an element $a$ in the universal set for which $P(a)$ is false.  So we have actually proved that the negation of the proposition is true.

When using a counterexample to prove a statement is false, we do not use the term ``proof'' since we reserve a proof for proving a proposition is true.  We could summarize our work as follows:


%\newpage
\indent
\parbox{4.5in}{\textbf{Conjecture}.  For each integer $n$, if 5 divides $\left(n^2 - 1 \right)$, then 5 divides $\left( n - 1 \right)$.}

\vskip6pt
\indent
\parbox{4.5in}{The integer $n = 4$ is a counterexample that proves this conjecture is false.  Notice that when $n = 4$, $n^2 - 1 = 15$ and 5 divides 15.  Hence, the hypothesis of the conjecture is true in this case.  In addition, $n - 1 = 3$ and 5 does not divide 3 and so the conclusion of the conjecture is false in this case.  Since this is an example where the hypothesis is true and the conclusion is false, the conjecture is false.}

\newpar
As a general rule of thumb, anytime we are trying to decide if a proposition is true or false, it is a good idea to try some examples first.  The examples that are chosen should be ones in which the hypothesis of the proposition is true.  If one of these examples makes the conclusion false, then we have found a counterexample and we know the proposition is false.  If all of the examples produce a true conclusion, then we have evidence that the proposition is true and can try to write a proof.
\hbreak

\begin{prog}[\textbf{Using a Counterexample}] \label{pr:counterexample} \hfill \\
Use a counterexample to prove the following statement is false.
\begin{list}{}
\item For all integers $a$ and $b$, if 5 divides $a$ or 5 divides $b$, then 5 divides $(5a + b)$. 
\end{list}
\end{prog}
\index{counterexample|)}%
\hbreak


%Outside of mathematics, a counterexample is often used as a rebuttal to a proposed general statement. For example, if someone states that ``Students who do not do well in school are invariably failures later in life,'' someone might say that Albert Einstein is a counterexample for this statement.  In mathematics, we only deal with statements that are true or are false and so we often use a counterexample to prove that a universally quantified statement is false.


\subsection*{Congruence}
\index{congruence|(}%
What mathematicians call congruence is a concept used to describe cycles in the world of the integers. For example, the day of the week is a cyclic phenomenon in that the day of the week repeats every seven days.  The time of the day is a cyclic phenomenon because it repeats every 12 hours if we use a 12-hour clock  or every 24 hours if we use a 24-hour clock.  We explored these two cyclic phenomena in \typeu Activity~\ref*{PA:calender}.

Similar to what we saw in \typeu Activity~\ref*{PA:calender}, if it is currently Monday, then it will be Wednesday 2 days from now, 9 days from now, 16 days from now, 23 days from now, and so on.  In addition, it was Wednesday 5 days ago, 12 days ago, 19 days ago, and so on.  Using negative numbers for time in the past, we generate the following list of numbers:

\[
 \ldots ,\; - 19,  - 12,  - 5, 2, 9, 16, 23,  \ldots .
\]

Notice that if we subtract any number in the list above from any other number in that list, we will obtain a multiple of 7.  For example,

\[
\begin{aligned}
  16 - 2 &= 14 = 7 \cdot 2 \\ 
  \left( { - 5} \right) - \left( 9 \right) &=  - 14 = 7 \cdot \left( { - 2} \right) \\ 
  16 - \left( { - 12} \right) &= 28 = 7 \cdot 4. \\ 
\end{aligned} 
\]

Using the concept of congruence, we would say that all the numbers in this list are congruent modulo 7, but we first have to define when two numbers are congruent modulo some natural number  $n$.

\begin{defbox}{congruence}{Let  $n \in \mathbb{N}$.  If  $a$  and  $b$  are integers, then we say that \textbf{$\boldsymbol{a}$  is congruent to  $\boldsymbol{b}$  modulo  $\boldsymbol{n}$}
\index{congruent modulo $n$}%
  provided that  $n$  divides  $a - b$.  A standard notation for this is   
$a \equiv b \pmod n$.
\label{sym:congruence}%
  This is read as ``$a$  is congruent to  $b$  modulo  $n$''   or  ``$a$  is congruent to  $b$  mod  $n$.''}
\end{defbox}
%
\noindent
Notice that we can use the definition of divides to say that  $n$ divides $(a - b)$  if and only if  there exists an integer  $k$  such that  $a - b = nk$.  So we can write

\[
\begin{aligned}
  a &\equiv b \pmod n \text{  means  }\left( {\exists k \in \mathbb{Z}} \right)\left( {a - b = nk} \right), \text{or} \hfill \\
  a &\equiv b \pmod n \text{  means  }\left( {\exists k \in \mathbb{Z}} \right)\left( {a = b + nk} \right). \hfill \\ 
\end{aligned}
\]

This means that in order to find integers that are congruent to $b$ modulo $n$, we only need to add multiples of $n$ to $b$.  For example, to find integers that are congruent to 2 modulo 5, we add multiples of 5 to 2.  This gives the following list:

\[
\ldots, -13, -8, -3, 2, 7, 12, 17, \ldots .
\]
We can also write this using set notation and say that

\[
\left\{ \left. a \in \Z \right| a \equiv 2 \pmod 5 \right\} = 
\left\{ \ldots, -13, -8, -3, 2, 7, 12, 17, \ldots \right\}.
\]
\hbreak

\begin{prog}[\textbf{Congruence Modulo 8}]\label{pr:congruence} \hfill
\begin{enumerate}
\item  Determine at least eight different integers that are congruent to 5 modulo 8.

\item Use set builder notation and the roster method to specify the set of all integers that are congruent to 5 modulo 8.

\item Choose two integers that are congruent to 5 modulo 8 and add them.  Then repeat this for at least five other pairs of integers that are congruent to 5 modulo 8.
\label{pr:congruence3}% 

\item Explain why all of the sums that were obtained in Part~(\ref{pr:congruence3}) are congruent to 2 modulo 8.
\end{enumerate}
\end{prog}
\hbreak

We will study the concept of congruence modulo  $n$  in much more detail later in the text.  For now, we will work with the definition of congruence modulo $n$  in the context of proofs.  For example, all of the examples used in Progress Check~\ref{pr:congruence} should provide evidence that the following proposition is true.
\begin{proposition} \label{prop:congruenceproof}
For all integers $a$ and $b$, if $\mod{a}{5}{8}$ and $\mod{b}{5}{8}$, then 
$\mod{(a + b)}{2}{8}$.
\end{proposition}
%
\begin{prog}[\textbf{Proving Proposition~\ref{prop:congruenceproof}}]\label{pr:congruence2} \hfill \\
We will use ``backward questions'' and ``forward questions'' to help construct a proof for Proposition~\ref{prop:congruenceproof}.  So, we might ask, ``How do we prove that 
$\mod{(a + b)}{2}{8}$?''  One way to answer this is to use the definition of congruence and state that $\mod{(a + b)}{2}{8}$ provided that 8 divides $(a + b - 2)$.
\begin{enumerate}
  \item Use the definition of divides to determine a way to prove that 8 divides $(a + b - 2)$.
\end{enumerate}

We now turn to what we know and ask, ``What can we conclude from the assumptions that 
$\mod{a}{5}{8}$ and $\mod{b}{5}{8}$?''  We can again use the definition of congruence and conclude that 8 divides $(a - 5)$ and 8 divides $(b - 5)$.
\end{prog}
\setcounter{oldenumi}{\theenumi}
\begin{enumerate} \setcounter{enumi}{\theoldenumi}
\item Use the definition of divides to make conclusions based on the facts that 8 divides 
$(a - 5)$ and 8 divides $(b - 5)$.
\item Solve an equation from part~(2) for $a$ and for $b$.
\item Use the results from part~(3) to prove that 8 divides $(a + b - 2)$.
\item Write a proof for Proposition~\ref{prop:congruenceproof}.
\end{enumerate}

\index{congruence|)}%
\hbreak

\endinput

\subsection*{Additional Writing Guidelines}
\index{writing guidelines|(}%
We will now be writing many proofs, and it is important to make sure we write according to accepted guidelines so that our proofs may be understood by others.  Some writing guidelines were introduced in Chapter~\ref{C:intro}.  The first four writing guidelines given below can be considered general guidelines, and the last three can be considered as technical guidelines specific to writing in mathematics.
\begin{enumerate}
\item \label{writing:know}%
\textbf{Know your audience.} 
Every writer should have a clear idea of the intended audience for a piece of writing.  In that way, the writer can give the right amount of information at the proper level of sophistication to communicate effectively.  This is especially true for mathematical writing.  For example, if a mathematician is writing a solution to a textbook problem for a solutions manual for instructors, the writing would be brief with many details omitted.  However, if the writing was for a students' solution manual, more details would be included.

\item \textbf{Use complete sentences and proper paragraph structure.}
Good grammar is an important part of any writing.  Therefore, conform to the accepted rules of grammar.  Pay careful attention to the structure of sentences.  Write proofs using \textbf{complete sentences} but avoid run-on sentences.  Also, do not forget punctuation, and always use a spell checker when using a word processor.

\item \textbf{Keep it simple}.
It is often difficult to understand a mathematical argument no matter how well it is written.  Do not let your writing help make it more difficult for the reader.  Use simple, declarative sentences and short paragraphs, each with a simple point.

\item \textbf{Write a first draft of your proof and then revise it.} 
Remember that a proof is written so that readers are able to read and understand the reasoning in the proof.  Be clear and concise.  Include details but do not ramble.  Do not be satisfied with the first draft of a proof.  Read it over and refine it.  Just like any worthwhile activity, learning to write mathematics well takes practice and hard work.  This can be frustrating.  Everyone can be sure that there will be some proofs that are difficult to construct, but remember that proofs are a very important part of mathematics.  So work hard and have fun.

\item \textbf{Do not use $*$ for multiplication or \^{} for exponents.}
Leave this type of notation for writing computer code.  The use of this notation makes it difficult for humans to read.  In addition, avoid using $/$ for division when using a complex fraction.  

For example, it is very difficult to read 
$\left(x^3 -3x^2 + 1/2 \right)\!/\!\left(2x/3 - 7\right)$; the fraction
\[
\frac{x^3 - 3x^2 +\dfrac{1}{2}}{\dfrac{2x}{3} - 7}
\]
is much easier to read.

\item \textbf{Do not use a mathematical symbol at the beginning of a sentence.}
For example, we should not write, ``Let $n$ be an integer.  $n$ is an odd integer provided that \ldots .''  Many people find this hard to read and often have to re-read it to understand it.  It would be better to write, ``An integer $n$ is an odd integer provided that \ldots .''

\item \textbf{Use English and minimize the use of cumbersome notation}.  Do not use the special symbols for quantifiers $\forall$ (for all), 
$\exists$ (there exists), $\mathrel\backepsilon$ (such that), or $\therefore $ (therefore) in formal mathematical writing.  It is often easier to write, and usually easier to read, if the English words are used instead of the symbols.  For example, why make the reader interpret
\[
\left( \forall x \in \R \right) \left( \exists y \in \R \right)\left( x + y = 0 \right)
\]
when it is possible to write
\begin{center}
For each real number $x$, there exists a real number $y$ such that $x + y = 0$,
\end{center}
or, more succinctly (if appropriate),
\begin{center}
Every real number has an additive inverse.
\end{center}
\end{enumerate}
\index{writing guidelines|)}%
\hbreak

\endinput

%



\endinput







\begin{center}
\begin{tabular}[h]{|p{0.4in}|p{2in}|p{1.8in}|}
  \hline
  \textbf{Step}  &  \textbf{Know}  &  \textbf{Reason}     \\ \hline
  $P$   &  $m \equiv 5 \pmod 6$  &  Hypothesis \\ \hline
  $P1$  &  $6 \mid \left( m-5 \right)$           &  Definition of ``congruence modulo 6''           \\ \hline
  $P2$  &  $\left( {\exists k \in \mathbb{Z}} \right)\left( {m - 5 = 6k} \right)$  &  Definition of ``divides''  \\ \hline
  $P3$  &                                         &  Algebra  \\ \hline
  \vdots  &  \vdots                         & \vdots      \\ \hline
  $Q1$    &  $  6 \mid \left(m^2-1 \right) $       &             \\  \hline  
  $Q$     &  $m^2  \equiv 1 \pmod 6$   &  Definition of ``congruence modulo 6''           \\ \hline
  \textbf{Step}  &  \textbf{Show}  &  \textbf{Reason}     \\ \hline
\end{tabular}
\end{center}
\hbreak













\endinput

\endinput

\section*{Section~\ref{S:moremethods} More Methods of Proof}
Plan about one and one-half class periods for this section.  It may be a good idea to plan on three class days for this section and Section~\ref{S:directproof}.  


\subsection*{Main Topics}
Proofs that use the contrapositive of a conditional statement, proofs of biconditional statements, the use of logical equivalencies in proofs, existence theorems and constructive proofs.

\subsection*{The Preview Activities}
\subsubsection*{Preview Activity~\ref{PA:attempt} (Attempting a Proof)} 
This can be a frustrating preview activity for students since the purpose is to convince them that a direct proof of a relatively simple statement is not always possible.

\subsubsection*{Preview Activity~\ref{PA:contrapositive} (The Contrapositive)}  
In this preview activity, students are asked to prove that the contrapositive of a conditional statement is logically equivalent to the conditional statement.  (This was also done in Preview Activity~\ref{PA:converse} in Section~\ref{S:logequiv}.)  Students will then use this fact to prove the statement in Preview Activity~\ref{PA:attempt}.

\subsubsection*{Preview Activity~\ref{PA:biconditional} (A Biconditional Statement)}
The purpose of this preview activity is to introduce a standard method for proving that a biconditional statement is true.  The logical equivalency was also done in 
Exercise~(\ref{exer:sec23-bicond}) in Section~\ref{S:logequiv}.
\hbreak


\subsection*{Activity~\ref{A:usingcontr} (Using a Logical Equivalency)}  
This a good activity for students to show students how to use a contrapositive in a nontrivial proof of a proposition that seems like it should be easy to prove.  It is not.  Students may find the proof in Part~(5) difficult since they are asked to assume that Proposition~X is true and they will have to use another logical equivalency. 
\hbreak

\subsection*{A Classroom Activity}
Following is an example that can be used as a class discussion item or for students to work on in small groups.  I recommend doing this as a class discussion with the instructor taking an active part in trying to lead the students through the activity.  Many students will have a difficult time formulating responses to the first two parts of the activity without assistance.

\vskip6pt
\noindent
Consider the following proposition:

\noindent
For all nonzero integers $a$ and $b$, if $a + b \ne 7$ and $49a + b \ne 1$, then the equation 
$ax^3 + bx - 7 = 0$ has no solution that is a natural number.

\begin{enumerate}
\item Focus on the conclusion in the conditional sentence, which is: the equation 
$ax^3 + bx - 7 = 0$ has no solution that is a natural number.  What does it mean to say that this equation has no solution that is a natural number?

\item Write the contrapositive of the proposition.

\item Outline a proof of the contrapositive of the proposition.
\end{enumerate}

In the first part of the activity, I usually try to lead the students to a formulation of the answer in terms of a universal quantifier.  Following is such an answer.

\vskip6pt
\begin{center}
$\left( \forall n \in \mathbb{N} \right) \left( an^3 + bn - 7 \ne 0 \right)$
\end{center}

Although this is not completely necessary, students need practice formulating sentences with quantifiers, and it will help to reinforce the process for negating a quantified sentence in the next part.  Also, it seems to help students better understand what a solution to an equation is and how to formulate a sentence involving a solution of an equation in a precise mathematical way.

The contrapositive of the proposition is:  For all nonzero integers $a$ and $b$, if the equation 
$ax^3 + bx - 7 = 0$ has a solution that is a natural number, then $a + b = 7$ or $49a + b = 1$.

In setting up a proof for this contrapositive, I tell the students that I like to use a letter other than $x$ for the solution that is assumed to exist.  This also helps with the idea that a solution is a specific number that is substituted for the variable $x$.  With this in mind, following is an outline of the proof.

Let $a$ and $b$ be nonzero integers and assume that the equation $ax^3 + bx - 7 = 0$ has a solution that is a natural number.  Let $n$ be a natural number that is a solution of this equation.  Then

\begin{equation} \notag
an^3 + bn - 7 = 0.
\end{equation}

The idea is now to rewrite the equation in the form $n \left(an^2 + b \right) = 7$ and use this to conclude that $n \mid 7$.  This in turn implies that $n = 1$ or $n = 7$.  Using $n = 1$ in the equation gives $a + b = 7$.  Using $n = 7$ in the equation gives $49a + b = 1$.

\vskip6pt
I usually do this as a discussion since this is a difficult proof at this stage for most students.  One difficulty is that there is not too much that can be done with the ``show'' portion of a 
know-show table.  How do you prove that $a + b = 7$ or $49a + b = 1$?  So I usually try to ask the students questions about what they might be able to do with the equation

\begin{equation} \notag
an^3 + bn - 7 = 0.
\end{equation}

I try to get them to rewrite it in equivalent forms and then ask if any conclusions can be made from these other equations.  I will remind them that they are working with integers and that some of the tools we have developed for studying integers are the divides relation, congruence modulo $m$, and even and odd integers.  If they do not see what to do, a question such as, ``Is there anything special about the number 7,'' might help.  This will usually get a few students to realize that we can conclude that $n$ divides 7, and once that is done, the proof can usually be completed.

Finally, I usually mention that the last step of the proof technically is a proof that uses cases and that we will study this proof method in Section \ref{S:cases}.
\hbreak 


\subsection*{The Exercises}
Assign as many of these exercises as possible.  Two of Exercises~(\ref{exer:sec32-2}), (\ref{exer:sec32-4}), and~(\ref{exer:sec32-congmod7}) should be included since they contain false statements. At least one of Exercises~(\ref{exer:sec32-6}) and ~(\ref{exer:sec32-8}) should be included as they involve the proof of a biconditional statement.  Exercise~(\ref{exer:sec32-9}) is considered optional at this point since it does get quite involved, and Exercise~(\ref{exer:sec32-equation}) provides a good use of using the contrapositive.  
Exercises~(\ref{exer:rationalbetween}) through~(15) involve existence proofs.
\vskip6pt
\noindent
Typical Assignment:  Exercises 2, 3, 6, 8, 9, 12, 13, 17


\hbreak
\endinput

\section*{Section~\ref{S:contradiction} Proof by Contradiction}
Plan at least one and one-half class periods for this section.  


\subsection*{Main Topics}
Proof by contradiction and the logic that justifies the method of proof by contradiction.  After studying this section, it is a good idea to have the students read the comparison of direct proofs, proof using the contrapositive, and proofs by contradiction that starts on page~\pageref{SS:proofcompare} in the summary for Chapter~\ref{C:proofs}.

\subsection*{The Preview Activities}
\subsubsection*{Preview Activity~\ref{PA:contradicton} (Proof by Contradiction)} 
The purpose of this preview activity is to provide a logical basis for the method of proof by contradiction.  A different rationale for the method of proof by contradiction is given in 
Exercise~(\ref{exer:sec33-1}).

\subsubsection*{Preview Activity~\ref{PA:contradiction2} (Proof by Contradiction (continued))}  
In this preview activity, students are asked state the assumptions that need to be made for a  proof by contradiciton.  They may have difficulty with this.  The example in this preview activity is discussed further in Example~\ref{E:contradiction}, and a complete proof is given in Proposition~\ref{P:contradiction}.   Progress Check~\ref{pr:start-con} can be done as a classroom activity after this preview activity and Preview Activity~\ref{PA:rational} are discussed in class.  

\subsubsection*{Preview Activity~\ref{PA:rational} (Rational Numbers)}
A version of the ``classic'' proof that the square root of 2 is irrational is given in 
Theorem~\ref{T:squareroot2}.  In order to understand this proof, most students need to review rational numbers and some properties of rational numbers.  This is done in this preview activity.  In particular, students need to understand that any rational number can be written as a quotient $\dfrac{m}{n}$, where $m$ and $n$ are integers, $ n > 0 $, and $m$ and $n$ have no common factor greater than 1.
\hbreak

%\subsection*{The Activities}
%There are four activities in this section.  Do not plan to complete all of them in class.  Activity~\ref{A:contradiction} should be completed and discussed in class since it is directly related to Preview Activity~\ref{PA:contradicton}.  If they are not completed in class, Activities~\ref{A:exploreproof} and~\ref{A:lineareq} should be assigned along with the exercises.  Activity~\ref{A:quadratic} may be too long to do in class, but it is an excellent activity to use as an out of class group (or individual) assignment.

%\subsection*{Activity~\ref{A:contradiction}}
%In this activity, students are asked to complete the proof by contradiction that was started in 
%Preview Activity~\ref{PA:contradicton}.  Specific directions are given for the algebraic steps to be performed, and of course, there are other algebraic steps that could be performed to reach a contradiction.
%
%\subsection*{Activity~\ref{A:exploreproof}}
%This activity will provide pracice at working with congruence notation.  If the definitions and notations are used correctly, students can arrive at a contradiction fairly easily.

\subsection*{Activity~\ref{A:lineareq} (A Proof by Contradiction)}
One thing that I encourage students to do with a proposition such as this is to use letters other than $x$ and $y$ to represent the solution that is assumed to exist.  This is not necessary but it does help reinforce the concept of a solution of an equation.  Also, it makes a distinction between the variables in the equation and specific values substituted for the variables.

\subsection*{Activity~\ref{A:quadratic} (Exploring a Quadratic Equation)}
The proof by contradiction in this activity is more involved than the ones in the other activities.  Again, encourage the students to use a letter other than $x$ to represent an integer  solution of the equation that is assume to exist.  As indicated earlier,  this activity may be too long to do in class, but it is an excellent activity to use as an out of class group (or individual) assignment.
\hbreak

\subsection*{The Exercises}
Many of these exercises can take students a long time to complete.  Be careful not to assign too many.  Seven or eight exercises (including any of the activities) should be sufficient.  If I have time, I usually discuss Exercise~(\ref{exer:sec33-1}) in class but do not assign it.  
Exercises~(\ref{exer:sec33-2}) and~(\ref{exer:sec33-10}) are good exercises to do after 
Theorem~\ref{T:squareroot2}.  

\vskip6pt
\noindent
Typical Assignment:  Exerices 3, 4, 7, 8, 9, two parts of 14, 15 or 16.


%\hbreak
\endinput

For Exercise~(\ref{exer:sec33-6}), some students will try to use the quadratic formula.  This can work but it gets quite messy.  A proof by contradiction provides a nice alternative method for this exercise.

\section*{Section~\ref{S:cases} Using Cases in Proofs}
Plan about one class period for this section.  


\subsection*{Main Topics}
How to set up a proof using cases, absolute value.

\subsection*{The Preview Activities}
\subsubsection*{Preview Activity~\ref{PA:alogicalequiv} (A Logical Equivalency)}  
In this preview activity, students will establish the logical equivalency that justifies the use of cases in a proof.

\subsubsection*{Preview Activity~\ref{PA:propintegers} (A Property of the Integers)}
This is a continuation of Preview Activity~\ref{PA:alogicalequiv}.  Students will use cases to prove that if $n$ is an integer, then $n^2 + n$ is an even integer.
\hbreak

\subsection*{Activity~\ref{A:triangleinequality} (Proof of the Triangle Inequality)}
This activity should be covered, either as a classroom discussion, a classroom activity, or as part of homework.
\hbreak

\subsection*{The Exercises}
Make sure you assign Exercise~(\ref{exer:consecutive}).  The result in this exercise is useful in Exercises~(\ref{exer:sec33-6}), (\ref{exer:sec34-6}), and~(\ref{exer:sec34-nsquared}).  I like to assign Exercise~(\ref{exer:sec34-6}) or~(\ref{exer:sec34-nsquared}) or both.    For Exercise~(\ref{exer:sec33-6}), some students will try to use the quadratic formula.  This can work but it gets quite messy.  A proof by contradiction (using cases) provides a nice alternative method for this exercise.  Some of the parts of Exercises~(\ref{exer:absvalue}) through~(\ref{exer:moreabsvalue}) should be assigned to provide practice with working with absolute value.

\vskip6pt
\noindent
Typical Assignment:  Exercises 1, 2, 3, 6 or 7, 9, 10, 11

\hbreak
\endinput

\section*{Section~\ref{S:divalgo} The Division Algorithm and Congruence}
Plan at least one and one-half class periods for this section.  


\subsection*{Main Topics}
The Division Algorithm, using the Division Algorithm to define cases, congruence and the relation between congruence and the Division Algorithm.  

Many of the exercises in this section can take students a long time to complete, especially if they work with the Division Algorithm rather than congruence arithmetic.  I try make sure that students undersand the importance of Theorem~\ref{T:propsofcong} and how it can be used in proofs.  Proposition~\ref{P:3dividesver2} is meant to illustrate this but I also another example in class.  This example could be done as a guided activty with something like Exercises~(\ref{exer:squaremod5}) or~(\ref{exer:remainderbycong}).  For example, the following question could be posed to the class:

\begin{list}{}
\item If an integer has a remainder of 7 when divided by 12, is it possible to make any conclusion about the remainder of the cube of that integer when it is divided by 12?
\end{list}

\vskip6pt
Some students will need guidance to start with something like, ``Let $n$ be an integer and assume that $n \equiv 7 \pmod {12}$.''  Quite a bit of algebra is needed if a student uses the fact that there exists an integer $k$ such that $n = 7 + 12k$ and then tries to find $n^3$.  Instead, try to guide the students to conclude that $n^3 \equiv 7^3 \pmod {12}$.  Then they can calculate $7^3 = 343$ and determine that $343 \equiv 1 \pmod {12}$ and conclude that 
$n^2 \equiv 7 \pmod {12}$.  This means that $n^3$ has a remainder of 7 when divided by 12.

\subsection*{The Preview Activities}
\subsubsection*{Preview Activity~\ref{PA:quotients} (Quotients and Remainders)} 
This preview activity serves as an introduction to the Division Algorithm.  It is intended to make sure that students will be careful when using the terms ``quotient'' and ``remainder.''  This will be especially true when negative integers are involved.

\subsubsection*{Preview Activity~\ref{PA:congruencereview} (A Review of Congruence)} 
Since congruence is still a relatively new concept to most students, I think this review is necesary before beginning this section.
\hbreak


\subsection*{Activity~\ref{A:lasttwo} (The Last Two Digits of a Large Integer)}
Although not really mathematically significant, students seem to enjoy this activity.  It does provide them with good practice with congruences.
\hbreak

\subsection*{The Exercises}
Exercises~(\ref{exer:congto3}), 
(\ref{exer:3divprod}), and~(\ref{exer:sqrt3}) are good exercises to assign together, as are 
Exercises~(\ref{exer:squaremod5}) and~(\ref{exer:sqrt5-irrational}). 
Exercise~(\ref{exer:falsecongruence}) can frustrate some students since all parts contain a false statement.  The one in Part~(d) takes some exploration to find a counterexample.  The student has to take effort to get to $a = 5$ and $b = 11$ or be willing to use negative integers such as $a = -1$ and $b = -7$.  Exercise~(\ref{exer:sec34-9}) is a good exercise that combines the use of cases with the method of proof by contradiction.

\vskip6pt
\noindent
Typical Assignment:  Exercises 2, 3, 5, 6, 7, 11, 12, 15 or 16, two parts of 17
\hbreak

\endinput


Plan at most one class periods for this section.  Be careful not to have the students get too ``bogged down'' in the algebra in this section.  Some may find the algebra overwhelming.  The important idea is the distinction between constructive proofs and non-constructive proofs.  I have at times omitted this section and discussed the distinction between constructive and non-constructive proofs at appropriate times when studying other parts of the text.

\subsection*{Main Topics}
Constructive proofs of existence theorems and non-constructive proofs of existence theorems.  Be careful not to have the students get too ``bogged down'' in the algebra in this section.  Some may find the algebra overwhelming.  The important idea is the distinction between constructive proofs and non-constructive proofs.

\subsection*{The Preview Activities}
\subsubsection*{Preview Activity~\ref{PA:linearsystems}} 
This preview activity reviews the methods of finding the solution of a system of two linear equations in two unknowns.  One system has a unique solution, one has infinitely many solutions, and one is inconsistent.

\subsubsection*{Preview Activity~\ref{PA:circles}}  
This preview activity reviews the standard equatin for a circle and asks them to solve a system of two equations in two unknowns to find the points of intersection of two circles.
\hbreak

\subsection*{The Activities}
There are two activities in this section.

\subsubsection*{Activity~\ref{A:linearsystem}}
This activity is actually part of the proof of Theorem~\ref{T:linearsystem}.  Students at this level should be able to handle the algebra involved in this proof, but do not let them get too frustrated.  If necessary, give them a short time to get started and then complete the proof in class.

\subsubsection*{Activity~\ref{A:circles}}
This activity is quite likely too involved for an in-class activity.  It will take the students most (if not all) of a class period to complete.  It might be best to skip this activity or assign it as a outside of class activity.
\hbreak

\subsection*{The Exercises}
Four or five of these exercises should be sufficient.  Exercise~(\ref{exer:sec35-4}) provides another opportunity to use cases in a proof.  Exercise~(\ref{exer:sec35-solution}) provides a method to use the concept of divides to actually prove there is no such integer in Part~(b).  Exercise~(\ref{exer:sec35-10}) is fairly difficult.  One method to prove this is to use an idea similar to the one used in Exercise~(\ref{exer:sec35-solution}).  I often use Exercise~(\ref{exer:sec35-10}) as part of an assignment.

\vskip6pt
\noindent
Typical Assignment:  Exercises 1, 4, 5, 6, 8

\hbreak



\chapter*{Chapter~\ref{C:settheory} \\Set Theory}

\section*{Main Objectives}
\begin{itemize}
\item Provide students with a thorough understanding of the common definitions and notations used in set theory.
\item  Introduce students to the concepts of the power set of a set and the cardinality of a finite set.
\item Provide students with a thorough understanding of the common definitions and notations used to define the basic set operations.
\item  Insure that students understand Venn diagrams and are proficient in using Venn diagrams to explore possible set relationships.
\item  Develop students' ability to use the so-called ``Choose Method'' to prove set relationships and other results in mathematics.
\item Develop students' understanding of the ``Algebra of Sets.''  Develop students' ability to use the algebra of sets to prove set relationships.
\item Develop students' understanding of the Cartesian product of two sets and develop their ability to prove results about Cartesian products of sets.
\item Develop students' ability to work with indexed families of sets and operations on indexed families of sets.
\end{itemize}
\hbreak

\input{xsection41}
\input{xsection42}
\input{xsection43}
\input{xsection44}
\input{xsection45}

\chapter*{Chapter~\ref{C:induction} \\Mathematical Induction}

\section*{Main Objectives}
\begin{itemize}
\item Provide students with an understanding of the process of mathematical induction.
\item Develop the ability to construct and write proofs using mathematical induction.
\item Provide an understanding of the various forms of mathematical induction.
\item Provide an understanding of definition by recursion.
\item Develop the ability to construct and write proofs about sequences that are defined recursively.
\item Gain an understanding of the Fibonacci numbers.
\end{itemize}
\hbreak

\input{xsection51}
\input{xsection52}
\input{xsection53}

\chapter*{Chapter~\ref{C:functions} \\Functions}

\section*{Main Objectives}
\begin{itemize}
\item To understand the concept of a function including the related notation and terminology, such as domain, codomain, and range.
\item To be able to work with functions in a variety of settings and be able to recognize certain mathematical processes as functions.
\item To understand the definitions of some of the special types of functions such as injections, surjections, and bijections.
\item To be able to prove or disprove that specific functions are injections or surjections.
\item To be able to work with the concepts of injections and surjections in proofs.
\item To understand the concept of the composition of two functions and to be able to prove results dealing with the composition of functions.
\item To understand the ordered pair reprentation of a function.
\item To understand the definition of the inverse of a function and to be able to determine when the inverse of a function is itself a function.
\item To be able to write proofs of results dealing with the inverse of a function.
\end{itemize}
\hbreak

\input{xsection61}
\input{xsection62}
\input{xsection63}
\input{xsection64}
\input{xsection65}
\input{xsection91}


\chapter*{Chapter~\ref{C:equivrelations} \\Equivalence Relations}

\section*{Main Objectives}
\begin{itemize}
\item Understand the definition of a relation and be able to use the related notation and terminology, such as the domain and range of a relation.
\item Be able to explain why a function is a relation to to be able to give examples of relations that are not functions.
%\item Understand some of the standard mathematical relations.
\item Understand the definition of the inverse of a relation and to be able to prove results about relations and the inverse of a relation.
\item Understand the properties of reflexive, symmetric, or transitive relations and to be able to determine which of these properties a certain relation satisfies.
\item Understand the definition of an equivalence relation and know some standard equivalence relations including congruence modulo $n$.
\item Be able to determine whether or not a given relation is an equivalence relation.
\item Understand what an equivalence relation is and know the basic properties of equivalence classes.
\item Be able to prove some of the basic properties of equivalence classes and be able to explain why these properties mean that the equivalence classes form a partition of the underlying set.
\item Understand the definitions of addition and multiplication for the equivalence classes that form the integers modulo $n$.
\item To be able to explain the issue of ``well-defined'' when making a definition of addition or multiplication of equivalence classes.
\item Be able to use modular arithmetic to prove properties of the integers, including certain divisibility tests.
\end{itemize}
\hbreak

\input{xsection71}
\input{xsection72}
\input{xsection73}
\input{xsection74}

\chapter*{Chapter~\ref{C:numbertheory} \\Topics in Number Theory}

\section*{Main Objectives}
\begin{itemize}
\item Understand the properties of the greatest common divisor of two integers.
\item Be able to use the Euclidean Algorithm to compute the greatest common divisor of two integers and to write the greatest common divisor as a linear combination of the two integers.
\item Be able to use the properties of the greatest common divisor of two integers to prove results in number theory.
\item Understand the proof of the Fundamental Theorem of Arithmetic.
\item Solve linear Diophantine equations in two variables.
\item Understand the proof of the theorem which states the facts about solutions of linear Diophantine equations in two variables.
\end{itemize}
\hbreak

\input{xsection81}
\input{xsection82}
\input{xsection83}


\chapter*{Chapter~\ref{C:topicsinsets} \\Topics in Set Theory}

\section*{Main Objectives}
\begin{itemize}
\item Know the definition of a finite set and some of the basic properties of finite sets.
\item Understand the Pigeonhole Principle and be able to use it in appropriate settings.
\item Understand and be able to prove basic facts about infinite sets and countably infinite sets.
\item Be able to explain why $\mathbb{Z}$ and $\mathbb{Q}$ are countably infinite.
\item Be able to use Cantor's diagonal argument to explain why the open interval 
$\left( 0, 1 \right)$ is uncountable.
\item Know and be able to prove Cantor's Theorem that a set and its power set do not have the same cardinality.

\end{itemize}
\hbreak

\input{xsection92}
\input{xsection93}
\input{xsection94}

\endinput

