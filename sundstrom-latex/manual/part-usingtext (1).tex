\part{Using the Text}
\label{part:usingtext}%
\markboth{Using the Text}{Using the Text}



\emph{Mathematical Reasoning and Writing} was written to assist students with the transition from calculus to upper level mathematics courses.  Students should be able to effectively use this text with a background of one semester of calculus.  Following are some of the important ways this text will help with this transition.

\section*{Emphasis on Writing in Mathematics}

\subsection*{The Writing Guidelines}

The issue of writing mathematical exposition is addressed throughout the book.  Guidelines for writing mathematical proofs are incorporated into the book.  These guidelines are introduced when needed and begin in Chapter~\ref{C:intro}.  Appendix~\ref{C:writingguides} contains a summary of all the guidelines for writing mathematical proofs that are introduced in the text.  In addition, every attempt has been made to insure that each proof presented in this text is written according to these guidelines.  This provides students with examples of well-written proofs.

\vskip6pt
Some of the writing guidelines deal with the use of a word processor.  At Grand Valley State University, this text has been used for several years for a course called ``Communicating in Mathematics.''  It is a sophomore level course with a Calculus I prerequisite that is required for all mathematics majors and minors.  For this course, the Department of Mathematics requires that students use a word processor capable of producing the appropriate mathematical symbols and equations.  (Microsoft Word and its equation editor is available on the student network.)  However, the writing guidelines can easily be implemented for courses where students do not have access to this type of word processing.

\subsection*{A Writing Assignment}

At Grand Valley, we do not require that every proof a student completes must be written on a word processor.  There is one major assignment where a word processor is required.  Many of us call this the ``Proofs Portfolio.''  Since our course is part of the University's Writing Skills Program, students must have an opportunity to work on a writing assignment, get feedback from the instructor, and then have the opportunity to revise their work.  We use the ``Proofs Portfolio'' to satisfy this requirement.  This portfolio consists of ten proofs (or propositions to be proven or disproven).  Students may hand in each proof to the professor two times to be critiqued.  Most of the critique is directed toward the student's writing, but we must occasionally give some ``mathematical direction'' to the student.  However, most of the time this is quite general, such as, ``You have an algebra mistake here.''

\vskip6pt
The goal is that each student will have a completed ``Portfolio of Proofs'' at the end of the semester.  The eight to ten proofs in the portfolio are chosen to illustrate the various proof techniques discussed in the course.  Hopefully, the students will be able to use their portfolios to provide examples of various proof techniques if they are required in later courses.

\vskip6pt
Sample instructions and sample portfolio problems that I have used are given later in this part of the Instructor's Manual.
\hbreak


\section*{Instruction in the Process of Constructing Proofs}

One of the primary goals of this book is to develop a student's ability to construct mathematical proofs and then to write the proof in a coherent manner that conveys an understanding of the proof to the reader.  These are two distinct skills.  Instruction on how to write proofs begins in Section~\ref{S:direct} and is developed further in Chapter~\ref{C:proofs}.  In addition, Chapter~\ref{C:induction} is devoted to developing students' abilities to construct proofs using mathematical induction.  

\subsection*{Know-Show Tables}

Students are taught to organize their thought processes when attempting to construct a proof with a so-called ``know-show table.'' (See Section~\ref{S:direct} and Section~\ref{S:directproof}.)  Students use this table to work backward from what it is they are trying to prove while at the same time working forward from the assumptions of the problem.  The know-show tables are used quite extensively in Chapter~\ref{C:proofs}.  However, the text gradually decreases the explicit use of know-show tables in the later chapters.  One reason for this is that these tables may work well when there appears to be only one way of proving a certain result.  As the proofs become more complicated or other methods of proof (such as proofs using cases) are used, these know-show ables become less useful.


So, the know-show ables are not to be considered an absolute necessity in using the text.  However, they are useful for students beginning to learn how to construct and write proofs.  They provide a convenient way for students to organize their work.  More importantly, they introduce students to a way of thinking about a problem.  Instead of immediately trying to write a complete proof, the know-show table forces students to stop, think, and ask questions such as:

\begin{itemize}
\item Just exactly what is it that I am trying to prove?
\item How can I prove this?
\item What methods do I have that may allow me to prove this?
\item What are the assumptions?
\item How can I use these assumptions to prove the result?
\end{itemize}

Being able to ask these questions is a big step in constructing a proof.  The next task is to answer the questions and to use those answers to construct a proof.
\hbreak

\section*{Emphasis on Active Learning}

One of the underlying premises of this text is that the best way to learn and understand mathematics is to be actively involved in the learning process.  However, it is unreasonable to expect students to just go out and learn mathematics on their own.  Students actively involved in learning mathematics need appropriate materials that will provide guidance and support in their learning of mathematics.  This text provides these by incorporating two or three preview activities for each section and some activities within each section based on the material in that section.  These activities can be done individually or in a collaborative learning setting where students work in groups to brainstorm, make conjectures, test each others' ideas, reach consensus, and hopefully, to develop sound mathematical arguments to support their work.

\subsection*{Using and Grading the Preview Activities}

I require students to complete all the preview activities for a section prior to the classroom discussion of that section.  However, it is quite possible to have students complete one of the preview activities in class (perhaps in small groups) prior to the discussion.  It would also be possible to incorporate the material in one or two of the preview activities into a classroom discussion or lecture.

I do not grade all the preview activities.  Although students are required to complete all of them, on the day of class, I announce which one of the them I will collect and grade.  Grading is not based on whether or not everything is correct, but rather on whether or not a serious and substantial effort was made to complete the preview activities.  Each Preview Activity that is graded will receive a score of 4 points, 2 points, or 0 points.  This provides enough incentive for students to complete the preview activities without burdening me with an enormous amount of grading.  I can usually grade a preview activity in ten to fifteen minutes.  

The grading is also made easier by the fact that I distribute solutions to the preview activities for a section after the one is handed in.  I usually do this by posting the solutions on my web page.  The Instructor's Manual for this text contains the solutions of all preview activities and Adobe pdf files containing the solutions are available to the instructor.

However it is done, the Preview Activities at the beginning of each section should be completed by the students prior to the classroom discussion of the section.  The purpose of the preview activities is to prepare the students for the classroom discussion of the section.  By completing these preview activities, the students will be better prepared to participate in the classroom discussion.  Some preview activities will review prior mathematical work that is necessary for the new section.  This prior work may contain material from previous mathematical courses or it may contain material covered earlier in this text.  Other preview activities will introduce new concepts and definitions that will be used when that section is discussed in class.

\subsection*{Using the Progress Checks}
Students should work on the progress checks as they are studying the material in the book.  The progress checks are either short exercises or short activities designed to help the students determine if they are understanding the material as it is presented.  Some progress checks are also intended to prepare the student for the next topic in the section.  Answers to the Progress Checks are provided at the end of each chapter.

%\hbreak

\subsection*{Using the Activities}

In addition to the Preview Activities, each section of the text contains one or two activities related to the material contained in that section.  These activities are located at the end of each section and can be used for in-class group work or can be assigned as homework in addition to the exercises at the end of each section.  Most of the activities are optional, but they do provide useful work and sometimes thought provoking material.  The instructor must choose which activities to do in class and which to assign for out of class work.  Comments about using each activity is included in Part~\ref{part:activities} of this manual.
\hbreak

%\item \textbf{Mathematical Content}

%Mathematical content is needed as a vehicle for learning how to construct and write proofs.  The mathematical content for this text is drawn primarily from elementary number theory including congruence arithmetic; elementary set theory; functions, including injections, surjections, and the inverse of a function; and relations and equivalence relations.  This material is needed for upper level mathematics courses.

\section*{The Role of Logic}

In order to learn how to construct mathematical proofs, students need to learn some logic and gain experience in the traditional language and proof methods used in mathematics.  Since this is a text that deals with constructing and writing mathematical proofs, the logic that is presented is intended to aid in the construction of proofs.  The goals are to provide students with a thorough understanding of conditional statements, quantifiers, and logical equivalencies.  Emphasis is placed on writing correct and useful negations of statements, especially those involving quantifiers.  The logical equivalencies that are presented are those that provide the logical basis for some of the standard proof techniques such as proving the contrapositive, proof by contradiction, and proof using cases.
\hbreak

\section*{An Example of a Course Schedule}
As is mentioned in the Preface of the text, the chapters in the text can roughly be divided into the following (possibly overlapping) classes:

\begin{itemize}
\item Constructing and Writing Proofs:  Chapters~\ref{C:intro}, \ref{C:proofs}, and~\ref{C:induction}
\item Content: Chapters~\ref{C:settheory}, \ref{C:functions}, \ref{C:equivrelations}, \ref{C:numbertheory}, and~\ref{C:topicsinsets}
\item Logic: Chapter~\ref{C:logic}
\end{itemize}

A standard one-semester course in constructing and writing proofs should cover the first six chapters of the text and at least one of Chapter~\ref{C:equivrelations}, Chapter~\ref{C:numbertheory}, or Chapter~\ref{C:topicsinsets}.  A class consisting of well-prepared and motivated students could cover two of the last three chapters.  In addition, there are a few options that an instructor could choose to tailor the course to her or his needs.  For example,

\begin{itemize}
\item Chapter~\ref{C:induction} can be covered before Chapter~\ref{C:settheory} if it is desired to cover all methods of proof before beginning the ``content'' portion of the course.  The only part of Chapter~\ref{C:induction} that would need to be skipped is the material in Section~\ref{S:otherinduction} dealing with the cardinality of the power set.  If desired, this material could be included when the power set is discussed in Chapter~\ref{C:settheory}.

\item Instructors who would like to cover topics in both Chapters~\ref{C:equivrelations} and~\ref{C:numbertheory} can omit a few selected sections from earlier chapters.  Although it is an important and interesting section, Section~\ref{S:recursion} is not used in the remainder of the book.  The same is true for Section~\ref{S:inversefunctions}.  Finally, 
Section~\ref{S:constructive} can be skipped as long as the concept of a constructive proof is discussed during other parts of the course.
\end{itemize}

Following is a way I have implemented a schedule for a one-semester course that meets 3 times per week for 50 minutes for 14 weeks.  This schedule includes the material from Chapters~1 through~6 and Chapter 7.  It is possible to substitute Chapter~8 or Chapter~9 for Chapter~7.

\begin{center}
\begin{tabular}[h]{| c | c | c | l |} \hline
     &  Previews &            &            \\
 Day &  Due      &  In Class  &  Exercises \\  \hline
  1  &           &  Section 1.1 &            \\  
     &           &  Preview 1   &            \\ \hline
  2  &  Section 1.1  &  Section 1.1  &  1, 2, 4, 5, (6, 7, or 8), 9  \\ \hline
  3  &  Section 1.2  &  Section 1.2  &  1, 2, 3(b), 4, 5(a), 6, 7, 9, 10(a)\\ \hline
     &               &               &               \\ \hline
  4  &  Section 2.1  &  Section 2.1  &  1, 3, 4, 5  \\ \hline
  5  &  Section 2.2  &  Section 2.2  &  1, 2, 4, 5, 6, 7, 8  \\ \hline
  6  &  Section 2.3  &  Section 2.3  &  1, 2, 3, 4(a),5(a), 6, 7, 8, 9 \\
     &               &               &                 \\ \hline
     &               &               &                 \\ \hline
  7  &  Section 2.4  &  Section 2.4  &  1(a, b, d), 2(a, c, d, f, h), \\
     &               &               &  3(b, c, f), 4(b, c, f), 6 \\ \hline
  8  &               &  Section 2.4  &  \\
     &  Section 3.1  &  Section 3.1  &  1(a), 3, 4  \\ \hline
  9  &               &  Section 3.1  &  6, 7, 9, 10  \\ \hline
     &               &               &  \\ \hline
 10  &               &  Section 3.1  &  11(a)  \\
     &  Section 3.2  &  Section 3.2  &  1, 2, 4 \\ \hline
 11  &               &  Section 3.2  &  6, 7, 9  \\ \hline
 12  &  Section 3.3  &  Section 3.3  &  2, 3, 4, 6  \\ \hline
     &               &               &   \\ \hline
 13  &               &  Section 3.3  &  9, 10, 12  \\ \hline
 14  &  Section 3.4  &  Section 3.4  &  1, 2, 3, 6 \\ \hline
 15  &  Section 3.5  &  Section 3.4  &  7, 8, 10, 14  \\
     &               &  Section 3.5  &  1, 4, 5  \\ \hline
     &               &               &  \\ \hline
 16  &               &  Section 3.5  &  6, 8  \\
     &               &  Review       &  \\ \hline
 17  &               &  Test \#1     &  \\ \hline
 18  &  Section 4.1  &  Section 4.1  &  1, 2, 3, 5, 6, 7, 8, 10, 12  \\ \hline
     &               &               &  \\ \hline
 19  &  Section 4.2  &  Section 4.2  &  1, 3, 4, 6(a, b)  \\ \hline
 20  &               &  Section 4.2  &  7, 9(b), 10(b), 12 \\
     &               &  Section 4.3  &  2, 3  \\ \hline
 21  &  Section 4.3  &  Section 4.3  &  4, 5, 7, 10  \\ \hline
     &               &               &  \\ \hline
\end{tabular}
\end{center}


\begin{center}
\begin{tabular}[h]{| c | c | c | l |} \hline
     &  Previews &            &            \\
 Day &  Due      &  In Class  &  Exercises \\  \hline
 22  &  Section 4.4  &  Section 4.4  &  1, 2(a, c, e, g), 3, 5, 7, 9 \\ \hline
 23  &  Section 5.1  &  Section 5.1  &  1, 2, 3(a, c), 5 \\ \hline
 24  &               &  Section 5.1  &  6(c), 8, 12 \\ \hline
     &               &               &  \\ \hline
 25  &  Section 5.3  &  Section 5.3  &  2, 3, 4(a, c) \\ \hline
 26  &               &  Section 5.3  &  4(d, e, g), 5 \\ \hline
 27  &               &  Review       &  \\ \hline
     &               &               &  \\ \hline
 28  &               &  Test \#2     &  \\ \hline
 29  &  Section 6.1  &  Section 6.1  &  1, 2, 3, 5, 6  \\ \hline
 30  &               &  Section 6.1  &  8, 9  \\
     &  Section 6.2  &  Section 6.2  &  1, 2, 3  \\ \hline
     &               &               &  \\ \hline
 31  &               &  Section 6.2  &  5, 6, 7  \\ \hline
 32  &  Section 6.3  &  Section 6.3  &  1, 2, 3, 5, 7, 8  \\ \hline
 33  &               &  Section 6.3  &  9, 10, 13, 14  \\ \hline
     &               &               &  \\ \hline
 34  &  Section 6.4  &  Section 6.4  &  2, 3, 4, 5  \\ \hline
 35  &               &  Section 6.4  &  6, 8 \\
     &  Section 6.5  &  Section 6.5  &  1, 4  \\ \hline
 36  &               &  Section 6.5  &  2, 3, 6, 7  \\ \hline
     &               &               &  \\ \hline
 37  &  Section 7.1  &  Section 7.1  &   2, 4, 5, 6, 7, 9  \\ \hline
 38  &  Section 7.2  &  Section 7.2  &  2, 4, 5, 9, 10  \\ \hline
 39  &  Section 7.3  &  Section 7.3  &  2, 3, 4, 5  \\ \hline
     &               &               &  \\ \hline
 40  &  Section 7.4  &  Section 7.4  &  1, 3, 4, 9  \\ \hline
 41  &               &  Section 7.4  &  10, 13  \\ \hline
 42  &               &  Review       &  \\ \hline
\end{tabular}
\end{center}
\hbreak
\newpage

\section*{Description of the Portfolio Project}
Following is an example of the guidelines I usually use for the Portfolio Project.  Some items may need some explanantion.
\begin{itemize}
\item These guidelines are written for a semester course that meets for fourteen weeks.
\item Incentives are built into the grading system to get the students working on the portfolio problems.  Without these incentives, manys students will procrastinate and wait until the course is almost over to hand in any of the problems.  I always encourate students to hand in portfolio problems one at a time as soon as they finish one.  If they do this, I can usually return their work by the next class day.
\item At Grand Valley, we have the Blackboard software that allows us to easily create course home pages.  Within the course home page, there is a digital drop box that allows students to hand in assignments electronically.  I require that students use this and hand in their portfolio problems as MS Word files.  MS Word is available on our student network with the built-in equation editor.  I can then use MS Word's editing capabilities to insert my comments and return the files to the students via the digital drop box.
\end{itemize}

\vskip6pt
\noindent
\subsection*{Guidelines for the Portfolio Project}
During the semester, ten problems will be posed to all students.  Each student will work on these problems and submit proposed solutions to the professor at the end of the semester. You may bring  each of your portfolio proofs to the professor two times to be critiqued.  The professor will make recommendations about these solutions (such as, "Start over", "You forgot the initial step in your induction proof", "This is not the way to start a proof by contradiction", "Very good, but improve the writing and correct spelling errors", "Wonderful - don't make any changes").  If necessary, students should then consider rewriting and resubmitting their proofs for further comment.  Basically, when you submit a proof, you are asking the professor, "Is this good enough for my portfolio?"  You do not have to do the proofs in order.

\subsubsection*{Grading of the Portfolio Project}
The Portfolio will be worth a total of 120 points. Each problem will be worth 10 points (for a total of 100 points), and there will be 20 points possible for submission of proofs for review by the professor.  To be eligible for the 20 points, a student must do all of the following:

\begin{itemize}
\item Submit the first draft of a portfolio problems by the end of the fourth week of the course;
\item Submit the first draft of a second portfolio problem (different from the first) by the end of the fifth week of the course; and 
\item Submit the first draft of a third portfolio problem (different than the first two) by the end of the sixth week of the course.
\end{itemize}

Each problem in your portfolio will be graded on a 10-point scale with the only possible grades being 10, 9, 6, 3, or 0 points.  There will be little partial credit because of the opportunity to submit problems for review, to re-write, and to re-submit. In order to receive full credit for a problem, your solution must be correct, complete, and well written.  Following is a description of the 10-point scale for grading each problem:

\begin{center}
\begin{tabular}[h]{| c | p{3in} |} \hline
Points  &  Description \\ \hline
10      &  The proof or solution is correct and well written according to the course guidelines.  \\ \hline
 9      &  The proof or solution is correct but there is a writing mistake.  \\ \hline
 6      &  The proof or solution is essentially correct but the solution is not written according to the guidelines.  \\ \hline
 3      &  Significant progress has been made in developing and writing the proof or solution. \\ \hline
 0      &  Little or no progress has been made in developing a proof for solution.  \\ \hline
\end{tabular}
\end{center}

In addition, there will be ten ``extra credit points'' available for the Portfolio Project.  These ten points  will be awarded to each student who has received a score of 10 on three portfolio problems by the end of the eleventh week of the course.

\subsubsection*{Honor System}
All work that you submit for the Portfolio Project must be your own work.  This means that you may not discuss the portfolio project with anyone except the instructor of the course.

This will also provide me with information regarding how students are doing with each problem.  So, if I find that a particular problem is causing more difficulties that anticipated, I can send a email message to all students with hints or points of clarification for that problem.

\subsubsection*{Electronic Submission of Portfolio Problems}
Each solution or proof must be done on a word processor capable of producing the appropriate mathematical symbols and equations.   Microsoft Word and its Equation Editor, which is available on the student network, is one such word processor.

Each solution or proof for a portfolio problem must be submitted to the instructor electronically through the Digital Drop Box that is on the course web page (through Grand Valley's Blackboard system).  The instructor will make comments on the problem and return them to the student using the Digital Drop Box.


\subsection*{Examples of Portfolio Problems}
Since a draft of a portfolio problem is expected by the end of the fourth week, and we do not start Chatper~\ref{C:proofs} until the end of the third week, I always include one problem that can be started after completing Chapter~\ref{C:intro}.  Also, many of the problems that I have included in past Portfolio Projects have been included in the exercises in the text.  I sometimes use Exercises from later in the text since they can often be done with the tools developed in Chapter~\ref{C:proofs} with perhaps a little extra guidance.

\subsubsection*{Problems that Can Be Started After Chapter~\ref{C:intro}.}

\begin{enumerate}
\item (Exercise~(\ref{exer:sec12-11}) in Section~\ref {S:direct}) Find all solutions of two quadratic equations of the form  $ax^2  + bx + c = 0$
 where  $a$, $b$, and  $c$  are real numbers, $a > 0$, and   $c < 0$.

Prove or disprove the following:
If  $a$, $b$, and  $c$  are real numbers with $a > 0$ and   $c < 0$, then at least one solution of the quadratic equation  $ax^2  + bx + c = 0$ is a positive real number.


\item (Similar to Exercise~(\ref{exer:sec12-morepythag}) in Section~\ref {S:direct})  The \textbf{Pythagorean Theorem} for right triangles states that if $a$ and $b$ are the lengths of the legs of a right triangle and $c$ is the length of the hypotenuse, then 
$a^2 + b^2 = c^2$.  For example, if the lengths of the legs of a right triangle are 4 and 7 units, then $c^2 = 4^2 + 7^2 = 63$, and the length of the hypotenuse must be $\sqrt{13}$ units (since the length must be a positive real number).

\eighth
\noindent
Prove that if $m$ is a real number the lengths of the three sides of a right triangle are $m$, $m + 7$ and $m + 8$ units, then the length of the hypotenuse must be 13 units.

\item (Part of Exercise~(\ref{exer:sec32-6}) in Section~\ref{S:moremethods}) For a right triangle, suppose that the hypotenuse has length  $c$  feet and the lengths of the sides are  $a$  feet and  $b$  feet.  If the area of the right triangle is  
$\dfrac{1}{4}c^2$, then the triangle is an isosceles triangle.

\item \begin{enumerate}
\item Is the following proposition true or false?  Justify your conclusion.

If  $x$  and  $y$  are real numbers and $xy > 0$, then  
$\dfrac{{x + y}}{2} \ge \sqrt {x{\kern 1pt} y}$.

\item If the proposition is true, write a complete proof for the proposition.  If it is false, add a reasonable condition to the hypothesis so that the new proposition is true.  Then, write a complete proof of this new proposition.

\end{enumerate}
\item Is the following proposition true or false?  Justify your conclusion.

For each integer  $n$,  $4n^2 - 6n + 3$  is an odd integer.

\item Let  $x$  be a real number. If  $0 < x < \dfrac{\pi }{2}$, then  
$\left( {\sin x + \cos x} \right) > 1$.

\underline{Note}:  This is Exercise~(\ref{exer:sec33-11}) in Section~\ref{S:contradiction}.  However, it can also be proven using a direct proof.  Many students shy away from this problem since it involves trigonometric identities.

\end{enumerate}

\subsubsection*{Problems that Can Be Started After Chapter~\ref{C:proofs}.}
\begin{enumerate}
\item Is the real number $\sqrt {12}$  a rational number or an irrational number?  Justify your conclusion.

\item (Exercise~(\ref{exer:sec32-equation17}), Section~\ref{S:moremethods})
\begin{enumerate}
\item Give examples of at least two different equations of the form  $ax^3  + bx + (b + a) = 0$   where $a$  and  $b$  are integers,  $a$  does not divide  $b$, and   $a$  is not equal to zero.

\item Find decimal approximations of all real number solutions of each of the equations from 
Part~(a).  Note:  You could use Maple or a graphing calculator to find these approximate solutions.

\item Assume that  $a$  and  $b$  are integers with  $a$  not equal to zero.  Consider the following statement:

If  $a$  does not divide  $b$, then the equation  $ax^3  + bx + (b + a) = 0$  has no solution that is a natural number.

Is this statement true or false?  Justify your conclusion.
\end{enumerate}

\item Is the following proposition true or false?  Justify your conclusion.

For all nonzero integers   $a$  and  $b$, if  $a + b \ne 7$  and  $49a + b \ne 1$, then the equation  $an^3  + bn - 7 = 0$ has no natural number solution.

\item Does the equation $n^7  - 3n^4  - 9n - 7 = 0$ have a solution that is a natural number?  Either find a natural number solution or prove that none exists.

\item Is the following proposition true or false?  Justify your conclusion.

Let  $n$  be a natural number.  If  3  does not divide  $\left( {n^2  + 2} \right)$, then  $n$  is not a prime number or  $n = 3$.


\item (Exercise~(\ref{exer:sec35-10}), Section ~\ref{S:contradiction}.) Prove or disprove the following:

There exist three consecutive natural numbers such that the cube of the largest one is equal to the sum of the cubes of the other two.

\item If  $n$  is a natural number and $m = n + 1$, then  $n$  and  $m$  are said to be consecutive natural numbers.  If  $n$  is an odd natural number and $m = n + 2$, then  $n$  and  
$m$  are said to be consecutive odd natural numbers.

Notice that  3, 5, and  7  are three consecutive odd natural numbers, all of which are prime.  Are there any others?  Either find three other consecutive odd natural numbers, all of which are prime, or prove that, except for 3, 5, and 7, every triple of consecutive odd natural numbers contains at least one composite number.

\item (Exercise~(\ref{exer:sec82-twinprimes}), Section~\ref{S:primefactorizations}) Give examples of three different pairs of prime numbers that differ by two.  Such pairs of numbers are said to be twin primes.  Calculate the product of  each of your examples of twin primes.  Is the following proposition true or false:

If  $p $ and  $q$  are twin primes other than the pair 3 and 5, then $pq + 1$ is a perfect square that is divisible by 36.

\item If  $x$, $y$, and  $z$  are natural numbers such that  $x^2  + y^2  = z^2 $
 and  $\gcd \left( {x, y, z} \right) = 1$, then exactly one of the natural numbers  $x$  and  $y$  is odd.

Note:  The greatest common divisor of three integers is the largest natural number that is a divisor of all three integers.  For example:

\begin{center}
$\gcd \left( {4, 20, 30} \right) = 2{\rm{   and   }}\gcd \left( {4, 20, 25} \right) = 1$.
\end{center}

\item \begin{enumerate}
\item Is the following proposition true or false?  Justify your conclusion.

For each integer  $n$,  if  $n$  is an odd integer, then  $n^2  \equiv 1 \pmod 8$.

\item Is the following proposition true or false?  Justify your conclusion.

For each integer  $n$, $n^2  \equiv 1 \pmod 8$  or  $n^2  \equiv 4 \pmod 8$.
\end{enumerate}

\item Is the following proposition true or false?  Justify your conclusion.

For all $a, b \in \mathbb{Z}$, if  $\left( {a^2  + b^2 } \right) \equiv 0 \pmod 3$, then  
$a \equiv 0 \pmod 4$  and  \linebreak $b \equiv 0 \pmod 3$.

\item Following are several examples of ordered triples  $\left( {x, y, z} \right)$
  where  $x$, $y$, and  $z$  are natural numbers that have no common factor except 1 and  
$x^2  + y^2  = z^2$.

\begin{multicols}{3}
$\left( {3,\,4,\,5} \right)$

$\left( {5,\,12,\,13} \right)$
	
$\left( {8,\,15,\,17} \right)$

$\left( {7,\,24,\,25} \right)$

$\left( {20,\,21,\,29} \right)$

$\left( {9,\,40,\,41} \right)$

$\left( {12,\,35,\,37} \right)$

$\left( {11,\,60,\,61} \right)$

%$\left( {28,\,45,\,53} \right)$

$\left( {33,\,56,\,65} \right)$

%$\left( {16,\,63,\,65} \right)$
\end{multicols}

Is the following statement true or false?  Justify your conclusion.

If   $x$, $y$, and  $z$  are natural numbers that have no common factor except 1 and  
$x^2  + y^2  = z^2 $, then one of  $x$ , $y$,  and  $z$  is divisible by 5.


\end{enumerate}

\subsubsection*{Problems that Can Be Started After Chapter~\ref{C:settheory}.}
\begin{enumerate}
\item (Exercise~(\ref{exer:sec43-6}), Section~\ref{S:setproperties}) Prove or disprove the following:

For any sets  $A$, $B$, and  $C$ that are subsets of a universal set  $U$, \linebreak 
$A - \left( {B \cap C} \right) = \left( {A - B} \right) \cup \left( {A - C} \right)$.

\item (Exercise~(\ref{exer:sec43-setdiff3x}), Section~\ref{S:setproperties}) Let  $A$,  $B$, and  $C$  be subsets of some universal set  $U$.  Use Venn diagrams to explore the relation between the two sets  $A - \left( {B - C} \right)$  and  
$\left( {A - B} \right) - C$. Based on these diagrams, what appears to be the relation between the sets  $\left( {A - B} \right) - C$  and   $A - \left( {B - C} \right)$?  

Formulate two propositions, each one of which states that one of these sets is (or is not) a subset of the other.  Then, justify the conclusions of these propositions.

\item (Exercise~(\ref{exer:goldbach}) in Section~\ref{S:provingset}) One of the most famous unsolved problems in mathematics is a conjecture made by Christian Goldbach in a letter to Leonhard Euler in 1742.  The conjecture made in this letter is now known as \textbf{Goldbach's Conjecture}.

State Goldbach's Conjecture and explain what it would take to prove that Goldbach's conjecture is false.  Then, prove the following:

If there exists an odd integer greater than 5 that is not the sum of three prime numbers, then Goldbach's Conjecture is false.

\item Is the following proposition true or false?  Justify your conclusion.

For any sets  $A$, $B$, and  $C$, 
$\left( {A - B} \right) \times C = \left( {A \times C} \right) - \left( {B \times C} \right)$.

If this proposition is false, you should investigate whether one of the sets is a subset of the other set.  If there is such a subset relation, you should include a proof.


\end{enumerate}


\subsubsection*{Problems that Can Be Started After Chapter~\ref{C:induction}.}
\begin{enumerate}
\item Let  $n$  be a natural number with $n \geq 3$.  A convex polygon with  $n$  sides is a polygon with  $n$  sides with the additional property that the straight line segment between any two points of the polygon lies entirely within the polygon.  So, for example, a triangle is a convex polygon with 3 sides.  In Euclidean geometry, what is the sum of the interior angles, in radians, of a triangle?

Develop a formula for the sum of the interior angles, in radians, of the interior angles of a convex polygon with  $n$  sides and prove that your formula is correct.

\item Is the following proposition true or false?  Justify your conclusion.

For each natural number $n$, 6 divides $n^3 - n$.

Note:  This proposition can be proven using induction or can be proven using cases based on congruence modulo 3.

\item (Exercise~(\ref{exer:sec52-1}), Section~\ref{S:otherinduction}) Prove the following:

For each natural number  $n$  that is greater than or equal to 3,  
$\left( {1 + \frac{1}{n}} \right)^n  < n$.

\item (Exercise~(\ref{exer:sec52-2}), Section~\ref{S:otherinduction})Make a conjecture about a formula for the product  
\[
\left( {1 - \frac{1}{4}} \right) \cdot \left( {1 - \frac{1}{9}} \right) \cdot \, \cdots \, \cdot \left( {1 - \frac{1}{{n^2 }}} \right)
\]
for all  natural numbers  $n$  with  $n \ge 2$.  Then, state a proposition and use mathematical induction to prove your proposition.

\item (Exercise~(\ref{exer:sec53-fib}), Section~\ref{S:recursion})  Prove or disprove the following:

Let  $f_1 ,\,f_2 ,\,f_3 ,\, \ldots ,\,f_m ,\, \ldots$ be the sequence of Fibonacci numbers.  Then, for all natural numbers  $n$,  $f_{5n} $  is a multiple of  5.

\item (Exercise~(\ref{exer:sec53-fib}), Section~\ref{S:recursion}) Let  
$f_1 ,\,f_2 ,\,f_3 ,\, \ldots ,\,f_m ,\, \ldots$ be the sequence of Fibonacci numbers.  Is the following proposition true or false?  Justify your conclusion.  

For each $n \in \mathbb{N}$ such that $n \not \equiv 0 \pmod 3$, $f_{n} $  is an odd natural number.



\item (Exercise~(\ref{exer:sec53-9}), Section~\ref{S:recursion})
\begin{enumerate} 
\item Compute $n!$ for the first ten natural numbers.

\item Let $a_1 = 1$, and for each natural number $k$, let
\[
a_{k + 1} = a_k + k \cdot k!.
\]
Compute $a_n$ for the first ten natural numbers.

\item Make a conjecture about a formula for $a_n$ in terms of $n$ that does not involve a summation or a recursion.

\item Prove your conjecture in Part~(c).
\end{enumerate}

\item \begin{enumerate}
\item Use mathematical induction to prove one of the following two propositions:

\begin{itemize}
\item For each natural number  $n$  that is greater than or equal to 2, 
$7^{\left( {2^n } \right)}  \equiv 1\left( {\bmod 100} \right)$�	.

\item For each natural number  $n$, $7^{4n}  \equiv 1\left( {\bmod 100} \right)$.
\end{itemize}

\item Use your proposition from Part~(a) to determine the last two digits in the decimal expansion of  $7^{331} $.  Carefully explain the procedure you used to do this.
\end{enumerate}

\item Do one of the following two problems:

\begin{itemize} 
\item For which natural numbers  $n$  do there exist natural numbers  $x$  and  $y$  such that  
$n = 4x + 5y$? 

\item For which natural numbers  $n$  do there exist non-negative integers  $x$  and  $y$  such that  $n = 4x + 5y$?
\end{itemize}
 
Use mathematical induction to prove that your conclusion is correct.

\item Is the following proposition true or false?  Justify your conclusion.

If  $a$  is any real number, then for every natural number  $n$,  
\[
\left[ {\begin{array}{*{20}c}
   1 & a  \\
   0 & 1  \\
\end{array}} \right]^n  = \left[ {\begin{array}{*{20}c}
   1 & {na}  \\
   0 & 1  \\
\end{array}} \right].
\]



\end{enumerate}

\subsubsection*{Problems that Can Be Started After Chapter~\ref{C:functions}.}
\begin{enumerate}
\item (Exercise~(\ref{exer:sec64-6}), Section~\ref{S:compositionoffunctions}). Let  $A$, $B$, and  $C$  be sets, and let  $f:A \to B$ and  $g:B \to C$ be functions.  

Prove or disprove each of the following:
\begin{itemize}
\item If the composite function  $g \circ f:A \to C$ is an injection, then the function  
$f:A \to B$  is an injection.

\item	If the composite function  $g \circ f:A \to C$ is an injection, then the function  
$g:B \to C$ is a injection.
\end{itemize}  

\item (Exercise~(\ref{exer:sec64-7}), Section~\ref{S:compositionoffunctions}). Let  $A$, $B$, and  $C$  be sets, and let  $f:A \to B$ and  $g:B \to C$ be functions.  

Prove or disprove each of the following:
\begin{itemize}
\item If the composite function  $g \circ f:A \to C$ is a surjection, then the function  
$f:A \to B$  is a surjection.

\item	If the composite function  $g \circ f:A \to C$ is a surjection, then the function  
$g:B \to C$ is surjection.
\end{itemize}  


\item \begin{enumerate}
\item Let $f: \mathbb{R} \times \mathbb{R}  \to  \mathbb{R} \times \mathbb{R}$  be defined by  
$f\left( {x,\;y} \right) = \left( {2x + y, x - y} \right)$. Is the function  $f$  an injection?  Is the function  $f$  a surjection?  Justify your conclusions.

\item Let $g: \mathbb{Z} \times \mathbb{Z} \to  \mathbb{Z} \times \mathbb{Z}$ be defined by  
$g\left( {x,\;y} \right) = \left( {2x + y, x - y} \right)$. Is the function  $g$  an injection?  Is the function  $g$  a surjection?  Justify your conclusions.
\end{enumerate}

\item Let  $M_{3, 3}$ represent the set of all  3 by 3  matrices over  $\mathbb{R}$.  \linebreak
Define  $F:M_{3, 3}  \to \mathbb{R}$  by  

\begin{center}
$F\left( {\begin{array}{*{20}c}
   a & b & c  \\
   d & e & f  \\
   g & h & i  \\
\end{array}} \right) = a^2  + e^2  + i^2  - c^2  - g^2 $
\end{center}  
for all 3 by 3 matrices  $\left( {\begin{array}{*{20}c}
   a & b & c  \\
   d & e & f  \\
   g & h & i  \\
\end{array}} \right)$
 in  $M_{3, 3} $.

Is the function  $F$  an injection?  Is the function  $F$  a surjection? Justify your conclusions.

\item Let  $M_{3,3}$ represent the set of all  3 by 3  matrices over  $\mathbb{R}$.  Define  
$D :M_{3, 3}  \to \mathbb{R}$  by  
\[
D \left( {\begin{array}{*{20}c}
   a & b & c  \\
   d & e & f  \\
   g & h & i  \\
\end{array}} \right) = aei - afh - bdi + bfg + cdh - ceg
\]
for all 3 by 3 matrices  $\left( {\begin{array}{*{20}c}
   a & b & c  \\
   d & e & f  \\
   g & h & i  \\
\end{array}} \right)$ in  $M_{3,3} $.

Is the function $D$ an injection?  Is the function $D$ a surjection? Justify your conclusions.




\end{enumerate}







\endinput
