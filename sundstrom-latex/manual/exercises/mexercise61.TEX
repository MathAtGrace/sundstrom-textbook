\section*{Section \ref{S:introfunctions} Introduction to Functions}

\begin{enumerate}
\item \begin{enumerate}
\item 
$f \left( -3 \right) = 15$, 
$f \left( -1 \right) = 3$, 
$f \left( 1 \right) = -1$, 
$f \left( 3 \right) = 3$.

\item The set of preimages of  0  is $\{0, 2 \}$. The set of preimages of 4  is   
$\left\{ \dfrac{{2 - \sqrt {20} }}{2}, \dfrac{{2 + \sqrt {20} }}{2} \right\}$.  (Use the quadratic formula.)

%\item There are no preimages of  -2 since the equation $x^2 - 2x = -2$ has no real number solutions.

\addtocounter{enumii}{1}
\item $\text{range} \left( f \right) = \left\{ y \in \mathbb{R} \mid y \geq  - 1 \right\}$.
\end{enumerate}

\item \begin{enumerate}
\item $s \left( -3 \right) = 9$,
$s \left( -1 \right) = 1$,
$s \left( 1 \right) = 1$,
$s \left( 3 \right) = 9$.




\item The set of preimages of  0  is $\{ 0 \}$. The set of preimages of 2 is 
$\{ -\sqrt{2}, \sqrt{2} \}$.

\addtocounter{enumii}{1}
\item $\text{range} \left( s \right) = \left\{ y \in \mathbb{R} \mid y \geq  0 \right\} = 
\mathbb{R}^*$.
\end{enumerate}



%\item Only (a) can be used to represent a function from  $A$  to  $B$.  In Part~(a), each element of $A$ is associated with exactly one element of $B$.  In Part~(b), the element 2 in $A$ is associated with 2 elements in $B$, and in Part~(c), the element 3 in $A$ is not associated with an element in $B$.


\item \begin{enumerate}
\item 
$f \left( -7 \right) = 10$, 
$f \left( -3 \right) = 6$, 
$f \left( 3 \right) = 0$, 
$f \left( 7 \right) = -4$.

\item The set of preimages of 5 is $\{ -2 \}$. The set of preimages of 4 is $\{ -1 \}$.

\item $\text{range} \left( f \right) = \mathbb{Z}$.
\end{enumerate}




\item \begin{enumerate}
\item 
$f \left( -7 \right) = -13$, 
$f \left( -3 \right) = -5$, 
$f \left( 3 \right) = 7$, 
$f \left( 7 \right) = 15$.

\item The set of preimages of  5 is  $\left\{ 2 \right\}$.  There set of preimages of 4 is 
the empty set.

\item $\text{range} \left( f \right) = \left\{ y \in \mathbb{Z} \mid y \text{ is odd } \right\}$.  That is, the range of $f$ is the set of all odd integers.
\end{enumerate}

\item \begin{enumerate}
\item $\text{dom}( k ) = \left\{ {x \in \mathbb{R}  \mid x \geq 3} \right\}$,  
$\text{range} ( k ) = \left\{ y \in \mathbb{R} \mid y \geq 0 \right\}$.

\item $\text{dom} ( F ) = \left\{ {x \in \mathbb{R} \mid x > \frac{1}{2}} \right\}$,  
$\text{range} ( F ) = \mathbb{R}$.

\item $\text{dom} ( f ) = \mathbb{R}$,
$\text{range} ( f ) = \left\{ y \in \mathbb{R} \mid -1 \leq y \leq 1 \right\}$.

\item $\text{dom} ( g ) = \left\{ {x \in \mathbb{R} \mid x \ne 2\text{  and  }x \ne  - 2} \right\}$, \\
$\text{range} ( g ) = \left\{ { {y \in \mathbb{R} } \mid y > 0} \right\} \cup 
\left\{ {y \in \mathbb{R} \mid y \leqslant  - 1} \right\}$.

\item $\text{dom} (G) = \R$, 
$\text{range} (G) = \left\{ y \in \R \mid 4 \leq y \leq 12 \right\}$.
\end{enumerate}



\item \begin{enumerate}
\item $d ( 1 ) = 1$, $d ( 2 ) = 2$, $d ( 3 ) = 2$, 
$d ( 4 ) = 3$, $d ( 8 ) = 4$, $d ( 9 ) = 3$

\item The only natural number $n$ such that $d ( n ) = 1$ is $n = 1$ since every other natural number has at least two divisors.  So the set of preimages of 1 is $\{1\}$.

\item The only natural numbers  $n$  such that $d( n ) = 2$   are the prime numbers. The set of preimages of the natural number  2 is the set of prime numbers.

\item The statement is false.  A counterexample is $m = 2$ and $n = 3$ since 
$d( 2 ) = 2$ and $d( 3 ) = 2$.

\item $d \left( 2^0 \right) = 1$, $d \left( 2^1 \right) = 2$, $d \left( 2^2 \right) = 3$, 
$d \left( 2^3 \right) = 4$, $d \left( 2^4 \right) = 5$, \\
$d \left( 2^5 \right) = 6$, and $d \left( 2^6 \right) = 7$.

\item Let $P \left( n \right)$ be, ``$d \left( 2^n \right) = n + 1$.''  $P \left( 0 \right)$ is true.  Let $k \in \mathbb{Z}$ with $k \geq 0$ and assume that $P \left( k \right)$ is true.  Then,
\begin {center}
$d \left( 2^k \right) = k + 1$.
\end{center}
This means that $2^k$ has $k + 1$ divisors.  Now, any divisor of $2^k$ is also a divisor of 
$2^{k+1}$.  The only other divisor of $2^{k+1}$ is $2^{k+1}$.  Thus,
\[
\begin{aligned}
d \left( 2^{k + 1} \right)&= ( k + 1 ) + 1 \\
                      &= k + 2.
\end{aligned}
\]
This proves that if $P( k )$ is true, then $P( k + 1 )$ is true.
\end{enumerate}


\item \begin{enumerate}
\item The domain of $S$ is $\mathbb{N}$, the codomain of $S$ is the power set of $\mathbb{N}$, and for each $n \in \mathbb{N}$, $S ( n )$ is the set of all the distinct natural number divisors of $n$.

\item For example:  $S( 3 ) = \left\{ {1, 3} \right\}$, 
$S( 8 ) = \left\{ {1, 2, 4, 8} \right\}$, \\
$S( {15} ) = \left\{ {1, 3, 5, 15} \right\}$.

\item For example:  $S( 3 ) = \left\{ {1, 3} \right\}$, 
$S( 11 ) = \left\{ {1,11} \right\}$, 
$S( 31 ) = \left\{ {1, 31} \right\}$.

\item $\left| S ( 1 ) \right| = 1$.  Since every natural number greater than 1 has at least 2 natural number divisors, there is no other natural number $n$ such that 
$\left| S ( n ) \right| = 1$.

\item If $n$ is a prime number, then $\left| S ( n ) \right| = 2$.

\item $d ( n ) = \left| S ( n ) \right|$.

\item The statement is true.  For example, if $m < n$, then $n \in S ( n )$ but 
$n \notin S ( m )$.  Therefore, $S ( m ) \ne S ( n )$.  Similarly, if $m > n$, then $S ( m ) \ne S ( n )$.

\item The statement is false.  For example, if $T = \left\{ 2 \right\}$, then using Part~(d), we see that there is no natural number $n$ such that $S ( n ) = T$.
\end{enumerate}
\end{enumerate}



\subsection*{Explorations and Activities}
\setcounter{oldenumi}{\theenumi}
\begin{enumerate} \setcounter{enumi}{\theoldenumi}
\item Let  $A = \left\{ {a, b, c, d} \right\}$, $B = \left\{ {a, b, c} \right\}$, and  
$C = \left\{ {s, t, u, v} \right\}$.  In each of the following exercises, draw an arrow diagram to represent your function when it is appropriate.

\begin{enumerate}
\item Define  $f:A \to C$  by  $f( a ) = s$, $f( b ) = t$, 
$f( c ) = u$, $f( d ) = v$.  The range of  $f$  is the set  $C$. 

\item Define  $f:A \to C$  by  $f( a ) = u$, $f( b ) = u$, 
$f( c ) = u$, $f( d ) = v$.  The range of  $f$  is the set   
$\left\{ {u, v} \right\}$.

\item It is not possible to have a function  $f:B \to C$ whose range is the set   $C$  since the range of such a function can contain at most 3 elements.

\item Define  $f:A \to C$  by  $f( a ) = u$, $f( b ) = u$, 
$f( c ) = u$, $f( d ) = u$.  The range of  $f$  is the set   
$\left\{ u \right\}$.

\item The function in Part (a) is a function  $f:A \to C$  that satisfies the following condition:

\begin{list}{}
\item For all  $x, y \in A$,  if  $x \ne y$, then  $f( x ) \ne f( y )$.
\end{list}

\item It is not possible to create a function  $f:A \to \left\{ {s, t, u} \right\}$ that satisfies the following condition:

\begin{list}{}
\item For all  $x, y \in A$,  if  $x \ne y$, then  $f( x ) \ne f( y )$.
\end{list}

If  $f( a ), f( b ), f( c )$ are all distinct values, then  
$f( d )$ must be equal to one of them.
\end{enumerate}


\end{enumerate}


\hbreak
\endinput


