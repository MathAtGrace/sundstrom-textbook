\section*{Section \ref{S:moreaboutfunctions} More about Functions}

\begin{enumerate}
\item \begin{enumerate}
\item $f(0) = 4$, $f(1) = 0$, $f(2) = 3$, $f(3) = 3$, $f(4) = 0$

\item $g(0) = 4$, $g(1) = 0$, $g(2) = 3$, $g(3) = 3$, $g(4) = 0$

\item The function $f$ is equal to the function $g$.
\end{enumerate}



\item \begin{enumerate}
\item $f(0) = 4$, $f(1) = 5$, $f(2) = 2$, $f(3) = 1$, $f(4) = 2$, $f(5) = 5$

\item $g(0) = 4$, $g(1) = 4$, $g(2) = 0$, $g(3) = 3$, $g(4) = 5$, $g(5)= 0$

\item The function $f$ is not equal to the function $g$.
\end{enumerate}



%\item \begin{enumerate}
%\item $f( 0 ) = 0$, $f( 1 ) = 1, f( 2 ) = 1$, 
%$f( 3 ) = 1$, $f( 4 ) = 1.$
%
%\addtocounter{enumii}{1}
%\item $f = \left\{ {( {0, 0} ), ( {1, 1} ), ( {2, 1} )( {3, 1} ), ( {4, 1} )} \right\}$.
%
%\item The function $f$ is not a constant function.
%\end{enumerate}
%
%\item \begin{enumerate}
%\item $g( 0 ) = 0$, $g( 1 ) = 1, g( 2 ) = 2$, 
%$g( 3 ) = 3$, $g( 4 ) = 1.$
%
%\addtocounter{enumii}{1}
%\item $g = \left\{ {( {0, 0} ), ( {1, 1} ), ( {2, 2} )( {3, 3} ), ( {4, 4} )} \right\}$.
%
%\item The function $g$ is equal to the indentity function on $\mathbb{Z}_5$.
%\end{enumerate}


\item \begin{enumerate}
\item $f(2) = 9$, $f(-2) = 9$, $f(3) = 14$, $f(\sqrt{2} = 7$

\item $g(0) = 5$, $g(2) = 9$, $g(-2) = 9$, $g(3) = 14$, $g(\sqrt{2}) = 7$

\item The function $f$ is not equal to the function $g$ since they do not have the same domain.

\item The function $h$ is equal to the function $f$ since if $x \ne 0$, then 
$\dfrac{x^2 + 5x}{x} = x^2 + 5$.

\end{enumerate}

\item \begin{enumerate}
\item $\left\langle {a_n } \right\rangle $  where  $a_n  = \dfrac{1}{{n^2 }}$  for each  
$n \in \mathbb{N}$.  The domain is  $\mathbb{N}$.

\item $\left\langle {a_n } \right\rangle $  where  $a_n  = \dfrac{1}{{3^n }}$  for each  
$n \in \mathbb{N}$.  The domain is  $\mathbb{N}$.

\item $\left\langle {a_n } \right\rangle $  where  $a_n  = ( -1 )^{n-1}$  for each  $n \in \mathbb{N}$.  The domain is  $\mathbb{N}$.

\item $\left\langle {a_n } \right\rangle $  where  $a_n  = \cos ( {n\pi } )$  for each  $n \in \mathbb{N}$.  The domain is  $\mathbb{N}$.  This is equal to the sequence in Part (c).
\end{enumerate}



\item \begin{enumerate}
\item
\begin{multicols}{3}
$p_1 ( 1, x ) = 1$

$p_1 ( 2, x ) = 2$

$p_1 ( 1, y ) = 1$

$p_1 ( 2, y ) = 2$

$p_1 ( 1, z ) = 1$

$p_1 ( 2, z ) = 2$
\end{multicols}

\item
\begin{multicols}{3}
$p_2 ( 1, x ) = x$

$p_2 ( 2, x ) = x$

$p_2 ( 1, y ) = y$

$p_2 ( 2, y ) = y$

$p_2 ( 1, z ) = z$

$p_2 ( 2, z ) = z$
\end{multicols}
\item $\text{range} ( p_1 ) = A$ and $\text{range} ( p_2 ) = B$

\item The statement is false.  For example, $ ( 1, x ) \ne ( 1, y )$ and \\
$p_1 ( 1, x ) = p_1 ( 1, y )$
\end{enumerate}

\item \begin{enumerate}
\item The domain of $S$ is $\mathbb{N}$, the codomain of $S$ is the power set of $\mathbb{N}$, and for each $n \in \mathbb{N}$, $S ( n )$ is the set of all the distinct natural number divisors of $n$.

\item For example:  $S( 3 ) = \left\{ {1, 3} \right\}$, 
$S( 8 ) = \left\{ {1, 2, 4, 8} \right\}$, \\
$S( {15} ) = \left\{ {1, 3, 5, 15} \right\}$.

\item For example:  $S( 3 ) = \left\{ {1, 3} \right\}$, 
$S( 11 ) = \left\{ {1,11} \right\}$, 
$S( 31 ) = \left\{ {1, 31} \right\}$.

\item $\left| S ( 1 ) \right| = 1$.  Since every natural number greater than 1 has at least 2 natural number divisors, there is no other natural number $n$ such that 
$\left| S ( n ) \right| = 1$.

\item If $n$ is a prime number, then $\left| S ( n ) \right| = 2$.

\item $d ( n ) = \left| S ( n ) \right|$.

\item The statement is true.  For example, if $m < n$, then $n \in S ( n )$ but 
$n \notin S ( m )$.  Therefore, $S ( m ) \ne S ( n )$.  Similarly, if $m > n$, then $S ( m ) \ne S ( n )$.

\item The statement is false.  For example, if $T = \left\{ 2 \right\}$, then using Part~(d), we see that there is no natural number $n$ such that $S ( n ) = T$.
\end{enumerate}

\item Let $P ( n )$ be, ``A convex polygon with $n$ sides has $\dfrac{n ( n-3 )}{2}$ sides.''  Then, $P ( 3 )$ is true since a triangle has no diagonals and hence $d ( 3 ) = 0$.
  
Let  $k \in D$  and assume that $P ( k )$ is true.  So, we assume that a convex polygon with  $k$  sides has  $\dfrac{{k( {k - 3} )}}{2}$  diagonals.   Now let  $Q$  be convex polygon with  $( {k + 1} )$ sides.  Let  $v$  be one of the 
$( {k + 1} )$ vertices of $P$  and let  $u$  and  $w$  be the two vertices adjacent to  $v$.  By drawing the line segment from  $u$  to  $w$ and omitting the vertex  $v$, we form a convex polygon with  $k$  sides.  Since we are assuming that $P ( k )$ is true, this polygon has $\dfrac{{k( {k - 3} )}}{2}$  diagonals.  These are also diagonals of the convex polygon $Q$.  The other diagonals of $Q$ are formed by connecting the vertex  $v$ to the 
$k - 1$ vertices of $Q$ that are not adjacent to $v$.  So, the total number of diagonals of $Q$ is
\[
\begin{aligned}
\frac{{k( {k - 3} )}}{2} + ( k - 1 ) &= 
\frac{k ( k - 3 ) + 2 ( k - 1 )}{2} \\
  &= \frac{k^2 - 3k + 2k - 2}{2} \\
  &= \frac{k^2 - k - 2}{2} \\
  &= \frac{( k + 1 ) ( k - 2 )}{2} \\
  &= \frac{( k + 1 ) \left[ ( k + 1 ) - 3 \right]}{2}. \\
\end{aligned}
\]
This proves that if $P ( k )$ is true, then $P ( k + 1 )$ is true.

\item \begin{enumerate}
\item $\det \left[ {\begin{array}{*{20}c}
   3 & 5  \\
   4 & 1  \\
\end{array} } \right] =  - 17, \det \left[ {\begin{array}{*{20}c}
   1 & 0  \\
   0 & 7  \\
 \end{array} } \right] = 7, \text{and det}\left[ {\begin{array}{*{20}c}
   3 & { - 2}  \\
   5 & 0  \\
 \end{array} } \right] = 10$

\item The process of finding the determinant of a 2 by 2 matrix over $\mathbb{R}$ can be thought of as a function from $\mathcal{M}_{2, 2}$ to $\mathbb{R}$.  So, the domain is 
$\mathcal{M}_{2, 2}$ and the codomain is $\mathbb{R}$.  That is, 
$\det : \mathcal{M}_{2, 2} \to \mathbb{R}$ where
\[
\det \left[ {\begin{array}{*{20}c}
   a & b  \\
   c & d  \\
\end{array} } \right] = ad - bc.
\]
\end{enumerate}

\item \begin{enumerate}
\item $\left[ {\begin{array}{*{20}c}
   3 & 5  \\
   4 & 1  \\
\end{array} } \right]^T =  \left[ {\begin{array}{*{20}c}
   3 & 4  \\
   5 & 1  \\
\end{array} } \right]$, \qquad
$\left[ {\begin{array}{*{20}c}
   1 & 0  \\
   0 & 7  \\
\end{array} } \right]^T =  \left[ {\begin{array}{*{20}c}
   1 & 0  \\
   0 & 7  \\
\end{array} } \right]$, \\
$\left[ {\begin{array}{*{20}c}
   3 & -2  \\
   5 & 0  \\
\end{array} } \right]^T =  \left[ {\begin{array}{*{20}c}
   3 & 5  \\
   -2 & 0  \\
\end{array} } \right]$

\item The process of finding the transpose of a 2 by 2 matrix over $\mathbb{R}$ can be thought of as a function from $\mathcal{M}_{2, 2}$ to $\mathcal{M}_{2, 2}$.  So, the domain is 
$\mathcal{M}_{2, 2}$ and the codomain is $\mathcal{M}_{2, 2}$.  That is, 
$\text{tran} : \mathcal{M}_{2, 2} \to \mathcal{M}_{2, 2}$ where
\[
\text{tran}\left[ {\begin{array}{*{20}c}
   a & b  \\
   c & d  \\
\end{array} } \right] = \left[ {\begin{array}{*{20}c}
   a & b  \\
   c & d  \\
\end{array} } \right]^T = \left[ {\begin{array}{*{20}c}
   a & c  \\
   b & d  \\
\end{array} } \right].
\]
\end{enumerate}


\end{enumerate}
\hbreak

\endinput
