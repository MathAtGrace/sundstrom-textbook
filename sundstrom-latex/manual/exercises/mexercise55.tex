\section*{Section \ref{S:indexfamily} Indexed Families of Sets}

\begin{enumerate}
\item For each natural number $n$, let $A_n = \{ n, n + 1, n + 2, n + 3 \}$.
\begin{multicols}{2}
\begin{enumerate}
\item $\bigcap\limits_{j=1}^{3}A_j = \{3, 4 \}$

\item $\bigcup\limits_{j=1}^{3}A_j = \{3, 4, 5, 6 \}$

\item $\bigcap\limits_{j=3}^{7}A_j = \emptyset$

\item $\bigcup\limits_{j=3}^{7}A_j = \{3, 4, 5, 6, 7, 8, 9, 10 \}$

\item $A_9 \cap \left(\:\bigcup\limits_{j=3}^{7}A_j \right) = \{9, 10 \}$

\item $\bigcup\limits_{j=3}^{7}\left( A_9 \cap A_j \right) = \{9, 10 \}$
\end{enumerate}
\end{multicols}


\item For each natural number $n$, let $A_n = \{ k \in \N \mid k \geq 7\}$.
\begin{multicols}{2}
\begin{enumerate}
\item $\bigcap\limits_{j=1}^{5}A_j = \{5, 6, 7, \ldots \, \}$

\item $\left(\: \bigcap\limits_{j=1}^{5}A_j \right)^c = \{1, 2, 3, 4 \}$

\item $\bigcap\limits_{j=1}^{5}A_j^c = \emptyset$

\item $\bigcup\limits_{j=1}^{5}A_j^c = \{1, 2, 3, 4 \}$

\item $\bigcup\limits_{j=1}^{5}A_j = \N$

\item $\left(\: \bigcup\limits_{j=1}^{5}A_j \right)^c = \emptyset$

\item $\bigcap\limits_{j \in \N}^{}A_j = \emptyset$

\item $\bigcup\limits_{j \in \N}^{}A_j = \N$
\end{enumerate}
\end{multicols}


\item \begin{enumerate}
\item $\bigcup\limits_{k \in \Lambda}^{}T_k = \{ x \in \R \mid -100 \leq x \leq 100 \}$

\item $\bigcap\limits_{k \in \Lambda}^{}T_k = \{ x \in \R \mid -1 \leq x \leq 1 \}$

\item $\bigcup\limits_{k \in \Lambda}^{}T_k = \R$

\item $\bigcap\limits_{k \in \Lambda}^{}T_k = \{ 0 \}$

\item $\bigcup\limits_{k \in \N}^{}T_k = \R$

\item $\bigcap\limits_{k \in \N}^{}T_k = \{ x \in \R \mid -1 \leq x \leq 1 \}$
\end{enumerate}


\item \begin{enumerate}
\item We let $\beta \in \Lambda$ and let $x \in A_\beta$.  Then $x \in A_\alpha$, for at least one 
$\alpha \in \Lambda$ and, hence, 
$x \in \bigcup\limits_{\alpha \in \Lambda}^{}A_\alpha$.  This proves that 
$A_\beta \subseteq \bigcup\limits_{\alpha \in \Lambda}^{}A_\alpha$.

\item We first let 
$x \in \left(\:\bigcup\limits_{\alpha \in \Lambda}^{}A_\alpha \right)^c$.  This means that $x \notin \left(\:\bigcup\limits_{\alpha \in \Lambda}^{}A_\alpha \right)$, and this means that for each $\alpha \in \Lambda$, $x \notin A_{\alpha}$.  Hence, for each $\alpha \in \Lambda$, $x \in A_{\alpha}^c$, which implies that 
$x \in \bigcap\limits_{\alpha \in \Lambda}^{}A_{\alpha}^c$.  Therefore, 
\[
\left(\:\bigcup\limits_{\alpha \in \Lambda}^{}A_\alpha \right)^c \subseteq \bigcap\limits_{\alpha \in \Lambda}^{}A_{\alpha}^c.
\]

We now let $y \in \bigcap\limits_{\alpha \in \Lambda}^{}A_{\alpha}^c$.  Then for each 
$\alpha \in \Lambda$, $y \in A_{\alpha}^c$ or $y \notin A_{\alpha}$, and therefore, 
$y \notin \left(\:\bigcup\limits_{\alpha \in \Lambda}^{}A_\alpha \right)$.  From this, we conclude that \linebreak
$y \in \left(\:\bigcup\limits_{\alpha \in \Lambda}^{}A_\alpha \right)^c$, and this proves that
\[
\bigcap\limits_{\alpha \in \Lambda}^{}A_{\alpha}^c \subseteq \left(\:\bigcup\limits_{\alpha \in \Lambda}^{}A_\alpha \right)^c.
\]
\end{enumerate}



\item \begin{enumerate}
\item We first let 
$x \in B \cap \left(\:\bigcup\limits_{\alpha \in \Lambda}^{}A_{\alpha} \right)$.  Then $x \in B$ and $x \in \bigcup\limits_{\alpha \in \Lambda}^{}A_{\alpha} $.  This means that there exists an $\alpha \in \Lambda$ such that $x \in A_\alpha$.  Hence, 
$x \in B \cap A_\alpha$, which implies that 
$x \in \bigcup\limits_{\alpha \in \Lambda}^{} \left( B \cap A_{\alpha} \right)$.  This proves that 
$B \cap \left(\bigcup\limits_{\alpha \in \Lambda}^{}A_{\alpha} \right) 
\subseteq \bigcup\limits_{\alpha \in \Lambda}^{} \left( B \cap A_{\alpha} \right)$.
 
We now let 
$y \in \bigcup\limits_{\alpha \in \Lambda}^{} \left( B \cap A_{\alpha} \right)$.  So there exists an $\alpha \in \Lambda$ such that $y \in B \cap A_{\alpha}$.  Then $y \in B$ and 
$y \in A_{\alpha}$, which implies that $y \in B$ and  
$y \in \bigcup\limits_{\alpha \in \Lambda}^{}A_{\alpha}$.  Therefore, 
$y \in B \cap \left(\:\bigcup\limits_{\alpha \in \Lambda}^{}A_{\alpha} \right)$, and this proves that 
$\bigcup\limits_{\alpha \in \Lambda}^{} \left( B \cap A_{\alpha} \right) \subseteq B \cap \left(\bigcup\limits_{\alpha \in \Lambda}^{}A_{\alpha} \right)$.

\item We first let 
$x \in B \cup \left(\:\bigcap\limits_{\alpha \in \Lambda}^{}A_{\alpha} \right)$.  Then $x \in B$ or $x \in \bigcap\limits_{\alpha \in \Lambda}^{}A_{\alpha} $.  This means for all 
 $\alpha \in \Lambda$, $x \in A_\alpha$ and, hence, for all $\alpha \in \Lambda$, 
$x \in B \cup A_\alpha$.  This implies that 
$x \in \bigcap\limits_{\alpha \in \Lambda}^{} \left( B \cup A_{\alpha} \right)$, and this proves that 
$B \cup \left(\bigcap\limits_{\alpha \in \Lambda}^{}A_{\alpha} \right) 
\subseteq \bigcap\limits_{\alpha \in \Lambda}^{} \left( B \cup A_{\alpha} \right)$.
 
We now let 
$y \in \bigcap\limits_{\alpha \in \Lambda}^{} \left( B \cup A_{\alpha} \right)$.  So for all $\alpha \in \Lambda$, $y \in B \cup A_{\alpha}$.  Then $y \in B$ or 
$y \in A_{\alpha}$.  So if $y \notin B$, then for all $\alpha \in \Lambda$, 
$y \in A_{\alpha}$, and so we can conclude that $y \in B$ or 
$y \in  \left(\:\bigcap\limits_{\alpha \in \Lambda}^{}A_{\alpha} \right)$.  Therefore, 
$y \in B \cup \left(\:\bigcap\limits_{\alpha \in \Lambda}^{}A_{\alpha} \right)$, and this proves that $\bigcap\limits_{\alpha \in \Lambda}^{} \left( B \cup A_{\alpha} \right) \subseteq B \cup \left(\:\bigcap\limits_{\alpha \in \Lambda}^{}A_{\alpha} \right)$.
\end{enumerate}



\item \begin{enumerate}
\item Using the distributive law twice, we obtain
\begin{align*}
\left(\: \bigcup\limits_{\alpha \in \Lambda}^{}A_\alpha\right) \cap \left(\: \bigcup\limits_{\beta \in \Gamma}^{}B_\beta \right) &= \bigcup\limits_{\beta \in \Gamma}^{} \left[ \left(\: \bigcup\limits_{\alpha \in \Lambda}^{}A_\alpha\right) \cap B_{\beta} \right] \\ &= 
\bigcup\limits_{\beta \in \Gamma}^{} \left[\: \bigcup\limits_{\alpha \in \Lambda}^{} \left(A_{\alpha} \cap B_{\beta} \right) \right].
\end{align*}

\item Using the distributive law twice, we obtain
\begin{align*}
\left(\: \bigcap\limits_{\alpha \in \Lambda}^{}A_\alpha\right) \cup \left(\: \bigcap\limits_{\beta \in \Gamma}^{}B_\beta \right) &= \bigcap\limits_{\beta \in \Gamma}^{} \left[ \left(\: \bigcap\limits_{\alpha \in \Lambda}^{}A_\alpha\right) \cup B_{\beta} \right] \\ &= 
\bigcap\limits_{\beta \in \Gamma}^{} \left[\: \bigcap\limits_{\alpha \in \Lambda}^{} \left(A_{\alpha} \cup B_{\beta} \right) \right].
\end{align*}
\end{enumerate}


\item \begin{enumerate}
\item Let $x \in \bigcup\limits_{\alpha \in \Gamma}^{}A_\alpha$.  Then there exists an 
$\alpha \in \Gamma$ such that $x \in A_{\alpha}$.  Since $\Gamma \subseteq \Lambda$, we conclude that $\alpha \in \Lambda$ and, hence, $x \in \bigcup\limits_{\alpha \in \Lambda}^{}A_\alpha$.

\item Let $x \in \bigcap\limits_{\alpha \in \Lambda}^{}A_\alpha$.  Then for each  
$\alpha \in \Lambda$, $x \in A_{\alpha}$.  Since $\Gamma \subseteq \Lambda$, we conclude that 
for each $\alpha \in \Gamma$, $x \in A_{\alpha}$ and, hence, 
$x \in \bigcap \limits_{\alpha \in \Gamma}^{}A_\alpha$.
\end{enumerate}



\item \begin{enumerate}
\item Let $x \in B$.  For each $\alpha \in \Lambda$, $B \subseteq A_\alpha$ and, hence, 
$x \in A_\alpha$.  This means that for each 
$\alpha \in \Lambda$, $x \in A_\alpha$  and, hence, 
$x \in \bigcap\limits_{\alpha \in \Lambda}^{}A_\alpha$.  Therefore, 
$B \subseteq \bigcap\limits_{\alpha \in \Lambda}^{}A_\alpha$.

\item Let $x \in \bigcup\limits_{\alpha \in \Lambda}^{}A_\alpha$.  So there exists an 
$\alpha \in \Lambda$ such that $x \in A_{\alpha}$.  Since $A_{\alpha} \subseteq C$, we see that 
$x \in C$ and, hence, $\bigcup\limits_{\alpha \in \Lambda}^{}A_\alpha \subseteq C$.
\end{enumerate}



\item Let $m, n \in \N$ with $m \ne n$.  Use a proof by contradiction and assume that 
$A_m \cap A_n \ne \emptyset$.  First assume that $m < n$.  There exists a real number $x$ such that $x \in A_m \cap A_n$.  From this, we conclude that $m - 1 < x < m$.  Now since $m < n$, we conclude that $m \leq n-1$ and hence, $x \leq n - 1$.  This means that $x \notin A_n$, which is a contradiction.  A similar proof can be used in the case where $n < m$.  This proves that 
if $m \ne n$, then $A_m \cap A_n = \emptyset$ and that $\left\{ A_n \left| n \in \N \right. \right\}$ is a pairwise disjoint family of sets.

To prove that $\bigcup\limits_{n \in \N}^{}A_n = \left( \R^+ - \N \right)$, we first notice that if $x \in A_n$ for some natural number $n$, then $n -1 < x < n$, and this implies that 
$x \in \R^+$ and $x \notin \N$.  Therefore, $x \in \R^+ - \N$.  From this, we conclude that 
$\bigcup\limits_{n \in \N}^{}A_n \subseteq \left( \R^+ - \N \right)$.  We now let 
$y \in \R^+ - \N$.  Let $n$ be the smallest natural number such that $y < n$.  Then, $y$ must be greater than $n - 1$ and, hence, $n - 1 < y < n$.  Therefore, $y \in A_n$ and 
$y \in \bigcup\limits_{n \in \N}^{}A_n$.  This proves that 
$\R^+ - \N \subseteq \bigcup\limits_{n \in \N}^{}A_n$.



\item \begin{enumerate}
\item This is false.  For example, if $j < k$, then $k \in A_j \cap A_k$.

\item This is true. If $\bigcap\limits_{k \in \N}^{}A_k \ne \emptyset$, then there exists a natural number $m$ such that $m \in \bigcap\limits_{k \in \N}^{}A_k$ and this implies that for every natural number $n$, $m \in A_n$ and, hence, $m \geq n$.  This is not possible.
\end{enumerate}



\item For each natural number $n$, let 
$A_n = \left\{ x \in R \left| \,\dfrac{1}{n} < x < 1 \right. \right\}$.  Then 
$\left\{ A_n \mid n \in \N \right\}$ is an indexed family of sets satisfying the three conditions.



\item \begin{enumerate}
\item We first rewrite the set difference and then use a distributive law.
\begin{align*}
\left(\:\bigcup\limits_{\alpha \in \Lambda}^{}A_{\alpha} \right) - B 
            &= \left(\:\bigcup\limits_{\alpha \in \Lambda}^{}A_{\alpha} \right) \cap B^c \\
            &= \bigcup\limits_{\alpha \in \Lambda}^{}\left( A_\alpha \cap B^c \right) \\
            &=\bigcup\limits_{\alpha \in \Lambda}^{} \left( A_{\alpha} - B \right) \\
\end{align*}

\item Let $x \in \left(\:\bigcap\limits_{\alpha \in \Lambda}^{}A_{\alpha} \right) - B$.  Then 
$x \notin B$ and for each $\alpha \in \Lambda$, $x \in A_{\alpha}$.  This means that for each 
$\alpha \in \Lambda$, $x \in A_{\alpha} - B$ and hence, 
$x \in \bigcap\limits_{\alpha \in \Lambda}^{} \left( A_{\alpha} - B \right)$.

Now let $y \in \bigcap\limits_{\alpha \in \Lambda}^{} \left( A_{\alpha} - B \right)$.  For each 
$\alpha \in \Lambda$, $y \in A_{\alpha} - B$ and hence, $y \in A_{\alpha}$ and $y \notin B$.  Therefore, $y \in \left(\:\bigcap\limits_{\alpha \in \Lambda}^{}A_{\alpha} \right) - B$.

\item We first rewrite the set difference and then use one of De Morgan's laws.
\begin{align*}
B - \left(\:\bigcup\limits_{\alpha \in \Lambda}^{}A_{\alpha} \right) &= 
B\cap \left(\:\bigcup\limits_{\alpha \in \Lambda}^{}A_{\alpha} \right)^c \\
  &= B \cap \left(\:\bigcap\limits_{\alpha \in \Lambda}^{}A_{\alpha}^c \right) \\
  &= \bigcap\limits_{\alpha \in \Lambda}^{}\left( B \cap A_{\alpha}^c \right) \\
  &= \bigcap\limits_{\alpha \in \Lambda}^{}\left( B -A \right).
\end{align*}

\item We first rewrite the set difference and then use one of De Morgan's laws and then the distributive law.
\begin{align*}
B - \left(\:\bigcap\limits_{\alpha \in \Lambda}^{}A_{\alpha} \right) &= 
B\cap \left(\:\bigcap\limits_{\alpha \in \Lambda}^{}A_{\alpha} \right)^c \\
  &= B \cap \left(\:\bigcup\limits_{\alpha \in \Lambda}^{}A_{\alpha}^c \right) \\
  &= \bigcup\limits_{\alpha \in \Lambda}^{}\left( B \cap A_{\alpha}^c \right) \\
  &= \bigcup\limits_{\alpha \in \Lambda}^{}\left( B -A \right).
\end{align*}
\end{enumerate}
\end{enumerate}



\subsection*{Explorations and Activities}
\setcounter{oldenumi}{\theenumi}
\begin{enumerate} \setcounter{enumi}{\theoldenumi}
\item \begin{enumerate}
\item $\bigcup\limits_{r \in \R^*}^{}C_r = \R \times \R$ \quad and \quad 
$\bigcap\limits_{r \in \R^*}^{}C_r = \emptyset$.

\item $\bigcup\limits_{r \in \R^*}^{}D_r = \R \times \R$ \quad and \quad 
$\bigcap\limits_{r \in \R^*}^{}D_r = \left\{ \left( 0, 0 \right) \right\}$.

\item $\bigcup\limits_{r \in \R^*}^{}T_r = \R \times \R - \left\{ \left( 0, 0 \right) \right\}$ \quad and \quad 
$\bigcap\limits_{r \in \R^*}^{}T_r = \emptyset$.

\item Only $\mathscr{C} = \left\{ C_r \mid r \in \R^* \right\}$ is a collection of pairwise disjoint sets.
\end{enumerate}

\noindent
Now let $I$ be the closed interval $[0, 2]$ and let $J$ be the closed interval $[1, 2]$.

\begin{enumerate} \setcounter{enumii}{4}
\item $\bigcup\limits_{r \in I}^{}C_r = D_2$ \qquad $\bigcap\limits_{r \in I}^{}C_r = \emptyset$

$\bigcup\limits_{r \in J}^{}C_r = D_2 - \left\{ (x, y) \in \R \times \R \mid x^2 + y^2 < 1 \right\}$ \qquad $\bigcap\limits_{r \in J}^{}C_r = \emptyset$

\item $\bigcup\limits_{r \in I}^{}D_r = D_2$ \qquad 
$\bigcap\limits_{r \in I}^{}D_r = \left\{ \left( 0, 0 \right) \right\} = D_0$

$\bigcup\limits_{r \in J}^{}D_r = D_2$ \qquad $\bigcap\limits_{r \in J}^{}D_r = D_1$


\item $\left( \bigcup\limits_{r \in I}^{}D_r \right)^c = {D_2}^c = T_2$ \qquad 
$\left( \bigcap\limits_{r \in I}^{}D_r \right)^c = {D_0}^c = T_0$
 
$\left( \bigcup\limits_{r \in J}^{}D_r \right)^c = {D_2}^c = T_2$ \qquad 
$\left( \bigcap\limits_{r \in J}^{}D_r \right)^c = {D_1}^c = T_1$.


\item $\bigcup\limits_{r \in I}^{}T_r = \R \times \R - \left\{ \left( 0, 0 \right) \right\} = T_0$ \qquad 
$\bigcap\limits_{r \in I}^{}T_r = T_2$

$\bigcup\limits_{r \in J}^{}T_r = T_1$ \qquad 
$\bigcap\limits_{r \in J}^{}T_r = T_2$


\item By De Morgan's Laws:

$\left( \bigcup\limits_{r \in I}^{}D_r \right)^c = \bigcap\limits_{r \in I}^{}{D_r}^c = \bigcap\limits_{r \in I}^{}T_r$ \\
$\left( \bigcap\limits_{r \in I}^{}D_r \right)^c = \bigcup\limits_{r \in I}^{}{D_r}^c = \bigcup\limits_{r \in I}^{}T_r$

$\left( \bigcup\limits_{r \in J}^{}D_r \right)^c = \bigcap\limits_{r \in J}^{}{D_r}^c = \bigcap\limits_{r \in J}^{}T_r$ \\
$\left( \bigcap\limits_{r \in J}^{}D_r \right)^c = \bigcup\limits_{r \in J}^{}{D_r}^c = \bigcup\limits_{r \in J}^{}T_r$
\end{enumerate}

\end{enumerate}

\hbreak
\endinput


\item We first rewrite the set difference.
\begin{align*}
\left(\:\bigcap\limits_{\alpha \in \Lambda}^{}A_{\alpha} \right) - B 
            &= \left(\:\bigcap\limits_{\alpha \in \Lambda}^{}A_{\alpha} \right) \cap B^c \\
            &= \bigcap\limits_{\alpha \in \Lambda}^{}\left( A_\alpha \cap B^c \right) \\
            &=\bigcap\limits_{\alpha \in \Lambda}^{} \left( A_{\alpha} - B \right) \\
\end{align*}
