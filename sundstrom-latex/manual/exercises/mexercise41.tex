\section*{Section \ref{S:mathinduction} The Principle of Mathematical Induction}

\begin{enumerate}
\item The sets in Parts~(a) and~(b) are inductive.  The set in Part~(c) is not inductive since 10 is in the set but 11 is not in the set.  The set in Part~(d) is not inductive since 500 is in the set but 501 is not in the set.



\item \begin{enumerate}
\item A finite non-empty subset of the integers $A$ has a largest element $M$.  Since $M \in A$ and $M+1 \notin A$, $A$ is not inductive.

\item The empty set is inductive since the statement ``For each integer $k$, if 
$x \in \emptyset$, then $x + 1 \in \emptyset$'' is a true conditional statement.  
\end{enumerate}



\item \begin{enumerate}
\item For each  $n \in \mathbb{N}$, let  $P\left( n \right)$ be, 
$2 + 5 + 8 +  \cdots  + \left( {3n - 1} \right) = \dfrac{{n\left( {3n + 1} \right)}}{2}$.  Verify that  
$P\left( 1 \right)$  is true.   For the inductive step, let $k \in \mathbb{N}$ and assume that 
$P \left( k \right)$ is true.  Then,
\[
  2 + 5 + 8 + \cdots  + \left( {3k - 1} \right) = \frac{{k \left( {3k + 1} \right)}}{2}. 
\]
We now add $3 \left( k + 1 \right) - 1$ to both sides of this equation.  This gives
\begin{align*}
  2 + 5 + 8 + \cdots  + 3 \left( {k + 1} \right)-1 &= 
           \frac{{k \left( {3k + 1} \right)}}{2} + \left( 3 \left( k + 1 \right) - 1 \right) \\   
   &= \frac{{k \left( {3k + 1} \right)}}{2} + \left( {3k + 2} \right)
\end{align*}
If we now combine the terms on the right side of the equation into a single fraction, we obtain
\begin{align*} 
 2 + 5 + 8 +  \cdots  + \left( {3k - 1} \right) + 3 \left( {k + 1} \right)-1  &= \frac{k \left( 3k + 1 \right) + 6k+4}{2} \\
   &= \frac{ 3k^2 + 7k + 4}{2} \\
   &= \frac{ \left( k + 1 \right) \left( 3k + 4 \right)}{2} \\
   &= \frac{ \left( k + 1 \right) \left( 3 \left( k + 1 \right) + 1 \right)}{2}. \\
\end{align*}
This proves that if  $P\left( k \right)$ is true, then $P\left( {k + 1} \right)$ is true.

\item For each  $n \in \mathbb{N}$, let  $P\left( n \right)$ be, 
$1 + 5 + 9 +  \cdots  + \left( {4n - 3} \right) = n(2n - 1)$.  Verify that  
$P\left( 1 \right)$  is true.   For the inductive step, let $k \in \mathbb{N}$ and assume that 
$P \left( k \right)$ is true.  Then,
\[
  1 + 5 + 9 + \cdots  + \left( {4k - 3} \right) = k(2k - 1). 
\]
We now add $4 \left( k + 1 \right) - 3$ to both sides of this equation.  This gives
\begin{align*}
  1 + 5 + 9 + \cdots  + 4 \left( {k + 1} \right)-3 &= 
           k(2k - 1) +( 4 \left( k + 1 \right) - 3) \\   
   &= k(2k - 1) +( 4k + 1).
\end{align*}
If we now combine  terms on the right side of the equation, we obtain
\begin{align*}
  1 + 5 + 9 + \cdots  + 4 \left( {k + 1} \right)-3 &= k(2k - 1) +( 4k + 1) \\
                                                   &= 2k^2 + 3k + 1\\
                                                   &= (k + 1)(2k + 1) \\
                                                   &= (k + 1)(2(k + 1) - 1).
\end{align*}
This proves that if  $P\left( k \right)$ is true, then $P\left( {k + 1} \right)$ is true.

\item For each $n \in \mathbb{N}$, let  $P\left( n \right)$ be 
$\sum\limits_{j = 1}^n {j^3 }  = \left[ {\dfrac{{n\left( {n + 1} \right)}}{2}} \right]^2$.  Verify that  $P\left( 1 \right)$  is true.   For the inductive step, let $k \in \mathbb{N}$ and assume that $P \left( k \right)$ is true.  Then,
\[
\sum\limits_{j = 1}^k {j^3 }  = \left[ {\frac{{k\left( {k + 1} \right)}}{2}} \right]^2.
\]
We now add $\left( k + 1 \right)^3$ to both sides of this equation.  This gives
\[
\begin{aligned}
\sum\limits_{j = 1}^k {j^3 } + \left( k + 1 \right)^3 &= 
\left[ {\frac{{k\left( {k + 1} \right)}}{2}} \right]^2 + \left( k + 1 \right)^3 \\
\sum\limits_{j = 1}^{k+1} {j^3} &= 
\frac{{\left[ {k\left( {k + 1} \right)} \right]^2  + 4\left( {k + 1} \right)^3 }}{4} \\
  &= \frac{\left( k + 1 \right)^2 \left( k^2 + 4k + 4 \right) }{4} \\
  &= \left[ \frac{\left( k + 1 \right) \left( k + 2 \right)}{2} \right]^2 \\
\end{aligned}
\]
This proves that if  $P\left( k \right)$ is true, then $P\left( {k + 1} \right)$ is true.
\end{enumerate}


\item $\left( {1^3  + 2^3  + 3^3  +  \cdots  + n^3 } \right) = \left( {1 + 2 + 3 +  \cdots  + n} \right)^2$.


\item \begin{enumerate}
\item We factor a 3 from the expression and obtain
\begin{align*} 3 + 6 + 9 + \cdots + 3n &=3 \left( 1 + 2 + 3 +  \cdots  + n \right) \\
                                       &= 3 \left( \frac{{n\left( {n + 1} \right)}}{2} \right) \\
                                       &= \frac{{3n\left( {n + 1} \right)}}{2}.
\end{align*}

\item Since there are $n$ terms on the left side of the equation, we can subtract $n$ from the left side by subtracting $1$ from each term.  This gives
\begin{align*}
(3 - 1) + (6 - 1) + (9 - 1) + \cdots + (3n - 1) &= \frac{{3n( {n + 1} )}}{2} - n \\
2 + 5 + 8 + \cdots (3n - 1) &= \frac{{3n( {n + 1} )}}{2} - n.
\end{align*}

\item Simplifying the result in Part~(b), we obtain
\begin{align*}
2 + 5 + 8 + \cdots (3n - 1) &= \frac{{3n( {n + 1} ) - 2n}}{2} \\
                            &= \frac{3n^2 + n}{2} \\
                            &= \frac{n(3n + 1)}{2}.
\end{align*}

\end{enumerate}

%\item \begin{enumerate}
%\item \[
%\begin{aligned} 3 \left( 1 + 2 + 3 +  \cdots  + n \right) &= 3 \left( \frac{{n\left( {n + 1} \right)}}{2} \right) \\
%  &= \frac{{3n\left( {n + 1} \right)}}{2}
%\end{aligned}
%\]
%\item $\left( {1^3  + 2^3  + 3^3  +  \cdots  + n^3 } \right) = \left( {1 + 2 + 3 +  \cdots  + n} \right)^2$.
%\end{enumerate}

\item \begin{enumerate} \setcounter{enumii}{1}
\item The conjecture is that  for each  $n \in \mathbb{N}$,  
$\sum\limits_{j = 1}^n {\left( {2j - 1} \right)}  = n^2 $.
\item The following indicates one way to prove the inductive step.
\begin{align*}
 \sum\limits_{j = 1}^{k + 1} {\left( {2j - 1} \right)}  &= \sum\limits_{j = 1}^k {\left( {2j - 1} \right)}  + \left[ {2\left( {k + 1} \right) - 1} \right] \\ 
   &= \sum\limits_{j = 1}^k {\left( {2j - 1} \right)}  + \left[ {2k + 1} \right] \\
   &= k^2 + 2k + 1 \\
   &= \left( k + 1 \right)^2.
\end{align*}
\end{enumerate}



\item \begin{enumerate} \setcounter{enumii}{4}
\item  For each natural number $n$, $4^n \equiv 1 \pmod 3$.

\item For each natural number $n$, let $P(n)$ be, ``$4^n \equiv 1 \pmod 3$.''  Since 
$4^1 \equiv 1 \pmod 3$, we see that $P(1)$ is true.  Now let $k \in \N$ and assume that 
$P(k)$ is true.  That is, assume that
\[
4^k \equiv 1 \pmod 3.
\]
Multiplying both sides of this congruence by 4 gives
\[
4^{k+1} \equiv 4 \pmod 3.
\]
However, $4 \equiv 1 \pmod 3$ and so by using the transitive property of congruence, we see that 
$4^{k+1} \equiv 1 \pmod 3$.  This proves that if $P(k)$ is true, then $P(k+1)$ is true.
\end{enumerate}



\item \begin{enumerate}
\item Let $P \left( n \right)$ be ``3 divides $\left( 4^n - 1 \right)$.''  For the inductive step, if 3 divides $ \left( 4^k -1 \right)$, then there exists an integer $m$ such that $4^k -1 = 3m$.  Hence,
\[
\begin{aligned}
4^k &= 1 + 3m \\
4^k \cdot 4 &= 4 \left( 1 + 3m \right) \\
4^{k+1} - 1 &= 3 \left(1 + 4m \right). \\
\end{aligned}
\]
So, if 3 divides $ \left( 4^k -1 \right)$, then 3 divides $ \left( 4^{k+1} -1 \right)$.

\item Let $P \left( n \right)$ be ``6 divides $\left( n^3 - n \right)$.''  For the inductive step, if 3 divides $ \left( k^3 -k \right)$, then there exists an integer $m$ such that $k^3 - k = 6m$.  Hence,
\[
\begin{aligned}
\left( k + 1 \right)^3 - \left( k  + 1 \right) &= \left( k^3 + 3k^2 + 3k + 1 \right) - k - 1 \\
  &= \left( k^3 - 1 \right) + 3k \left( k + 1 \right) \\
  &= 6m + 3k \left( k + 1 \right) \\
\end{aligned}
\]
The next step is to prove that $k \left( k + 1 \right)$ is an even number.  Then, there exists an integer $q$ such that $k \left( k + 1 \right) = 2q$.  We then see that
\[
\begin{aligned}
\left( k + 1 \right)^3 - \left( k  + 1 \right) &= 6m + 6q \\
                                               &= 6 \left( m + q \right). \\
\end{aligned}
\]
This proves that if 3 divides $k^3 - k$, then 3 divides 
$\left( k + 1 \right)^3 - \left( k + 1 \right)$.
\end{enumerate}



\item $4^n  \equiv 1 \pmod 3$  if and only if  3 divides $4^n - 1$.


\item Let $P \left( n \right)$ be ``3 divides $(n^3 + 23n)$.''  Verify that $P(1)$ is true.  For the inductive step, if $k \in \N$ and $P(k)$ is true, then 3 divides 
$(k^3 + 23k) $.  So there exists an integer $m$ such that $k^3 +23k = 6m$.  Hence,
\begin{align*}
( k + 1 )^3 + 23( k  + 1) &= \left( k^3 + 3k^2 + 3k + 1 \right) +23(k + 1) \\
  &= (k^3 + 23k) + (3k^2 + 3k + 24) \\
  &= 6m + 3( k^2 + k + 8) \\
  &= 3(2m + k^2 + k + 8).
\end{align*}
This proves that if $P(k)$ is true, then $P(k+1)$ is true.

In Exercise~(18) from Section 3.5, it was proved that 3 divides $(n^3 + 23n)$ for all integers $n$.  Using induction, this was proved only for all natural numbers.



\item \begin{enumerate} \setcounter{enumii}{1}
\item The conjecture is that 3 divides $5^n - 2^n$.

\item If 3 divides $5^k - 2^k$, then there exists an integer $m$ such that \\
$5^k - 2^k = 3m$.  Then
\[
\begin{aligned}
5^k &= 3m + 2^k \\
5^{k+1}  &= 5 \left( 3m + 2^k \right) \\
         &= 15m + 5 \cdot 2^k \\
         &= 15m + \left( 3 + 2 \right) \cdot 2^k \\
         &= 3 \left( 5m + 2^k \right) + 2^{k+1}. \\
\end{aligned}
\]
So $5^{k+1} - 2^{k+1} = 3 \left( 5m + 2^k \right)$.  This proves that if 3 divides $5^k - 2^k$, then 3 divides $5^{k+1} - 2^{k+1}$.
\end{enumerate}


\item If $(x - y)$ divides $\left( x^k - y^k \right)$, then there exists an integer $m$ such that \\
$\left( x^k - y^k \right) = m(x - y)$.  Then
\[
\begin{aligned}
x^k &= m(x - y) + y^k \\
x^{k+1}  &=  x\left( m(x - y) + y^k \right) \\
         &= mx(x - y) + xy^k \\
         &= mx(x - y) + \left( (x - y) + y \right)y^k \\
         &= mx(x - y) + (x - y)y^k + y^{k+1} \\
x^{k+1} - y^{k+1} &= (x - y)\left( mx + y^k \right).
\end{aligned}
\]
This proves that if $(x - y)$ divides $x^k - y^k$, then $(x - y)$ divides 
$x^{k+1} - y^{k+1}$, which proves the inductive step for a proof by induction.

If we use $x = 5$ and $y = 2$, we will get the result in Exercise~(11).



\item The Basis Step is the assumption that   $a \equiv b \pmod n$.  For the Inductive Step, assume that $a^k \equiv b^k \pmod n$.  Then, use Property~(\ref{T:propsofcong2}) of Theorem~\ref{T:propsofcong} to conclude that
\[
\begin{aligned}
a^k \cdot a &\equiv b^k \cdot b \pmod n \\
a^{k+1} &\equiv b^{k+1} \pmod n. \\
\end{aligned}
\]
This proves the Inductive Step.



\item Let $P(n)$ be, ``9 divides $\left( n^3 + (n + 1)^3 + (n + 2)^3 \right)$.  Then $P(1)$ is true since $1^3 + 2^3 + 3^3 = 36$.  So let $k$ be a natural number and assume that 
$P(k)$ is true.  This means that there exists a natural number $m$ such that
\[
k^3 + (k + 1)^ + (k + 2)^3 = 9m.
\]
We do the following algebra for $P(k + 1)$.
%\begin{align*}
%(k + 1)^3 + (k + 2)^3 + (k + 3)^3 &= (k + 1)^3 + (k + 2)^3 + \left(k^3 + 9k^2 + 27k + 27 \right) \\
%          &= k^3 + (k + 1)^3 + (k + 2)^3 + 9\left( k^2 + 3k + 3 \right) \\
%          &= 9m + 9\left( k^2 + 3k + 3 \right) \\
%          &= 9 \left( m + k^2 + 3k + 3 \right).
%\end{align*}
\begin{align*}
(k + 1)^3 + (k + 2)^3 &+ (k + 3)^3 \\
 &= (k + 1)^3 + (k + 2)^3 + \left(k^3 + 9k^2 + 27k + 27 \right) \\
          &= k^3 + (k + 1)^3 + (k + 2)^3 + 9\left( k^2 + 3k + 3 \right) \\
          &= 9m + 9\left( k^2 + 3k + 3 \right) \\
          &= 9 \left( m + k^2 + 3k + 3 \right).
\end{align*}
The last equation can be used to prove that if $P(k)$ is true, then $P(k + 1)$ is true.


%\item Let $P(n)$ be the predicate, ``$5^n \equiv 1 \pmod 4$.''  $P(1)$ is true since 
%$5 \equiv 1 \pmod 4$.  So let $k$ be a natural number and assume that $P(k)$ is true.  So
%\[
%5^k \equiv 1 \pmod 4.
%\]
%We now multiply both sides of this congruence by 5 and obtain $5^{k+1} \equiv 5 \pmod 4$.  Therefore, $5^{k+1} \equiv 1 \pmod 4$ and this proves that if $P(k)$ is true, then 
%$P(k+1)$ is true.



\item %\begin{multicols}{2}
\begin{enumerate}
\item $\dfrac{{d^2 y}}{{dx^2 }} = a^2 e^{ax}$

\item $\dfrac{{d^3 y}}{{dx^3 }} = a^3 e^{ax}$
%\end{enumerate}
%\end{multicols}

%\begin{enumerate} \setcounter{enumii}{2}
\item The conjecture is that $\dfrac{{d^n y}}{{dx^n }} = a^n e^{ax}$.

\item $P \left( n \right)$ is ``$\dfrac{{d^n y}}{{dx^n }} = a^n e^{ax}$.''
If $\dfrac{{d^k y}}{{dx^k }} = a^k e^{ax}$, then 
\begin{align*}
\frac{{d^{k+1} y}}{{dx^{k+1} }} &= \frac{d}{dx} \left( \dfrac{{d^k y}}{{dx^k }} \right) \\
  &= \frac{d}{dx} \left( a^k e^{ax} \right) \\
  &=  a^k  \frac{d}{dx} \left( e^{ax} \right) \\
  &=  a^{k+1} e^{ax}.
\end{align*}
This proves that if $P \left( k \right)$ is true, then $P \left( k + 1 \right)$ is true.
\end{enumerate}



\item \begin{enumerate}
\item $\displaystyle\int_{0}^{\pi/2} \sin^{\,2} x \, dx = \left( \left. \frac{x}{2} - \frac{1}{2} \sin x \cos x \right) \right|_0^{\frac{\pi}{2}} = \dfrac{\pi}{4}$ and 

$\displaystyle\int_{0}^{\pi/2} \sin^{\,4} x \, dx = \left( \left. -\frac{1}{4} \sin^{3}x \cos x \right) \right|_0^{\frac{\pi}{2}} + \frac{3}{4} \displaystyle\int_{0}^{\pi/2} \sin^{\,2} x \, dx = \dfrac{3 \pi}{16}$

\item The basis step (with $n = 1$) was done in Part~(a).  For the inductive step, let $k$ be a natural number and assume that
\[
\int_{0}^{\pi/2} \sin^{\,2k}x \, dx = \frac{1 \cdot 3 \cdot 5 \cdots (2k - 1)}{2 \cdot 4 \cdot 6 \cdots (2k)} \frac{\pi}{2}.
\]
Then,
\begin{align*}
\int_{0}^{\pi/2} \sin^{\,2(k+1)}x \, dx &= \int_{0}^{\pi/2} \sin^{\,2k + 2}x \, dx \\
  &= \left( \left. -\frac{1}{2k + 2} \sin^{2k+1}x \cos x \right) \right|_0^{\frac{\pi}{2}} \\ &+ \frac{2k+1}{2k+2} \int_{0}^{\pi/2} \sin^{2k}x \, dx \\
  &= 0 + \frac{2k+1}{2k} \left(\frac{1 \cdot 3 \cdot 5 \cdots (2k - 1)}{2 \cdot 4 \cdot 6 \cdots (2k)} \frac{\pi}{2} \right) \\
  &= \frac{1 \cdot 3 \cdot 5 \cdots (2k - 1)(2k + 1)}{2 \cdot 4 \cdot 6 \cdots (2k)(2k + 2)} \frac{\pi}{2},
\end{align*}
and this proves the inductive step and proves that for all natural numbers $n$,
\[
\int_{0}^{\pi/2} \sin^{\,2n}x \, dx = \frac{1 \cdot 3 \cdot 5 \cdots (2n - 1)}{2 \cdot 4 \cdot 6 \cdots (2n)} \frac{\pi}{2}.
\]

For the second Wallis Sine Formula, we first see that
\begin{align*}
\int_{0}^{\pi/2} \sin^{\,3} x \, dx &= \left( \left. -\frac{1}{3} \sin^{2}x \cos x \right) \right|_0^{\frac{\pi}{2}} + \frac{2}{3} \int_{0}^{\pi/2} \sin x \, dx  \\
             &= 0 + \frac{2}{3} = \frac{2}{3}.
\end{align*}
This proves the basis step ($n = 1$) for the induction proof.  For the inductive step, let $k$ be a natural number and assume that
\[
\int_{0}^{\pi/2} \sin^{\,2k+1}x \, dx = \frac{2 \cdot 4 \cdot 6 \cdots (2k)}{1 \cdot 3 \cdot 5 \cdots (2k + 1)}.
\]
Then,
\begin{align*}
\int_{0}^{\pi/2} \sin^{\,2(k+1) + 1}x \, dx &= \int_{0}^{\pi/2} \sin^{\,2k + 3}x \, dx \\
  &= \left( \left. -\frac{1}{2k + 3} \sin^{2k+2}x \cos x \right) \right|_0^{\frac{\pi}{2}} \\ &+ \frac{2k+2}{2k+3} \int_{0}^{\pi/2} \sin^{2k+1}x \, dx \\
  &= 0 + \frac{2k+2}{2k+3} \left(\frac{2 \cdot 4 \cdot 6 \cdots (2k)}{1 \cdot 3 \cdot 5 \cdots (2k+1)} \right) \\
  &= \frac{2 \cdot 4 \cdot 6 \cdots (2k)(2k + 2)}{1 \cdot 3 \cdot 5 \cdots (2k+1)(2k + 3)},
\end{align*}
and this proves the inductive step and proves that for all natural numbers $n$,
\[
\int_{0}^{\pi/2} \sin^{\,2n+1}x \, dx = \frac{2 \cdot 4 \cdot 6 \cdots (2n)}{1 \cdot 3 \cdot 5 \cdots (2n + 1)}.
\]

\item For the basis step for the first Wallis Cosine Formula,
\[
\int_{0}^{\pi/2} \cos^{\,2} x \, dx = \left( \left. \frac{x}{2} + \frac{1}{2} \sin x \cos x \right) \right|_0^{\frac{\pi}{2}} = \dfrac{\pi}{4}.
\]
For the inductive step, let $k$ be a natural number and assume that
\[
\int_{0}^{\pi/2} \cos^{\,2k}x \, dx = \frac{1 \cdot 3 \cdot 5 \cdots (2k - 1)}{2 \cdot 4 \cdot 6 \cdots (2k)} \frac{\pi}{2}.
\]
Then,
\begin{align*}
\int_{0}^{\pi/2} \cos^{\,2(k+1)}x \, dx &= \int_{0}^{\pi/2} \cos^{\,2k + 2}x \, dx \\
  &= \left( \left. \frac{1}{2k + 2} \cos^{2k+1}x \sin x \right) \right|_0^{\frac{\pi}{2}} \\ &+ \frac{2k+1}{2k+2} \int_{0}^{\pi/2} \cos^{2k}x \, dx \\
  &= 0 + \frac{2k+1}{2k} \left(\frac{1 \cdot 3 \cdot 5 \cdots (2k - 1)}{2 \cdot 4 \cdot 6 \cdots (2k)} \frac{\pi}{2} \right) \\
  &= \frac{1 \cdot 3 \cdot 5 \cdots (2k - 1)(2k + 1)}{2 \cdot 4 \cdot 6 \cdots (2k)(2k + 2)} \frac{\pi}{2},
\end{align*}
and this proves the inductive step and proves that for all natural numbers $n$,
\[
\int_{0}^{\pi/2} \cos^{\,2n}x \, dx = \frac{1 \cdot 3 \cdot 5 \cdots (2n - 1)}{2 \cdot 4 \cdot 6 \cdots (2n)} \frac{\pi}{2}.
\]
For the second Wallis Cosine Formula, we first see that
\begin{align*}
\int_{0}^{\pi/2} \cos^{\,3} x \, dx &= \left( \left. \frac{1}{3} \cos^{2}x \sin x \right) \right|_0^{\frac{\pi}{2}} + \frac{2}{3} \int_{0}^{\pi/2} \cos x \, dx  \\
             &= 0 + \frac{2}{3} = \frac{2}{3}.
\end{align*}
This proves the basis step ($n = 1$) for the induction proof.  For the inductive step, let $k$ be a natural number and assume that
\[
\int_{0}^{\pi/2} \cos^{\,2k+1}x \, dx = \frac{2 \cdot 4 \cdot 6 \cdots (2k)}{1 \cdot 3 \cdot 5 \cdots (2k + 1)}.
\]
Then,
\begin{align*}
\int_{0}^{\pi/2} \cos^{\,2(k+1) + 1}x \, dx &= \int_{0}^{\pi/2} \cos^{\,2k + 3}x \, dx \\
  &= \left( \left. \frac{1}{2k + 3} \cos^{2k+2}x \sin x \right) \right|_0^{\frac{\pi}{2}} \\ &+ \frac{2k+2}{2k+3} \int_{0}^{\pi/2} \cos^{2k+1}x \, dx \\
  &= 0 + \frac{2k+2}{2k+3} \left(\frac{2 \cdot 4 \cdot 6 \cdots (2k)}{1 \cdot 3 \cdot 5 \cdots (2k+1)} \right) \\
  &= \frac{2 \cdot 4 \cdot 6 \cdots (2k)(2k + 2)}{1 \cdot 3 \cdot 5 \cdots (2k+1)(2k + 3)},
\end{align*}
and this proves the inductive step and proves that for all natural numbers $n$,
\[
\int_{0}^{\pi/2} \cos^{\,2n+1}x \, dx = \frac{2 \cdot 4 \cdot 6 \cdots (2n)}{1 \cdot 3 \cdot 5 \cdots (2n + 1)}.
\]
\end{enumerate}


\item \begin{enumerate}
\item The set $Q$ is not an inductive set.

\item The set of all real numbers greater than or equal to 1 is not an inductive set.
\end{enumerate}
\end{enumerate}


\subsection*{Evaluation of Proofs}
\setcounter{oldenumi}{\theenumi}
\begin{enumerate} \setcounter{enumi}{\theoldenumi}
\item \begin{enumerate}
\item This proposition is true, but the proposed induction proof has several flaws.  First, the predicate $P(n)$ is not defined correclty.  The predicate $P(n)$ should be,
\[
1 + 4 + 7 + \cdots + \left( 3n - 2 \right) = \frac{n(3n - 1)}{2}.
\]
In addition, the basis step is not proven correctly, and in the induction step, the display starts with the conclusion of the conditional statement instead of starting with $P(k)$ and using it to prove that $P(k + 1)$ must then be true.  Following is a well-written induction proof of this proposition.

\quarter
\setcounter{equation}{0}
\begin{myproof}
We will prove this proposition using mathematical induction.  So we let $P ( n )$ be the predicate
\[
1 + 4 + 7 + \cdots + \left( 3n - 2 \right) = \frac{n(3n - 1)}{2}.
\]
Using $n = 1$, the sum on the left side is equal to 1, and we see that 
$\dfrac{n(3n - 1)}{2} = 2$.  Hence, $P \left( 1 \right)$ is true.

For the induction step, we assume that $P( k )$ is true.  That is, we assume that
\begin{equation}
1 + 4 + 7 + \cdots + \left( 3k - 2 \right) = \frac{k \left(3k -1 \right)}{2} .
\end{equation}
We now need to prove that $P(k + 1)$ is true.  That is, we need to prove that
\[
1 + 4 + 7 + \cdots + ( 3k - 2 ) + \left( 3(k+1) - 2 \right) = \frac{(k + 1) \left(3(k + 1) -1 \right)}{2}
\]
or that
\begin{equation}
1 + 4 + 7 + \cdots + ( 3k - 2 ) + (3k - 1) = \frac{(k + 1) (3k + 2)}{2}.
\end{equation}

\noindent
Using the assumption that $P(k)$ is true (equation~(1)) and some algebra, we see that
\begin{align*}
1 + 4 + 7 + \cdots &+ ( 3k - 2 ) + \left( 3 (k + 1) - 2 \right) \\
  &= 
\left[ 1 + 4 + 7 + \cdots + ( 3k - 2 ) \right] + (3k + 1) \\
    &= \frac{k(3k - 1)}{2} + (3k + 1) \\
    &= \frac{(3k^2 - k) + 2(3k + 1)}{2} \\
    &= \frac{3k^2 + 5k + 2}{2} \\
    &= \frac{(k + 1)(3k + 2)}{2}
\end{align*}
Comparing this last equation to equation~(2), we see that we have proved that if $P(k)$ is true, then $P(k + 1)$ is true.  Hence, by the Principle of Mathematical Induction, we have proved that for each natural number $n$, $1 + 4 + 7 + \cdots + \left( 3n - 2 \right) = 
\dfrac{n \left(3n - 1 \right)}{2}$.
\end{myproof}

\item This proposition is true, but the proposed induction proof has several flaws.  First, the predicate $P(n)$ is not defined correclty.  This is supposed to be a predicate or open sentence.  $P(n)$ in the proposed proof is defined as the output of a function.  Given a natural number $n$, $P(n)$ would then be a number.  How can a number be true or false?  The 
\textbf{predicate} $P(n)$ should be,
\[
1 + 4 + 7 + \cdots + \left( 3n - 2 \right) = \frac{n(3n - 1)}{2}.
\]
So any of the references to $P(1)$ and $P(n)$ in the proposed proof need to be fixed.  A  well-written induction proof of this proposition is given in Part~(a) of this exercise.


\item This proposition is clearly false since there do exist dogs that are not the same breed.  The problem with the proposed induction proof lies in the induction step.  In particular, the problem is that the argument presented does not apply when $k = 1$.  In this case, the set $D_1$ has 1 element, the set $D_2$ has one element, but $D_1 \cap D_2 = \emptyset$.  So in this case, we cannot conclude that the two dogs in the set $D_1 \cup D_2$ have the same color.  So in the induction step, it is possible to prove that

\begin{list}{}
\item For all natural numbers $k$ with $k \geq 2$, if $P(k)$ is true, then \\$P(k + 1)$ is true.
\end{list}

\vskip6pt
\noindent
However, the conditional statement, if $P(1)$ is true, then $P(2)$ is true, cannot be proven.

\vskip6pt
\noindent
So it is possible to prove that if any set of two dogs consists of dogs of the same breed, then for each $n \in \N$ with $n \geq 2$, any set of $n$ dogs consists of dogs of the same breed.
\end{enumerate}
\end{enumerate}


\subsection*{Explorations and Activities}
\setcounter{oldenumi}{\theenumi}
\begin{enumerate} \setcounter{enumi}{\theoldenumi}
\item \setcounter{equation}{0}
Let  $P\left( n \right)$ be, $1 + 2 +  \cdots  + n = \dfrac{{n^2  + n + 1}}
{2}$.

\begin{enumerate}
\item Let  $k \in \mathbb{N}$.  Complete the following proof that if  $P\left( k \right)$ is true, then  $P\left( {k + 1} \right)$ is true.

Let  $k \in \mathbb{N}$.  Assume that  $P\left( k \right)$ is true.  That is, assume that
\begin{equation} \label{eq:act52a}
1 + 2 +  \cdots  + k = \frac{{k^2  + k + 1}}{2}.
\end{equation}

The goal is to prove that  $P\left( {k + 1} \right)$ is true.  That is, we need to prove that
\begin{equation} \label{eq:act52b}
1 + 2 +  \cdots  + k + \left( {k + 1} \right) = \frac{{\left( {k + 1} \right)^2  + \left( {k + 1} \right) + 1}}{2}.
\end{equation}

To do this, we add  $\left( {k + 1} \right)$ to both sides of Equation~(\ref{eq:act52a}).  This gives
\[
\begin{aligned}
  1 + 2 +  \cdots  + k + \left( {k + 1} \right) &= \frac{{k^2  + k + 1}}{2} + \left( {k + 1} \right) \\ 
               &= \frac{{k^2  + 3k + 3}}{2} \\ 
               &= \frac{{\left( {k^2  + 2k + 1} \right) + \left( {k + 2} \right)}}{2} \\ 
               &= \frac{{\left( {k + 1} \right)^2  + \left( {k + 1} \right) + 1}}{2} \\ 
\end{aligned}
\]
This proves that if  $P\left( k \right)$ is true, then  $P\left( {k + 1} \right)$ is true.

\item The propositions  $P\left( 1 \right)$, $P\left( 2 \right)$,  $P\left( 3 \right)$, and 
$P\left( 4 \right)$  are all false.
\end{enumerate}

\item \begin{enumerate}
\item There are 8 regions when there are 4 equally spaced points on the circle.

\item The pattern seems to indicate that when a point is added, the number of regions doubles.  If this pattern continues, there would be 16 regions when there are 5 equally spaced points on the circle.

\item If the pattern continues, there would be 32 regions when there are 6 equally spaced points on the circle.

\item There are 16 regions when there are 5 equally spaced points on the circle, and there are 30 
regions when there are 6 equally spaced points on the circle.

\item This activity is intended to show that we cannot assume that a pattern continues, we must prove that the pattern continues.  One way to do this is to use mathematical induction, and this activity shows that the inductive step is a necessary part of a proof by induction.

\end{enumerate}

\end{enumerate}



\hbreak
\endinput


\left( -1 \right)^{n-1} \dfrac{ \left( n - 1 \right)!}{x^n}$


\item For the inductive step, the following trigonometric identities are useful:
\begin{list}{}
\item $\cos \left( \alpha + \beta \right) = \cos \alpha \cos \beta - \sin \alpha \sin \beta$.
\item $\sin \left( \alpha + \beta \right) = \sin \alpha \cos \beta + \cos \alpha \sin \beta$.

\end{list}
If $\left[ {\cos x + i(\sin x)} \right]^k  = \cos (kx) + i\left( {\sin (kx)} \right)$, then
\[
\begin{aligned}
\left[ {\cos x + i(\sin x)} \right]^{k+1} &= \left[ {\cos x + i(\sin x)} \right]^k \left[ {\cos x + i \left(\sin x \right)} \right] \\
  &= \left( \cos (kx) + i {\sin (kx)} \right) \left( \cos x + i \left(\sin x \right) \right) \\
  &= \left( \cos \left( kx \right) \cos x - \sin \left( kx \right) sin x \right) + 
      i \left( \sin \left( kx \right) cos x + \cos \left( kx \right) \sin x \right)\\
  &= \cos \left(kx + x \right) + i \sin \left( kx + x \right) \\
  &= \cos \left( \left( k + 1 \right ) x \right) + i \sin \left( \left( k + 1 \right ) x \right).\\ 
\end{aligned}
\]


\item \begin{enumerate}
\item $\dfrac{{dy}}{{dx}} = \dfrac{1}{x}$,$\dfrac{{d^2 y}}{{dx^2 }} = -\dfrac{1}{x^2}$,
$\dfrac{{d^3 y}}{{dx^3 }} = \dfrac{2}{x^3}$, and $\dfrac{{d^4 y}}{{dx^4 }} = -\dfrac{6}{x^4}$.

\item $P \left( n \right)$ is 
``$\dfrac{{d^n y}}{{dx^n }} = \left( -1 \right)^{n-1} \dfrac{ \left( n - 1 \right)!}{x^n}$.''

If $\dfrac{{d^k y}}{{dx^k }} = \left( -1 \right)^{k-1} \dfrac{ \left( k - 1 \right)!}{x^k}$, then 
\[
\begin{aligned}
\frac{{d^{k+1} y}}{{dx^{k+1} }} &= \frac{d}{dx} \left( \dfrac{{d^k y}}{{dx^k }} \right) \\
  &= \frac{d}{dx} \left( ( -1 \right)^{k-1} \frac{ \left( k - 1 \right)!}{x^k} \\
  &=  \left( -1 \right)^{k-1} \cdot \left( -k \right) \left( k - 1 \right)! x^{-k-1} \\
  &=  \left( -1 \right)^k \frac{k!}{x^{k+1}}. \\
\end{aligned}
\]
This proves that if $P \left( k \right)$ is true, then $P \left( k + 1 \right)$ is true.
\end{enumerate}



\item Let $n$ be a natural number.  $P \left( n \right)$ is 
``$n^3 + \left( n+1 \right)^3 + \left( n+2 \right)^3$ is a multiple of 9.''  $P \left( 1 \right)$ is true since $1^3 + 2^3 + 3^3 = 36$.  So, let $k$ be a natural number and assume that
\begin{center}
$k^3 + \left( k + 1 \right)^3 + \left( k + 2 \right)^3$ is a multiple of 9.
\end{center}
So, there exists an integer $m$ such that
\[
k^3 + \left( k + 1 \right)^3 + \left( k + 2 \right)^3 = 9m.
\]
The goal is to prove that 
$\left( k + 1 \right)^3 + \left( k + 2 \right)^3 + \left( k + 3 \right)$ is a multiple of 9.  Using algebra, we see that
\[
\begin{aligned}
\left( k + 1 \right)^3 + \left( k + 2 \right)^3 &+ \left( k + 3 \right)^3 \\
  &= \left( k + 1 \right)^3 + \left( k + 2 \right)^3 + \left( k^3 + 9k^2 + 27k + 27 \right) \\
  &= k^3 + \left( k + 1 \right)^3 + \left( k + 2 \right)^3 + \left( 9k^2 + 27k + 27 \right) \\
  &= 9m + \left( 9k^2 + 27k + 27 \right) \\
  &= 9 \left( m + k^2 + 3k + 3 \right). \\
\end{aligned}
\]
This proves that if $P \left( k \right)$ is true, then $P \left( k + 1 \right)$ is true.




