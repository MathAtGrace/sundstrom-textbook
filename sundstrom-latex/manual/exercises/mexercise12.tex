\section*{Section \ref{S:direct} Constructing Direct Proofs}

\begin{enumerate}
\item \begin{enumerate} \usecounter{enumii} \item 
\begin{tabular}[t]{|p{0.4in}|p{1.6in}|p{1.6in}|}
  \hline
  \textbf{Step}  &  \textbf{Know}  &  \textbf{Reason} \\ \hline
  $P$  &  $m$ is an even integer.  &  Hypothesis \\ \hline
  $P1$ &  There exists an integers $k$ such that $m = 2k$. &  Definition of an even integer. \\ \hline
  $P2$  &  $m + 1 = 2k + 1$  &  Algebra \\ \hline
  $Q1$  &  There exists an integer $q$ such that $m +1 = 2q+1$  &  Substitution of $k = q$. \\ \hline
  $Q$  &  $m + 1$ is an odd integer. &  Definition of an odd integer. \\ \hline
\end{tabular}

A formal proof would be similar to the formal proof for Part~(b) given below.

\item Similar to Part~(a).  The main difference is that if there exist an integer $k$ such that $m = 2k + 1$, then $m + 1 = \left(2k + 1 \right) + 1 = 2 \left( k + 1 \right)$.

\noindent
\emph{\textbf{Proof}}.  Assume that $m$ is an odd integer.  We will prove that $m + 1$ is an even integer.  Since 
$m$ is odd, there exists an integer $k$ such that $m = 2k + 1$.  By adding 1 to both sides of this equation, we obtain
\[
m + 1 = \left( 2k + 1 \right) + 1.
\]
Then, using algebra, we obtain
\[
\begin{aligned}
m + 1 &= 2k + 2 \\
      &= 2 \left( k + 1 \right)
\end{aligned}
\]
Now, by the closure properties of the integers, $k + 1$ is an integer.  So, this proves that 
$m + 1$ is an even integer.  Therefore, we have proven that if $m$ is an odd integer, then 
$m + 1$ is an even integer.
\end{enumerate}



\item \begin{enumerate} \usecounter{enumii} 
\item Similar to Part~(c) except:  If there exist integers $m$ and $n$ such that $x = 2m$ and 
$y = 2n$, then $x + y = 2m + 2n = 2 \left(m + n \right)$. 

\item Similar to Part~(c) except:  If there exist integers $m$ and $n$ such that $x = 2m$ and 
$y = 2n+1$, then $x + y = 2m + \left(2n + 1 \right) = 2 \left(m + n \right) + 1$. 

\item \begin{tabular}[t]{|p{0.4in}|p{1.6in}|p{1.6in}|}
  \hline
  \textbf{Step}  &  \textbf{Know}  &  \textbf{Reason} \\ \hline
  $P$  &  $x$ and $y$ are odd integers.  &  Hypothesis \\ \hline
  $P1$ &  There exist integers $m$ and $n$ such that $x = 2m+1$ and $y = 2n+1$.  &  Definition of an odd integer. \\ \hline
  $P2$  &  $x + y = \left( 2m+1 \right) + \left( 2n+1 \right)$  &  Substitution \\ \hline
  $P3$  &  $x + y = 2m + 2n + 2$  &  Algebra  \\
        &  $x + y = 2 \left( {m+n+1} \right)$  &  \\ \hline
  $P4$  &  $\left( {m+n+1} \right)$ is an integer  &  Closure properties of the integers \\ \hline
  $Q1$  &  There exists an integer $q$ such that $x + y = 2q$  & $q = m+n+1$.  \\ \hline
  $Q$  &  $x + y$ is an even integer. &  Definition of an even integer. \\ \hline
\end{tabular}
\end{enumerate}



\item \begin{enumerate} \setcounter{enumii}{1}
%\emph{\item \textbf{Proof of (b)}}.  
\item Assume that $x$ is an even integer and $y$ is an odd integer.  We will prove that $x + y$ is an odd integer.  Since $x$ is even and $y$ is odd, there exist integers $m$ and $n$ such that
\[
x = 2m \quad \text{ and } \quad y = 2n + 1.
\]
By adding $x$ and $y$, we see that
\[
\begin{aligned}
x + y &= 2m + \left( 2n + 1 \right) \\
      &= 2m + 2n + 1 \\
      &= 2 \left( m + n \right) + 1 \\
\end{aligned}
\]
By the closure properties of the integers, $m + n$ is an integer.  So, the last equation proves that $x + y$ is an odd integer.  Therefore, we have proven that if $x$ is an even integer and $y$ is an odd integer, then $x + y$ is an odd integer.
\end{enumerate}


%\item \begin{enumerate} \usecounter{enumii} \item
%\begin{tabular}[t]{|p{0.4in}|p{1.6in}|p{1.6in}|}
%  \hline
%  \textbf{Step}  &  \textbf{Know}  &  \textbf{Reason} \\ \hline
%  $P$  &  $m$ is an even integer and $n$ is an integer.  &  Hypothesis \\ \hline
%  $P1$ &  There exists an integer $k$ $m = 2k$    &  Definition of an even integer. \\ \hline
%  $P2$  &  $m \cdot n = \left( 2k \right) n$  &  Substitution \\ \hline
%  $P3$  &  $m \cdot n = 2 \left( {kn} \right)$  &  Algebra  \\ \hline
%  $P4$  &  $\left( {kn} \right)$ is an integer  &  Closure properties of the integers \\ \hline
%  $Q1$  &  There exists an integer $q$ such that $m \cdot n = 2q$  & $q = kn$.  \\ \hline
%  $Q$  &  $m \cdot n$ is an even integer. &  Definition of an even integer. \\ \hline
%\end{tabular}
%
%\item \begin{tabular}[t]{|p{0.4in}|p{1.6in}|p{1.6in}|}
%  \hline
%  \textbf{Step}  &  \textbf{Know}  &  \textbf{Reason} \\ \hline
%  $P$  &  $n$ is an even integer.  &  Hypothesis \\ \hline
%  $P1$ &  $n^2 = n \cdot n$        &  Definition of an exponent. \\ \hline
%  $Q$  &  $n^2$ is an even integer. &  Exercise~(\ref{exer:evenoddmult}), Part~(a). \\ \hline
%\end{tabular}

%\item \begin{tabular}[t]{|p{0.4in}|p{1.6in}|p{1.6in}|}
%  \hline
%  \textbf{Step}  &  \textbf{Know}  &  \textbf{Reason} \\ \hline
%  $P$  &  $n$ is an odd integer.  &  Hypothesis \\ \hline
%  $P1$ &  $n^2 = n \cdot n$        &  Definition of an exponent. \\ \hline
%  $Q$  &  $n^2$ is an odd integer. &  Theorem~\ref{T:xyodd}. \\ \hline
%\end{tabular}
%\end{enumerate}


\item \begin{enumerate} %\usecounter{enumii}
\item We assume that $m$ is an even integer.  Then, by Part~(a) of 
Exercise~(3), $5m$ is an even integer, and by Part~(b) of 
Exercise~(3), $5m + 7$ is an odd integer.

\item We assume that $m$ is an odd integer.  Then, by Theorem~\ref{T:xyodd}, $5m$ is an odd integer, and by Part~(c) of Exercise~(\ref{exer:evenoddadd}), $5m + 7$ is an even integer.

\item We assume that $m$ and $n$ are odd integers.  Then, by Theorem~\ref{T:xyodd}, $mn$ is an odd integer, and by Part~(c) of Exercise~(3), $mn + 7$ is an even integer.
\end{enumerate}


\item \begin{enumerate} \usecounter{enumii}
\item We assume that $m$ is an even integer.  This means that there exists an integer $k$ such that $m = 2k$.  Substituting for $m$ in the expression $3m^2 + 2m + 3$, we obtain
\begin{align*}
3m^2 + 2m + 3 &= 3 \left( 2k \right)^2 + 2 \left( 2k \right) + 3 \\
              &= 12k^2 + 4k + 3 \\
              &= 2 \left( 6k^2 + 2k +1 \right) + 1
\end{align*}
By the closure properties of the integers, we conclude that $\left( 6k^2 + 2k +1 \right)$, and hence, the last equation proves that $3m^2 + 2m + 3$ is an odd integer.

\item We assume that $m$ is an odd integer.  By the Theorem~\ref{T:xyodd} and the results in Exercise~(\ref{exer:evenoddmult}), we conclude that $3m^2$ and $7m$ are both odd integers.  We can the use the results in Exercise~(\ref{exer:evenoddadd}) to concude that 
$3m^2 + 7m$ is an even integer and then that $3m^2 + 7m + 12$ is an even integer.
\end{enumerate}



\item \begin{enumerate} \usecounter{enumii}
\item Prove that they are not zero and their quotient is equal to 1; or prove that their difference is equal to 0.

\item Prove that it is greater than or equal to 0 and that it is less than or equal to 0.  Prove that its square is equal to 0.  Prove that its absolute value is equal to 0.

\item Prove that the two lines have a common perpendicular.  Prove that a transversal cuts the lines so that the alternate interior angles are equal.
\item Prove that two of the sides have the same length.  Prove that the triangle has two congruent angles.  Prove that an altitude of the triangle is a perpendicular bisector of a side of the triangle.
\end{enumerate}


%\begin{enumerate}[\exer{exer:sec12-pythag}]
%\item We assume that $m$ is a real number and that $m$, $m+1$, and $m+2$ are the lengths of the three sides of a right triangle.  Then, $m+2$ is the length of the hypotenuse and we can use the Pythagorean Theorem to obtain:
%\begin{align*}
%m^2 + \left( m + 1 \right)^2 &= \left(m + 2 \right)^2 \\
%               2m^2 + 2m + 1 &= m^2 + 4m + 4 \\
%                m^2 - 2m - 3 &= 0
%\end{align*}
%We now factor the left side of this equation to solve for $m$.  This gives \linebreak
%$(m - 3)(m + 1) = 0$ and we see that $m = 3$ or $m = -1$.  However, $m$ represents a length and hence, cannot be negative.  We conclude that $m = 3$.
%\end{enumerate}


\item \begin{enumerate} \usecounter{enumii}
\item The statement is false.  A counterexample is:  $a = 1$, $b = 2$, and $c = 1$.  With these values, we see that
\[
ab + ac = 1 \cdot 2 + 1 \cdot 1 = 3.
\]
So for this example, the hypothesis is true and the conclusion is false.

\item \emph{\textbf{Proof}}.  We assume that $b$ and $c$ are odd integers and that $a$ is an integer.  Since $b$ and $c$ are odd, there exist integers $m$ and $n$ such that $b = 2m + 1$ and $c = 2n + 1$.  We then see that
\begin{align*}
ab + ac &= a(2m + 1) + a(2n + 1) \\
        &= 2am + a + 2an + a \\
        &= 2am + 2an + 2a \\
        &= 2(am + an + a)
\end{align*}
Since $a$, $m$, and $n$ are integers, we can use the closure properties of the integers to conclude that $am + an + a$ is an integer.  So the last equation implies that $ab + ac$ is an even integer.  This proves that if $b$ and $c$ are odd integers and $a$ is an integer, then $ab + ac$ is an even integer.
\end{enumerate}


\item
\begin{tabular}[t]{|p{0.4in}|p{1.6in}|p{1.6in}|}
  \hline
  \textbf{Step}  &  \textbf{Know}  &  \textbf{Reason} \\ \hline
  $P$  &  $a$ and $b$ are non-negative real numbers and $a + b = 0$.  &  Hypothesis \\ \hline
  $P1$ &  $\left( a + b \right) + \left( -b \right) = 0 + \left( -b \right)$        &  Add $\left( -b \right)$ to both sides of the equation. \\ \hline
  $P2$ &  $a = -b$  &  Algebra. \\ \hline
  $P3$ &  $a \geq 0$ and $-b \leq 0$ &  Hypothesis and $b \geq 0$. \\ \hline
  $P4$ &  $a \geq 0$ and $a \leq 0$  &  Step $P3$ and substitution of $a = -b$. \\ \hline
  $Q$  &  $a = 0$. &  Zero is the only real number greater than or equal to 0 and less than or equal to 0.\\ \hline
\end{tabular}


\item \begin{enumerate} \usecounter{enumii}
\item  Some examples of type 1 integers are: $-5, -2, 1, 4, 7, 10$.
\item  Some examples of type 2 integers are: $-4, -1, 2, 5, 8, 11$.
\item  The statement appears to be true.
\end{enumerate}

%\begin{list}{} 
%\item \begin{list}{(b)}
%\item  Some examples are: -4, -1, 2, 5, 8, 11.
%\end{list}
%\end{list}
%
%\begin{list}{} 
%\item \begin{list}{(c)}
%\item  The statement appears to be true.
%\end{list}
%\end{list}

\item \begin{enumerate} \usecounter{enumii}
\item Let $a$ and $b$ be integers and assume that $a$ and $b$ are both type 1 integers.  Then, there exist integers $m$ and $n$ such that $a = 3m + 1$ and $b = 3n + 1$.  By adding $a$ and 
$b$, we see that
\[
\begin{aligned}
a + b &= \left( 3m + 1 \right) + \left( 3n + 1 \right) \\
      &= \left( 3m + 3n \right) + 2 \\
      &= 3 \left( m + n \right) + 2
\end{aligned}
\]
The closure properties of the integers imply that $m + n$ is an integer.  Therefore, the last equation tells us that $a + b$ is a type 2 integer.  Hence, we have proven that if $a$ and $b$ are both type 1 integers, then $a + b$ is a type 2 integer.

\item Similar to Part~(a) except:   If there exist integers $m$ and $n$ such that $a = 3m + 2$ and $b = 3n + 2$, then
\[
\begin{aligned}
a + b &= \left( 3m + 2 \right) + \left( 3n + 2 \right) \\
      &= 3m + 3n + 4 \\
      &= 3m + 3n + 3 + 1 \\
      &= 3 \left( m + n + 1 \right) + 1
\end{aligned}
\]

\item Similar to Part~(a) except:   If there exist integers $m$ and $n$ such that $a = 3m + 1$ and $b = 3n + 2$, then
\[
\begin{aligned}
a \cdot b &= \left( 3m + 1 \right) \cdot \left( 3n + 2 \right) \\
      &= 9mn + 6m + 3n + 2 \\
      &= 3 \left( 3mn + 2m + n \right) + 2
\end{aligned}
\]

\item Similar to Part~(a) except:   If there exist integers $m$ and $n$ such that $a = 3m + 2$ and $b = 3n + 2$, then
\[
\begin{aligned}
a \cdot b &= \left( 3m + 2 \right) \cdot \left( 3n + 2 \right) \\
      &= 9mn + 6m + 6n + 4 \\
      &= 9mn + 6m + 6n + 3 + 1 \\
      &= 3 \left( 3mn + + 2m + 2n + 1 \right) + 1
\end{aligned}
\]
\end{enumerate}


\item \begin{enumerate} \usecounter{enumii}
\item The sum of the two roots is
\begin{align*}
x_1 +  x_2 &= \frac{-b + \sqrt{b^2 - 4ac}}{2a} + \frac{-b - \sqrt{b^2 - 4ac}}{2a} \\
          &= \frac{-b + \sqrt{b^2 - 4ac} + -b - \sqrt{b^2 - 4ac}}{2a} \\
          &= \frac{-2b}{2a} \\
          &= - \frac{b}{a}
\end{align*}

\item The product of the two roots is
\begin{align*}
x_1 \cdot x_2 &= \frac{-b + \sqrt{b^2 - 4ac}}{2a} \cdot \frac{-b - \sqrt{b^2 - 4ac}}{2a} \\
          &= \frac{(-b)^2 - \left( \sqrt{b^2 - 4ac} \right)^2}{(2a)^2} \\
          &= \frac{b^2 - \left(b^2 - 4ac \right)}{4a^2} \\
          &= \frac{4ac}{4a^2} \\
          &= \frac{c}{a}
\end{align*}

\end{enumerate}




\item \begin{enumerate} \usecounter{enumii}
\item The quadratic formula is:  $x = \dfrac{-b \pm \sqrt{b^2 - 4ac}}{2a}$.  The quadratic equation $ax^2 + bx + c = 0$ has:
\begin{itemize}
\item No real number solution if $b^2 - 4ac < 0$.
\item One real number solution if $b^2 - 4ac = 0$.
\item Two real number solutions if $b^2 - 4ac > 0$.
\end{itemize}

\item The key is that if $a > 0$ and $c < 0$, then $-4ac > 0$ and hence 
\[
b^2 - 4ac = b^2 + \left( -4ac \right) > 0.
\]
In addition, $b^2 - 4ac > b^2$ and so, $\sqrt{b^2 - 4ac} > \left| b \right|$.  Thus, \\
$-b + \sqrt{b^2 - 4ac} > 0$, and hence, $\dfrac{-b + \sqrt{b^2 - 4ac}}{2a} > 0$.

\item The key is that if $\dfrac{b}{2} < \sqrt{ac}$, then by squaring both sides of this inequality, we get $\dfrac{b^2}{4} < ac$, and this implies that $b^2 - 4ac < 0$.  Hence, the equation $ax^2 + bx + c = 0$ has no real number solutions.
\end{enumerate}
\end{enumerate}
%\hbreak

\subsection*{Explorations and Activities}
%\begin{enumerate}[\exer{exer:pythag}]
\setcounter{oldenumi}{\theenumi}
\begin{enumerate} \setcounter{enumi}{\theoldenumi}
\item  \begin{enumerate} \usecounter{enumii} \setcounter{enumii}{1}
\item If $m = 5$, then $m + 7 = 12$ and $m + 8 = 13$, and 5, 12, 13 form a Pythagorean triple.  To determine if there are any other Pythagorean triples of this form, we let $m$ be a natural number and solve the equation
\[
m^2 + (m + 7)^2 = (m + 8)^2.
\]
Expanding the terms in parentheses and then using algebra, we obtain
\begin{align*}
m^2 + \left( m^2 + 14m + 49 \right) &= m^2 + 16m + 64 \\
          m^2 - 2m - 15 &= 0 \\
          (m - 5)(m + 3) &= 0
\end{align*}
This shows that $m = 5$ is the only natural number solution of this equation.  Therefore, 5, 12, 13 is the only Pythagorean triple of the form $m$, $m + 7$, $m + 8$, for some natural number $m$.


\item To determine if there are any Pythagorean triples of the form $m$, $m + 11$, $m + 12$ for some natural number $m$, we let $m$ be a natural number and solve the equation
\[
m^2 + (m + 11)^2 = (m + 12)^2.
\]
Expanding the terms in parentheses and then using algebra, we obtain
\begin{align*}
m^2 + \left( m^2 + 22m + 121 \right) &= m^2 + 24m + 144 \\
          m^2 - 2m - 23 &= 0 \\
\end{align*}
We can use the quadratic formula to find the solutions of this equation and note that there are no natural number solutions to this equation.  We therefore conclude that there is no Pythagorean triple of the form $m$, $m + 11$, $m + 12$, for some natural number $m$.
\end{enumerate}
\end{enumerate}



\begin{enumerate}[\exer{exer:morepythag}]
\item If we focus on the least even natural number in each Pythagorean triple and let $n$  be this even natural number with $n = 2m$, for some natural number $m$, we can see that each Pythagorean triple in the list is of the form $2m$, $m^2 - 1$, and $m^2 + 1$.  (Not necessarily in that order.)  So we will prove the folloiwng proposition:

\newpar
\textbf{Proposition}.  If $m$ is a natural number with $m \geq 2$, then the natural numbers $2m$, $m^2 - 1$, and $m^2 + 1$ form a Pythagorean triple.

\begin{myproof}
Let $m$ be a natural number with $m \geq 2$ and consider the natural numbers $2m$, $m^2 - 1$, and $m^2 + 1$.  Since $m \geq 2$, we see that $m^2 + 1$ is the largest of these three numbers.  We will now prove that these numbers form a Pythagorean triple by proving that $(2m)^2 + \left( m^2 - 1 \right)^2 = \left( m^2 + 1 \right)$.  We see that
\begin{align*}
(2m)^2 + \left( m^2 - 1 \right)^2 &= 4m^2 + \left( m^4 - 2m^2 + 1 \right) \\
                                  &= m^4 + 2m^2 + 1 \\
                                  &= \left( m^2 + 1 \right)
\end{align*}
This proves that if $m$ is a natural number with $m \geq 2$, then the natural numbers $2m$, $m^2 - 1$, and $m^2 + 1$ form a Pythagorean triple.              
\end{myproof}
\end{enumerate}
\hbreak

\endinput

\begin{tabular}[t]{|p{0.4in}|p{1.6in}|p{1.6in}|}
  \hline
  \textbf{Step}  &  \textbf{Know}  &  \textbf{Reason} \\ \hline
  $P$  &  $m$ is an odd integer.  &  Hypothesis \\ \hline
  $P1$ &  There exists an integers $k$ such that $m = 2k +1$. &  Definition of an odd integer. \\ \hline
  $P2$  &  $m +1 = 2k + 2$  &  Algebra \\ \hline
  $P3$  &  $m +1 = 2 \left( k + 1 \right)$  &  Algebra \\ \hline
  $Q1$  &  There exists an integer $q$ such that $m + 1 = 2q$  &  Substitution of $q = k + 1$. \\ \hline
  $Q$  &  $m + 1$ is an even integer. &  Definition of an even integer. \\ \hline
\end{tabular}

