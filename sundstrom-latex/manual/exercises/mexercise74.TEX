\section*{Section \ref{S:modulararithmetic} Modular Arithmetic}

\begin{enumerate}

\item \begin{enumerate}
\item
%\begin{center}
\begin{tabular}{ c | c  c  c  c p{0.5in} c | c  c  c c}
$\oplus$ & $\left[ 0 \right]$ & $\left[ 1 \right]$ & $\left[ 2 \right]$ & $\left[ 3 \right]$ & &   $\odot$ & $\left[ 0 \right]$ & $\left[ 1 \right]$ & $\left[ 2 \right]$ & $\left[ 3 \right]$  \\ \cline{1-5} \cline{7-11}

$\left[ 0 \right]$ & $\left[ 0 \right]$ & $\left[ 1 \right]$ & $\left[ 2 \right]$ & $\left[ 3 \right]$&  & 
$\left[ 0 \right]$ & $\left[ 0 \right]$ & $\left[ 0 \right]$ & $\left[ 0 \right]$ & $\left[ 0 \right]$ 
\\ 

$\left[ 1 \right]$ & $\left[ 1 \right]$ & $\left[ 2 \right]$ & $\left[ 3 \right]$ & $\left[ 0 \right]$ &  & 
$\left[ 1 \right]$ & $\left[ 0 \right]$ & $\left[ 1 \right]$ & $\left[ 2 \right]$ & $\left[ 3 \right]$
\\ 

$\left[ 2 \right]$ & $\left[ 2 \right]$ & $\left[ 3 \right]$ & $\left[ 0 \right]$ & $\left[ 1 \right]$ &  & 
$\left[ 2 \right]$ & $\left[ 0 \right]$ & $\left[ 2 \right]$ & $\left[ 0 \right]$ & $\left[ 2 \right]$
\\ 

$\left[ 3 \right]$ & $\left[ 3 \right]$ & $\left[ 0 \right]$ & $\left[ 1 \right]$ & $\left[ 2 \right]$ &  & 
$\left[ 3 \right]$ & $\left[ 0 \right]$ & $\left[ 3 \right]$ & $\left[ 2 \right]$ & $\left[ 1 \right]$
\\ 

\end{tabular}
%\end{center}

\item
\begin{tabular}{ c | c  c  c  c  c  c  c}
$\oplus$ & $\left[ 0 \right]$ & $\left[ 1 \right]$ & $\left[ 2 \right]$ & $\left[ 3 \right]$ & 
$\left[ 4 \right]$ & $\left[ 5 \right]$ & $\left[ 6 \right]$  \\ \hline

$\left[ 0 \right]$ & $\left[ 0 \right]$ & $\left[ 1 \right]$ & $\left[ 2 \right]$ & 
$\left[ 3 \right]$ & $\left[ 4 \right]$ & $\left[ 5 \right]$ & $\left[ 6 \right]$  \\ 

$\left[ 1 \right]$ & $\left[ 1 \right]$ & $\left[ 2 \right]$ & $\left[ 3 \right]$ & 
$\left[ 4 \right]$ & $\left[ 5 \right]$ & $\left[ 6 \right]$ & $\left[ 0 \right]$  \\ 

$\left[ 2 \right]$ & $\left[ 2 \right]$ & $\left[ 3 \right]$ & $\left[ 4 \right]$ & 
$\left[ 5 \right]$ & $\left[ 6 \right]$ & $\left[ 0 \right]$ & $\left[ 1 \right]$  \\ 

$\left[ 3 \right]$ & $\left[ 3 \right]$ & $\left[ 4 \right]$ & $\left[ 5 \right]$ & 
$\left[ 6 \right]$ & $\left[ 0 \right]$ & $\left[ 1 \right]$ & $\left[ 2 \right]$ \\ 

$\left[ 4 \right]$ & $\left[ 4 \right]$ & $\left[ 5 \right]$ & $\left[ 6 \right]$ & 
$\left[ 0 \right]$ & $\left[ 1 \right]$ & $\left[ 2 \right]$ & $\left[ 3 \right]$  \\ 

$\left[ 5 \right]$ & $\left[ 5 \right]$ & $\left[ 6 \right]$ & $\left[ 0 \right]$ & 
$\left[ 1 \right]$ & $\left[ 2 \right]$ & $\left[ 3 \right]$ & $\left[ 4 \right]$  \\ 

$\left[ 6 \right]$ & $\left[ 6 \right]$ & $\left[ 0 \right]$ & $\left[ 1 \right]$ & 
$\left[ 2 \right]$ & $\left[ 3 \right]$ & $\left[ 4 \right]$ & $\left[ 5 \right]$  \\ 
\end{tabular}

\vskip 9pt

\begin{tabular}{ c | c  c  c  c  c  c  c}
$\odot$ & $\left[ 0 \right]$ & $\left[ 1 \right]$ & $\left[ 2 \right]$ & $\left[ 3 \right]$ & 
$\left[ 4 \right]$ & $\left[ 5 \right]$ & $\left[ 6 \right]$  \\ \hline

$\left[ 0 \right]$ & $\left[ 0 \right]$ & $\left[ 0 \right]$ & $\left[ 0 \right]$ & 
$\left[ 0 \right]$ & $\left[ 0 \right]$ & $\left[ 0 \right]$ & $\left[ 0 \right]$  \\ 

$\left[ 1 \right]$ & $\left[ 1 \right]$ & $\left[ 2 \right]$ & $\left[ 3 \right]$ & 
$\left[ 4 \right]$ & $\left[ 5 \right]$ & $\left[ 6 \right]$ & $\left[ 0 \right]$  \\ 

$\left[ 2 \right]$ & $\left[ 0 \right]$ & $\left[ 2 \right]$ & $\left[ 4 \right]$ & 
$\left[ 6 \right]$ & $\left[ 1 \right]$ & $\left[ 3 \right]$ & $\left[ 5 \right]$  \\ 

$\left[ 3 \right]$ & $\left[ 0 \right]$ & $\left[ 3 \right]$ & $\left[ 6 \right]$ & 
$\left[ 2 \right]$ & $\left[ 5 \right]$ & $\left[ 1 \right]$ & $\left[ 4 \right]$  \\ 

$\left[ 4 \right]$ & $\left[ 0 \right]$ & $\left[ 4 \right]$ & $\left[ 1 \right]$ & 
$\left[ 5 \right]$ & $\left[ 2 \right]$ & $\left[ 6 \right]$ & $\left[ 3 \right]$  \\ 

$\left[ 5 \right]$ & $\left[ 0 \right]$ & $\left[ 5 \right]$ & $\left[ 3 \right]$ & 
$\left[ 1 \right]$ & $\left[ 6 \right]$ & $\left[ 4 \right]$ & $\left[ 2 \right]$  \\ 

$\left[ 6 \right]$ & $\left[ 0 \right]$ & $\left[ 6 \right]$ & $\left[ 5 \right]$ & 
$\left[ 4 \right]$ & $\left[ 3 \right]$ & $\left[ 2 \right]$ & $\left[ 1 \right]$  \\ 
\end{tabular}

\item
\begin{tabular}{ c | c  c  c  c  c  c  c c}
$\oplus$ & $\left[ 0 \right]$ & $\left[ 1 \right]$ & $\left[ 2 \right]$ & $\left[ 3 \right]$ & 
$\left[ 4 \right]$ & $\left[ 5 \right]$ & $\left[ 6 \right]$ & $\left[ 7 \right]$  \\ \hline

$\left[ 0 \right]$ & $\left[ 0 \right]$ & $\left[ 1 \right]$ & $\left[ 2 \right]$ & 
$\left[ 3 \right]$ & $\left[ 4 \right]$ & $\left[ 5 \right]$ & $\left[ 6 \right]$ & 
$\left[ 7 \right]$  \\ 

$\left[ 1 \right]$ & $\left[ 1 \right]$ & $\left[ 2 \right]$ & $\left[ 3 \right]$ & 
$\left[ 4 \right]$ & $\left[ 5 \right]$ & $\left[ 6 \right]$ & $\left[ 7 \right]$ & 
$\left[ 0 \right]$ \\ 

$\left[ 2 \right]$ & $\left[ 2 \right]$ & $\left[ 3 \right]$ & $\left[ 4 \right]$ & 
$\left[ 5 \right]$ & $\left[ 6 \right]$ & $\left[ 7 \right]$ & $\left[ 0 \right]$ & 
$\left[ 1 \right]$  \\ 

$\left[ 3 \right]$ & $\left[ 3 \right]$ & $\left[ 4 \right]$ & $\left[ 5 \right]$ & 
$\left[ 6 \right]$ & $\left[ 7 \right]$ & $\left[ 0 \right]$ & $\left[ 1 \right]$ & 
$\left[ 2 \right]$  \\ 

$\left[ 4 \right]$ & $\left[ 4 \right]$ & $\left[ 5 \right]$ & $\left[ 6 \right]$ & 
$\left[ 7 \right]$ & $\left[ 0 \right]$ & $\left[ 1 \right]$ & $\left[ 2 \right]$ & 
$\left[ 3 \right]$  \\ 

$\left[ 5 \right]$ & $\left[ 5 \right]$ & $\left[ 6 \right]$ & $\left[ 7 \right]$ & 
$\left[ 0 \right]$ & $\left[ 1 \right]$ & $\left[ 2 \right]$ & $\left[ 3 \right]$ & 
$\left[ 4 \right]$  \\ 

$\left[ 6 \right]$ & $\left[ 6 \right]$ & $\left[ 7 \right]$ & $\left[ 0 \right]$ & 
$\left[ 1 \right]$ & $\left[ 2 \right]$ & $\left[ 3 \right]$ & $\left[ 4 \right]$ & 
$\left[ 5 \right]$  \\ 

$\left[ 7 \right]$ & $\left[ 7 \right]$ & $\left[ 0 \right]$ & $\left[ 1 \right]$ & 
$\left[ 2 \right]$ & $\left[ 3 \right]$ & $\left[ 4 \right]$ & $\left[ 5 \right]$ & 
$\left[ 6 \right]$  \\ 
\end{tabular}

\vskip 9pt

\begin{tabular}{ c | c  c  c  c  c  c  c c}
$\odot$ & $\left[ 0 \right]$ & $\left[ 1 \right]$ & $\left[ 2 \right]$ & $\left[ 3 \right]$ & 
$\left[ 4 \right]$ & $\left[ 5 \right]$ & $\left[ 6 \right]$ & $\left[ 7 \right]$  \\ \hline

$\left[ 0 \right]$ & $\left[ 0 \right]$ & $\left[ 0 \right]$ & $\left[ 0 \right]$ & 
$\left[ 0 \right]$ & $\left[ 0 \right]$ & $\left[ 0 \right]$ & $\left[ 0 \right]$ & 
$\left[ 0 \right]$  \\ 

$\left[ 1 \right]$ & $\left[ 0 \right]$ & $\left[ 1 \right]$ & $\left[ 2 \right]$ & 
$\left[ 3 \right]$ & $\left[ 4 \right]$ & $\left[ 5 \right]$ & $\left[ 6 \right]$ & 
$\left[ 7 \right]$ \\ 

$\left[ 2 \right]$ & $\left[ 0 \right]$ & $\left[ 2 \right]$ & $\left[ 4 \right]$ & 
$\left[ 6 \right]$ & $\left[ 0 \right]$ & $\left[ 2 \right]$ & $\left[ 4 \right]$ & 
$\left[ 6 \right]$  \\ 

$\left[ 3 \right]$ & $\left[ 0 \right]$ & $\left[ 3 \right]$ & $\left[ 6 \right]$ & 
$\left[ 1 \right]$ & $\left[ 4 \right]$ & $\left[ 7 \right]$ & $\left[ 2 \right]$ & 
$\left[ 5 \right]$  \\ 

$\left[ 4 \right]$ & $\left[ 0 \right]$ & $\left[ 4 \right]$ & $\left[ 0 \right]$ & 
$\left[ 4 \right]$ & $\left[ 0 \right]$ & $\left[ 4 \right]$ & $\left[ 0 \right]$ & 
$\left[ 4 \right]$  \\ 

$\left[ 5 \right]$ & $\left[ 0 \right]$ & $\left[ 5 \right]$ & $\left[ 2 \right]$ & 
$\left[ 7 \right]$ & $\left[ 4 \right]$ & $\left[ 1 \right]$ & $\left[ 6 \right]$ & 
$\left[ 3 \right]$  \\ 

$\left[ 6 \right]$ & $\left[ 0 \right]$ & $\left[ 6 \right]$ & $\left[ 4 \right]$ & 
$\left[ 2 \right]$ & $\left[ 0 \right]$ & $\left[ 6 \right]$ & $\left[ 4 \right]$ & 
$\left[ 2 \right]$  \\ 

$\left[ 7 \right]$ & $\left[ 0 \right]$ & $\left[ 7 \right]$ & $\left[ 6 \right]$ & 
$\left[ 5 \right]$ & $\left[ 4 \right]$ & $\left[ 3 \right]$ & $\left[ 2 \right]$ & 
$\left[ 1 \right]$  \\ 
\end{tabular}
\end{enumerate}




\item \begin{enumerate}
\item $\left[ x \right] = \left[ 1 \right]$ or $\left[ x \right] = \left[ 3 \right]$. 

\item $\left[ x \right] = \left[ 1 \right]$, $\left[ x \right] = \left[ 3 \right]$, 
$\left[ x \right] = \left[ 5 \right]$, or $\left[ x \right] = \left[ 7 \right]$.

\item $\left[ x \right] = \left[ 1 \right]$, $\left[ x \right] = \left[ 2 \right]$, 
$\left[ x \right] = \left[ 3 \right]$, or $\left[ x \right] = \left[ 4 \right]$.

\item The equation has no solution.

\item $\left[ x \right] = \left[ 2 \right]$ or $\left[ x \right] = \left[ 3 \right]$.

\item $\left[ x \right] = \left[ 1 \right]$.

\item The equation has no solution.

\item The equation has no solution.
\end{enumerate}



\item \begin{enumerate}
\item The statement is false.  By using the multiplication table for $\mathbb{Z}_6$, we see that a counterexample is $\left[ a \right] = \left[ 2 \right]$.

\item The statement is true.  By using the multiplication table for $\mathbb{Z}_5$, we see that:

\begin{multicols}{2}
\begin{list}{}
\item $\left[ 1 \right] \odot \left[ 1 \right] = \left[ 1 \right]$.

\item $\left[ 2 \right] \odot \left[ 3 \right] = \left[ 1 \right]$.

\item $\left[ 3 \right] \odot \left[ 2 \right] = \left[ 1 \right]$.

\item $\left[ 4 \right] \odot \left[ 4 \right] = \left[ 1 \right]$.
\end{list}
\end{multicols}
\end{enumerate}



\item \begin{enumerate}
\item The statement is false.  A counterexample is $\left[ a \right] = \left[ 2 \right]$ and $\left[ b \right] = \left[ 3 \right]$.

\item The statement is true.  By using the multiplication table for $\mathbb{Z}_5$, we see that there are no entries equal to $\left[ 0 \right]$ when $\left[ a \right] \ne \left[ 0 \right]$ and 
$\left[ b \right] \ne \left[ 0 \right]$.
\end{enumerate}



\item \begin{enumerate}
\item The proof consists of the following computations:
\begin{multicols}{2}
\begin{list}{}
\item $[ 1 ]^2 = [ 1 ]$
\item $[ 2 ]^2 = [ 4 ]$
\item $[ 3 ]^2 = [ 9 ] = [ 4 ]$
\item $[ 4 ]^2 = [ 16 ] = [ 1 ]$.
\end{list}
\end{multicols}

\item If there exists an integer $a$ such that $a^2 = 5,158,232,468,953,153$, then 
$a^2 \equiv 3 \pmod 5$.  This contradicts the result in Part~(a).  Therefore, no such integer exists.
\end{enumerate}




\item The key idea in the inductive step of the proof is that if
\[
10^k \equiv 1 \pmod 9,
\]
then $10 \cdot 10^k \equiv 10 \cdot 1 \pmod 9$.  This proves that $10^{k+1} \equiv 10 \pmod 9$ and since $10 \equiv 1 \pmod 9$, we conclude that $10^{k+1} \equiv 1 \pmod 9$.


\item The key idea in the inductive step of the proof is that if
\[
10^k \equiv 1 \pmod 3,
\]
then $10 \cdot 10^k \equiv 10 \cdot 1 \pmod 3$.  This proves that $10^{k+1} \equiv 10 \pmod 3$ and since $10 \equiv 1 \pmod 3$, we conclude that $10^{k+1} \equiv 1 \pmod 3$.


\item \begin{enumerate}
\item 
\begin{align}
  \left[ n \right] &= \left[ {\left( {a_k  \times 10^k } \right) + \left( {a_{k - 1}  \times 10^{k - 1} } \right) +  \cdots  + \left( {a_1  \times 10^1 } \right) + \left( {a_0  \times 10^0 } \right)} \right] \notag \\ 
   &= \left[ {a_k  \times 10^k } \right] \oplus \left[ {a_{k - 1}  \times 10^{k - 1} } \right] \oplus  \cdots  \oplus \left[ {a_1  \times 10^1 } \right] \oplus \left[ {a_0  \times 10^0 } \right] \notag \\
  &= \left( {\left[ {a_k } \right] \otimes \left[ {10^k } \right]} \right) \oplus \left( {\left[ {a_{k - 1} } \right] \otimes \left[ {10^{k - 1} } \right]} \right) \oplus  \cdots \notag \\ 
  & \hspace{5cm} \oplus \left( {\left[ {a_1 } \right] \otimes \left[ {10^1 } \right]} \right) \oplus \left( {\left[ {a_0 } \right] \otimes \left[ {10^0 } \right]} \right) \notag  
\end{align}

So, the result in Exercise~(\ref{exer:poweroftenmod3}) implies that
\begin{align} \notag
[ n ] &= \left( {[ {a_k } ] \odot [ 1 ]} \right) \oplus \left( {[ {a_{k - 1} } ] \odot [ 1 ]} \right) \oplus  \cdots  \oplus \left( {[ {a_1 } ] \odot [ 1 ]} \right) \oplus \left( {[ {a_0 } ] \odot [ 1 ]} \right)  \notag \\ 
                 &= [ {a_k } ] \oplus [ {a_{k - 1} } ] \oplus  \cdots  \oplus [ {a_1 } ] \oplus [ {a_0 } ] \notag \\ 
                 &= [ {a_k  + a_{k - 1}  +  \cdots  + a_1  + a_0 } ]. \notag \\
%\label{eq:divideby9-5} \\
                 &= [ s( n ) ]. \notag
\end{align}
\item Converting the result in Part~(a) to a congruence, we see that \\ 
$n \equiv s(n)  \pmod 3$.

\item By Part~(b),  $n \equiv 0 \pmod 3$  if and only if  
$ s(n)  \equiv 0 \pmod 3$.  This means that $3 \mid n$ if and only if  
$5 \mid s(n) $.
\end{enumerate}



\item The key idea in the inductive step of the proof is that if
\[
10^k \equiv 0 \pmod 5,
\]
then $10 \cdot 10^k \equiv 10 \cdot 0 \pmod 5$.  This proves that $10^{k+1} \equiv 0 \pmod 5$.



\item \begin{enumerate}
\item 
\begin{align}
  \left[ n \right] &= \left[ {\left( {a_k  \times 10^k } \right) + \left( {a_{k - 1}  \times 10^{k - 1} } \right) +  \cdots  + \left( {a_1  \times 10^1 } \right) + \left( {a_0  \times 10^0 } \right)} \right] \notag \\ 
   &= \left[ {a_k  \times 10^k } \right] \oplus \left[ {a_{k - 1}  \times 10^{k - 1} } \right] \oplus  \cdots  \oplus \left[ {a_1  \times 10^1 } \right] \oplus \left[ {a_0  \times 10^0 } \right] \notag \\
  &= \left( {\left[ {a_k } \right] \otimes \left[ {10^k } \right]} \right) \oplus \left( {\left[ {a_{k - 1} } \right] \otimes \left[ {10^{k - 1} } \right]} \right) \oplus  \cdots \notag \\ 
  & \hspace{5cm} \oplus \left( {\left[ {a_1 } \right] \otimes \left[ {10^1 } \right]} \right) \oplus \left( {\left[ {a_0 } \right] \otimes \left[ {10^0 } \right]} \right) \notag  
\end{align}

So, the result in Exercise~(\ref{exer:poweroftenmod5}) implies that
\[
\begin{aligned}
  \left[ n \right] &= \left( {\left[ {a_0 } \right] \otimes \left[ {10^0 } \right]} \right) \\ 
                   &= \left[ {a_0 } \right] \\ 
\end{aligned}
\]
\item Converting the result in Part~(a) to a congruence, we see that \\ 
$n \equiv  {a_0 }  \pmod 5$.

\item By Part~(b),  $n \equiv 0 \pmod 5$  if and only if  
$ {a_0 }  \equiv 0 \pmod 5$.  This means that $5 \mid n$ if and only if  
$5 \mid  {a_0 } $.
\end{enumerate}



\item The key idea in the inductive step of the proof is that if
\[
10^k \equiv 0 \pmod 4,
\]
then $10 \cdot 10^k \equiv 10 \cdot 0 \pmod 4$.  This proves that $10^{k+1} \equiv 0 \pmod 4$.

\item \begin{enumerate}
\item Following is an outline of a proof.
\[
  \left[ n \right] = \left( {\left[ {a_k } \right] \otimes \left[ {10^k } \right]} \right) \oplus \left( {\left[ {a_{k - 1} } \right] \otimes \left[ {10^{k - 1} } \right]} \right) \oplus  \cdots  \oplus \left( {\left[ {a_1 } \right] \otimes \left[ {10^1 } \right]} \right) \oplus \left( {\left[ {a_0 } \right] \otimes \left[ {10^0 } \right]} \right)
\]
So, the result in Exercise~(\ref{exer:poweroftenmod4}) implies that
\[
\begin{aligned}
  \left[ n \right] &= \left( {\left[ {a_2 } \right] \otimes \left[ {10^2 } \right]} \right) \oplus \left( {\left[ {a_1 } \right] \otimes \left[ {10^1 } \right]} \right) \oplus \left( {\left[ {a_0 } \right] \otimes \left[ {10^0 } \right]} \right) \\ 
   &= \left[ {10a_1  + a_0 } \right] \\ 
\end{aligned}
\]

\item Converting this result to a congruence, we see that  
\[
n \equiv \left( {10a_1  + a_0 } \right) \pmod 8.
\]

\item Hence, $4 \mid n$ if and only if $4 \mid \left( {10a_1  + a_0 } \right)$.
\end{enumerate}




\item The key idea in the inductive step of the proof is that if
\[
10^k \equiv 0 \pmod 8,
\]
then $10 \cdot 10^k \equiv 10 \cdot 0 \pmod 8$.  This proves that $10^{k+1} \equiv 0 \pmod 8$.

\item The divisibility test for 8 is:   $8 \mid n$ if and only if  
$8 \mid \left( {100a_2 + 10a_1  + a_0 } \right)$.  Following is an outline of a proof.

\[
  \left[ n \right] = \left( {\left[ {a_k } \right] \otimes \left[ {10^k } \right]} \right) \oplus \left( {\left[ {a_{k - 1} } \right] \otimes \left[ {10^{k - 1} } \right]} \right) \oplus  \cdots  \oplus \left( {\left[ {a_1 } \right] \otimes \left[ {10^1 } \right]} \right) \oplus \left( {\left[ {a_0 } \right] \otimes \left[ {10^0 } \right]} \right)
\]
So, the result in Exercise~(\ref{exer:poweroftenmod8}) implies that
\[
\begin{aligned}
  \left[ n \right] &= \left( {\left[ {a_2 } \right] \otimes \left[ {10^2 } \right]} \right) \oplus \left( {\left[ {a_1 } \right] \otimes \left[ {10^1 } \right]} \right) \oplus \left( {\left[ {a_0 } \right] \otimes \left[ {10^0 } \right]} \right) \\ 
   &= \left[ {100a_2 + 10a_1  + a_0 } \right] \\ 
\end{aligned}
\]
Converting this result to a congruence, we see that  
\[
n \equiv \left( {100a_2 + 10a_1  + a_0 } \right) \pmod 8.
\]
Hence, $8 \mid n$ if and only if $8 \mid \left( {100a_2 + 10a_1  + a_0 } \right)$.




\item The key idea in the inductive step of the proof is that if
\[
10^k \equiv \left( -1 \right)^k \pmod {11},
\]
then $10 \cdot 10^k \equiv 10 \cdot \left( -1 \right)^k \pmod {11}$.  Since 
$10 \equiv -1 \pmod {11}$, we can conclude that
\[
10^{k+1} \equiv \left( -1 \right) \cdot \left( -1 \right)^k \pmod {11},
\]
and hence that $10^{k+1} \equiv \left( -1 \right)^{k+1} \pmod {11}$.

\item \begin{enumerate}
\item If  $n = \sum\limits_{j = 0}^k \left( {a_j \times 10^j} \right)$, then 
$n \equiv \sum\limits_{j = 0}^k \left[ { a_j \times 10^j \pmod {11}} \right]$. Therefore,  
$n \equiv \sum\limits_{j = 0}^k \left[ {\left( { - 1} \right)^j a_j } \pmod {11} \right]$, and hence, \\$n \equiv \sum\limits_{j = 0}^k {\left( { - 1} \right)^j a_j } \pmod {11}$.

\item This is a direct consequence of Part~(a).

\item By Part~(b), $\left[ n \right] = \left[ 0 \right]$ if and only if 
$\left[ {\sum\limits_{j = 0}^k {\left( { - 1} \right)^j a_j } }\right] = \left[ 0 \right]$.  This means that  $11 \mid n$ if and only if  
$11 \mid \sum\limits_{j = 0}^k {\left( { - 1} \right)^j a_j }$. 
\end{enumerate}

\item \begin{enumerate}
\item First, use the multiplication table to show that if $\left[ x \right] \in \mathbb{Z}_3$ and 
$\left[ x \right] \ne \left[ 0 \right]$, then $\left[ x \right]^2 = \left[ 1 \right]$.  This can then be used to prove the contrapositive: If $\left[ a \right] \ne \left[ 0 \right]$ and 
$\left[ b \right] \ne \left[ 0 \right]$, then 
$\left[ a \right]^2 + \left[ b \right]^2 \ne \left[ 0 \right]$.

\item By translating the statement in Part~(a) from congruence classes to congruence modulo 3, we see that if  $\left( {a^2  + b^2 } \right) \equiv 0 \pmod 3$, then  
$a \equiv 0 \pmod 3$  and  $b \equiv 0 \pmod 3$.

\item This follows directly from Part~(b) since $x \equiv 0 \pmod 3$ if and only if $3 \mid x$.
\end{enumerate}

\item Examine all 10 cases in $\mathbb{Z}_{10}$ to prove that if 
$\left[ x \right] \in \mathbb{Z}_{10}$, then $\left[ x \right]^4$ must be $\left[ 0 \right]$, 
$\left[ 1 \right]$, $\left[ 5 \right]$, or $\left[ 6 \right]$.  Then, construct an addition table for these 4 values using addition modulo 10.  The only entries in this table will be 
$\left[ 0 \right]$, $\left[ 1 \right]$, $\left[ 2 \right]$, or $\left[ 5 \right]$, 
$\left[ 6 \right]$, and $\left[ 7 \right]$.  This proves that $\left [ a \right]$ must be one of these 6 congruence classes and hence $a$ must be congruent modulo 10 to 0, 1, 2, 5, 6, or 7.  Since a non-negative integer is congruent modulo 10 to its units digit, this proves that the units digit of $a$ must be 0, 1, 2, 5, 6, or 7.


\item Let $n \in \mathbb{Z}$.  If $n$ is odd, then $n$ must be congruent to 1, 3, 5, or 7 modulo 8.  For each of these values of $n$, verify that in 
$\mathbb{Z}_8$, $\left[ n \right]^2 = \left[ 1 \right]$.  This implies that 
$n^2 \equiv 1 \pmod 8$ and hence that $8 \mid \left( n^2 - 1 \right)$.

\item First prove that if $\left[ x \right] \in \mathbb{Z}_8$, then $\left[ x \right]^2$ must be 
$\left[ 0 \right]$, $\left[ 1 \right]$, or $\left[ 4 \right]$.  Then argue (by cases if necessary), that the sum of three of these congruence classes will always be unequal to 
$\left[ 7 \right]$.  This proves that if $a, b ,c \in \mathbb{N}$, then 
$\left( a^2 + b^2 +c^2 \right) \not\equiv 7 \pmod 8$.  Hence, if $n$ is a natural number and 
$n \equiv 7 \pmod 8$, then $n$ is not the sum of three squares.
\end{enumerate}



\subsection*{Explorations and Activities}
\setcounter{oldenumi}{\theenumi}
\begin{enumerate} \setcounter{enumi}{\theoldenumi}
\item \begin{enumerate}
\item If  $n \in \mathbb{Z}$, then in  $\mathbb{Z}_4 $, $\left[ n \right] = \left[ 0 \right]$,
$\left[ n \right] = \left[ 1 \right]$, $\left[ n \right] = \left[ 2 \right]$, or 
$\left[ n \right] = \left[ 3 \right]$.  So, using the multiplication table for  $\mathbb{Z}_4 $, we see that  $\left[ n \right]^2  = \left[ 0 \right]$  or  
$\left[ n \right]^2  = \left[ 1 \right]$.  Since  $\left[ n \right]^2  = \left[ {n^2 } \right]$, we conclude that  $\left[ {n^2 } \right] = \left[ 0 \right]$  or  
$\left[ {n^2 } \right] = \left[ 1 \right]$.

\item For each  $n \in \mathbb{Z}$,  $n^2  \equiv 0 \pmod 4$  or  $n^2  \equiv 1 \pmod 4$.

\item Since  $59 \equiv 3 \pmod 4$, we use the results in Part~(b) of Exercise~(12) to conclude that  \\
$104 257 833 259 \equiv 3 \pmod 4$.  For a more basic approach, we note that
\[
104 257 833 259 = 104 257 833 200 + 59.
\]
Since $104 257 833 200 \equiv 0 \pmod 4$ and $59 \equiv 3 \pmod 4$, we can conclude that
$104 257 833 259 \equiv (0 + 3) \pmod 4$ and hence, $104 257 833 259 \equiv 3 \pmod 4$.

\item The number  104,257,833,259 is not a perfect square since for each  $n \in \mathbb{Z}$,  
$n^2  \equiv 0 \pmod 4$  or  $n^2  \equiv 1 \pmod 4$, and  
$104,257,833,259 \equiv 3 \pmod 4$.
\end{enumerate}

\end{enumerate}

\hbreak
\endinput

