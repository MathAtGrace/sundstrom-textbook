\section*{Section \ref{S:equivrelations} Equivalence Relations}

\begin{enumerate} 
\item The relation $R$ is not reflexive on $A$.  For example, $\left( a, a \right) \notin R$.
The relation $R$ is not symmetric since $\left( a, b \right) \in R$ and 
$\left( b, a \right) \notin R$.  
The relation $R$ is transitive since the conditional statement, ``If $x \mathrel{R} y$ and $y \mathrel{R} z$, then $x \mathrel{R} z$'', is a true conditional statement.
The relation $R$ is not an equivalence relation.



\vskip6pt
\BeginTable
\BeginFormat
| p(2.25in) | p(2.25in) | 
\EndFormat
" \textbf{2. (a)} " \textbf(b) " \\
" \setlength{\unitlength}{0.5cm}
\begin{picture}(8,6)
\put(1,5){\circle*{.25}}
\put(7,5){\circle*{.25}}
\put(7,1){\circle*{.25}}
\put(1.2,5.2){\vector(1,0){5.6}}
%\put(7,4.8){\vector(0,-1){3.6}}
\put(6.8,4.8){\vector(-1,0){5.6}}
\put(0.6,5.3){$a$}
\put(7.3,5.3){$b$}
\put(7.3,0.8){$c$}
\end{picture} " 
\setlength{\unitlength}{0.5cm}
\begin{picture}(8,6)
\put(1,5){\circle*{.25}}
\put(7,5){\circle*{.25}}
\put(7,1){\circle*{.25}}
\put(1.2,5){\vector(1,0){5.6}}
\put(7,4.8){\vector(0,-1){3.6}}
\put(1.2,5){\vector(3,-2){5.6}}
\put(0.6,5.3){$a$}
\put(7.3,5.3){$b$}
\put(7.3,0.8){$c$}
\end{picture}
" \\
" \textbf{(c)} " \textbf{(d)} " \\
" \setlength{\unitlength}{0.5cm}
\begin{picture}(8,6)
\put(1,5){\circle*{.25}}
\put(7,5){\circle*{.25}}
\put(7,1){\circle*{.25}}
\put(1.2,5.2){\vector(1,0){5.6}}
\put(6.8,4.8){\vector(-1,0){5.6}}
\put(1,4){\circle{2}}
\put(1,3){\vector(1,0){0}}
\put(7,4){\circle{2}}
\put(7,3){\vector(-1,0){0}}
\put(0.6,5.3){$a$}
\put(7.3,5.3){$b$}
\put(7.3,0.8){$c$}
\end{picture} " 
\setlength{\unitlength}{0.5cm}
\begin{picture}(8,6)
\put(1,5){\circle*{.25}}
\put(7,5){\circle*{.25}}
\put(7,1){\circle*{.25}}
\put(1.2,5){\vector(1,0){5.6}}
\put(7,4.8){\vector(0,-1){3.6}}
%\put(1.2,5){\vector(3,-2){5.6}}
\put(0.6,5.3){$a$}
\put(7.3,5.3){$b$}
\put(7.3,0.8){$c$}
\end{picture} " \\
\EndTable

\vskip6pt
\BeginTable
\BeginFormat
| p(2.25in) | p(2.25in) | 
\EndFormat
" \hspace{6pt} \textbf{(e)} " " \\
" \setlength{\unitlength}{0.5cm}
\begin{picture}(8,6)
\put(1,5){\circle*{.25}}
\put(7,5){\circle*{.25}}
\put(7,1){\circle*{.25}}
\put(1.2,5.2){\vector(1,0){5.6}}
\put(6.8,4.8){\vector(-1,0){5.6}}
\put(1,4){\circle{2}}
\put(1,3){\vector(1,0){0}}
\put(7,4){\circle{2}}
\put(7,3){\vector(-1,0){0}}
\put(8,1){\circle{2}}
\put(8,2){\vector(-1,0){0}}
\put(0.6,5.3){$a$}
\put(7.3,5.3){$b$}
\put(7.3,0.8){$c$}
\end{picture} " " \\
\EndTable


\addtocounter{enumi}{1}


\item One example of an equivalence relation on $A =\left\{ 1, 2, 3, 4, 5 \right\}$ is: 

$R = \left\{ \left( 1, 1 \right), \left( 2, 2 \right), \left( 3, 3 \right), \left( 4, 4 \right), \left( 5, 5 \right), \left( 1, 2 \right), \left( 2, 1 \right) \right\}$.



\item The relation $R$ is not reflexive on $A$.  For example, $\left( 4, 4 \right) \notin R$.
The relation $R$ is symmetric.  If $\left( a, b \right) \in R$, then 
$\left| a \right| + \left| b \right| = 4$.  Therefore, $\left| b \right| + \left| a \right| = 4$, and hence, $\left( b, a \right) \in R$.
The relation $R$ is not transitive.  For example, $\left( 4, 0 \right) \in R$, 
$\left( 0, 4 \right) \in R$, and $\left( 4, 4 \right) \notin R$.
The relation $R$ is not an equivalence relation.


\item The relation $R$ is reflexive since for each $a \in \Z$, $|a - a| = 0$ and hence, 
$a \mathrel{R} a$.  The relation $R$ is symmetric since for all $a, b \in \Z$, 
$|a - b| = |b - a|$.   Therefore, if $a \mathrel{R} b$, then $b \mathrel{R} a$.  The relation $R$ is not transitive.  For example, $1 \mathrel{R} 4$, $4 \mathrel{R} 7$, and 
$1 \mathrel{\not \negthickspace R} 7$.  So the relation $R$ is not an equivalence relation.






\item For  $a, b \in \mathbb{R}$,  $a \sim b$ if and only if  
$f\left( a \right) = f\left( b \right)$.
\begin{enumerate}
\item The relation $\sim$ is an equivalence relation on $\mathbb{R}$.  

For $a \in \mathbb{R}$, $a \sim a$ since $f \left( a \right) = f \left( a \right)$.  So, $\sim$ is reflexive.

For $a, b \in \mathbb{R}$, if $a \sim b$, then  $f \left( a \right) = f \left( b \right)$.  So, 
$f \left( b \right) = f \left( a \right)$. Hence, $b \sim a$ and $\sim$ is symmetric.

For $a, b,c  \in \mathbb{R}$, if $a \sim b$ and $b \sim c$, then  
$f \left( a \right) = f \left( b \right)$ and $f \left( b \right) = f \left( c \right)$.  So, 
$f \left( a \right) = f \left( c \right)$. Hence, $a \sim c$ and $\sim$ is transitive.

\item  $C = \left\{ { {x \in \mathbb{R} } \mid x \sim 5} \right\} = \left\{ -5, 5 \right\}$.
\end{enumerate}



\item \begin{enumerate}
\item The proof that $\sim$ is an equivalence relation is identical to the proof in Exercise~(6).

\item  $C = \left\{ { {x \in \mathbb{R} } \mid x \sim 5} \right\} = \left\{ -2, 5 \right\}$.
\end{enumerate}



\item $C = \left\{ { {x \in \mathbb{R} } \mid x \sim \pi} \right\} = 
\left\{ n \pi \mid n \in \mathbb{Z} \right\}$.



\item For  $a, b \in \mathbb{R}$,  $a \sim b$  if and only if   $a - b$  is an integer.
\begin{enumerate}
%\item The relation $\sim$ is an equivalence relation.
%
%For $x \in \mathbb{R}$, $x - x = 0$ and 0 is an integer.  Therefore, $x \sim x$ and $\sim$ is reflexive.
%
%For $x, y \in \mathbb{R}$, if $x \sim y$, then $x - y$ is an integer.  This implies that $y - x$ is an integer and so, $y \sim x$.  Therefore, $\sim$ is symmetric.
%
%For $x, y, z \in \mathbb{R}$, if $x \sim y$ and $y \sim z$, then $x - y$ is an integer and 
%$y - z$ is an integer.  Since, $x - z = \left( x - y \right) + \left( y - z \right)$, we see that $x - z$ is an integer.  Therefore,  $x \sim z$ and $\sim$ is transitive.

\item Four elements in the set $C = \left\{ { {x \in \Q } \mid x \sim \dfrac{5}{7} } \right\}$ are:  $\dfrac{5}{7}$, $\dfrac{12}{7}$, $\dfrac{19}{7}$, and $\dfrac{-2}{7}$.

\item $C = \left\{ x \in \mathbb{Q}  \mid x \sim \dfrac{5}{7} \right\} = 
\left\{ n + \dfrac{5}{7} \mid n \in \mathbb{Z} \right\}$

\item $C = \left\{ \ldots, \dfrac{-9}{7}, \dfrac{-2}{7}, \dfrac{5}{7}, \dfrac{12}{7}, \dfrac{19}{7}, \ldots \, \right\}$
\end{enumerate}



\item \begin{enumerate}
\item The relation $\sim$ is an equivalence relation on $\Z$.  It is reflexive since for each integer $a$, $a + a = 2a$ and hence, $2$ divides $a + a$.  Now let $a, b \in \Z$ and assume that 2 divides $a + b$.  Then, $a$ and $b$ must either both be odd or both be even.  In both cases, 2 divides $b + a$ and hence, $\sim$ is symmetric.  Finally, let $a, b, c \in \Z$ and assume that 
$a \sim b$ and $b \sim c$.  Since 2 divides $a + b$, $a$ and $b$ must both be odd or both be even.  In the case that $a$ and $b$ are both odd, then $b \sim c$ implies that $c$ must be odd.  Hence, $a + c$ is even and $a \sim c$.  A similar proof shows that if $a$ and $b$ are both even, then $a \sim c$.  Therefore, $\sim$ is transitive.

\item The relation $\approx$ is not reflexive on $\Z$ since $2 \not\approx 2$.  The relation $\approx$ is not transitive since $2 \approx 1$, $1 \approx 5$, and $2 \not\approx 5$.  However, the relation $\approx$ is symmetric.  To prove this, if $a, b \in \Z$ and $a \approx b$, then 
$a + b = b + a$ and 3 divided $a + b$ if and only if 3 divides $b + a$.  Therefore, $b \approx a$.
\end{enumerate}



\item \begin{enumerate}
\item The relation  $\sim$ is not reflexive on  $\mathcal{P}\left( U \right)$  since  $U \cap U \ne \emptyset $.

The relation  $\sim$ is symmetric since  if  $A \cap B = \emptyset $ then  
$B \cap A = \emptyset $.

The relation  $\sim$  is not transitive.  For an example, let  
$U = \left\{ {1, 2, 3, 4, 5} \right\}$, $A = \left\{ {1, 2} \right\}$, 
$B = \left\{ {3, 4} \right\}$, and  $C = \left\{ {1, 5} \right\}$.  Then,  
$A \cap B = \emptyset $,  $B \cap C = \emptyset $, but  $A \cap C \ne \emptyset $.

\item The relation  $ \approx $  is reflexive on  $\mathcal{P}\left( U \right)$  since for all  
$A \in \mathcal{P}\left( U \right)$,  $\left| A \right| = \left| A \right|$.

The relation  $ \approx $ is symmetric since  for all  $A, B \in \mathcal{P}\left( U \right)$, if  $\left| A \right| = \left| B \right|$, then  $\left| B \right| = \left| A \right|$.  That is, if  $A$  has the same number of elements as  $B$, then  $B$  has the same number of elements as  $A$.

The relation  $ \approx $ is transitive since  for all  $A, B, C \in \mathcal{P}\left( U \right)$, if  $\left| A \right| = \left| B \right|$ and  $\left| B \right| = \left| C \right|$, then  $\left| A \right| = \left| C \right|.$  That is,  if  $A$  and  $B$  have the same number of elements and  $B$  and  $C$  have the same number of elements, then  $A$  and  $C$  have the same number of elements.

Therefore, the relation $\approx$ is an equivalence relation on $\mathcal{P}\left( U \right)$.
\end{enumerate}




\item For $A \in \mathcal{P} \left( U \right)$, the identity function $I_A$ is a bijection.  Therefore, $A \sim A$ and $\sim$ is reflexive.

Let $A, B \in \mathcal{P} \left( U \right)$ and assume that $A \sim B$.  This means there is a bijection $f: A \to B$.  Then, by Exercise~(\ref{exer:finversebijection}) in 
Section~\ref{S:inversefunctions}, $f^{-1}:B \to A$ is a bijection, and hence, $B \sim A$.  Therefore, $\sim$ is symmetric.

Let $A, B, C \in \mathcal{P} \left( U \right)$ and assume that $A \sim B$ and $B \sim C$.  This means there is a bijection $f: A \to B$ and a bijection $g:B \to C$.  Then, by 
Theorem~\ref{T:compositefunctions} in Section~\ref{S:compositionoffunctions}, 
$g \circ f: A \to C$ is a bijection, and hence, $A \sim C$.  Therefore, $\sim$ is transitive.



\item \begin{enumerate}
\item The relation $\sim$ is an equivalence relation on $\Z$.  For each $a \in \Z$, 
$2a + 3a = 5a$ and, hence, $2a + 3a \equiv \pmod 5$ and $a \sim a$.  Therefore, the relation 
$\sim$ is reflexive on $\Z$.  Now let $a, b \in \Z$ and assume that $a \sim b$.  Then 
$2a + 3b \equiv 0 \pmod 5$.  If we multiply both sides of this congruence by 4, we obtain 
$8a + 12b \equiv 0 \pmod 5$, and this implies that $2b + 3a \equiv 0 \pmod 5$ or that 
$b \sim a$.  This proves that $\sim$ is symmetric.

Finally, let $a, b, c \in \Z$ and assume that $a \sim b$ and $b \sim c$.  Then
\[
2a + 3b \equiv 0 \pmod 5 \qquad \text{and} \qquad 2b + 3c \equiv 0 \pmod 5.
\]
If we add the corresponding sides of these two congruences, we obtain 
$(2a + 5b + 3c) \equiv \pmod 5$ or $2a + 3c \equiv 0$.  So $a \sim c$ and the relation $\sim$ is transitive.

\item The relation $\approx$ is not an equivalence relation on $\Z$.  First, it is not reflexive as is shown by the fact that $1 \not\approx 1$.  Also, the relation $\approx$ is not symmetric since $2 \approx 1$ but $1 \not\approx 2$.  Finally, the relation $\approx$ is not transitive.  For example, $2 \approx 1$, $1 \approx 3$, and $2 \not\approx 3$.
\end{enumerate}



\item \begin{enumerate}
\item The relation $\sim$ is not an equivalence relation.  It is a symmetric relation since for each $x \in \R$, $x^2 \geq 0$ and hence, $x \sim x$.  The relation $\sim$ is also symmetric since for all $x, y \in \R$, $xy \geq 0$ if and only if $yx \geq 0$.  However, the relation 
$\sim$ is not transitive.  For example, $2 \sim 0$, $0 \sim -2$, and $2 \not\sim -2$. 

\item The relation $\approx$ is not an equivalence relation on $\R$.  For example, it is not reflexive since $2 \not\approx 2$.  It is a symmetric relation since for all $x, y \in \R$, 
$xy \leq 0$ if and only if $yx \leq 0$.  The relation $\approx$ is also not transitive.  For example, $2 \approx -2$, $-2 \approx 2$, and $2 \not\approx 2$.
\end{enumerate}



\item For  $\left( {a, b} \right), \left( {c, d} \right) \in \mathbb{R} \times \mathbb{R}$,  $\left( {a, b} \right) \approx \left( {c, d} \right)$ if and only if  \\
$a^2  + b^2  = c^2  + d^2 $.

\begin{enumerate}
\item Let $\left( {a, b} \right) \in \mathbb{R} \times \mathbb{R}$.  Since 
$a^2 + b^2 = a^2 + b^2$, we see that $\left( {a, b} \right) \approx \left( {a, b} \right)$ and hence, $\approx$ is reflexive.

Let $\left( {a, b} \right), \left( {c, d} \right) \in \mathbb{R} \times \mathbb{R}$ and assume that $\left( {a, b} \right) \approx \left( {c, d} \right)$.  Then, $a^2  + b^2  = c^2  + d^2 $, and this implies that $c^2  + d^2  = a^2  + b^2 $.  Hence, 
$\left( {a, b} \right) \approx \left( {c, d} \right)$ and $\approx$ is symmetric.

Let $\left( {a, b} \right), \left( {c, d} \right), \left( {p, q} \right) \in \mathbb{R} \times \mathbb{R}$ and assume that $\left( {a, b} \right) \approx \left( {c, d} \right)$ and 
$\left( {c, d} \right) \approx \left( {p, q} \right)$.  Then, $a^2  + b^2  = c^2  + d^2 $ and 
$c^2  + d^2  = p^2  + q^2 $.  From these two equations, we see that 
$a^2  + b^2  = p^2  + q^2 $ and hence that $\left( {a, b} \right) \approx \left( {p, q} \right)$.  Therefore, $\approx$ is transitive.

\item The ordered pairs $\left( 4, 3 \right)$, $\left( -4, 3 \right)$, 
$\left( -3, -4 \right)$, and $\left( 5, 0 \right)$ are in \\
$C = \left\{ { {\left( {x, y} \right) \in \mathbb{R} \times \mathbb{R} } \mid                 \left( {x, y} \right) \approx \left( {4, 3} \right)} \right\}$.

\item The set $C$ consists of all points on the circle of radius 5 whose center is at the origin.
\end{enumerate}
\end{enumerate}




\subsection*{Evaluations of Proofs}
\setcounter{oldenumi}{\theenumi}
\begin{enumerate} \setcounter{enumi}{\theoldenumi}
\item \begin{enumerate}
\item This proposition is false.  If $A = \{1, 2, 3, 4 \}$, then the relation  
$R  = \left\{ {\left( {1, 1} \right), \left( {2, 2} \right), \left( {1, 2} \right), \left( {2, 1} \right)} \right\}$ is symmetric and transitive but is not reflexive on  
$A$.

\noindent
The mistake in this proof is in the use of quantifiers and conditional statements.  To prove that a relation $R$ on a set $A$ is reflexive, we need to prove that for every $x \in A$, 
$x \mathrel{R} x$.  In order to use the argument in the proposed proof that $x \mathrel{R} x$, there has to exist an element $y \in A$ such that $x \mathrel{R} y$.  This is not guaranteed by the definition of a relation.  Notice that in the counterexample, there is no $y$ in $A$ such that $3 \mathrel{R} y$.

\item The essential ideas of a proof of this proposition are in the proposed proof.  An introductory paragraph is needed in the proof so that it is a self-contained proof.  In addition, there are several revisions that can be made to make this a better proof.  First, the proof that the relation is reflexive is not done correctly.  The proof starts with the assumption that $a \sim a$, which is what needs to be proven.  In addition, some more explanation is needed for the proofs of the symmetric and transitive properties.  Following is a revised proof of this proposition.

\setcounter{equation}{0}
\begin{myproof}
Let $\sim$ be a relation on $\Z$ where for all $a, b \in \Z$,  
$a \sim b$ if and only if $\left( a + 2b \right) \equiv 0 \pmod 3$.  We will prove the relation 
$\sim$ is an equivalence relation on $\Z$.  To prove that $\sim$ is reflexive, we let $a \in \Z$.
Then $\left( a + 2a \right) \equiv 0 \pmod 3$ since 
$\left( 3a \right) \equiv 0 \pmod 3$. Therefore, $\sim$ is reflexive on $\Z$.  

Now let $a, b \in \Z$ and assume that $a \sim b$.  Then 
$\left( a + 2b \right) \equiv 0 \pmod 3$, and if we multiply both sides of this congruence by 2, we get
\begin{align}
2\left( a + 2b \right) &\equiv 2 \cdot 0 \pmod 3 \notag \\
\left( 2a + 4b \right) &\equiv 0 \pmod 3.
\end{align}
We now use the fact that $3b \equiv 3b \pmod 3$ and congruence~(1) to obtain
\begin{align*}
\left( 2a + 4b \right) - 3b  &\equiv 0 - 3b \pmod 3  \\
\left( b + 2a \right) &\equiv 0 \pmod 3. 
\end{align*}
This means that $b \sim a$ and hence, $\sim$ is symmetric.

Finally, we let $a, b, c \in \Z$ and assume that $a \sim b$ and $b \sim c$.  This means that 
$\left( a + 2b \right) \equiv 0\pmod 3$ and 
$\left( b + 2c \right) \equiv 0\pmod 3$.  By adding the corresponding sides of these two congruences, we obtain
\begin{align}
\left( a + 2b \right) + \left( b + 2c \right) &\equiv 0 + 0 \pmod 3 \notag \\
\left( a + 3b + 2c \right) &\equiv 0 \pmod 3.
\end{align}
We now use the fact that $3b \equiv 3b \pmod 3$ and congruence~(2) to obtain
\begin{align*}
\left( a + 3b + 2c \right) - 3b  &\equiv 0 - 3b \pmod 3 \\
\left( a + 2c \right) &\equiv 0 \pmod 3.
\end{align*}
This means that $a \sim c$.  Hence, the relation$\sim$ is transitive and we have proved that 
$\sim$ is an equivalence relation on $\Z$.
\end{myproof}
\end{enumerate}
\end{enumerate}



\subsection*{Explorations and Activities}
\setcounter{oldenumi}{\theenumi}
\begin{enumerate} \setcounter{enumi}{\theoldenumi}
\item \begin{enumerate}
\item A relation $R$ on a set $A$ is not circular provided that there exist $x$, $y$, and $z$ in $A$ such that 
$x \mathrel{R} y$, $y \mathrel{R} z$, and $z \mathrel{\not \negthickspace R} x$.

\item For an example of a relation that is circular, draw a directed graph in which $R = \{ (1, 2), (2, 3), (3, 1) \}$.  So the only arrows would be for  $1 \mathrel{R} 2$, $2 \mathrel{R} 3$, and $3 \mathrel{R} 1$.

\noindent
For an example of a relation that is not circular, draw a directed graph in which $R = \{ (1, 2), (2, 3) \}$.  So the only arrows would be for  $1 \mathrel{R} 2$ and  $2 \mathrel{R} 3$.

\item For an example of a relation that is transitive and not circular, draw a directed graph in which $R = \{ (1, 2), (2, 3), (3, 1) \}$.  So the only arrows would be for  $1 \mathrel{R} 2$, $2 \mathrel{R} 3$, and $3 \mathrel{R} 1$.

\noindent
For an example of a relation that is transitive and not circular, draw a directed graph in which $R = \{ (1, 2), (2, 3), (1, 3) \}$.  So the only arrows would be for  $1 \mathrel{R} 2$, $2 \mathrel{R} 3$, and $1 \mathrel{R} 3$.


  
\item First, assume the relation $R$ is an equivalence relation on $A$.  Since it is then a reflexive relation, we need only prove it is circular.  So let $a, b, c \in A$ and assume that 
$a \sim b$ and $b \sim c$.  Then since $\sim$ is transitive, $a \sim c$ and since $\sim$ is symmetric, we conclude that $c \sim a$.  This proves that $\sim$ is circular.

Now assume that $\sim$ is reflexive and circular.  To prove that $\sim$ is symmetric, we let 
$a, b \in A$ and assume that $a \sim b$.  However, we also know that $\sim$ is reflexive and hence, $b \sim b$.  Now, since $a \sim b$ and $b \sim b$ and $\sim$ is circular, we conclude that $b \sim a$.  This proves that $\sim$ is symmetric.  To prove it is transitive, let 
$a, b, c \in A$ and assume that $a \sim b$ and $b \sim c$.  Since $\sim$ is circular, we conclude that $c \sim a$ and since $\sim$ is symmetric, we conclude that $a \sim c$.  This proves that $\sim$ is transitive and hence, $\sim$ is an equivalence relation on $A$.



%\item \begin{enumerate}
\item A relation $R$ on a set $A$ is not antisymmetric provided that there exist $x$ and $y$ in $A$ such that 
$x \mathrel{R} y$, $y \mathrel{R} x$, and $x \ne y$.

\item The equality relation on $A$ (in which the only arrows are loops at each of the three vertices) is an antisymmetric relation.

\noindent
For an example of a relation that is not antisymmetric, draw a directed graph in which $R = \{ (1, 2), (2, 1) \}$.  So the only arrows would be for  $1 \mathrel{R} 2$ and  $2 \mathrel{R} 1$.



\item To prove the first proposition, assume that $R$ is both symmetric and antisymmetric.  We will prove that $R$ is transitive.  So let $a, b, c \in A$ and assume that $a \mathrel{R} b$ and 
$b \mathrel{R} c$. Since $R$ is symmetric, we can conclude that $b \mathrel{R} a$.  So we have 
$a \mathrel{R} b$ and $b \mathrel{R} a$.  Since $R$ is antisymmetric, this implies that 
$a = b$.  However, we then have $a = b$ and $b \mathrel{R} c$.  Hence, $a \mathrel{R} c$ and we have proven that $R$ is transitive.

\noindent
The second proposition is false.  A counterexample is $A = \{ x, y \}$ and $R = \{ (x, x) \}$.  This relation is both symmetric and antisymmetric but is not reflexive.
\end{enumerate}

\end{enumerate}

\hbreak
\endinput
















\item \begin{multicols}{2} 
\begin{enumerate}
\item 
\setlength{\unitlength}{0.5cm}
\begin{picture}(8,6)
\put(1,5){\circle*{.25}}
\put(7,5){\circle*{.25}}
\put(7,1){\circle*{.25}}
\put(1.2,5.2){\vector(1,0){5.6}}
%\put(7,4.8){\vector(0,-1){3.6}}
\put(6.8,4.8){\vector(-1,0){5.6}}
\put(0.6,5.3){$a$}
\put(7.3,5.3){$b$}
\put(7.3,0.8){$c$}
\end{picture}

\item 
\setlength{\unitlength}{0.5cm}
\begin{picture}(8,6)
\put(1,5){\circle*{.25}}
\put(7,5){\circle*{.25}}
\put(7,1){\circle*{.25}}
\put(1.2,5){\vector(1,0){5.6}}
\put(7,4.8){\vector(0,-1){3.6}}
\put(1.2,5){\vector(3,-2){5.6}}
\put(0.6,5.3){$a$}
\put(7.3,5.3){$b$}
\put(7.3,0.8){$c$}
\end{picture}

\item 
\setlength{\unitlength}{0.5cm}
\begin{picture}(8,6)
\put(1,5){\circle*{.25}}
\put(7,5){\circle*{.25}}
\put(7,1){\circle*{.25}}
\put(1.2,5.2){\vector(1,0){5.6}}
\put(6.8,4.8){\vector(-1,0){5.6}}
\put(1,4){\circle{2}}
\put(1,3){\vector(1,0){0}}
\put(7,4){\circle{2}}
\put(7,3){\vector(-1,0){0}}
\put(0.6,5.3){$a$}
\put(7.3,5.3){$b$}
\put(7.3,0.8){$c$}
\end{picture}

\item 
\setlength{\unitlength}{0.5cm}
\begin{picture}(8,6)
\put(1,5){\circle*{.25}}
\put(7,5){\circle*{.25}}
\put(7,1){\circle*{.25}}
\put(1.2,5){\vector(1,0){5.6}}
\put(7,4.8){\vector(0,-1){3.6}}
%\put(1.2,5){\vector(3,-2){5.6}}
\put(0.6,5.3){$a$}
\put(7.3,5.3){$b$}
\put(7.3,0.8){$c$}
\end{picture}

\item 
\setlength{\unitlength}{0.5cm}
\begin{picture}(8,6)
\put(1,5){\circle*{.25}}
\put(7,5){\circle*{.25}}
\put(7,1){\circle*{.25}}
\put(1.2,5.2){\vector(1,0){5.6}}
\put(6.8,4.8){\vector(-1,0){5.6}}
\put(1,4){\circle{2}}
\put(1,3){\vector(1,0){0}}
\put(7,4){\circle{2}}
\put(7,3){\vector(-1,0){0}}
\put(8,1){\circle{2}}
\put(8,2){\vector(-1,0){0}}
\put(0.6,5.3){$a$}
\put(7.3,5.3){$b$}
\put(7.3,0.8){$c$}
\end{picture}
\end{enumerate}
\end{multicols}
