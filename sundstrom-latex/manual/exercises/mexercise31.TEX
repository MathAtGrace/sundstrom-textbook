\section*{Section \ref{S:directproof}  Direct Proofs}

\begin{enumerate}
\item  \begin{enumerate}
\item Since $a \mid b$ and $a \mid c$, there exist integers $m$ and $n$ such that $b = am$ and $c = an$.  Hence,
\begin{align*}
b - c &= am - an \\
      &= a(m - n)
\end{align*}
Since $m - n$ is an integer (by the closure properties of the integers), the last equation implies that $a$ divides $b - c$.

\item Since $n$ is an odd integer, there exists an integer $k$ such that $n = 2k + 1$.  Then
\begin{align*}
n^3 &= (2k + 1)^3 \\
    &= 8k^3 + 12k^2 + 6k + 1 \\
    &= 2\left( 4k^3 + 6k^2 + 3k \right) + 1
\end{align*}
From the closure properties of the integers, we conclude that $\left( 4k^3 + 6k^2 + 3k \right)$ is an integer and so the last equation implies that $n^3$ is an odd integer.

\item We assume $a$ is an integer and that 4 divides $(a - 1)$.  Hence, there exists an integer $m$ such that $a - 1 = 4m$ or $a = 4m + 1$.  We then see that
\begin{align*}
a^2 - 1 &= (4m + 1)^2 - 1 \\
        &= 16m^2 + 8m + 1 - 1 \\
        &= 4 \left( 4m^2 + 2m \right)
\end{align*}
Since the integers are closed under addition and multiplication, we conclude that  $\left( 4m^2 + 2m \right)$ is an integer.  Hence, the last equation implies that 4 divides $\left( a^2 - 1 \right)$.
\end{enumerate}


\item \begin{enumerate}
\item The natural number 9 is a counterexample.
\item The natural number 5 is a counterexample.  With $n = 5$, $\left( 3 \cdot 2^5 + 2 \cdot 3^5 + 1 \right) = 583$ and $583 = 11 \cdot 53$.
\item A counterexample is $x = 0$ and $y = 0$.  With these values $\sqrt{x^2 + y^2} = 0$ and $2xy = 0$.
\item The integer 3 is a counterexample.  Using $a = 3$, $a^2 - 1 = 8$ and $a - 1 = 2$.  The integer 4 divides 8 but 4 does not divide 2.
\end{enumerate}


\item \begin{enumerate}
\item If $a \mid b$, then there exists an integer $m$ such that $b = am$.  Then, \\
$bc = \left( am \right)c = a \left( mc \right)$ and hence, $a \mid \left( bc \right)$.

\item The statement is false.  One counterexample is $b = 3$ and $c = 4$.

\item Since $a$ divides $(b - 1)$ and $a$ divides $(c - 1)$, there exist integers $m$ and $n$ such that $b - 1 = am$ and $c - 1 = an$.  So $b = 1 + am$ and $c = 1 + an$.  We then see that
\begin{align*}
bc - 1 &= (1 + am)(1 + an) - 1 \\
       &= (1 + am + an + a^2 mn) - 1 \\
       &= a(m + n + amn)
\end{align*}
Since $(m + n + amn)$ is an integer, this proves that $a$ divides $(bc - 1)$.

\item This statement is false.  One counterexample is $n = 5$.  With this value, $n^2 - 1 = 21$ and 7 divides 21.  However, $n - 2 = 3$ and 7 does not divide 3.

\item This statement is false.  One counterexample is $n = 0$.  Actually, any even integer will be a counterexample.

\item If $n$ is odd, then there exists an integer $m$ such that $n = 2m + 1$, and
\begin{align*}
4n^2 + 7n + 6 &= 6m^2 + 30m + 17 \\
              &= 2 \left( 3m^2 + 15m + 8 \right) + 1.
\end{align*}
So $4n^2 + 7n + 6$ is an odd integer.

\item The statement is false.  One counterexample is $d = 2$, $a = 5$, and $b = 3$.

\item This statement is false.  One counterexample is $a = 6$, $b = 3$, and $c = 4$.
\end{enumerate}


\item \begin{enumerate}
\item The only divisors of 1 are 1 and $-1$.
\item If $a \mid b$ and $b \mid a$, then there exist integers $m$ and $n$ such that 
$b = am$ and $a = bn$.  Substituting for $a$ gives $b = \left( bn \right) m$.  Since $b \ne 0$, this implies that $1 = nm$.  But then $n = \pm 1$ and hence, $a = \pm b$.
\end{enumerate}


\item If $a \mid ( 4n + 3 )$ and $a \mid ( 2n + 1 )$, then there exist integers $m$ and $q$ such that
\[
4n + 3 = am \qquad \text{and} \qquad 2n + 1 = aq.
\]
Multiplying the second equation by 2 yields $4n + 2 = 2aq$.  If we subtract this from the first equation, we obtain $1 = a ( m - 2q )$.  This implies that $a \mid 1$ and hence, 
$a = \pm 1$.


\item \begin{enumerate}
\item If $a \mid ( 8n + 7 )$ and $a \mid ( 4n + 1 )$, then there exist integers $m$ and $q$ such that
\[
8n + 7 = am \qquad \text{and} \qquad 4n + 1 = aq.
\]
Multiplying both sides of the second equation by 2 yields $8n + 2 = 2aq$.  If we subtract the sides of this from the corresponding sides of the first equation, we obtain $5 = a ( m - 2q )$.  This implies that $a$ divides 5. 

\item If $a \mid ( 9n + 5 )$ and $a \mid ( 6n + 1 )$, then there exist integers $m$ and $q$ such that
\[
9n + 5 = am \qquad \text{and} \qquad 6n + 1 = aq.
\]
If we multiply both sides of the first equation by 2 and multiply both sides of the second equation by 3, we obtain
\[
18n + 10 = 2am \qquad \text{and} \qquad 18n + 3 = 3aq.
\]
If we now subtract the corresponding sides of these two equations, we obtain $7 = 2am - 3 aq$ or 
$7 = a(2m - 3q)$.  This implies that $a$ divides 7. 

\item Since $n$ is odd, there exists an integer $k$ such that $n = 2k + 1$.  We substitute and use algebra to obtain
\begin{align*}
n^4 + 4n^2 + 11 &= (2k + 1)^4 + 4(2k + 1)^2 + 11 \\
                &= 16k^4 + 32k^3 + 40k^2 + 24k + 16 \\
                &= 8 \left( 2k^4 + 4k^3 + 5k^2 + 3k + 2 \right)
\end{align*}
This proves that 8 divides $\left( n^4 + 4n^2 + 11 \right)$.

\item This statement is false.  A counterexample is $n = 1$.
\end{enumerate}



\item \begin{enumerate}
\item If $a \equiv 0 \pmod n$, then $n \mid \left( a - 0 \right)$, or $n \mid a$.
\item If $n \mid a$, then $n \mid \left( a - 0 \right)$ and hence, $a \equiv 0 \pmod n$.
\end{enumerate}



\item Assuming that $a$ and $b$ are both congruent to 2 modulo 3, there exist integers $m$ and $n$ such that $a = 3m + 2$ and $b = 3n + 2$.  Then,
\begin{align*}
a + b &= (3m + 2) + (3n + 2) \\
      &= 3m + 3n + 3 + 1 \\
      &= 3(m + n + 1) + 1
\end{align*}
and this proves that $a + b \equiv 1 \pmod 3$.  Also,
\begin{align*}
ab &= (3m + 2)(3n + 2) \\
   &= 9mn + 6m + 6n + 4 \\
   &= 3(3mn + 2m + 2n + 1 ) + 1
\end{align*}
and this proves that $ab \equiv 1 \pmod 3$.



\item Assuming that $a \equiv 7 \pmod 8$ and $b \equiv 3 \pmod 8$, there exist integers $m$ and $n$ such that $a = 8m + 7$ and $b = 8n + 3$.  Then,
\begin{align*}
a + b &= (8m + 7) + (8n + 3) \\
      &= 8m + 8n + 8 + 2 \\
      &= 8(m + n + 1) + 2
\end{align*}
and this proves that $a + b \equiv 2 \pmod 8$.  Also,
\begin{align*}
ab &= (8m + 7)(8n + 3) \\
   &= 64mn + 24m + 56n + 21 \\
   &= 8(8mn + 3m + 7n + 2) + 5
\end{align*}
and this proves that $ab \equiv 5 \pmod 8$.



\item \begin{enumerate}
\item This statement is false.  A counterexample is $a = 3$ and $b = 2$.  With these values 
$\mod{ab}{0}{6}$ and $a \not \equiv 0 \pmod 6$  and $b \not \equiv 0 \pmod 6$.


\item Assuming $\mod{a}{2}{8}$, we know that 8 divides $a - 2$ and hence, there exists an integer $m$ such that $a - 2 = 8m$.  So $a = 2 + 8m$ and
\begin{align*}
a^2 - 4 &= (2 + 8m)^2 - 4 \\
        &= 32m + 64m^2 \\
        &= 8 \left( 4m + 8m^2 \right)
\end{align*}
This proves that 8 divides $\left( a^2 - 4 \right)$ and hence, $\mod{a^2}{4}{8}$.


\item This statement is false.  Notice that if $a = 6$, then $a^2 = 36$ and $\mod{36}{4}{8}$.  However, 
$6 \not \equiv 2 \pmod 8$, and so $a = 6$ is a counterexample.
\end{enumerate}


\item \begin{enumerate}
\item Since $a - a = 0$ and $n$ divides 0, $\mod{a}{a}{n}$.

\item If  $a \equiv b \pmod n$,  then  $n \mid ( a - b )$.  Hence, there exists an integer $k$ such that $a - b = nk$.  Then, $b - a = n ( -k )$ and $b \equiv a \pmod n$.

\item If $a \equiv b \pmod n$ and $b \equiv b \pmod n$,  then  $n \mid ( a - b )$ and $n \mid (b - c)$.  Hence, there exists an integer $k$ such that $a - b = nk$ and there exists an integer $m$ such that 
$b - c = nm$.  If we add the corresponding sides of these two equations, we obtain
\begin{align*}
(a - b) + (b - c) &= nk + nm \\
          (a - c) &= n(k + m)
\end{align*}
Since $(k + m)$ is an integer, the last equation shows that $n$ divides $(a - c)$ and hence, $\mod{a}{c}{b}$.
\end{enumerate}


\item If $a \equiv b \pmod n$ and $c \equiv d \pmod n$, then $n \mid \left( a - b \right)$ and 
$n \mid \left( c - d \right)$.  Hence, there exist integers $m$ and $q$ such that 
$a - b = nm$ and $c - d = nq$.  Solving for $a$ and $c$, we obtain
\[
a = b + nm \qquad \text{and} \qquad c = d + nq.
\]
\begin{enumerate}
\item So, $a + c = \left( b + nm \right) + \left( d + nq \right)$.  This implies that
\[
\left( a + c \right) - \left( b + d \right) = n \left( m + q \right),
\]
and hence, $\left( a + c \right) \equiv \left( b + d \right) \pmod n$.

\item Also, $ac = \left( b + nm \right) \left( d + nq \right)$.  This implies that
\[
ac - bd = n \left( bq + dm + mnq \right),
\]
and hence, $ac \equiv bd \pmod n$.
\end{enumerate}


\item \begin{enumerate}
\item The quadratic formula gives the solutions for the quadratic equation $ax^2 + bx + c = 0$ and is
\[
x = \frac{-b \pm \sqrt{b^2 - 4ac}}{2a}.
\]
The discriminant is $b^2 - 4ac$.
\begin{itemize}
  \item If $b^2 - 4ac > 0$, then the quadratic equation has two real number solutions.
  \item If $b^2 - 4ac = 0$, then the quadartic equation has one real number solution.
  \item if $b^2 - 4ac < 0$, then the quadratic equation has no real number solutions.
\end{itemize}

\item If $a > 0$ and $c < 0$, then $-4ac > 0$ and hence, $b^2 - 4ac > 0$.  Therefore, the discriminant is positive and the quadratic equation has two real number solutions.  Since $-4ac > 0$, we also see that
\[
b^2 - 4ac > b^2.
\]
From this, we conclude that $\sqrt{b^2 - 4ac} > \sqrt{b^2}$ and so, $\sqrt{b^2 - 4ac} > b$ and hence, 
$-b + \sqrt{b^2 - 4ac} > 0$.  In addition $a > 0$ and so the solution $\dfrac{-b + \sqrt{b^2 - 4ac}}{2a}$ is a positive real number.

\item If $a \ne 0$, $b > 0$, and $\dfrac{b}{2} < \sqrt{ac}$, then $\dfrac{b^2}{4} < ac$ and hence, $b^2 < 4ac$.  This implies that $b^2 - 4ac < 0$ and so the quadrtic equation has no real number solutions.
\end{enumerate}


\item If $(a, b)$ is on or inside the circle $(x - 1)^2 + (y - 2)^2 = 4$, then we know that 
$(a - 1)^2 + (b - 2)^2 \leq 4$, and so 
\[
(a - 1)^2 \leq 4 \qquad \text{and} \qquad (b - 2)^2 \leq 4.  
\]
This two inequalities imply that
\[
-2 \leq a - 1 \leq 2 \qquad \text{and} \qquad -2 \leq b - 2 \leq 4.
\]
We can then conclude that $-1 \leq a \leq 3$ and $0 \leq b \leq 4$.  We can now rewrite the inequality $(a - 1)^2 + (b - 2)^2 \leq 4$ as follows:
\begin{align*}
(a^2 - 2a + 1) + (b^2 - 4b + 4) &\leq 4 \\
a^2 + b^2 &\leq -1 + 2a + 4b \\
a^2 + b^2 &\leq -1 + 2(3) + 4(4) \\
a^2 + b^2 &\leq 21.
\end{align*}
This implies that $a^2 + b^2 < 26$ and hence, $(a, b)$ is inside the circle $x^2 + y^2 = 26$.




\item With $x^2 + y^2 = r^2$, we have the following:
\begin{enumerate}
\item $2x + 2y \dfrac{dy}{dx} = 0$.  Solving for $\dfrac{dy}{dx}$ gives 
$\dfrac{dy}{dx} = -\dfrac{x}{y}$.

\item If $(a, b)$ is on the circle, the slope of the line tangent to the circle at $(a, b)$ is 
$-\dfrac{a}{b}$.

\item The slope of the radius from the origin to the point $(a, b)$ is $\dfrac{b}{a}$.  So the product of this slope with the slope of the tangent line in~(b) is $-1$.  This means that the radius is perpendicular to the line tangent to the circle.
\end{enumerate}


\item \begin{enumerate}
\item This statement is false.  One counterexample is $x = -1$ and $y = -1$.

\item Let $x, y \in \R$.  We start with the fact that $(x - y)^2 \geq 0$.  We then have 
$0 \leq x^2 - 2xy + y^2$.  We now add $4xy$ to both sides of this inequality.  This gives
\begin{align*}
4xy &\leq x^2 + 2xy + y^2 \\
 xy &\leq \frac{(x + y)^2}{4} \\
 xy &\leq \left( \frac{x + y}{2} \right)^2.
\end{align*}

\item Let $x$ and $y$ be nonnegative real numbers.  We can use the result in Part~(b) to prove this result.  Since $xy \geq 0$ and $\dfrac{x+y}{2} \geq 0$, we can take the square root of both sides of the inequality in Part~(b) to obtain
\[
\sqrt{xy} \leq \frac{x + y}{2}.
\]
\end{enumerate}


\item Let $a$ be a real number and let $y = x(a - x)$.  When $x = \dfrac{a}{2}$, we see that $y = \dfrac{a^2}{4}$.  Now for any real number $x$, we can use the result in Part~(b) of Exercise~(16) to see that
\begin{align*}
x(a - x) &\leq \left( \frac{x + (a - x)}{2} \right)^2 \\
x(a - x) &\leq \left( \frac{a}{2} \right)^2 \\
x(a - x) &\leq \frac{a^2}{4}
\end{align*}
This shows that the maximum value for $y = x(a - x)$ occurs when $x = \dfrac{a}{2}$.


\item \begin{enumerate} \setcounter{enumii}{1}
\item The area of the right triangle is $\dfrac{1}{2}xh$.  Using the Pythagorean Theorem, we see that $h^2 + \dfrac{x^2}{4} = x^2$ and this implies that $h^2 = \dfrac{3}{4} x^2$.  Since $h$ and $x$ are positive, $h = \dfrac{\sqrt{3}}{2} x$, and so the area of the right triangle is
\[
\frac{1}{2}xh = \frac{\sqrt{3}}{4}x^2.
\]

\item The area of the equilateral triangle on the hypotenuse is $\dfrac{\sqrt{3}}{4}c^2$ and by the Pythagorean Theorem $c^2 = a^2 + b^2$.  Therefore,
\begin{align*}
\frac{\sqrt{3}}{4}c^2 &= \frac{\sqrt{3}}{4}(a^2 + b^2) \\
                      &= \frac{\sqrt{3}}{4}a^2 + \frac{\sqrt{3}}{4}b^2
\end{align*}
This shows that the area of the equilateral triangle on the hypotenuse is equal to the sum of the areas of the equilateral triangles on the other two sides.
\end{enumerate}
\end{enumerate}



\subsection*{Evaluation of Proofs}
\setcounter{oldenumi}{\theenumi}
\begin{enumerate} \setcounter{enumi}{\theoldenumi}
\item \begin{enumerate}
\item This proof is not well-written.  The assumptions for the proposition are not stated at the beginning of the proof and there is no indication of what will be proven.  In particular, the variables $m$ and $n$ are not defined in the body of the proof. Also,  the important equations are not displayed.  A better way to write the proof is as follows:

\begin{myproof}
Let $m$ be an even integer.  Then there exists an integer $n$ such that $m = 2n$.  Using this, we see that
\begin{align*}
5m + 4 &= 5(2n) + 4 \\
       &= 2(5n + 2).
\end{align*}
Using the closure properties of the integers, we know that $(5n + 2)$ is an integer and, so, the last equation implies that $5m + 4$ is an even integer.  This proves that if $m$ is an even  integer, then $5m + 4$ is an even integer.
\end{myproof}


\item The proposition is true, but the proof is not a valid proof.  Basically, the proof begins with the conclusion of the proposition and proceeds to prove the hypothesis.  This is essentially a proof of the converse of the proposition.


In addition, the assumptions for the proposition are not stated at the beginning of the proof and there is no indication of what will be proven.  In particular, the variables $x$ and $y$ are not defined in the body of the proof.  
Following is a valid proof of this proposition.

\begin{myproof}
Let $x$ and $y$ be positive real numbers and assume that $x \ne y$.  We will prove that 
$\dfrac{x}{y} + \dfrac{y}{x} > 2$.  Since $x \ne y$, we can conclude that $x - y \ne 0$ and hence, $(x - y)^2 > 0$.  From this, we obtain
\begin{align*}
      (x - y)^2 &> 0 \\
x^2 - 2xy + y^2 &> 0 \\
      x^2 + y^2 &> 2xy.
\end{align*}
We also know that $xy \ne 0$ and so, we can divide both sides of the last inequality to obtain
\begin{align*}
           \frac{x^2 + y^2 }{xy}&> \frac{2xy}{xy} \\
\frac{x^2}{xy} + \frac{y^2}{xy} &> 2 \\
      \frac{x}{y} + \frac{y}{x} &> 2.
\end{align*}
So we have proved that for all real numbers $x$ and $y$, if $x \ne y$, $x > 0$, and $y >0$, then $\dfrac{x}{y} + \dfrac{y}{x} > 2$.
\end{myproof}


\item This proposition is false.  A counterexample is $a = 4$, $b = 2$, and $c = 3$.  In this case, 
$bc = 12$ and so, $a \mid (bc)$.  However, $a$ does not divide $b$, and $a$ does not divide $c$.

\vskip6pt
\noindent
The problem with the proposed proof is that from the fact that $bc = mna$, it is not possible to conclude that $b = ma$ or $c = na$.


\item We should not start the counterexample with the equation $\left( a^b \right)^c = a^{\left( b^c \right)}$ since by doing so, we are implying the equation is true.  We could write that with $a = 2$, $b = 3$, and $c = 2$, we see that

\begin{align*}
\left( a^b \right)^c &= \left( 2^3 \right)^2 &\text{and}& &a^{\left( b^c \right)} &= 2^{\left( 3^2 \right)} \\
                   &= 8^2  & & & &= 2^9\\
                   &= 64   & & & &= 512
\end{align*}
Since $64 \ne 512$, this is a counterexample that shows the proposition is false.
\end{enumerate}
\end{enumerate}


\subsection*{Explorations and Activities}
\setcounter{oldenumi}{\theenumi}
\begin{enumerate} \setcounter{enumi}{\theoldenumi}
\item \begin{enumerate}
\item The integers that are congruent to 5 modulo 6 are in the set \\  
$\left\{ { \ldots ,  - 13,  - 7,  - 1, 5, 11, 17,  \ldots } \right\}$.

\item For each integer  $m$  from Part (1), $m^2  \equiv 1 \pmod 6$.  

\item For each integer $m$, if $\mod{m}{5}{6}$, then $\mod{m^2}{1}{6}$.

\item \textbf{Proposition}:  For all  $m \in \mathbb{Z}$, if  $m \equiv 5 \pmod 6$, then  
$m^2  \equiv 1 \pmod 6$.
\begin{myproof}
Let $m$ be an integer and assume that $m \equiv 5 \pmod 6$.  We will prove that 
$m^2  \equiv 1 \pmod 6$.  From the assumption, we know that 6 divides $(m - 5)$ and hence, there exists an integer $k$ such that $m - 5 = 6k$.  We can then conclude that 
$m = 5 + 6k$ and use algebra as follows:
\begin{align*}
m^2 - 1 &= \left( 5 + 6k \right)^2 - 1 \\
        &= \left( 25 + 60k + 36k^2 \right) - 1 \\
        &= 24 + 60k + 36k^2 \\
        &= 6 \left( 4 + 10k + 6k^2 \right)
\end{align*}
From the closure properties of the integers, we know that 
$\left( {4 + 10k + 6k^2 } \right)$ is an integer, and hence, the last equation proves that 6 divides $\left( m^2 - 1 \right)$, which in turn means that $m^2  \equiv 1 \pmod 6$.  This proves that if $m \equiv 5 \pmod 6$, then  $m^2  \equiv 1 \pmod 6$.
\end{myproof}
\end{enumerate}


\item \begin{enumerate}
\item \textbf{Theorem}.  The only Pythagorean triple consisting of three consecutive natural numbers is 3, 4, 5.

\begin{myproof}
Let $n$ be a natural number.  We can then represent .  For these three numbers to form a Pythgorean triple, we need $n^2 + (n + 1)^2 = (n + 2)^2$.  We rewrite this equation as follows:
\begin{align*}
n^2 + (n + 1)^2 &= (n + 2)^2 \\
n^2 + \left( n^2 + 2n + 1 \right) &= n^2 + 4n + 4 \\
n^2 - 2n - 3 &= 0
\end{align*}
We can factor the left side of the last equation and solve the resulting equation.  This gives $(n - 3)(n + 1) = 0$.  The solutions are $n = 3$ and $n = -1$.  Since $n$ is a natural number, we conclude $n = 3$.  This proves that the only Pythagorean triple consisting of three consecutive natural numbers is 3, 4, 5.
\end{myproof}


\item We represent three natural numbers as $m$, $m + 7$, and $m +8$, for some integer $m$.  For these three numbers to form a Pythgorean triple, we need $m^2 + (m + 7)^2 = (m + 8)^2$.  We rewrite this equation as follows:
\begin{align*}
m^2 + (m + 7)^2 &= (m + 8)^2 \\
m^2 + \left( m^2 + 14m + 49 \right) &= m^2 + 16m + 64 \\
m^2 - 2m - 15 &= 0
\end{align*}
We can factor the left side of the last equation and solve the resulting equation.  This gives $(m - 5)(m + 3) = 0$.  The solutions are $m = 5$ and $m = -3$.  Since $m$ is a natural number, we conclude $m = 5$.  This proves that the only Pythagorean triple of the form $m$, $m + 7$, and $m + 8$ is 5, 12, 13.
\end{enumerate}

\end{enumerate}

\hbreak
\endinput
