\section*{Section \ref{S:quantifier} Quantifiers and Negations}

\begin{enumerate}
\item \begin{enumerate}
\item There exists a rational number $x$ such that $x^2 - 3x - 7 = 0$.  Using the quadratic formula, the two solutions of the equation are $\dfrac{3 \pm \sqrt{37}}{2}$, neither of which is a rational number.  Therefore, the statement is false.
\item There exist a real number $x$ such that $x ^ 2 + 1 = 0$.  This statement is false since the square of any real number is greater than or equal to 0, and so for each real number $x$, $x^2 + 1 \geq 1$.
\item There exists a natural number $m$ such that $m^2 < 1$.  This is false since for each natural number $m$, $m \geq 1$ and so $m^2 \geq 1$.
\end{enumerate}



\item \begin{enumerate}
\item Any odd integer is a counterexample.
\item $x = 0$ is a counterexample.
\item Any negative real number is a counterexample.
\item $m = 1$ is a counterexample.
\item Any negative integer is a counterexample.
\item $x = \frac{\pi}{2}$ is a counterexample.
%\item Both solutions for the equation are irrational.
%\item Any odd integer is a counterexample.
%\item For each $x \in \mathbb{R}, x^2 + 1 \ne 0$.
\end{enumerate}

\item \begin{enumerate}
\item There exists a rational number $x$ such that $x > \sqrt{2}$. \\
The negation is, $\left( {\forall x \in \mathbb{Q}} \right)\left( {x \leq \sqrt 2 } \right)$, or\\ for each rational number $x$, $x \le \sqrt{2}$.

\item For each rational number $x$, $x^2 - 2 \ne 0$. \\
The negation is, $\left( \exists x \in \mathbb{Q} \right) \left( x^2 - 2 = 0 \right)$, or\\ 
there exists a rational number $x$ such that $x^2 - 2 = 0$.

\item For each integer $x$, $x$ is even or $x$ is odd. \\
The negation is, $\left( \exists x \in \mathbb{Z} \right) \left( x \text{ is odd and $x$ is even} \right)$, or \\
there exists an integer $x$ such that $x$ is odd and $x$ is even.

\item There exists a rational number $x$ such that $\sqrt{2} < x < \sqrt{3}$. \\
The negation is, $\left( {\forall x \in \mathbb{Q}} \right)\left( {x \leq \sqrt 2 \text{ or } x \geq \sqrt 3 } \right)$, or \\
for each rational number $x$, $x \leq \sqrt 2$ or $x \geq \sqrt 3$.

\item For each integer $x$, if $x^2$ is odd, then $x$ is odd. \\
The negation is, $\left( \exists x \in \mathbb{Z} \right) \left( x^2 \text{ is odd and $x$ is even} \right)$, or \\
there exists an integer $x$ such that $x^2$ is odd and $x$ is even.

\item For each natural number $n$, if $n$ is a perfect square, then $\left( 2^n - 1 \right)$ is not a prime number. \\
The negation is, $\left( \exists n \in \mathbb{N} \right) \left( n \text{ is a perfect square and 
$\left( 2^n - 1 \right)$ is a prime number} \right)$, or 
there exists a natural number $n$ such that $n$ is a perfect square and $\left( 2^n - 1 \right)$ is a prime number.

\item For each natural number $n$, $n^2 - n + 41$ is a prime number. \\
The negation is, $\left( \exists n \in \mathbb{N} \right) \left( \left( n^2 - n + 41 \right) \text{ is not a prime number} \right)$, or 
there exists a natural number $n$ such that $n^2 - n + 41$ is not a prime number.

\item There exists a real number $x$ such that $\cos \left( 2x \right) = 2 \left( \cos x \right)$. \\
The negation is, $\left( {\forall x \in \mathbb{R}} \right) \left( \cos \left( 2x \right) \ne 2 \left( \cos x \right) \right)$, or \\
for each real number $x$, $\cos \left( 2x \right) \ne 2 \left( \cos x \right)$.
\end{enumerate}

\item \begin{enumerate}
\item There exist integers $m$ and $n$ such that $m > n$.
\item There exists an integer $m$ such that for each integer $n$, $m > n$.
\item For each integer $m$, thee exists an integer $n$ such that $m > n$.
\item For each integer $m$ and for each integer $n$, $m > n$.
\item There exists an integer $n$ such that for each integer $m$, $m^2 > n$.
\item For each integer $n$, there exists an integer $m$ such that $m^2 > n$.
\end{enumerate}

\item \begin{enumerate}
\item $\left( \forall m \right) \left( \forall n \right) \left( m \le n \right)$.  \\
For all integers $m$ and $n$ , $m \le n$.

\item $\left( \forall m \right) \left( \exists n \right) \left( m \le n \right)$.  \\
For each integer $m$, there exists an integer $n$ such that $m \le n$.

\item $\left( \exists m \right) \left( \forall n \right) \left( m \le n \right)$.  \\
There exists an integer $m$ such that for each integer $n$ such that $m \le n$.

\item $\left( \exists m \right) \left( \exists n \right) \left( m \le n \right)$.  \\
There exist integers $m$ and $n$ such that $m \le n$.

\item $\left( \forall n \right) \left( \exists m \right) \left( m \le n \right)$.  \\
For each integer $n$, there exists an integer $m$ such that  $m^2 \le n$.

\item $\left( \exists n \right) \left( \forall m \right) \left( m \le n \right)$.  \\
There exists an integer $n$ such that for each integer $m$, $m^2 \le n$.
\end{enumerate}



\item \begin{enumerate}
\item It is not a statement since $x$ is an unquantified variable.
\item It is a true statement.
\item It is a false statement.
\item It is a true statement.
\item $\left\{ -20, -10, -5, -4, -2, -1, 1, 2, 4, 5, 10, 20 \right\}$
\end{enumerate}


\item \begin{enumerate}
\item It is not a statement since $x$ is an unquantified variable.
\item It is a true statement.
\item It is a true statement.
\item It is a false statement.
\item $\left\{ x \in \R \mid x \ne 0 \right\}$
\end{enumerate}



\item \begin{enumerate}
\item We can use $x = 2$ for a counterexample.  For $x = 2$, there is no integer $y$ such that $2y = 1$.  Any integer except $x = 1$ and $x = -1$ is a counterexample.
\item $\left( \forall x \in \Z^* \right) \left( \exists y \in \Z^* \right) (xy = 1)$.
\item $\left( \exists x \in \Z^* \right) \left( \forall y \in \Z^* \right) (xy \ne 1)$.
\item There exists a nonzero integer $x$ such that for each nonzero integer $y$, $xy \ne 1$.
\end{enumerate}


\item \begin{enumerate}
\item The integer $m$ satisfies the divides property provided that \\$\left( \forall a, b \in \Z \right) \left( \text{if } m \mid (ab), \text{ then } m \mid a \text{ or } m \mid b \right)$.
\item The integer $m$ does not satisfy the divides property provided that \\$\left( \exists a, b \in \Z \right) \left(  m \mid (ab), \text{ and } m \text{ does not divide } a \text{ and } m \text{ does not divide } b \right)$.
\item The integer $m$ does not satisfy the divides property provided that there exist integers $a$ and $b$ such that $m$ divides $ab$, $m$ does not divide $a$, and $m$ does not divide $b$.
\end{enumerate}



\item \begin{enumerate}
\item A function $f$ with domain $\mathbb{R}$ is strictly increasing provided that \\
$\left( \forall x, y \in \mathbb{R} \right) \left( f \left( x \right) < \left( y \right) \text{ whenever } x < y \right)$.

\item A function $f$ with domain $\mathbb{R}$ is not strictly increasing provided that \\
$\left( \exists x, y \in \mathbb{R} \right) \left( f \left( x \right) \ge \left( y \right) \text{ and } x < y \right)$.

\item A function $f$ with domain $\mathbb{R}$ is not strictly increasing provided that there exist real numbers $x$ and $y$ such that $f \left( x \right) \ge \left( y \right)$ and $x < y$.
\end{enumerate}

\item \begin{enumerate}
\item A function $f$ is continuous at the real number $a$ provided that \\
$\left( \forall \epsilon > 0 \right) \left( \exists \delta > 0 \right) 
\left( \left| x - a \right| < \delta \to \left| f \left( x \right) - f \left( a \right) \right| < \epsilon \right)$

\item A function $f$ is not continuous at the real number $a$ provided that \\
$\left( \exists \epsilon > 0 \right) \left( \forall \delta > 0 \right) 
\left[ \left( \exists x \right) \left( \left| x - a \right| < \delta \text{ and } 
\left| f \left( x \right) - f \left( a \right) \right| \ge \epsilon \right) \right]$.  Notice the existential quantifier for $x$ due to the hidden universal quantifier for the conditional sentence in Part~(a).

\item A function $f$ is not continuous at the real number $a$ provided that there exists an $\epsilon > 0$ such that for all $\delta > 0$, there exists an $x$ such that 
$\left| x - a \right| < \delta$ and 
$\left| f \left( x \right) - f \left( a \right) \right| \ge \epsilon$.
\end{enumerate}


\item \begin{enumerate}
\item  An operation $*$ on a set $A$ is not commutative provided that there exist $x, y \in A$ such that 
$x*y \ne y*x$.
\item  A nonzero element $a$ is a ring $R$ is not a zero divisor provided that for each nonzero element $b$ in $R$, $ab \ne 0$.
\item A set $M$ of real numbers is not a neighborhood of the real number $a$ provided that for each positive real number $\epsilon$, the open interval $(a - \epsilon, a + \epsilon)$ is not contained in $M$.
\item A sequence of real numbers $\{x_1, x_2, \ldots, \x_k, \ldots \}$ is not a Cauchy sequence provided that there exists a positive real number $\epsilon$ such that for every natural number $N$, there exist $m,n \in \N$ such that $m > N$, $n > N$, and $|x_n - x_m | \geq \epsilon$.
\end{enumerate}
\end{enumerate}


\subsection*{Explorations and Activities}
\setcounter{oldenumi}{\theenumi}
\begin{enumerate} \setcounter{enumi}{\theoldenumi}
\item \begin{enumerate}
\item Examples of natural numbers that are prime numbers are:  11, 13, 17, 19, 23, 29, 31, 37.

\item The conditional statement,``For all  $d \in \mathbb{N}$, if  $d$  is a factor of  $p$, then  $d = 1$ or  $d = p$'', means that the only natural number factors of  $p$  are  1  and  $p$.  Hence, this means that  $p$  is a prime number.

\item Examples of composite numbers are:  4, 6, 8, 9, 14, 15, 16, 18, 20, 21 since each of these numbers can be factored into a product of two natural numbers, neither of which is 1.

\item An integer  $n$ is a composite number provided that it is greater than one and that there exists a $d \in \mathbb{N}$  such that  $d$  is a factor of  $n$  and  $d \ne 1$ and  $d \ne n$.
\end{enumerate}



\item \begin{enumerate}
\item Let  $A$  be a subset of the real numbers.  A number  $b$  is called an upper bound for the set  $A$ provided that $\left( \forall x \in A\right) \left( x \leq b \right)$.

\item Three different upper bounds for the set  
$A = \left\{ {\left. {x \in \mathbb{R} } \right| 1 \leq x \leq 3} \right\}$
 are  3, $\pi $, and 20.

\item The set  $A = \left\{ {x \in \mathbb{R}\left.   \right| x > 0} \right\}$
 does not have an upper bound since for all real numbers  $b$, there exists an element  $a \in A$
 such that  $a > b$.

\item Three different real numbers that are not upper bounds for the set  
$A = \left\{ {\left. {x \in \mathbb{R} } \right| 1 \leq x \leq 3} \right\}$
 are 2.99, 1, and $-5$.

\item Let  $A$  be a subset of the real numbers.  A number  $b$  is not an upper bound for the set  $A$   provided that  $\left( \exists x \in A\right) \left( x > b \right)$.

\item Let  $A$  be a subset of the real numbers.  A number  $b$  is not an upper bound for the set  $A$  provided that there exist an element  $x$  in  $A$  such that  $x > b$.

\item The examples in Part (4) are consistent with  Part (6) as the following shows:

\begin{itemize}
\item The number 2.99  is not an upper bound for  $A$  since  $3 \in A$ and  $3 > 2.99$.

\item The number 1  is not an upper bound for  $A$  since  $3 \in A$ and  $3 > 1$.

\item The number $-5$  is not an upper bound for  $A$  since  $3 \in A$ and  $3 >  - 5$.
\end{itemize}

\end{enumerate}




\item \begin{enumerate}
\item The universal quantifier was used for the real number  $\beta $
  since  $\beta $  represents an upper bound for  $A$  and the least upper bound  $\alpha $
 must be less than or equal to every other upper bound for  $A$.

%\item We use one of De Morgan's Laws to negate a conjunction: 
%\[
%\mynot \left( {P \wedge Q} \right) \equiv \mynot P \vee \mynot Q.
%\]

\item A real number  $\alpha $ is not the least upper bound for  $A$  provided that  
\[
\mynot P\left( \alpha  \right) \vee \left[ {\left( {\exists \beta  \in \mathbb{R}} \right)\left( {P\left( \beta  \right) \wedge \left( {\alpha  > \beta } \right)} \right)} \right].
\]

\item A real number  $\alpha $ is not the least upper bound for  $A$  provided that  $\alpha $ is not an upper bound for  $A$  or there exists an upper bound  $\beta $  for  $A$  such that  $\alpha  > \beta $.
\end{enumerate}



\end{enumerate}
\hbreak
\endinput

