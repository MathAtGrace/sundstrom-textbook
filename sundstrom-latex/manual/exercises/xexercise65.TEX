\section*{Section \ref{S:inversefunctions} Inverse Functions}

\begin{enumerate}
\item \begin{enumerate}
\item The inverse of $f$ will not be a function in this situation.  One example is $f:A \to B$ by $f \left( 1 \right) = a$, $f \left( 2 \right) = b$, and $f \left( 3 \right) = a$.  Then, 
$f^{-1} = \left\{ \left( a, 1 \right), \left( b, 2 \right), \left( a, 3 \right) \right\}$.

\item The inverse of $g$ will be a function in this situation.  One example is $g:A \to B$ by 
$g \left( 1 \right) = c$, $g \left( 2 \right) = b$, and $g \left( 3 \right) = a$.  Then, 
$g^{-1} = \left\{ \left( a, 3 \right), \left( b, 2 \right), \left( c, 1 \right) \right\}$.
\end{enumerate}



\item \begin{enumerate}
\item The function $f$ is a bijection.

\item $f^{ - 1}  = \left\{ {\left( {c,a} \right),\left( {b,b} \right),\left( {d,c} \right),\left( {a,d} \right)} \right\}$.

\addtocounter{enumii}{1}
\item $\left( {f^{ - 1}  \circ f} \right)\left( x \right) = x = \left( {f \circ f^{ - 1} } \right)\left( x \right)$.  This illustrates Corollary~\ref{C:inversecomposition}.
\end{enumerate}



\item \begin{enumerate}
\item This is a use of Corollary~\ref{C:inversecomposition} since the cube root function and the cubing function are inverse functions of each other and consequently, the composition of the cubing function with the cube root function is the identity function.

\item This is a use of Corollary~\ref{C:inversecomposition} since the natural logarithm function  and the exponential function with base $e$ are inverse functions of each other and consequently, the composition of the natural logarithm function with the exponential function with base $e$ is the identity function.

\item They are similar because they both use the concept of an inverse function to ``undo'' one side of the equation.
\end{enumerate}



\item Let $y \in B$.  Since $f$ is a surjection, there exists an $x$ in $A$ such that 
$f(x) = y$.  By Theorem~6.29, we conclude that $f^{-1}(y) = x$.  We then see that 
\begin{align*}
\left( f \circ f^{-1} \right)(y) &= f \left( f^{-1}(y) \right) \\
                                 &= f(x) \\
                                 &= y.
\end{align*}
Hence, for each $y \in B$, $\left( f \circ f^{-1} \right)(y) = y$.


\item Let $A$ and $B$ be nonempty sets and let $f\x A \to B$ be a bijection.  Then
\begin{itemize}
\item $f^{-1} \circ f = I_A$

\item $f \circ f^{-1} = I_B$
\end{itemize}



\item \begin{enumerate}
\item If  $g \circ f = I_A $, then  $f$  is an injection.

\textbf{\emph{Proof}.}  Let $x, y \in A$ and assume that 
$f \left( x \right) = f \left( y \right)$.  Then, applying $g$ to both sides of this equation yields
\[
\begin{aligned}
g \left( f \left( x \right) \right) &= g \left( f \left( y \right) \right) \\
\left( g \circ f \right) \left( x \right) &= \left( g \circ f \right) \left( y \right). \\
\end{aligned}
\]
Since $g \circ f = I_A $, the last equation implies that $x = y$ and hence that $f$ is an injection.

\item If  $f \circ g = I_B $, then  $f$  is a surjection.

\textbf{\emph{Proof}.}  Assume that $f \circ g = I_B $, and let $ y \in B$.  Then,
\[
\begin{aligned}
\left( f \circ g \right) \left( y \right) &= I_B \left( y \right) \\
f \left( g \left( y \right) \right) &= y. \\
\end{aligned}
\]
Notice that $g \left( y \right) \in A$ and so if $a = g \left( y \right)$, then there exists an 
$a \in A$ such that $f \left( a \right) = y$.  This proves that $f$ is a surjection.

\item If $g \circ f = I_A$ and $f \circ g = I_B$, then by Part~(a), $f$ is an injection, and by Part~(b), $f$ is a surjection.  Similary, by Part~(a), $g$ is a surjection, and by Part~(b), $g$ is an injection.  Hence, both $f$ and $g$ are bijections.

Now let $x \in A$ and $y \in B$.  Since $g \circ f = I_A$ and $f \circ g = I_B$, we can see that 
$y = f \left( x \right)$ if and only if $x = g \left( y \right)$.  Hence, $g = f^{-1}$. 
\end{enumerate}



\item \begin{enumerate}
\item 
\[
\begin{aligned}
y &= e^{2x - 1} \\
\ln y &= 2x - 1 \\
x &= \frac{1}{2}\left( \ln y + 1 \right)\\
\end{aligned}
\]

\item $g: \mathbb{R}^+ \to \mathbb{R}$ by 
$g \left( y \right) = \dfrac{1}{2}\left( \ln y + 1 \right)$.

\item \begin{multicols}{2}
\[
\begin{aligned}
\left( g \circ f \right) \left( x \right) &= g \left( e^{2x-1} \right) \\
                               &= \frac{1}{2} \ln \left( \left( 2x - 1 \right) + 1 \right) \\ 
                               &= \frac{1}{2} \ln \left( 2x \right) \\
                               &= \frac{1}{2} \left( 2 \ln x \right) \\
                               &= \ln{x}.\\
\end{aligned}
\]

\[
\begin{aligned}
\left( f \circ g \right) \left( y \right) &= f \left( \frac{1}{2} \left( \ln y + 1 \right)\right) \\
                         &= e^{2 \frac{1}{2} \left( \ln y + 1 \right) - 1} \\
                         &= e^{\left( \ln y + 1 \right) - 1 } \\
                         &= e^{\ln y} \\
                         &= y. \\
\end{aligned}
\]
\end{multicols}

\item  In this case, $g \circ f = I_{\mathbb{R}}$ and 
$f \circ g = I_{\mathbb{R}^+}$.
\end{enumerate}



\item \begin{enumerate}
\item $f:\mathbb{R} \to \mathbb{R}$ is defined by $f\left( x \right) = e^{ - x^2 } $.  Since this function is not an injection, the inverse of $f$ is not a function.

\item $g:\mathbb{R}^*  \to \left( {0, 1} \right]$ is defined by $g\left( x \right) = e^{ - x^2 }$.  In this case, $g$ is a bijection and hence, the inverse of $g$ is a function.

To see that $g$ is an injection, assume that $x, y \in \mathbb{R}^*$ and that 
$e^{-x^2} = e^{-y^2}$.  Then, $x^2 = y^2$ and since $x, y \geq 0$, we see that $x = y$.  To see that $g$ is a surjection, let $y \in \left( 0, 1 \right]$.  Then, $\ln y < 0$ and $- \ln y > 0$, and $g \left( \sqrt{-\ln y} \right) = y$.
\end{enumerate}



\item \begin{enumerate}
\item If $f:\mathbb{R} \to \mathbb{R}$ is defined by  $f\left( x \right) = x^2 $, then the inverse of $f$ is not a function since $f$ is not a bijection.

\item The standard choice is to let 
$\mathbb{R}^*  = \left\{ { {x \in \mathbb{R} } \mid x \geq 0} \right\}$ and to define 
$s: \mathbb{R}^* \to \mathbb{R}^*$ by $s \left( x \right) = x^2$.

\item The inverse of $s$ is a function and for $y \in \mathbb{R}^*$, 
$s^{-1} \left( y \right) = x$ if and only if $x^2 = y$.  Hence, $x$ is the non-negative square root of $y$ and so, $s^{-1} \left( y \right) = \sqrt y$.

\item True.  
\item False.
\end{enumerate}



\item If  $f:A \to B$ is a bijection, then  $f^{ - 1} :B \to A$ is also a bijection.

\textbf{\emph{Proof}.}  If $b_1, b_2 \in B$ and 
$f^{-1} \left( b_1 \right) = f^{-1} \left( b_2 \right)$, then
\[
\begin{aligned}
f \left( f^{-1} \left( b_1 \right) \right) &= f \left( f^{-1} \left( b_1 \right) \right) \\
                                       b_1 &= b_2, \\
\end{aligned}
\]
and hence, $f^{-1}$ is an injection.  Now, let $a \in A$.  Then, $f \left( a \right) = b$ and hence, $f^{-1} \left( f \left( a \right) \right) = a$.  Hence, $f^{-1}$ is a surjection.

\item For each natural number  $k$, let  $A_k $ be a set, and for each natural number  $n$, let  $f_n :A_n  \to A_{n + 1} $.  For each natural number  $n$  with  $n \geq 2$, if  $f_1 , f_2 ,  \ldots , f_n $ are all bijections, then  
$f_n  \circ f_{n - 1}  \circ  \cdots  \circ f_2  \circ f_1 $ is a bijection and
\[
\left( {f_n  \circ f_{n - 1}  \circ  \cdots  \circ f_2  \circ f_1 } \right)^{ - 1}  = f_1^{ - 1}  \circ f_2^{ - 1}  \circ  \cdots  \circ f_{n - 1}^{ - 1}  \circ f_n^{ - 1}.
\]

\textbf{\emph{Proof}.} (By mathematical induction)  Using the given notation, let 
$P \left( n \right)$ be if  $f_1 , f_2 ,  \ldots , f_n $ are all bijections, then  
$f_n  \circ f_{n - 1}  \circ  \cdots  \circ f_2  \circ f_1 $ is a bijection and
\[
\left( {f_n  \circ f_{n - 1}  \circ  \cdots  \circ f_2  \circ f_1 } \right)^{ - 1}  = f_1^{ - 1}  \circ f_2^{ - 1}  \circ  \cdots  \circ f_{n - 1}^{ - 1}  \circ f_n^{ - 1}.
\]
The basis step $\left( n = 2 \right)$ is Theorem~\ref{compositionofbijections}.  Now assume that $k \in \mathbb{N}$, $k \geq 2$, and $P \left( k \right)$ is true.  To prove that 
$P \left( k + 1 \right)$ is true, we assume that $f_1 , f_2 ,  \ldots , f_k, f_{k+1}$ are all bijections.  Since $P \left( k \right)$ is true, we conclude that 
$f_k  \circ f_{k - 1}  \circ  \cdots  \circ f_2  \circ f_1$ is a bijection.  Now, 
\[
f_{k+1}  \circ f_{k}  \circ  \cdots  \circ f_2  \circ f_1 = 
f_{k+1} \circ \left[ f_k  \circ f_{k - 1}  \circ  \cdots  \circ f_2  \circ f_1 \right],
\]
and so by Theorem~\ref{compositionofbijections} is a bijection.  In addition,
\[
\begin{aligned}
\left( {f_{k+1}  \circ f_{k}  \circ  \cdots  \circ f_2  \circ f_1 } \right)^{ - 1}  &= 
\left[ f_{k+1} \circ \left( f_k  \circ f_{k - 1}  \circ  \cdots  \circ f_2  \circ f_1 \right) \right]^{-1} \\
  &= \left( f_k  \circ f_{k - 1}  \circ  \cdots  \circ f_2  \circ f_1 \right)^{-1} \circ f_{k+1}^{-1} \\
  &= \left( f_1^{ - 1}  \circ f_2^{ - 1}  \circ  \cdots  \circ f_{k - 1}^{ - 1}  \circ f_k^{ - 1} \right) \circ f_{k+1}^{-1} \\
  &=f_1^{ - 1}  \circ f_2^{ - 1}  \circ  \cdots  \circ f_{k - 1}^{ - 1}  \circ f_k^{ - 1} \circ f_{k+1}^{-1}. \\
\end{aligned}
\]
This proves that if $P \left( k \right)$ is true, then $P \left( k + 1 \right)$ is true.



\item \begin{enumerate}
\item Since the function $f$ is not an injection and is not a surjection, the inverse of 
$f$ is not a function.

\item We first prove that $F$ is an injection.  Let $a, b \in \R^*$ and assume that 
$F(a) = F(b)$.  Then $a^2 - 4 = b^2 - 4$, which implies that $a^2 = b^2$.  Since both $a$ and $b$ are nonnegative, we can take the square root of both sides of this equation to prove that $a = b$.  Therefore, $F$ is an injection.

Now let $y \in T$.  Then $y \in \R$ and $y \geq -4$.  We can then conclude that 
$y + 4 \geq 0$ and hence, $\sqrt{y + 4} \in \R^*$.  So if $x = \sqrt{y + 4}$, then
\begin{align*}
F(x) &= x^2 - 4 \\
     &= (y + 4) - 4 \\
     &= y.
\end{align*}
This shows that $F$ is a surjection and that for each $y \in T$, $F^{-1}(y) = \sqrt{y + 4}$.
\end{enumerate}



\item \begin{enumerate}
\item $f^{-1} = \{ (4, 0), (0, 1), (3, 2), (3, 3), (0, 4) \}$.  $f^{-1}$ is not a function since $f^{-1}$ contains ordered pairs that have the same first coordinate and different second coordinates.

\item $g^{-1} = \{ (4, 0), (0, 1), (2, 2), (1, 3), (3, 4) \}$.  $g^{-1}$ is a function and 
$g^{-1}(0) = 1$, $g^{-1}(1) = 3$, $g^{-1}(2) = 2$, $g^{-1}(3) = 4$, and $g^{-1}(4) = 0$.

\item \begin{multicols}{2}
$\sqrt[3]{0} = 0$ since $0^3 = 0$.

$\sqrt[3]{1} = 1$ since $1^3 = 1$.

$\sqrt[3]{2} = 3$ since $3^3 = 2$.

$\sqrt[3]{3} = 2$ since $2^3 = 3$.

$\sqrt[3]{4} = 4$ since $4^3 = 4$.
\end{multicols}

\item Since it possible to define a cube root for each element of $\Z_5$, we can write 
$g^{-1}(y) = \sqrt[3]{y - 4} \pmod 5 = \sqrt[3]{y + 1} \pmod 5$.
\end{enumerate}
\end{enumerate}
\hbreak

\endinput
