\section*{Section~\ref{S:prop} Conditional Statements}
\noindent

\begin{enumerate}[\exer{exer:sec11-1}]
\item Sentences (a) , (c), (e), (f), (j), and  (k) are statements.  Sentence (h) is a statement if we are assuming that $n$ is a prime number means that  $n$ is a natural number.
\end{enumerate}

%\exer{exer:sec11-1}{Sentences (a) , (d), and  (h) are propositions.  Sentence (f) is a proposition if we are assuming that n is a prime number means that  $n$ is an integer.}


\begin{enumerate}[\exer{exer:sec11-2}] \item
%\begin{list}{\bf{\ref{exer:sec11-2}}.} \item
\begin{tabular}[t]{| c | p{2.0in} | p{2.0in} |} \hline
  &  Hypothesis  &  Conclusion \\ \hline
\bf{(a)}  &  $n$ is a prime number  &  $n^2$ has three positive divisors \\ \hline
\bf{(b)}  &  $a$ is an irrational number and $b$ is an irrational number  &  $a \cdot b$ is an irrational number \\ \hline
\bf{(c)}  &  $p$ is a prime number  &  $p = 2$ or $p$ is an odd number  \\ \hline
\bf{(d)}  &  $p$ is a prime number and $p \ne 2$  &  $p$ is an odd number \\ \hline
\bf{(e)}  &  $p \ne 2$ and $p$ is even  &  $p$ is not prime \\  \hline
\end{tabular}
\end{enumerate}

\begin{enumerate}[\exer{exer:sec11-3}]  
\item Statements (a), (c), and (d) are true.
\end{enumerate}

\begin{enumerate}[\exer{exer:sec11-4}]
\item (a) True when $a \ne 3$.  (b) True when $a= 3$.
\end{enumerate}


\begin{enumerate}[\exer{exer:sec11-5}]
\item In Part (c), the instructor would have lied.  Part (b) corresponds to the first row of the truth table for $P \to Q$ , Part (c) corresponds to the second row, and Part (d) corresponds to the last two rows of this truth table.
\end{enumerate}


\begin{enumerate}[\exer{exer:sec11-6}]
\item \begin{enumerate} \usecounter{enumii}
\item The function $g$ has a maximum value when $x = \dfrac{5}{16}$.

\item The function $h$ has a maximum value when $x = \dfrac{9}{2}$.

\item No conclusion can be made about the function $k$ from this theorem.

\item The function $j$ has a maximum value when $x = 0$.

\item The function $f$ has a maximum value when $x = \dfrac{-3}{8}$.

\item No conclusion can be made about the function $F$ from this theorem.
\end{enumerate}
\end{enumerate}





\begin{enumerate}[\exer{exer:sec11-7}]
\item \begin{enumerate} \usecounter{enumii} 
\item No conclusion can be made about the function $g$ from this theorem.

\item No conclusion can be made about the function $h$ from this theorem.

\item The function $k$ has two $x$-intercepts.

\item The function $j$ has two $x$-intercepts.

\item The function $f$ has two $x$-intercepts.

\item No conclusion can be made about the function $F$ from this theorem.
\end{enumerate}
\end{enumerate}



\begin{enumerate}[\exer{exer:sec1-1-8}]
\item \begin{enumerate} \usecounter{enumii}
\item The function $f$ has exactly one $x$-intercept.

\item No conclusion can be made about the function $g$ from these theorems.

\item By writing $h(x) = (-1)\left(x^3 - x + 5 \right)$, we can use Theorem A and Theorem B to conclude that the function $h$ has exactly one $x$-intercept.

\item No conclusion can be made about the function $k$ from these theorems.

\item No conclusion can be made about the function $r$ from these theorems.

\item By writing $F(x) = 2 \left(x^3 - x + \dfrac{7}{2} \right)$, we can use Theorem A and Theorem B to conclude that the function $F$ has exactly one $x$-intercept.
\end{enumerate}
\end{enumerate}




\begin{enumerate}[\exer{exer:sec11-8}]
\item \begin{enumerate} \usecounter{enumii}
\item The set of natural numbers is not closed under division.

\item The set of rational numbers is not closed under division since we cannot divide by zero.

\item The set of non-zero rational numbers is closed under division.

\item The set of positive rational numbers is closed under division.

\item The set of positive real numbers is not closed under subtraction.

\item The set of negative rational numbers is not closed under division.

\item The set of negative integers is closed under addition.
\end{enumerate}
\end{enumerate}




\hbreak
\endinput
