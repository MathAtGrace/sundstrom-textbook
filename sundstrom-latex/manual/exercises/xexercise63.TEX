\section*{Section \ref{S:typesoffunctions} Types of Functions}

\begin{enumerate}
\item \begin{enumerate}
\item One possibility is to have $A = \{a, b \}$,$B = \{ 1, 2, 3 \}$ and draw an arrow diagram for $f\x A \to B$ with $f(a) = 1$ and $f(b) = 2$.

\item One possibility is to have $A = \{a, b \}$,$B = \{ 1, 2 \}$ and draw an arrow diagram for $f\x A \to B$ with $f(a) = 1$ and $f(b) = 2$.

\item One possibility is to have $A = \{a, b \}$,$B = \{ 1, 2, 3 \}$ and draw an arrow diagram for $f\x A \to B$ with $f(a) = 1$ and $f(b) = 1$.

\item One possibility is to have $A = \{a, b, c \}$,$B = \{ 1, 2 \}$ and draw an arrow diagram for $f\x A \to B$ with $f(a) = 1$, $f(b) = 2$, and $f(c) = 1$.

\item The arrow diagrams in (a), (c), or (d) can be used.
\end{enumerate}


\item \begin{enumerate}
\item The function $f$ is not an injection and is not a surjection.

\item The function $g$ is not an injection and is not a surjection.

\item The function $F$ is an injection and is a surjection.
\end{enumerate}


\item \begin{enumerate}
\item The function $f$ is an injection.  To prove this, let $x_1, x_2 \in \mathbb{Z}$ and assume that $f ( x_1 ) = f ( x_2 )$.  hen,
\[
\begin{aligned}
3x_1 + 1 &= 3x_2 + 1 \\
    3x_1 &= 3x_2 \\
     x_1 &= x_2. \\
\end{aligned}
\]
Hence, $f$ is an injection.  Now, for each $x \in \mathbb{Z}$, $3x + 1 \equiv 0 \pmod 3$, and hence 
$f ( x ) \equiv 0 \pmod 3$.  This means that there is no integer $x$ such that 
$f ( x ) = 0$.  Therefore, $f$ is not a surjection.

\item The proof that $F$ is an injection is similar to the proof in Part~(a) that $f$ is an injection.  To prove that $F$ is a surjection, let $y \in \mathbb{Q}$.  Then, 
$\dfrac{y-1}{3} \in \mathbb{Q}$ and $F \left( \dfrac{y-1}{3} \right) = y$ and hence, $F$ is a surjection.
\end{enumerate}


\item \begin{enumerate}
\item The function $g$ is a bijection.  The essential idea is that if $a^3 = b^3$, then 
$\sqrt[3]{a^3} = \sqrt[3]{b^3}$ and hence $a = b$.  In addition, if $y \in \mathbb{R}$, then 
$\sqrt[3]{y} \in \mathbb{R}$ and $g ( \sqrt[3]{y} ) = y$.  Hence, $g$ is a surjection.

\item The function $f$ is an injection.  The proof is similar to the proof in Part~(a).  However, the function $f$ is not a surjection.  For example, $\sqrt[3]{2} \notin \mathbb{Q}$ and hence, there is no $x \in \mathbb{Q}$ such that $f ( x ) = 2$.
\end{enumerate}



\item \begin{enumerate}
\item Let $F: \mathbb{R} \to \mathbb{R}$ be defined by $F ( x ) = 5x + 3$ for all 
$x \in \mathbb{R}$.  Let  $x_1, x_2 \in \mathbb{R}$ and assume that 
$F ( x_1 ) = F ( x_2 )$.  Then,
\[
\begin{aligned}
5x_1 + 3 &= 5x_2 + 3 \\
    5x_1 &= 5x_2 \\
     x_1 &= x_2. \\
\end{aligned}
\]
Hence, $F$ is an injection.  Now let $y \in \mathbb{R}$.  Then, $\dfrac{y - 3}{5} \in \mathbb{R}$ and 
\[
\begin{aligned}
F \left( \frac{y - 3}{5} \right) &= 5 \left( \frac{y - 3}{5} \right) + 3 \\
                         &= ( y - 3 ) + 3 \\
                         &= y. \\
\end{aligned}
\]
Thus, $F$ is a surjection and hence $F$ is a bijection.


\item The proof that $G$ is an injection is similar to the proof in Part~(a) that $F$ is an injection.  Now, for each $x \in \mathbb{Z}$, $5x + 3 \equiv 5 \pmod 3$, and hence 
$G ( x ) \equiv 3 \pmod 5$.  This means that there is no integer $x$ such that 
$G ( x ) = 0$.  Therefore, $G$ is not a surjection.

\item Let  $a, b \in \mathbb{R} - \{4 \}$ and assume that 
$f( a ) = f( a )$.  Then,
\begin{align*}
\frac{3a}{a - 4} &= \frac{3b}{b - 4} \\
       3a(b - 4) &= 3b(a - 4) \\
       3ab - 12a &= 3ab - 12b \\
            -12a &= -12b \\
               a &= b.
\end{align*}
So $f$ is an injection.

Use a proof by contradiction to show there is no $a \in \R - \{4 \}$ such that $f(a) = 3$.  Assume such an $a$ exists.  Then
\begin{align*}
\frac{3a}{a - 4} &= 3 \\
3a &= 3a - 4 \\
 0 &= - 4,
\end{align*}
and this is a contradiction.  Therefore, for all $x \in \R - \{4\}$, $f(x) \ne 3$ and $f$ is not a surjection.

\item The function $g$ is a bijection.  The proof that is an injection is similar to the proof that $f$ is an injection in Part~(c).  To prove that it is a surjection let 
$y \in \R - \{3 \}$.  Then, $\dfrac{4y}{y - 3} \in \R - \{4\}$ and
\begin{align*}
g \left(\frac{4y}{y - 3}\right) &= \frac{3 \left(\dfrac{4y}{y - 3}\right)}{\left(\dfrac{4y}{y - 3}\right) - 4} \\
   &= \frac{12y}{4y - 4(y - 3)} \\
   &= \frac{12y}{12} \\
   &= y.
\end{align*}
This proves that $g$ is a surjection.
\end{enumerate}



\item The function $f$ is not an injection.  For example,
\[
f(0) = -\frac{1}{4} \qquad \text{and} \qquad f (8) = -\frac{1}{4}.
\]
The function $f$ is not a surjection since $f(x) \ne 1$ for all $x \in \R$.  To prove this, assume there exists an $x \in \R$ such that $f(x) = 1$.  Then
\begin{align*}
2x - 1 &= x^2 + 4 \\
x^2 -2x + 5 &= 0.
\end{align*}
The quadratic formula can then be used to show that $x \notin \R$ and this is a contradiction.


\item \begin{enumerate}
\item The function $f$ is not an injection since $f(1) = f(-1)$.  Now we also know that 
$e^{x^2} \geq 1$ and so
\[
e^{-x^2} = \frac{1}{e^{x^2}} \leq 1.
\]
This can be used to explain why $f$ is not a surjection.

\item The function $g$ is not a surjection for the same reason that the function $f$ in Part~(a) is not a surjection.  To prove that $g$ is an injection, let $a, b \in \R^*$ and assume that 
$g(a) = g(b)$.  Then
\[
e^{-a^2} = e^{-b^2},
\]
and this implies that $-a^2 = -b^2$ or that $a^2 = b^2$.  We can now use the facts that 
$a \geq 0$ and $b \geq 0$ to conclude that $a = b$.  Hence, $g$ is an injection.

\item The proof that $h$ is an injection is similar to the proof that the function $g$ in 
Part~(b) is an injection.  To prove that $h$ is a surjection, let $y \in T$.  This means that 
$y \in \R$ and $0 < y \leq 1$.  This means that $\ln y < 0$ and hence, $-\ln y > 0$.  So, if 
$x = \sqrt{-\ln y}$, then $x \in \R*$ and $x^2 = -ln y$.  Therefore,
\begin{align*}
h(x) &=  e^{-(-\ln y)} \\
     &= e^{\ln y} \\
     &= y.
\end{align*}
This proves that $h$ is a surjection.
\end{enumerate}


\item The function $f$ is an injection but is not a surjection.  To prove that it is an injection, let $a, b \in \R$ and assume that $f(a) = f(b)$.  There are four cases.
\begin{itemize}
\item $a \geq 0$ and $b < 0$.  This is not possible since in this case, $f(a) \geq 0$ and 
$f(b) < -1$.
\item $a < 0$ and $b \geq 0$.  This is not possible since in this case, $f(a) < -1$ and 
$f(b) \geq 0$.
\item $a \geq 0$ and $b \geq 0$.  In this case, $a^2 = b^2$ and since both $a$ and $b$ are nonnegative, $a = b$.
\item $a < 0$ and $b < 0$.  In this case, $a - 1 = b - 1$ and hence, $a = b$.
\end{itemize}
Therefore, $f$ is an injection.  To prove that $f$ is not a surjection, we note that if 
$x \geq 0$, then $f(x) \geq 0$ and if $x < 0$, then $f(x) < -1$.  Therefore, for all 
$x \in \R$, $f(x) \ne -0.5$.
%\begin{figure}[h]
%\begin{center}
%\scalebox{0.5}{\includegraphics{xfig-exer8-63.eps}}
%\end{center}
%\end{figure}




\item \begin{enumerate}
\item Notice that if $x \in [0, 1]$ and $0 < x < 1$, then $0 < g(x) < 0.5$.
Let $a, b \in [0,1]$ and assume that $g(a) = g(b)$.  Consider three cases.
\begin{itemize}
\item If $a = 0$, then $g(a) = 0.8$.  If $b \ne 0$, then $g(b) < 0.8$, which is not possible since $g(b) = g(a) =0.8$.  Therefore, $b = 0$ and hence, $a = b$.

\item If $a = 1$, then $g(a) = 0.6$.  If $b \ne 1$, then $g(b) \ne 0.6$, which is not possible since $g(b) = g(a) =0.6$.  Therefore, $b = 1$ and hence, $a = b$.

\item If $0 < a < 1$, then $0 < g(a) < 0.5$.  Since $g(0) = 0.8$ and $g(1) = 0.6$, in order for $g(b)$ to equal $g(a)$, we must have $0 < b < 1$.  Hence, $0.5a = 0.5b$ and hence, $a = b$.
\end{itemize}
Therefore, $g$ is an injection. The function $g$ is not an injection since for all 
$x \in [0, 1]$, $g(x) \ne 0.9$.

\item The function $h$ is not an injection since $h(0) = h(2)$.  The function $h$ is a surjection since $h(0) = 0$ and $h(1) = 1$.
\end{enumerate}



\item The sum of the divisors function $s$ is not an injection.  For example, 
$s ( 6 ) = s ( 11 )$.  This function is also not a surjection.  For example, for all 
$x \in \mathbb{N}$, $s ( x ) \ne 2$ and for all  
$x \in \mathbb{N}$, $s ( x ) \ne 5$.



\item The number of divisors function $d$ is not an injection since for every prime number $p$, $d ( p ) = 2$.  This function is, however, a surjection.  To see this, use 
Exercise~(\ref{exer:numberofdivisors}) from Section~\ref{S:introfunctions}.  In this exercise, it was proven that $d ( 2^n ) = n + 1$ for each integer $n$ with $n \geq 0$.  Hence, $d ( 1 ) = 1$ and if $k \in \mathbb{N}$, then $2^{k-1} \in \mathbb{N}$ and 
$d ( 2^{k-1} ) = k$.



\item The birthday function is not an injection since there exists different people with the same birthday.  The birthday function is a surjection since there has been a person born on each day of the year.

\item \begin{enumerate}
\item Let $f:\mathbb{Z} \times \mathbb{Z} \to \mathbb{Z}$ be defined by  
$f( {m, n} ) = 2m + n$.  The function $f$ is not an injection.  For example, 
$f ( 1, 1 ) = 3$ and $f ( 0, 3 ) = 3$.  The function $f$ is a surjection.  To see this, let $n \in \mathbb{Z}$.  Then, 
$( 0, n ) \in \mathbb{Z} \times \mathbb{Z}$ and 
$f ( 0, n ) = n$.

\item Let $g:\mathbb{Z} \times \mathbb{Z} \to \mathbb{Z}$ be defined by  
$f( {m, n} ) = 6m + 3n$.  The function $g$ is not an injection.  For example, 
$g ( 1, 1 ) = 9$ and $g ( 0, 3 ) = 9$.  For each 
$( m, n ) \in \mathbb{Z} \times \mathbb{Z}$, $g ( m, n )$ is a multiple of 3.  Therefore, $g$ is not a surjection.
\end{enumerate}



\item \begin{enumerate}
\item Let 
$f:\mathbb{R} \times \mathbb{R} \to \mathbb{R} \times \mathbb{R}$ be defined by  
$f( {x, y} ) = ( {2x, x + y} )$. To prove that $f$  is an injection, we assume that  
$( {a, b} ) \in \mathbb{R} \times \mathbb{R}$, 
$( {c, d} ) \in \mathbb{R} \times \mathbb{R}$, and that  
$f( {a, b} ) = f( {c, d} )$.  This means that
\[
( {2a, a + b} ) = ( {2c, c + d} ).
\]
Since this equation is an equality of ordered pairs, we see that
\[
\begin{aligned}
       2a &= 2c \text{, and} \\ 
    a + b &= c + d. \\ 
\end{aligned}
\]
The first equation implies that $a = c$.  Substituting this into the second equation shows that 
$b = d$.  Hence, 
\[
( {a, b} ) = ( {c, d} ),
\]
and we have shown that if $f( {a, b} ) = f( {c, d} )$, then  
$( {a, b} ) = ( {c, d} )$.  Therefore,  $f$  is an injection.

Now, to determine if  $f$  is a surjection, we let  
$( {r, s} ) \in \mathbb{R} \times \mathbb{R}$. To find an ordered pair 
$( {a, b} ) \in \mathbb{R} \times \mathbb{R}$ such that  
$f( {a, b} ) = ( {r, s} )$, we need
\[
( {2a, a + b} ) = ( {r, s} ).
\]
That is, we need
\[
\begin{aligned}
      2a &= r\text{, and} \\ 
   a + b &= s. \\ 
\end{aligned}
\]
Solving this system for  $a$  and  $b$  yields  
\[
a = \frac{{r}}{2} \text{ and } b = \frac{{2s - r}}{2}.
\]
Since  $r, s \in \mathbb{R}$, we can conclude that  $a \in \mathbb{R}$ and 
$b \in \mathbb{R}$ and hence that  
\mbox{$( {a, b} ) \in \mathbb{R} \times \mathbb{R}$}.  So,
\[
\begin{aligned}
  f( {a, b} ) &= f( {\frac{{r}}{2}, \frac{{2s - r}}{2}} ) \\ 
                         &= ( {2( {\frac{{r}}{2}} ),                             \frac{{r}}{2} + \frac{{2s - r}}{2}} ) \\ 
                         &= ( {r, s} ). \\ 
\end{aligned} 
\]
This proves that for all  $( {r, s} ) \in \mathbb{R} \times \mathbb{R}$, there exists  $( {a, b} ) \in \mathbb{R} \times \mathbb{R}$ such that  
$f( {a, b} ) = ( {r, s} )$.  Hence, the function  $f$  is a surjection.  Since  $f$  is both an injection and a surjection,  it is a bijection.

\item Let $g:\mathbb{Z} \times \mathbb{Z} \to \mathbb{Z} \times \mathbb{Z}$ be defined by  
$g( {x, y} ) = ( {2x, x + y} )$.  The proof that $g$ is an injection is almost identical to the proof that $f$ is an injection in Part~(a).  Now, for each 
$( x, y ) \in \mathbb{Z} \times \mathbb{Z}$, the first coordinate of 
$g ( x, y )$ is an even integer.  Therefore, $g$ is not a surjection.
\end{enumerate}


\item The function $f$ is not an injection.  For example, $f(0, 0) = 0$ and 
$f(1, 0) = 0$. The function $f$ is a surjection.  To prove this, let $z \in \R$.  Then 
$\dfrac{z}{3} \in \R$ and $f \left(0, \dfrac{z}{3} \right) = z$.


\item The function $g$ is not an injection.  For example, $g(0, 0) = o$ and 
$g \left( 0, \pi \right) = 0$.  The function $g$ is a surjection.  To prove this, let 
$z \in \R$.  Then $\sqrt[3]{z - 2} \in \R$ and 
$g \left( \sqrt[3]{z - 2}, \dfrac{\pi}{2} \right) = z$.


\item Let $A$ be a nonempty set.  The identity function $I_A$ on the set $A$ is a bijection.

\item Let  $A$  and  $B$  be two non-empty sets.  Define  
$p_1 :A \times B \to A \text{ by }  \\
p_1 ( {a, b} ) = a$ for every  
$( {a, b} ) \in A \times B$.

\begin{enumerate}
\item Let $a \in A$.  Since $B \ne \emptyset$, there exists an element $b$ in $B$.  Therefore, 
$p_1 ( a, b ) = a$, and hence, $p_1$ is a surjection.

\item If $B = \left\{ b \right\}$, then the second coordinate of every element of $A \times B$ is $b$.  So, if $( a_1, b )$ and $( a_2, b )$ are in $A \times B$ and 
$p_1 ( {a_1, b} ) = p_1 ( {a_2, b} )$, then $a_1 = a_2$, and hence,
$( a_1, b ) = ( a_2, b )$.  Therefore, $p_1$ is an injection.

\item If the set $B$ contains more than one element, then $p_1$ is not an injection.  To see this, assume $b, c \in B$ with $b \ne c$.  If $a \in A$, then 
$( a, b ) \ne ( a, c )$ and 
$p_1 ( a, b ) = p_1 ( a, c )$.
\end{enumerate}

\item \begin{enumerate}
\item The determinant function is not an injection.  For example,
\[
\det \left[ {\begin{array}{*{20}c}
   1 & 0  \\
   0 & 1  \\
\end{array} } \right] =  \det \left[ {\begin{array}{*{20}c}
   1 & 2  \\
   0 & 1  \\
 \end{array} } \right].
\]
The determinant function is a surjection.  To prove this, let $a \in \mathbb{R}$.  Then
\[
\det \left[ {\begin{array}{*{20}c}
   a & 0  \\
   0 & 1  \\
\end{array} } \right] =  a.
\]

\item The transpose function is a bijection.  To prove it is an injection, let 
$\left[ {\begin{array}{*{20}c}
   a & b  \\
   c & d  \\
\end{array} } \right], \left[ {\begin{array}{*{20}c}
   p & q  \\
   r & s  \\
\end{array} } \right] \in \mathcal{M}_{2, 2}$ and assume that
\[
\text{tran}\left[ {\begin{array}{*{20}c}
   a & b  \\
   c & d  \\
 \end{array} } \right] = \text{tran}\left[ {\begin{array}{*{20}c}
   p & q  \\
   r & s  \\
 \end{array} } \right].
\]
Then, $\left[ {\begin{array}{*{20}c}
   a & c  \\
   b & d  \\
\end{array} } \right] = \left[ {\begin{array}{*{20}c}
   p & r  \\
   q & s  \\
\end{array} } \right]$.  Therefore, $a = p$, $b = q$, $c = r$, and $d = s$ and hence, 
$\left[ {\begin{array}{*{20}c}
   a & b  \\
   c & d  \\
\end{array} } \right] = \left[ {\begin{array}{*{20}c}
   p & q  \\
   r & s  \\
\end{array} } \right]$.  To prove that the transpose function is a surjection, let 
$\left[ {\begin{array}{*{20}c}
   a & b  \\
   c & d  \\
\end{array} } \right] \in \mathcal{M}_{2, 2}$.  Then,
\[
\text{tran}\left[ {\begin{array}{*{20}c}
   a & c  \\
   b & d  \\
 \end{array} } \right] = \left[ {\begin{array}{*{20}c}
   a & b  \\
   c & d  \\
 \end{array} } \right].
\]

\item The function $F$ is not an injection.  For example
\[
F \left[ {\begin{array}{*{20}c}
   0 & 0  \\
   0 & 0  \\
\end{array} } \right] =  0 \qquad \text{and} \qquad
F \left[ {\begin{array}{*{20}c}
   1 & 1  \\
   1 & 1  \\
\end{array} } \right] =  0.
\]
The function $F$ is a surjection.  To prove this, let $y \in \R$.  Consider three cases.
\begin{itemize}
\item If $y = 0$, then $F \left[ {\begin{array}{*{20}c}
   0 & 0  \\
   0 & 0  \\
\end{array} } \right] =  0 = y$.

\item If $y > 0$, then $\sqrt{y} \in \R$ and $F \left[ {\begin{array}{*{20}c}
   \sqrt{y} & 0  \\
   0 & 0  \\
\end{array} } \right] =  \left( \sqrt{y} \right)^2 = y$.

\item If $y < 0$, then $\sqrt{-y} \in \R$ and $F \left[ {\begin{array}{*{20}c}
   0 & \sqrt{-y}  \\
   0 & 0  \\
\end{array} } \right] =  -\left( \sqrt{-y} \right)^2 = y$.

\end{itemize}

\end{enumerate}



\item The function $A$ is not an injection.  For example, if $f$ and $g$ are real functions defined by $f ( x ) = x$ and $g ( x ) = \dfrac{1}{2}$ for each 
$x \in \mathbb{R}$, then $f, g \in C$ and
\[
\begin{aligned}
A ( f ) &= \int_0^1 {f( x ) \, dx} = \frac{1}{2}, \text{and} \\
A ( g ) &= \int_0^1 {g( x ) \, dx} = \frac{1}{2}. \\
\end{aligned}
\]
The function $A$ is a surjection.  To prove this, let $y \in \mathbb{R}$.  Define the real function $k$ by $k ( x ) = y$ for each $x \in \mathbb{R}$.  Then, $k \in C$ and 
\[
A ( k ) = \int_0^1 {k( x ) \, dx} = y.
\]


\item Let $A = \left\{ ( m, n ) \mid m \in \mathbb{Z}, n \in \mathbb{Z}, \text{ and } n \ne 0 \right\}$.  Define $f\x A \to \mathbb{Q}$ as follows:

\begin{list}{}
\item For each $( m, n ) \in A$, $f ( m, n ) = \dfrac{m+n}{n}$.
\end{list}
\begin{enumerate}
\item The function $f$ is not an injection.  For example, $f(0, 1) = 0$ and $f(0, 2) = 0$.  

\item The function $f$ is a surjection.  To prove this, let $y \in \Q$.  Then, there exist integers $a$ and $b$ with $b \ne 0$ such that $y = \frac{a}{b}$.  Then, $a - b \in \Z$ and
\[
f(a - b, b) = \frac{(a - b) + b}{b} = \frac{a}{b} = y.
\]
\end{enumerate}



\setcounter{equation}{0}
\item \begin{enumerate}
\item This proof is not set up correctly, and there are several problems with the way the proof is written.   A minor point is that the proof does not contain a definition of the function $f$.  This should be included so that the proof is self-contained.  A standard procedure to prove that the function $f$ is an injection is to assume that $f(a, b) = f(c, d)$ and then prove that $(a, b) = (c, d)$.  This proof starts with an assumption that $(a, b) = (c, d)$.  In addition, a little more explanation is needed for the work shown in the second display of the proof, and the notation is incorrect in the equation in the last display.  Following is a well-written proof of this proposition.


\quarter
\begin{myproof}
We let $f$ be the function $f\x\R \times \R \to \R \times \R$ defined by  
$f( {x, y} ) = ( {2x + y, x - y} )$.  To prove that $f$ is an injection , we assume 
$(a, b)$ and $(c, d)$ are in $\R \times \R$ and $f(a, b) = f(c, d)$.  We will prove that 
$(a, b) = (c, d)$.  From the assumption that $f(a, b) = (c, d)$, we obtain
\[
(2a + b, a - b) = (2c + d, c - d).
\]
We will use systems of equations to prove that $a = c$ and $b = d$.  By equating the corresponding coordinates in the ordered pairs in the previous equation, we see that
\begin{align}
2a + b &= 2c + d \\
 a - b &=  c - d
\end{align}
We now add the corresponding sides of these two equations to obtain $3a = 3c$ and, hence, $a = c$.
  Since $a = c$, we can use equation~(2) to obtain $a - b = a - d$, which implies that $b = d$. Hence, we may conclude that $(a, b) = (c, d)$, and this proves that the function $f$ is an injection.
\end{myproof}

\item This is a good proof of the proposition, but the start of the proof should be rewritten.  In particular, the function $f$ should be defined in the body of the proof and it should be stated that we will prove that $f$ is a surjection.  Following is a revised version of this proof.

\begin{myproof}
We let $f$ be the function $f\x\R \times \R \to \R \times \R$ defined by  
$f( {x, y} ) = ( {2x + y, x - y} )$.  To prove that $f$ is an surjection, we let 
$(a, b) \in \R \times \R$ and will prove that there exists an $(x, y) \in \R \times \R$ such that $f(x, y) = (a, b)$.  So, we need to find an ordered pair such that $f(x, y) = (a, b)$. That is, we need $(2x + y, x - y) = (a, b)$, or
\[
2x + y = a \qquad \text{and} \qquad x - y = b.
\]
Treating these two equations as a system of equations and solving for $x$ and $y$, we find that
\[
x = \frac{a + b}{3} \qquad \text{and} \qquad y = \frac{a - 2b}{3}.
\]
Hence, $x$ and $y$ are real numbers, $(x, y) \in \R \times \R$, and
\begin{align*}
f(x, y) &= f \left( \frac{a + b}{3}, \frac{a - 2b}{3} \right) \\
        &= \left(2 \left( \frac{a + b}{3} \right) + \frac{a - 2b}{3}, \frac{a + b}{3} - \frac{a - 2b}{3} \right) \\
        &= \left( \frac{2a + 2b + a - 2b}{3}, \frac{a + b - a + 2b}{3} \right) \\
        &= \left( \frac{3a}{3}, \frac{3b}{3} \right) \\
        &= (a, b).
\end{align*}
Therefore, we have proved that for every $(a, b) \in \R \times \R$, there exists an 
$(x, y) \in \R \times \R$ such that $f(x, y) = (a, b)$.  This proves that the function $f$ is a surjection.
\end{myproof}
\end{enumerate}
\end{enumerate}
\hbreak
\endinput
