\section*{Section \ref{S:functionsonsets} Functions Acting on Sets}

\begin{enumerate}
\item \begin{enumerate}
\item There exists an $x \in A \cap B$ such that $f \left( x \right) = y$.

\item There exists an $x \in A \cup B$ such that $f \left( x \right) = y$.

\item There exists an $a \in A$ such that $f \left( a \right) = y$ and there exists a 
$b \in B$ such that $f \left( b \right) = y$.

\item There exists an $a \in A$ such that $f \left( a \right) = y$ or there exists a 
$b \in B$ such that $f \left( b \right) = y$.

\item $f \left( x \right) \in C \cap D$

\item $f \left( x \right) \in C \cup D$

\item $f \left( x \right) \in C$ and $f \left( x \right) \in D$

\item $f \left( x \right) \in C$ or $f \left( x \right) \in D$
\end{enumerate}

\item \begin{multicols}{2}
\begin{enumerate}
\item $f \left( A \right) = \left[ -9, -3 \right]$.

\item $f^{-1} \left( f \left( A \right) \right) = \left[ 2, 5 \right]$.

\item $f^{-1} \left( C \right) = \left[ -1, \dfrac{3}{2} \right]$.

\item $f \left( f^{-1} \left( C \right) \right) = \left[ -2, 3 \right]$.

\item $f \left( A \cap B \right) = \left[ -5, -3 \right]$.

\item $f \left( A \right) \cap f \left( B \right) = \left[ -5, -3 \right]$.

\item $f^{-1} \left( C \cap D \right) = \left[ -1, 0 \right]$.

\item $f^{1} \left( C \right) \cap f^{-1} \left( D \right) = \left[ -1, 0 \right]$.
\end{enumerate}
\end{multicols}

\item \begin{enumerate}
\item $g \left( A \times A \right) = \left\{6, 12, 18, 24, 36, 54, 72, 108, 216 \right\}$.

\item $g^{-1} \left( C \right) = \left\{ \left( 1, 1 \right), \left( 2, 1 \right), 
\left( 1, 2 \right) \right\}$.

\item $g^{-1} \left( g \left( A \times A \right) \right) = A \times A$.

\item $g \left( g^{1} \left( C \right) \right) = \left\{ 6, 12, 18 \right\}$
\end{enumerate}

\item \begin{enumerate}
\item $\text{range} \left( F \right) = F \left( S \right) = \left\{1, 4, 9, 16 \right\}$.

\item $f \left( A \right) = \left\{ f \left( x \right) \mid x \in A \right\} = \text{range} \left( f \right)$.

\item Let $y \in f \left( A \right)$.  Then, there exists an $x$ in $A$ such that 
$f \left( x \right) = y$.  Since $g \left( x \right) = f \left( x \right)$, we see that 
$g \left( x \right) = y$, and so the function $g$ is a surjection.
\end{enumerate}

\item To prove $f \left( A \cup B \right) \subseteq f \left( A \right) \cup f \left( B \right)$, let $y \in f \left( A \cup B \right)$.  Then there exists an $x \in A \cup B$ such that 
$f \left( x \right) = y$.  

\begin{itemize}
\item If $x \in A$, then $y = f \left( x \right)$ is in $f \left( A \right)$.

\item If $x \in B$, then $y = f \left( x \right)$ is in $f \left( B \right)$.
\end{itemize}
In both cases, $y = f \left( x \right) \in f \left( A \right) \cup f \left( B \right)$ and hence, 
$f \left( A \cup B \right) \subseteq f \left( A \right) \cup f \left( B \right)$.

Now let $y \in f \left( A \right) \cup f \left( B \right)$.  If $y \in f \left( A \right)$, then there exists an $x \in A$ such that $y = f \left( x \right)$.  Since $A \subseteq A \cup B$, this implies that $y = f \left( x \right) \in f \left( A \cup B \right)$.  In a similar manner, we can prove that if $y \in f \left( B \right)$, then $y \in f \left( A \cup B \right)$.  Therefore, 
$f \left( A \right) \cup f \left( B \right) \subseteq f \left( A \cup B \right)$.

\item Let $x \in f^{-1} \left( C \cap D \right)$.  This means that 
$f \left( x \right) \in C \cap D$.  Hence, $f \left( x \right) \in C$ and 
$f \left( x \right) \in D$ and therefore, $x \in f^{-1} \left( C \right)$ and 
$x \in f^{-1} \left( D \right)$.  Consequently, $x \in f^{-1} \left( C \right) \cap f^{-1} \left( D \right)$ and so 
$f^{-1} \left( C \cap D \right) \subseteq f^{-1} \left( C \right) \cap f^{-1} \left( D \right)$.

Now let $x \in f^{-1} \left( C \right) \cap f^{-1} \left( D \right)$.  Then 
$f \left( x \right) \in C$ and $f \left( x \right) \in D$.  This means that 
$f \left( x \right) \in C \cap D$ and so, $x \in f^{-1} \left( C \cap D \right)$.  Therefore, \linebreak
$f^{-1} \left( C \right) \cap f^{-1} \left( D \right) \subseteq f \left( x \right) \in C$.

\item Let $y \in f \left( f^{-1} \left( C \right) \right)$.  Then there exists an $x$ in 
$f^{-1} \left( C \right)$ such that \linebreak
$f \left( x \right) = y$. However, if  $x \in f^{-1} \left( C \right)$, then 
$f \left( x \right) \in C$.  Therefore, $y \in C$ and hence, 
$f \left( f^{-1} \left( C \right) \right) \subseteq C$.



\item \begin{enumerate}
\item  Assume $A \subseteq B$ and let $y \in f \left( A \right)$.  Then there exists an 
$x \in A$ such that $f \left( x \right) = y$.  Since $A \subseteq B$, we see that $x \in B$ and hence, $y = f \left( x \right) \in f \left( B \right)$.  Therefore, 
$f \left( A \right) \subseteq f \left( B \right)$.

\item The statement is false.  Following is one counterexample.
\begin{multicols}{4}
$S = \left\{ 1, 2, 3 \right\}$

$T = \left\{ a, b \right\}$

$A = \left\{ 1 \right\}$

$B = \left\{ 2, 3 \right\}$
\end{multicols}
$f: S \to T$ by $f \left( 1 \right) = a$, $f \left( 2 \right) = a$, $f \left( 3 \right) = b$.

Then,

$f \left( A \right) = \left\{a \right\}$ and $f \left( B \right) = \left\{ a, b \right\}$.  So, 
$f \left( A \right) \subseteq f \left( B \right)$ and $A \not \subseteq B$.
\end{enumerate}


\item \begin{enumerate}
\item Assume $C \subseteq D$ and let $x \in f^{-1} \left( C \right)$.  This means that 
$f \left( x \right) \in C$ and hence, $f \left( x \right) \in D$.  Therefore,  
$x \in f^{-1} \left( D \right)$ and $f^{-1} \left( C \right) \subseteq f^{-1} \left( D \right)$.

\item The statement is false.  Following is one counterexample.
\begin{multicols}{4}
$S = \left\{ 1, 2, 3 \right\}$

$T = \left\{ a, b \right\}$

$C = \left\{ a \right\}$

$D = \left\{ b \right\}$
\end{multicols}
$f: S \to T$ by $f \left( 1 \right) = b$, $f \left( 2 \right) = b$, $f \left( 3 \right) = b$.

Then,

$f^{-1} \left( C \right) = \emptyset$ and $f^{-1} \left( D \right) = S$.  So, 
$f^{-1} \left( C \right) \subseteq f^{-1} \left( D \right)$ and $C \not \subseteq D$.
\end{enumerate}

\item The statement is false.  Following is one counterexample.
\begin{multicols}{4}
$S = \left\{ 1, 2, 3 \right\}$

$T = \left\{ a, b \right\}$

$A = \left\{ 1 \right\}$

$B = \left\{ 2, 3 \right\}$
\end{multicols}
$f: S \to T$ by $f \left( 1 \right) = a$, $f \left( 2 \right) = a$, $f \left( 3 \right) = b$.

Then,

$f \left( A \right) \cap f \left( B \right) = \left\{ a \right\}$ and 
$f \left( Af \left( A \right) \cap f \left( B \right) \cap B \right) = \emptyset$.

\item Let $f: \mathbb{R} \to \mathbb{R}$ by $f \left( x \right) = x^2$ for all 
$x \in \mathbb{R}$.  
\begin{enumerate}
\item If $A = \left[ 0, 1 \right]$, then 
$f \left( A \right) = \left[ 0, 1 \right]$ and 
$f^{-1} \left( f \left( A \right) \right) = \left[ -1, 1 \right]$.  So in this case, 
$A \subset f^{-1} \left( f \left( A \right) \right)$.

\item Let $C = \left[ - 1, 1 \right]$.  Then, $f^{-1} \left( C \right) = \left[ 0, 1 \right]$ and 
$f \left( f^{-1} \left( C \right) \right) = \left[ 0, 1 \right]$.  So in this case, 
$f \left( f^{-1} \left( C \right) \right) \subset C$.
\end{enumerate}

\item The proposition is true.  By Theorem~\ref{T:imageofinvimage}, 
$A \subseteq f^{-1} \left( f \left( A \right) \right)$.  Now let 
$x \in f^{-1} \left( f \left( A \right) \right)$.  Then, 
$f \left( xf \left( A \right) - f \left( B \right) \right) \in f \left( A \right)$, which means that there exists an $a \in A$ such that 
$f \left( a \right) = f \left( x \right)$.  Since $f$ is an injection, $x = a$ and hence, 
$x \in A$.  Thefore, $f^{-1} \left( f \left( A \right) \right) \subseteq A$.

\item The proposition is true.  By Theorem~\ref{T:imageofinvimage},
$f \left( f^{-1} \left( C \right) \right) \subseteq C$.  Let $y \in C$.  Since $f$ is a surjection, there exists an $x \in S$ such that $f \left( x \right) = y$.  This means that 
$x \in f^{-1} \left( C \right)$ and hence, $f \left( x \right) \in C$.  Therefore, 
$y \in C$ and hence, $C \subseteq f \left( f^{-1} \left( C \right) \right)$.

\item Assume that $f : S \to T$ is an injection and let $A$ and $B$ be subsets of $S$.  By 
Theorem~\ref{T:imageofoperations}, 
$f \left( A \cap B \right) \subseteq f \left( A \right) \cap f \left( B \right)$.  Now let 
$y \in f \left( A \right) \cap f \left( B \right)$.  Since $y \in f \left( A \right)$, there exists an $a \in A$ such that $f \left( a \right) = y$.  Similarly, there exists a $b \in B$ such that $f \left( b \right) = y$.  Therefore, $f \left( a \right) = f \left( b \right)$ and since 
$f$ is an injection, $a = b$.  Therefore, $a \in A \cap B$ and hence, 
$y = f \left( a \right) \in f \left( A \cap B \right)$.  So, 
$f \left( A \right) \cap f \left( B \right) \subseteq f \left( A \cap B \right)$.

Now assume that $f \left( A \cap B \right) \subseteq f \left( A \right) \cap f \left( B \right)$ for all subsets $A$ and $B$ of $S$.  Let $a, b \in A$ and assume that 
$a \ne b$.  Let $A = \left\{ a \right\}$ and $B = \left\{ b \right\}$.  Then 
$A \cap B = \emptyset$ and so $f \left( A \cap B \right) = \emptyset$.  By our assumption, this means that $f \left( A \right) \cap f \left( B \right) = \emptyset$ and this implies that 
$f \left( a \right) \ne f \left( b \right)$.  Therefore, $f$ is an injection.

\item It is possible to prove that 
$f \left( A \right) - f \left( B \right) \subseteq f \left( A - B \right)$.  Let \linebreak
$y \in f \left( A \right) - f \left( B \right)$.  Then, $y \in f \left( A \right)$ and 
$y \notin f \left( B \right)$.  So there exists an $a$ in $A$ such that $f \left( a \right) = y$, 
and since $y \notin f \left( B \right)$, $f \left( a \right) \notin f \left( B \right)$. Therefore, $a \notin B$ and so $a \in A - B$.  From this, we see that 
$y = f \left( a \right) \in f \left( A - B \right)$.

It is also possible to prove that if $f$ is an injection, then 
$f \left( A - B \right) = f \left( A \right) - f \left( B \right)$.  Let 
$y \in f \left( A - B \right)$.  There exists an $x \in A - B$ such that $f \left( x \right) = y$.  Since $x \in A$, $y = f \left( x \right) \in f \left( A \right)$.  Now assume that 
$f \left( x \right) \in f \left( B \right)$.  Then there exists a $b$ in $B$ such that 
$f \left( x \right) = f \left( b \right)$.  Since $f$ is an injection, we conclude that $x = b$.  But this means that $x \in B$, which contradicts the fact that $x \in A - B$.  Therefore 
$y = f \left( x \right) \notin f \left( B \right)$ and so 
$y \in f \left( A \right) - f \left( B \right)$, and we have proven that 
$f \left( A - B \right) = f \left( A \right) - f \left( B \right)$.

\hbreak
\end{enumerate}

\endinput
