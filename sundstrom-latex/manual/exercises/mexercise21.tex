\section*{Section \ref{S:logop} Statements and Logical Operators}

\begin{enumerate}
\item The statement is not false and so must be true. When the hypothesis is false, the conditional statement is true.

\item \begin{multicols}{3}
\begin{enumerate}
\item $P$ is false.
\item $P \wedge Q$ is false.
\item $P \vee Q$ is false.
\end{enumerate}
\end{multicols}

\item \begin{multicols}{3}
\begin{enumerate}
\item $\mynot P \to Q$ is true.
\item $Q \to P$ is true.
\item $P \vee Q$ is true.
\end{enumerate}
\end{multicols}

\item \begin{enumerate}
\item $\mynot Q \to P$ is true.
\item $P$ is true.
\item Cannot determine if $P \wedge R$ is true or false.
\item Cannot determine if $R \to \mynot P$ is true or false.
\end{enumerate}


\item  $$
\BeginTable
    \BeginFormat
    | c | c | c | c |
    \EndFormat
     \_6
      | $P$ | $Q$ \|6 $P \to Q$ | $Q \to P$ |\\+22 \_6
      | T   |  T  \|6  T | T | \\ 
      | T   |  F  \|6  F | T | \\ 
      | F   |  T  \|6  F | F | \\ 
      | F   |  F  \|6  T | T | \\ \_6
 \EndTable
 $$
Statements (a) and (d) have the same truth table.  Statements (b) and (c) have the same truth table.


\item $$
\BeginTable
    \BeginFormat
    | c | c | c | c | c | c |
    \EndFormat
     \_6
      | $P$ | $Q$ \|6 $P \vee \mynot Q$ | $\mynot (P \vee Q)$ | $\mynot P \vee \mynot Q$ | $\mynot P \wedge \mynot Q$ | \\+22 \_6
      | T   |  T  \|6  T | F | F | F | \\ 
      | T   |  F  \|6  T | F | T | F | \\ 
      | F   |  T  \|6  F | F | T | F | \\ 
      | F   |  F  \|6  T | T | T | T | \\ \_6
 \EndTable
 $$
Statements (b) and (d) have the same truth table.


\item $$
\BeginTable
\BeginFormat
| c | c | c | c | c |
\EndFormat
\_6
       | $P$  |  $Q$  |  $R$  \|6  $P \wedge (Q \vee R)$  |  $(P \wedge Q) \vee (R \wedge R)$  | \\+22 \_6
          | T | T | T \|6 T | T | \\ 
          | T | T | F \|6 T | T |  \\ 
          | T | F | T \|6 T | T | \\ 
          | T | F | F \|6 F | F | \\ 
          | F | T | T \|6 F | F | \\ 
          | F | T | F \|6 F | F | \\ 
          | F | F | T \|6 F | F |  \\ 
          | F | F | F \|6 F | F |  \\ \_6
\EndTable
$$
The two statements have the same truth table.



\item \begin{multicols}{3}
\begin{enumerate}
\item True
\item False
\item True
\end{enumerate}
\end{multicols}

\item \begin{enumerate}
\item If the integer $x$ is even, then $x^2$ is even.
\item The integer $x$ is even implies that $x^2$ is even.
\item The integer  $x$  is even only if $x^2$   is even.
\item The integer $x^2$ is even is necessary for $x$ to be even.
\item The integer $x$ is even is sufficient for $x^2$ to be even.
\end{enumerate}

\item \begin{enumerate}
\item If $x^2$ is even, then the integer $x$ is even.
\item The integer $x^2$ is even implies that the integer $x$ is even.
\item The integer  $x^2$  is even only if the integer $x$ is even.
\item The integer $x$ is even is necessary for $x^2$ to be even.
\item The integer $x^2$ is even is sufficient for the integer $x$ to be even.
\end{enumerate}


\item $$
\BeginTable
    \BeginFormat
    | c | c | c | c | c | c |
    \EndFormat
     \_6
      | $P$ | $Q$ \|6 $\mynot Q \vee (P \to Q)$ | $Q \wedge (P \wedge \mynot Q$ | $(Q \wedge P) \wedge (P \to \mynot Q)$ | $\mynot Q \to (P \wedge \mynot P) $ | \\+22 \_6
      | T   |  T  \|6  T | F | F | T | \\ 
      | T   |  F  \|6  T | F | F | F | \\ 
      | F   |  T  \|6  T | F | F | T | \\ 
      | F   |  F  \|6  T | F | F | F | \\ \_6
 \EndTable
 $$


\item \begin{enumerate}
\item $$
\BeginTable
    \BeginFormat
    | c | c | c | c | c |
    \EndFormat
     \_6
      | $P$ | $Q$ \|6 $P \to Q$ | $(P \to Q) \wedge P$ | $[(P \to Q) \wedge P] \to Q$ | \\+22 \_6
      | T   |  T  \|6  T | T | T |  \\ 
      | T   |  F  \|6  F | F | T |  \\ 
      | F   |  T  \|6  T | F | T |  \\ 
      | F   |  F  \|6  T | F | T |  \\ \_6
 \EndTable
 $$


\item $$
\BeginTable
\BeginFormat
| c | c | c | c | c |
\EndFormat
\_6
       | $P$  |  $Q$  |  $R$  \|6  $(P \to Q) \wedge (Q \to R)$  |  $[(P \to Q) \wedge (Q \to R)] \to (P \to R)$  | \\+22 \_6
          | T | T | T \|6 T | T | \\ 
          | T | T | F \|6 F | T |  \\ 
          | T | F | T \|6 F | T | \\ 
          | T | F | F \|6 F | T | \\ 
          | F | T | T \|6 T | T | \\ 
          | F | T | F \|6 F | T | \\ 
          | F | F | T \|6 T | T |  \\ 
          | F | F | F \|6 T | T |  \\ \_6
\EndTable
$$
\end{enumerate}
\end{enumerate}




\subsection*{Explorations and Activities}
\setcounter{oldenumi}{\theenumi}
\begin{enumerate} \setcounter{enumi}{\theoldenumi}
\item $$
\BeginTable
\BeginFormat
|l|c|c|c|
\EndFormat
  \_
 | \textbf{English Form}  |  \textbf{Hypothesis}  |  \textbf{Conclusion} |  \textbf{Symbolic Form}  | \\+22  \_
 | If $P$, then $Q$.              |  $P$  |  $Q$  |  $P \to Q$  | \\ \_1
|  $Q$ only if $P$.               |  $Q$  |  $P$  |  $Q \to P$  | \\ \_1
|  $P$ is necessary for $Q$.      |  $Q$  |  $P$  |  $Q \to P$  | \\ \_1
|  $P$ is sufficient for $Q$.     |  $P$  |  $Q$  |  $P \to Q$  | \\ \_1
|  $Q$ is necessary for $P$.      |  $P$  |  $Q$  |  $P \to Q$  | \\ \_1
|  $P$ implies $Q$.               |  $P$  |  $Q$  |  $P \to Q$  | \\ \_1
|  $P$ only if $Q$.               |  $P$  |  $Q$  |  $P \to Q$  | \\ \_1
|  $P$ if $Q$.                    |  $Q$  |  $P$  |  $Q \to P$  | \\ \_1
|  If $Q$ then $P$.               |  $Q$  |  $P$  |  $Q \to P$  | \\ \_1
|  If  $\neg  Q$, then $\neg  P$. |  $\mynot Q$| $\mynot P$   | $\mynot Q \to \mynot P$       | \\ \_1
|  If $P$, then $Q \wedge R$.     |  $P$  |  $Q \wedge R$     | $P \to (Q \wedge R)$           | \\ \_1
|  If $P \vee Q$, then $R$.       |  $P \vee Q$   | $R$       | $(P \vee Q) \to R$           | \\ \hline
\EndTable
$$


\item \begin{enumerate}
  \item $(P \vee Q) \vee (U \wedge W)$ is true since $P \vee Q$ is true.
  \item $P \wedge (Q \to W)$.  It is not possible to tell if this is true or false since it is not possible to tell if $Q \to W$ is true or false.
  \item $P \wedge (W \to Q)$ is true since both $P$ and $W \to Q$ are true.
  \item $W \to (P \wedge U)$.  It is not possible to tell if this is true or false since $P \wedge U$ is false.
  \item $W \to (P \wedge \mynot U)$ is true since $P \wedge \mynot U$ is true.
  \item $(\mynot P \vee \mynot U) \wedge (Q \vee \mynot V)$ is true since both $(\mynot P \vee \mynot U)$ and 
$(Q \vee \mynot V)$ are true.
  \item $(P \wedge \mynot V) \wedge (U \vee W)$.  It is not possible to determine if this is true or false since it is not possible to determine if $U \vee W$ is true or false.
  \item $(P \vee \mynot Q) \to (U \wedge W)$ is false since $(P \vee \mynot Q)$ is true and $(U \wedge W)$ is false
  \item $(P \vee W) \to (U \wedge W)$ is false since $P \vee W$ is true and $U \wedge W$ is false.
  \item $(U \wedge \mynot V) \to (P \wedge W)$ is true since $(U \wedge \mynot V)$ is false.
\end{enumerate}


\end{enumerate}


\hbreak
\endinput
