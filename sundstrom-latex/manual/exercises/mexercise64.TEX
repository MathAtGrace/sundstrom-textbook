\section*{Section \ref{S:compositionoffunctions} Composition of Functions}

\begin{enumerate}
\item \begin{enumerate}
\item It is possible to determine $\left( {g \circ f} \right)\left( x \right)$  for all  
$x \in \mathbb{R}$.  In fact,
\[
\begin{aligned}
\left( {g \circ f} \right)\left( x \right) &= g \left( x^2 + 1 \right) \\
                                           &= \frac{1}{x^2 + 1}. \\
\end{aligned}
\]
\item Let  $f:A \to T$ and  $g:B \to C$.  If $T \subseteq B$, then the composite function 
$g \circ f:A \to C$ can be defined.
\end{enumerate}


\item $( {g \circ h} ):\mathbb{R} \to \mathbb{R}$  by  
$( {g \circ h} )( x ) = g( {h( x )} ) = g\!\left( {x^3 } \right) = 3x^3  + 2$.

$( {h \circ g} ):\mathbb{R} \to \mathbb{R}$  by  
$( {h \circ g} )( x ) = h( {g( x )} ) = h( {3x + 2} ) = ( {3x + 2} )^3 $.

\noindent
This shows that $h \circ g \ne g \circ h$ or that composition of functions is not commutative.


\item \begin{enumerate}
\item $F\left( x \right) = \left( {g \circ f} \right)\left( x \right)$, where 
$f\left( x \right) = e^x$  and $g\left( x \right) = \cos x$.

\item $G\left( x \right) = \left( {g \circ f} \right)\left( x \right)$ where 
$f\left( x \right) = \cos x$ and $g\left( x \right) = e^x $.

\item $H\left( x \right) = \left( {g \circ f} \right)\left( x \right)$, 
$f\left( x \right) = \sin x$, $g\left( x \right) = \frac{1}{x}$.

\item $K\left( x \right) = \left( {g \circ f} \right)\left( x \right)$, 
$f\left( x \right) = e^{-x^2}$, $g\left( x \right) = \cos x$.
\end{enumerate}




\item \begin{enumerate}
\item For each  $x \in A$, $\left( {f \circ I_A } \right)\left( x \right) = f\left( {I_A \left( x \right)} \right) = f\left( x \right)$.  Therefore, $f \circ I_A  = f$.

\item For each  $x \in A$, $\left( {I_B \circ f } \right)\left( x \right) = I_B \left( {f \left( x \right)} \right) = f\left( x \right)$.  Therefore, $I_B \circ f  = f$.
\end{enumerate}


\item \begin{enumerate}
\item $\left[ {\left( {h \circ g} \right) \circ f} \right]\left( x \right) = \sqrt[3]{{\sin \left( {x^2 } \right)}}$; 
$\left[ {h \circ \left( {g \circ f} \right)} \right]\left( x \right) = \sqrt[3]{{\sin \left( {x^2 } \right)}}$.

\item 
$\left[ {\left( {h \circ g} \right) \circ f} \right]\left( x \right) 
  = \left[ \left( {h \circ g} \right) \right] \left( f \left( x \right) \right)
  = h \left( g \left( f \left( x \right) \right) \right).$

$\left[ {h \circ \left( {g \circ f} \right)} \right]\left( x \right) 
  = h \left( \left( g \circ f \right) \left( x \right) \right) 
  = h \left( g \left( f \left( x \right) \right) \right)$. 
\end{enumerate}



\item Let  $A$, $B$, and  $C$  be nonempty sets and let  $f:A \to B$  and  $g:B \to C$.  If  $f$  and  $g$  are both injections, then  $g \circ f$  is an injection. 

\textbf{\emph{Proof:}} Let  $A$, $B$, and  $C$  be nonempty sets and let  $f:A \to B$  and  
$g:B \to C$.  Assume that $f$  and  $g$  are both injections.  Let  $x, y \in A$ and assume that  
$\left( {g \circ f} \right)\left( x \right) = \left( {g \circ f} \right)\left( y \right)$.  Then,
\[
g \left( f \left( x \right) \right) = g \left( f \left( y \right) \right),
\] 
and since $g$ is an injection, we conclude that $f \left( x \right) = f \left( y \right)$.  Now, since $f$ is an injection, we see that $x = y$ and hence $g \circ f$ is an injection.





\item \begin{enumerate}
\item $f: \mathbb{R} \to \mathbb{R}$ by $f \left( x \right) = x$, 
$g: \mathbb{R} \to \mathbb{R}$ by $g \left( x \right) = x^2$.  The function $f$ is a surjection, but $g \circ f$ is not a surjection.

\item $f: \mathbb{R} \to \mathbb{R}$ by $f \left( x \right) = x$, 
$g: \mathbb{R} \to \mathbb{R}$ by $g \left( x \right) = x^2$.  The function $f$ is an injection, but $g \circ f$ is not an injection.

\item $f: \mathbb{R} \to \mathbb{R}$ by $f \left( x \right) = x^2$, 
$g: \mathbb{R} \to \mathbb{R}$ by $g \left( x \right) = x$.  The function $g$ is a surjection, but $g \circ f$ is not a surjection.

\item $f: \mathbb{R} \to \mathbb{R}$ by $f \left( x \right) = x^2$, 
$g: \mathbb{R} \to \mathbb{R}$ by $g \left( x \right) = x$.  The function $g$ is an injection, but $g \circ f$ is not an injection.

\item $f:\mathbb{R}^+ \to \mathbb{R}$ by $f \left( x \right) = e^x$, 
$g:\mathbb{R} \to \mathbb{R}$ by $g \left( x \right) = \ln x$.  The function $f$ is not a surjection but the function $g \circ f$ is a surjection.

\item By Part~(\ref{T:morecompositefunctions1}) of Theorem~\ref{T:morecompositefunctions}, this is not possible since if $g \circ f$ is an injection, then $f$ is an injection.

\item By Part~(\ref{T:morecompositefunctions2}) of Theorem~\ref{T:morecompositefunctions}, this is not possible since if $g \circ f$ is a surjection, then $g$ is a surjection.

\item $f:\mathbb{R} \to \mathbb{R}$ by $f \left( x \right) = e^x$, 
$g:\mathbb{R} \to \mathbb{R}$ by $g \left( x \right) = x^2$.  The function $g$ is not an injection but the function $g \circ f$ is an injection. 
$\left[ \left( g \circ f \right) \left( x \right) = e^{2x}.\right]$
\end{enumerate}



\item \begin{enumerate}
\item Verify that $f^2(x) = x + 2$, $f^3(x) = x + 3$, and $f^4(x) = x + 4$.  Let $P(n)$ be 
``$f^n(x) = x + n$.  We have verified that the basis step is true.  So let $k \in \N$ and assume that $P(k)$ is true.  So $f^k(x) = x + k$.  Then
\begin{align*}
f^{k+1}(x) &= \left( f \circ f^k \right)(x) \\
           &= f \left( f^k(x) \right) \\
           &= f(x + k) \\
           &= (x + k) + 1 \\
           &= x + (k + 1).
\end{align*}
This proves that if $P(k)$ is true, then $P(k + 1)$ is true.

\item Verify that $f^2(x) = a^2x + ab + b$, $f^3(x) = a^3x + a^2b+ab+b$, and $f^4(x) = a^4x + a^3b + a^2b+ab+b$.  Let $P(n)$ be \\``$f^n(x) = a^nx + a^{n-1}b + a^{n-2}b + \cdots a^2b + ab + b$''.  We have verified that the basis step is true.  So let $k \in \N$ and assume that $P(k)$ is true.  So $f^k(x) = a^kx + a^{k-1}b + \cdots a^2b + ab + b$.  Then
\begin{align*}
f^{k+1}(x) &= \left( f \circ f^k \right)(x) \\
           &= f \left( f^k(x) \right) \\
           &= f\left(a^kx + a^{k-1}b + \cdots a^2b + ab + b \right) \\
           &= a\left(a^kx + a^{k-1}b + \cdots a^2b + ab + b \right) + b \\
           &= a^{k+1}x + a^kb + a^{k-1}b + \cdots + a^2b + ab + b.
\end{align*}
This proves that if $P(k)$ is true, then $P(k + 1)$ is true.

\item Let $P(n)$ be ``$f^{n+1} = f^n \circ f$''.  Then $P(1)$ is true since 
$f^2 = f \circ f = f^1 \circ f$.  Now let $k \in \N$ and assume that $P(k)$ is true.  That is, assume that $f^{k+1} = f^k \circ f$.  We then see that
\begin{align*}
f^{k+2} &= f \circ f^{k+1} \\
        &= f \circ \left( f^k \circ f \right) \\
        &= \left( f \circ f^k \right) \circ f \\
        &= f^{k+1} \circ f.
\end{align*}
This proves that if $P(k)$ is true, then $P(k + 1)$ is true.
\end{enumerate}
\end{enumerate}



\subsection*{Explorations and Activities}
\setcounter{oldenumi}{\theenumi}
\begin{enumerate} \setcounter{enumi}{\theoldenumi}
\item Let  $A$, $B$, and  $C$  be nonempty sets and let  $f\x A \to B$  and  $g\x B \to C$.

\begin{enumerate} 
\item One example is:  $A = \left\{ {a, b} \right\}$, $B = \left\{ {p, q, r} \right\}$, 
$C = \left\{ {x, y} \right\}$.

$f\x A \to B$  with  $f\left( a \right) = p$ and  $f\left( b \right) = q$.

$g\x B \to C$ with  $g\left( p \right) = x$, $g\left( q \right) = y$, and  $g\left( r \right) = x$. 

Then,
$g \circ f\x A \to C$  with  $\left( {g \circ f} \right)\left( a \right) = x$  and   
$\left( {g \circ f} \right)\left( b \right) = y$, the function  $f$  is an injection,  $g$  is not an injection, and  $g \circ f$ is an injection.

\item This is not possible.  If  $f$  is not an injection, then there exist  $s, t \in A$ with  
$s \ne t$  and  $f\left( s \right) = f\left( t \right)$.  Then,  
$g\left( {f\left( s \right)} \right) = g\left( {f\left( t \right)} \right)$ and hence, $g \circ f$
is not an injection.

\item This is not possible.  If  $g$  is not a surjection, then there exists a  $y \in C$ such that  $g\left( x \right) \ne y$ for all  $x \in B$.  So, for each  $a \in A$,  
$f\left( a \right) \in B$ and hence  $g\left( {f\left( a \right)} \right) \ne y$.  This means that  $g \circ f$ is not a surjection.

\item One example is:  $A = \left\{ {a, b} \right\}$, $B = \left\{ {p, q, r} \right\}$, 
$C = \left\{ {x, y} \right\}$.

$f\x A \to B$  with  $f\left( a \right) = p$ and  $f\left( b \right) = q$.

$g\x B \to C$ with  $g\left( p \right) = x$, $g\left( q \right) = y$, and  $g\left( r \right) = x$. 

Then, $g \circ f\x A \to C$  with  $\left( {g \circ f} \right)\left( a \right) = x$  and   
$\left( {g \circ f} \right)\left( b \right) = y$, the function  $f$  is not a surjection,  $g$  is not a surjection, and  $g \circ f$ is a surjection.

\end{enumerate}



\item \begin{enumerate}
\item Let  $A$, $B$, and  $C$  be nonempty sets and let  $f:A \to B$  and  $g:B \to C$.  If  
$g \circ f:A \to C$  is an injection, then  $f:A \to B$  is an injection.

\textbf{\emph{Proof:}}  Let  $A$, $B$, and  $C$  be nonempty sets and let  $f:A \to B$  and  
$g:B \to C$.  Assume that $g \circ f$ is an injection.  To prove that $f$ is an injection, we let $x$ and $y$ be in $A$ and assume that $f \left( x \right) = f \left( y \right)$.  Since $f \left( x \right)$ is in $B$, we can use the function $g$ to conclude that
\[
\begin{aligned}
g \left( f \left( x \right) \right) &= g \left( f \left( y \right) \right) \\
\left( g \circ f \right) \left( x \right) &= \left( g \circ f \right) \left( y \right). \\
\end{aligned}
\]
Since $g \circ f$ is an injection, the last equation implies that $x = y$. Hence, $f$ is an injection.

\item Let  $A$, $B$, and  $C$  be nonempty sets and let  $f:A \to B$  and  $g:B \to C$.  If  
$g \circ f:A \to C$  is a surjection, then  $g:B \to C$  is a surjection.

\textbf{\emph{Proof:}}  Let  $A$, $B$, and  $C$  be nonempty sets and let  $f:A \to B$  and  
$g:B \to C$, and assume that $g \circ f$ is a surjection.  To prove that $g$ is a surjection, we let $z \in C$.  Now, since $g \circ f$ is a surjection, there exists an $x \in A$ such that
\[
\left( g \circ f \right) \left( x \right) = z,
\]
and this implies that $g \left( f \left( x \right) \right) = z$.  In addition, 
$f \left( x \right) \in B$ and hence using $y = f \left( x \right)$, we have proven that there exists an element $y \in B$ such that $g \left( y \right) = z$.  Therefore, $g$ is a surjection.
\end{enumerate}


\end{enumerate}

\hbreak


\endinput
