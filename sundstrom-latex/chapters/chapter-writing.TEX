\chapter{Guidelines for Writing Mathematical Proofs} \label{C:writingguides}
\markboth{Appendix~\ref{C:writingguides}. Writing Guidelines}{Appendix~\ref{C:writingguides}. Writing Guidelines}
\index{writing guidelines|(}%

One of the most important forms of mathematical writing is writing mathematical proofs.  The writing of mathematical proofs is an acquired skill and takes a lot of practice. Throughout the textbook, we have introduced various guidelines for writing proofs.  These guidelines are in Sections~\ref{S:prop}, \ref{S:direct}, \ref{S:directproof}, \ref{S:moremethods}, \ref{S:contradiction}, and \ref{S:mathinduction}.

Following is a summary of all the writing guidelines introduced in the text.  This summary contains some standard conventions that are usually followed when writing a mathematical proof.

\begin{enumerate}
\item \textbf{Know your audience}. 
\label{writing:know}%
Every writer should have a clear idea of the intended audience for a piece of writing.  In that way, the writer can give the right amount of information at the proper level of sophistication to communicate effectively.  This is especially true for mathematical writing.  For example, if a mathematician is writing a solution to a textbook problem for a solutions manual for instructors, the writing would be brief with many details omitted.  However, if the writing was for a students' solution manual, more details would be included.  %This is why the instructions for Preview Activity~\ref{PA:equation} stated that your descriptions should be written for someone who already knows basic algebra and how to solve quadratic equations.


\item \textbf{Begin with a carefully worded statement of the theorem or result to be proven.}
The statement should be a simple declarative statement of the problem.  Do not simply rewrite the problem as stated in the textbook or given on a handout.  Problems often begin with phrases such as ``Show that'' or ``Prove that.''  This should be reworded as a simple declarative statement of the theorem.  Then skip a line and write ``Proof''  in italics or boldface font (when using a word processor).  Begin the proof on the same line.  Make sure that all paragraphs can be easily identified.  Skipping a line between paragraphs or indenting each paragraph can accomplish this.

As an example, an exercise in a text might read, ``Prove that if $x$  is an odd integer, then $x^2$ is an odd integer.''  This could be started as follows:

\textbf{Theorem.} 
If  $x$  is an odd integer, then $x^2$ is an odd integer.

\textbf{\emph{Proof}}:  We assume that  $x$  is an odd integer  \ldots .

\item \textbf{Begin the proof with a statement of your assumptions.}
Follow the statement of your assumptions with a statement of what you will prove.
\begin{flushleft}
\emph{\textbf{Proof}}.  We assume that  $x$  and  $y$  are odd integers and will prove that $x \cdot y$   is an odd integer.
\end{flushleft}

\item \textbf{Use the pronoun ``we.''}
If a pronoun is used in a proof, the usual convention is to use ``we'' instead of ``I.''  The idea is to stress that you and the reader are doing the mathematics together.  It will help encourage the reader to continue working through the mathematics.  Notice that we started the proof of Theorem~\ref{T:xyodd} with ``We assume that $\ldots .$''

%If a pronoun is used in a proof, the usual convention is to use ``we'' instead of ``I.''  The idea is that the author and the reader are proving the theorem together.



\item \textbf{Use italics for variables when using a word processor.}
When using a word processor to write mathematics, the word processor needs to be capable of producing the appropriate mathematical symbols and equations.  The mathematics that is written with a word processor should look like typeset mathematics.  This means that variables need to be italicized, boldface is used for vectors, and regular font is used for mathematical terms such as the names of the trigonometric functions and logarithmic functions.  

For example, we do not write sin x or $sin \: x$.  The proper way to typeset this is $\sin x$.



\item \textbf{Do not use $*$ for multiplication or \^{} for exponents.}
Leave this type of notation for writing computer code.  The use of this notation makes it difficult for humans to read.  In addition, avoid using $/$ for division when using a complex fraction.  

For example, it is very difficult to read 
$\left(x^3 -3x^2 + 1/2 \right)/\left(2x/3 - 7\right)$; the fraction
\[
\frac{x^3 - 3x^2 +\dfrac{1}{2}}{\dfrac{2x}{3} - 7}
\]
is much easier to read.


\item \textbf{Use complete sentences and proper paragraph structure.}
Good grammar is an important part of any writing.  Therefore, conform to the accepted rules of grammar.  Pay careful attention to the structure of sentences.  Write proofs using \textbf{complete sentences} but avoid run-on sentences.  Also, do not forget punctuation, and always use a spell checker when using a word processor.



\item \textbf{Keep the reader informed.} 
Sometimes a theorem is proven by proving the contrapositive or by using a proof by contradiction.  If either proof method is used, this should be indicated within the first few lines of the proof.  This also applies if the result is going to be proven using mathematical induction.

Examples:
\begin{itemize}
\item We will prove this result by proving the contrapositive of the statement.

%\item We will prove this result using a proof by contraposition.

\item We will prove this statement using a proof by contradiction.

\item We will assume to the contrary that $\ldots .$

\item We will use mathematical induction to prove this result.
\end{itemize}

In addition, make sure the reader knows the status of every assertion that is made.  That is, make sure it is clearly stated whether an assertion is an assumption of the theorem, a previously proven result, a well-known result, or something from the reader's mathematical background.


\item \textbf{Display important equations and mathematical expressions.}
Equations and manipulations are often an integral part of the exposition.  Do not write equations, algebraic manipulations, or formulas in one column with reasons given in another column (as is often done in geometry texts).   Important equations and manipulations should be displayed.  This means that they should be centered with blank lines before and after the equation or manipulations, and if one side of an equation does not change, it should not be repeated.  For example,

Using algebra, we obtain
	
\begin{align}
  x \cdot y &= \left( {2m + 1} \right)\left( {2n + 1} \right)  \notag \\ 
            &= 4mn + 2m + 2n + 1  \notag \\ 
            &= 2\left( {2mn + m + n} \right) + 1.  \notag  
\end{align} 

Since  $m$  and  $n$  are integers, we conclude that $ \ldots .$


\item \textbf{Equation numbering guidelines.} 
If it is necessary to refer to an equation later in a proof, that equation should be centered and displayed, and it should be given a number.  The number for the equation should be written in parentheses on the same line as the equation at the right-hand margin.

\textbf{Example:}
\setcounter{equation}{0}

Since  $x$  is an odd integer, there exists an integer  $n$  such that
\begin{equation}\label{wg.eqnum}
x = 2n + 1.
\end{equation}

\begin{flushleft}
Later in the proof, there may be a line such as
\begin{center}
Then, using the result in equation~(\ref{wg.eqnum}), we obtain $\ldots .$
\end{center}
Please note that we should only number those equations we will be referring to later in the proof.  
%Also, note that the word ``Equation'' begins with a capital ``E'' when we are referring to an equation by number.
Also, note that the word ``equation'' is not capitalized when we are referring to an equation by number.  Although it may be appropriate to use a capital ``E,'' the usual convention in mathematics is not to capitalize.
\end{flushleft}

\item \textbf{Do not use a mathematical symbol at the beginning of a sentence.}\\
For example, we should not write, ``Let $n$ be an integer.  $n$ is an odd integer provided that $\ldots .$''  Many people find this hard to read and often have to re-read it to understand it.  It would be better to write, ``An integer $n$ is an odd integer provided that $\ldots .$''

\item \textbf{Use English and minimize the use of cumbersome notation}.  Do not use the special symbols for quantifiers $\forall$ (for all), 
$\exists$ (there exists), $\mathrel\backepsilon$ (such that), or $\therefore $ (therefore) in formal mathematical writing.  It is often easier to write, and usually easier to read, if the English words are used instead of the symbols.  For example, why make the reader interpret
\[
\left( \forall x \in \R \right) \left( \exists y \in \R \right)\left( x + y = 0 \right)
\]
when it is possible to write
\begin{center}
For each real number $x$, there exists a real number $y$ such that $x + y = 0$,
\end{center}
or more succinctly (if appropriate)
\begin{center}
Every real number has an additive inverse.
\end{center}
\index{writing guidelines|)}%


\item \textbf{Tell the reader when the proof has been completed.}
Perhaps the best way to do this is to say outright that, ``This completes the proof.''  Although it may seem repetitive, a good alternative is to finish a proof with a sentence that states precisely what has been proven.  In any case, it is usually good practice to use  some ``end of proof symbol'' such as  $\blacksquare$.  
 


\item \textbf{Keep it simple.}  
It is often difficult to understand a mathematical argument no matter how well it is written.  Do not let your writing help make it more difficult for the reader.  Use simple, declarative sentences and short paragraphs, each with a simple point.


\item \textbf{Write a first draft of your proof and then revise it.} 
Remember that a proof is written so that readers are able to read and understand the reasoning in the proof.  Be clear and concise.  Include details but do not ramble.  Do not be satisfied with the first draft of a proof.  Read it over and refine it.  Just like any worthwhile activity, learning to write mathematics well takes practice and hard work.  This can be frustrating.  Everyone can be sure that there will be some proofs that are difficult to construct, but remember that proofs are a very important part of mathematics.  So work hard and have fun.
\end{enumerate}


\endinput

\begin{enumerate}
\item \textbf{Begin with a carefully worded statement of the theorem or result to be proven.}
\label{wg:begin} 
The statement should be a simple declarative statement of the problem. Do not simply rewrite the problem as stated in the textbook or given on a handout.  Problems often begin with phrases such as ``Show that'' or ``Prove that.''  This should be reworded as a simple declarative statement of the theorem.  Then skip a line and write ``Proof''  in boldface font (when using a word processor).  Begin the proof on the same line.  Make sure that all paragraphs can be easily identified.  Skipping a line between paragraphs or indenting each paragraph can accomplish this.

As an example, an exercise in a text might read, ``Prove that if $x$  is an odd integer, then $x^2$ is an odd integer.''  This could be started as follows:

\textbf{Theorem.} 
If  $x$  is an odd integer, then $x^2$ is an odd integer.

\textbf{\emph{Proof}}:  We assume that  $x$  is an odd integer  \ldots

\newpage
\item \textbf{Begin the proof with a statement of the assumptions.}  \label{wg:begin2} 
This is illustrated in the example in Part~(\ref{wg:begin}).  Follow the statement of the assumptions with a statement of what will be proven.

\textbf{Theorem}
If  $x$  is an odd integer, then $x^2$ is an odd integer.

\textbf{\emph{Proof}}:  We assume that  $x$  is an odd integer, and we will prove that $x^2$ is an odd integer.

\item \textbf{Use the pronoun ``we.''}
If a pronoun is used in a proof, the usual convention is to use ``we'' instead of ``I.''  The idea is that the author and the reader are proving the theorem together.

\item \textbf{Use italics for variables.}
When using a word processor to write mathematics, the word processor needs to be capable of producing the appropriate mathematical symbols and equations.  The mathematics that is written with a word processor should look like typeset mathematics.  This means that variables need to be italicized, boldface is used for vectors, and regular font is used for mathematical terms such as the names of the trigonometric functions and logarithmic functions.  The use of italics is illustrated in the example in Part~(\ref{wg:begin2}).

\item \textbf{Use complete sentences and proper paragraph structure.}
Good grammar is an important part of any writing.  Therefore, conform to the accepted rules of grammar.  Pay careful attention to the structure of sentences.  Write proofs using \textbf{complete sentences} but avoid run-on sentences.  Part of good grammar is correct spelling.  Always use a spell checker when using a word processor.

\item \textbf{Keep the reader informed.}
Sometimes a theorem is proven by proving the contrapositive or by using a proof by contradiction.  If either proof method is used, this should be indicated within the first few lines of the proof.  This also applies if the result is going to be proven using mathematical induction.

Examples:
\begin{itemize}
\item We will prove this result by proving the contrapositive of the statement.

%\item We will prove this result using a proof by contraposition.

\item We will prove this statement using a proof by contradiction.

\item We will assume to the contrary that \ldots

\item We will use mathematical induction to prove this result.
\end{itemize}

In addition, make sure the reader knows the status of every assertion that is made.  That is, make sure it is clearly stated whether an assertion is an assumption of the theorem, a previously proven result, a well-known result, or something from the reader's mathematical background.

\item \textbf{Display important equations and mathematical expressions.}
Equations and manipulations are often an integral part of the exposition.  Do not write equations, algebraic manipulations, or formulas in one column with reasons given in another column (as is often done in geometry texts).   Important equations and manipulations should be displayed.  This means that they should be centered with blank lines before and after the equation or manipulations.  If several steps are shown together, the equals signs should be aligned, and if one side of an equation does not change, it should not be repeated.  For example,

Using algebra, we obtain
	
\begin{align}
  x \cdot y &= \left( {2m + 1} \right)\left( {2n + 1} \right)  \notag \\ 
            &= 4mn + 2m + 2n + 1  \notag \\ 
            &= 2\left( {2mn + m + n} \right) + 1 . \notag  
\end{align} 


Since  $m$  and  $n$  are integers, we conclude that $ \ldots $

\item \textbf{Equation numbering guidelines}
If it is necessary to refer to an equation later in a proof, that equation should be centered and displayed, and it should be given a number.  The number for the equation should be written in parentheses on the same line as the equation at the right-hand margin.

\textbf{Example:}
\setcounter{equation}{0}

Since  $x$  is an odd integer, there exists an integer  $n$  such that
\begin{equation} \label{wg.eqnum}
x = 2n + 1.
\end{equation}

\begin{flushleft}
Later in the proof, there may be a line such as
\begin{center}
Then, using the result in Equation~(\ref{wg.eqnum}), we obtain  . . .
\end{center}
Please note that we should only number those equations we will be referring to later in the proof.  Also, note that the word ``Equation'' begins with a capital ``E'' when we are referring to an equation by number.
\end{flushleft}

\item \textbf{Do not begin a sentence with a mathematical symbol.}
In addition to not beginning a sentence with a mathematical symbol, in formal writing in mathematics, we do not use the special symbols $\forall $ (for all), $\exists $ (there exists), 
$\mathrel\backepsilon  $ (such that), or $\therefore $ (therefore).

\item \textbf{Tell the reader when the proof has been completed.}
Perhaps the best way to do this is to say outright that, ``This completes the proof.''  Although it may seem repetitive, a good alternative is to finish a proof with a sentence that states precisely what has been proven.  In any case, it is usually good practice to use  some ``end of proof symbol'' such as  $\blacksquare$.

\item \textbf{Write a first draft of your proof and then revise it.}
Remember that a proof is written so that readers are able to read and understand the reasoning in the proof.  Be clear and concise.  Include details but do not ramble.  Do not be satisfied with the first draft of a proof.  Read it over and refine it.  Just like any worthwhile activity, learning to write mathematics well takes practice and hard work.  This can be frustrating.  Everyone can be sure that there will be some proofs that are difficult to construct, but remember that proofs are a very important part of mathematics.  So work hard and have fun.

\end{enumerate}

\endinput
