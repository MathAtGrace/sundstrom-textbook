
\chapter{Note to Students}
\markboth{Note to Students}{Note to Students}
This book may be different than other mathematics textbooks you have used since one of the main goals of this book is to help you to develop the ability to construct and write mathematical proofs.  So this book is not just about mathematical content but is also about the process of doing mathematics.  Along the way, you will also learn some important mathematical topics that will help you in your future study of mathematics.    

This book is designed not to be just casually read but rather to be \emph{engaged}.  
%One of the most important things for you to do in this journey is to actively read and study the material in this textbook.  
It may seem like a clich\'{e} (because it is in almost every mathematics book now) but there is truth in the statement that \emph{mathematics is not a spectator sport}.  To learn and understand mathematics, you must \emph{engage} in the process of doing mathematics.  So you must actively read and study the book, which means to have a pencil and paper with you and be willing to follow along and fill in missing details.  This type of engagement is not easy and is often frustrating, but if you do so, you will learn a great deal about mathematics and more importantly, about doing mathematics.

Recognizing that actively studying a mathematics book is often not easy, several features of the textbook have been designed to help you become more engaged as you study the material.  Some of the features are:
\begin{itemize}
  \item \textbf{Preview Activities.}  With the exception of Sections~\ref{S:prop} and~\ref{S:reviewproofs}, each section has exactly two preview activities.  Some Preview Activities will review prior mathematical work that is necessary for the new section.  This prior work may contain material from previous mathematical courses or it may contain material covered earlier in this text.  Other preview activities will introduce new concepts and definitions that will be used when that section is discussed in class.   It is very important that you work on these preview activities before starting the rest of the section.   Please note that answers to these preview activities are not included in the text.  This book is designed to be used for a course and it is left up to the discretion of each individual instructor as to how to distribute the answers to the preview activities.

\item \textbf{Progress Checks.}  Several Progress Checks are included in each section.  These are either short exercises or short activities designed to help you determine if you are understanding the material as it is presented.  As such, it is important to work through these progress checks to test your understanding, and if necessary, study the material again before proceeding further.  Answers to the Progress Checks are provided in Appendix~\ref{C:progress}.

\item \textbf{Chapter Summaries.}  To assist you with studying the material in the text, there is a summary at the end of each chapter.  The summaries usually list the important definitions introduced in the chapter and the important results proven in the chapter.  If appropriate, the summary also describes the important proof techniques discussed in the chapter.

\item \textbf{Answers for Selected Exercises.}  Answers or hints for several exercises are included in an appendix.    Those exercises with an answer or a hint in the appendix are preceded by a star 
$\left( ^\star \right)$.
\end{itemize}

Although not part of the textbook, there are now 107 online ideos with about 14 hours of content that span the first seven chapters of this book. These videos are freely available online at Grand Valley's Department of Mathematics YouTube channel on this playlist: 
 
\begin{center}
\href{http://www.youtube.com/playlist?list=PL2419488168AE7001}
{http://www.youtube.com/playlist?list=PL2419488168AE7001}
\end{center}
These online videos were created and developed by Dr. Robert Talbert of Grand Valley State University.

There is also a web site for the textbook at
\begin{center}
\href{https://sites.google.com/site/mathematicalreasoning3ed/}
{https://sites.google.com/site/mathematicalreasoning3ed/}
\end{center}
You may find some things there that could be of help.  For example, there currently is a link to study guides for the sections of this textbook.
Good luck with your study of mathematics and please make use of the online videos and the resources available in the textbook.  If there are things that you think would be good additions to the book or the web site, please feel free to send me a message at mathreasoning@gmail.com.




\endinput

