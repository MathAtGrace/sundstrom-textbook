\section{Chapter \ref{C:settheory} Summary} \label{Su:settheory}

\subsection*{Important Definitions}
\begin{multicols}{2}
\begin{itemize}
\item Equal sets, page~\pageref*{sym:setequal12}
\item Subset, page~\pageref*{sym:subset2}
\item Proper subset, page~\pageref*{sym:propersub}
\item Power set, page~\pageref*{D:powerset}
\item Cardinality of a finite set, page~\pageref*{D:cardinality}
\item Intersection of two sets, page~\pageref*{intersection}
\item Union of two sets, page~\pageref*{intersection}
\item Set difference, page~\pageref*{setdiff}
\item Complement of a set, page~\pageref*{setdiff}
\item Disjoint sets, page~\pageref*{D:disjointsets}
\item Cartesian product of two sets, pages~\pageref*{cartesianproduct}%, \pageref*{cartesianproduct2}
\item Ordered pair, page~\pageref*{orderedpair}
\item Union over a family of sets, page~\pageref*{D:familyoper}
\item Intersection over a family of sets, page~\pageref*{D:familyoper}
\item Indexing set, page~\pageref*{D:indexfamily}
\item Indexed family of sets, page~\pageref*{D:indexfamily}
\item Union over an indexed family of sets, page~\pageref*{D:indexoper}
\item Intersection over an indexed family of sets, page~\pageref*{D:indexoper}
\item Pairwise disjoint family of sets, page~\pageref*{D:disjointfamily}
\end{itemize}
\end{multicols}
\hbreak



\subsection*{Important Theorems and Results about Sets} \label{SS:setresults}
\index{algebra of sets}%
\begin{itemize}
\item \textbf{Theorem \ref{T:powerset}}.  
\emph{Let  $n$  be a nonnegative integer and let  $A$  be a subset of some universal set.  If  $A$  is a finite set with  $n$  elements, then  $A$  has  $2^n $ subsets.  That is,  if  $\left| A \right| = n$, then  $\left| {\mathcal{P}\left( A \right)} \right| = 2^n $.}

\item \textbf{Theorem~\ref{T:algebraofsets}}.
\emph{Let  $A$, $B$, and  $C$  be subsets of some universal set  $U$\!.  Then all of the following equalities hold}.

%\vskip2pt
\noindent
\BeginTable
\def\L{\JustLeft}
\BeginFormat
| p(1.7in) | p(1.25in) | p(1.25in) |
\EndFormat
" \emph{Properties of the Empty Set
\index{empty set!properties}%
 and the Universal Set}
\index{universal set!properties} " %  
$A \cap \emptyset  = \emptyset$ \\    
$A \cup \emptyset  = A$ 
" 
$A \cap U = A$ \\       
$A \cup U = U$ 
" \\
\EndTable

%\vskip2pt
\noindent
\BeginTable
\def\L{\JustLeft}
\BeginFormat
| p(1.7in) | p(1.25in) | p(1.25in) |
\EndFormat
"\emph{Idempotent Laws} " $A \cap A = A$ " $A \cup A = A$ " \\
\EndTable
\index{idempotent laws for sets}%

%\vskip6pt
\noindent
\BeginTable
\def\L{\JustLeft}
\BeginFormat
| p(1.7in) | p(1.25in) | p(1.25in) |
\EndFormat
"\emph{Commutative Laws} " $A \cap B = B \cap A$ " $A \cup B = B \cup A$ " \\
\EndTable
\index{commutative laws!for sets}%

%\vskip6pt
\noindent
\BeginTable
\def\L{\JustLeft}
\BeginFormat
| p(1.7in) | p(2.5in) |
\EndFormat"\emph{Associative Laws} "  \L $\left( {A \cap B} \right) \cap C = A \cap \left( {B \cap C} \right)$ " \\
"                  "  \L $\left( {A \cup B} \right) \cup C = A \cup \left( {B \cup C} \right) $ " \\
\EndTable
\index{associative laws!for sets}%

%\vskip6pt
\noindent
\BeginTable
\def\L{\JustLeft}
\BeginFormat
| p(1.7in) | p(2.5in) |
\EndFormat"\emph{Distributive Laws} "  \L $A \cap \left( {B \cup C} \right) = \left( {A \cap B} \right) \cup \left( {A \cap C} \right)$ " \\
"                  "  \L $A \cup \left( {B \cap C} \right) = \left( {A \cup B} \right) \cap \left( {A \cup C} \right)$ " \\
\EndTable
\index{distributive laws!for sets}%

\item \textbf{Theorem~\ref{T:propsofcomplements}}.
\emph{Let  $A$  and  $B$  be subsets of some universal set  $U$.  Then the following are true}:

\noindent
\BeginTable
\BeginFormat
| p(1.7in) |  p(2.2in) |
\EndFormat
"\emph{Basic Properties}    "  $\left( A^c \right)^c = A$ " \\+20
"                    "  $A - B = A \cap B^c$   " \\+03
\EndTable

\noindent
\BeginTable
\BeginFormat
| p(1.7in) |  p(2.2in) |
\EndFormat
"\emph{Empty Set, Universal Set}      "  $A - \emptyset = A$ and $A - U = \emptyset$ " \\
"                    "  ${\emptyset}^c = U$ and $U^c = \emptyset$ " \\+02
\EndTable

\noindent
\BeginTable
\BeginFormat
| p(1.7in) |  p(2.2in) |
\EndFormat
"\emph{De Morgan's Laws}
\index{De Morgan's Laws!for sets}%
    "  $\left ({A \cap B} \right)^c = A^c \cup B^c$ " \\
"                    "  $\left ({A \cup B} \right)^c = A^c \cap B^c$ " \\+02
\EndTable

\noindent
\BeginTable
\BeginFormat
| p(1.7in) |  p(2.2in) |
\EndFormat
"\emph{Subsets and Complements}        "  $A \subseteq B$ if and only if $B^c \subseteq A^c$. " \\
\EndTable
%\newpage
\item \textbf{Theorem~\ref{T:propsofcartprod}}.
\emph{Let  $A$, $B$, and  $C$  be sets.  Then}
\begin{enumerate}
\item $A \times \left( {B \cap C} \right) = \left( {A \times B} \right) \cap \left( {A \times C} \right)$
 
\item $A \times \left( {B \cup C} \right) = \left( {A \times B} \right) \cup \left( {A \times C} \right)$

\item $\left( {A \cap B} \right) \times C = \left( {A \times C} \right) \cap \left( {B \times C} \right)$ 

\item $\left( {A \cup B} \right) \times C = \left( {A \times C} \right) \cup \left( {B \times C} \right)$ 

\item $A \times \left( {B - C} \right) = \left( {A \times B} \right) - \left( {A \times C} \right)$ 

\item $\left( {A - B} \right) \times C = \left( {A \times C} \right) - \left( {B \times C} \right)$ 

\item If  $T \subseteq A$, then  $T \times B \subseteq A \times B$. 

\item If  $Y \subseteq B$, then  $A \times Y \subseteq A \times B$. 
\end{enumerate}


\item \textbf{Theorem \ref{T:indexproperties}}.
\emph{Let $\Lambda$ be a nonempty indexing set and let 
$\mathscr{A} = \left\{ A_\alpha \mid \alpha \in \Lambda \right\}$ be an indexed family of sets.  Then
\begin{enumerate}
\item For each $\beta \in \Lambda$, $\bigcap\limits_{\alpha \in \Lambda}^{}A_\alpha \subseteq A_\beta$. %\label{T:indexproperties1}
\item For each $\beta \in \Lambda$, $A_\beta \subseteq \bigcap\limits_{\alpha \in \Lambda}^{}A_\alpha$. %\label{T:indexproperties2}
\item $\left(\bigcap\limits_{\alpha \in \Lambda}^{}A_\alpha \right)^c = 
\bigcup\limits_{\alpha \in \Lambda}^{}A_{\alpha}^c$ %\label{T:indexproperties3}
\item $\left(\bigcup\limits_{\alpha \in \Lambda}^{}A_\alpha \right)^c = 
\bigcap\limits_{\alpha \in \Lambda}^{}A_{\alpha}^c$ %\label{T:indexproperties4}
\end{enumerate}
Parts~(\ref{T:indexproperties3}) and~(\ref{T:indexproperties4}) are known as 
\textbf{De Morgan's Laws}.}

\item \textbf{Theorem~\ref{T:distributeindex}}.
\emph{Let $\Lambda$ be a nonempty indexing set, let 
$\mathscr{A} = \left\{ A_\alpha \mid \alpha \in \Lambda \right\}$ be an indexed family of sets, and let $B$ be a set.  Then
\begin{enumerate}
\item $B \cap \left(\:\bigcup\limits_{\alpha \in \Lambda}^{}A_{\alpha} \right) 
= \bigcup\limits_{\alpha \in \Lambda}^{} \left( B \cap A_{\alpha} \right)$, and
\item $B \cup \left(\:\bigcap\limits_{\alpha \in \Lambda}^{}A_{\alpha} \right) 
= \bigcap\limits_{\alpha \in \Lambda}^{} \left( B \cup A_{\alpha} \right)$.
\end{enumerate}}

\end{itemize}
\hbreak

\subsection*{Important Proof Method}

\noindent
\textbf{The Choose-an-Element Method} \hfill \\
\index{choose an element method}%
The choose-an-element method is frequently used when we encounter a universal quantifier in a statement in the backward process of a proof.  This statement often has the form

\begin{center}
For each element with a given property, something happens.
\end{center}
In the forward process of the proof, we then we choose an arbitrary element with the given property.  
\vskip6pt
\begin{center}
\parbox{4in}{\emph{Whenever we choose an arbitrary element with a given property, we are not selecting a specific element.  Rather, the only thing we can assume about the element is the given property.}}
\end{center}
For more information, see page~\pageref{SS:choosemethod}.
\hbreak
\input blankpage
\endinput
