\newpage
\section{Chapter \ref{C:induction} Summary}

\subsection*{Important Definitions}
\begin{multicols}{2}
\begin{itemize}
\item Inductive set, page~\pageref*{D:inductiveset}
\item Factorial, page~\pageref*{factorial2}
\item Recursive definition, page~\pageref*{recursivedef}
\item Fibonacci numbers, page~\pageref*{fibonacci}
\item Geometric sequence, page~\pageref*{geomseq}
\item Geometric series, page~\pageref*{geometricseries}
\end{itemize}
\end{multicols}
\hbreak



\subsection*{The Various Forms of Mathematical Induction} 
\label{SS:induction}
\begin{enumerate}
\item \textbf{The Principle of Mathematical Induction}\\  
\index{Principle of Mathematical Induction}%
\index{mathematical induction!Principle}%
If  $T$  is a subset of  $\mathbb{N}$ such that
\begin{enumerate}
  \item $1 \in T$\!, and   
  \item For every  $k \in \mathbb{N}$, if  $k \in T$\!, then  $\left( {k + 1} \right) \in T$\!,
\end{enumerate}
then  $T = \mathbb{N}$.

%\pagebreak
\noindent
\textbf{Procedure for a Proof by Mathematical Induction}\\  
To prove  $\left( {\forall n \in \mathbb{N}} \right)\left( {P( n )} \right)$

\begin{tabular}{r p{3in}}
Basis step:        &    Prove  $P( 1 )$.  \\
                   &                                 \\
Inductive step:    &  Prove that for each  $k \in \mathbb{N}$, if  $P( k )$ is true, then  
$P( {k + 1} )$ is true. \\
\end{tabular}

\item \textbf{The Extended Principle of Mathematical Induction}\\  
\index{Extended Principle of Mathematical Induction}%
\index{mathematical induction!Extended Principle}%
Let  $M$  be an integer.  If  $T$  is a subset of  $\mathbb{Z}$ such that
\begin{enumerate}
  \item $M \in T$, and
  \item For every  $k \in \mathbb{Z}$ with $k \geq M$, if  $k \in T$, then  $\left( {k + 1} \right) \in T$,
\end{enumerate}
then  $T$  contains all integers greater than or equal to  $M$\!.  
%That is, $\left\{ n \in \mathbb{Z}  \mid n \geq M \right\} \subseteq T$.

\textbf{Using the Extended Principle of Mathematical Induction} \\
Let  $M$  be an integer.  To prove $\left( {\forall n \in \mathbb{Z} \text{ with } n \geq M} \right)\left( {P( n )} \right)$

\begin{tabular}{r p{3in}}
 Basis step:    &  Prove  $P( M )$. \\
                &                              \\
Inductive step: &	 Prove that for every  $k \in \mathbb{Z} $ with $k \geq M$, if  $P( k )$ is true, then  $P( {k + 1})$ is true. \\
\end{tabular}
%\vskip10pt

We can then conclude that  $P( n )$ is true for all  $n \in \mathbb{Z}$ with  
$n \geq M$\!.

\item \textbf{The Second Principle of Mathematical Induction} \\
\index{Second Principle of Mathematical Induction}%
\index{mathematical induction!Extended Principle}%
Let  $M$  be an integer.  If  $T$  is a subset of  $\mathbb{Z}$ such that
\begin{enumerate}
  \item $M \in T$\!, and

  \item For every  $k \in \mathbb{Z}$ with $k \geq M$\!, if  
$\left\{ {M, M + 1, \ldots ,k} \right\} \subseteq T$\!, then  $\left( {k + 1} \right) \in T$\!,
\end{enumerate}

then  $T$  contains all integers greater than or equal to  $M$\!.  
%That is,  $\left\{ { {n \in \mathbb{Z} } \mid n \geq M} \right\} \subseteq T$.

\textbf{Using the Second Principle of Mathematical Induction} \\
Let  $M$  be an integer.  To prove 
$\left( {\forall n \in \mathbb{Z} \text{ with } n \geq M} \right)\left( {P( n )} \right)$

\begin{tabular}{r p{3in}}
Basis step:    &  Prove  $P( M )$. \\
                &                              \\
Inductive step: &	 Let  $k \in \mathbb{Z}$ with  $k \geq M$.  Prove that if $P( M ),P( {M + 1} ), \ldots ,P( k )$ are true, then  $P( {k + 1} )$ is true. \\
\end{tabular}

We can then conclude that  $P( n )$ is true for all  $n \in \mathbb{Z}$ with  $n \geq M$.
\end{enumerate}
\hbreak

\subsection*{Important Results}

\begin{itemize}
\item \textbf{Theorem \ref{T:primefactors}}.  
\emph{Each natural number greater than 1 either is a prime number or is a product of prime numbers.}

%\item \textbf{Theorem \ref{P:powersetcardinality}}.  
%\emph{Let  $n$  be a nonnegative integer and let  $A$  be a subset of some universal set.  If  $A$  is a finite set with  $n$  elements, then  $A$  has  $2^n $ subsets.  That is,  if  $\left| A \right| = n$, then  $\left| {\mathcal{P}\left( A \right)} \right| = 2^n $.}

\item \textbf{Theorem \ref{P:geometricsequence}}.  
\emph{Let  $a, r \in \mathbb{R}$.  If a geometric sequence is defined by  $a_1  = a$ and for each  $n \in \mathbb{N}$,  $a_{n + 1}  = r \cdot a_n $, then for each  $n \in \mathbb{N}$,   $a_n  = a \cdot r^{n - 1} $.}

\item \textbf{Theorem \ref{P:geometricseries}}.  
\emph{Let  $a, r \in \mathbb{R}$.  If the sequence  $S_1 ,S_2 , \ldots ,S_n , \ldots $ is defined by  $S_1  = a$ and for each  $n \in \mathbb{N}$,  $S_{n + 1}  = a + r \cdot S_n $, then for each  $n \in \mathbb{N}$,   $S_n  = a + a \cdot r + a \cdot r^2  +  \cdots  + a \cdot r^{n - 1} $.  That is, the geometric series  $S_n $ is the sum of the first  $n$  terms of the corresponding geometric sequence.}

\item \textbf{Theorem \ref{P:geometricseries2}}.  
\emph{Let  $a, r \in \mathbb{R}$ and  $r \ne 1$.  If the sequence  $S_1 ,S_2 , \ldots ,S_n , \ldots $ is defined by  $S_1  = a$ and for each  $n \in \mathbb{N}$,  $S_{n + 1}  = a + r \cdot S_n $, then for each  $n \in \mathbb{N}$, $S_n  = a\left( {\dfrac{{1 - r^n }}{{1 - r}}} \right)$.}

\end{itemize}








\endinput
