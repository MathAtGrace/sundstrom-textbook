\section{Chapter \ref{C:numbertheory} Summary}


\subsection*{Important Definitions}
%\begin{multicols}{2}
\begin{itemize}
\item Greatest common divisor of two integers, page~\pageref*{gcd}
\item Linear combination of two integers, page~\pageref*{linearcombination}
\item Prime number, page~\pageref*{def:prime}
\item Composite number, page~\pageref*{def:prime}
\item Prime factorization, page~\pageref*{def:primefactorization}
\item Relatively prime integers, page~\pageref*{relativelyprime}
\item Diophantine equation, page~\pageref*{diophantineequation}
\item Linear Diophantine equation in two variables, page~\pageref*{lineardiophantine}
\end{itemize}
%\end{multicols}
%\hbreak



\subsection*{Important Theorems and Results about Relations, Equivalence Relations, and Equivalence Classes} 
\begin{itemize}
\item \textbf{Theorem~\ref{T:euclidalgo}}.
\emph{Let  $a$  and  $b$  be integers with $a \ne 0$  and  $b > 0$.  Then  
$\gcd ( {a, b} )$ is the only natural number  $d$  such that
\renewcommand{\labelenumi}{(\textbf{\alph{enumi}})}
\renewcommand{\theenumi}{({\alph{enumi}})}
\begin{enumerate}
\item $d$ divides  $a$, \label{T:euclidalgo1}
\item $d$ divides  $b$, and  \label{T:euclidalgo2}
\item if  $k$  is an integer that divides both $a$  and  $b$, then  $k$  divides  $d$\!.  
\end{enumerate}}

\item \textbf{Theorem~\ref{T:gcdaslincomb}}.
\emph{Let  $a$  and  $b$  be integers, not both  0.  Then  $\gcd ( {a, b} )$
 can be written as a linear combination of  $a$  and  $b$.  That is, there exist integers  $u$  and  $v$  such that  $\gcd ( {a, b} ) = au + bv$.}



%\item \textbf{Theorem~\ref{T:gcddivideslincombs}}.
%\emph{Let  $a, b \in \mathbb{Z}$, not both 0.  The greatest common divisor of $a$ and $b$ divides every linear combination of $a$ and $b$ and is the least positive integer that is a linear combination of $a$ and $b$.}
%Let  $a, b \in \mathbb{Z}$, not both 0.  %The greatest common divisor of $a$ and $b$ divides every linear combination of $a$ and $b$ and is the least positive integer that is a linear combination of $a$ and $b$.
\item \textbf{Theorem~\ref{T:gcddivideslincombs}}.
\begin{enumerate}
\item \emph{The greatest common divisor,  $d$, is a linear combination of  $a$  and  $b$.  That is, there exist integers  $m$  and  $n$  such that  $d = am + bn$}.

\item \emph{The greatest common divisor,  $d$,  divides every linear combination of  $a$  and  $b$.  That is, for all  $x, y \in \mathbb{Z}$\,,  $d \mid \left( {ax + by} \right)$}.

\item \emph{The greatest common divisor, $d$, is the smallest positive number that is a linear combination of $a$ and $b$}.
\end{enumerate}



\item \textbf{Theorem~\ref{T:relativelyprime}}.
\emph{Let  $a$  and  $b$  be nonzero integers, and let  $p$  be a  prime number.}

\begin{enumerate}
\item \emph{If  $a$  and  $b$  are relatively prime, then there exist integers  $m$  and  $n$  such that  $am + bn = 1$.  That is,  1  can be written as linear combination of  $a$  and  $b$.}

\item \emph{If  $p \mid a$, then  $\gcd ( {a, p} ) = p$.}

\item \emph{If  $p$  does not divide  $a$, then  $\gcd ( {a, p} ) = 1$.} 
\end{enumerate}

\item \textbf{Theorem~\ref{T:relativelyprimeprop}}
\emph{Let  $a$, $b$, and  $c$ be integers.  If  $a$  and  $b$  are relatively prime  and  
$a \mid ( {bc} )$, then  $a \mid c$.}

\item \textbf{Corollary~\ref{C:primedivides}}
\begin{enumerate}
\item \emph{Let  $a, b \in \mathbb{Z}$, and let  $p$  be a prime number.  If  
$p \mid ( {ab} )$, then  $p \mid a$  or  $p \mid b$.}  

\item \emph{Let  $p$  be a prime number, let  $n \in \mathbb{N}$, and let  
$a_1 ,a_2 , \ldots , a_n  \in \mathbb{Z}$.  If  
$p \mid \left( {a_1 a_2  \cdots a_n } \right)$, then there exists a natural number $k$ 
 with  $1 \leq k \leq n$ such that  $p \mid a_k $.}
\end{enumerate}

\item \textbf{Theorem~\ref{T:fundtheorem}, The Fundamental Theorem of Arithmetic} 
\begin{enumerate}
\item \emph{Each natural number greater than 1 is either a prime number or is a product of prime numbers.}

\item \emph{Let   $n \in \mathbb{N}$ with  $n > 1$.  Assume that
\[
n = p_1 p_2  \cdots p_r \text{  and that  }n = q_1 q_2  \cdots q_s,
\]
where  $p_1 , p_2 ,  \ldots, p_r $ and  $q_1 , q_2 ,  \ldots, q_s $ are primes with  
$p_1  \leq p_2  \leq  \cdots  \leq p_r $ and  
$q_1  \leq q_2  \leq  \dots  \leq q_s $.  Then  $r = s$, and for each  $j$  from  1  to  r,  $p_j  = q_j $.}
\end{enumerate}

\item \textbf{Theorem~\ref{T:infiniteprimes}}.
\emph{There are infinitely many prime numbers.}

\item \textbf{Theorem~\ref{T:lindioph2}}.
\emph{Let $a$, $b$, and $c$ be integers with $a \ne 0$ and $b \ne 0$, and let 
$d = gcd ( a,b )$.}

\begin{enumerate}
\item \emph{If $d$ does not divide $c$, then the linear Diophantine equation $ax + by = c$ has no solution.}

\item \emph{If $d$ divides $c$, then the linear Diophantine equation $ax + by = c$ has infinitely many solutions.  In addition, if  $( x_0, y_0 )$ is a particular solution of this equation, then all the solutions of the equation are given by
\[
x = x_0 + \frac{b}{d} k  \quad \text{and} \quad y = y_0 - \frac{a}{d} k,
\]
where $k \in \mathbb{Z}$.}
\end{enumerate}

\item \textbf{Corollary~\ref{C:lindioph2}}.
\emph{Let $a$, $b$, and $c$ be integers with $a \ne 0$ and $b \ne 0$.  If $a$ and $b$ are relatively prime, then the linear Diophantine equation $ax + by = c$ has infinitely many solutions.  In addition, if  $x_0$, $y_0$ is a particular solution of this equation, then all the solutions of the equation are given by
\[
x = x_0 + b k  \quad \text{and} \quad y = y_0 - a k,
\]
where $k \in \mathbb{Z}$.}



\end{itemize}
\hbreak

\endinput
