%Note:  The first part of this chapter summary is at the end of the exercises for Section 1.1.   This was done to make the chapter summary start right after the exercises rather than starting on a new page.

\section{Chapter \ref{C:intro} Summary} \label{Su:intro}

\subsection*{Important Definitions}
\begin{multicols}{2}
\begin{itemize}
\item Statement, page~\pageref{D:prop}
\item Conditional statement, pages~\pageref{PA:prop}, ~\pageref{D:conditional}
\item Even integer, page~\pageref{D:even}
\item Odd integer, page~\pageref{D:even}
\item Pythagorean triple, page~\pageref{exer:pythag}
\end{itemize}
\end{multicols}

\subsection*{Important Number Systems and Their Properties} 
\begin{itemize}
\item The natural numbers, $\N$; the integers, $\Z$; the rational numbers, $\Q$; and the real numbers, $\R$.  See page~\pageref{sym:reals}.
\item Closure Properties of the Number Systems
$$
\BeginTable
\BeginFormat
| p(1.5in) | p(2.7in) |
\EndFormat
\_
| \textbf{Number System}   |   \textbf{Closed Under} | \\+22 \_
| Natural Numbers, $\N$    |  addition and multiplication | \\ \_1
| Integers, $\Z$           |  addition, subtraction, and multiplication | \\ \_1
| \Lower{Rational Numbers, $\Q$}   |  addition, subtraction, multiplication, and division by nonzero rational numbers | \\ \_1
| \Lower{Real Numbers, $\R$}   |  addition, subtraction, multiplication, and division by nonzero real numbers | \\ \_
\EndTable
$$
\item Inverse, commutative, associative, and distributive properties of the real numbers.  See page~\pageref{Ta:propertiesofreals}.
\end{itemize}
%\begin{itemize}
%\item The real numbers, $\R$, page~\pageref{sym:reals}
%\item The rational numbers, $\Q$, page~\pageref{sym:rationals}
%\item The natural numbers, $\N$, page~\pageref{sym:natural}
%\item The integers, $\Z$, page~\pageref{sym:integers}
%\end{itemize}

\subsection*{Important Theorems and Results}
%\begin{theorem}[Properties of Even and Odd Integers] \label{T:evenodd} \hfill
\index{even integer!properties}%
\index{odd integer!properties}%

%
\begin{itemize}
\item \textbf{Exercise~(\ref{exer:nextint}), Section~\ref{S:direct}} \\
\emph{
If $m$ is an even integer, then $m+1$ is an odd integer. \\ 
If $m$ is an odd integer, then $m+1$ is an even integer}.

\item \textbf{Exercise~(\ref{exer:integeradd}), Section~\ref{S:direct}} \\
\emph{
If $x$ is an even integer and $y$ is an even integer, then $x+y$ is an even integer.  \\
If $x$ is an even integer and $y$ is an odd integer, then $x+y$ is an odd integer. \\
If $x$ is an odd integer and $y$ is an odd integer, then $x+y$ is an even integer}.

\item \textbf{Exercise~(\ref{exer:integermult}),  Section~\ref{S:direct}}.  \emph{If $x$ is an even integer and $y$ is an integer, then $ x \cdot y $ is an even integer}.

\item \textbf{Theorem~\ref{T:xyodd}}.  \emph{If $x$ is an odd integer and $y$ is an odd integer, then $ x \cdot y $ is an odd integer}.

\item The \textbf{Pythagorean Theorem}, page~\pageref{pr:pythag}.  \emph{If $a$ and $b$ are the lengths of the legs of a right triangle and $c$ is the length of the hypotenuse, then $a^2 + b^2 = c^2$}. 
\end{itemize}
%\end{theorem}

\endinput
