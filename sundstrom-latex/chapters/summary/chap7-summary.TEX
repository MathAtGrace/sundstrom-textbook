\newpage
\section{Chapter \ref{C:equivrelations} Summary} \label{Su:equivrelations}
\subsection*{Important Definitions}
\begin{multicols}{2}
\begin{itemize}
\item Relation from $A$ to $B$, page~\pageref*{relation}
\item Relation on $A$, page~\pageref*{relation}
\item Domain of a relation, page~\pageref*{domrangeofrelation}
\item Range of a relation, page~\pageref*{domrangeofrelation}
\item Inverse of a relation, page~\pageref*{inverseofrelation}
\item Reflexive relation, page~\pageref*{ref-sym-trans}
\item Symmetric relation, page~\pageref*{ref-sym-trans}
\item Transitive relation, page~\pageref*{ref-sym-trans}
\item Equivalence relation, page~\pageref*{equivalencerelation}
\item Equivalence class, page~\pageref*{equivalenceclass}
\item Congruence class, page~\pageref*{congclass}
\item Partition of a set, page~\pageref*{partition}
\item Integers modulo $n$, page~\pageref*{integersmodn}
\item Addition in $\Z_n$, page~\pageref*{modulararithmetic}
\item Multiplication in $\Z_n$, page~\pageref*{modulararithmetic}
\end{itemize}
\end{multicols}
\hbreak



\subsection*{Important Theorems and Results about Relations, Equivalence Relations, and Equivalence Classes} 
\begin{itemize}
\item \textbf{Theorem~\ref{T:inverserelations}}.
\emph{Let  $R$  be a relation from the set  $A$  to the set  $B$.  Then}

\begin{enumerate}
\item \emph{The domain of  $R^{ - 1} $ is the range of  $R$.  That is, 
$\text{dom}( {R^{ - 1} } ) = \text{range}( R )$}.  \label{T:inverserelations1}

\item \emph{The range of  $R^{ - 1} $  is the domain of  $R$.   That is, 
$\text{range}( {R^{ - 1} } ) = \text{dom}( R )$}.  \label{T:inverserelations2}

\item \emph{The inverse of  $R^{ - 1} $  is  $R$.  That is, 
$\left( {R^{ - 1} } \right)^{ - 1}  = R$}.
\end{enumerate}

\item \textbf{Theorem~\ref{T:congruence-remainder}}.
\emph{Let  $n \in \mathbb{N}$ and let  $a, b \in \mathbb{Z}$.  Then 
$a \equiv b \pmod n$  if and only if  $a$  and  $b$  have the same remainder when divided by $n$.}

\item \textbf{Theorem~\ref{T:propsofequivclasses}}.  
\emph{Let  $A$  be a nonempty set and let  $\sim$  be an equivalence relation on  $A$}.  

\begin{enumerate}
\item \emph{For each  $a \in A$,  $a \in \left[ a \right]$}.  \label{T:propsofequivclasses1}

\item \emph{For each  $a, b \in A$, $a \sim b$  if and only if \,  
$\left[ a \right] = \left[ b \right]$}.  \label{T:propsofequivclasses2}

\item \emph{For each  $a, b \in A$,   $\left[ a \right] = \left[ b \right]$  or  
$\left[ a \right] \cap \left[ b \right] = \emptyset $}.  \label{T:propsofequivclasses3}
\end{enumerate}


%\pagebreak
\item \textbf{Corollary~\ref{C:propsofcongclasses}}.
\emph{Let  $n \in \mathbb{N}$.  For each  $a \in \mathbb{Z}$, let  $\left[ a \right]$ represent the congruence class of  $a$  modulo  $n$}.

\begin{enumerate}
\item \emph{For each  $a \in \mathbb{Z}$,  $a \in \left[ a \right]$}.

\item \emph{For each  $a, b \in \mathbb{Z}$, $ a \equiv  b \pmod n$ if and only if   $\left[ a \right] = \left[ b \right]$}.

\item \emph{For each  $a, b \in \mathbb{Z}$, $\left[ a \right] = \left[ b \right]$  or  
$\left[ a \right] \cap \left[ b \right] = \emptyset$}.
\end{enumerate}

\item \textbf{Corollary~\ref{C:propsofcongclasses2}}.
\emph{Let  $n \in \mathbb{N}$.  For each  $a \in \mathbb{Z}$, let  $\left[ a \right]$ represent the congruence class of  $a$  modulo  $n$}.

\begin{enumerate}
\item \emph{$\mathbb{Z} = \left[ 0 \right] \cup \left[ 1 \right] \cup \left[ 2 \right] \cup   \cdots  \cup  \left[ {n - 1} \right]$}

\item \emph{For $j, k \in \left\{ {0, 1, 2,  \ldots , n - 1} \right\}$, if  $j \ne k$, then  
$\left[ j \right] \cap \left[ k \right] = \emptyset$}.
\end{enumerate}



\item \textbf{Theorem~\ref{T:equivclasses-partition}}.
\emph{Let  $\sim$  be an equivalence relation on the nonempty set  $A$.  Then the  collection  $\mathcal{C}$  of all equivalence classes determined by  $\sim$  is a partition of the set  $A$}.
\end{itemize}
\hbreak

\endinput
