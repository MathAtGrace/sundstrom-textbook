\newpage
\section{Chapter \ref{C:topicsinsets} Summary} \label{Su:topicsinsets}

\subsection*{Important Definitions}
\begin{multicols}{2}
\begin{itemize}
\item Equivalent sets, page~\pageref*{equivsets}
\item Sets with the same cardinality, page~\pageref*{equivsets}
\item Finite set, page~\pageref*{cardinalityfinite}
\item Infinite set, page~\pageref*{cardinalityfinite}
\item Cardinality of a finite set, page~\pageref*{cardinalityfinite}
\item Cardinality of $\N$, page~\pageref*{aleph0}
\item $\aleph_0$, page~\pageref*{aleph0}
\item Countably infinite set, page~\pageref*{countinfinite}
\item Denumerable set, page~\pageref*{countinfinite}
\item Uncountable set, page~\pageref*{countinfinite}
\end{itemize}
\end{multicols}
\hbreak



\subsection*{Important Theorems and Results about Finite and Infinite Sets} 
\begin{itemize}
\item \textbf{Theorem~\ref{T:equivfinitesets}}.
\emph{Any set equivalent to a finite nonempty set $A$ is a finite set and has the same cardinality as $A$.}

\item \textbf{Theorem~\ref{T:finitesubsets}}.  
\emph{If $S$ is a finite set and $A$ is a subset of $S$, then $A$ is finite and 
$\text{card} ( A ) \leq \text{card} ( S )$.}

\item \textbf{Corollary~\ref{C:propersubsets}}.  
\emph{A finite set is not equivalent to any of its proper subsets.}

\item \textbf{Theorem~\ref{T:pigeonhole} [The Pigeonhole Principle]}.  
\emph{Let $A$ and $B$ be finite sets.  If $\text{card} ( A ) > \text{card} ( B )$, then any function $f\x A \to B$ is not an injection.}


\item \textbf{Theorem~\ref{T:subsetisinfinite}}.  
\emph{Let $A$ and $B$ be sets.
\begin{enumerate}
\item If $A$ is infinite and $A \approx B$, then $B$ is infinite.
\item If $A$ is infinite and $A \subseteq B$, then $B$ is infinite.
\end{enumerate}}


\item \textbf{Theorem~\ref{T:ZequivtoN}}.  
\emph{The set $\Z$ of integers is countably infinite, and so 
$\text{card} ( \Z ) = \aleph_0$.}


\item \textbf{Theorem~\ref{T:positiverationals}}. 
\emph{The set of positive rational numbers is countably infinite.}


\item \textbf{Theorem~\ref{T:addfinitetocountable}}.  
\emph{If $A$ is a countably infinite set and $B$ is a finite set, then $A \cup B$ is a countably infinite set.}


 
\item \textbf{Theorem~\ref{T:unionofcountable}}.  
\emph{If $A$ and $B$ are disjoint countably infinite sets, then $A \cup B$ is a countably infinite set.}


\item \textbf{Theorem~\ref{T:Qiscountable}}.  
\emph{The set $\mathbb{Q}$ of all rational numbers is countably infinite.}


\item \textbf{Theorem~\ref{T:subsetsofN}}.  
\emph{Every subset of the natural numbers is countable.}


\item \textbf{Corollary~\ref{C:subsetofcountable}}.  
\emph{Every subset of a countable set is countable.}


\item \textbf{Theorem~\ref{T:uncountableinterval}}.
\emph{The open interval $( 0, 1 )$ is an uncountable set.}


\item \textbf{Theorem~\ref{T:openintervals}}.
\emph{Let $a$ and $b$ be real numbers with $a < b$.  The open interval $( a, b )$ is uncountable and has cardinality $\boldsymbol{c}$.}


\item \textbf{Theorem~\ref{T:realsuncount}}.
\emph{The set of real numbers $\mathbb{R}$ is uncountable and has cardinality $\boldsymbol{c}$.}


\item \textbf{Theorem~\ref{T:cantor} [Cantor's Theorem]}.
\emph{For every set $A$, $A$ and $\mathcal{P} ( A )$ do not have the same cardinality.}


\item \textbf{Corollary~\ref{C:cantor}}.
\emph{$\mathcal{P} ( \mathbb{N} )$ is an infinite set that is not countably infinite.}


\item \textbf{Theorem~\ref{T:bernstein} [Cantor-Schr\"{o}der-Bernstein]}.
\emph{Let $A$ and $B$ be sets.  If there exist injections $f_1:A \to B$ and $f_2:B \to A$, then $A \approx B$.}





\end{itemize}

\hbreak

\endinput
