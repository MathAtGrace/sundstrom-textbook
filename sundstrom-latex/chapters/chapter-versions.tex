
\chapter{Version 3 and Version 2.1} \label{C:versions}
\markboth{}{}
There are only minor changes in Version 3 compared to Version 2.1.  Both versions should be able to be used in the same class.  Following is a summary of these changes.
\begin{enumerate}
  \item The main change is that the preview activities have been renamed and are now beginning activities.  This was done to emphasize that these activities are meant to be completed before starting the rest of the section and are not just a short preview of what is to come in the rest of the section.
  \item Two beginning activities have been added to Section 1.1.  (There are no preview activities in Section 1.1 of Version 2.1.)  Some minor changes were made in the rest of Chapter~\ref{C:intro} to make this chapter have the same number of pages in Version~3 as there are in Version~2.1.  So the content and pagination in the rest of the book is the same in both versions.
  \item (This is a technical note for instructors.)  There has been one change in notation dealing with functions involving congruences starting on page~\pageref{sub:functioncong} in Section~\ref{S:moreaboutfunctions} (dated before May 26, 2020).  In those copies of Version 2.1, we defined $\Z_n$ to be the set
\[
\Z_n = \{0, 1, 2, \ldots, n-1 \}
\]
for each natural number $n$.  Unfortunately, this could be confused with the definition of the integers modulo $n$ in Section~\ref{S:modulararithmetic},  The standard notation for this set of congruence classes is $\Z_n$.  So the notation in Section~\ref{S:moreaboutfunctions} has been changed from $\Z_n$ to $R_n$, which represents the set of all possible remainders upon division by $n$ in the set of integers.  So starting in Section~\ref{S:moreaboutfunctions} and then for the rest of the book, we use the notation
\[
R_n = \{0, 1, 2, \ldots, n-1 \}.
\]

\end{enumerate}



\endinput

