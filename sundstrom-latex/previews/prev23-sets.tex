\begin{previewactivity}[\textbf{Sets and Set Notation}]\label{PA:sets} \hfill \\
The theory of sets is fundamental to mathematics in the sense that many areas of mathematics use set theory and its language and notation.  This language and notation must be understood if we are to communicate effectively in mathematics.  At this point, we will give a very brief introduction to some of the terminology used in set theory.  

A \textbf{set} is a well-defined collection of objects that can be thought of as a single entity itself.  For example, we can think of the set of integers that are greater than 4.  Even though we cannot write down all the integers that are in this set, it is still a perfectly well-defined set.  This means that if we are given a specific integer, we can tell whether or not it is in the set of integers greater than 4.  %So for this course, a \textbf{set} is simply a well-defined collection of objects.
%\begin{itemize}
%  \item If  $A$  is a set and  $y$  is one of the objects in the set $A$, we write  $y \in A$ 
%\label{sym:elementof}%
% and read this as ``$y$ is an element of  $A$''  or ``$y$ is a member of  $A$.''  For example, if $B$ is the set of all integers greater than 4, then we could write $5 \in B$ and $10 \in B$.
%  \item If an object  $z$  is not an element in the set  $A$, we write  $z \notin A$ 
%\label{sym:notelement}%
% and read this as ``$z$  is not an element of  $A$.''  For example, if $B$ is the set of all integers greater than 4, then we could write $-2 \notin B$ and $4 \notin B$.
%\end{itemize}



The most basic way of specifying the elements of a set is to list the elements of that set.  This works well when the set contains only a small number of objects.  The usual practice is to list these elements between braces.  For example, if the set  $C$  consists of the integer solutions of the equation  $x^2  = 9$, we would write
\[
C = \left\{ { - 3,3} \right\}.
\]

For larger sets, it is sometimes inconvenient to list all of the elements of the set.  In this case, we often list several of them and then write a series of three dots ($ \ldots $) to indicate that the pattern continues.  For example,
\[
D = \left\{ 1, 3, 5, 7, \ldots, 49 \right\}
\]
is the set of all odd natural numbers from 1 to 49, inclusive.


For some sets, it is not possible to list all of the elements of a set; we then list several of the elements in the set and again use a series of three dots ($ \ldots $) to indicate that the pattern continues.  For example, if  $F$  is the set of all even natural numbers, we could write
\[
F = \left\{ {2,4,6, \ldots } \right\}.
\]

We can also use the three dots before listing specific elements to indicate the pattern prior to those elements.  For example, if  $E$  is the set of all even integers, we could write
\[
E = \left\{ {\ldots -6, -4, -2, 0, 2,4,6, \ldots } \right\}.
\]
Listing the elements of a set inside braces is called the \textbf{roster method}
\index{roster method}%
\index{set!roster method}%
\label{sym:roster}%
 of specifying the elements of the set.  We will learn other ways of specifying the elements of a set later in this section.



\begin{enumerate}
  \item Use the roster method to specify the elements of each of the following sets:
\begin{enumerate}
\item The set of real numbers that are solutions of the equation $x^2 - 5x = 0$.
\item The set of natural numbers that are less than or equal to 10.
\item The set of integers that are greater than $-2$.
\end{enumerate}
  \item Each of the following sets is defined using the roster method.  For each set, determine four elements of the set other than the ones listed using the roster method.
\begin{multicols}{2}
$A = \{1, 4, 7, 10, \ldots \}$\\
$B = \{2, 4, 8, 16, \ldots \}$ \\
$C = \{ \ldots, -8, -6, -4, -2, 0 \}$ \\
$D = \{\ldots, -9, -6, -3, 0, 3, 6, 9, \ldots \}$
\end{multicols}
\end{enumerate}




%\begin{prog}[Using the Roster Method]\label{pr:roster} \hfill \\
%Let $S = \left\{ 1, 4, 7, 10, \ldots \right\}$ and let 
%$T = \left\{ 2, 4, 8, 16, \ldots \right\}$.  Determine four more elements in each set other than the ones used in specifying the sets with the roster method.
%\end{prog}
%\hbreak

%When working with a mathematical object, such as set, we usually need to define what we usually need to define when two of these objects are equal.
%
%\begin{defbox}{D:setequality}{Two sets, $A$ and $B$,  are \textbf{equal}
%\index{equal sets}%
%\index{set equality}%
%\index{set!equality}%
% when they have precisely the same elements.  In this case, we write  $A = B$.  If the sets  $C$  and  $D$  are not equal, we write  $C \ne D$.}
%\end{defbox}  
%Using this definition of set equality, we see that
%
%\begin{multicols}{2}
%\begin{itemize}
%\item $\left\{ {1, 3, 5} \right\} = \left\{ {3, 5, 1} \right\}$
%
%\item $\left\{ {4, 8, 12} \right\} = \left\{ {4, 4, 8, 12, 12} \right\}$
%
%\item $\left\{ {5, 10} \right\} = \left\{ {5, 10, 5} \right\}$
%
%\item $\left\{ {5, 10} \right\} \ne \left\{ {5, 10, 15} \right\}$
%\end{itemize}
%\end{multicols}
%
%In each of the first three examples, the two sets have exactly the same elements even though the elements may be repeated or written in a different order.

\end{previewactivity}
\hbreak
\endinput

