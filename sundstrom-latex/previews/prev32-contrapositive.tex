\begin{previewactivity}[\textbf{Using the Contrapositive}] \label{PA:attempt} \hfill \\
The following statement was proven in Exercise (\ref{exer:x2odd}) on page~\pageref{exer:x2odd} in Section \ref{S:direct}.

\begin{list}{}
  \item If  $n$  is an odd integer, then  $n^2$ is an odd integer.
\end{list}
\vskip10pt
%It is also a direct consequence of Theorem \ref{T:xyodd} on page \pageref{T:xyodd} in Section \ref{S:direct}.
%
%\begin{list}{}
%  \item If  $x$  and  $y$  are odd integers, then  $x \cdot y$ is an odd integer.
%\end{list}
%\vskip10pt
\noindent
Now consider the following proposition:
\begin{list}{}
  \item For each integer $n$, if  $n^2 $ is an odd integer, then  $n$  is an odd integer.
\end{list}
%
\begin{enumerate}
  \item After examining several examples, decide whether you think this proposition is true or false.

  \item Try completing the following know-show~table for a direct proof of this proposition.  The question is, ``Can we perform algebraic manipulations to get from the `know' portion of the table to the `show' portion of the table?''  Be careful with this!  Remember that we are working with integers and we want to make sure that we can end up with an integer  $q$  as stated in Step $Q1$.
\end{enumerate}
$$
\BeginTable
\def\C{\JustCenter}
\BeginFormat
|p(0.4in)|p(2in)|p(1.8in)|
\EndFormat
\_
 | \textbf{Step}  |  \textbf{Know}  |  \textbf{Reason}   |  \\+02 \_
 | $P$   |  $n^2$ is an odd integer.  |  Hypothesis  | \\ \_1
 | $P1$  |   $\left( \exists k \in \Z \right) \left( n^2 = 2k + 1 \right)$          |  Definition of ``odd integer''          |  \\ \_1
 | \C $\vdots$  |  \C $\vdots$                         | \C $\vdots$     |  \\ \_1
 | $Q1$    |  $\left( \exists q \in \Z \right) \left( n=2q+1 \right) $  |       | \\  \_1  
 | $Q$     | $n$ is an odd  integer.     |  Definition of ``odd integer''       |    \\ \_
 | \textbf{Step}  |  \textbf{Show}  |  \textbf{Reason}    | \\+20 \_
\EndTable
$$
Recall that the contrapositive of the conditional statement $P \to Q$ is the conditional statement $\mynot Q \to \mynot P$, which is logically equivalent to the original conditional statement.   %We have seen in Section~\ref{S:logequiv} that the contrapositive of a conditional statement is logically equivalent to the conditional statement.  
(It might be a good idea to review \typeu Activity~\ref*{PA:converse} from Section~\ref{S:logequiv} on page~\pageref*{PA:converse}.)  Consider the following proposition once again:
\begin{list}{}
  \item For each integer $n$, if  $n^2 $ is an odd integer, then  $n$  is an odd integer.
\end{list}
\setcounter{oldenumi}{\theenumi}
\begin{enumerate} \setcounter{enumi}{\theoldenumi}
    \item Write the contrapositive of this conditional statement.  Please note that ``not odd'' means ``even.''  (We have not proved this, but it can be proved using the Division Algorithm in Section~\ref{S:divalgo}.)
\label{PA:contrapositive2}%

  \item Complete a know-show~table for the contrapositive statement from 
Part~(\ref{PA:contrapositive2}).
\label{PA:contrapositive3}%

  \item By completing the proof in Part~(\ref{PA:contrapositive3}), have you proven the given proposition?  That is, have you proven that if  $n^2$  is an odd integer, then  $n$  is an odd integer?  Explain.
\end{enumerate}
\hbreak
\end{previewactivity}

\endinput
