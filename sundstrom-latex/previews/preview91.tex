%\begin{previewactivity}[Bijections for Finite Sets] \label{PA:bijections} \hfill

%For each of the following, construction a bijection from the first set to the second set or %explain why no such bijection exists.

%\begin{tabular}[t]{l l}
%\textbf{1}. $A = \left\{ 1, 2, 3 \right\}$ & $B = \left\{ a, b, c \right\}$ \\
%\textbf{2}. $C = \left\{ 1, 2, 3, 4 \right\}$ & $B = \left\{ a, b, c \right\}$ \\
%\textbf{3}. $X = \left\{ 1, 2, 3, \ldots, 10 \right\}$ & 
%$Y = \left\{ 57, 58, 59, \ldots, 66 \right\}$ \\
%\end{tabular}

%\end{previewactivity}
%\hbreak
%
\begin{previewactivity}[Equivalent Sets, Part 1]\label{PA:equivalentsets} \hfill
\begin{enumerate}
\item Let $f:A \to B$.  Complete each of the following:  (If necessary, review the material in Section~\ref{S:typesoffunctions}.)
\begin{enumerate}
\item The function $f$ is an injection provided that $\ldots .$
\item The function $f$ is not an injection provided that $\ldots .$
\item The function $f$ is a surjection provided that $\ldots .$
\item The function $f$ is not a surjection provided that $\ldots .$
\item The function $f$ is a bijection provided that $\ldots .$
\end{enumerate}
\end{enumerate}

\begin{defbox}{equivsets}{Let $A$ and $B$ be subsets of a universal set $U$.  The set $A$ is \textbf{equivalent} to the set $B$ 
\index{equivalent sets}%
provided that there exists a bijection from the set $A$ onto the set $B$.  In this case, we write 
$A \approx B$.  
\label{sym:AequivB}%

\vskip6pt
When $A \approx B$, we also say that the set $A$ is in \textbf{one-to-one correspondence} 
\index{one-to-one correspondence}%
with the set $B$ and that the set $A$ has the same \textbf{cardinality} 
\index{cardinality}%
as the set $B$.
}
\end{defbox}
\noindent
\note  When $A$ is not equivalent to $B$, we write $A \not \approx B$.

%\enlargethispage{\baselineskip}
\begin{enumerate} \addtocounter{enumi}{1}
\item For each of the following, use the definition of equivalent sets to determine if the first set is equivalent to the second set.

\begin{enumerate}
\item $A = \left\{ 1, 2, 3 \right\}$ and $B = \left\{ a, b, c \right\}$  

\item $C = \left\{ 1, 2 \right\}$ and $B = \left\{ a, b, c \right\}$  

\item $X = \left\{ 1, 2, 3, \ldots, 10 \right\}$   and 
$Y = \left\{ 57, 58, 59, \ldots, 66 \right\}$
\end{enumerate}  

\item Let $D^+$ be the set of all odd natural numbers.   Prove that the function \linebreak
$f\x \mathbb{N} \to D^+$ defined by 
$f \left( x \right) = 2x - 1$, for all $x \in \mathbb{N}$,  is a bijection and hence that $\mathbb{N} \approx D^+$. 
\label{PA:equivalentsets5}%

%\item Let $r$ be an real number with $r > 0$.  Let $\left( 0, 1 \right)$ and 
%$\left( 0, r \right)$ be the open intervals from 0 to 1 and 0 to $r$, respectively, and let  
%$g: \left( 0, 1 \right) \to \left( 0, r \right)$ by $g \left( x \right) = rx$, for all $x \in \mathbb{R}$.  Prove that the function $g$ is a bijection and hence that 
%$\left( 0, 1 \right) \approx \left( 0, r \right)$. \label{PA:equivalentsets6}

\item Let $\R^+$ be the set of all positive real numbers.  Prove that the function 
\linebreak
$g\x \R \to \R^+$ defined by $g (x ) = e^x$, for all $x \in \R$ is a bijection and hence, that 
$\R \approx \R^+$.
\end{enumerate}
\end{previewactivity}
\hrule

\begin{previewactivity}[Equivalent Sets, Part 2]\label{PA:equivsets2} \hfill
\begin{enumerate}
\item Review Theorem~\ref{T:compositefunctions} in Section~\ref{S:compositionoffunctions}, Theorem~\ref{T:inversenotation} in Section~\ref{S:inversefunctions}, and 
Exercise~(\ref{exer:finversebijection}) in Section~\ref{S:inversefunctions}.

\item Review the definitions of a reflexive relation on a set, a symmetric relation, a transitive relation, and an equivalence relation on a set in Section~\ref{S:equivrelations}.

\item Let $A$, $B$, and $C$ be subsets of some universal set $U$.  That is, $A$, $B$, and $C$ are elements of the power set of $U$, which is denoted by 
$\mathcal{P} \left( U \right)$.  Prove that equivalence of sets $\left( \approx \right)$ is an equivalence relation on $\mathcal{P}( U )$ by proving each of the following three propositions.

\begin{enumerate}
\item For all $A \in \mathcal{P}( U )$, $A \approx A$.  That is, $\approx$ is a reflexive relation on $\mathcal{P} \left( U \right)$.

\item For all $A, B \in \mathcal{P}( U )$, if $A \approx B$, then 
$B \approx A$.  That is, $\approx$ is a symmetric relation.

\item For all $A, B, C \in \mathcal{P}( U )$, if $A \approx B$ and 
$B \approx C$, then $A \approx C$.  That is,  $\approx$ is a transitive relation.
\end{enumerate}
\end{enumerate}
\end{previewactivity}
\hbreak

%\begin{previewactivity}[Cardinality of a Finite Set] \label{PA:cardinalityfinite} \hfill

%\end{previewactivity}
\endinput
