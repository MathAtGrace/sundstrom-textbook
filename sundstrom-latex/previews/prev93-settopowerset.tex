\begin{previewactivity}[\textbf{Functions from a Set to Its Power Set}]\label{PA:powerset} \hfill \\
Let $A$ be a set.  In Section~\ref{S:setoperations}, we defined the \textbf{power set}
\index{power set}% 
$\mathcal{P} ( A )$ of $A$ to be the set of all subsets of $A$.  This means that
\begin{center}
$X \in \mathcal{P} ( A )$ if and only if $X \subseteq A$.
\end{center}
%
Theorem~\ref{T:powerset} in Section~\ref{S:setoperations} states that if a set $A$ has $n$ elements, then $A$ has $2^n$ subsets or that $\mathcal{P} ( A )$ has 
$2^n$ elements.  Using our current notation for cardinality, this means that
%
\begin{center}
if $\text{card} ( A ) = n$, then 
$\text{card} ( \mathcal{P} ( A ) ) = 2^n$.
\end{center}
%
(The proof of this theorem was Exercise~(\ref{exer:powerset}) on page~\pageref{exer:powerset}.)
%\begin{enumerate}
%\item Determine the power set of each of the following sets:
%\begin{multicols}{2}
%\begin{enumerate}
%\item $A = \left\{a, b \right\}$
%\item $B = \left\{a, b, c \right\}$
%\end{enumerate}
%\end{multicols}
%\end{enumerate}
%

We are now going to define and explore some functions from a set $A$ to its power set 
$\mathcal{P} ( A )$.  This means that the input of the function will be an element of $A$ and the output of the function will be a subset of $A$.

\begin{enumerate} 
\item Let $A = \left\{1, 2, 3, 4 \right\}$.  Define $f\x A \to \mathcal{P} ( A )$ by
%
\begin{multicols}{2}
$f ( 1 ) = \left\{ 1, 2, 3 \right\}$

$f ( 2 ) = \left\{ 1, 3, 4 \right\}$

$f ( 3 ) = \left\{ 1, 4 \right\}$

$f ( 4 ) = \left\{ 2, 4 \right\}$.
\end{multicols}
%
\begin{enumerate}
\item Is $1 \in f ( 1 )$?  Is $2 \in f ( 2 )$? Is 
$3 \in f ( 3 )$?  Is $4 \in f ( 4 )$?

\item Determine $S = \left\{ x \in A \mid x \notin f ( x ) \right\}$.

\item Notice that $S \in \mathcal{P} ( A )$.  Does there exist an element $t$ in $A$ such that $f ( t ) = S$?  That is, is $S \in \text{range} ( f )$?
 
\end{enumerate}

\item Let $A = \left\{1, 2, 3, 4 \right\}$.  Define $f\x A \to \mathcal{P} ( A )$ by 
%
\begin{center}
$f ( x ) = A - \left\{ x \right\}$ for each $x \in A$.
\end{center}
%
\begin{multicols}{2}
\begin{enumerate}
\item Determine $f ( 1 )$.  Is $1 \in f ( 1 )$?
\item Determine $f ( 2 )$.  Is $2 \in f ( 2 )$?
\item Determine $f ( 3 )$.  Is $3 \in f ( 3 )$?
\item Determine $f ( 4 )$.  Is $4 \in f ( 4 )$?
\end{enumerate}
\end{multicols}
%
\begin{enumerate} \setcounter{enumii}{4}
\item Determine $S = \left\{ x \in A \mid x \notin f ( x ) \right\}$.

\item Notice that $S \in \mathcal{P} ( A )$.  Does there exist an element $t$ in $A$ such that $f ( t ) = S$?  That is, is $S \in \text{range} ( f )$?
\end{enumerate}
%
\item Define $f\x \mathbb{N} \to \mathcal{P} ( \mathbb{N} )$ by
%
\begin{center}
$f ( n ) = \mathbb{N} - \left\{n^2, n^2-2n \right\}$, for each $n \in \mathbb{N}$.
\end{center}
%
\begin{enumerate}
\item Determine $f ( 1 )$, $f ( 2 )$, $f ( 3 )$, and 
$f ( 4 )$.  In each of these cases, determine if $k \in f ( k )$.
%
\item Prove that if $n > 3$, then $n \in f ( n )$.  \hint  Prove that if $n >3$, then 
$n^2 > n$ and $n^2 - 2n >n$.

\item Determine $S = \left\{ x \in \mathbb{N} \mid x \notin f ( x ) \right\}$.

\item Notice that $S \in \mathcal{P} ( \mathbb{N} )$.  Does there exist an element $t$ in $\mathbb{N}$ such that $f ( t ) = S$?  That is, is 
$S \in \text{range} ( f )$?

\end{enumerate}
\end{enumerate}
\end{previewactivity}
%\hbreak

\endinput


