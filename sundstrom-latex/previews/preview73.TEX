\begin{previewactivity}[\textbf{Sets Associated with a Relation}] \label{PA:setswithrelation} \hfill

Let  $A = \left\{ {a, b, c, d, e} \right\}$, and  let 
\[
\begin{aligned}
R &= \left\{ {( {a, a} ), ( {b, b} ), ( {c, c} ), ( {d, d} ), ( {e, e} ), ( {a, b} ), ( {b, a} ), ( {c, d} ), ( {d, c} )} \right\}; \\
S &= \left\{ { ( {b, b} ), ( {c, c} ), ( {d, d} ), ( {e, e} ), ( {a, b} ), ( {b, c} ), ( {a, d} ), ( {c, d} ), ( {d, c} )} \right\}. \\
\end{aligned}
\]

\begin{enumerate}
\item Draw a digraph that represents the relation  $R$  on  $A$.  Explain why  $R$  is an equivalence relation on  $A$.

\item Draw a digraph that represents the relation  $S$  on  $A$.  Explain why  $S$  is not an equivalence relation on  $A$.
\end{enumerate}
For each $y \in A$, define the following subsets of   $A$:
\[
\begin{aligned}
  R[ y ] &= \left\{ { {x \in A } \mid x \mathrel{R} y} \right\} = \left\{ { {x \in A } \mid \left( {x, y} \right) \in R} \right\}\!, \text{ and } \\
  S[ y ] &= \left\{ { {x \in A } \mid x \mathrel{S} y} \right\} = \left\{ { {x \in A } \mid \left( {x, y} \right) \in S} \right\}\!. \\ 
\end{aligned}
\]
%For example, $R\left[ a \right] = \left\{ a, b \right\}$. 
%since $(a, a) \in R$ and $(b, a) \in R$, and there is no other $y \in A$ such that $(y, a) \in R$. 
\begin{enumerate} \setcounter{enumi}{2} \item \begin{enumerate}

\item Determine  $R[ a ]$, $R[ b ]$, $R[ c ]$, 
      $R[ d ]$, and $R[ e ]$.

\item Which of the sets  $R[ a ]$, $R[ b ]$, $R[ c ]$, 
      $R[ d ]$,  and $R[ e ]$ are equal?

\item Which of the sets  $R[ a ]$, $R[ b ]$, $R[ c ]$, 
      $R[ d ]$,  and $R[ e ]$ are disjoint?
\end{enumerate}

\item \begin{enumerate}
\item Determine  $S[ a ]$, $S[ b ]$, $S[ c ]$, 
      $S[ d ]$, and $S[ e ]$.

\item Which of the sets  $S[ a ]$, $S[ b ]$, $S[ c ]$, 
      $S[ d ]$, and $S[ e ]$ are equal?

\item Which of the sets  $S[ a ]$, $S[ b ]$, $S[ c ]$, 
      $S[ d ]$, and $S[ e ]$ are disjoint?
\end{enumerate}

\end{enumerate}

\end{previewactivity}
\hbreak
%
\begin{previewactivity}[\textbf{Congruence Modulo 3}]\label{PA:congruencemodulo3} \hfill
%\enlargethispage{\baselineskip}
\begin{enumerate}
\item Use the roster method to specify each of the following sets:

\begin{enumerate}
\item The set  $C[ 0 ]$ of all integers  $a$  that are congruent to 0 modulo 3.   That is, 
\label{PA:congruencemodulo3-1}%
$C[ 0 ] = \left\{ { {a \in \mathbb{Z} } \mid a \equiv 0 \pmod 3} \right\}\!.$
%
\item The set  $C[ 1 ]$ of all integers  $a$  that are congruent to 1 modulo 3.   That is, 
\label{PA:congruencemodulo3-2}%
$C[ 1 ] = \left\{ { {a \in \mathbb{Z} } \mid a \equiv 1 \pmod 3} \right\}\!.$
%
\item The set  $C[ 2 ]$ of all integers  $a$  that are congruent to 2 modulo 3.   That is, 
\label{PA:congruencemodulo3-3}%
$C[ 2 ] = \left\{ { {a \in \mathbb{Z} } \mid a \equiv 2 \pmod 3} \right\}\!.$
%
\item The set  $C[ 3 ]$ of all integers  $a$  that are congruent to 3 modulo 3.   That is, 
$C[ 3 ] = \left\{ { {a \in \mathbb{Z} } \mid a \equiv 3 \pmod 3} \right\}\!.$
\end{enumerate}

\item Consider the three sets, $C[ 0 ]$, $C[ 1 ]$,  and 
$C[ 2 ]$ in Parts~(\ref{PA:congruencemodulo3-1}), (\ref{PA:congruencemodulo3-2}), 
and~(\ref{PA:congruencemodulo3-3}).
\begin{enumerate}
  \item Determine the intersection of any two of these sets.  That is,  determine  
  $C[ 0 ] \cap C[ 1 ]$, $C[ 0 ] \cap C[ 2 ]$,        and  $C[ 1 ] \cap C[ 2 ]$. 

  \item Let  $n = 734$.  What is the remainder when  $n$ is divided by 3?  Which of the three sets, if any,  contains  $n = 734$?  
\label{PA:congruencemodulo3-5b}%

  \item Repeat Part~(\ref{PA:congruencemodulo3-5b}) for  $n = 79$ and for $n=-79$.

  %\item Repeat Part~(\ref{PA:congruencemodulo3-5b}) for  $n =  - 79$.

  \item Do you think that  
        $C[ 0 ] \cup C[ 1 ] \cup C[ 2 ] = \mathbb{Z}$?  Explain.
\end{enumerate}

\end{enumerate}
\end{previewactivity}
\hbreak


\endinput
