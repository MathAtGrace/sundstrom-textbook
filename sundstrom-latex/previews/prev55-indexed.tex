\begin{previewactivity}[\textbf{An Indexed Family of Sets}] \label{PA:indexfamily} \hfill \\
We often use subscripts to identify sets.  For example, in 
\typeu Activity~\ref*{PA:foursets}, instead of using $A$, $B$, $C$, and $D$ as the names of the sets, we could have used $A_1$, $A_2$, $A_3$, and $A_4$.  When we do this, we are using the subscript as an identifying tag, or index, for each set.  We can also use this idea to specify an infinite family of sets.  For example, for each natural number $n$, we define
\[
C_n = \left\{ n, n+1, n+2, n+3, n+4 \right\}\!.
\]
So if we have a family of sets $\mathscr{C} = \left\{ C_1, C_2, C_3, C_4 \right\}$, we use the notation 
$\bigcup\limits_{j=1}^{4}C_j$ to mean the same thing as 
$\bigcup\limits_{X \in \mathscr{C}}^{}X$\!.
\begin{enumerate}
\item Determine $\bigcup\limits_{j=1}^{4}C_j$  and  $\bigcap\limits_{j=1}^{4}C_j$\!.
\end{enumerate}
We can see that with the use of subscripts, we do not even have to define the family of sets 
$\mathscr{A}$.  We can work with the infinite family of sets
\[
\mathscr{C}^* = \left\{ A_n \mid n \in \N \right\}
\]
and use the subscripts to indicate which sets to use in a union or an intersection.

\begin{enumerate} \setcounter{enumi}{1}
\item Use the roster method to specify each of the following pairs of sets.  The universal set is 
$\N$.
\begin{multicols}{2}
\begin{enumerate}
\item $\bigcup\limits_{j=1}^{6}C_j$  and  $\bigcap\limits_{j=1}^{6}C_j$
\item $\bigcup\limits_{j=1}^{8}C_j$  and  $\bigcap\limits_{j=1}^{8}C_j$
\item $\bigcup\limits_{j=4}^{8}C_j$  and  $\bigcap\limits_{j=4}^{8}C_j$
\item $\left( \bigcap\limits_{j=1}^{4}C_j \right)^c$  and  $\bigcup\limits_{j=1}^{4}C_j^c$
\end{enumerate}
\end{multicols}
\end{enumerate}
\end{previewactivity}
\hbreak

\endinput
