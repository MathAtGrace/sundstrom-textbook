\begin{previewactivity}[\textbf{Equivalent Sets, Part 2}]\label{PA:equivsets2} \hfill
\begin{enumerate}
\item Review Theorem~\ref{T:compositefunctions} in Section~\ref{S:compositionoffunctions}, Theorem~\ref{T:inversenotation} in Section~\ref{S:inversefunctions}, and 
Exercise~(\ref{exer:finversebijection}) in Section~\ref{S:inversefunctions}.

%\item Review the definitions of a reflexive relation on a set, a symmetric relation, a transitive relation, and an equivalence relation on a set in Section~\ref{S:equivrelations}.

\item Prove each part of the following theorem.
\begin{theorem} \label{T:equivsets} Let $A$, $B$, and $C$ be sets.

\begin{enumerate}
\item For each set $A$, $A \approx A$.  \label{T:equivsets1}

\item For all sets $A$ and $B$, if $A \approx B$, then 
$B \approx A$.  \label{T:equivsets2}

\item For all sets $A$, $B$, and $C$, if $A \approx B$ and 
$B \approx C$, then $A \approx C$.  \label{T:equivsets3}
\end{enumerate}
\end{theorem}
\end{enumerate}
%\textbf{Technical Note}:   The three properties we proved in this activity are very similar to the concepts of reflexive, symmetric, and transitive relations.  However, we do not consider equivalence of sets to be an equivalence relation on a set $U$ since an equivalence relation requires an underlying (universal) set 
%$U$.  In this case, our elements would be the sets $A$, $B$, and $C$, and these would then have to subsets of some universal set $W$ (elements of the power set of $W$).  For equivalence of sets, we are not requiring that the sets $A$, $B$, and $C$ be subsets of the same universal set.  So we do not use the term relation in regards to the equivalence of sets.  However, if $A$ and $B$ are sets and $A \equiv B$, then we often say that $A$ and $B$ are \textbf{equivalent sets}.
\end{previewactivity}
\hbreak

\endinput
