\begin{previewactivity}[Attempting a Proof]\label{PA:attempt} \hfill \\
The following statement was proven in Exercise (\ref{exer:x2odd}) on page~\pageref{exer:x2odd} in Section \ref{S:direct}.

\begin{list}{}
  \item If  $n$  is an odd integer, then  $n^2$ is an odd integer.
\end{list}
\vskip10pt
%It is also a direct consequence of Theorem \ref{T:xyodd} on page \pageref{T:xyodd} in Section \ref{S:direct}.
%
%\begin{list}{}
%  \item If  $x$  and  $y$  are odd integers, then  $x \cdot y$ is an odd integer.
%\end{list}
%\vskip10pt
\noindent
Now consider the following proposition:
%\begin{flushleft}
\begin{list}{}
  \item For each integer $n$, if  $n^2 $ is an odd integer, then  $n$  is an odd integer.
\end{list}
%\end{flushleft}
%
\begin{enumerate}
  \item After examining several examples, decide whether you think this proposition is true or false.

  \item Try completing the following know-show~table for a direct proof of this proposition.
\end{enumerate}
$$
\BeginTable
\def\C{\JustCenter}
\BeginFormat
|p(0.4in)|p(2in)|p(1.8in)|
\EndFormat
\_
 | \textbf{Step}  |  \textbf{Know}  |  \textbf{Reason}   |  \\+02 \_
 | $P$   |  $n^2$ is an odd integer.  |  Hypothesis  | \\ \_1
 | $P1$  |   $\left( \exists k \in \Z \right) \left( n^2 = 2k + 1 \right)$          |  Definition of ``odd integer''          |  \\ \_1
 | \C $\vdots$  |  \C $\vdots$                         | \C $\vdots$     |  \\ \_1
 | $Q1$    |  $\left( \exists q \in \Z \right) \left( n=2q+1 \right) $  |       | \\  \_1  
 | $Q$     | $n$ is an odd  integer.     |  Definition of ``odd integer''       |    \\ \_
 | \textbf{Step}  |  \textbf{Show}  |  \textbf{Reason}    | \\+20 \_
\EndTable
$$

%\begin{center}
%\begin{tabular}[h]{|p{0.4in}|p{2in}|p{1.8in}|}
%  \hline
%  \textbf{Step}  &  \textbf{Know}  &  \textbf{Reason}     \\ \hline
%  $P$   &  $n^2$ is an odd integer.  &  Hypothesis \\ \hline
%  $P1$  &   $\left( \exists k \in \Z \right) \left( n^2 = 2k + 1 \right)$          &  Definition of ``odd integer''            \\ \hline
%  \vdots  &  \vdots                         & \vdots      \\ \hline
%  $Q1$    &  $\left( \exists q \in \Z \right) \left( n=2q+1 \right) $         &             \\  \hline  
%  $Q$     & $n$ is an odd  integer.     &  Definition of ``odd integer''           \\ \hline
%  \textbf{Step}  &  \textbf{Show}  &  \textbf{Reason}     \\ \hline
%\end{tabular}
%\end{center}
The question is, ``Can we perform algebraic manipulations to get from the `know' portion of the table to the `show' portion of the table?''
Be careful with this!  Remember that we are working with integers and we want to make sure that we can end up with an integer  $q$  as stated in Step $Q1$.
\end{previewactivity}
\hbreak
%
\begin{previewactivity}[The Contrapositive]\label{PA:contrapositive}\index{contrapositive} \hfill
\begin{enumerate}
  \item Complete the following truth table.  What does this truth table prove?  
\label{PA:contrapositive1}%
$$
\BeginTable
\BeginFormat
|c|c|c|c|c|c|
\EndFormat
\_6
   |   $P$ | $Q$ \|6 $\mynot Q$  |  $\mynot P$  |  $P \to Q$  |  $\mynot Q \to \mynot P$  | \\+22 \_6
    |   T  |  T  \|6  |  |  |  |   \\ \_1
    |   T  |  F  \|6  |  |  |  |   \\ \_1
    |   F  |  T  \|6  |  |  |  |   \\ \_1
    |   F  |  F  \|6  |  |  |  |   \\ \_6
\EndTable
$$

%
%\begin{center}
%    \begin{tabular}{|c|c ||c|c|c|c|}
%     \hline
%      $P$ & $Q$ & $ \mynot  Q $ & $ \mynot P $  & $P \to Q $  &  $ \mynot  Q \to \mynot  P $ \\ \hline
%       T  &  T  &  &  &  &   \\ \hline
%       T  &  F  &  &  &  &  \\ \hline
%       F  &  T  &  &  &  &  \\ \hline
%       F  &  F  &  &  &  &   \\ \hline
%     \end{tabular}
%\end{center}
%
Consider the following proposition again:
 
  \begin{list}{}
    \item For each integer $n$, if $n^2$   is an odd integer, then  $n$  is an odd integer.
  \end{list}
%
  \item Write the contrapositive of this conditional statement in this proposition.  Remember that ``not odd'' means ``even.''
\label{PA:contrapositive2}%

  \item Complete a know-show~table for the contrapositive statement from 
Part~(\ref{PA:contrapositive2}).
\label{PA:contrapositive3}%

  \item By completing the proof in Part~(\ref{PA:contrapositive3}), have you proven the proposition from Part~(\ref{PA:contrapositive1})?  That is, have you proven that if  $n^2$  is an odd integer, then  $n$  is an odd integer?  Explain.
\end{enumerate}
\end{previewactivity}
\hbreak
%
\pagebreak
\begin{previewactivity}[A Biconditional Statement]\label{PA:biconditional} \hfill
\index{biconditional statement}%
\index{statement!biconditional}%
%
\begin{enumerate}
  %\item In Exercise (\ref{exer:bicond}) from Section \ref{S:logequiv}, we constructed a truth table to prove that the biconditional statement, $P \leftrightarrow Q$, is logically equivalent to  $\left( {P \to Q} \right) \wedge \left( {Q \to P} \right)$.  Complete this exercise if you have not already done so.  \label{PA:biconditional1}

  \item Recall that  
$P \leftrightarrow Q \equiv \left( {P \to Q} \right) \wedge \left( {Q \to P} \right)$.

Suppose that we want to prove a biconditional statement of the form  
\makebox{$P \leftrightarrow Q$}.  Explain a method for completing this proof based on this logical equivalency.
%in Part~(\ref{PA:biconditional1}).

  \item Let  $n$  be an integer.  Assume that we have completed the proofs of the following two statements:

  \begin{itemize}
    \item If  $n$  is an odd integer, then  $n^2 $ is an odd integer.
    \item If  $ n^2 $ is an odd integer, then  $n$  is an odd integer.
  \end{itemize}

(See Preview Activity \ref{PA:attempt} and Preview Activity \ref{PA:contrapositive}.)  Have we completed the proof of the following proposition?

  \begin{list}{}
    \item The integer  $n$  is an odd integer  if and only if  $ n^2 $ is an odd integer.
  \end{list}

Explain.
 
\end{enumerate}
\hbreak
\end{previewactivity}

\endinput
