%
 \begin{previewactivity}[\textbf{Statements}] \hfill \label{PA:prop}\\
Much of our work in mathematics deals with statements.  In mathematics, a \textbf{statement}
\label{D:prop}%
\index{statement}%
  is a declarative sentence that must have a definite truth value, either true or false but not both.    
A statement is sometimes called a \textbf{proposition}.
\index{proposition}%
The key is that there must be no ambiguity.  To be a statement, a sentence must be  true or false, and it cannot be both.  So a sentence such as ``The sky is beautiful'' is not a statement since whether the sentence is true or not is a matter of opinion.  A question such as ``Is it raining?'' is not a statement because it is a question and is not declaring or asserting that something is true.

Some sentences that are mathematical in nature often are not statements because we may not know precisely what a variable represents.  For example, the equation $2x + 5 = 10$ is not a statement since we do not know what $x$ represents.  If we substitute a specific value for $x$ (such as $x = 3$), then the resulting equation, $2 \cdot 3 + 5 = 10$ is a statement (which is a false statement). 

%\noindent
%In addition, a sentence such as,
%\begin{center}
%There exists a real number $x$ such that $2x + 5 = 10$,
%\end{center}
%is a statement since either such a real number $x$ exists or it does not.

\noindent
Which of the following sentences are statements?  Do not worry about determining the truth value of those that are statements; just determine whether each sentence is a statement or not.
\begin{multicols}{2}
\begin{enumerate}
\item $3 \cdot 4 + 7 = 19$.
\item $3 \cdot 5 + 7 = 19$.
\item $3x + 7 = 19$.
\end{enumerate}
\end{multicols}
\begin{enumerate} \setcounter{enumi}{3}
  \item There exists an integer $x$ such that $3x + 7 = 19$.
\item The derivative of $f(x) = \sin x$ is $f'(x) = \cos x$.
\item Does the equation $3x^2 - 5x - 7 = 0$ have two real number solutions?
\end{enumerate}

%\begin{multicols}{2}
%\begin{enumerate}
%\item $3 + 4 = 8$.
%
%\item $2 \cdot 7 + 8 = 22$.
%
%\item	$\left( {x - 1} \right) = \sqrt {x + 11}$.
%
%\item $2x + 5y = 7$.\label{PA:prop3}
%\end{enumerate}
%\end{multicols}
%\begin{enumerate} \setcounter{enumi}{4}
%\item There are integers  $x$  and  $y$  such that $2x + 5y = 7.$\label{PA:prop4}
%
%\item There are integers  $x$  and  $y$  such that $23x + 37y = 52.$
%
%\item If  $x$  and  $y$  are odd integers, then $x \cdot y$ is an odd integer.
%
%\item Given a line  $L$  and a point  $P$  not on that line, there is a unique line through  $P$  that does not intersect  $L$.
%
%\item $\left( {a + b} \right)^2  = a^2  + b^2.$\label{PA:prop8}
%
%\item $\left( {a + b} \right)^2  = a^2  + b^2$ for all real numbers  $a$  and  $b$.
%\label{PA:prop9}
%
%\item If $ABC$ is a right triangle with right angle at vertex $B$, and if $D$ is the midpoint of the hypotenuse, then the line segment connecting vertex $B$ to $D$ is half the length of the hypotenuse.
%
%%\item If you pick  $N$  distinct points on the circumference of a circle and draw line segments connecting them all with each other, then the interior of the circle will be divided into  
%%$2^{N - 1}$ portions.
%
%\item There do not exist three real numbers  $x$, $y$, and  $z$ such that \\
% $x^3  + y^3  = z^3.$
%
%\end{enumerate}
\hbreak
\end{previewactivity}
\endinput

%
\begin{previewactivity}[Conditional Statements]\label{PA:conditional} \hfill \\
Given statements $P$ and $Q$, a statement of the form ``If $P$ then $Q$'' is called a 
\textbf{conditional statement}.
\index{conditional statement}%
\index{statement!conditional}%
 It seems reasonable that the truth value (true or false) of the conditional statement 
``If $P$ then $Q$'' depends on the truth values of $P$  and  $Q$.  The statement ``If $P$ then $Q$'' means that $Q$  must be true whenever $P$ is true.  The statement $P$ is called the \textbf{hypothesis}
\index{conditional statement!hypothesis}%
 of the conditional statement, and the statement $Q$ is called the \textbf{conclusion}
\index{conditional statement!conclusion}%
 of the conditional statement.  We will now explore some examples.

\begin{enumerate}
\item ``If it is raining, then Laura is at the theater.''
Under what conditions is this conditional statement false?  For example,
\begin{enumerate}
\item Is it false if it is raining and Laura is at the theater?
\item Is it false if it is raining and Laura is not at the theater?
\item Is it false if it is not raining and Laura is at the theater?
\item Is it false if it is not raining and Laura is not at the theater?
\end{enumerate}

\item Which of the following conditional statements do you believe are true and which do you believe are false?
\begin{multicols}{2}
\begin{enumerate}
\item If $3 + 2 = 5$, then $5 < 8$.
\item If $3 + 2 = 5$, then $8 < 5$.
\item If $8 < 5$, then $3 + 2 = 5$.
\item If $8 < 5$, then $3 + 2 = 9$.
\end{enumerate}
\end{multicols}

\item Now consider the following sentence:
\begin{center}
If $x$ is a positive real number, then $x^2 + 8x$ is a positive real number.
\end{center}

Although the hypothesis and conclusion of this conditional sentence are not statements, the conditional sentence itself can be considered to be a statement as long as we know what possible numbers may be used for the variable $x$.  From the context of this sentence, it seems that we can substitute any real number for $x$.  (In Chapter~\ref{C:logic}, we will learn how to more careful and precise with these types of conditional statements.)

\begin{enumerate}
\item Notice that if $x = -3$, then $x^2 + 8x = -15$, which is negative.  Does this mean that the given conditional statement is false?

\item Notice that if $x = 4$, then $x^2 + 8x = 48$, which is positive.  Does this mean that the given conditional statement is true?

\item Do you think this conditional statement is true or false?  Record the results for at least five different examples where the hypothesis of this conditional statement is true.
\end{enumerate}

%\item ``If  $x$  and  $y$  are odd integers, then  $x \cdot y$ is an odd integer.''
%   \begin{enumerate}
%   \item Notice that  if  $x = 7$ and  $y = 2$, then  $x \cdot y = 14$.  So the statement that  $x \cdot y$ is odd is false in this case.  Does this mean that the given conditional statement is false?
%   \item Do you think this statement is true or false?  Try and record at least five different examples where the hypothesis of this conditional statement is true.
%   \end{enumerate}

\item ``If  $n$  is a positive integer, then $(n^2-n+41)$    is a prime number.''  (Remember that a prime number is a positive integer greater than 1 whose only positive factors are 1  and itself.) 
\label{PA:conditional3}%

Do you think this statement is true?  Record your results for $n=1$, $n=2$, $n=3$, 
$n=4$, $n=5$, and $n=10$.  Then record the result for at least four other values of  $n$.  

\end{enumerate}
\hbreak
\end{previewactivity}
\endinput

%\begin{previewactivity}[Solving an Equation] \label{PA:equation} \hfill
%
%The following are some steps that can be used to begin solving the equation  
%\[
%\ x - 1  = \sqrt {x + 11}, 
%\]
%where $x$ represents a real number.
%
%\begin{itemize}
%\item Square both sides of the equation.
%\item Expand the left side of the equation.
%\item Write the resulting equation in standard quadratic form \newline $\left( {ax^2  + bx + c = 0} \right).$
%\item Solve the resulting quadratic equation.
%\item Check the solutions of the quadratic equation in the original equation.
%\end{itemize}
%
%Write a description of how to solve this equation.  This description should be written for someone who already knows basic algebra and how to solve quadratic equations.
%\hbreak
%\end{previewactivity} 
%\newpage
