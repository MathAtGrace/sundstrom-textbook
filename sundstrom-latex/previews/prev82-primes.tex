\begin{previewactivity}[\textbf{Prime Factorizations}]\label{PA:primefactorization25} \hfill \\
Recall that a natural number  $p$  is  a \textbf{prime number}
\index{prime number}%
\label{def:prime}%
 provided that it is greater than 1 and the only natural numbers that divide  $p$  are  1  and  $p$.  A natural number other than 1 that is not a prime number is a \textbf{composite number}.
\index{composite number}%
  The number 1 is neither prime nor composite.  (See Exercise~\ref{exer:prime} from Section~\ref{S:quantifier} on page~\pageref{exer:prime}.)

\begin{enumerate}
\item Give examples of four natural numbers that are prime and four natural numbers that are composite.
\end{enumerate}

Theorem~\ref{T:primefactors} in Section~\ref{S:otherinduction} states that every natural number greater than  1  is either a prime number or a product of prime numbers.

When a composite number is written as a product of prime numbers, we say that we have obtained a \textbf{prime factorization}
\index{prime factorization}%
\label{def:primefactorization}%
 of that composite number.  For example, since $60 = 2^2 \cdot 3 \cdot 5$, we say that $2^2 \cdot 3 \cdot 5$ is a prime factorization of 60.

\begin{enumerate}
\setcounter{enumi}{1}
\item Write the number  40  as a product of prime numbers by first writing  $40 = 2 \cdot 20$
 and then factoring  20  into a product of primes.  Next, write the number  40  as a product of prime numbers by first writing  $40 = 5 \mspace{1mu}\cdot\mspace{1mu} 8$  and then factoring  8  into a product of primes.  
\label{PA:primefactorization25-2}%

\item In Part~(\ref{PA:primefactorization25-2}), we used two different methods to obtain a prime factorization of  40.  Did these methods produce the same prime factorization or different prime factorizations?  Explain.  
\label{PA:primefactorization25-3}%

\item Repeat Parts~(\ref{PA:primefactorization25-2}) and~(\ref{PA:primefactorization25-3}) with 150.  First, start with  
$150 = 3 \cdot 50$, and then start with  
$150 = 5 \cdot 30$.
\end{enumerate}
\end{previewactivity}

\hbreak
%


\endinput
