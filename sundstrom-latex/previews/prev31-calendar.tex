\begin{previewactivity}[\textbf{Calendars and Clocks}]\label{PA:calender} \hfill \\
This activity is intended to help with understanding the concept of congruence, which will be studied at the end of this section.
\begin{enumerate}
  \item Suppose that it is currently Tuesday.
\label{PA:calender1}%
  \begin{enumerate}
    \item What day will it be 3 days from now?
    \item What day will it be 10 days from now?
    \item What day will it be 17 days from now?  What day will it be 24 days from now?
    \item Find several other natural numbers  $x$  such that it will be Friday  $x$  days from now.
\label{PA:calender1d}%
    \item Create a list (in increasing order) of the numbers $3, 10, 17, 24$, and the numbers you generated in Part~(\ref{PA:calender1d}).  Pick any two numbers from this list and subtract one from the other. Repeat this several times.
\label{PA:calender1e}%
    \item What do the numbers you obtained in Part~(\ref{PA:calender1e}) have in common?
  \end{enumerate}

  \item Suppose that we are using a twelve-hour clock with no distinction between {\smallc a.m.} and {\smallc p.m.}  Also, suppose that the current time is 5:00.
  \begin{enumerate}
    \item What time will it be  4 hours from now?
    \item What time will it be 16 hours from now?  What time will it be 28 hours from now?
    \item Find several other natural numbers  $x$  such that it will be 9:00  $x$  hours from now.
\label{PA:calender2c}%
    \item Create a list (in increasing order) of the numbers $4, 16, 28$, and the numbers you generated in Part~(\ref{PA:calender2c}).  Pick any two numbers from this list and subtract one from the other. Repeat this several times.
\label{PA:calender2e}%
    \item What do the numbers you obtained in Part~(\ref{PA:calender2e}) have in common? 
  \end{enumerate}

  \item This is a continuation of Part~(\ref{PA:calender1}).  Suppose that it is currently Tuesday.
  \begin{enumerate}
    \item What day was it 4 days ago?
    \item What day was it 11 days ago?  What day was it 18 days ago?
    \item Find several other natural numbers  $x$  such that it was Friday  $x$  days ago.  \label{PA:calender3c}%
    \item Create a list (in increasing order) consisting of the numbers 
\linebreak
$-18, -11, -4$, the opposites of the numbers you generated in Part~(\ref{PA:calender3c}) and the positive numbers in the list from Part~(\ref{PA:calender1e}).  Pick any two numbers from this list and subtract one from the other.  Repeat this several times.
\label{PA:calender3d}%
    \item What do the numbers you obtained in Part~(\ref{PA:calender3d}) have in common?
  \end{enumerate}
%
\end{enumerate}
\end{previewactivity}
\hbreak


\endinput
