\begin{previewactivity}[\textbf{Introduction to Infinite Sets}]\label{PA:introtoinfinite} \hfill \\
In Section~\ref{S:finitesets}, we defined a \textbf{finite set}
\index{finite set}%
 to be the empty set or a set $A$ such that $A \approx \mathbb{N}_k$ for some natural number $k$.    We also defined an 
\textbf{infinite set} 
\index{infinite set}%
to be a set that is not finite, but the question now is, ``How do we know if a set is infinite?''  
One way to determine if a set is an infinite set is to use 
Corollary~\ref{C:propersubsets}, which states that a finite set is not equivalent to any of its subsets.  We can write this as a conditional statement as follows:

\begin{list}{}
\item If $A$ is a finite set, then $A$ is not equivalent to any of its proper subsets.
\end{list}
\noindent
or more formally as
\begin{list}{}
\item For each set $A$, if $A$ is a finite set, then for each proper subset $B$ of $A$, $A \not\approx B$.
\end{list}

\begin{enumerate}
\item Write the contrapositive of the preceding conditional statement.  Then explain how this statement can be used to determine if a set is infinite.

\item Let $D^+$ be the set of all odd natural numbers.  In \typeu Activity~\ref*{PA:equivalentsets} from Section~\ref{S:finitesets}, we proved that $\mathbb{N} \approx D^+$. 
\label{PA:introtoinfinite2}%

\begin{enumerate}
\item Use this to explain carefully why $\mathbb{N}$ is an infinite set.

\item Is $D^+$ a finite set or an infinite set?  Explain carefully  how you know.
\end{enumerate}

\item Let $b$ be a positive real number.  Let $( 0, 1 )$ and 
$( 0, b )$ be the open intervals from 0 to 1 and 0 to $b$, respectively.  In 
Part~(\ref{A:equivsets4}) of Progress Check~\ref{prog:equivsets} 
(on page~\pageref{prog:equivsets}), 
%Beginning Activity~\ref{PA:equivalentsets} from Section~\ref{S:finitesets}, 
we proved that 
$( 0, 1 ) \approx ( 0, b )$. 
\label{PA:introtoinfinite3}%

\begin{enumerate}
\item Use a value for $b$ where $0 < b < 1$ to explain why $( 0, 1 )$ is an infinite set.

\item Use a value for $b$ where $b > 1$ to explain why $( 0, b )$ is an infinite set.
\end{enumerate}

\end{enumerate}
\end{previewactivity}
%\hbreak

\endinput


