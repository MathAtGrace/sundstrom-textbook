%
\begin{previewactivity}[A Logical Equivalency]\label{PA:alogicalequiv} \hfill
\begin{enumerate}

\item Complete a truth table to show that  $\left( {P \vee Q} \right) \to R$
  is logically equivalent to  $\left( {P \to R} \right) \wedge \left( {Q \to R} \right)$.
%\begin{flushleft}
%\begin{tabular}[h]{|c|c|c||c|c||c|c|c|}
%  \hline
%        $P$  &  $Q$  &  $R$  &  $P \vee Q$  &  $\left( {P \vee Q} \right) \to R$ & $P \to Q$ & %$Q \to R$ & $\left( {P \to R} \right) \wedge \left( {Q \to R} \right)$  \\ \hline
%        T & T & T &  &  &  &  &    \\ \hline
%        T & T & F &  &  &  &  & \\ \hline
%        T & F & T &  &  &  &  & \\ \hline
%        T & F & F &  &  &  &  & \\ \hline
%        F & T & T &  &  &  &  & \\ \hline
%        F & T & F &  &  &  &  & \\ \hline
%        F & F & T &  &  &  &  & \\ \hline
%        F & F & F &  &  &  &  & \\ \hline
%\end{tabular}
%\end{flushleft}
%
\item Suppose that you are trying to prove a statement that is written in the form  
$\left( {P \vee Q} \right) \to R$.  Explain why you can complete this proof by writing separate and independent proofs of   $P \to R$ and   $Q \to R$.

\item Explain why the statement, ``If  $n$  is an integer, then  $n^2  + n$ is an even integer.''

is logically equivalent to

\begin{list}{}
  \item If  $n$  is an even integer, then  $n^2  + n$ is an even integer  and if   $n$  is an odd integer, then  $n^2  + n$ is an even integer.
\end{list}
\end{enumerate}
\end{previewactivity}
\hbreak
%
\begin{previewactivity}[A Property of the Integers]\label{PA:propintegers} \hfill
\begin{enumerate}
\item Complete the proof for the following proposition:

\textbf{Proposition 1:}  If  $n$  is an even integer, then  $n^2  + n$ is an even integer.

\textbf{\emph{Proof}}.  Let  $n$  be an even integer.  Then there exists an integer  $m$  such that $n = 2m$.  Substituting this into the expression  $n^2  + n$ yields \ldots .

\item Construct a proof for the following proposition:

\textbf{Proposition 2:}  If  $n$  is an odd integer, then  $n^2  + n$ is an even integer.

\item Assume that you have completed the proofs of  Proposition 1 and Proposition 2.   Do these two proofs constitute a proof of the following proposition?

\textbf{Proposition 3:}  If  $n$  is an integer, then  $n^2  + n$ is an even integer.
\end{enumerate}
\end{previewactivity}
\hbreak





\endinput
%
