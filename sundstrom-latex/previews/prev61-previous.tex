\begin{previewactivity}[\textbf{Functions from Previous Courses}] \label{PA:previousfunctions} \hfill \\
One of the most important concepts in modern mathematics is that of a \textbf{function}.  In previous mathematics courses, we have often thought of a function as some sort of input-output rule that assigns exactly one output to each input.  So in this context, a \textbf{function}
\index{function}%
 can be thought of as a procedure for associating with each element of some set, called the \textbf{domain of the function},
\index{domain!of a function}%
\index{function!domain}%
 exactly one element of another set, called the \textbf{codomain of the function}.
\index{codomain}%
\index{function!codomain}%
  This procedure can be considered an input-output rule.  The function takes the input, which is an element of the domain, and produces an output, which is an element of the codomain.  In calculus and precalculus, the inputs and outputs were almost always real numbers.  So the notation $f( x ) = x^2 \sin x$ means the following:

\begin{itemize}
\item $f$  is the name of the function.

\item $f( x )$  is a real number.  It is the output of the function when the input is the real number  $x$.  For example,
\[
\begin{aligned}
  f\left( {\frac{\pi }{2}} \right) &= \left( {\frac{\pi }{2}} \right)^2 \sin \left( {\frac{\pi }
{2}} \right) \\ 
                                   &= \frac{{\pi ^2 }}{4} \cdot 1 \\ 
                                   &= \frac{{\pi ^2 }}{4}. \\ 
\end{aligned}
\]
\end{itemize}
For this function, it is understood that the domain of the function is the set  $\R$ of all real numbers.  In this situation, we think of the domain as the set of all possible inputs.  That is, the domain is the set of all possible real numbers  $x$  for which a real number output can be determined.

This is closely related to the equation  
$y = x^2 \sin x $.  With this equation, we frequently think of  $x$  as the input and  $y$  as the output.  In fact, we sometimes write  $y = f( x )$.  The key to remember is that a function must have exactly one output for each input.  When we write an equation such as  
\[
y = \dfrac{1}{2}x^3  - 1,
\]
we can use this equation to define  $y$  as a function of  $x$.  This is because when we substitute a real number for  $x$  (the input), the equation produces exactly one real number for  $y$  (the output).  We can give this function a name, such as  $g$, and write
\[
y = g( x ) = \frac{1}{2}x^3  - 1.
\]
However, as written, an equation such as  
\[
y^2  = x + 3
\]
cannot be used to define  $y$  as a function of  $x$  since there are real numbers that can be substituted for  $x$  that will produce more than one possible value of  $y$.  For example,  if  $x = 1$, then $y^2  = 4$, and  $y$  could be  $-2$  or  2.

Which of the following equations can be used to define a function with  $x \in \R$
as the input and  $y \in \R$  as the output?

\begin{multicols}{2}
\begin{enumerate}

\item $y = x^2  - 2$

\item $y^2  = x + 3$

\item $y = \dfrac{1}{2}x^3  - 1$

\item $y = \dfrac{1}{2} x\sin x$

\item $x^2  + y^2  = 4$

\item $y = 2x - 1$

\item $y = \dfrac{x}{{x - 1}}$

\end{enumerate}
\end{multicols}
\end{previewactivity}
\hbreak

\endinput
