\begin{previewactivity}[Functions and Sets, Part 1] \label{PA:functionsandsets} \hfill \\
Let $S = \left\{ a, b, c, d \right\}$ and $T = \left\{ s, t, u \right\}$.  Define $f\x S \to T$ by
%\begin{multicols}{4}
%$f ( a ) = s$
%
%$f ( b ) = t$
%
%$f ( c ) = t$
%
%$f ( d ) = s$.
%\end{multicols}
$$
\BeginTable
\BeginFormat
| c | c | c | c |
\EndFormat
" $f ( a ) = s$ " $f ( b ) = t$ " $f ( c ) = t$ 
" $f ( d ) = s$. \\
\EndTable
$$

\begin{enumerate}
\item Let $A = \left\{ a, b ,c \right\}$ and $B = \left\{ a, d \right\}$.  Notice that $A$ and $B$ are subsets of $S$.  Use the roster method to describe the following two sets: 
\begin{multicols}{2}
\begin{enumerate}
\item $\left\{ f ( x ) \mid x \in A \right\}$  
\item $\left\{ f ( x ) \mid x \in B \right\}$
\end{enumerate}
\end{multicols}

\item Let $C = \left\{ s, t \right\}$ and $D = \left\{ s, u \right\}$.  Notice that $C$ and $D$ are subsets of $T$. Use the roster method to describe the following two sets:
\begin{multicols}{2}
\begin{enumerate}
\item $\left\{ x \in S \mid f ( x ) \in C \right\}$
\item $\left\{ x \in S \mid f ( x ) \in D \right\}$
\end{enumerate}
\end{multicols}
\end{enumerate}
\end{previewactivity}
\hbreak

\begin{previewactivity}[Functions and Sets, Part 2] \label{PA:functionsandsets2} \hfill \\
Let $f\x  \mathbb{R} \to \mathbb{R}$ be defined by $f ( x ) = x^2$ for all 
$x \in \mathbb{R}$.

\begin{enumerate}
%\item Determine $f ( 1 )$, $f ( 2 )$, $f ( 3 )$, and 
%$f ( -1 )$.

\item Let $A = \left\{ 1, 2, 3, -1 \right\}$.  Use the roster method to describe the set 
$\left\{ f ( x ) \mid x \in A \right\}$.

\item Use the roster method to describe each of the following sets:
\begin{multicols}{2}
\begin{enumerate}
\item $\left\{ x \in \R \mid f( x ) = 1 \right\}$
\item $\left\{ x \in \R \mid f( x ) = 9 \right\}$
\item $\left\{ x \in \R \mid f( x ) = 15 \right\}$
\item $\left\{ x \in \R \mid f( x ) = -1 \right\}$
\end{enumerate}
\end{multicols}
%\begin{enumerate}
%\item Find all $x \in \mathbb{R}$ such that $f ( x ) = 1$.
%\item Find all $x \in \mathbb{R}$ such that $f ( x ) = 4$.
%\item Find all $x \in \mathbb{R}$ such that $f ( x ) = 9$.
%\item Find all $x \in \mathbb{R}$ such that $f ( x ) = -1$.
%
%\end{enumerate}

\item Let $B = \left\{ 1, 9, 15, -1 \right\}$.  Use the roster method to describe the set \linebreak
$\left\{ x \in \mathbb{R} \mid f ( x ) \in B \right\}$.
\end{enumerate}
\end{previewactivity}
\hbreak

\begin{previewactivity}[Functions and Intervals] \label{PA:functionsandint} \hfill \\
Let $f\x \mathbb{R} \to \mathbb{R}$ be defined by $f ( x ) = x^2$ for all 
$x \in \mathbb{R}$.

\begin{enumerate}
\item We will first determine where $f$ maps the closed interval $\left[ 1, 2 \right]$.  That is, we will describe, in simpler terms, the set 
$\left\{ f ( x ) \mid x \in \left[ 1, 2 \right] \right\}$.  This is the set of all images of the real numbers in the closed interval $\left[ 1, 2 \right]$.

\begin{enumerate}
\item Draw a graph of the function $f$ using $-3 \leq x \leq 3$.

\item On the graph, draw the vertical lines $x = 1$ and $x = 2$ from the $x$-axis to the graph.   Label the points $P \!\left(1, f ( 1 ) \right)$ and 
$Q \!\left(2, f ( 2 ) \right)$ on the graph.

\item Now draw horizontal lines from the points $P$ and $Q$ to the $y$-axis.  Use this information from the graph to describe the set 
$\left\{ f ( x ) \mid x \in \left[ 1, 2 \right] \right\}$ in simpler terms.  Use interval notation or set builder notation.
\end{enumerate}

\item We will now determine all real numbers that map into the closed interval 
$\left[ 1, 4 \right]$.  That is, we will describe the set 
$\left\{ x \in \mathbb{R} \mid f ( x ) \in \left[ 1, 4 \right] \right\}$ in simpler terms.  This is the set of all preimages of the real numbers in the closed interval $\left[ 1, 4 \right]$.

\begin{enumerate}
\item Draw a graph of the function $f$ using $-3 \leq x \leq 3$.

\item On the graph, draw the horizontal lines $y = 1$ and $y = 4$ from the $y$-axis to the graph.   Label all points where these two lines intersect the graph.

\item Now draw vertical lines from the points in Part~(2) to the $x$-axis, and then use the resulting information  to describe the set \linebreak
$\left\{ x \in \mathbb{R} \mid f ( x ) \in \left[ 1, 4 \right] \right\}$ in simpler terms.  (You will need to describe this set as a union of two intervals.  Use interval notation or set builder notation.)
\end{enumerate}
\end{enumerate}
\end{previewactivity}
\hbreak

\endinput
