\begin{previewactivity}[\textbf{The GCD and the Division Algorithm}] \label{PA:gcdanddivalgo} \hfill \\
%\begin{enumerate}
%\item Write a complete statement of the Division Algorithm for integers.  (See 
%Section~\ref{S:divalgo}, page~\pageref{divalgorithm}.)
%\end{enumerate}
When we speak of the quotient and the remainder when we ``divide an integer  $a$  by the positive integer  $b$,'' we will always mean the quotient $q$    and the remainder  $r$  guaranteed by the Division Algorithm.  (See Section~\ref{S:divalgo}, page~\pageref{divalgorithm}.)

\begin{enumerate}
%\setcounter{enumi}{1}
\item Each row in the following table contains values for the integers  $a$  and  $b$.  In this table, the value of  $r$  is the remainder (from the Division Algorithm) when  $a$  is divided by  $b$.  Complete each row in this table by determining  $\gcd( {a, b} )$, $r$, and  
$\gcd( {b, r} )$.  
\label{PA:gcdanddivalgo2}%

%\begin{center}
%\begin{tabular}{| c | c | c |  c | c |} \hline
%$a$   &  $b$   &   $\gcd( {a, b} )$   &   Remainder  $r$   &  $\gcd( {b, r} )$ \\ \hline
%44   &   12  &  &  &  \\ \hline
%75   &   21	 &  &  &  \\ \hline
%50   &   33	 &  &  &  \\ \hline
%\end{tabular}
%\end{center}		
$$
\BeginTable
\BeginFormat
| c | c | c | c | c |
\EndFormat
\_
|  $a$  |  $b$  |  $\gcd(a, b)$  |  Remainder $r$  |  $\gcd(b,r)$ | \\+22 \_
|  44   |  12   |                |                 |              | \\+22 \_
|  75   |  21   |                |                 |              | \\+22 \_
|  50   |  33   |                |                 |              | \\+22 \_
\EndTable
$$
\item Formulate a conjecture based on the results of the table in Part~(\ref{PA:gcdanddivalgo2}).
\end{enumerate}
\end{previewactivity}
\hbreak



\endinput
