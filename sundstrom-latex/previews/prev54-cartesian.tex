\begin{previewactivity}[\textbf{The Cartesian Product of Two Sets}]\label{PA:cartesianproduct} \hfill \\
\label{prev54-cartesian}
In \typeu Activity~\ref*{PA:equationwith2vars}, we worked with ordered pairs without providing a formal definition of an ordered pair.  We instead relied on your previous work with ordered pairs, primarily from graphing equations with two variables.  Following is a formal definition of an ordered pair.

\begin{defbox}{orderedpair}{Let $A$ and $B$ be sets.  An \textbf{ordered pair}
\index{ordered pair}%
 (with first element from $A$ and second element from $B$) is a single pair of objects, denoted by  
$(a, b)$\label{sym:orderedpair}, with $a \in A$ and $b \in B$ and an implied order.  This means that for two ordered pairs to be equal, they must contain exactly the same objects in the same order.  That is, if 
$a, c \in A$ and $b, d \in B$, then  
\[
\left( {a,b} \right) = \left( {c,d} \right)  \text{ if and only if }  a = c \text{ and } b = d.  
\] 
The objects in the ordered pair are called the \textbf{coordinates}
\index{coordinates}%
 of the ordered pair.  In the ordered pair  $\left( {a,b} \right)$,  $a$  is the \textbf{first coordinate} and  $b$  is the \textbf{second coordinate}.}
\end{defbox}

We will now introduce a new set operation that gives a way of combining elements from two given sets to form ordered pairs.  The basic idea is that we will create a set of ordered pairs.
  
\begin{defbox}{cartesianproduct}{If  $A$  and  $B$  are sets, then the \textbf{Cartesian product},
\index{Cartesian product}%
 $A \times B$, of  $A$  and  $B$  is the set of all ordered pairs  $\left( {x,y} \right)$ where  $x \in A$ and  $y \in B$.  We use the notation $A \times B$ for the Cartesian product of $A$ and $B$, and using  set builder notation, we can write
%\[
\begin{center}
$A \times B = \left\{ { {\left( {x,y} \right)} \mid x \in A\text{ and }y \in B} \right\}.$
\end{center}
%\] 
\label{sym:cartprod}
We frequently read  $A \times B$ as  ``$A$  cross  $B$.''
In the case where the two sets are the same, we will write  $A^2 $ for  $A \times A$.  That is,
\[
A^2  = A \times A = \left\{ {\left( {a,b} \right) \mid a \in A\text{ and }b \in A} \right\}.
\]}
\end{defbox}

\noindent
Let $A = \left\{ {1,2,3} \right\}$ and  $B = \left\{ {a,b} \right\}$.

\begin{enumerate}
\item Is the ordered pair  $\left( {3,a} \right)$ in the Cartesian product  $A \times B$?  Explain.

\item Is the ordered pair  $\left( {3,a} \right)$ in the Cartesian product $A \times A$?  Explain.

\item Is the ordered pair  $\left( {3,1} \right)$ in the Cartesian product  $A \times A$?  Explain.

\item Use the roster method to specify all the elements of  $A \times B$.  (Remember that the elements of $A \times B$ will be ordered pairs.)

\item Use the roster method to specify all of the elements of the set  $A \times A = A^2$.

\item For any sets  $C$  and  $D$, explain carefully what it means to say that the ordered pair  $\left( {x,y} \right)$ is not in the Cartesian product  $C \times D$.

\end{enumerate}
\end{previewactivity}
\hbreak

\endinput
