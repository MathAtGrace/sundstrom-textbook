\begin{previewactivity}[\textbf{A Property of the Natural Numbers}] \label{PA:propertyofN} \hfill \\
Intuitively, the natural numbers begin with the number  1, and then there is 2, then 3, then 4, and so on.  Does this process of ``starting with 1'' and ``adding 1 repeatedly'' result in all the natural numbers?  We will use the concept of an inductive set to explore this idea in this activity.

\begin{defbox}{D:inductiveset}{A set  $T$  that is a subset of  $\mathbb{Z}$ is an 
\textbf{inductive set}
\index{inductive set}%
 provided that for each integer $k$, if $k \in T$, then  $k + 1 \in T$.}
\end{defbox}

%Consider the following property for a set   $T$  that is a subset of  $\mathbb{Z}$, the set of all integers.
%\begin{center}
%\textbf{Property I:} For every  $k \in \mathbb{Z}$, if  $k \in T$, then  $k + 1 \in T$.
%\end{center}

\begin{enumerate}
  \item Carefully explain what it means to say that a subset $T$ of the integers $\Z$ is not an inductive set.  This description should use an existential quantifier.

\item Use the definition of an inductive set to determine which of the following sets are inductive sets and which are not.  Do not worry about formal proofs, but if a set is  not inductive, be sure to provide a specific counterexample that proves it is not inductive.

\begin{multicols}{2}
\begin{enumerate}
\item $A = \left\{ {1,2,3, \ldots ,20} \right\}$

\item The set of natural numbers, $\mathbb{N}$

\item $B = \left\{ { {n \in \mathbb{N}} \mid n \geq 5} \right\}$
 
\item $S = \left\{ { {n \in \mathbb{Z}} \mid n \geq  - 3} \right\}$
 
\item $R = \left\{ { {n \in \mathbb{Z}} \mid n \leq  100} \right\}$

\item The set of integers, $\mathbb{Z}$
\item The set of odd natural numbers.
\end{enumerate}
\end{multicols}

\item This part will explore one of the underlying mathematical ideas for a proof by induction.  Assume that  $T \subseteq \mathbb{N}$ and assume that  $1 \in T$ and that  $T$ is an inductive set.  Use the definition of an inductive set to answer each of the following:  \label{PA:propertyofN6}

\begin{multicols}{2}
\begin{enumerate}
  \item Is  $2 \in T$?  Explain.
  \item Is  $3 \in T$?  Explain.
  \item Is  $4 \in T$?  Explain.
  \item Is  $100 \in T$?  Explain.
  \item Do you think that  $T = \mathbb{N}$?  Explain.
\end{enumerate}
\end{multicols}

\end{enumerate}
\end{previewactivity}
\hbreak
\endinput
