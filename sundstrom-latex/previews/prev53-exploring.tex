\begin{previewactivity}[\textbf{Exploring a Relationship between Two Sets}] \label{PA:workingwithvenn2} \hfill \\
Let $A$ and $B$ be subsets of some universal set $U$.

\begin{enumerate}
\item Draw two general Venn diagrams for the sets  $A$  and  $B$.  On one, shade the region that represents  $\left( {A \cup B} \right)^c $, and on the other, shade the region that represents  $A^c  \cap B^c $.  Explain carefully how you determined these regions. 
\label{PA:workingwithvenn2-1}%

\item Based on the Venn diagrams in Part~(\ref{PA:workingwithvenn2-1}), what appears to be the relationship between the sets   $\left( {A \cup B} \right)^c $ and   $A^c  \cap B^c $?

%\item Draw two general Venn diagrams for the sets  $A$  and  $B$.  On one, shade the region that represents  $\left( {A \cap B} \right)^c $, and on the other, shade the region that represents  $A^c \cup B^c $.  Explain carefully how you determined these regions.  \label{PA:workingwithvenn2-3}
%
%\item Based on the Venn diagrams in Part~(\ref{PA:workingwithvenn2-3}), what appears to be the relationship between the sets      $\left( {A \cap B} \right)^c $ and  $A^c \cup B^c $?

\end{enumerate}
%\end{previewactivity}
%\hbreak
%%
%\begin{previewactivity}[Working with Definitions] \label{PA:workingwithdef} \hfill \\
Some of the properties of set operations are closely related to some of the logical operators we studied in Section~\ref{S:logop}.  This is due to the fact that set intersection is defined using a conjunction (and), and set union is defined using a disjunction (or).  For example, if  $A$  and  $B$ are subsets of some universal set  $U$, then an element  $x$  is in  $A \cup B$  if and only if  $x \in A$  or  $x \in B$.


%consider one of De Morgan's Laws for statements  $P$  and  $Q$.
%\[
%\mynot \left( {P \vee Q} \right) \equiv \mynot P \wedge \mynot Q.
%\]
%
%Now, if  $A$  and  $B$ are subsets of some universal set  $U$, then an element  $x$  is in  $A \cup B$  if and only if  $x \in A$  or  $x \in B$.

\setcounter{oldenumi}{\theenumi}
\begin{enumerate} \setcounter{enumi}{\theoldenumi}
  \item Use one of De Morgan's Laws (Theorem~\ref{T:logequiv} on page~\pageref{T:logequiv}) to explain carefully what it means to say that an element  $x$  is not in  $A \cup B$. 
\label{part1PA1}
  \item What does it mean to say that an element  $x$  is in  $A^c $? What does it mean to say that an element  $x$  is in  $B^c $? 
  \item Explain carefully what it means to say that an element  $x$   is in  $A^c  \cap B^c $. \label{part3PA1}
  \item Compare your response in Part~(\ref{part1PA1}) to your response in Part~(\ref{part3PA1}).  Are they equivalent?  Explain.
  \item How do you think the sets  $\left( {A \cup B} \right)^c $ and $A^c  \cap B^c $ are related?  Is this consistent with the Venn diagrams from Part~(\ref{PA:workingwithvenn2-1})?
\end{enumerate}
\end{previewactivity}
\hbreak
%



\endinput
