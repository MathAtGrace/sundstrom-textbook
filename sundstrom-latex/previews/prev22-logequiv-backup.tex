\begin{previewactivity}[\textbf{Logically Equivalent Statements}]\label{PA:logequiv} \hfill \\
%We now define what we mean by logically equivalent statements.
In Exercises~(\ref{exer:sec22-5}) and~(\ref{exer:sec22-4}) from Section~\ref{S:logop}, we observed situations where two different statements have the same truth tables.  Basically, this means these statements are equivalent, and we make the following definition:
\begin{defbox}{D:logequiv}{Two expressions are \textbf{logically equivalent}
\index{logically equivalent}%
 provided that they have the same truth value for all possible combinations of truth values for all variables appearing in the two expressions.  In this case, we write  $X \equiv Y$  
\label{sym:logequiv}%
and say that  $X$  and  $Y$  are logically equivalent.} 
%(Another notation that is sometimes used to mean  ``$X$  is logically equivalent to  $Y$'' is 
%$X \Leftrightarrow Y$.)}
\end{defbox}

\begin{enumerate}
\item Complete truth tables for  $\mynot  \left( {P \wedge Q} \right)$  and  $\mynot  P \vee \mynot  Q$.


\item Are the expressions  $\mynot  \left( {P \wedge Q} \right)$  and   $\mynot  P \vee \mynot  Q$  logically equivalent?

\item Suppose that the statement ``I will play golf and I will mow the lawn'' is false.  Then its negation is true.  Write the negation of this statement in the form of a disjunction.  Does this make sense?
\end{enumerate}
\end{previewactivity}


%
%\begin{previewactivity}[Conditional Statements]\label{PA:conditional2} \hfill \\
Sometimes we actually use logical reasoning in our everyday living!  Perhaps you can imagine a parent making the following two statements.
\begin{center}
\begin{tabular}{l p{3in}}
Statement 1  &  If you do not clean your room, then you cannot watch TV. \\
Statement 2  &  You clean your room or you cannot watch TV. \\
\end{tabular}
\end{center}
%\noindent
%(Perhaps you can imagine a parent making these statements.)

\setcounter{oldenumi}{\theenumi}
\begin{enumerate} \setcounter{enumi}{\theoldenumi}
%  \item Do these two statements mean the same thing?  Explain.

  \item \label{PA:conditional2-2}%
   Let  $P$  be  ``you do not clean your room,'' and let  $Q$  be ``you cannot watch TV.''  Use these to translate Statement~1 and Statement~2 into symbolic forms. 

  \item Construct a truth table for each of the expressions you determined in 
Part~(\ref{PA:conditional2-2}).  Are the expressions logically equivalent? \label{PA:logequiv6}

  \item \label{PA:conditional2-4}%
     Assume that Statement~1 and Statement~2 are false.  In this case, what is the truth value of $P$ and what is the truth value of $Q$?  Now, write a true statement in symbolic form that is a conjunction and involves $P$ and $Q$.


  \item Write a truth table for the (conjunction) statement in Part~(\ref{PA:conditional2-4}) and compare it to a truth table for  $\mynot  \left( {P \to Q} \right)$.  What do you observe?
\end{enumerate}
%\end{previewactivity}
\hrule
\vskip6pt

\endinput

What would it mean to say that Statement \#1 is false?  What would it mean to say that Statement \#2 is false?  Write your answers to these questions in symbolic form using  $P$  and  $Q$. 
