\begin{previewactivity}[\textbf{Some Other Types of Functions}] \label{PA:otherfunctions} \hfill \\
The domain and codomain of each of the functions in \typeu Activity~\ref*{PA:previousfunctions} are the set  $\R$ of all real numbers, or some subset of   $\R$.  In most of these cases, the way in which the function associates elements of the domain with elements of the codomain is by a rule determined by some mathematical expression.  For example, when we say that $f$  is the function such that
\[
f( x ) = \dfrac{x}{{x - 1}},
\]
then the algebraic rule that determines the output of the function  $f$  when the input is  $x$  is  
$\dfrac{x}{{x - 1}}$.  In this case, we would say that the domain of  $f$  is the set of all real numbers not equal to  1 since division by zero is not defined.

However, the concept of a function is much more general than this.  The domain and codomain of a function can be any set, and the way in which a function associates elements of the domain with elements of the codomain can have many different forms. The input-output rule for a function can be a formula, a graph, a table, a random process, or a verbal description.  We will explore two different examples in this activity.

\begin{enumerate}
  \item Let  $b$  be the function that assigns to each person his or her birthday (month and day).  The domain of the function  $b$  is the set of all people and the codomain of  $b$  is the set of all days in a leap year (i.e., January 1 through December 31, including February 29).

\begin{enumerate}
\item Explain why  $b$  really is a function.  We will call this the \textbf{birthday function}.
\index{birthday function}%

\item In 1995, Andrew Wiles
\index{Wiles, Andrew}%
 became famous for publishing a proof of Fermat's Last Theorem.  (See A. D. Aczel, \textit{Fermat's Last Theorem: Unlocking the Secret of an Ancient Mathematical Problem}, Dell Publishing, New York, 1996.)  Andrew Wiles's birthday is April 11, 1953.  Translate this fact into functional notation using the ``birthday function'' $b$.  That is, fill in the spaces for the following question marks:
\[
b( {\,?\,} ) = \,?.
\]
\item Is the following statement true or false?  Explain.

\begin{list}{}
\item For each day  $D$  of the year, there exists a person  $x$  such that  
\linebreak $b( x ) = D$.
\end{list}

\item Is the following statement true or false?  Explain.

\begin{list}{}
\item For any people  $x$  and  $y$,  if  $x$  and  $y$  are different people, then  
\linebreak $b( x ) \ne b( y )$.
\end{list}
\end{enumerate}
  
\item Let  $s$  \label{sym:sumdivisors} be the function that associates with each natural number the sum of its distinct natural number divisors.  This is called the \textbf{sum of the divisors function}.  
\index{sum of divisors function}%
For example, the natural number divisors of 6 are 1, 2, 3, and 6, and so
\[
\begin{aligned}
  s( 6 ) &= 1 + 2 + 3 + 6 \\ 
                    &= 12. \\ 
\end{aligned} 
\]
\begin{enumerate}
\item Calculate  $s( k )$ for each natural number  $k$  from  1  through 15.

%\item Are the numbers $\sqrt{5}$, $\pi$, and $-6$ in the domain of the function $s$?  What is the domain of the function  $s$?

%\item Is  $s\left( {\sqrt 5 } \right)$  defined?  Explain.  Is  $s\left( \pi  \right)$  defined?  Is  $s\left( { - 6} \right)$  defined?

\item Does there exist a natural number  $n$  such that  $s( n ) = 5$?  Justify your conclusion.

\item Is it possible to find two different natural numbers  $m$  and  $n$  such that  \linebreak
$s( m ) = s( n )$?  Explain.

\item Use your responses in (b) and (c) to determine the truth value of each of the following statements.

\begin{enumerate}
  \item For each  $m \in \mathbb{N}$, there exists a natural number  $n$  such that  
$s( n ) = m$.

  \item For all $m, n \in \mathbb{N}$, if  $m \ne n$, then  $s( m ) \ne s( n )$.
\end{enumerate}

\end{enumerate}
\end{enumerate}

\end{previewactivity}
\hbreak

\endinput

