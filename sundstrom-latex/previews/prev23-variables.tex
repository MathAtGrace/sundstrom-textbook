\begin{previewactivity}[\textbf{Variables}] \label{PA:variable} \hfill \\
Not all mathematical sentences are statements.  For example, an equation such as
\[
x^2  - 5 = 0
\]
is not a statement.  In this sentence, the symbol  $x$  is a \textbf{variable}.   It represents a number that may be chosen from some specified set of numbers.  The sentence (equation) becomes true or false when a specific number is substituted for $x$.
%\item Not all mathematical sentences are statements.  For example, an equation such as
%\[
%x^2  - 5 = 0
%\]
%is not a statement.  In this sentence, the symbol  $x$  is a \textbf{variable}.   It represents a number that may be chosen from some specified set of numbers.  The sentence (equation) becomes true or false when a specific number is selected for $x$.
\begin{enumerate}
\item   \begin{enumerate}
    \item Does the equation $x^2 - 25 = 0$ become a true statement if $-5$ is substituted for $x$? 
    \item Does the equation $x^2 - 25 = 0$ become a true statement if $\sqrt{5}$ is substituted for $x$?
  \end{enumerate}

%\item Compare the following two sentences:
%  \begin{itemize}
%  \item $ \sqrt{x}$  is  a real number.
%  \item For each real number  $x$,  if  $x \geq 0$, then  $\sqrt{x}$ is a real number.
%  \end{itemize}

%What is the difference between these two sentences?
\end{enumerate}
%
\begin{defbox}{D:universal}{A \textbf{variable}
\index{variable}%
 is a symbol representing an unspecified object that can be chosen from a given set $U$.  The set $U$ is called the \textbf{universal set for the variable.}
\index{universal set}%
  It is the set of specified objects from which objects may be chosen to substitute for the variable.  A \textbf{constant}
\index{constant}%
 is a specific member of the universal set.}
\end{defbox}



%For this course, we will consider a \textbf{set} to be simply some well-defined collection of objects.  
Some sets that we will use frequently are the usual number systems.  Recall that we use the symbol  $\mathbb{R}$ to stand for the set of all \textbf{real numbers},
\index{real numbers}%
 the symbol  $\mathbb{Q}$ to stand for the set of all \textbf{rational numbers},
\index{rational numbers}%
 the symbol  $\mathbb{Z}$ to stand for the set of all \textbf{integers},
\index{integers}%
 and the symbol $\mathbb{N}$ 
\label{sym:naturals}%
 to stand for the set of all 
\textbf{natural numbers}.
\index{natural numbers}%
%\index{$\mathbb{N}$}%

%A \textbf{variable} 
%\index{variable}% 
%is a symbol representing an unspecified object that can be chosen from some specified set of objects.  This specified set of objects is agreed to in advance and is frequently called the \textbf{universal set}.
%\index{universal set}%

\setcounter{oldenumi}{\theenumi}
\begin{enumerate} \setcounter{enumi}{\theoldenumi}
  \item What real numbers will make the sentence ``$y^2 - 2y - 15 = 0$'' a true statement when substituted for  $y$?

  \item What natural numbers will make the sentence ``$y^2 - 2y - 15 = 0$'' a true statement when substituted for  $y$?
  %\item What integers will make the sentence ``$\sqrt{x}$  is a real number'' a true statement when substituted for $x$?

\item What real numbers will make the sentence ``$\sqrt{x}$  is a real number'' a true statement when substituted for $x$? \newpage

  \item What real numbers will make the sentence ``$\sin ^2 x + \cos ^2 x = 1$'' a true statement when substituted for  $x$?


  \item What natural numbers will make the sentence ``$\sqrt{n}$ is a natural number'' a true statement when substituted for $n$?

  \item What real numbers will make the sentence 
  \[
   \int_0^y {t^2 dt > 9}
  \]
   a true statement when substituted for  $y$?
\end{enumerate}
\end{previewactivity}
\hbreak
\endinput

