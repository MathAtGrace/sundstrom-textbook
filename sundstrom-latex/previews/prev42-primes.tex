\begin{previewactivity}[\textbf{Prime Factors of a Natural Number}] \label{PA:primefactors} \hfill \\
Recall that a natural number  $p$  is  a \textbf{prime number}
\index{prime number}%
 provided that it is greater than 1 and the only natural numbers that divide  $p$  are  1  and  $p$.  A natural number other than 1 that is not a prime number is a \textbf{composite number}.
\index{composite number}%
  The number 1 is neither prime nor composite.
\begin{enumerate}
\item Give examples of four natural numbers that are prime and four natural numbers that are composite.

\item Write each of the natural numbers  20, 40, 50, and 150  as a product of prime numbers.  \label{PA:primefactors2}

%\item Repeat Part~(\ref{PA:primefactors2}) using  50  and  150. \label{PA:primefactors3}

\item Do you think that any composite number can be written as a product of prime numbers?

\item Write a useful description of what it means to say that a natural number is a composite number (other than saying that it is not prime).

\item Based on your work in Part~(\ref{PA:primefactors2}), do you think it would be possible to use induction to prove that any composite number can be written as a product of prime numbers?
\end{enumerate}
\end{previewactivity}
\hbreak

\endinput
