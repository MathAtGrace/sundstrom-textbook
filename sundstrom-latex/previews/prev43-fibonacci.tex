\begin{previewactivity}[\textbf{The Fibonacci Numbers}] \label{PA:fibonaccinumbers} \hfill \\
The \textbf{Fibonacci numbers} \label{fibonacci}
\index{Fibonacci numbers}%
 are a sequence of natural numbers  $f_1 ,f_2 ,f_3 , \ldots ,f_n , \ldots $ 
\label{sym:fibonacci} defined recursively as follows:
\begin{itemize}
\item $f_1  = 1$ and  $f_2  = 1$, and

\item For each natural number  $n$,   $f_{n + 2}  = f_{n + 1}  + f_n $.
\end{itemize}
In words, the recursion formula states that for any natural number $n$ with $n \geq 3$, the $n^{th}$ Fibonacci number is the sum of the two previous Fibonacci numbers.  So we see that 
%
\begin{align*}
f_3  &= f_2  + f_1  = 1 + 1 = 2, \\
f_4  &= f_3  + f_2  = 2 + 1 = 3, \text{ and} \\
f_5  &= f_4  + f_3 = 3 + 2 = 5. \\
\end{align*}


%\begin{itemize}
%\item $f_3  = f_2  + f_1  = 1 + 1 = 2$,
%\item $f_4  = f_3  + f_2  = 2 + 1 = 3$, and
%\item $f_5  = f_4  + f_3 = 3 + 2 = 5$.
%\end{itemize}
%
\begin{enumerate}
\item Calculate  $f_6$ through $f_{20}$.

%\item Now calculate  $f_{11} $ through  $f_{20} $.

\item Which of the Fibonacci numbers $f_1$ through $f_{20}$ are even?  Which are multiples of 3?  
\item For $n = 2$, $n = 3$, $n = 4$, and $n = 5$, how is the sum of the first $(n - 1)$ Fibonacci numbers related to the $(n + 1)^{st}$ Fibonacci number?

\item Record any other observations about the values of the Fibonacci numbers or any patterns that you observe in the sequence of Fibonacci numbers.  If necessary, compute more Fibonacci numbers.

\end{enumerate}
\end{previewactivity}
\hbreak

\endinput
