\begin{previewactivity}[\textbf{Definition of Even and Odd Integers}]\label{PA:even} \hfill \\
Definitions play a very important role in mathematics.  A direct proof of a proposition in mathematics is often a demonstration that the proposition follows logically from certain definitions and previously proven propositions.  A \textbf{definition}
\index{definition}%
 is an agreement that a particular word or phrase will stand for some object, property, or other concept that we expect to refer to often.  In many elementary proofs, the answer to the question, ``How do we prove a certain proposition?'', is often answered by means of a definition.  For example, in Progress Check~\ref{pr:explores} on page~\pageref{pr:explores}, all of the examples should have indicated that the following conditional statement is true:
\begin{center}
If  $x$  and  $y$  are odd integers, then $x \cdot y$ is an odd integer.
\end{center}
In order to construct a mathematical proof of this conditional statement, we need a precise definition of what it means to say that an integer is an even integer and what it means to say that an integer is an odd integer.
%
\begin{defbox}{D:even}{An integer  $a$  is an \textbf{even integer} 
\index{even integer}%
 provided that there exists an integer  $n$  such that  $a = 2n$. An integer  $a$  is an 
\textbf{odd integer}
\index{odd integer}%
 provided there exists an integer  $n$  such that  $a = 2n + 1$.}
\end{defbox}
\newpar
\label{def:even}
Using this definition, we can conclude that the integer 16 is an even integer since $16 = 2 \cdot 8$ and 8 is an integer.  By answering the following questions, you should obtain a better understanding of these definitions.  These questions are not here just to have questions in the textbook.  Constructing and answering such questions is a way in which many mathematicians will try to gain a better understanding of a definition.
%
%\begin{flushleft}
%\fbox{\parbox{5in}{\begin{definition}
%An integer  $a$  is an \textbf{even integer} if there exists an integer  $n$  such that  $a = 2n$. An integer  $a$  is an \textbf{odd integer} if there exists an integer  $n$  such that  $a = 2n + 1$.
%\end{definition}}}
%\end{flushleft}
\begin{enumerate}
\item Use the definitions given above to
\begin{enumerate}
\item Explain why  28, $-42$, 24, and 0 are even integers.

\item Explain why 51, $-11$, 1, and $-1$  are odd integers.
\end{enumerate}
\end{enumerate}

\noindent
It is important to realize that mathematical definitions are not made randomly.  In most cases, they are motivated by a mathematical concept that occurs frequently.
%
\begin{enumerate}
\setcounter{enumi}{1}
\item Are the definitions of even integers and odd integers consistent with your previous ideas about even and odd integers?	
\end{enumerate}
\hbreak
\end{previewactivity}

\endinput
