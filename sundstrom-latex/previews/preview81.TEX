\begin{previewactivity}[A Review of Divisibility] \label{PA:reviewdivides} \hfill
\begin{enumerate}
\item Explain what it means to say that a nonzero integer  $m$  divides an integer $n$.  Recall that we use the notation  $m \mid n$  to indicate that the integer  $m$  divides the integer  $n$.

%\item Give two different examples of a pair of integers where the first integer divides the second integer, and give two different examples of a pair of integers where the first integer does not divide the second integer.

\item Let  $m$  and  $n$  be integers with $m \ne 0$.  Explain what it means to say that  $m$  does not divide  $n$.

%\item According to the definition of ``divides,'' does the integer  0 divide the integer  10?  Explain.
%
%\item According to the definition of ``divides,'' does the integer  10  divide the integer  0?  Explain.

\item Is the following proposition true or false?  Justify your conclusion.
\begin{center}
For all  $a, b, c \in \Z$ with $a \ne 0$, if  $a \mid \left( {b \cdot c} \right)$, then  $a \mid b$
or  $a \mid c$.
\end{center}
\end{enumerate}
\end{previewactivity}
\hbreak
%
\begin{previewactivity}[The Greatest Common Divisor] \label{PA:gcd} \hfill
\begin{defbox}{gcd}{Let  $a$  and  $b$  be integers, not both 0. A \textbf{common divisor}
\index{common divisor}%
 of  $a$  and  $b$  is any nonzero integer that divides both  $a$  and  $b$.  The \emph{largest} natural number that divides both  $a$  and  $b$  is called the \textbf{greatest common divisor}
\index{greatest common divisor}%
  of  $a$  and  $b$.  The greatest common divisor of  $a$  and  $b$  is denoted by  $\gcd \left( {a, b} \right)$.
\label{sym:gcd}}
\end{defbox}

\begin{enumerate}
\item Use the roster method to list the elements of the set that contains all the natural numbers that are divisors of 48. \label{PA:gcd1}

\item Use the roster method to list the elements of the set that contains all the natural numbers that are divisors of 84. \label{PA:gcd2}

\item Determine the intersection of the two sets in Parts~(\ref{PA:gcd1}) and~(\ref{PA:gcd2}).  This set contains all the natural numbers that are common divisors of  48  and  84. 

\item What is the greatest common divisor of  48  and  84?  
\label{PA:gcd4}%

\item Use the method suggested in Parts~(\ref{PA:gcd1}) through~(\ref{PA:gcd4}) to determine each of the following:  $\gcd( {8,  - 12}) $, 
$\gcd( {0, 5} )$, $\gcd( {8, 27} )$, and $\gcd( {14, 28})$.  
\label{PA:gcd5}%

\item If $a$ and $b$ are integers, make a conjecture about how the common divisors of  $a$  and  $b$  are related to the greatest common divisor of  $a$  and  $b$.
\end{enumerate}
\end{previewactivity}
\hbreak
%
\begin{previewactivity}[The GCD and the Division Algorithm] \label{PA:gcdanddivalgo} \hfill \\
%\begin{enumerate}
%\item Write a complete statement of the Division Algorithm for integers.  (See 
%Section~\ref{S:divalgo}, page~\pageref{divalgorithm}.)
%\end{enumerate}
When we speak of the quotient and the remainder when we ``divide an integer  $a$  by the positive integer  $b$,'' we will always mean the quotient $q$    and the remainder  $r$  guaranteed by the Division Algorithm.  (See Section~\ref{S:divalgo}, page~\pageref{divalgorithm}.)

\begin{enumerate}
%\setcounter{enumi}{1}
\item Each row in the following table contains values for the integers  $a$  and  $b$.  In this table, the value of  $r$  is the remainder (from the Division Algorithm) when  $a$  is divided by  $b$.  Complete each row in this table by determining  $\gcd( {a, b} )$, $r$, and  
$\gcd( {b, r} )$.  
\label{PA:gcdanddivalgo2}%

%\begin{center}
%\begin{tabular}{| c | c | c |  c | c |} \hline
%$a$   &  $b$   &   $\gcd( {a, b} )$   &   Remainder  $r$   &  $\gcd( {b, r} )$ \\ \hline
%44   &   12  &  &  &  \\ \hline
%75   &   21	 &  &  &  \\ \hline
%50   &   33	 &  &  &  \\ \hline
%\end{tabular}
%\end{center}		
$$
\BeginTable
\BeginFormat
| c | c | c | c | c |
\EndFormat
\_
|  $a$  |  $b$  |  $\gcd(a, b)$  |  Remainder $r$  |  $\gcd(b,r)$ | \\+22 \_
|  44   |  12   |                |                 |              | \\+22 \_
|  75   |  21   |                |                 |              | \\+22 \_
|  50   |  33   |                |                 |              | \\+22 \_
\EndTable
$$
\item Formulate a conjecture based on the results of the table in Part~(\ref{PA:gcdanddivalgo2}).
\end{enumerate}
\end{previewactivity}
\hbreak



\endinput
