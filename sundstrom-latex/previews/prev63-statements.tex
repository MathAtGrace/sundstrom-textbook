\begin{previewactivity}[\textbf{Statements Involving Functions}]
\label{PA:functionstatements} \hfill \\
Let $A$ and $B$ be nonempty sets and let $f\x A \to B$.  
In \typeu Activity~\ref*{PA:functionswithfinitedom}, we determined whether or not certain functions satisfied some specified properties.  These properties were written in the form of statements, and we will now examine these statements in more detail.
\begin{enumerate}
\item Consider the following statement:
\begin{list}{}
\item For all $x, y \in A$, if $x \ne y$, then $f ( x ) \ne f ( y )$.
\end{list}

\begin{enumerate} \label{PA:functionstatements1}
\item Write the contrapositive of this conditional statement.

\item Write the negation of this conditional statement.
\end{enumerate}

\item Now consider the statement:
\label{PA:functionstatements2}%
\begin{list}{}
\item For all $y \in B$, there exists an $x \in A$ such that $f ( x ) = y$.
\end{list}
Write the negation of this statement.

\item Let $g:\R \to \R$ be defined by $g ( x ) = 5x + 3$, for all $x \in \R$.  Complete the following proofs of the following propositions about the function $g$.
\label{PA:functionstatements3}

\newpar
\textbf{Proposition 1}.  For all $a, b \in \R$, if $g ( a ) = g ( b )$, then $a = b$.

\noindent
\textbf{\emph{Proof}}.  We let $a, b \in \R$, and we assume that 
$g ( a ) = g ( b )$ and will prove that $a = b$.  Since $g(a) = g(b)$, we know that
\[
5a + 3 = 5b + 3.
\]
(Now prove that in this situation, $a = b$.)

\newpar
\textbf{Proposition 2}.  For all $b \in \R$, there exists an $a \in \R$ such that $g ( a ) = b$.

\noindent
\textbf{\emph{Proof}}.  We let $b \in \R$.  We will prove that there exists an $a \in \R$ such that 
$g ( a ) = b$ by constructing such an $a$ in $\R$.  In order for this to happen, we need $g(a) = 5a + 3 = b$.

\noindent
(Now solve the equation for $a$ and then show that for this real number $a$, 
$g ( a ) = b$.)
\end{enumerate}

\end{previewactivity}
\hbreak

\endinput
