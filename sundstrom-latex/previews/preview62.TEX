\begin{previewactivity}[A Function Defined by a Congruence] \label{PA:functioncongruence} \hfill \\
Write complete statements of Theorem~\ref{T:congtorem}  and  Corollary~\ref{C:congtorem}   from Section~\ref{S:cases}.
%\vskip10pt
%\noindent
%\textbf{Theorem~\ref{T:congtorem}} \emph{Let  $n \in \mathbb{N}$ and let  $a \in \mathbb{Z}$.  If  $a = nq + r\text{  and  }0 \leq r < n$ for some integers  $q$  and  $r$, then  
%$a \equiv r \pmod n$.}
%\vskip10pt

Theorem~\ref{T:congtorem} and Corollary~\ref{C:congtorem} state that an integer is congruent (mod $n$) to its remainder when it is divided by  $n$.  (Recall that we always mean the remainder guaranteed by the Division Algorithm, which is the least nonnegative remainder.)  Since this remainder is unique and since the only possible remainders for division by $n$  are  $0, 1, 2,  \ldots , n - 1$, we then know that each integer is congruent, modulo $n$, to precisely one of the integers $0,1,2, \ldots ,n - 1$.
%\vskip10pt
%\noindent
%\textbf{Corollary~\ref{C:congtorem}} \emph{If  $n \in \mathbb{N}$, then each integer is congruent, modulo n, to precisely one of the integers $0,1,2, \ldots ,n - 1$.}
%\vskip10pt
%
\begin{enumerate}
\item Define the set  $\mathbb{Z}_6 $ to be  $\mathbb{Z}_6  = \left\{ {0, 1, 2, 3, 4, 5} \right\}$.  For each  $x \in \mathbb{Z}_6 $, compute  $x^2  + 3$ and then determine the value of  $r$  in  $\mathbb{Z}_6 $ so that
\[
\left( {x^2  + 3} \right) \equiv r\pmod 6.
\]
For example,  $2^2  + 3 = 7$ and so  $\left( {2^2  + 3} \right) \equiv 1 \pmod 6$.  Organize your results in a table with one column for the value of $x$ and another column for the value of $r$, where $r \in \mathbb{Z}_6$ and $\left( {x^2  + 3} \right) \equiv r\pmod 6$. 
\label{PA:functioncongruence1}

\item Explain how your work in Part~(\ref{PA:functioncongruence1}) can be used to define a function from   $\mathbb{Z}_6 $
to  $\mathbb{Z}_6 $.

\end{enumerate}
\end{previewactivity}
\hbreak
%
%
\begin{previewactivity}[The Number of Diagonals of a Polygon] \label{PA:diagonals} \hfill \\
A \textbf{polygon}
\index{polygon}%
 is a closed plane figure formed by the joining of three or more straight lines. For example, a \textbf{triangle}
\index{triangle}%
 is a polygon that has three sides; a \textbf{quadrilateral}
\index{quadrilateral}%
 is a polygon that  has four sides and includes squares, rectangles, and parallelograms; a \textbf{pentagon}
\index{pentagon}%
 is a polygon that  has five sides; and an \textbf{octagon}
\index{octagon}%
 is a polygon that has eight sides. A \textbf{regular polygon}
\index{regular polygon}%
\index{polygon!regular}%
 is one that has equal-length sides and congruent interior angles.

A \textbf{diagonal of a polygon}
\index{diagonal}%
\index{polygon!diagonal}%
 is a line segment that connects two nonadjacent vertices of the polygon.  In this activity, we will assume that all polygons are convex polygons so that, except for the vertices, each diagonal lies inside the polygon.

\begin{enumerate}
\item How many diagonals does a triangle have?  How many diagonals does a square have?  How many diagonals does any quadrilateral have?

\item Let   $D = \mathbb{N} - \left\{ {1, 2} \right\}$.  Define   
$d\x D \to \mathbb{N} \cup \left\{ 0 \right\}$ so that   $d( n )$ is the number of diagonals of a  convex polygon with  $n$  sides.   Determine the values of $d(3)$, $d(4)$, $d(5)$, $d(6)$, $d(7)$, and $d(8)$.  Arrange the results in the form of a table of values for the function $d$.
\label{PA:diagonals2}
%Complete the following table, which shows the values of  $d\left( n \right)$ for selected values of  $n$.  
%
%\begin{center}
%\begin{tabular}{ c | c  c  c | c }
%  $n$   &  $d\left( n \right)$  &  &  $n$  &  $d\left( n \right)$ \\ \cline{1-2} \cline{4-5}
%   3    &                       &  &   6   &  \\ \cline{1-2} \cline{4-5}
%   4    &                       &  &   7   &  \\ \cline{1-2} \cline{4-5}
%   5    &                       &  &   8   &  \\ \cline{1-2} \cline{4-5}
%\end{tabular}
%\end{center}

\item Let  $f\x \mathbb{R} \to \mathbb{R}$  be defined by  
\[
f( x ) = \frac{{x\left( {x - 3} \right)}}{2}.
\]
Determine the values of $f(0)$, $f(1)$, $f(2)$, $f(3)$, $f(4)$, $f(5)$, $f(6)$, $f(7)$, 
$f(8)$, and $f(9)$.  Arrange the results in the form of a table of values for the function $f$\!.
\label{PA:diagonals3}%

%Complete the following table, which shows the values of  $f\left( x \right)$  for selected values of  $x$.  
%\begin{center}
%\begin{tabular}{ c | c  c  c | c }
%  $x$   &  $f\left( x \right)$  &  &  $x$  &  $f\left( x \right)$ \\ \cline{1-2} \cline{4-5}
%   0    &                       &  &   5   &  \\ \cline{1-2} \cline{4-5}
%   1    &                       &  &   6   &  \\ \cline{1-2} \cline{4-5}
%   2    &                       &  &   7   &  \\ \cline{1-2} \cline{4-5}
%   3    &                       &  &   8   &  \\ \cline{1-2} \cline{4-5}
%   4    &                       &  &   9   &  \\ \cline{1-2} \cline{4-5}
%\end{tabular}
%\end{center}

\item	What (if any) are the differences between the functions described in Parts~(\ref{PA:diagonals2}) and~(\ref{PA:diagonals3})?  Explain.

\end{enumerate}
\end{previewactivity}
\hbreak

\begin{previewactivity}[Derivatives] \label{PA:derivatives} \hfill \\
In calculus, we learned how to find the derivatives
\index{derivative}%
 of certain functions.  For example, if
\[
f( x ) = x^2( {\sin x} ),
\]
then we can use the product rule to obtain
\[
f'( x ) = 2x( {\sin x} ) + x^2( {\cos x} ).
\]
\begin{enumerate}
\item If possible, find the derivative of each of the following functions:
\begin{multicols}{2}
\begin{enumerate}
  \item $f( x ) = x^4  - 5x^3  + 3x - 7$

  \item $g( x ) = \cos ( {5x} )$

  \item $h( x ) = \dfrac{{\sin x}}{x}$
	
  \item $k( x ) = e^{ - x^2 } $

  \item $r( x ) = \left| x \right|$
\end{enumerate}
\end{multicols}

\item Is it possible to think of differentiation as a function?  Explain.  If so, what would be the domain of the function, what could be the codomain of the function, and what is the rule for computing the element of the codomain (output) that is associated with a given element of the domain (input)?

\end{enumerate}
\end{previewactivity}
\hbreak


\endinput

\begin{center}
\begin{tabular}[h]{| c | c |}
\hline
             &  $r \in \mathbb{Z}_6 $ where \\
   $x$       &  $\left( {x^2  + 3} \right) \equiv r \pmod 6$ \\ \hline
    0        &   \\ \hline
    1        &   \\ \hline
    2        &   \\ \hline
    3        &   \\ \hline
    4        &   \\ \hline
    5        &   \\ \hline
\end{tabular}
\end{center}
