\begin{previewactivity}[\textbf{The Greatest Common Divisor}] \label{PA:gcd} \hfill
\begin{enumerate}
  \item Explain what it means to say that a nonzero integer  $m$  divides an integer $n$.  Recall that we use the notation  $m \mid n$  to indicate that the nonzero integer  $m$  divides the integer  $n$.

\item Let  $m$  and  $n$  be integers with $m \ne 0$.  Explain what it means to say that  $m$  does not divide  $n$.
\end{enumerate}
\begin{defbox}{gcd}{Let  $a$  and  $b$  be integers, not both 0. A \textbf{common divisor}
\index{common divisor}%
 of  $a$  and  $b$  is any nonzero integer that divides both  $a$  and  $b$.  The \emph{largest} natural number that divides both  $a$  and  $b$  is called the \textbf{greatest common divisor}
\index{greatest common divisor}%
  of  $a$  and  $b$.  The greatest common divisor of  $a$  and  $b$  is denoted by  $\gcd \left( {a, b} \right)$.
\label{sym:gcd}}
\end{defbox}

%\begin{enumerate}
\setcounter{oldenumi}{\theenumi}
\begin{enumerate} \setcounter{enumi}{\theoldenumi}

\item Use the roster method to list the elements of the set that contains all the natural numbers that are divisors of 48. \label{PA:gcd1}

\item Use the roster method to list the elements of the set that contains all the natural numbers that are divisors of 84. \label{PA:gcd2}

\item Determine the intersection of the two sets in Parts~(\ref{PA:gcd1}) and~(\ref{PA:gcd2}).  This set contains all the natural numbers that are common divisors of  48  and  84. 

\item What is the greatest common divisor of  48  and  84?  
\label{PA:gcd4}%

\item Use the method suggested in Parts~(\ref{PA:gcd1}) through~(\ref{PA:gcd4}) to determine each of the following:  $\gcd( {8,  - 12}) $, 
$\gcd( {0, 5} )$, $\gcd( {8, 27} )$, and $\gcd( {14, 28})$.  
\label{PA:gcd5}%

\item If $a$ and $b$ are integers, make a conjecture about how the common divisors of  $a$  and  $b$  are related to the greatest common divisor of  $a$  and  $b$.
\end{enumerate}
\end{previewactivity}
\hbreak

\endinput
