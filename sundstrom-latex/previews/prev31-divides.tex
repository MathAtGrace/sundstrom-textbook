\begin{previewactivity}[\textbf{Definition of Divides, Divisor, Multiple}]\label{PA:divisor} \hfill \\
%
In Section~\ref{S:direct}, we studied the concepts of even integers
\index{even integer}%
 and odd integers.  The definition of an even integer was a formalization of our concept of an even integer as being one that is ``divisible by 2,'' or a ``multiple of 2.''  We could also say that if ``2 divides an integer,'' then that integer is an even integer.  We will now extend this idea to integers other than 2.  Following is a formal definition of what it means to say that a nonzero integer $m$ divides an integer $n$.

\begin{defbox}{divides}{A nonzero integer $m$  \textbf{divides}
\index{divides}%
 an integer  $n$  provided that there is an integer  $q$  such that  $n = m \cdot q$.  We also say that  $m$  is a \textbf{divisor}
\index{divisor}%
 of  $n$, $m$ is a \textbf{factor}
\index{factor}%
 of $n$, and $n$  is a \textbf{multiple}
\index{multiple}%
 of  $m$.  The integer 0 is not a divisor of any integer.  If $a$ and $b$ are integers and $a \ne 0$, we frequently use the notation $a \mid b$ as a shorthand for ``$a$ divides $b$.''}
\label{sym:divides}%}  
\end{defbox}
%
\noindent
\textbf{A Note about Notation}:  Be careful with the notation $a \mid b$.  This does not represent the rational number  $\dfrac{a}{b}$.  The notation  $a \mid b$  represents a relationship between the integers  
$a$  and  $b$  and is simply a shorthand for ``$a$  divides  $b$.''

\newpar
\textbf{A Note about Definitions}:  Technically, a definition in mathematics should almost always be written using ``if and only if.''  It is not clear why, but the convention in mathematics is to replace the phrase ``if and only if'' with ``if'' or an equivalent.  Perhaps this is a bit of laziness or the ``if and only if'' phrase can be a bit cumbersome.  In this text, we will often use the phrase ``provided that'' instead.
%\vskip6pt

The definition for ``divides'' can be written in symbolic form using appropriate quantifiers as follows:
A nonzero integer  $m$  \textbf{divides} an integer  $n$  provided that $\left( {\exists q \in \mathbb{Z}} \right)\left( {n = m \cdot q} \right)$.
%
\begin{enumerate}
  \item Use the definition of divides to explain why 4 divides 32 and to explain why 8 divides $-96$.
%  \item Give three different examples of three integers where the first integer divides the second integer and the second integer divides the third integer.  
%\label{PA:divisor1}%
%
%  \item In your examples in Part~(\ref{PA:divisor1}), is there any relationship between the first and the third integer?  Explain, and formulate a conjecture.  \textbf{Write your conjecture in the form of a conditional statement with appropriate quantifiers}.

  \item Give several examples of two integers where the first integer does not divide the second integer.

  %\item According to the definition of ``divides,'' does the integer  0  divide the integer 10?  That is, is  0  a divisor of 10?  Explain.

  \item According to the definition of ``divides,'' does the integer  10  divide the integer  0?  That is, is  10  a divisor of  0?  Explain.

  \item Use the definition of ``divides'' to complete the following sentence in symbolic form:  ``The nonzero integer  $m$ does not divide the integer $n$ means that \ldots .''

  \item Use the definition of ``divides'' to complete the following sentence without using the symbols for quantifiers:  ``The nonzero integer  $m$  does not divide the integer $n \ldots .$''
  \item Give three different examples of three integers where the first integer divides the second integer and the second integer divides the third integer.  
\label{PA:divisor1}%
\end{enumerate}
As we have seen in Section~\ref{S:direct}, a definition is frequently used when constructing and writing mathematical proofs.  Consider the following conjecture:

\eighth
\setlength{\hangindent}{60pt}
\noindent
\textbf{Conjecture:} \emph{Let $a$, $b$, and $c$ be integers with $a \ne 0$ and $b \ne 0$.  If $a$ divides $b$ and $b$ divides $c$, then $a$ divides $c$.}

\setcounter{oldenumi}{\theenumi}
\begin{enumerate} \setcounter{enumi}{\theoldenumi}
\item Explain why the examples you generated in part~(\ref{PA:divisor1}) provide evidence that this conjecture is true.
\end{enumerate}
\setlength{\hangindent}{0pt}
In Section~\ref{S:direct}, we also learned how to use a \textbf{know-show table} to help organize our thoughts when trying to construct a proof of a statement.  If necessary, review the appropriate material in Section~\ref{S:direct}.
\setcounter{oldenumi}{\theenumi}
\begin{enumerate} \setcounter{enumi}{\theoldenumi}
  \item  State precisely what we would assume if we were trying to write a proof of the preceding conjecture. \label{PA:divisor2}
  \item Use the definition of ``divides'' to make some conclusions based on your assumptions in part~(\ref{PA:divisor2}).
  \item State precisely what we would be trying to prove if we were trying to write a proof of the conjecture. \label{PA:divisor3}
  \item Use the definition of divides to write an answer to the question, ``How can we prove what we stated in part~(\ref{PA:divisor3})?''
\end{enumerate}

\hbreak
\end{previewactivity}


\endinput
