\begin{previewactivity}[\textbf{Properties of Relations}] \label{PA:propsofrelaitons} \hfill \\
In previous mathematics courses, we have worked with the equality relation.  For example, let $R$ be the relation on $\Z$ defined as follows:  For all $a, b \in \Z$,  $a \mathrel{R} b$ if and only if $a = b$.  We know this equality relation on $\Z$ has the following properties:
\begin{itemize}
  \item For each $a \in \Z$, $a = a$ and so $a \mathrel{R} a$.
  \item For all $a, b \in \Z$, if $a = b$, then $b = a$.  That is, if $a \mathrel{R} b$, then $b \mathrel{R} a$.
  \item For all $a, b, c \in \Z$, if $a = b$ and $b = c$, then $a = c$.  That is, if $a \mathrel{R} b$ and 
         $b \mathrel{R} c$, then $a \mathrel{R} c$.
\end{itemize}
In mathematics, when something satisfies certain properties, we often ask if other things satisfy the same properties.  Before investigating this, we will give names to these properties.

%In Beginning Activity~\ref{PA:directedgraphs}, the same three questions were asked about two different relations on the set $A = \left\{ {1, 2, 3, 4} \right\}$.  These were questions about certain properties of relations, and these properties occur frequently enough to warrant names.

\begin{defbox}{ref-sym-trans}{Let  $A$  be a nonempty set and let  $R$  be a relation on  $A$.
\begin{itemize}
\item The relation  $R$  is \textbf{reflexive on}
\index{reflexive}%
\index{relation!reflexive on}%
 $\boldsymbol{A}$  provided that for each  
$x \in A$,  $x \mathrel{R} x$ or, equivalently,  $\left( {x, x} \right) \in R$.

\item The relation  $R$  is \textbf{symmetric}
\index{symmetric}%
\index{relation!symmetric}%
  provided that for every  $x, y \in A$,  if  
$x \mathrel{R} y$, then  $y \mathrel{R} x$ or, equivalently, for every  $x, y \in A$,  if  $\left( {x, y} \right) \in R$, then  $\left( {y, x} \right) \in R$.

\item The relation  $R$  is \textbf{transitive}
\index{transitive}%
\index{relation!transitive}%
  provided that for every $x, y, z \in A$,  if  $x \mathrel{R} y$ and  $y \mathrel{R} z$, then  $x \mathrel{R} z$ or, equivalently, for every  $x, y, z \in A$,  if  $\left( {x, y} \right) \in R$ and $\left( {y, z} \right) \in R$, then  $\left( {x, z} \right) \in R$.
\end{itemize}
}
\end{defbox}
Before exploring examples, for each of these properties, it is a good idea to understand what it means to say that a relation does not satisfy the property.  So let  $A$  be a nonempty set and let  $R$  be a relation on  $A$. 
\begin{enumerate}
\item Carefully explain what it means to say that the relation  $R$  is not reflexive on the set  $A$.

\item Carefully explain what it means to say that the relation  $R$  is not symmetric.

\item Carefully explain what it means to say that the relation  $R$  is not transitive.
\end{enumerate}
To illustrate these properties, we let  $A = \left\{ {1, 2, 3, 4} \right\}$ and define the relations $R$ and $T$ on $A$ as follows:
\begin{align*}
R &= \left\{ {( {1, 1} ), ( {2, 2} ), ( {3, 3} ), ( {4, 4} ), ( {1, 3} ), ( {3, 2} )} \right\} \\
T &= \left\{ {( {1, 1} ), ( {1, 4} ), ( {2, 4} ), ( {4, 1} ), ( {4, 2} )} \right\}
\end{align*}
\setcounter{oldenumi}{\theenumi}
\begin{enumerate} \setcounter{enumi}{\theoldenumi}
\item Draw a directed graph for the relation $R$. Then explain why the relation $R$ is reflexive on $A$, is not symmetric, and is not transitive.
\item Draw a directed graph for the relation $T$.  Is the relation $T$ reflexive on $A$?  Is the relation $T$ symmetric?  Is the relation $T$ transitive?  Explain. 
\end{enumerate}
%   We then see that the relation $R$ is reflexive on $A$.  (Why?)  However:
%\begin{itemize}
%  \item The relation $R$ is not symmetric since $(1, 3) \in R$ but $(3, 1) \notin R$.  ($1 \mathrel{R} 3$ but $3 \not \negthickspace \negthinspace \mathrel{R} 1$.)
%  \item The relation $R$ is not transitive since $(1, 3) \in R$ and $(3, 2) \in R$, but $(1, 3) \notin R$.  ($1 \mathrel{R} 3$ and $3 \mathrel{R} 2$ but $3 \not \negthickspace \negthinspace \mathrel{R} 2.)$
%\end{itemize}
%\begin{enumerate}
%  \item Let  $A = \left\{ {1, 2, 3, 4} \right\}$ and  
%$S = \left\{ {( {1, 1} ), ( {1, 4} ), ( {2, 4} ), ( {4, 1} ), ( {4, 2} )} \right\}$.  Draw a digraph for this relation and then determine if the relation $S$ is reflexive on $A$, if the relation $S$ is symmetric, and if the relation $S$ is transitive.
%\end{enumerate}
%
%\noindent
%Define the relations  $\sim$  and  $ \approx $  on  $\Q$  as follows:  For all $a, b \in Q$,
%\begin{itemize}
%\item $a \sim b$  if and only if  $a - b \in \Z $.	
%
%\item $a \approx b$  if and only if  $a + b \in \Z$.
%\end{itemize}
%
%\noindent
%For example, $\dfrac{3}{4} \sim \dfrac{7}{4}$ since $\dfrac{3}{4} - \dfrac{7}{4} = -1$, and 
%$\dfrac{3}{4} \not  \approx \dfrac{7}{4}$ since $\dfrac{3}{4} + \dfrac{7}{4} = \dfrac{5}{2}$.
%
%\begin{enumerate}
%\item Is the relation  $\sim$  a reflexive relation on  $\Q$?  Is it a symmetric relation?  Is it a transitive relation? 
%
%\item Is the relation  $ \approx $  a reflexive relation on  $\Q$?  Is it a symmetric relation?  Is it a transitive relation?
%\end{enumerate}
\end{previewactivity}
\hbreak

\endinput
