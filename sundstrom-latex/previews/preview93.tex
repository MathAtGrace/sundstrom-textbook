\begin{previewactivity}[The Game of Dodge Ball]\label{PA:dodgeball} \hfill \\
\index{Dodge Ball}%
(From \emph{The Heart of Mathematics: An Invitation to Effective Thinking} by Edward B. Burger and Michael Starbird, Key Publishing Company, \copyright 2000 by Edward B. Burger and Michael Starbird.)

Dodge Ball is a game for two players.  It is played on a game board such as the one shown in Figure~\ref{fig:dodgeball}.
\begin{figure}[h]
\begin{center}
\setlength{\unitlength}{0.5cm}
\begin{picture}(15,21)
\put(1,20){Player One's Array}
\put(1,19){\line(1,0){13}}
\put(1,17){\line(1,0){13}}
\put(1,15){\line(1,0){13}}
\put(1,13){\line(1,0){13}}
\put(1,11){\line(1,0){13}}
\put(1,9){\line(1,0){13}}
\put(1,7){\line(1,0){13}}

\put(1,7){\line(0,1){12}}
\put(2,7){\line(0,1){12}}
\put(4,7){\line(0,1){12}}
\put(6,7){\line(0,1){12}}
\put(8,7){\line(0,1){12}}
\put(10,7){\line(0,1){12}}
\put(12,7){\line(0,1){12}}
\put(14,7){\line(0,1){12}}

\put(1.3,17.5){1}
\put(1.3,15.5){2}
\put(1.3,13.5){3}
\put(1.3,11.5){4}
\put(1.3,9.5){5}
\put(1.3,7.5){6}

\put(1,5){Player Two's Row}
\put(2,4){\line(1,0){12}}
\put(2,3){\line(1,0){12}}
\put(2,1){\line(1,0){12}}

\put(2,1){\line(0,1){3}}
\put(4,1){\line(0,1){3}}
\put(6,1){\line(0,1){3}}
\put(8,1){\line(0,1){3}}
\put(10,1){\line(0,1){3}}
\put(12,1){\line(0,1){3}}
\put(14,1){\line(0,1){3}}

\put(2.8,3.3){1}
\put(4.8,3.3){2}
\put(6.8,3.3){3}
\put(8.8,3.3){4}
\put(10.8,3.3){5}
\put(12.8,3.3){6}

\put(0,21){\line(1,0){15}}
\put(0,0){\line(1,0){15}}
\put(0,0){\line(0,1){21}}
\put(15,0){\line(0,1){21}}
\end{picture}
\caption{Game Board for Dodge Ball}\label{fig:dodgeball}
\end{center}
\end{figure}

Player One has a 6 by 6 array to complete and Player Two has a 1 by 6 row to complete.  Each player has six turns as described next.
\begin{itemize}
\item Player One begins by filling in the first horizontal row of his or her table with a sequence of six X's and O's, one in each square in the first row.

\item Then Player Two places either an X or an O in the first box of his or her row.  At this point, Player One has completed the first row and Player Two has filled in the first box of his or her row with one letter.

\item The game continues with Player One completing a row with six letters (X's and O's), one in each box of the next row followed by Player Two writing one letter (an X or an O) in the next box of his or her row.  The game is completed when Player One has completed all six rows and Player Two has completed all six boxes in his or her row.
\end{itemize}

\textbf{Winning the Game}
\begin{itemize}
\item Player One wins if any horizontal row in the 6 by 6 array is identical to the row that Player Two created.  (Player One matches Player Two.)

\item Player Two wins if Player Two's row of six letters is different than each of the six rows produced by Player One.  (Player Two ``dodges'' Player One.)
\end{itemize}

There is a winning strategy for one of the two players.  This means that there is plan by which one of the two players will always win.  Which player has a winning strategy?  Carefully describe this winning strategy.
\end{previewactivity}
\hbreak
%
\begin{previewactivity}[Functions from a Set to Its Power Set]\label{PA:powerset} \hfill \\
Let $A$ be a set.  In Section~\ref{S:setoperations}, we defined the \textbf{power set}
\index{power set}% 
$\mathcal{P} ( A )$ of $A$ to be the set of all subsets of $A$.  This means that
\begin{center}
$X \in \mathcal{P} ( A )$ if and only if $X \subseteq A$.
\end{center}
%
In Proposition~\ref{P:powersetcardinality} in Section~\ref{S:otherinduction}, we proved that if a set $A$ has $n$ elements, then $A$ has $2^n$ subsets or that $\mathcal{P} ( A )$ has 
$2^n$ elements.  Using our current notation for cardinality, this means that
%
\begin{center}
if $\text{card} ( A ) = n$, then 
$\text{card} ( \mathcal{P} ( A ) ) = 2^n$.
\end{center}
%
%\begin{enumerate}
%\item Determine the power set of each of the following sets:
%\begin{multicols}{2}
%\begin{enumerate}
%\item $A = \left\{a, b \right\}$
%\item $B = \left\{a, b, c \right\}$
%\end{enumerate}
%\end{multicols}
%\end{enumerate}
%
We are now going to define and explore some functions from a set $A$ to its power set 
$\mathcal{P} ( A )$.  This means that the input of the function will be an element of $A$ and the output of the function will be a subset of $A$.

\begin{enumerate} 
\item Let $A = \left\{1, 2, 3, 4 \right\}$.  Define $f\x A \to \mathcal{P} ( A )$ by
%
\begin{multicols}{2}
$f ( 1 ) = \left\{ 1, 2, 3 \right\}$

$f ( 2 ) = \left\{ 1, 3, 4 \right\}$

$f ( 3 ) = \left\{ 1, 4 \right\}$

$f ( 4 ) = \left\{ 2, 4 \right\}$.
\end{multicols}
%
\begin{enumerate}
\item Is $1 \in f ( 1 )$?  Is $2 \in f ( 2 )$? Is 
$3 \in f ( 3 )$?  Is $4 \in f ( 4 )$?

\item Determine $S = \left\{ x \in A \mid x \notin f ( x ) \right\}$.

\item Notice that $S \in \mathcal{P} ( A )$.  Does there exist an element $t$ in $A$ such that $f ( t ) = S$?  That is, is $S \in \text{range} ( f )$?
 
\end{enumerate}

\item Let $A = \left\{1, 2, 3, 4 \right\}$.  Define $f\x A \to \mathcal{P} ( A )$ by 
%
\begin{center}
$f ( x ) = A - \left\{ x \right\}$ for each $x \in A$.
\end{center}
%
\begin{multicols}{2}
\begin{enumerate}
\item Determine $f ( 1 )$.  Is $1 \in f ( 1 )$?
\item Determine $f ( 2 )$.  Is $2 \in f ( 2 )$?
\item Determine $f ( 3 )$.  Is $3 \in f ( 3 )$?
\item Determine $f ( 4 )$.  Is $4 \in f ( 4 )$?
\end{enumerate}
\end{multicols}
%
\begin{enumerate} \setcounter{enumii}{4}
\item Determine $S = \left\{ x \in A \mid x \notin f ( x ) \right\}$.

\item Notice that $S \in \mathcal{P} ( A )$.  Does there exist an element $t$ in $A$ such that $f ( t ) = S$?  That is, is $S \in \text{range} ( f )$?
\end{enumerate}
%
\item Define $f\x \mathbb{N} \to \mathcal{P} ( \mathbb{N} )$ by
%
\begin{center}
$f ( n ) = \mathbb{N} - \left\{n^2, n^2-2n \right\}$, for each $n \in \mathbb{N}$.
\end{center}
%
\begin{enumerate}
\item Determine $f ( 1 )$, $f ( 2 )$, $f ( 3 )$, and 
$f ( 4 )$.  In each of these cases, determine if $k \in f ( k )$.
%
\item Prove that if $n > 3$, then $n \in f ( n )$.  \hint  Prove that if $n >3$, then 
$n^2 > n$ and $n^2 - 2n >n$.

\item Determine $S = \left\{ x \in \mathbb{N} \mid x \notin f ( x ) \right\}$.

\item Notice that $S \in \mathcal{P} ( \mathbb{N} )$.  Does there exist an element $t$ in $\mathbb{N}$ such that $f ( t ) = S$?  That is, is 
$S \in \text{range} ( f )$?

\end{enumerate}
\end{enumerate}
\end{previewactivity}
\hbreak

\endinput


\begin{previewactivity}[Some Real Functions]\label{PA:realfunctions} \hfill

\begin{enumerate}
\item In Part~(\ref{PA:equivalentsets6}) of Preview Activity~\ref{PA:equivalentsets} in 
Section~\ref{S:finitesets}, we proved that if $r \in \mathbb{R}$ and $r > 0$, then the open interval $( 0, 1 )$ is equivalent to the open interval $( 0, r )$.

Now let $a$ and $b$ be real numbers with $a < b$.  Find a function 
\[
f\x  ( 0, 1 ) \to ( a, b )
\]
that is a bijection and conclude that $( 0, 1 ) \approx ( a, b )$.

\hint  Find a linear function that passes through the points $( 0, a )$ and 
$( 1, b )$.  Use this to define the function $f$.  Make sure you prove that this function $f$ is a bijection.

\item Let $a, b, c, d$ be real numbers with $a < b$ and $c < d$.  Prove that \\
$( a, b ) \approx ( c, d )$.


\end{enumerate}



\end{previewactivity}
