\begin{previewactivity}[Factorials and Recurrence Relations] \label{PA:factorialsandrecur} \hfill \\
In Preview Activity~\ref{PA:factorials} in Section~\ref{S:otherinduction}, we defined  $n!$, read  \textbf{$n$  factorial}, \label{factorial2}
\index{factorial}%
 for each natural number  $n$  as the product of the first  $n$  natural numbers.  
%That is,
%\[
%n! = 1 \cdot 2 \cdot  \cdots  \cdot n.
%\]
We also defined  $0!$  to be equal to 1.  Define a sequence of numbers  $a_0 ,a_1 ,a_2 , \ldots $  as follows:
%
\begin{center}
\fbox{\parbox{4in}{
\begin{list}{}
\item $a_0  = 1$, and 
\item 
\item for each nonnegative integer  $n$,  $a_{n + 1}  = \left( {n + 1} \right) \cdot a_n $.
\end{list}
}}
\end{center}
%
\vskip10pt
\noindent
Using $n=0$, we see that this implies that $a_1 =1 \cdot a_0 = 1 \cdot 1 = 1$\!.

\begin{enumerate}
\item Calculate  $a_2, a_3, a_4, a_5$, and $a_6 $.

\item Do you think that it is possible to calculate  $a_{20} $ and  $a_{100} $\!?  Explain.

\item Do you think it is possible to calculate  $a_n $ for any natural number  $n$?  Explain.

\item Compare the values of  $a_0 ,a_1 ,a_2 ,a_3 ,a_4 ,a_5$, and $a_6 $ with those of  \\$0!,1!,2!,3!,4!,5!$, and $6!$.  What do you observe?
\end{enumerate}

The boxed formulas in this preview activity provide another way to define  $n!$  for each nonnegative integer  $n$.  This is an example of a \textbf{recursive definition.}
\index{recursive definition}%
\index{definition!recursive}%
    In a recursive definition of a sequence, the value of a beginning term is specified (or beginning terms are specified), and succeeding terms are defined in terms of the previous values.  In this case,  $a_0 $  is specified to be 1, and  for each nonnegative integer  $n$,  $a_{n + 1} $  is defined in terms of  $a_n $.  The relation that defines the succeeding terms is often called the \textbf{recurrence relation}
\index{recurrence relation}%
 for the recursive definition.

\end{previewactivity}
\hbreak
%
\begin{previewactivity}[The Fibonacci Numbers] \label{PA:fibonaccinumbers} \hfill \\
The \textbf{Fibonacci numbers}
\index{Fibonacci numbers}%
 are a sequence of natural numbers  \\$f_1 ,f_2 ,f_3 , \ldots ,f_n , \ldots $ 
\label{sym:fibonacci} defined as follows:
\begin{itemize}
\item $f_1  = 1$ and  $f_2  = 1$, and

\item For each natural number  $n$,   $f_{n + 2}  = f_{n + 1}  + f_n $.
\end{itemize}
%
This means that
\[
\begin{aligned}
f_3  &= f_2  + f_1  = 1 + 1 = 2, \\
f_4  &= f_3  + f_2  = 2 + 1 = 3, \text{ and} \\
f_5  &= f_4  + f_3 = 3 + 2 = 5. \\
\end{aligned}
\]

%\begin{itemize}
%\item $f_3  = f_2  + f_1  = 1 + 1 = 2$,
%\item $f_4  = f_3  + f_2  = 2 + 1 = 3$, and
%\item $f_5  = f_4  + f_3 = 3 + 2 = 5$.
%\end{itemize}
%
\begin{enumerate}
\item Calculate  $f_6$ through $f_{20}$.

%\item Now calculate  $f_{11} $ through  $f_{20} $.

\item Record any observations about the values of the Fibonacci numbers or any patterns that you observe in the sequence of Fibonacci numbers.  (For example, which terms are even?  Which terms are odd?  Which terms are multiples of 3?)  If necessary, compute more Fibonacci numbers.

\end{enumerate}
\end{previewactivity}
\hbreak
%
\begin{previewactivity}[Recursively Defined Sequences] \label{PA:recursivesequences} \hfill
\begin{enumerate}
\item Define a sequence recursively as follows: \label{PA:recursivesequences1}

\begin{list}{}
\item $b_1  = 16$, and	for each  $n \in \mathbb{N}$,  $b_{n + 1}  = \dfrac{1}{2}b_n $.
\end{list}

Calculate  $b_2 $ through  $b_{10} $. What seems to be happening to the values of  $b_n $
as  $n$  gets larger?

\item Define a sequence recursively as follows: \label{PA:recursivesequences2}

\begin{list}{}
\item $T_1  = 16$, and	for each  $n \in \mathbb{N}$,  $T_{n + 1}  = 16 + \dfrac{1}{2}T_n $.
\end{list}

Calculate  $T_2 $ through  $T_{10} $.  What seems to be happening to the values of  $T_n $
as  $n$  gets larger?

\end{enumerate}

\noindent
The sequences in Parts~(\ref{PA:recursivesequences1}) and~(\ref{PA:recursivesequences2}) can be generalized as follows:  Let  $a$  and  $r$  be real numbers.  Define two sequences recursively as follows:

\begin{list}{}
\item $a_1  = a$, and for each  $n \in \mathbb{N}$,  $a_{n + 1}  = r \cdot a_n $.

\item $S_1  = a$, and for each  $n \in \mathbb{N}$,  $S_{n + 1}  = a + r \cdot S_n $.
\end{list}

\begin{enumerate}
\setcounter{enumi}{2}
\item Determine  formulas (in terms of  $a$  and  $r$) for  $a_2 $ through  $a_6 $.  What do you think  $a_n $ is equal to (in terms of  $a$, $r$, and  $n$)?

\item Determine  formulas (in terms of  $a$  and  $r$) for  $S_2 $ through  $S_6 $.  What do you think  $a_n $ is equal to (in terms of  $a$, $r$, and  $n$)?

\end{enumerate}

\end{previewactivity}
\hbreak






\endinput

