\begin{previewactivity}[\textbf{Functions and Sets of Ordered Pairs}] \label{PA:functionasordered} \hfill \\
When we graph a real function, we plot ordered pairs in the Cartesian plane where the first coordinate is the input of the function and the second coordinate is the output of the function.  For example, if  $g\x \mathbb{R} \to \mathbb{R}$, then every point on the graph of  $g$  is an ordered pair  $( {x, y} )$ of real numbers where  $y = g( x )$.  This shows how we can generate ordered pairs from a function.  It happens that we can do this with any function.
For example, let  
\[
A = \left\{ {1, 2, 3} \right\} \text{ and } B = \left\{ {a, b} \right\}.
\]
Define the function  $F\x A \to B$ by  
\[
  F( 1 ) = a, \quad  
  F( 2 ) = b, \quad \text{ and} \quad 
  F( 3 ) = b. 
\]

%\[
%\begin{aligned}
%  F( 1 ) &= a, \\ 
%  F( 2 ) &= b,\text{ and} \\ 
%  F( 3 ) &= b. \\ 
%\end{aligned}
%\]
We can convert each of these to an ordered pair in  $A \times B$ by using the input as the first coordinate and the output as the second coordinate.  For example, 
$F( 1 ) = a$ is converted to $( 1, a )$, 
$F( 2 ) = b$ is converted to $( 2, b )$, and 
$F( 3 ) = b$ is converted to $( 3, b )$.  So we can think of this function as a set of ordered pairs, which is a subset of  $A \times B$, and write
\[
F = \left\{ {( {1, a} ), ( {2, b} ), ( {3, b} )} \right\}\!.
\]

\noindent
\note  Since  $F$  is the name of the function, it is customary to use  $F$  as the name for the set of ordered pairs.
%Any function  $f\x A \to B$ can be thought of as a set of ordered pairs that is a subset of  
%$A \times B$.  This subset is
%\begin{multicols}{2}
%$f = \left\{ { {( {a, f( a )} ) } \mid a \in A} \right\}$, \qquad or
% 
%$f = \left\{ {( {a, b} ) \in A \times B   \mid b = f( a )} \right\}$.
%\end{multicols}
%\noindent
%\underline{Note}:  Since  $f$  is the name of the function, it is customary to use  $f$  as the name for the set of ordered pairs.
%\vskip10pt
\noindent
 \begin{enumerate}
\item Let  $A = \left\{ {1, 2, 3} \right\}$ and let  $C = \left\{ {a, b, c, d} \right\}$.  Define the function  $g\x A \to C$ by  $g( 1 ) = a$, $g( 2 ) = b$, and 
$g( 3 ) = d$.  Write the function  $g$  as a set of ordered pairs in  $A \times C$. 
\end{enumerate}
For another example, if we have a real function, such as  $g\x \mathbb{R} \to \mathbb{R}$  by  $g( x ) = x^2  - 2$, then we can think of  $g$  as the following infinite subset of  $\mathbb{R} \times \mathbb{R}$:
\[
g = \left\{ { {( {x, y} ) \in \mathbb{R} \times \mathbb{R}} \mid y = x^2  - 2} \right\}\!.
\]
We can also write this  as 
$g = \left\{ {( {x, x^2 - 2} )} \mid x \in \mathbb{R} \right\}$.  
%\[
%g = \left\{ {( {x, y} )} \mid y = x^2  - 2 \right\} \quad \text{ or } \quad
%g = \left\{ {( {x, x^2 - 2} )} \mid x \in \mathbb{R} \right\}\!.
%\]


\begin{enumerate} \setcounter{enumi}{1}
\item Let $f\x \Z \to \Z$ be defined by $f(m) = 3m +5$, for all $m \in \Z$.  Use set builder notation to write the function $f$ as a set of ordered pairs, and then use the roster method to write the function $f$ as a set of ordered pairs.
\end{enumerate}
So any function  $f\x A \to B$ can be thought of as a set of ordered pairs that is a subset of  $A \times B$.  This subset is
\[
f = \left\{ { {( {a, f( a )} ) } \mid a \in A} \right\} \quad \text{or} 
\quad f = \left\{ {( {a, b} ) \in A \times B   \mid b = f( a )} \right\}\!.
\]

\noindent
On the other hand, if we started with  $A = \left\{ {1, 2, 3} \right\}$, 
$B = \left\{ {a, b} \right\}$, and 
\[
G = \left\{ {( {1, a} ), ( {2, a} ), ( {3, b} )} \right\} \subseteq A \times B ,
\]
then we could think of  $G$  as a function from  $A$  to  $B$  with  $G( 1 ) = a$, $G( 2 ) = a$, and $G( 3 ) = b.$  The idea is to use the first coordinate of each ordered pair as the input, and the second coordinate as the output.  However, not every subset of  $A \times B$ can be used to define a function from  $A$  to  $B$.  This is explored in the following questions.

\setcounter{oldenumi}{\theenumi}
\begin{enumerate} \setcounter{enumi}{\theoldenumi}
\item Let  $f = \left\{ {( {1, a} ), ( {2, a} ), ( {3, a} ), ( {1, b} )} \right\}$. Could this set of ordered pairs be used to define a function from  $A$  to  $B$?  Explain.

\item Let $g = \left\{ {( {1, a} ), ( {2, b} ), ( {3, a} )} \right\}$.  Could this set of ordered pairs be used to define a function from  $A$  to  $B$?  Explain.

\item Let $h = \left\{ {( {1, a} ), ( {2, b} )} \right\}$.  Could this set of ordered pairs be used to define a function from  $A$  to  $B$?  Explain.
\end{enumerate}
\hbreak

\end{previewactivity}


\endinput

