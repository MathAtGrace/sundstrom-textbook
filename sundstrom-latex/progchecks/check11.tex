\section*{Section~\ref{S:prop}}

\subsection*{Progress Check \ref{prog:statements2}}
In the list of sentences,  1, 3, 4, 6, 8, 9 are statements.

\subsection*{Progress Check \ref{pr:explores}}
\begin{enumerate}
\item %$\left( {a + b} \right)^2  = a^2  + b^2$  for all real numbers  $a$  and  $b$.
This proposition is false.  A counterexample is  $a = 2$ and $b = 1$.  For these values,
$\left( {a + b} \right)^2 = 9$ and $a^2 + b^2 = 5$.

\item %There are integers  $x$  and  $y$  such that  $2x + 5y = 41$.  
This proposition is true, as we can see by using  $x = 3$ and  $y = 7$.  We could also use  
$x =  -2$ and  $y = 9$.  There are many other possible choices for  $x$  and  $y$.

\item %If  $x$ is an even integer, then $x^2$ is an even integer.
This proposition appears to be true.  Anytime we use an example where  $x$  is  an even integer, the number  $x^2$ is an even integer.  However, we cannot claim that this is true based on examples since we cannot list all of the examples where  $x$  is  an even integer.

\item %If  $x$ is an even integer, then $x^2$ is an even integer.
This proposition appears to be true.  Anytime we use an example where  $x$ and $y$ are both integers, the number  $x \cdot y$ is an odd integer.  However, we cannot claim that this is true based on examples since we cannot list all of the examples where both $x$ and $y$ are odd integers.
\end{enumerate}


\subsection*{Progress Check \ref{prog:condition}}
\begin{enumerate}
  \item \begin{enumerate}
      \item  This does not mean the conditional statement is false since when $x = -3$, the hypothesis is false, and the only time a conditional statement is false is when the hypothesis is true and the conclusion is false.
      \item This does not mean the conditional statment is true since we have not checked all positive real numbers, only the one where $x = 4$.
      \item All examples should indicate that the conditional statement is true.
  \end{enumerate}
  \item The number $\left( n^2 - n + 41 \right)$ will be a prime number for all examples of $n$ that are less than 41.  However, when $n = 41$, we get
\begin{align*}
  n^2  - n + 41 &= 41^2  - 41 + 41 \\ 
  n^2  - n + 41 &= 41^2 
\end{align*}
So in the case where  $n = 41$, the hypothesis is true  (41 is a positive integer) and the conclusion is false $\left( {41^2 \text{ is not prime}} \right)$.  Therefore, 41 is a counterexample that shows the conditional statement is false.  There are other counterexamples (such as $n = 42$, $n = 45$, and  $n = 50$), but only one counterexample is needed to prove that the statement is false.
\end{enumerate}



\subsection*{Progress Check \ref{pr:conditional}}
\begin{enumerate}
\item We can conclude that this function is continuous at 0.
\item We can make no conclusion about this function from the theorem.
\item We can make no conclusion about this function from the theorem.
\item We can conclude that this function is not differentiable at 0.
\end{enumerate}



\subsection*{Progress Check \ref{pr:closure}}
\begin{enumerate}
\item The set of rational numbers is closed under addition since 
$\dfrac{a}{b} + \dfrac{c}{d} = \dfrac{ad+bc}{bd}$.

\item The set of integers is not closed under division.  For example, $\dfrac{2}{3}$ is not an integer.

\item The set of rational numbers is closed under subtraction since 
$\dfrac{a}{b} - \dfrac{c}{d} = \dfrac{ad-bc}{bd}$.
\end{enumerate}
\hbreak

\endinput
