\section*{Section~\ref{S:logequiv}}

\subsection*{Progress Check~\ref{pr:workingeq2}}
\begin{enumerate}
%\item The two truth tables show that  $\left( {P \wedge \mynot  Q} \right) \to R$
 % is logically equivalent to  $P \to \left( {Q \vee R} \right)$.
\item Starting with the suggested equivalency, we obtain 
\begin{align*}
\left( {P \wedge \mynot  Q} \right) \to R &\equiv \mynot \left( P \wedge \mynot Q \right) \vee R \\
                  &\equiv \left( \mynot P \vee \mynot \left( \mynot Q \right) \right) \vee R \\
                  &\equiv \mynot P \vee \left( Q \vee R \right) \\
                  &\equiv P \to \left( Q \vee R \right)
\end{align*}
\item For this, let  $P$  be, ``3 is a factor of  $a \cdot b$,'' let  $Q$  be, ``3 is a factor of  $a$,'' and let  $R$  be, ``3 is a factor of  $b$.''  Then the stated proposition is written in the form  $P \to \left( {Q \vee R} \right)$.  Since this is logically equivalent to  
$\left( {P \wedge \mynot  Q} \right) \to R$, if we prove that

\begin{center}
if  3  is a factor of   $a \cdot b$ and  3  is not a factor of   $a$, then  3  is a factor of $b$,
\end{center}

then we have proven the original proposition.
\end{enumerate}
\hbreak

\endinput
