\section*{Section~\ref{S:mathinduction}}

\subsection*{Progress Check \ref{prog:inductivesets}}
\begin{enumerate}
\item It is not possible to tell if $1 \in T$ and $5 \in T$.
\item True.
\item True.  The contrapositive is, ``If $2 \in T$, then $5 \in T$,'' which is true.
\item True.
\item True, since ``$k \notin T$ or $k + 1 \in T$'' is logically equivalent to ``If $k \in T$, then 
$k + 1 \in T$.''
\item False.  If $k \in T$, then $k + 1 \in T$.
\item It is not possible to tell if this is true.  It is the converse of the conditional statement, ``For each integer $k$, if $k \in T$, then $k + 1 \in T$.''
\item True.  This is the contrapositive of the conditional statement, ``For each integer $k$, if $k \in T$, then $k + 1 \in T$.''
\end{enumerate}



\subsection*{Progress Check \ref{prog:indexample}}
\begin{myproof}
Let $P( n )$ be the predicate, 
``$1 + 2 + 3 +  \cdots  + n = \dfrac{n \left( n + 1 \right)}{2}$.''  For the basis step, notice that the equation $1 = \dfrac{1 \left( 1 + 1 \right)}{2}$ shows that 
$P( 1 )$ is true.  Now \vskip3pt \noindent
 let $k$ be a natural number and assume that 
$P(k )$ is true.  That is, assume that
\setcounter{equation}{0}
\begin{equation}
1 + 2 + 3 +  \cdots  + k = \frac{k \left( k + 1 \right)}{2}. \label{prog:indexample1}
\end{equation}
We now need to prove that $P(k +1 )$ is true or that
\begin{equation}
1 + 2 + 3 +  \cdots  + k + \left( k + 1 \right) = 
\frac{\left( k + 1 \right) \left(k + 2 \right)}{2}. \label{prog:indexample2}
\end{equation}
By adding $\left(k + 1 \right)$ to both sides of equation~(\ref{prog:indexample1}), we see that
\[
\begin{aligned}
1 + 2 + 3 +  \cdots  + k + \left( k + 1 \right) &= 
\frac{k \left( k + 1 \right)}{2} + \left( k + 1 \right) \\
                                 &= \frac{k \left( k + 1 \right) + 2 \left( k + 1 \right)}{2} \\
                                 &= \frac{k^2 + 3k + 2}{2} \\
                                 &= \frac{ \left(k + 1 \right) \left( k + 2 \right)}{2}.
\end{aligned}
\]
By comparing the last equation to equation~(\ref{prog:indexample2}), we see that we have proved that if $P(k )$ is true, then $P( k + 1 )$ is true, and the inductive step has been established.  Hence, by the Principle of Mathematical Induction, we  have proved that for each integer $n$, $1 + 2 + 3 +  \cdots  + n = \dfrac{n \left( n + 1 \right)}{2}$.
\end{myproof}



\subsection*{Progress Check~\ref{P:congruenceinduction}}
For the inductive step, let $k$ be a natural number and assume that $P(k)$ is true.  That is, assume that $\mod{ 5^k }{1}{4}$.
\begin{enumerate}
  \item To prove that $P(k+1)$ is true, we must prove $\mod{5^{k+1}}{1}{4}$.
  \item Since $5^{k+1} = 5 \cdot 5^k$, we multiply both sides of the congruence $\mod{ 5^k }{1}{4}$ by 5 and obtain
\[
\mod{5 \cdot 5^k}{5 \cdot 1}{4} \quad \text{or} \quad \mod{5^{k+1}}{5}{4}.
\]
  \item Since $\mod{5^{k+1}}{5}{4}$ and we know that $\mod{5}{1}{4}$, we can use the transitive property of congruence to obtain $\mod{5^{k+1}}{1}{4}$.  This proves that if $P(k)$ is true, then $P(k+1)$ is true, and hence, by the Principle of Mathematical Induction, we have proved that for each natural number $n$, 
$\mod{5^n}{1}{4}$.
\end{enumerate}
\hbreak

\endinput
