\section*{Section ~\ref{S:infinitesets}}

\subsection*{Progress Check~\ref{E:infinitesets}}
\begin{enumerate}
\item The set of natural numbers $\N$ is a subset of $\Z$, $\Q$, and $\R$.  Since $\N$ is an infinite set, we can use Part~(2) of Theorem~\ref{T:subsetisinfinite} to conclude that $\Z$, 
$\Q$, and $\R$ are infinite sets.

\item Use Part~(1) of Theorem~\ref{T:subsetisinfinite}.

\item Prove that $E^+ \approx \N$ and use Part~(1) of Theorem~\ref{T:subsetisinfinite}.
\end{enumerate}



\subsection*{Progress Check~\ref{prog:countablyinfinitesets}}
\begin{enumerate}
\item Use the definition of a countably infinite set.

\item Since $E^+ \approx \N$, we can conclude that $\text{card} ( E^+ ) = \aleph_0$.

\item One function that can be used is $f\x S \to \N$ defined by $f(m) = \sqrt{m}$ for all 
$m \in S$.
\end{enumerate}


\hbreak

\endinput
