\section*{Section~\ref{S:directproof}}


\subsection*{Progress Check \ref{pr:divisors}}
\begin{enumerate} \setcounter{enumi}{1}
%\item Four examples of integers  $a$, $b$, and  $c$ such that  $a$  divides  $b$  and  $a$  divides  $c$.
%
%\begin{center}\begin{tabular}{l l l}
%-2  divides  8	& and	& -2  divides  -32. \\
%5   divides  -10	& and	& 5   divides  20. \\
%1   divides  10	& and	& 1   divides  30. \\
%6   divides  12	& and	& 6   divides  24. \\
%\end{tabular}
%\end{center}

\item For each example in Part (1), the integer  $a$  divides the sum  $b + c$.

\item Conjecture:  For all integers $a$, $b$, and  $c$  with $a \ne 0$,  if $a$  divides  $b$  and  $a$  divides  $c$, then  $a$  divides  $b + c$.

\item A Know-show table for a proof of the conjecture in Part (3).
\begin{center}
\begin{tabular}[h]{|p{0.4in}|p{2in}|p{1.8in}|}
  \hline
  \textbf{Step}  &  \textbf{Know}  &  \textbf{Reason}     \\ \hline
  $P$     &  $a \mid b$ and  $a \mid c$     &  Hypothesis \\ \hline
  $P1$    &  $\left( {\exists s \in \mathbb{Z}} \right)\left( {b = a \cdot s} \right)$

$\left( {\exists t \in \mathbb{Z}} \right)\left( {c = a \cdot t} \right)$                               &  Definition of ``divides''           \\ \hline
  $P2$    &     $b + c = as + at$     & Substituting for $b$ and $c$   \\  \hline
  $P3$    &     $b + c = a\left( {s + t} \right)$     & Distributive property   \\  \hline  
  $Q1$    &     $s + t$ is an integer     & $\mathbb{Z}$ is closed under addition   \\  \hline  
  
  $Q$     &  $\left. {a } \right| \left( {b + c} \right)$  &  Definition of ``divides''    \\ \hline
  \textbf{Step}  &  \textbf{Show}  &  \textbf{Reason}     \\ \hline
\end{tabular}
\end{center}
\end{enumerate}


\subsection*{Progress Check~\ref{pr:counterexample}}
A counterexample for this statement will be values of $a$ and $b$ for which 5 divides $a$ or 5 divides $b$, and 5 does not divide $5a + b$.  One counterexample for the statement is $a = 5$ and $b = 1$.  For these values, the hyothesis is true since 5 divides $a$ and the conclusion is false since $5a + b = 26$ and 5 does not divide 26.  

\subsection*{Progress Check \ref{pr:congruence}}
\begin{enumerate}
\item Some integers that are congruent to 5 modulo 8 are $-11, -3, 5, 13$, and 21.

\item $\left\{ \left. x \in \Z \right| x \equiv 5 \pmod 8 \right\} =
\left\{ \ldots, -19, -11, -3, 5, 13, 21, 29, \ldots \right\}$.

\item For example, $-3 + 5 = 2$, $-11 + 29 = 18$, $13 + 21 = 34$. \label{pr:congans3}

\item If we subtract 2 from any of the sums obtained in Part~(\ref{pr:congans3}), the result will be a multiple of 8.  This means that the sum is congruent to 2 modulo 8.  For example, 
$2 - 2 = 0$, $18 - 2 = 16$, $34 - 2 = 32$.
\end{enumerate}


\subsection*{Progress Check~\ref{pr:congruence2}}
\begin{enumerate}
  \item To prove that 8 divides $(a + b - 2)$, we can prove that there exists an integer $q$ such that $(a + b - 2) = 8q$.
  \item Since 8 divides $(a - 5)$ and $(b - 5)$, there exist integers $k$ and $m$ such that 
$a - 5 = 8k$ and $b - 5 = 8m$.
  \item $a = 5 + 8k$ and $b = 5 + 8m$.
  \item $a + b - 2 = (5 + 8k) + (5 + 8m) - 2 = 8 + 8k + 8m = 8(1 + k + m)$.
  \item 
\begin{myproof}
Let $a$ and $b$ be integers and assume that $\mod{a}{5}{8}$ and $\mod{b}{5}{8}$.  We will prove that $\mod{(a + b)}{2}{8}$.  Since 8 divides $(a - 5)$ and $(b - 5)$, there exist integers $k$ and $m$ such that $a - 5 = 8k$ and $b - 5 = 8m$.  We then see that
\begin{align*}
a + b - 2 &= (5 + 8k) + (5 + 8m) - 2 \\
      &= 8 + 8k + 8m \\
      &= 8(1 + k + m)
\end{align*}
By the closure properties of the integers, $(1 + k + m)$ is an integer and so the last equation proves that 8 divides $(a + b - 2)$ and hence, $\mod{(a + b)}{2}{8}$.  This proves that if 
$\mod{a}{5}{8}$ and $\mod{b}{5}{8}$, then $\mod{(a + b)}{2}{8}$.
\end{myproof}
\end{enumerate}
\hbreak


\endinput

