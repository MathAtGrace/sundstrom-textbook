\section*{Section \ref{S:primefactorizations}}

\subsection*{Progress Check~\ref{prog:relativelyprime}}
\begin{enumerate}
\item If  $a, p \in \mathbb{Z}$, $p$  is prime, and  $p$ divides  $a$, then  
$\gcd ( {a, p} ) = p$.

\item If  $a, p \in \mathbb{Z}$, $p$  is prime, and  $p$ does not divide  $a$, then  
$\gcd ( {a, p} ) = 1$.

\item Three examples are  $\gcd ( {4, 9} ) = 1$,  
$\gcd ( {15, 16} ) = 1$,  $\gcd ( {8, 25} ) = 1$.
\end{enumerate}


\subsection*{Progress Check~\ref{prog:relativelyprimeprop}}
\setcounter{equation}{0}
\noindent
\textbf{Theorem~\ref{T:relativelyprimeprop}}.  Let  $a$, $b$, and  $c$ be integers.  If  $a$  and  $b$  are relatively prime  and  $a \mid ( {bc} )$, then  $a \mid c$.

\begin{myproof}
Let  $a$, $b$, and  $c$ be integers.  Assume that  $a$  and  $b$  are relatively prime  and  
$a \mid ( {bc} )$.  We will prove that  $a$  divides  $c$.

Since  $a$  divides  $bc$, there exists an integer  $k$  such that
\begin{equation} \label{eq:814a}
bc = ak.
\end{equation}
In addition, we are assuming that  $a$  and  $b$  are relatively prime  and hence  
$\gcd ( {a, b} ) = 1$.  So by Theorem~\ref{T:gcddivideslincombs}, there exist integers  $m$  and  $n$  such that
\begin{equation} \label{eq:814b}
am + bn = 1.
\end{equation}
We now multiply both sides of equation~(\ref{eq:814b}) by  $c$.  This gives
\begin{align}
  ( {am + bn} )c &= 1 \cdot c \notag \\ 
                  acm + bcn &= c \label{eq:814c} 
\end{align} 
We can now use equation~(\ref{eq:814a}) to substitute  $bc = ak$ in equation~(\ref{eq:814c}) and obtain
\[
acm + akn = c.
\]
If we now factor the left side of this last equation, we see that  
$a( {cm + kn} ) = c$.  Since  $( {cm + kn} )$ is an integer, this proves that  $a$  divides  $c$.  Hence, we have proven that if  $a$  and  $b$  are relatively prime  and  $a \mid ( {bc} )$, then  $a \mid c$.
\end{myproof}

\hbreak

\endinput
