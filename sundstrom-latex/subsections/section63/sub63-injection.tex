\subsection*{Injections}
In previous sections and in \typeu Activity~\ref*{PA:functionswithfinitedom}, we have seen examples of functions for which there exist different inputs that produce the same output.   Using more formal notation, this means that there are functions  $f:A \to B$ for which there exist  \mbox{$x_1 , x_2  \in A$}
 with  $x_1  \ne x_2 $  and  $f( {x_1 } ) = f( {x_2 } )$.    
%What does it mean to say that this does not happen?  This was explored in Beginning Activity~\ref{PA:functionstatements}, and using the input-output version of a function, it means that if the inputs are different then the outputs are different.  More formally, it means that  if  $x_1  \ne x_2 $, then  
%$f( {x_1 } ) \ne f( {x_2 } )$.  
The work in the \typel activities was intended to motivate the following definition.
%
\begin{defbox}{injection}{Let  $f:A \to B$  be a function from the set  $A$  to the set  $B$.  The function  $f$  is called an \textbf{injection}
\index{injection}%
 provided that
\begin{center}
for all  $x_1 , x_2  \in A$, if  $x_1  \ne x_2 $, then  
$f( {x_1 } ) \ne f( {x_2 } )$.
\end{center}
When  $f$  is an injection, we also say that  $f$  is a \textbf{one-to-one function},
\index{one-to-one function}%
\index{function!one-to-one}%
 or that  $f$  is an \textbf{injective function}.}
\index{function!injective}%

\end{defbox}
%
Notice that the condition that specifies that a function  $f$  is an injection is given in the form of a conditional statement.  As we shall see, in proofs, it is usually easier to use the contrapositive of this conditional statement.  Although we did not define the term then, we have already written the contrapositive for the conditional statement in the definition of an injection in Part~(\ref{PA:functionstatements1}) of \typeu Activity~\ref*{PA:functionstatements}.  In that activity, we also wrote the negation of the definition of an injection.  Following is a summary of this work giving the conditions for  $f$  being an injection or not being an injection.

\begin{center}
\fbox{\parbox{4.68in}{
\begin{center}
\textbf{Let } $\boldsymbol{f\x A \to B}$\!.
\end{center}
\textbf{``The function $\boldsymbol{f}$ is an injection'' means that}
\begin{itemize}
\item For all  $x_1 , x_2  \in A$, if  $x_1  \ne x_2 $, then  
$f( {x_1 } ) \ne f( {x_2 } )$; or

\item For all  $x_1 , x_2  \in A$, if  $f( {x_1 } ) = f( {x_2 } )$, then  $x_1  = x_2 $.
\end{itemize}

\noindent
\textbf{``The function $\boldsymbol{f}$ is not an injection'' means that}
\begin{itemize}
\item There exist  $x_1 , x_2  \in A$ such that  $x_1  \ne x_2 $  and  
$f( {x_1 } ) = f( {x_2 } )$.
\end{itemize}
}}
\end{center}
%
%We will see how to use these ideas in the activities and examples that follow the discussion of surjections.
\begin{prog}[\textbf{Working with the Definition of an Injection}] \label{pr:injections} \hfill \\
Now that we have defined what it means for a function to be an injection, we can see that in 
Part~(\ref{PA:functionstatements3}) of \typeu Activity~\ref*{PA:functionstatements}, we proved that the function $g\x \R \to \R$ is an injection, where $g ( x ) = 5x + 3$ for all $x \in \R$.  Use the definition (or its negation) to determine whether or not the following functions are injections.
\begin{enumerate}
\item $k:A \to B$, where $A = \left\{a, b, c \right\}$, $B = \left\{1, 2, 3, 4 \right\}$, and 
$k (a) = 4$, $k(b) = 1$, and $k(c) = 3$.

\item $f:A \to C$, where $A = \left\{a, b, c \right\}$, $C = \left\{1, 2, 3\right\}$, and 
$f(a) = 2$, $f(b) = 3$, and $f(c) = 2$.

\item $F: \Z \to \Z$ defined by $F ( m ) = 3m + 2$ for all $m \in \Z$

\item $h: \R \to \R$ defined by $h ( x ) = x^2 - 3x$ for all $x \in \R$

\item $R_5 = \{0, 1, 2, 3, 4 \}$ and $s:R_5 \to R_5$ defined by $s ( x ) = x^3 \pmod 5$ for all $x \in R_5$.

\end{enumerate}
\end{prog}
\hbreak

\endinput
