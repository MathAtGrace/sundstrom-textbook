\subsection*{The Cardinality of a Finite Set}
In our discussion of the power set, we were concerned with the number of elements in a set.  In fact, the number of elements in a finite set is a distinguishing characteristic of the set,  so we give it the following name. 
%
\begin{defbox}{D:cardinality}{The number of elements in a finite set  $A$ is called the \textbf{cardinality}
\index{cardinality}%
\index{cardinality!finite set}%
 of  $A$  and is denoted by  $\card(A)$.} \label{sym:finitecard}
\end{defbox}
%
\noindent
For example,  $\card (\emptyset) = 0$; \qquad	
$\card (\left\{ {a,b} \right\}) = 2$; \qquad	
$\card \left( \mathcal{P}( \left\{ {a,b} \right\}) \right) = 4$.


%\textbf{A Word about Notation:}  We are using the notation  $\left| A \right|$ to denote the cardinality of a set  $A$.  Do not confuse this with the absolute value of a real number.  It is common practice in mathematics to use the same notation for two different concepts when there is little chance of confusion.  We must be careful to use (and understand) the notation in the proper context.  In set theory, $\left| A \right|$ is the cardinality of the set  $A$, and in the algebra of real numbers, $\left| x \right|$ is the absolute value of the real number  $x$.  
\begin{flushleft}
\textbf{Theoretical Note:}  There is a mathematical way to distinguish between finite and infinite sets, and there is a way to define the cardinality of an infinite set.  We will not concern ourselves with this at this time.  More about the cardinality of finite and infinite sets is discussed in Chapter~\ref{C:topicsinsets}.
\end{flushleft}

\endinput
