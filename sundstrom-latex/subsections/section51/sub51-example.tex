\subsection*{An Example of Using Set Operations} \label{ss:example-sets}
Following is an example that uses the set operations discussed in Beginning Activity~\ref{PA:setops} where 
we used 
\[
  U = \left\{ {0,1,2,3, \ldots ,10} \right\}\!, \quad  A = \left\{ {0,1,2,9} \right\}\!, \quad  B = \left\{ {2,3,4,5,6} \right\}\!. 
\]
So in this case,

\[
\begin{aligned}
A \cap B &= \left\{ {x \in U \left| {x \in A\text{  and  }x \in B} \right.} \right\} = \left\{ 2 \right\} \\
%  & \\
A \cup B &= \left\{ {x \in U\,\left| {x \in A\text{  or  }x \in B} \right.} \right\} = \left\{ {0,1,2,3,4,5,6,9} \right\} \\
%  &  \\
A^c  &= \left\{ {\left. {x \in U\,} \right|x \notin A} \right\} = \left\{ {3,4,5,6,7,8,10} \right\} \\
%  &  \\
A - B &= \left\{ {\left. {x \in A} \right|x \notin B} \right\} = \left\{ {0,1,9} \right\}\!. \\
\end{aligned}
\]
We can also use these definitions in combinations to form other sets.  For example,
\[
\begin{aligned}
B^c  &= \left\{ {\left. {x \in U} \right|x \notin B} \right\} = \left\{ {0,1,7,8,9,10} \right\} \\
%  &  \\
A^c  \cup  B^c  &= \left\{ {\left. {x \in U} \right|x \in A^c \text{  or  }x \in B^c } \right\} = \left\{ {0,1,3,4,5,6,7,8,9,10} \right\}  \\
%  &  \\
\left( {A \cap B} \right)^c  &= \left\{ {x \in U\left.  \right|x \notin A \cap B} \right\} = \left\{ {0,1,3,4,5,6,7,8,9,10} \right\}\!. \\
\end{aligned}
\]
%
\hbreak
\endinput

