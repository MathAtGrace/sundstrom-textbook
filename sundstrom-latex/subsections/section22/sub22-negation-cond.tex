\subsection*{The Negation of a Conditional Statement}
\index{negation!of a conditional statement}%
The logical equivalency $\mynot  \left( {P \to Q} \right) \equiv P \wedge \mynot  Q$ is interesting because it shows us that \textbf{the negation of a conditional statement is not another conditional statement.}
\index{conditional statement!negation}%
  The negation of a conditional statement can be written in the form of a conjunction.
%\end{flushleft}
So what does it mean to say that the conditional statement 
\begin{center}
If you do not clean your room, then you cannot watch TV,
\end{center}
is false?  To answer this, we can use the logical equivalency $\mynot  \left( {P \to Q} \right) \equiv P \wedge \mynot  Q$.  The idea is that if $P \to Q$ is false, then its negation must be true.  So the negation of this 
%
can be written as
\begin{center}
You do not clean your room and you can watch TV.
\end{center}
%
For another example, consider the following conditional statement:
%
\begin{center}
If $-5 < -3$, then $\left( -5 \right)^2 < \left( -3 \right)^2$.
\end{center}
%
This conditional statement is false since its hypothesis is true and its conclusion is false.  Consequently, its negation must be true.  Its negation is not a conditional statement.  The negation can be written in the form of a conjunction by using the logical equivalency 
$\mynot  \left( {P \to Q} \right) \equiv P \wedge \mynot  Q$.  So, the negation can be written as follows:
%
\begin{center}
$-5 < -3$  and  $\mynot \left( {\left( -5 \right)^2 < \left( -3 \right)^2} \right)$.
\end{center}
%
However, the second part of this conjunction can be written in a simpler manner by noting that ``not less than'' means the same thing as ``greater than or equal to.''  So we use this to write the negation of the original conditional statement as follows:
%
\begin{center}
$-5 < -3$  and  $\left( -5 \right)^2 \geqslant \left( -3 \right)^2$.
\end{center}
This conjunction is true since each of the individual statements in the conjunction is true.
\hbreak

\endinput
