\subsection*{Properties of Union and Intersection}
In Theorem~\ref{T:indexproperties}, we will prove some properties of set operations for indexed families of sets.  Some of these properties are direct extensions of corresponding properties for two sets.  For example, we have already proved De Morgan's Laws for two sets in 
Theorem~\ref{T:propsofcomplements} on page~\pageref{T:propsofcomplements}.  The work in the \typel activities and Progress Check~\ref{prog:indexfamily2} suggests that we should get similar results using set operations with an indexed family of sets.  For example, in \typeu Activity~\ref*{PA:indexfamily}, we saw that
\[
\left( \bigcap_{j=1}^{4}A_j \right)^c = \bigcup_{j=1}^{4}A_j^c.
\]

\begin{theorem} \label{T:indexproperties}
Let $\Lambda$ be a nonempty indexing set and let 
$\mathscr{A} = \left\{ A_\alpha \mid \alpha \in \Lambda \right\}$ be an indexed family of sets, each of which is a subset of some universal set $U$.  Then
\begin{enumerate}
\item For each $\beta \in \Lambda$, $\bigcap\limits_{\alpha \in \Lambda}^{}A_\alpha \subseteq A_\beta$. \label{T:indexproperties1}
\item For each $\beta \in \Lambda$, $A_\beta \subseteq \bigcup\limits_{\alpha \in \Lambda}^{}A_\alpha$. \label{T:indexproperties2}
\item $\left(\:\bigcap\limits_{\alpha \in \Lambda}^{}A_\alpha \right)^c = 
\bigcup\limits_{\alpha \in \Lambda}^{}A_{\alpha}^c$ \label{T:indexproperties3}
\item $\left(\:\bigcup\limits_{\alpha \in \Lambda}^{}A_\alpha \right)^c = 
\bigcap\limits_{\alpha \in \Lambda}^{}A_{\alpha}^c$ \label{T:indexproperties4}
\end{enumerate}
Parts~(\ref{T:indexproperties3}) and~(\ref{T:indexproperties4}) are known as 
\textbf{De Morgan's Laws}.
\index{De Morgan's Laws!for indexed family of sets}%
\index{indexed family of sets!De Morgan's Laws}%
\end{theorem}

\setcounter{equation}{0}
\begin{myproof}
We will prove Parts~(\ref{T:indexproperties1}) and~(\ref{T:indexproperties3}).  The proofs of Parts~(\ref{T:indexproperties2}) and~(\ref{T:indexproperties4}) are included in  
Exercise~(\ref{exer:indexproperties}).  So we let $\Lambda$ be a nonempty indexing set and let 
$\mathscr{A} = \left\{ A_\alpha \mid \alpha \in \Lambda \right\}$ be an indexed family of sets.  To prove Part~(\ref{T:indexproperties1}), we let $\beta \in \Lambda$ and note that if 
$x \in \bigcap\limits_{\alpha \in \Lambda}^{}A_\alpha$, then $x \in A_\alpha$, for all 
$\alpha \in \Lambda$.  Since $\beta$ is one element in $\Lambda$, we may conclude that 
$x \in A_\beta$.  This proves that 
$\bigcap\limits_{\alpha \in \Lambda}^{}A_\alpha \subseteq A_\beta$.

To prove Part~(\ref{T:indexproperties3}), we will prove that each set is a subset of the other set.  We first let $x \in \left(\:\bigcap\limits_{\alpha \in \Lambda}^{}A_\alpha \right)^c$.  This means that $x \notin \left(\:\bigcap\limits_{\alpha \in \Lambda}^{}A_\alpha \right)$, and this means that \pagebreak
\begin{center}
there exists a $\beta \in \Lambda$ such that $x \notin A_\beta$.
\end{center}
Hence, $x \in A_{\beta}^c$,  which implies that 
$x \in \bigcup\limits_{\alpha \in \Lambda}^{}A_{\alpha}^c$.  Therefore, we have proved that 
\begin{equation} \label{eq:indexproperties1}
\left(\:\bigcap_{\alpha \in \Lambda}^{}A_\alpha \right)^c \subseteq 
\bigcup_{\alpha \in \Lambda}^{}A_{\alpha}^c.
\end{equation}

We now let $y \in \bigcup\limits_{\alpha \in \Lambda}^{}A_{\alpha}^c$.  This means that there exists a $\beta \in \Lambda$ such that $y \in A_{\beta}^c$ or $y \notin A_\beta$.  However, since 
$y \notin A_\beta$, we may conclude that $y \notin \bigcap\limits_{\alpha \in \Lambda}^{}A_\alpha$ and, hence, $y \in \left(\:\bigcap\limits_{\alpha \in \Lambda}^{}A_\alpha \right)^c$.  This proves that
\begin{equation} \label{eq:indexproperties2}
\bigcup_{\alpha \in \Lambda}^{}A_{\alpha}^c \subseteq 
\left(\:\bigcap_{\alpha \in \Lambda}^{}A_\alpha \right)^c.
\end{equation}
Using the results in~(\ref{eq:indexproperties1}) and~(\ref{eq:indexproperties2}), we have proved that $\left(\:\bigcap\limits_{\alpha \in \Lambda}^{}A_\alpha \right)^c = 
\bigcup\limits_{\alpha \in \Lambda}^{}A_{\alpha}^c$.
\end{myproof}
%\hbreak

Many of the other properties of set operations are also true for indexed families of sets.  Theorem~\ref{T:distributeindex} states the \textbf{distributive laws}
\index{distributive laws!for indexed family of sets}%
\index{indexed family of sets!distributive laws}%
 for set operations.

\begin{theorem} \label{T:distributeindex}
Let $\Lambda$ be a nonempty indexing set, let 
$\mathscr{A} = \left\{ A_\alpha \mid \alpha \in \Lambda \right\}$ be an indexed family of sets, and let $B$ be a set.  Then
\begin{enumerate}
\item $B \cap \left(\:\bigcup\limits_{\alpha \in \Lambda}^{}A_{\alpha} \right) 
= \bigcup\limits_{\alpha \in \Lambda}^{} \left( B \cap A_{\alpha} \right)$, and
\item $B \cup \left(\:\bigcap\limits_{\alpha \in \Lambda}^{}A_{\alpha} \right) 
= \bigcap\limits_{\alpha \in \Lambda}^{} \left( B \cup A_{\alpha} \right)$.
\end{enumerate}
\end{theorem}

\noindent
The proof of Theorem~\ref{T:distributeindex} is Exercise~(\ref{exer:distributeindex}).

\endinput
