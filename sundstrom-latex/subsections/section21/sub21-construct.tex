\subsection*{Constructing Truth Tables}
Truth tables for compound statements can be constructed by using the truth tables for the basic connectives.  To illustrate this, we will construct a truth table for  
$\left( {P \wedge \mynot  Q} \right) \to R$.  The first step is to determine the number of rows needed.
\begin{itemize}
  \item For a truth table with two different simple statements, four rows are needed  since there are four different combinations of truth values for the two statements.  We should be consistent with how we set up the rows.  The way we will do it in this text is to label the rows for the first statement with (T, T, F, F) and the rows for the second statement with (T, F, T, F).  All truth tables in the text have this scheme.

  \item For a truth table with three different simple statements, eight rows are needed  since there are eight different combinations of truth values for the three statements.  Our standard scheme for this type of truth table is shown in 
Table~\ref{Ta:compoundtruthtable}.
\end{itemize}


The next step is to determine the columns to be used.  One way to do this is to work backward from the form of the given statement.  For $\left( {P \wedge \mynot  Q} \right) \to R$, the last step is to deal with the conditional operator $\left(  \to  \right)$.  To do this, we need to know the truth values of  $\left( {P \wedge \mynot  Q} \right)$ and  $R$.  To determine the truth values for  $\left( {P \wedge \mynot  Q} \right)$, we need to apply the rules for the conjunction operator $\left(  \wedge  \right)$ and we need to know the truth values for  $P$  and  $\mynot  Q$.

Table~\ref{Ta:compoundtruthtable} is a  completed truth table for  
$\left( {P \wedge \mynot  Q} \right) \to R$ with the step numbers indicated at the bottom of each column.  The step numbers correspond to the order in which the columns were completed.

\begin{table}[h]
$$
\BeginTable
\BeginFormat
| c | c | c | c | c | c |
\EndFormat
\_6
       | $P$  |  $Q$  |  $R$  \|6  $\mynot  Q$  |  $P \wedge \mynot  Q$  |  $\left( P \wedge \mynot  Q \right) \to R$ | \\+22 \_6
          | T | T | T \|6 F | F | T  | \\ 
         | T | T | F \|6 F | F | T  | \\ 
          | T | F | T \|6 T | T | T  | \\ 
           | T | F | F \|6 T | T | F  | \\ 
          | F | T | T \|6 F | F | T  | \\ 
          | F | T | F \|6 F | F | T  | \\ 
          | F | F | T \|6 T | F | T  | \\ 
          | F | F | F \|6 T | F | T |  \\ \_6
 | 1 | 1 | 1 \|6 2 | 3 | 4 |   \\ \_6
\EndTable
$$
\caption{Truth Table for $\left( P \wedge \mynot  Q \right) \to R$}
\label{Ta:compoundtruthtable}%
\end{table}
%\begin{table}[h]
%$$
%\BeginTable
%\BeginFormat
%| l | c | c | c | c | c | c |
%\EndFormat
%"  & \use6 \=6 & \\0
%    "    | $P$  |  $Q$  |  $R$  |  $\mynot  Q$  |  $P \wedge \mynot  Q$  |  $\left( P \wedge \mynot  Q \right) \to R$ | \\+22 
%"  & \use6 \=6 &  \\0
%   "        | T | T | T | F | F | T  | \\ 
%   "        | T | T | F | F | F | T  | \\ 
%   "        | T | F | T | T | T | T  | \\ 
%   "        | T | F | F | T | T | F  | \\ 
%   "        | F | T | T | F | F | T  | \\ 
%   "        | F | T | F | F | F | T  | \\ 
%   "        | F | F | T | T | F | T  | \\ 
%   "        | F | F | F | T | F | T |  \\ \_6
% "Step No. | 1 | 1 | 1 | 2 | 3 | 4 |   \\ \_6
%\EndTable
%$$
%\caption{Truth Table for $\left( P \wedge \mynot  Q \right) \to R$}
%\label{Ta:compoundtruthtable}%
%\end{table}

\begin{itemize}
  \item When completing the column for  $P \wedge \mynot  Q$, remember that the only time the conjunction is true is when both  $P$  and  $\mynot  Q$ are true.  
  \item When completing the column for  $\left( {P \wedge \mynot  Q} \right) \to R$, remember that the only time the conditional statement is false is when the hypothesis $\left( {P \wedge \mynot  Q} \right)$ is true and the conclusion, $R$, is false.  
\end{itemize}
The last column entered is the truth table for the statement  $\left( {P \wedge \mynot  Q} \right) \to R$ using the setup in the first three columns.
%hbreak
%
\begin{prog}[\textbf{Constructing Truth Tables}]\label{pr:truthtables} \hfill \\
Construct a truth table for each of the following statements: 
\label{exer:sec22-4}%
  \begin{multicols}{2}
  \begin{enumerate}
    \item $P \wedge \mynot  Q$
    \item $\mynot  \left( {P \wedge Q} \right)$
    \item $\mynot  P \wedge \mynot  Q$
    \item $\mynot  P \vee \mynot  Q$
  \end{enumerate}
  \end{multicols}
\noindent
Do any of these statements have the same truth table?
\end{prog}
\hbreak


\endinput

