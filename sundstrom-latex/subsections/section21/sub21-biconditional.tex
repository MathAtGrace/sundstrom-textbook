
\subsection*{The Biconditional Statement}
\index{biconditional statement}%
Some mathematical results are stated in the form  ``$P$  if and only if  $Q$'' or ``$P$  is necessary and sufficient for  $Q$.''  An example would be, ``A triangle is equilateral if and only if its three interior angles are congruent.''
%\begin{center}
%\parbox{4in}{A triangle is equilateral if and only if its three interior angles are congruent.}
%\end{center}
The symbolic form for the biconditional statement  ``$P$  if and only if  $Q$''   is  $P \leftrightarrow Q$. 
\label{sym:bicond}%
  In order to determine a truth table for a biconditional statement, it is instructive to look carefully at the form of the phrase  ``$P$  if and only if  $Q$.''  The word ``and'' suggests that this statement is a conjunction.  Actually it is a conjunction of the statements 
``$P$ if $Q$'' and ``$P$ only if $Q$.''  
The symbolic form of this conjunction is  $\left[ {\left( {Q \to P} \right) \wedge \left( {P \to Q} \right)} \right]$.
%\hbreak
%\enlargethispage{\baselineskip}
\begin{prog}[\textbf{The Truth Table for the Biconditional Statement}]\label{pr:biconditional} \hfill \\
Complete a truth table for $\left[ {\left( {Q \to P} \right) \wedge \left( {P \to Q} \right)} \right]$.  Use the following columns:  $P$, $Q$, $Q \to P$, $P \to Q$, and $\left[ {\left( {Q \to P} \right) \wedge \left( {P \to Q} \right)} \right]$.  The last column of this table will be the truth table for $P \leftrightarrow Q$.
\end{prog}
\vskip6pt
%\hrule

\subsection*{Other Forms of the Biconditional Statement}
\index{biconditional statement!forms of}%
As with the conditional statement, there are some common ways to express the biconditional statement,  $P \leftrightarrow Q$, in the English language. For example,
\begin{multicols}{2}
\begin{itemize}
  \item $P$  if and only if  $Q$.
  \item $P$ implies  $Q$  and  $Q$  implies  $P$.
  \item $P$ is necessary and sufficient for  $Q$.
  %\item $P$  is equivalent to  $Q$.
\end{itemize}
\end{multicols}
\hbreak
%\index{biconditional statement|)}%
%\pagebreak


\endinput
