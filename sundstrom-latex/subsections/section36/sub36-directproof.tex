\subsection*{Direct Proof of a Conditional Statement  \mathversion{bold} $\left( P \to Q \right)$}
%\textbf{{Direct Proof of} {\mathversion{bold} $P \to Q$}}
\index{direct proof}%
\index{proof!direct}%

\begin{itemize}
\item \textbf{When is it indicated}?  This type of proof is often used when the hypothesis and the conclusion are both stated in  a ``positive'' manner.  That is, no negations are evident in the hypothesis and conclusion.  

\item \textbf{Description of the process}.  Assume that  $P$  is true and use this to conclude that  $Q$  is true.  That is, we use the forward-backward method and work forward from  $P$  and backward from  $Q$.

\item \textbf{Why the process makes sense}.  We know that the conditional statement  $P \to Q$  is automatically true when the hypothesis is false.  Therefore, because our goal is to prove that  $P \to Q$  is true, there is nothing to do in the case that  $P$  is false.  Consequently, we may assume that  $P$  is true.  Then, in order for  $P \to Q$  to be true,  the conclusion  $Q$  must also be true.  (When  $P$  is true, but  $Q$ is false, $P \to Q$  is false.)  Thus, we must use our assumption that  $P$  is true to show that  $Q$  is also true.
\end{itemize}

\endinput
