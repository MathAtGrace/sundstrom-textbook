\subsection*{Other Methods of Proof}
The methods of proof that were just described are three of the most common types of proof.  However, we have seen other methods of proof and these are described below.
\subsubsection*{Proofs that Use a Logical Equivalency}
\setcounter{equation}{0}
As was indicated in Section~\ref{S:moremethods}, we can sometimes use  a logical equivalency to help prove a statement.  For example, in order to prove a statement of the form
\begin{equation}\label{sub36-state1}
P \to \left( {Q \vee R} \right),
\end{equation}
it is sometimes possible to use the logical equivalency
\[\left[ {P \to \left( {Q \vee R} \right)} \right] \equiv \left[ {\left( {P \wedge \mynot  Q} \right) \to R} \right].
\]
We would then prove the statement
\begin{equation} \label{sub36-state2}
\left( {P \wedge \mynot Q} \right) \to R.
\end{equation}
Most often, this would use a direct proof for statement~(\ref{sub36-state2}) but other methods could also be used.  Because of the logical equivalency, by proving statement~(\ref{sub36-state2}), we have also proven the  statement~(\ref{sub36-state1}).

\subsubsection*{Proofs that Use Cases}
\index{cases, proof using}%
\index{proof!using cases}%
When we are trying to prove a proposition or a theorem, we often run into the problem that there does not seem to be enough information to proceed.  In this situation, we will sometimes use cases to provide additional assumptions for the forward process of the proof.  When this is done, the  original proposition is divided into a number of separate cases that are proven independently of each other.  The cases must be chosen so that they exhaust all possibilities for the hypothesis of the original proposition.  This method of case analysis is justified by the logical equivalency
\[
\left( {P \vee Q} \right) \to R \equiv \left( {P \to R} \right) \wedge \left( {Q \to R} \right),
\]
which was established in \typeu Activity~\ref*{PA:logicalequiv} in Section~\ref{S:cases}.


\subsubsection*{Constructive Proof} 
\index{constructive proof}%
\index{proof!constructive}%
This is a technique that is often used to prove a so-called \textbf{existence theorem.}
\index{existence theorem}%
  The objective of an existence theorem is to prove that a certain mathematical object exists.  That is, the goal is usually to prove a statement of the form  
\begin{center}
There exists an $x$  such that  $P( x )$.
\end{center}
For a constructive proof of such a proposition, we actually name, describe, or explain how to construct  some object in the universe that makes  $P( x )$ true.

\subsubsection*{Nonconstructive Proof}
Another type of proof that is often used to prove an existence theorem is the so-called \textbf{nonconstructive proof.}
%\index{existence theorem}%
\index{proof!non-constructive}%
  For this type of proof, we make an argument that an object  in the universal set that makes  
$P\left( x \right)$ true must exist but we never construct or name the object that makes  
$P\left( x \right)$  true.

\endinput
