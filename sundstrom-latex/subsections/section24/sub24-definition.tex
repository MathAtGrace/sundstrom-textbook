\subsection*{Quantifiers in Definitions}
\index{quantifier}%
Definitions of terms in mathematics often involve quantifiers.  These definitions are often given in a form that does not use the symbols for quantifiers.  Not only is it important to  know a definition, it is also important to be able to write a negation of the definition.  This will be illustrated with the definition of what it means to say that a natural number is a perfect square.

%Recall that the natural numbers, denoted by  $\mathbb{N}$, consist of the positive whole numbers.  That is, $\mathbb{N} = \left\{ {1,\;2,\;3,\; \ldots } \right\}$. 
%
\begin{defbox}{D:square}{A natural number  $n$  is a \textbf{perfect square}
\index{perfect square}%
 provided that there exists a natural number  $k$  such that  $n = k^2$.}  
\end{defbox}
%
This definition can be written in symbolic form using appropriate quantifiers as follows:
\begin{center}
A natural number  $n$  is a \textbf{perfect square} provided  $\left( {\exists k \in \mathbb{N}} \right) \! \left( {n = k^2 } \right)$.
\end{center}

We frequently use the following steps to gain a better understanding of a definition.

\begin{enumerate}
  \item Examples of natural numbers that are perfect squares are 1, 4, 9, and 81 since 
$1 = 1^2$, $4 = 2^2$, $9 = 3^2$, and $81 = 9^2$.

  \item Examples of natural numbers that are not perfect squares are 2, 5, 10, and 50.

  \item This definition gives two ``conditions.''  One is that the natural number $n$ is a perfect square and the other is that there exists a natural number $k$ such that $n = k^2$.  The definition states that these mean the same thing.  So when we say that a natural number $n$ is not a perfect square, we need to negate the condition that  there exists a natural number $k$ such that $n = k^2$.  We can use the symbolic form to do this.

\[
\mynot \left( {\exists k \in \mathbb{N}} \right)\left( {n = k^2 } \right) \equiv 
\left( \forall k \in \N \right) \left( n \ne k^2 \right)
\]

Notice that instead of writing $\mynot \left(n = k^2 \right)$, we used the equivalent form of 
$\left(n \ne k^2 \right)$.  This will be easier to translate into an English sentence.  So we can write,

\begin{list}{}
\item A natural number $n$  is not a perfect square provided that for every natural number $k$, $n \ne k^2$.
\end{list}
\end{enumerate}


The preceding method illustrates a good method for trying to understand a new definition.  Most textbooks will simply define a concept and leave it to the reader to do the preceding steps.  Frequently, it is not sufficient just to read a definition and expect to understand the new term.  We must provide examples that satisfy the definition, as well as examples that do not satisfy the definition, and we must be able to write a coherent negation of the definition.
\hbreak

%\pagebreak
\begin{prog}[\textbf{Multiples of Three}]\label{pr:mutliple3} \hfill 
\begin{defbox}{D:multiple3}{An integer  $n$  is a \textbf{multiple of 3} provided that there exists an integer  $k$  such that  $n = 3k$.}  
\end{defbox}

\begin{enumerate}
  \item Write this definition in symbolic form using quantifiers by completing the following:

\begin{list}{}
\item An integer $n$ is a multiple of 3 provided that \ldots .
\end{list}
  \item Give several examples of integers (including negative integers) that are multiples of 3.
  \item Give several examples of integers (including negative integers) that are not multiples of 3.
  \item Use the symbolic form of the definition of a multiple of 3 to complete the following sentence: ``An integer $n$  is not a multiple of 3 provided that \ldots .''

  \item Without using the symbols for quantifiers, complete the following sentence:  ``An integer  $n$ is not a multiple of 3 provided that  \ldots .''
\end{enumerate}
\end{prog}
\hbreak


\endinput
