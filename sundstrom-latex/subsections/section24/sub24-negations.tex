\subsection*{Negations of Quantified Statements}
\index{negation!of a quantified statement|(}%
In \typeu Activity~\ref*{PA:quantifier}, we wrote negations of some quantified statements.  This is a very important mathematical activity.  As we will see in future sections, it is sometimes just as important to be able to describe when some object does not satisfy a certain property as it is to describe when the object satisfies the property.  Our next task is to learn how to write negations of quantified statements in a useful English form.

We first look at the negation of a statement involving a universal quantifier.  The general form for such a statement can be written as
$\left( {\forall x} \in U \right)\left( {P( x )} \right)$,
where  $P( x )$ is an open sentence and $U$ is the universal set for the variable $x$.  When we write
\[
\mynot  \left( {\forall x} \in U \right)\left[ {P\left( x \right)} \right],
\]
we are asserting that the statement  $\left( {\forall x} \in U \right)\left[ {P( x )} \right]$ is false.  This is equivalent to saying that the truth set of the open sentence   
$P( x )$ is not the universal set.  That is, there exists an element  $x$  in the universal set  $U$  such that  $P( x )$ is false.  This in turn means that there exists an element  $x$  in  $U$  such that  $\mynot  P( x )$ is true,   which is equivalent to saying that  $\left( {\exists x} \in U \right)\left[ {\mynot  P( x )} \right]$ is true.  This explains why the following result is true:
%
\[
\mynot  \left( {\forall x} \in U \right)\left[ {P( x )} \right] \equiv \left( {\exists x} \in U \right)\left[ {\mynot  P( x )} \right].
\]
Similarly, when we write
\[
\mynot  \left( {\exists x} \in U \right)\left[ {P( x )} \right],
\]
we are asserting that the statement  $\left( {\exists x} \in U \right)\left[ {P( x )} \right]$ is false.  This is equivalent to saying that the truth set of the open sentence  $P( x )$ is the empty set.  That is, there is no element  $x$  in the universal set  $U$  such that  
$P( x )$ is true.  This in turn means that for each element  $x$  in  $U$, 
$\mynot  P( x )$ is true, and this is equivalent to saying that  
$\left( {\forall x} \in U \right)\left[ {\mynot  P( x )} \right]$ is true.  This explains why the following result is true: 
\[
\mynot  \left( {\exists x} \in U \right)\left[ {P( x )} \right] \equiv \left( {\forall x} \in U \right)\left[ {\mynot  P( x )} \right].
\]
We summarize these results in the following theorem.
%\hbreak
\begin{theorem}\label{T:negations}
For any open sentence  $P( x )$,
\[
\begin{aligned}
  \mynot  \left( {\forall x} \in U \right)\left[ {P( x )} \right] &\equiv \left( {\exists x} \in U \right)\left[ {\mynot  P( x )} \right]\text{, and} \\ 
  \mynot  \left( {\exists x} \in U \right)\left[ {P( x )} \right] &\equiv \left( {\forall x} \in U \right)\left[ {\mynot  P( x )} \right]. \\ 
\end{aligned}
\] 
\end{theorem}
\hbreak
%
\begin{example}[\textbf{Negations of Quantified Statements}]\label{E:negations} \hfill \\
Consider the following statement:  $\left( {\forall x \in \mathbb{R}} \right)\left( {x^3  \geq x^2 } \right)$.

We can write this statement as an English sentence in several ways.  Following are two different ways to do so.
\begin{itemize}
  \item For each real number $x$, $x^3  \geq x^2 $.
  \item If  $x$  is a real number, then  $x^3 $ is greater than or equal to  $x^2 $.
\end{itemize}
The second statement shows that in a conditional statement, there is often a hidden universal quantifier.  This statement is false since there are real numbers  $x$  for which  $x^3 $ is not greater than or equal to  $x^2 $. For example, we could use  $x =  - 1$ or  $x = \frac{1}{2}$.

%Since the phrase ``is not greater than or equal to'' means the same thing as ``is less than,'' we usually say that there are real numbers  $x$  for which  $x^3  < x^2 $. 
This means that the negation must be true.  We can form the negation as follows:
\[
\mynot  \left( {\forall x \in \mathbb{R}} \right)\left( {x^3  \geq x^2 } \right) \equiv \left( {\exists x \in \mathbb{R}} \right)\mynot  \left( {x^3  \geq x^2 } \right).
\]
In most cases, we want to write this negation in a way that does not use the negation symbol.  In this case, we can now write the open sentence $\mynot  \left( {x^3  \geq x^2 } \right)$ as  $\left( {x^3  < x^2 } \right)$.  (That is, the negation of ``is greater than or equal to'' is ``is less than.'')  So we obtain the following:
\[
\mynot  \left( {\forall x \in \mathbb{R}} \right)\left( {x^3  \geq x^2 } \right) \equiv \left( {\exists x \in \mathbb{R}} \right)\left( {x^3  < x^2 } \right).
\]
The statement $\left( {\exists x \in \mathbb{R}} \right)\left( {x^3  < x^2 } \right)$
could be written in English as follows:
\begin{itemize}
  \item There exists a real number  $x$  such that  $x^3  < x^2 $.
  \item There exists an  $x$  such that  $x$  is a real number and  $x^3  < x^2 $.
\end{itemize}
%
\end{example}
\hbreak
%
\begin{prog}[\textbf{Negating Quantified Statements}]\label{pr:negating} \hfill \\
For each of the following statements
\renewcommand{\theenumi}{\alph{enumi}}
\begin{itemize}
  \item Write the statement in the form of an English sentence that does not use the symbols for quantifiers.
  \item Write the negation of the statement in a symbolic form that does not use the negation symbol.
  \item Write the negation of the statement in the form of an English sentence that does not use the symbols for quantifiers.
\end{itemize}

%\renewcommand{\theenumi}{\arabic{enumi}}
\begin{enumerate}
  \item $\left( \forall a \in \mathbb{R}\right) \left( a + 0 = a \right)$.
  \item $\left( \forall x \in \mathbb{R} \right) \left[ \sin ( {2x} ) = 2 ( {\sin x} )( {\cos x} ) \right]$.
  \item $\left( \forall x \in \mathbb{R} \right) \left( \tan ^2 x + 1 = \sec ^2 x \right)$.
  \item $\left( \exists x \in \mathbb{Q} \right) \left( x^2  - 3x - 7 = 0 \right)$.
  \item $\left( \exists x \in \mathbb{R} \right) \left( x^2  + 1 = 0 \right)$.
\end{enumerate}
\end{prog}
\index{negation!of a quantified statement|)}%
\index{counterexample}%
\hbreak


\endinput
