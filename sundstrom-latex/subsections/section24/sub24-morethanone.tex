
\subsection*{Statements with More than One Quantifier}
When a predicate contains more than one variable, each variable must be quantified to create a statement.  For example, assume the universal set is the set of integers, $\mathbb{Z}$, and let  $P\left( {x, y} \right)$ be the predicate, ``$x + y = 0$.''  We can create a statement from this predicate in several ways.
\begin{enumerate}
  \item $\left( {\forall x \in \mathbb{Z}} \right)\left( {\forall y \in \mathbb{Z}} \right)\left( {x + y = 0} \right)$. \label{twoquantifiers1}%

We could read this as, ``For all integers  $x$  and  $y$, $x + y = 0$.''  This is a false statement since it is possible to find two integers whose sum is not zero $\left( {2 + 3 \ne 0} \right)$.

  \item $\left( {\forall x \in \mathbb{Z}} \right)\left( {\exists y \in \mathbb{Z}} \right)\left( {x + y = 0} \right)$. \label{twoquantifiers2}%

We could read this as, ``For every integer  $x$, there exists an integer  $y$  such that 
$x + y = 0$.''  This is a true statement.

  \item $\left( {\exists x \in \mathbb{Z}} \right)\left( {\forall y \in \mathbb{Z}} \right)\left( {x + y = 0} \right)$. \label{twoquantifiers3}%

We could read this as, ``There exists an integer  $x$  such that for each integer   $y$, $x + y = 0$.''  This is a false statement since there is no integer  whose sum with each integer is zero.

  \item $\left( {\exists x \in \mathbb{Z}} \right)\left( {\exists y \in \mathbb{Z}} \right)\left( {x + y = 0} \right)$.

We could read this as, ``There exist integers  $x$  and  $y$  such that \\
$x + y = 0$.''  This is a true statement.  For example, $2 + \left( { - 2} \right) = 0$.  
\end{enumerate}
%
When we negate a statement with more than one quantifier, we consider each quantifier in turn and apply the appropriate part of Theorem~\ref{T:negations}.  As an example, we will negate Statement~(\ref{twoquantifiers3}) from the preceding list.  The statement is
\[
\left( {\exists x \in \mathbb{Z}} \right)\left( {\forall y \in \mathbb{Z}} \right)\left( {x + y = 0} \right).
\]
We first treat this as a statement in the following form:  
$\left( {\exists x \in \mathbb{Z}} \right)\left( {P( x )} \right)$  where  $P( x )$ is the predicate  $\left( {\forall y \in \mathbb{Z}} \right)\left( {x + y = 0} \right)$.  Using Theorem~\ref{T:negations}, we have
\[
\mynot  \left( {\exists x \in \mathbb{Z}} \right)\left( {P( x )} \right) \equiv \left( {\forall x \in \mathbb{Z}} \right)\left( {\mynot  P( x )} \right).
\]
%
Using Theorem~\ref{T:negations} again, we obtain the following:
\[
\begin{aligned}
  \mynot  P( x ) &\equiv \mynot  \left( {\forall y \in \mathbb{Z}} \right)\left( {x + y = 0} \right) \\ 
   &\equiv \left( {\exists y \in \mathbb{Z}} \right)\mynot  \left( {x + y = 0} \right) \\ 
   &\equiv \left( {\exists y \in \mathbb{Z}} \right)\left( {x + y \ne 0} \right). \\ 
\end{aligned} 
\]
%
Combining these two results, we obtain
\[
\mynot  \left( {\exists x \in \mathbb{Z}} \right)\left( {\forall y \in \mathbb{Z}} \right)\left( {x + y = 0} \right) \equiv \left( {\forall x \in \mathbb{Z}} \right)\left( {\exists y \in \mathbb{Z}} \right)\left( {x + y \ne 0} \right).
\]
%
%This process can be written as follows:
%\[
%\begin{aligned}
%  \mynot  \left( {\exists x \in \mathbb{Z}} \right)\left( {\forall y \in \mathbb{Z}} \right)\left( {x + y = 0} \right) &\equiv \left( {\forall x \in \mathbb{Z}} \right)\left[ {\mynot  \left( {\forall y \in \mathbb{Z}} \right)\left( {x + y = 0} \right)} \right] \\ 
%   &\equiv \left( {\forall x \in \mathbb{Z}} \right)\left[ {\left( {\exists y \in \mathbb{Z}} \right)\mynot  \left( {x + y = 0} \right)} \right] \\ 
%   &\equiv \left( {\forall x \in \mathbb{Z}} \right)\left( {\exists y \in \mathbb{Z}} \right)\left( {x + y \ne 0} \right). \\ 
%\end{aligned}
%\]
%
The results are summarized in the following table.

$$
\BeginTable
\BeginFormat
|p(0.75in)|p(2in)|p(1.5in)|
\EndFormat
\_
|             |  \textbf{Symbolic Form}  |  \textbf{English Form} | \\+22 \_
|  \Lower{Statement}  |  \Lower{$\left( {\exists x \in \mathbb{Z}} \right)\left( {\forall y \in \mathbb{Z}} \right)\left( {x + y = 0} \right)$}  |  There exists an integer  $x$  such that for each integer   $y$, $x + y = 0$. | \\  \_1
|  \Lower{Negation}   |  \Lower{$\left( {\forall x \in \mathbb{Z}} \right)\left( {\exists y \in \mathbb{Z}} \right)\left( {x + y \ne 0} \right)$}  |  For each integer  $x$, there exists an integer  $y$  such that  $x + y \ne 0$. | \\  \_
\EndTable
$$

%
%\begin{center}
%\begin{tabular}[h]{|p{0.75in}|p{2in}|p{1.5in}|}
%  \hline
%             &  \textbf{Symbolic Form}  &  \textbf{English Form} \\ \hline
%  Statement  &  $\left( {\exists x \in \mathbb{Z}} \right)\left( {\forall y \in \mathbb{Z}} \right)\left( {x + y = 0} \right)$  &  There exists an integer  $x$  such that for each integer   $y$, $x + y = 0$.  \\  \hline
%  Negation   &  $\left( {\forall x \in \mathbb{Z}} \right)\left( {\exists y \in \mathbb{Z}} \right)\left( {x + y \ne 0} \right)$  &  For each integer  $x$, there exists an integer  $y$  such that  $x + y \ne 0$.  \\  \hline
%\end{tabular}
%\end{center}
%
\noindent
Since the given statement is false, its negation is true.

We can construct a similar table for each of the four statements.  The next table shows Statement~(\ref{twoquantifiers2}), which is true, and its negation, which is false.

$$
\BeginTable
\BeginFormat
|p(0.75in)|p(2in)|p(1.5in)|
\EndFormat
\_
 |            |  \textbf{Symbolic Form}  |  \textbf{English Form}  | \\+22 \_
 | \Lower{Statement}  |  \Lower{$\left( {\forall x \in \mathbb{Z}} \right)\left( {\exists y \in \mathbb{Z}} \right)\left( {x + y = 0} \right)$}  |  For every integer  $x$, there exists an integer  $y$  such that $x + y = 0$. | \\  \_1
 | \Lower{Negation}   |  \Lower{$\left( {\exists x \in \mathbb{Z}} \right)\left( {\forall y \in \mathbb{Z}} \right)\left( {x + y \ne 0} \right)$}  |  There exists an integer  $x$  such that for every integer  $y$,  $x + y \ne 0$. | \\  \_
\EndTable
$$
%Since the given statement is true, its negation is false.
%\hbreak

\begin{prog}[\textbf{Negating a Statement with Two Quantifiers}]\label{pr:twoquant} \hfill \\ 
Write the negation of the statement
\[
\left( {\forall x \in \mathbb{Z}} \right)\left( {\forall y \in \mathbb{Z}} \right)\left( {x + y = 0} \right)
\]
\noindent
in symbolic form and as a sentence written in English.
\end{prog}
\hbreak

\endinput
