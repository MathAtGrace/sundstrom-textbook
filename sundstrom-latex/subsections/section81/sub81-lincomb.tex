\subsection*{Writing $\boldsymbol{\gcd (a, b)}$  in Terms of \emph{a} and \emph{b}}
We will use Example~\ref{E:euclidalgo} to illustrate another use of the Euclidean Algorithm.  It is possible to use the steps of the Euclidean Algorithm in reverse order to write  
$\gcd( {a, b} )$  in terms of  $a$  and  $b$.  We will use these steps in reverse order to find integers  $m$  and  $n$  such that  
$\gcd( {234, 42} ) = 234m + 42n$.  The idea is to start with the row with the last nonzero remainder and work backward as shown in the following table:

\begin{center}
\begin{tabular}{| l | l |} \hline
\textbf{Explanation}  &  \textbf{Result} \\ \hline
First, use the equation in Step 3 to  &  $6 = 24 - 18 \cdot 1$  \\ 
write 6 in terms of  24  and  18. &  \\ \hline
Use the equation in Step 2 to write   &   $6 = 24 - 18 \cdot 1$ \\
$18 = 42 - 24 \cdot 1$. Substitute this into &  $\ \ \, = 24 - \left( {42 - 24 \cdot 1} \right)$ \\
the preceding result and simplify.  & $\ \ \, = 42 \cdot \left( { - 1} \right) + 24 \cdot 2 $ \\ \hline
We now have written  6  in terms of &  $6 = 42 \cdot \left( { - 1} \right) + 24 \cdot 2 $ \\
42 and 24.  Use the equation in  & $\ \ \, = 42 \cdot \left( { - 1} \right) + \left( {234 - 42 \cdot 5} \right) \cdot 2 $ \\
Step 1 to write $24 = 234 - 42 \cdot 5$.   &  $\ \ \, = 234 \cdot 2 + 42 \cdot \left( { - 11} \right)$  \\ 
Substitute this into the preceding &  \\ 
result and simplify.  &  \\ \hline
\end{tabular}
\end{center}
Hence, we can write  
\[
\gcd( {234, 42} ) = 234 \cdot 2 + 42 \cdot ( { - 11} ).
\]
(Check this with a calculator.)  In this case, we say that we have written \linebreak 
$\gcd( {234, 42} )$ as a linear combination of  234 and  42.  More generally, we have the following definition.
%
\begin{defbox}{linearcombination}{Let  $a$  and  $b$  be integers.  A \textbf{linear combination}
\index{linear combination}%
 of  $a$  and  $b$  is an integer of the form  $ax + by$, where  $x$  and  $y$  are integers.}
\end{defbox}
%
\begin{prog}[\textbf{Writing the gcd as a Linear Combination}]\label{prog:gcdaslincomb} \hfill \\ 
Use the results from Progress Check~\ref{prog:usingeuclid} to
\begin{enumerate}
\item Write  $\gcd( {180, 126} )$  as a linear combination of  180  and  126.

\item Write  $\gcd( {4208, 288} )$  as a linear combination of  4208  and  288.
\end{enumerate}
\end{prog}
\hbreak
%
The previous example and progress check illustrate the following important result in number theory, which will be used in the next section to help prove some other significant results.
%
\begin{theorem}\label{T:gcdaslincomb}
Let  $a$  and  $b$  be integers, not both  0.  Then  $\gcd( {a, b} )$
 can be written as a linear combination of  $a$  and  $b$.  That is, there exist integers  $u$  and  $v$  such that  $\gcd( {a, b} ) = au + bv$.
\end{theorem}
%
We will not give a formal proof of this theorem.  Hopefully, the examples and activities provide evidence for its validity.  The idea is to use the steps of the Euclidean Algorithm in reverse order to write  $\gcd( {a, b} )$  as a linear combination of  $a$  and  $b$.  For example, assume the completed table for the Euclidean Algorithm is

\begin{center}
\begin{tabular}[h]{| c | l | l | l |}
  \hline
\textbf{Step} & \textbf{Original}  &  \textbf{Equation from}  &    \textbf{New}    \\
  & \textbf{Pair}     &  \textbf{Division Algorithm}  &    \textbf{Pair}  \\ \hline
1 & $\left( {a, b} \right)$  &  	$a = b \cdot q_1  + r_1 $  &  $\left( {b, r_1 } \right)$  \\ \hline
2 & $\left( {b, r_1 } \right)$  &  $b = r_1  \cdot q_2  + r_2 $  &  $\left( {r_1 , r_2 } \right)$  \\ \hline
3 & $\left( {r_1 , r_2 } \right)$  &  $r_1  = r_2  \cdot q_3  + r_3$   &	$\left( {r_2 , r_3 } \right)$  \\ \hline
4 & $\left( {r_2 , r_3 } \right)$  &  $r_2  = r_3  \cdot q_4  + 0 $  &    \\ \hline
\end{tabular}
\end{center}
%
We can use Step 3 to write  $r_3  = \gcd( {a, b} )$ as a linear combination of  
$r_1 $  and  $r_2 $.  We can then solve the equation in Step 2 for  $r_2 $ and use this to write  $r_3  = \gcd( {a, b} )$ as a linear combination of  $r_1 $  and  $b$.  We can then use the equation in Step 1 to solve for  $r_1 $  and use this to write  
$r_3  = \gcd( {a, b} )$  as a linear combination of  $a$  and  $b$.  

In general,  if we can write  $r_p  = \gcd( {a, b} )$
 as a linear combination of a pair in a given row, then we can use the equation in the preceding step to write  $r_p  = \gcd( {a, b} )$ as a linear combination of the pair in this preceding row.

The notational details of this induction argument get quite involved.  Many mathematicians prefer to prove Theorem~\ref{T:gcdaslincomb} using a property of the natural numbers called the Well-Ordering Principle.  \textbf{The Well-Ordering Principle}
\index{Well Ordering Principle}%
 for the natural numbers states that any nonempty set of natural numbers must contain a least element.  It can be proven that the Well-Ordering Principle is equivalent to the Principle of Mathematical Induction.
\hbreak

%\begin{activity}[Linear Combinations and the GCD] \label{A:lincomb-gcd} \hfill
%\begin{enumerate}
%\item What is the greatest common divisor of 20 and 12?
%
%\item Let  $d = \gcd( {20, 12} )$.  Write  $d$  as a linear combination of  20  and  12.
%
%\item Generate at least six different linear combinations of  20  and  12.  Does 
%$\gcd( {20, 12} )$ divide each of these linear combinations?
%
%\item Generate at least six different linear combinations of  21  and  $-6$.  Does 
%$\gcd( {21, -6} )$ divide each of these linear combinations?
%
%\item Complete the proof of Proposition~\ref{P:divlinearcomb} in Section~\ref{S:provingset}.
%
%\noindent
%\textbf{Proposition \ref{P:divlinearcomb}}  \emph{Let a, b, and  t  be integers with $t \ne 0$.  If  t  divides  a  and  t  divides  b, then for all integers  x  and  y,  t  divides  
%\text{(}ax + by\text{)}.}
%
%\noindent
%\textbf{\textit{Proof}}. Let $a$, $b$, and  $t$  be integers with $t \ne 0$, and assume that $t$  divides  $a$  and  $t$  divides  $b$.  We will prove that for all integers  $x$  and  $y$,  $t$  divides  $(ax + by)$.
%
%So let  $x \in \mathbb{Z}$ and let  $y \in \mathbb{Z}$.  Since  $t$  divides  $a$, there exists an integer  $m$  such that $ \ldots .$
%
%\item Now let $a$ and $b$ be integers, not both zero and let $d = \gcd(a, b )$.  Theorem~\ref{T:gcdaslincomb} states that $d$ is a linear combination of $a$ and $b$.  In addition, let $S$ and $T$ be the following sets:
%\[
%S = \left\{ ax + by \mid x, y \in \Z \right\} \qquad \text{and} \qquad 
%T = \left\{ kd \mid k \in \Z \right\}\!.
%\]
%That is, $S$ is the set of all linear combinations of $a$ and $b$, and $T$ is the set of all multiples of the greatest common divisor of $a$ and $b$.  Prove that $S = T$\!.
%\end{enumerate}
%\end{activity}
%\hbreak


\endinput
