\subsection*{The System of Integers}
Number theory is a study of the system of integers, 
\index{integers!system}%
 which consists of the set of integers, 
$\mathbb{Z} = \left\{ {\, \ldots ,  - 3,  - 2,  - 1, 0, 1, 2, 3,  \ldots \, } \right\}$
 and the various properties of this set under the usual operations of addition and multiplication and under the usual ordering relation of ``less than.''  The properties of the integers in Table~\ref{Ta:axiomsforintegers} will be considered axioms in this text.

We will also assume the properties of the integers shown in Table~\ref{Ta:propertiesofintegers}.  These properties can be proven from the properties in Table~\ref{Ta:axiomsforintegers}.  (However, we will not do so here.)


\newpage
\begin{table}[h]
\begin{center}
\begin{tabular}{| p{2.3in} | p{2.3in} |} 
 \multicolumn{1}{ l }{ } & \multicolumn{1}{ l }{For all integers $a$, $b$, and $c$:}  \\ \hline
Closure Properties for Addition and Multiplication   &  
$a + b \in \mathbb{Z}$ and 

$ab \in \mathbb{Z}$ \\ \hline

Commutative Properties for Addition and Multiplication  &
$a + b = b + a$, and

$ab = ba$ \\ \hline
Associative Properties for Addition and Multiplication  &
$\left( {a + b} \right) + c = a + \left( {b + c} \right)$ and

$\left( {ab} \right)c = a\left( {bc} \right)$ \\ \hline
Distributive Properties of Multiplication over Addition  &
$a\left( {b + c} \right) = ab + ac$, and

$\left( {b + c} \right)a = ba + ca$  \\ \hline
Additive and Multiplicative Identity Properties  &
$a + 0 = 0 + a = a$, and 

$a \cdot 1 = 1 \cdot a = a$ \\ \hline
Additive Inverse Property  &
$a + \left( { - a} \right) = \left( { - a} \right) + a = 0$  \\ \hline

\end{tabular}
     \caption{Axioms for the Integers}
     \label{Ta:axiomsforintegers}
\end{center} 
\end{table} 


\begin{table}[h]
\begin{center}
\begin{tabular}{| p{1.6in} | p{3in} |} \hline
Zero Property of Multiplication  &   If  $a \in \mathbb{Z}$, then  $a \cdot 0 = 0 \cdot a = 0$.
 \\ \hline
Cancellation Properties &  If  $a, b, c \in \mathbb{Z}$ and  
$a + b = a + c$, then  $b = c$.  \\ 
of Addition and Multiplication   &  If  $a, b, c \in \mathbb{Z}$, $a \ne 0$ and  $ac = bc$, then  $b = c$.  \\ \hline
%Trichotomy Law  &  If  $a \in \mathbb{Z}$, then exactly one of the following statements is true:  $a < 0$, $a = 0$, or $a > 0$.\\ \hline
\end{tabular}
     \caption{Properties of the Integers}
     \label{Ta:propertiesofintegers}
\end{center} 
\end{table} 
%
We have already studied a good deal of number theory in this text in our discussion of proof methods.  In particular, we have studied even and odd integers, divisibility of integers, congruence, and the Division Algorithm.  See the summary for Chapter~\ref{C:proofs} on 
page~\pageref{SS:evenodd} for a summary of results concerning even and odd integers as well as results concerning properties of divisors.
%See Theorems~\ref{T:evenodd} and~\ref{T:divisors} at the end of Section~\ref{S:moremethods} for a summary of results concerning even and odd integers as well as  some results concerning properties of divisors.  
We reviewed some of these properties and the Division Algorithm in the \typel activities.
\hbreak

\endinput
