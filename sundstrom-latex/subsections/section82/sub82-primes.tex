\subsection*{Prime Numbers and Prime Factorizations}
We are now ready to prove the Fundamental Theorem of Arithmetic.  The first part of this theorem was proved in Theorem~\ref{T:primefactors} in Section~\ref{S:otherinduction}.  This theorem states that each natural number greater than 1 is either a prime number or is a product of prime numbers.  Before we state the Fundamental Theorem of Arithmetic, we will discuss some notational conventions that will help us with the proof.  We start with an example.

We will use  $n = 120$.  Since  $5 \mid 120$, we can write  $120 = 5 \cdot 24$.  In addition, we can factor 24 as  $24 = 2 \cdot 2 \cdot 2 \cdot 3$.  So we can write
\[
\begin{aligned}
  120 &= 5 \cdot 24 \\ 
      &= 5\left( {2 \cdot 2 \cdot 2 \cdot 3} \right). \\ 
\end{aligned} 
\]
This is a prime factorization of 120, but it is not the way we usually write this factorization.  Most often, we will write the prime number factors in ascending order.  So we write
\[
120 = 2 \cdot 2 \cdot 2 \cdot 3 \cdot 5\text{  or  }120 = 2^3  \cdot 3 \cdot 5.
\]
Now, let  $n \in \mathbb{N}$.  To write the prime factorization of  $n$ with the prime factors in ascending order requires that if we write  
$n = p_1 p_2  \cdots p_r $, where  $p_1 , p_2 ,  \ldots, p_r $ are prime numbers, we will have 
$p_1  \leq p_2  \leq  \cdots  \leq p_r $.
\hbreak
%
\begin{theorem}[\textbf{The Fundamental Theorem of Arithmetic}] \label{T:fundtheorem}
\index{Fundamental Theorem! of Arithmetic}%
 \hfill
\begin{enumerate}
\item Each natural number greater than 1 is either a prime number or is a product of prime numbers.

\item Let   $n \in \mathbb{N}$ with  $n > 1$.  Assume that
\[
n = p_1 p_2  \cdots p_r \text{  and that  }n = q_1 q_2  \cdots q_s,
\]
where  $p_1 , p_2 ,  \ldots, p_r $ and  $q_1 , q_2 ,  \ldots, q_s $ are primes with  
$p_1  \leq p_2  \leq  \cdots  \leq p_r $ and  
$q_1  \leq q_2  \leq  \cdots  \leq q_s $.  Then  $r = s$, and for each  $j$  from  1  to  r,  $p_j  = q_j $.
\end{enumerate}
\end{theorem}
%
\begin{myproof}
The first part of this theorem was proved in Theorem~\ref{T:primefactors}.  We will prove the second part of the theorem by induction on  $n$  using the Second Principle of Mathematical Induction.  (See Section~\ref{S:otherinduction}.)  For each natural number  $n$  with  $n > 1$,  let  $P( n )$ be

\begin{list}{}
\item If  $n = p_1 p_2  \cdots p_r$ and $n = q_1 q_2  \cdots q_s $, where  
$p_1 , p_2 ,  \ldots, p_r $ and  
$q_1 , q_2 ,  \ldots, q_s $ are primes with  
$p_1  \leq p_2  \leq  \cdots  \leq p_r $  and  
$q_1  \leq q_2  \leq  \cdots  \leq q_s $, then  $r = s$, and for each  $j$  from  1  to  $r$,  $p_j  = q_j $.
\end{list}

%\vskip9pt
For the basis step, we notice that since  2  is a  prime number, its only factorization is  
$2 = 1 \cdot 2$.  This means that the only equation of the form  $2 = p_1 p_2  \cdots p_r $, where  $p_1 , p_2 ,  \ldots, p_r $ are prime numbers, is the case where  $r = 1$ and  $p_1  = 2$.  This proves that  $P( 2 )$  is true.

%\vskip9pt
For the inductive step, let  $k \in \mathbb{N}$ with  $k \geq 2$.  We will assume that \linebreak
$P( 2 ), P( 3 ),  \ldots , P( k )$ are true.  The goal now is to prove that  $P( {k + 1} )$ is true.  To prove this, we assume that  $(k + 1)$ has two prime factorizations and then prove that these prime factorizations are the same.  So we assume that

\begin{list}{}
\item $k + 1 = p_1 p_2  \cdots p_r$ and that $k + 1 = q_1 q_2  \cdots q_s $, where  
$p_1 , p_2 ,  \ldots, p_r $ and  $q_1 , q_2 ,  \ldots, q_s $ are primes with  
$p_1  \leq p_2  \leq  \cdots  \leq p_r $  and  
$q_1  \leq q_2  \leq  \cdots  \leq q_s $.
\end{list}

\vskip6pt
\noindent
We must now prove that  $r = s$, and for each  $j$  from  1  to  $r$,  $p_j  = q_j $.  We can break our proof into two cases:  (1)  $p_1  \leq q_1 $; and (2) $q_1  \leq p_1 $.  Since one of these must be true, and since the proofs will be similar, we can assume, without loss of generality,  that  $p_1  \leq q_1 $.

Since  $k + 1 = p_1 p_2  \cdots p_r $, we know that  $p_1 \mid \left( {k + 1} \right)$, and hence we may conclude that $p_1 \mid \left( {q_1 q_2  \cdots q_s } \right)$.  We now use Corollary~\ref{C:primedivides} to conclude that there exists a  $j$  with  
$1 \leq j \leq s$   such that  $p_1 \mid q_j $.  Since  $p_1 $  and  $q_j $ are primes, we conclude that
\[
p_1  = q_j.
\]
We have also assumed that  $q_1  \leq q_j$ for all $j$ and, hence, we know that  
$q_1  \leq p_1 $.  However, we have also assumed that  $p_1  \leq q_1 $.  Hence,
\[
p_1  = q_1.
\]
We now use this and the fact that  $k + 1 = p_1 p_2  \cdots p_r  = q_1 q_2  \cdots q_s $ to conclude that
\[
p_2  \cdots p_r  = q_2  \cdots q_s.
\]
The product in the previous equation is less that  $k + 1$. Hence, we can apply our induction hypothesis to these factorizations and conclude that  $r = s$, and for each  $j$  from  2  to  $r$,  $p_j  = q_j $.

This completes the proof that if  
$P( 2 ), P( 3 ),  \ldots , P( k )$ are true, then 
$P( {k + 1} )$ is true.  Hence, by the Second Principle of Mathematical Induction, we conclude that  $P( n )$ is true for all  $n \in \mathbb{N}$ with  $n \geq 2$.  This completes the proof of the theorem.
\end{myproof}
%
\hbreak
%
\noindent
\note  We often shorten the result of the Fundamental Theorem of Arithmetic by simply saying that each natural number greater than one that is not a prime has a \textbf{unique factorization}
\index{unique factorization}%
 as a product of primes.  This simply means that if  $n \in \mathbb{N}$, $n > 1$, and  $n$  is not prime, then no matter how we choose to factor  $n$  into a product of primes, we will always have the same prime factors.  The only difference may be in the order in which we write the prime factors.


\endinput
