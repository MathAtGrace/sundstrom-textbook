\subsection*{Summation Notation}
The result in Proposition~\ref{P:suminduction} could be written using summation notation as follows:
%
\begin{center}
For each natural number $n$, $\sum\limits_{j = 1}^n {j^2 }  = \dfrac{{n(n + 1)(2n + 1)}}{6}$.
\end{center}
%
\noindent
In this case, we use  $j$  for the index for the summation, and the notation 
%\[
$\sum\limits_{j = 1}^n {j^2 }$
%\]
tells us to add all the values of  $j^2 $ for  $j$  from  1  to  $n$, inclusive.  That is, 
\[
\sum\limits_{j = 1}^n {j^2 }  = 1^2  + 2^2  +  \cdots  + n^2.
\]
So in the proof of Proposition~\ref{P:suminduction}, we would let $P( n )$  be  
%\[
$\sum\limits_{j = 1}^n {j^2 }  = \dfrac{{n(n + 1)(2n + 1)}}{6}$,
%\]
and we would use the fact that for each natural number  $k$,
\[
\sum\limits_{j = 1}^{k + 1} {j^2 }  = \left( {\sum\limits_{j = 1}^k {j^2 } } \right) + \left( {k + 1} \right)^2 .
\]
\hbreak
\setcounter{equation}{0}

%
\begin{prog}[\textbf{An Example of a Proof by Induction}] \label{prog:indexample} \hfill
\begin{enumerate}
\item Calculate  $1 + 2 + 3 +  \cdots  + n$ and $\dfrac{n(n + 1)}{2}$ 
for several natural numbers  $n$. What do you observe?

\item Use mathematical induction to prove that 
$1 + 2 + 3 +  \cdots  + n = \dfrac{n \left( n + 1 \right)}{2}$.

To do this, let $P ( n )$ be the open sentence, 
``$1 + 2 + 3 +  \cdots  + n = \dfrac{n \left( n + 1 \right)}{2}$.''  For the basis step, notice that the equation $1 = \dfrac{1 \left( 1 + 1 \right)}{2}$ shows that 
$P( 1 )$ is true.  Now let $k$ be a natural number and assume that 
$P (k )$ is true.  That is, assume that
\[
1 + 2 + 3 +  \cdots  + k = \frac{k \left( k + 1 \right)}{2},
\]
and complete the proof.
\end{enumerate}
\end{prog}
\hbreak

\endinput
