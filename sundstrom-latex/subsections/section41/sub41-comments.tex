\subsection*{Some Comments about Mathematical Induction}
\setcounter{equation}{0}
\begin{enumerate}
\item The basis step is an essential part of a proof by induction.  See 
Exercise~(\ref{exer:basis}) for an example that shows that the basis step is needed in a proof by induction.


\item Exercise~(\ref{exer:circleregions}) provides an example that shows the inductive step is also an essential part of a proof by mathematical induction.

\item It is important to remember that the inductive step in an induction proof is a proof of a conditional statement.  Although we did not explicitly use the forward-backward process in the inductive step for Proposition~\ref{P:suminduction}, it was implicitly used in the discussion prior to Proposition~\ref{P:suminduction}.  The key question was, ``How does knowing the sum of the first  $k$  squares help us find the sum of the first  $\left( {k + 1} \right)$
 squares?''

%\item The proof in Activity~\ref{A:basisstep} is a legitimate proof of the proposition that if  $P(  k )$  is true, then  $P\left( {k + 1} \right)$ is true.  This is a true conditional statement.  The point is that even though this conditional statement is true, nothing has been proved about the individual statements  $P\left( 1 \right)$, $P\left( 2 \right)$, 
%$P\left( 3 \right)$, and so on.

\item When proving the inductive step in a proof by induction, the key question is, 
\begin{center}
How does knowing  $P(  k )$ help us prove  $P( {k + 1} )$?
\end{center}
\setcounter{equation}{0}
In Proposition~\ref{P:suminduction}, we were able to see that the way to answer this question was to add a certain expression to both sides of the equation given in  $P(  k )$. Sometimes the relationship between  $P(  k )$  and  $P( {k + 1} )$ is not as easy to see.  For example, in \typeu Activity~\ref*{PA:exploringstatements}, we explored the following proposition:
\begin{center}
For each natural number $n$,  4  divides  $\left( {5^n  - 1} \right)$.
\end{center}
%
This means that the open sentence, $P( n )$, is  ``4  divides  $\left( {5^n  - 1} \right)$.''  So in the inductive step, we assume  $k \in \mathbb{N}$ and that  4  divides  $\left( {5^k  - 1} \right)$.  This means that there exists an integer  $m$  such that
%
\begin{equation} \label{eq:5a}
5^k  - 1 = 4m.
\end{equation}
%
In the backward process, the goal is to prove that  4  divides  $\left( {5^{k + 1}  - 1} \right)$.  This can be accomplished if we can prove that there exists an integer  $s$  such that
\begin{equation} \label{eq:5b}
5^{k + 1}  - 1 = 4s.
\end{equation}
%
We now need to see if there is anything in equation~(\ref{eq:5a}) that can be used in equation~(\ref{eq:5b}).  The key is to find something in the equation $5^k - 1 = 4m$ that is related to something similar in the equation $5^{k + 1}  - 1 = 4s$.  In this case, we notice that
\[
5^{k + 1}  = 5 \cdot 5^k .
\]
So if we can solve $5^k  - 1 = 4m$  for  $5^k $, we could make a substitution for $5^k$.  This is done in the proof of the following proposition.
\end{enumerate}
%\hbreak
%
\begin{proposition} \label{P:divideinduction}
For every natural number  $n$,  4  divides  $\left( {5^n  - 1} \right)$.
\end{proposition}
%
\setcounter{equation}{0}
\begin{myproof}(Proof by Mathematical Induction)
For each natural number  $n$, let  $P( n )$ be ``4  divides  $\left( {5^n  - 1} \right)$.'' 
%\vskip10pt
%\noindent
We first prove that $P \left( 1 \right)$ is true.  Notice that when  
$n = 1$, $\left( {5^n  - 1} \right) = 4$.  Since  4  divides 4,  $P\left( 1 \right)$  is true.
%\vskip6pt
%\noindent

For the inductive step, we prove that for all $k \in \mathbb{N}$, if $P \left( k \right)$ is true, then $P \left( k + 1 \right)$ is true.  So let  $k$  be a natural number and assume that  $P(  k )$  is true.  That is, assume that
\[
4\text{  divides  }\left( {5^k  - 1} \right).
\]
This means that there exists an integer  $m$  such that  
\[
5^k  - 1 = 4m.
\]
Thus,
\begin{equation} \label{eq:5c}
5^k  = 4m + 1.
\end{equation}
In order to prove that  $P( k + 1 )$ is true, we must show that  4 divides $\left( {5^{k + 1}  - 1} \right)$.  Since  $5^{k + 1}  = 5 \cdot 5^k $, we can write
\begin{equation} \label{eq:5d}
5^{k + 1}  - 1 = 5 \cdot 5^k  - 1.
\end{equation}
We now substitute the expression for  $5^k $ from equation~(\ref{eq:5c})  into equation~(\ref{eq:5d}).  This gives
%
\begin{align}
  5^{k + 1}  - 1 &= 5 \cdot 5^k  - 1 \notag \\
                 &= 5( {4m + 1} ) - 1 \notag \\ 
                 &= ( {20m + 5} ) - 1 \notag \\ 
                 &= 20m + 4 \notag \\ 
                 &= 4( {5m + 1} ) \label{eq:5e} 
\end{align} 
%
Since  $\left( {5m + 1} \right)$ is an integer, equation~(\ref{eq:5e}) shows that 4 divides $\left( {5^{k + 1}  - 1} \right)$.  Therefore,  if  $P(  k )$ is true, then  $P( k + 1 )$ is true and the inductive step has been established.
Thus, by the Principle of Mathematical Induction, for every natural number  $n$,  4  divides  $\left( {5^n  - 1} \right)$.
\end{myproof}
%\hbreak

Proposition~\ref{P:divideinduction} was stated in terms of ``divides.''  We can use congruence to state a proposition that is equivalent to Proposition~\ref{P:divideinduction}.  The idea is that the sentence, 4 divides $\left(5^n - 1 \right)$ means that $\mod{ 5^n }{1}{4}$.  So the following proposition is equivalent to Proposition~\ref{P:divideinduction}.
\begin{proposition} \label{P:congruenceinduction}
For every natural number  $n$,  $\mod{ 5^n }{1}{4}$.
\end{proposition}
Since we have proved Proposition~\ref{P:divideinduction}, we have in effect proved 
Proposition~\ref{P:congruenceinduction}.  However, we could have proved 
Proposition~\ref{P:congruenceinduction} first by using the results in Theorem~\ref{T:propsofcong} on page~\pageref{T:propsofcong}.  This will be done in the next progress check.
\hbreak

\begin{prog}[\textbf{Proof of Proposition~\ref{P:congruenceinduction}}] \hfill \\
To prove Proposition~\ref{P:congruenceinduction}, we let $P(n)$ be $\mod{ 5^n }{1}{4}$ and notice that $P(1)$ is true since $\mod{5}{1}{4}$.  For the inductive step, let $k$ be a natural number and assume that $P(k)$ is true.  That is, assume that $\mod{ 5^k }{1}{4}$.
\begin{enumerate}
  \item What must be proved in order to prove that $P(k+1)$ is true?
  \item Since $5^{k+1} = 5 \cdot 5^k$, multiply both sides of the congruence $\mod{ 5^k }{1}{4}$ by 5.  The results in Theorem~\ref{T:propsofcong} on page~\pageref{T:propsofcong} justify this step.
  \item Now complete the proof that for each $k \in \N$, if $P(k)$ is true, then $P(k+1)$ is true and complete the induction proof of Proposition~\ref{P:congruenceinduction}.
\end{enumerate}
It might be nice to compare the proofs of Propositions~\ref{P:divideinduction} and~\ref{P:congruenceinduction} and decide which one is easier to understand.
\end{prog}


%\noindent
%\note  We proved Proposition~\ref{P:congruenceinduction} using mathematical induction so that we could practice constructing and writing a proof by induction.  However, there is another way to prove this result that uses the concept of congruence.  See Exercise~(\ref{exer:51-congruence}).
\hbreak

\endinput
