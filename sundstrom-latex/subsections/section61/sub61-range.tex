\subsection*{The Codomain and Range of a Function}
Besides the domain and codomain, there is another important set associated with a function.  The need for this was illustrated in the example of the function $g$ on page~\pageref{example:function}.  For this function, it was noticed that there are elements in the codomain that have no preimage or, equivalently, there are elements in the codomain that are not the image of any element in the domain.  The set we are talking about is the subset of the codomain consisting of all images of the elements of the domain of the function, and it is called the range of the function.
%
\begin{defbox}{rangeandimage}{Let  $f\x A \to B$.  The set  
$\left\{ {f( x ) \mid x \in A} \right\}$  is called the \textbf{range of the function}
\index{range}%
\index{function!range}%
  $\boldsymbol{f}$  and is denoted by  $\text{range}\left( f \right)$\!. \label{sym:rangef}  The range of  $f$  is sometimes called the \textbf{image of the function}  $\boldsymbol{f}$ (or the \textbf{image of} 
$\boldsymbol{A}$ \textbf{under} $\boldsymbol{f}$).}
\end{defbox}

\noindent
The range of  $f\x A \to B $ could equivalently be defined as follows:
\[
\text{range}( f ) = \left\{ { {y \in B} \mid y = f\left( x \right)\text{ for some }x \in A} \right\}\!.
\]
Notice that this means that $\text{range} (f) \subseteq \text{codom} (f)$ but does not necessarily mean that 
$\text{range} (f) = \text{codom} (f)$.  Whether we have this set equality or not depends on the function $f$.  More about this will be explored in Section~\ref{S:typesoffunctions}.
\hbreak
%
\begin{prog}[\textbf{Codomain and Range}] \label{pr:codomainandrange} \hfill 
\begin{enumerate}
\item Let  $b$  be the function that assigns to each person his or her birthday (month and day).
\begin{enumerate}
  \item What is the domain of this function?

  \item What is a codomain for this function?

  \item In \typeu Activity~\ref*{PA:otherfunctions}, we determined that the following statement is true:  For each day  $D$  of the year, there exists a person  $x$  such that  
$b( x ) = D$.
%  \begin{list}{}
%    \item For each day  $D$  of the year, there exists a person  $x$  such that  
%$b\left( x \right) = D$.
%  \end{list}
What does this tell us about the range of the function  $b$?  Explain.
\end{enumerate}

\item Let  $s$  be the function that associates with each natural number the sum of its distinct natural number factors.

\begin{enumerate}
  \item What is the domain of this function?

  \item What is a codomain for this function?

  \item In \typeu Activity~\ref*{PA:otherfunctions}, we determined that the following statement is false:

  \begin{list}{}
    \item For each  $m \in \mathbb{N}$, there exists a natural number  $n$  such that  
\linebreak $s( n ) = m$.
  \end{list}

Give an example of a natural number  $m$  that shows this statement is false, and explain what this tells us about the  range of the function  $s$\!.
\end{enumerate}

\end{enumerate}
\end{prog}
\hbreak

\endinput
