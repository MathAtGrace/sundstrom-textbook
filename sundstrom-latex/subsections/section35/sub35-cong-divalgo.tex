\subsection*{Using Cases Based on Congruence Modulo \emph{n}}
\index{congruence!Division Algorithm|(}%
\index{Division Algorithm!congruence|(}%
Notice that the set of all integers that are congruent to 2  modulo 7 is
\[
\left\{  n \in \Z \mid n \equiv 2 \pmod7 \right\} = 
 \left\{ \ldots , - 19, - 12, - 5, 2, 9, 16, 23, \ldots \right\}.
\]
If we divide any integer in this set by 7 and write the result according to the Division Algorithm, we will get a remainder of 2.  For example,
\begin{align}
2 &= 7 \cdot 0 + 2 &  -5 &= 7 \left( -1 \right) + 2 \notag \\ 
9 &= 7 \cdot 1 + 2 &  -12 &= 7 \left( -2 \right) +2 \notag \\
16 &= 7 \cdot 2 + 2 &  -19 &= 7 \left( -3 \right) +2 \notag \\
23 &= 7 \cdot 3 + 2. &   & \notag 
\notag
\end{align}
%
Is this a coincidence or is this always true?  Let's look at the general case.  For this, let  $n$  be a natural number and let  $a \in \mathbb{Z}$.  By the Division Algorithm, there exist unique integers  $q$  and  $r$  such that
\[
a = nq + r\text{  and  }0 \leq r < n.
\]
By subtracting  $r$  from both sides of the equation  $a = nq + r$, we obtain
\[
a - r = nq.
\]
But this implies that $n \mid \left( a - r \right)$ and hence that  
$a \equiv r \pmod n$.  We have proven the following result.
\begin{theorem}\label{T:congtorem}
Let  $n \in \mathbb{N}$ and let  $a \in \mathbb{Z}$.  If  $a = nq + r\text{  and  }0 \leq r < n$ for some integers  $q$  and  $r$, then  $a \equiv r \pmod n$.
\end{theorem}
%
This theorem says that an integer is congruent (mod $n$) to its remainder when it is divided by  $n$.  Since this remainder is unique and since the only possible remainders for division by $n$  are  $0,1,2, \ldots ,n - 1$, we can state the following result.
%
\begin{corollary}\label{C:congtorem}
If  $n \in \mathbb{N}$, then each integer is congruent, modulo $n$, to precisely one of the integers $0,1,2, \ldots ,n - 1$.  That is, for each integer $a$, there exists a unique integer 
$r$ such that
\[
a \equiv r \pmod n \quad \text{and} \quad 0 \leq r < n.
\]
\end{corollary}

Corollary~\ref{C:congtorem} can be used to set up cases for an integer in a proof.  If  $n \in \mathbb{N}$  and  $a \in \mathbb{Z}$, then we can consider  $n$  cases for  $a$.  The integer  $a$  could be congruent to  
$0,1,2, \ldots , \text{or } n - 1$  modulo  $n$.  For example, if we assume that 5 does not divide an integer $a$, then we know  $a$ is not congruent to 0 modulo 5, and hence, that $a$ must be congruent to 1, 2, 3, or 4 modulo 5.  We can use these as 4 cases within a proof.  %This is illustrated in the next progress check.
%\begin{prog}[\textbf{Using Properties of Congruence}] \label{prog:propertiesofcong} \hfill \\
For example, suppose we wish to determine the values of $a^2$ modulo 5 for integers that are not congruent to 0 modulo 5.  We begin by squaring some integers that are not congruent to 0 modulo 5.  We see that
\begin{align*}
1^2 &= 1 & \text{and} & &1 &\equiv 1 \pmod 5. \\
3^2 &= 9 & \text{and} & &9 &\equiv 4 \pmod 5. \\
6^2 &= 36 & \text{and} & &36 &\equiv 1 \pmod 5. \\
8^2 &= 64 & \text{and} & &64 &\equiv 4 \pmod 5. \\
9^2 &= 81 & \text{and} & &81 &\equiv 1 \pmod 5. 
\end{align*}
These explorations indicate that the following proposition is true and we will now outline a method to prove it.   
\begin{proposition} \label{prop:congmod5}
For each integer $a$, if $a \not \equiv 0 \pmod 5$, then $\mod{a^2}{1}{5}$ or $\mod{a^2}{4}{5}$.
\end{proposition}
\newpar
\textbf{\emph{Proof}}. We will prove this proposition using cases for $a$ based on congruence modulo 5.  In doing so, we will use the results in Theorem~\ref{T:propsofcong} and Theorem~\ref{T:modprops}.  Because the hypothesis is $a \not \equiv 0 \pmod 5$, we can use four cases, which are:  (1) $\mod{a}{1}{5}$, (2) $\mod{a}{2}{5}$, (3) $\mod{a}{3}{5}$, and (4) $\mod{a}{4}{5}$.  Following are proofs for the first and fourth cases.

\newpar
\textbf{Case 1}.  $\left( \mod{a}{1}{5} \right)$.  In this case, we use Theorem~\ref{T:propsofcong} to conclude that
\[
\mod{a^2}{1^2}{5} \quad \text{or} \quad \mod{a^2}{1}{5}.
\]
This proves that if $\mod{a}{1}{5}$, then $\mod{a^2}{1}{5}$.

\newpar
\textbf{Case 4}.  $\left( \mod{a}{4}{5} \right)$.  In this case, we use Theorem~\ref{T:propsofcong} to conclude that
\[
\mod{a^2}{4^2}{5} \quad \text{or} \quad \mod{a^2}{16}{5}.
\]
We also know that $\mod{16}{1}{5}$.  So we have $\mod{a^2}{16}{5}$ and $\mod{16}{1}{5}$, and we can now use the transitive property of congruence (Theorem~\ref{T:modprops}) to conclude that $\mod{a^2}{1}{5}$.  This proves that if $\mod{a}{4}{5}$, then $\mod{a^2}{1}{5}$.

\newpar
\begin{prog}[\textbf{Using Properties of Congruence}] \label{prog:propertiesofcong} \hfill \\
Complete a proof of Proposition~\ref{prop:congmod5} by completing proofs for the other two cases.  \note It is possible to prove Proposition~\ref{prop:congmod5} using only the definition of congruence instead of using the properties that we have proved about congruence.  However, such a proof would involve a good deal of algebra.  One of the advantages of using the properties is that it avoids the use of complicated algebra in which it is easy to make mistakes.
\end{prog}
%\end{myproof}
\hbreak
In the proof of Proposition~\ref{prop:congmod5}, we used four cases.  Sometimes it may seem a bit overwhelming when confronted with a proof that requires several cases.  For example, if we want to prove something about some integers modulo 6, we may have to use six cases.  However, there are sometimes additional assumptions (or conclusions) that can help reduce the number of cases that must be considered.  This will be illustrated in the next progress check.

\begin{prog}[\textbf{Using Cases Modulo 6}] \label{prog:casesmod6} \hfill \\ 
Suppose we want to determine the possible values for $a^2$ modulo 6 for odd integers that are not multiples of 3.  Before beginning to use congruence arithmetic (as in the proof of Proposition~\ref{prop:congmod5}) in each of the possible six cases, we can show that some of the cases are not possible under these assumptions.  (In some sense, we use a short proof by contradiction for these cases.)  So assume that $a$ is an odd integer.  Then:
\begin{itemize}
  \item If $\mod{a}{0}{6}$, then there exists an integer $k$ such that $a = 6k$.  But then $a = 2(3k)$ and hence, $a$ is even.  Since we assumed that $a$ is odd, this case is not possible.
  \item If $\mod{a}{2}{6}$, then there exists an integer $k$ such that $a = 6k + 2 $.  But then $a = 2(3k + 1)$ and hence, 
$a$ is even.  Since we assumed that $a$ is odd, this case is not possible.
\end{itemize}
\begin{enumerate}
  \item Prove that if $a$ is an odd integer, then $a$ cannot be congruent to 4 modulo 6.
  \item Prove that if $a$ is an integer and 3 does not divide $a$, then $a$ cannot be congruent to 3 modulo 6.
  \item So if $a$ is an odd integer that is not a multiple of 3, then $a$ must be congruent to 1 or 5 modulo 6.  Use these two cases to prove the following proposition: 
\end{enumerate}
\begin{proposition} \label{prop:congmod6}
For each integer $a$, if $a$ is an odd integer that is not multiple of 3, then $\mod{a^2}{1}{6}$.
\end{proposition}
\end{prog}
\hbreak

%\begin{proposition}\label{P:3dividesver2}
%If $n$ is an integer, then 3 divides $n^3-n$.
%\end{proposition}
%\begin{myproof}
%Let $n$ be an integer.  We will show that 3 divides $n^3-n$ by examining three cases:  
%$n \equiv 0 \pmod 3$, $n \equiv 1 \pmod 3$, or $n \equiv 2 \pmod 3$.
%\vskip10pt
%\noindent
%\textit{Case 1} ($n \equiv 0 \pmod 3$):  Using Part~(\ref{T:propsofcong3}) of Theorem~\ref{T:propsofcong}, we see that
%\[
%\begin{aligned}
%  n^3  &\equiv 0^3\pmod 3,\text{ or} \\
%  n^3  &\equiv 0\pmod 3.  \\ 
%\end{aligned} 
%\]
%Since  $n^3  \equiv 0 \pmod 3$ and  $n \equiv 0 \pmod 3$, we can apply Part~(\ref{T:propsofcong1}) of Theorem~\ref{T:propsofcong} to obtain
%\[
%\begin{aligned}
%  \left( {n^3  - n} \right) &\equiv \left( {0 - 0} \right) \pmod 3 \\ 
%  \left( {n^3  - n} \right) &\equiv 0 \pmod 3. \\ 
%\end{aligned}
%\]
%The last congruence tells us that  $3 \mid \left[ {\left( {n^3  - n} \right) - 0} \right]$
%or that  $3 \mid \left( {n^3  - n} \right)$.
%
%%The other two cases are handled similarly.  The computations using Theorem~\ref{T:propsofcong} are shown below.
%
%\noindent
%\textit{Case 2} ($n \equiv 1 \pmod 3$):  In this case, $n^3  \equiv 1^3 \pmod 3$ or  
%$n^3  \equiv 1 \pmod 3$.  So we obtain
%\[
%\begin{aligned}
%  \left( {n^3  - n} \right) &\equiv \left( {1 - 1} \right) \pmod 3 \\ 
%  \left( {n^3  - n} \right) &\equiv 0 \pmod 3. \\ 
%\end{aligned}
%\]
%
%\noindent
%\textit{Case 3} ($n \equiv 2 \pmod 3$):  The details of this case are part of Exercise~(\ref{exer:prop3divides}).
%%$n \equiv 2 \pmod 3$.  In this case, $n^3  \equiv 2^3 \pmod 3$ or  
%%$n^3  \equiv 8 \pmod 3$.  So we obtain
%%\[
%%\begin{aligned}
%%  \left( {n^3  - n} \right) &\equiv \left( {8 - 2} \right) \pmod 3 \\ 
%%  \left( {n^3  - n} \right) &\equiv 6 \pmod 3 \\ 
%%\end{aligned}
%%\]
%%Since $6 \equiv 0 \pmod 3$, we can use the last equation to conclude that 
%%\[
%%\left( {n^3  - n} \right) \equiv 0 \pmod 3.
%%\]
%
%In all three cases, we have proven that  $\left( {n^3  - n} \right) \equiv 0 \pmod 3$, and this implies that  $3 \mid \left( n^3  - n  \right)$.  Hence, we can conclude that if  $n$  is an integer, then  3  divides  $n^3  - n$.
%\end{myproof}
%\index{congruence!Division Algorithm|)}%
%\index{Division Algorithm!congruence|)}%
%\hbreak
%\begin{example}[The Last Digit of a Large Number] \hfill \\
%Theorem~\ref{T:propsofcong} provides tools to explore certain properties of natural numbers.  For example, since we know that  $3^4  = 81$, we can conclude that
%\[
%3^4  \equiv 1 \pmod {10}.
%\]
%This is not much of a surprise, but if we use this fact with Theorem~\ref{T:propsofcong}, we can arrive at some not quite so obvious results.  Using the second part of Theorem~\ref{T:propsofcong} with  $a = c = 3^4 $ and  $b = d = 1$, we see that
%\[
%\begin{aligned}
%  3^4  \cdot 3^4  &\equiv 1 \cdot 1 \pmod {10} \\ 
%  3^8  &\equiv 1 \pmod {10}. \\ 
%\end{aligned} 
%\]
%This tells us that the last digit in the decimal representation of  $3^8$ is 1.  We can also use Part~(\ref{T:propsofcong3}) of Theorem~\ref{T:propsofcong} and the fact that  
%$3^4  \equiv 1 \pmod {10}$ to conclude that
%\[
%\begin{aligned}
%  \left( {3^4 } \right)^{100}  &\equiv 1^{100}\pmod {10} \\ 
%  3^{400}  &\equiv 1 \pmod {10}. \\ 
%\end{aligned}
%\]
%This tells us that the last digit in the decimal representation of  $3^{400} $ is 1.
%\end{example}
%\hbreak
%%
%\begin{activity}[The Last Two Digits of a Large Integer]\label{A:lasttwo} \hfill 
%\begin{enumerate}
%\item Use the fact that  $3^4  \equiv 81 \pmod {100}$ to prove that 
%$3^{16}  \equiv 21 \pmod {100}$.  What does this tell you about the last two digits in the decimal representation of  $3^{16} $?
%\label{A:lasttwo1}%
%
%\item Use the two congruences in Part~(\ref{A:lasttwo1}) and laws of exponents to determine  $r$  where  $3^{20}  \equiv r \pmod {100}$ and  $r \in \mathbb{Z}$ with  $0 \leq r < 100$
%. What does this tell you about the last two digits in the decimal representation of  $3^{20} $?
%
%\item Determine the last two digits in the decimal representation of  $3^{400} $.
%
%\item Determine the last two digits in the decimal representation of  $4^{804} $.
%
%\end{enumerate}
%\hint  One way is to determine the  ``mod 100 values''  for  $4^2$, $4^4$, $4^8$, $4^{16}$, $4^{32}$, $4^{64}$,~and so on.  Then use these values and laws of exponents to determine 
%$r$,  where  $4^{804}  \equiv r \pmod {100}$ and  $r \in \mathbb{Z}$ with  $0 \leq r < 100$.
%\end{activity}
%\hbreak

\endinput
