\subsection*{Using Cases Determined by the Division Algorithm}
The Division Algorithm
\index{Division Algorithm!using cases|(}%
 can sometimes be used to construct cases that can be used to prove a statement that is true for all integers.  We have done this when we divided the integers into the even integers and the odd integers since even integers have a remainder of 0 when divided by 2 and odd integers have a remainder of 1 when divided by 2.

%The Division Algorithm states that when we divide an integer  $a$  by  2, there are two possible remainders:  0  and  1.  The remainder is  0  if and only if  $a$  is even, and the remainder is  1  if and only if  $a$  is odd.

Sometimes it is more useful to divide the integer  $a$  by an integer other than  2.  For example, if  $a$  is divided by  3, there are three possible remainders:  0, 1, and 2.  If  $a$  is divided by 4, there are four possible remainders:  0, 1, 2, and 3.  The remainders form the basis for the cases.  

If the hypothesis of a proposition is that  ``$n$  is an integer,'' then we can use the Division Algorithm to claim that there are unique integers  $q$  and  $r$  such that
\[
n = 3q + r\text{  and  }0 \leq r < 3.
\]
We can then divide the proof into the following three cases:  (1)  $r = 0$;  (2)  $r = 1$;  and (3) $r = 2$.  This is done in Proposition~\ref{P:3divides}.
%\hbreak
%\begin{activity}[A Proof Using the Division Algorithm] \label{A:divalgoproof} \hfill
%
%Complete the details for Case (2) and Case (3) in the proof of the following proposition:

\begin{proposition}\label{P:3divides}
If $n$ is an integer, then 3 divides $n^3-n$.
\end{proposition}
\begin{myproof}
Let $n$ be an integer.  We will show that 3 divides $n^3-n$ by examining the three cases for the remainder when $n$ is divided by 3.  By the Division Algorithm, there exist unique integers  $q$  and  $r$  such that
\[
n = 3q + r\text{, and  }0 \leq r < 3.
\]
This means that we can consider the following three cases:  (1)  $r = 0$;  (2)  $r = 1$;  and 
 (3)  $r = 2$.
%\vskip10pt

%\noindent
%\textit{Case 1} ($r = 0$): 
In the case where $r = 0$, we have $n = 3q$.  By substituting this into the expression  $n^3  - n$, we get
\[
\begin{aligned}
  n^3  - n &= \left( {3q} \right)^3  - \left( {3q} \right) \\ 
   &= 27q^3  - 3q \\ 
   &= 3\left( {9q^3  - q} \right). \\ 
\end{aligned} 
\]
Since $\left( {9q^3 -q} \right)$ is an integer, the last equation proves that 
$3 \mid \left( n^3 - n \right)$.
%\vskip10pt

%\noindent
%\underline{Case 2} ($r = 1$):  
In the second case,  $r = 1$ and $n = 3q + 1$.  When we substitute this into 
$\left( n^3 - n \right)$, we obtain
\[
\begin{aligned}
  n^3  - n &= \left( {3q + 1} \right)^3  - \left( {3q + 1} \right) \\ 
           &= \left( 27q^3 + 27q^2 +9q + 1\right) - \left( 3q + 1 \right) \\
           &= 27q^3 + 27q^2 + 6q \\                            
           &= 3 \left( 9q^3 + 9q^2 + 2q \right). \\
\end{aligned} 
\]
Since $\left( {9q^3 +9q^2 + 2q} \right)$ is an integer, the last equation proves that 
$3 \mid \left( n^3 - n \right)$.

The last case is when $r = 2$.  The details for this case are part of Exercise~(\ref{exer:prop3divides}).
%In this case,  $n = 3q + 2$, and when we substitute this into 
%$\left( n^3 - n \right)$, we obtain
%\[
%\begin{aligned}
%  n^3  - n &= \left( {3q + 2} \right)^3  - \left( {3q + 2} \right) \\ 
%           &= \cdots. \\ 
%\end{aligned} 
%\]
%\vskip10pt
Once this case is completed, we will have proved that  3 divides  $n^3  - n$ in all three cases.  Hence, we may conclude that if  $n$  is an integer, then  3  divides  $n^3  - n$.
\end{myproof}
\index{Division Algorithm!using cases|)}%

%\end{activity}
\hbreak

\endinput
