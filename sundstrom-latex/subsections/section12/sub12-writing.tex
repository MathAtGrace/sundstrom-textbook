\subsection*{Writing Guidelines for Mathematics Proofs}
\index{writing guidelines|(}%
%In this section, the emphasis is on constructing an outline of a proof using a know-show table.  However, some proof writing will be done, and 
At the risk of oversimplification, doing mathematics can be considered to have two distinct stages.  The first stage is to convince yourself that you have solved the problem or proved a conjecture.  This stage is a creative one and is quite often how mathematics is actually done.  The second equally important stage is to convince other people that you have solved the problem or proved the conjecture.  This second stage often has little in common with the first stage in the sense that it does not really communicate the process by which you solved the problem or proved the conjecture.   However, it is an important part of the process of communicating mathematical results to a wider audience.

A \textbf{mathematical proof} is a convincing argument (within the accepted standards of the mathematical community) that a certain \index{proof}%
mathematical statement is necessarily true.  A proof generally uses deductive reasoning and logic but also contains some amount of ordinary language (such as English).  A mathematical proof that you write should convince an appropriate audience that the result you are proving is in fact true. So we do not consider a proof complete until there is a well-written proof.  So it is important to introduce some writing guidelines.  The preceding proof was written according to the following basic guidelines for writing proofs.  More writing guidelines will be given in Chapter~\ref{C:proofs}.
\begin{enumerate}
%\item \label{writing:know}%
%\textbf{Know Your Audience}. 
%
%Every writer should have a clear idea of the intended audience for a piece of writing.  In that way, the writer can give the right amount of information at the proper level of sophistication to communicate effectively.  This is especially true for mathematical writing.  For example, if a mathematician is writing a solution to a textbook problem for a solutions manual for instructors, the writing would be brief with many details omitted.  However, if the writing was for a students' solution manual, more details would be included.  %This is why the instructions for Beginning Activity~\ref{PA:equation} stated that your descriptions should be written for someone who already knows basic algebra and how to solve quadratic equations.


\item \textbf{Begin with a carefully worded statement of the theorem or result to be proven.}
This should be a simple declarative statement of the theorem or result.  Do not simply rewrite the problem as stated in the textbook or given on a handout.  Problems often begin with phrases such as ``Show that'' or ``Prove that.''  This should be reworded as a simple declarative statement of the theorem.  Then skip a line and write ``Proof''  in italics or boldface font (when using a word processor).  Begin the proof on the same line.  Make sure that all paragraphs can be easily identified.  Skipping a line between paragraphs or indenting each paragraph can accomplish this.

As an example, an exercise in a text might read, ``Prove that if $x$  is an odd integer, then $x^2$ is an odd integer.''  This could be started as follows:

\textbf{Theorem.} 
If  $x$  is an odd integer, then $x^2$ is an odd integer.

\textbf{\emph{Proof}}:  We assume that  $x$  is an odd integer  $\ldots$

\item \textbf{Begin the proof with a statement of your assumptions.}
Follow the statement of your assumptions with a statement of what you will prove.

\noindent
\textbf{Theorem.} 
If  $x$  is an odd integer, then $x^2$ is an odd integer.

%\begin{flushleft}
\noindent
\emph{\textbf{Proof}}.  We assume that  $x$  is an odd integer and will prove that $x^2$   is an odd integer.
%\end{flushleft}

\item \textbf{Use the pronoun ``we.''}
If a pronoun is used in a proof, the usual convention is to use ``we'' instead of ``I.''  The idea is to stress that you and the reader are doing the mathematics together.  It will help encourage the reader to continue working through the mathematics.  Notice that we started the proof of Theorem~\ref{T:xyodd} with ``We assume that $\ldots$ .''

%If a pronoun is used in a proof, the usual convention is to use ``we'' instead of ``I.''  The idea is that the author and the reader are proving the theorem together.



\item \textbf{Use italics for variables when using a word processor.}
When using a word processor to write mathematics, the word processor needs to be capable of producing the appropriate mathematical symbols and equations.  The mathematics that is written with a word processor should look like typeset mathematics.  This means that italics is used for variables, boldface font is used for vectors, and regular font is used for mathematical terms such as the names of the trigonometric and logarithmic functions.  

For example, we do not write sin (x) or \emph{sin (x)}.  The proper way to typeset this is $\sin (x)$.



%\item \textbf{Do not use $*$ for multiplication or \^{} for exponents.} \\
%Leave this type of notation for writing computer code.  The use of this notation makes it difficult for humans to read.  In addition, avoid using $/$ for division when using a complex fraction.  
%
%For example, it is very difficult to read 
%$\left(x^3 -3x^2 + 1/2 \right)/\left(2x/3 - 7\right)$; the fraction
%\[
%\frac{x^3 - 3x^2 +\dfrac{1}{2}}{\dfrac{2x}{3} - 7}
%\]
%is much easier to read.


%\item \textbf{Use complete sentences and proper paragraph structure.}
%
%Good grammar is an important part of any writing.  Therefore, conform to the accepted rules of grammar.  Pay careful attention to the structure of sentences.  Write proofs using \textbf{complete sentences} but avoid run-on sentences.  Also, do not forget punctuation, and always use a spell checker when using a word processor.


\item \textbf{Display important equations and mathematical expressions.}
Equations and manipulations are often an integral part of mathematical exposition.  Do not write equations, algebraic manipulations, or formulas in one column with reasons given in another column. 
%(as is often done in geometry texts).
   Important equations and manipulations should be displayed.  This means that they should be centered with blank lines before and after the equation or manipulations, and if the left side of the equations does not change, it should not be repeated.  For example,

Using algebra, we obtain	
\begin{align}
  x \cdot y &= \left( {2m + 1} \right)\left( {2n + 1} \right)  \notag \\ 
            &= 4mn + 2m + 2n + 1  \notag \\ 
            &= 2\left( {2mn + m + n} \right) + 1.  \notag  
\end{align} 
Since  $m$  and  $n$  are integers, we conclude that $ \ldots $ .

%\item \textbf{Do not use a mathematical symbol at the beginning of a sentence.}
%For example, we should not write, ``Let $n$ be an integer.  $n$ is an odd integer provided that \ldots''  Many people find this hard to read and often have to re-read it to understand it.  It would be better to write, ``An integer $n$ is an odd integer provided that \ldots''


\item \textbf{Tell the reader when the proof has been completed.}
Perhaps the best way to do this is to simply write, ``This completes the proof.''  Although it may seem repetitive, a good alternative is to finish a proof with a sentence that states precisely what has been proven.  In any case, it is usually good practice to use  some ``end of proof symbol'' such as  $\blacksquare$.
\index{writing guidelines|)}%


%\item \textbf{Keep it simple}.
%
%It is often difficult to understand a mathematical argument no matter how well it is written.  Do not let your writing help make it more difficult for the reader.  Use simple, declarative sentences and short paragraphs, each with a simple point.
\end{enumerate}
\hbreak

\begin{prog}[\textbf{Proving Propositions}] \label{prog:proving} \hfill \\
Construct a know-show table for each of the following propositions and then write a formal proof for one of the propositions.
\begin{enumerate}
  \item If $x$ is an even integer and $y$ is an even integer, then $x + y$ is an even integer.
  \item If $x$ is an even integer and $y$ is an odd integer, then $x + y$ is an odd integer.
  \item If $x$ is an odd integer and $y$ is an odd integer, then $x + y$ is an even integer.
\end{enumerate}
\end{prog}
\hbreak



\endinput
