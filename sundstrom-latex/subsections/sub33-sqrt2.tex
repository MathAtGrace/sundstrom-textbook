\subsection*{The Square Root of 2 Is an Irrational Number}
%\begin{activity}[\textbf{The Square Root of 2 Is Irrational}] \hfill \\
The proof that the square root of  2 is an irrational number is one of the classic proofs in mathematics, and  every mathematics student should know this proof.  %This is why we are completing this proof through a ``guided activity'' rather than simply writing a proof.  The proposition can be stated as follows:  
This is why we will be doing some preliminary work with rational numbers and integers before completing the proof.  The theorem we will be proving can be stated as follows:

\begin{list}{}
  \item If  $r$  is a real number such that  $r^2  = 2$, then  $r$  is an irrational number.
\end{list}
\newpar
This is stated in the form of a conditional statement, but it basically means that  $\sqrt 2$
 is irrational (and that   $ - \sqrt 2$ is irrational).  That is,  $\sqrt 2$  cannot be written as a quotient of integers with the denominator not equal to zero.

In order to complete this proof, we need to be able to work with some basic facts about rational numbers and a result about even integers.  The following questions are meant to review these properties.

\begin{enumerate}
%  \item Write a complete definition of a rational number.  See Exercise~(\ref{exer:sec32-rational}) from Section~\ref{S:moremethods} on 
%page~\pageref{exer:sec32-rational}.  Give examples of at least five different rational numbers.

  \item Are integers rational numbers?  Explain.

  \item Notice that $\dfrac{2}{3} = \dfrac{4}{6}$, since
\begin{align*}
\frac{4}{6} &= \frac{2 \cdot 2}{3 \cdot 2} = \frac{2}{2} \cdot \frac{2}{3} = \frac{2}{3}
\end{align*}
Show that $\dfrac{15}{12} = \dfrac{5}{4}$, $\dfrac{10}{-8} = \dfrac{-5}{4}$, and $\dfrac{-30}{-16} = \dfrac{15}{8}$ \label{LA:learn33-eq2}%

%  \item Are any of the rational numbers $\dfrac{-5}{4}$, $\dfrac{-10}{8}$, $\dfrac{18}{27}$, and $\dfrac{-8}{-12}$ equal? \label{q:4}%

%  \item What does it mean to say that a real number $r$ is an irrational number?  Explain by writing a precise negation of the definition of a rational number. 
\end{enumerate}
Question~(\ref{LA:learn33-eq2}) was included to illustrate the fact that a rational number can be written as a fraction in ``lowest terms'' with a positive denominator.  This means that any rational number can be written as a quotient $\dfrac{m}{n}$, where $m$ and $n$ are integers, $ n > 0 $, and $m$ and $n$ have no common factor greater than 1.  This fact will be used in a proof by contradiction that the square root of 2 is an irrational number.  (Theorem~\ref{T:squareroot2})

\setcounter{oldenumi}{\theenumi}
\begin{enumerate} \setcounter{enumi}{\theoldenumi}
  \item If $n$ is an integer and $n^2$ is even, what can be conclude about $n$.  Refer to Theorem~\ref{T:n2odd} on page~\pageref{T:n2odd}.
\end{enumerate}




\setcounter{oldenumi}{\theenumi}
\begin{enumerate} \setcounter{enumi}{\theoldenumi}
\item Complete the indicated steps in the proof of Theorem~\ref{T:squareroot2}.
\end{enumerate}


\setcounter{equation}{0}
\begin{theorem}\label{T:squareroot2}
If  $r$  is a real number such that  $r^2  = 2$, then  $r$  is an irrational number.
\end{theorem}

\begin{myproof}
We will use a proof by contradiction.  So we assume that the statement of the theorem is false.  That is, we assume that
\begin{list}{}
  \item there exists a real number $r$ such that $r^2  = 2$ and that  $r$  is a rational number.
\end{list}
\vskip6pt

\noindent
Since  $r$  is a rational number, there exist integers  $m$  and  $n$  with  $n > 0$ such that  
\begin{equation} \label{eq:sqrt-1}
r = \frac{m}{n}
\end{equation}
and  $m$  and  $n$  have no common factor greater than 1.  We will obtain a contradiction by showing that  $m$  and  $n$  must both be even.
\begin{itemize} 
  \item Square both sides of equation~(\ref{eq:sqrt-1}) and then substitute 2 for $r^2$.  This will be equation (2).
  \item Multiply both sides of equation~(2) by $n^2$.  This will be equation~(3).
  \item Why is $m^2$ even?  Using Theorem~\ref{T:n2odd}, what can be concluded about the integer $m$?
  \item So there exists an integer $p$ such that $m = 2p$.  Use this to make a substitution in equation~(3). This will be equation~(4).
  \item Show that equation~(4) can be rewritten in the form $2p^2 = n^2$.  This is equation~(5). 
  \item As with $m^2$ and $m$, use equation~(5) to prove that $n$ must be even. 
\end{itemize}
We have now established that both  $m$  and  $n$  are even.  This means that  2  is a common factor of  $m$  and  $n$, which contradicts the assumption that $m$  and  $n$  have no common factor greater than 1.  Consequently, the statement of the theorem cannot be false, and we have proven that if  $r$  is a real number such that  $r^2  = 2$, then  $r$  is an irrational number.
\end{myproof}
%\end{activity}

\hbreak


\endinput

%\setcounter{equation}{0}
%\begin{theorem}\label{T:squareroot2}
%If  $r$  is a real number such that  $r^2  = 2$, then  $r$  is an irrational number.
%\end{theorem}
%%
%\begin{myproof}
%We will use a proof by contradiction.  So we assume that the statement of the theorem is false.  That is, we assume that
%\begin{list}{}
%  \item $r$  is a real number, $r^2  = 2$, and that  $r$  is a rational number.
%\end{list}
%\vskip6pt
%
%\noindent
%Since  $r$  is a rational number, there exist integers  $m$  and  $n$  with  $n > 0$ such that  
%\begin{equation} \label{eq:sqrt-1}
%r = \frac{m}{n}
%\end{equation}
%and  $m$  and  $n$  have no common factor greater than 1.  We will obtain a contradiction by showing that  $m$  and  $n$  must both be even.
%\begin{itemize} 
%  \item Square both sides of equation~(\ref{eq:sqrt-1}) and then substitute 2 for $r^2$.  This will be equation (2).
%  \item Multiply both sides of equation~(2) by $n^2$.  This will be equation~(3).
%  \item Why is $m^2$ even?  Using Theorem~\ref{T:n2odd}, what can be concluded about the integer $m$?
%  \item So there exists an integer $p$ such that $m = 2p$.  Use this to make a substitution in equation~(3). This will be equation~(4).
%  \item Show that equation~(4) can be rewritten in the form $2p^2 = n^2$.  This is equation~(5). 
%  \item As with $m^2$ and $m$, use equation~(5) to prove that $n$ must be even. 
%\end{itemize}
%We have now established that both  $m$  and  $n$  are even.  This means that  2  is a common factor of  $m$  and  $n$, which contradicts the assumption that $m$  and  $n$  have no common factor greater than 1.  Consequently, the statement of the theorem cannot be false, and we have proven that if  $r$  is a real number such that  $r^2  = 2$, then  $r$  is an irrational number.
%\end{myproof}
%\end{Lactivity}
%\hbreak

\endinput
