\subsection*{Congruence}
\index{congruence|(}%
What mathematicians call congruence is a concept used to describe cycles in the world of the integers. For example, the day of the week is a cyclic phenomenon in that the day of the week repeats every seven days.  The time of the day is a cyclic phenomenon because it repeats every 12 hours if we use a 12-hour clock  or every 24 hours if we use a 24-hour clock.  We explored these two cyclic phenomena in \typeu Activity~\ref*{PA:calender}.

Similar to what we saw in \typeu Activity~\ref*{PA:calender}, if it is currently Monday, then it will be Wednesday 2 days from now, 9 days from now, 16 days from now, 23 days from now, and so on.  In addition, it was Wednesday 5 days ago, 12 days ago, 19 days ago, and so on.  Using negative numbers for time in the past, we generate the following list of numbers:

\[
 \ldots ,\; - 19,  - 12,  - 5, 2, 9, 16, 23,  \ldots .
\]

Notice that if we subtract any number in the list above from any other number in that list, we will obtain a multiple of 7.  For example,

\[
\begin{aligned}
  16 - 2 &= 14 = 7 \cdot 2 \\ 
  \left( { - 5} \right) - \left( 9 \right) &=  - 14 = 7 \cdot \left( { - 2} \right) \\ 
  16 - \left( { - 12} \right) &= 28 = 7 \cdot 4. \\ 
\end{aligned} 
\]

Using the concept of congruence, we would say that all the numbers in this list are congruent modulo 7, but we first have to define when two numbers are congruent modulo some natural number  $n$.

\begin{defbox}{congruence}{Let  $n \in \mathbb{N}$.  If  $a$  and  $b$  are integers, then we say that \textbf{$\boldsymbol{a}$  is congruent to  $\boldsymbol{b}$  modulo  $\boldsymbol{n}$}
\index{congruent modulo $n$}%
  provided that  $n$  divides  $a - b$.  A standard notation for this is   
$a \equiv b \pmod n$.
\label{sym:congruence}%
  This is read as ``$a$  is congruent to  $b$  modulo  $n$''   or  ``$a$  is congruent to  $b$  mod  $n$.''}
\end{defbox}
%
\noindent
Notice that we can use the definition of divides to say that  $n$ divides $(a - b)$  if and only if  there exists an integer  $k$  such that  $a - b = nk$.  So we can write

\[
\begin{aligned}
  a &\equiv b \pmod n \text{  means  }\left( {\exists k \in \mathbb{Z}} \right)\left( {a - b = nk} \right), \text{or} \hfill \\
  a &\equiv b \pmod n \text{  means  }\left( {\exists k \in \mathbb{Z}} \right)\left( {a = b + nk} \right). \hfill \\ 
\end{aligned}
\]

This means that in order to find integers that are congruent to $b$ modulo $n$, we only need to add multiples of $n$ to $b$.  For example, to find integers that are congruent to 2 modulo 5, we add multiples of 5 to 2.  This gives the following list:

\[
\ldots, -13, -8, -3, 2, 7, 12, 17, \ldots .
\]
We can also write this using set notation and say that

\[
\left\{ \left. a \in \Z \right| a \equiv 2 \pmod 5 \right\} = 
\left\{ \ldots, -13, -8, -3, 2, 7, 12, 17, \ldots \right\}.
\]
\hbreak

\begin{prog}[\textbf{Congruence Modulo 8}]\label{pr:congruence} \hfill
\begin{enumerate}
\item  Determine at least eight different integers that are congruent to 5 modulo 8.

\item Use set builder notation and the roster method to specify the set of all integers that are congruent to 5 modulo 8.

\item Choose two integers that are congruent to 5 modulo 8 and add them.  Then repeat this for at least five other pairs of integers that are congruent to 5 modulo 8.
\label{pr:congruence3}% 

\item Explain why all of the sums that were obtained in Part~(\ref{pr:congruence3}) are congruent to 2 modulo 8.
\end{enumerate}
\end{prog}
\hbreak

We will study the concept of congruence modulo  $n$  in much more detail later in the text.  For now, we will work with the definition of congruence modulo $n$  in the context of proofs.  For example, all of the examples used in Progress Check~\ref{pr:congruence} should provide evidence that the following proposition is true.
\begin{proposition} \label{prop:congruenceproof}
For all integers $a$ and $b$, if $\mod{a}{5}{8}$ and $\mod{b}{5}{8}$, then 
$\mod{(a + b)}{2}{8}$.
\end{proposition}
%
\begin{prog}[\textbf{Proving Proposition~\ref{prop:congruenceproof}}]\label{pr:congruence2} \hfill \\
We will use ``backward questions'' and ``forward questions'' to help construct a proof for Proposition~\ref{prop:congruenceproof}.  So, we might ask, ``How do we prove that 
$\mod{(a + b)}{2}{8}$?''  One way to answer this is to use the definition of congruence and state that $\mod{(a + b)}{2}{8}$ provided that 8 divides $(a + b - 2)$.
\begin{enumerate}
  \item Use the definition of divides to determine a way to prove that 8 divides $(a + b - 2)$.
\end{enumerate}

We now turn to what we know and ask, ``What can we conclude from the assumptions that 
$\mod{a}{5}{8}$ and $\mod{b}{5}{8}$?''  We can again use the definition of congruence and conclude that 8 divides $(a - 5)$ and 8 divides $(b - 5)$.
\end{prog}
\setcounter{oldenumi}{\theenumi}
\begin{enumerate} \setcounter{enumi}{\theoldenumi}
\item Use the definition of divides to make conclusions based on the facts that 8 divides 
$(a - 5)$ and 8 divides $(b - 5)$.
\item Solve an equation from part~(2) for $a$ and for $b$.
\item Use the results from part~(3) to prove that 8 divides $(a + b - 2)$.
\item Write a proof for Proposition~\ref{prop:congruenceproof}.
\end{enumerate}

\index{congruence|)}%
\hbreak

\endinput
