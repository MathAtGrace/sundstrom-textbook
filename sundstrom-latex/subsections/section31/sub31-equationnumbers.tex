\subsection*{Writing Guidelines for Equation Numbers}
\index{writing guidelines}%

We wrote the proof for Theorem~\ref{T:transdivide} according to the guidelines introduced in Section~\ref{S:direct}, but a new element that appeared in this proof was the use of equation numbers.  Following are some guidelines that can be used for \textbf{equation numbers}.
\index{equation numbers}%

If it is necessary to refer to an equation later in a proof, that equation should be centered and displayed.  It should then be given a number.  The number for the equation should be written in parentheses on the same line as the equation at the right-hand margin as in shown in the following example.

%\begin{example} \label{E:eqnum} \hfill
\setcounter{equation}{0}

\eighth
Since  $x$  is an odd integer, there exists an integer  $n$  such that
\begin{equation}\label{eq:3e}
x = 2n + 1.
\end{equation}

\begin{flushleft}
Later in the proof, there may be a line such as
\begin{center}
Then, using the result in equation~(\ref{eq:3e}), we obtain  \ldots .
\end{center}
Notice that we did not number every equation in Theorem~\ref{T:transdivide}.  We should only number those equations we will be referring to later in the proof, and we should only number equations when it is necessary.  For example, instead of numbering an equation, it is often better to use a phrase such as, ``the previous equation proves that \ldots" or ``we can rearrange the terms on the right side of the previous equation.'' 
Also, note that the word ``equation'' is not capitalized when we are referring to an equation by number.  Although it may be appropriate to use a capital ``E,'' the usual convention in mathematics is not to capitalize.
%Also, note that the word ``Equation'' begins with a capital ``E'' when we are referring to an equation by number.
\end{flushleft}
%\end{example}
\hbreak

\begin{prog}[\textbf{A Property of Divisors}]\label{pr:divisors} \hfill
%In this activity, the universal set for each variable is the set of integers.
\begin{enumerate}
\item Give at least four different examples of integers  $a$, $b$, and  $c$  with $a \ne 0$ such that  $a$  divides  $b$  and  $a$  divides  $c$.
\label{pr:divisors1}%

\item For each example in Part~(\ref{pr:divisors1}), calculate the sum  $b + c$.  Does the integer  $a$  divide the sum  $b + c$?
\label{pr:divisors2}%

\item Construct a know-show table for the following proposition:  For all integers $a$, $b$, and  $c$  with 
$a \ne 0$, if $a$ divides $b$ and $a$ divides $c$, then $a$ divides $(b + c)$.
\label{pr:divisors3}%
 
%\item Construct a Know-show table for a proof of the conjecture in Part~(\ref{pr:divisors3}).

\end{enumerate}
\end{prog}
\hbreak

\endinput
