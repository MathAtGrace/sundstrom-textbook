\subsection*{Additional Writing Guidelines}
\index{writing guidelines|(}%
We will now be writing many proofs, and it is important to make sure we write according to accepted guidelines so that our proofs may be understood by others.  Some writing guidelines were introduced in Chapter~\ref{C:intro}.  The first four writing guidelines given below can be considered general guidelines, and the last three can be considered as technical guidelines specific to writing in mathematics.
\begin{enumerate}
\item \label{writing:know}%
\textbf{Know your audience.} 
Every writer should have a clear idea of the intended audience for a piece of writing.  In that way, the writer can give the right amount of information at the proper level of sophistication to communicate effectively.  This is especially true for mathematical writing.  For example, if a mathematician is writing a solution to a textbook problem for a solutions manual for instructors, the writing would be brief with many details omitted.  However, if the writing was for a students' solution manual, more details would be included.

\item \textbf{Use complete sentences and proper paragraph structure.}
Good grammar is an important part of any writing.  Therefore, conform to the accepted rules of grammar.  Pay careful attention to the structure of sentences.  Write proofs using \textbf{complete sentences} but avoid run-on sentences.  Also, do not forget punctuation, and always use a spell checker when using a word processor.

\item \textbf{Keep it simple}.
It is often difficult to understand a mathematical argument no matter how well it is written.  Do not let your writing help make it more difficult for the reader.  Use simple, declarative sentences and short paragraphs, each with a simple point.

\item \textbf{Write a first draft of your proof and then revise it.} 
Remember that a proof is written so that readers are able to read and understand the reasoning in the proof.  Be clear and concise.  Include details but do not ramble.  Do not be satisfied with the first draft of a proof.  Read it over and refine it.  Just like any worthwhile activity, learning to write mathematics well takes practice and hard work.  This can be frustrating.  Everyone can be sure that there will be some proofs that are difficult to construct, but remember that proofs are a very important part of mathematics.  So work hard and have fun.

\item \textbf{Do not use $*$ for multiplication or \^{} for exponents.}
Leave this type of notation for writing computer code.  The use of this notation makes it difficult for humans to read.  In addition, avoid using $/$ for division when using a complex fraction.  

For example, it is very difficult to read 
$\left(x^3 -3x^2 + 1/2 \right)\!/\!\left(2x/3 - 7\right)$; the fraction
\[
\frac{x^3 - 3x^2 +\dfrac{1}{2}}{\dfrac{2x}{3} - 7}
\]
is much easier to read.

\item \textbf{Do not use a mathematical symbol at the beginning of a sentence.}
For example, we should not write, ``Let $n$ be an integer.  $n$ is an odd integer provided that \ldots .''  Many people find this hard to read and often have to re-read it to understand it.  It would be better to write, ``An integer $n$ is an odd integer provided that \ldots .''

\item \textbf{Use English and minimize the use of cumbersome notation}.  Do not use the special symbols for quantifiers $\forall$ (for all), 
$\exists$ (there exists), $\mathrel\backepsilon$ (such that), or $\therefore $ (therefore) in formal mathematical writing.  It is often easier to write, and usually easier to read, if the English words are used instead of the symbols.  For example, why make the reader interpret
\[
\left( \forall x \in \R \right) \left( \exists y \in \R \right)\left( x + y = 0 \right)
\]
when it is possible to write
\begin{center}
For each real number $x$, there exists a real number $y$ such that $x + y = 0$,
\end{center}
or, more succinctly (if appropriate),
\begin{center}
Every real number has an additive inverse.
\end{center}
\end{enumerate}
\index{writing guidelines|)}%
\hbreak

\endinput
