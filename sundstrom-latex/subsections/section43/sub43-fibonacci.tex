\subsection*{The Fibonacci Numbers}

The Fibonacci numbers form a famous sequence in mathematics that was investigated by Leonardo of Pisa
\index{Leonardo of Pisa}%
 (1170 -- 1250), who is better known as Fibonacci.  Fibonacci introduced this sequence to the Western world as a solution of the following problem:

\begin{list}{}
\item Suppose that a pair of adult rabbits (one male, one female) produces a pair of rabbits (one male, one female) each month.  Also, suppose that newborn rabbits become adults in two months and produce another pair of rabbits.  Starting with one adult pair of rabbits, how many pairs of rabbits will be produced each month for one year?
\end{list}
\vskip10pt

Since we start with one adult pair, there will be one pair produced the first month, and since there is still only one adult pair,  one pair will also be produced in the second month (since the new pair produced in the first month is not yet mature).  In the third month, two pairs will be produced, one by the original pair and one by the pair which was produced in the first month. In the fourth month, three pairs will be produced, and in the fifth month, five pairs will be produced.     

The basic rule is that in a given month after the first two months, the number of adult pairs is the number of adult pairs one month ago plus the number of pairs born two months ago.  This is summarized in Table~\ref{T:fibonacci}, 
%on page~\pageref{T:fibonacci}, 
where the number of pairs produced is equal to the number of adult pairs, and the number of adult pairs follows the Fibonacci sequence of numbers that we developed in \typeu Activity~\ref*{PA:fibonaccinumbers}.

\begin{table}[h]
$$
\BeginTable
\def\C{\JustCenter}
\BeginFormat
| c | c | c | c | c | c | c | c | c | c | c |
\EndFormat
\_
| Months | 1 | 2 | 3 | 4 | 5 | 6 | 7 | 8 | 9 | 10 | \\ \_
| Adult Pairs | 1 | 1 | 2 | 3 | 5 | 8 | 13 | 21 | 34 | 55 | \\ \_1
| Newborn Pairs | 1 | 1 | 2 | 3 | 5 | 8 | 13 | 21 | 34 | 55 | \\ \_1
| Month-Old Pairs | 0 | 1 | 1 | 2 | 3 | 5 | 8 | 13 | 21 | 34 | \\ \_
\EndTable
$$
\caption{Fibonacci Numbers}
\label{T:fibonacci}
\end{table}

Historically, it is interesting to note that Indian mathematicians were studying these types of numerical sequences well before Fibonacci.  In particular, about fifty years before Fibonacci introduced his sequence, Acharya Hemachandra (sometimes spelled Hemchandra)
\index{Hemachandra, Acharya}%
 (1089 -- 1173) considered the following problem, which is from the biography of Hemachandra in the 
\emph{MacTutor History of Mathematics Archive} at {https://mathshistory.st-andrews.ac.uk/Biographies/Hemchandra/}.
%http://www-groups.dcs.st-and.ac.uk/~history/Biographies/Hemachandra.html.
\begin{list}{}
\item  Suppose we assume that lines are composed of syllables which are either short or long. Suppose also that each long syllable takes twice as long to articulate as a short syllable. A line of length $n$ contains $n$ units where each short syllable is one unit and each long syllable is two units. Clearly a line of length $n$ units takes the same time to articulate regardless of how it is composed. Hemchandra asks: How many different combinations of short and long syllables are possible in a line of length $n$? 
\end{list}
This is an important problem in the Sanskrit language since Sanskrit meters are based on duration rather than on accent as in the English Language.  The answer to this question generates a sequence similar to the Fibonacci sequence.  Suppose that $h_n$ is the number of patterns of syllables of length $n$.  We then see that  $h_1 = 1$ and $h_2 = 2$.  Now let $n$ be a natural number and consider pattern of length $n + 2$.  This pattern either ends in a short syllable or a long syllable.  If it ends in a short syllable and this syllable is removed, then there is a pattern of length $n+1$, and there are $h_{n+1}$ such patterns.  Similarly, if it ends in a long syllable and this syllable is removed, then there is a pattern of length $n$, and there are $h_n$ such patterns.  From this, we conclude that
\[
h_{n+2} = h_{n+1} + h_n.
\]
This actually generates the sequence 1, 2, 3, 5, 8, 13, 21, \ldots.
%
For more information about Hemachandra, see the article \emph{Math for Poets and Drummers} by Rachel Wells Hall in the February 2008 issue of \emph{Math Horizons}.

We will continue to use the Fibonacci sequence in this book.  This sequence may not seem all that important or interesting.  However, it turns out that this sequence occurs in nature frequently and has applications in computer science.  There is even a scholarly journal, \textit{The Fibonacci Quarterly},
\index{Fibonacci Quarterly}%
 devoted to the Fibonacci numbers.

The sequence of Fibonacci numbers is one of the most studied sequences in mathematics, due mainly to the many beautiful patterns it contains. Perhaps one observation you made in \typeu Activity~\ref*{PA:fibonaccinumbers} is that every third Fibonacci number is even.  This can be written as a proposition as follows:
\[
\text{For each natural number } n, f_{3n} \text{ is an even natural number}.
\]
As with many propositions associated with definitions by recursion, we can prove this using mathematical induction.  The first step is to define the appropriate open sentence. For this, we can let $P(n)$ be, 
``$f_{3n} $  is an even natural number.''

Notice that  $P( 1 )$ is true since  $f_3  = 2$.  We now need to prove the inductive step.  To do this, we need to prove that for each  $k \in \mathbb{N}$,

\begin{center}
if  $P( k )$  is true, then  $P( {k + 1} )$  is true.
\end{center}
That is, we need to prove that for each  $k \in \mathbb{N}$, if  $f_{3k} $ is even, then  $f_{3\left( {k + 1} \right)} $ is even.

So let's analyze this conditional statement using a know-show table.
%
$$
\BeginTable
\def\C{\JustCenter}
\BeginFormat
|p(0.4in)|p(2in)|p(1.8in)|
\EndFormat
\_
 | \textbf{Step}  |  \textbf{Know}  |  \textbf{Reason}   |  \\+02 \_
 | $P$     |  $f_{3k} $  is even.     |  Inductive hypothesis | \\ \_1
 | $P1$    |   $\left( {\exists m \in \mathbb{N}} \right)\left( {f_{3k}  = 2m} \right)$                              |  Definition of ``even integer''  |         \\ \_1
 | \C $\vdots$  |  \C $\vdots$                         | \C $\vdots$     |  \\ \_1
 | $Q1$    |  $\left( {\exists q \in \mathbb{N}} \right)\left( {f_{3\left( {k + 1} \right)}  = 2q} \right)$                                 |           |  \\  \_1  
 | $Q$     |  $f_{3\left( {k + 1} \right)} $ is even.   |   Definition of ``even integer''     |     \\ \_
 | \textbf{Step}  |  \textbf{Show}  |  \textbf{Reason}    | \\+20 \_
\EndTable
$$
%
The key question now is, ``Is there any relation between  $f_{3\left( {k + 1} \right)} $
and  $f_{3k} $?''  We can use the recursion formula that defines the Fibonacci sequence to find such a relation. 

The recurrence relation for the Fibonacci sequence states that a Fibonacci number (except for the first two) is equal to the sum of the two previous Fibonacci numbers.  If we write  $3\left( {k + 1} \right) = 3k + 3$, then we get  $f_{3\left( {k + 1} \right)}  = f_{3k + 3} $.  For  $f_{3k + 3} $, the two previous Fibonacci numbers are  $f_{3k + 2} $ and  $f_{3k + 1} $.  This means that
\[
f_{3k + 3}  = f_{3k + 2}  + f_{3k + 1}\!.
\]  
Using this and continuing to use the Fibonacci relation, we obtain the following:

\[
\begin{aligned}
f_{3\left( {k + 1} \right)}  &= f_{3k + 3}  \\ 
                             &= f_{3k + 2}  + f_{3k + 1}  \\ 
                             &= \left( {f_{3k + 1}  + f_{3k} } \right) + f_{3k + 1} . \\ 
\end{aligned} 
\]

The preceding equation states that  $f_{3\left( {k + 1} \right)}  = 2f_{3k + 1}  + f_{3k} $.  This equation can be used to complete the proof of the induction step.
\hbreak
%
\begin{prog}[\textbf{Every Third Fibonacci Number Is Even}] \label{prog:thirdfibonacci} \hfill \\
Complete the proof of Proposition~\ref{P:thirdfibonacci}.

\begin{proposition} \label{P:thirdfibonacci}
For each natural number $n$, the Fibonacci number $f_{3n}$ is an even natural number.
\end{proposition}

\noindent
\hint  We have already defined the predicate  $P\left( n \right)$ to be used in an induction proof and have proved the basis step.  
%[The predicate, $P\left( n \right)$, is 
%``$f_{3n} $  is an even natural number.'']  
Use the information in and after the preceding know-show table to help prove that if  $f_{3k} $ is even, then  $f_{3\left( {k + 1} \right)} $  is even.
\end{prog}
\hbreak

\endinput
