\subsection*{How Do We Decide If a Statement Is True or False?}\label{SS:true}
In mathematics, we often establish that a statement is true by writing a mathematical proof.  To establish that a statement is false, we often find a so-called counterexample.  (These ideas will be explored later in this chapter.)  So mathematicians must be able to discover and construct proofs.  In addition, once the discovery has been made, the mathematician must be able to communicate this discovery to others who speak the language of mathematics.  We will be dealing with these ideas throughout the text.

For now, we want to focus on what happens  before we start a proof.  One thing that mathematicians often do is to make a conjecture beforehand as to whether the statement is true or false.  This is often done through exploration.  The role of exploration in mathematics is often difficult because the goal is not to find a specific answer but simply to investigate.  Following are some techniques of exploration that might be helpful.
\index{statement|)}%
%
\subsection*{Techniques of Exploration}
\begin{itemize}
\item \textbf{Guesswork and conjectures}.
Formulate and write down questions and conjectures.  When we make a guess in mathematics, we usually call it a conjecture.
\index{conjecture}%

\item \textbf{Examples}.
\textbf{Constructing appropriate examples is extremely important}.  Exploration often requires looking at lots of examples.  In this way, we can gather information that provides evidence that a statement is true, or we might find an example that shows the statement is false.  This type of example is called a \textbf{counterexample}.
\index{counterexample}%

For example, if someone makes the conjecture that $\sin \! \left(2x \right) = 2 \sin (x)$, for all real numbers $x$, we can test this conjecture by substituting specific values for 
$x$.  One way to do this is to choose values of $x$ for which $\sin (x)$ is known.  Using 
$x = \dfrac{\pi}{4}$, we see that

\[
\begin{aligned}
\sin \! \left( 2 \left( \frac{\pi}{4} \right) \right) &= \sin \left(\frac{\pi}{2}\right) = 1, \text{ and} \\
2 \sin \left(\frac{\pi}{4}\right) &= 2 \left( \frac{\sqrt{2}}{2} \right) = \sqrt{2}.
\end{aligned}
\]

Since $1 \ne \sqrt{2}$, these calculations show that this conjecture is false.  However, if we do not find a counterexample for a conjecture, we usually cannot claim the conjecture is true.  The best we can say is that our examples indicate the conjecture is true.  As an example, consider the conjecture that

\begin{center}
If $x$ and $y$ are odd integers, then $x + y$ is an even integer.
\end{center}

We can do lots of calculations, such as $3 + 7 = 10$ and $5 + 11 = 16$, and find that every time we add two odd integers, the sum is an even integer.  However, it is not possible to test every pair of odd integers, and so we can only say that the conjecture appears to be true.  (We will prove that this statement is true in the next section.)

\item \textbf{Use of prior knowledge}.
This also is very important.  We cannot start from square one every time we explore a statement.  We must make use of our acquired mathematical knowledge.  For the conjecture that 
$\sin\left(2x \right) = 2 \sin (x)$, for all real numbers $x$, we might recall that there are trigonometric identities called ``double angle identities.''  We may even remember the correct identity for $\sin\left( 2x \right)$, but if we do not, we can always look it up.  We should recall (or find) that
\[
\text{for all real numbers } x, \sin \! \left( 2x \right) = 2  \sin (x) \!  \cos (x).
\]
We could use this identity to argue that the conjecture ``for all real numbers $x$, 
$\sin (2x) = 2 \sin(x)$'' is false, but if we do, it is still a good idea to give a specific counterexample as we did before.

\item \textbf{Cooperation and brainstorming}.
Working together is often more fruitful than working alone.  When we work with someone else, we can compare notes and articulate our ideas.  Thinking out loud is often a useful brainstorming method that helps generate new ideas.
\end{itemize}
%\hbreak

%\begin{test} \label{test:explores} \hfill
\begin{prog}[\textbf{Explorations}]\label{pr:explores} \hfill \\
Use the techniques of exploration to investigate each of the following statements.  Can you make a conjecture as to whether the statement is true or false?  Can you determine whether it is true or false?
\begin{enumerate}
  \item $\left(a+b \right)^2=a^2+b^2$, for all real numbers  $a$  and  $b$.
  \item There are integers  $x$  and  $y$  such that $2x+5y=41$.
  \item If  $x$  is an even integer, then $x^2$  is an even integer.
  \item If $x$ and $y$ are odd integers, then $x \cdot y$ is an odd integer.
\end{enumerate}
\end{prog}
\hbreak
