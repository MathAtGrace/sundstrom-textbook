\subsection*{Some Standard Mathematical Relations}
There are many different relations in mathematics.  For example, two real numbers can be considered to be related if one number is less than the other number.  We call this the ``less than'' relation on  $\mathbb{R}$. If  $x, y \in \mathbb{R}$ and  $x$  is less than  $y$, we often write  $x < y$.  As a set of ordered pairs, this relation is  $R_{  < }$, where
\[
R_{  < }  = \left\{ { {\left( {x, y} \right) \in \mathbb{R} \times \mathbb{R} } \mid x < y} \right\}\!.
\]
With many mathematical relations, we do not write the relation as a set of ordered pairs even though, technically, it is a set of ordered pairs.  Table~\ref{Ta:standardrelations} describes some standard mathematical relations.  

\begin{table}[h]
\begin{center}
\begin{tabular}{| p{1.2in} | p{1in} | p{2.2in} |}  \hline
\textbf{Name}   &  \textbf{Open}   &  \textbf{Relation as a Set of}  \\ 
  &  \textbf{Sentence}  &  \textbf{Ordered Pairs} \\ \hline

The ``less than'' relation on  $\mathbb{R}$  &  $x < y$  &  
$\left\{ { {\left( {x, y} \right) \in \mathbb{R} \times \mathbb{R} } \mid x < y} \right\}$  \\ \hline

The ``equality'' relation on $\mathbb{R}$  &  $x=y$  & 
$\left\{ { {\left( {x, y} \right) \in \mathbb{R} \times \mathbb{R} } \mid x = y} \right\}$  \\ \hline

The ``divides'' relation on  $\mathbb{Z}$  &  $m \mid n$  &
$\left\{ { {\left( {m, n} \right) \in \mathbb{Z} \times \mathbb{Z} } \mid m\text{  divides  }n} \right\}$  \\ \hline

The ``subset'' relation on  $\mathcal{P}\left( U \right)$  &  $S \subseteq T$  & 
$\left\{ { {\left( {S, T} \right) \in \mathcal{P}\left( U \right) \times \mathcal{P}\left( U \right) } \mid S \subseteq T} \right\}$  \\ \hline

The ``element of'' relation from  $U$  to  $\mathcal{P}\left( U \right)$  &  $x \in S$  & 
$\left\{ { {\left( {x, S} \right) \in U \times \mathcal{P}\left( U \right) } \mid x \in S} \right\}$  \\ \hline

The ``congruence modulo $n$'' relation on $\mathbb{Z}$  &  $a \equiv b \pmod n$ & 
$\left\{ {\left( {a, b} \right) \in \mathbb{Z} \times \mathbb{Z}   \mid a \equiv b \pmod n} \right\}$ \\ \hline
\end{tabular}
\caption{Standard Mathematical Relations}
\label{Ta:standardrelations}
\end{center}
\end{table}

\endinput
