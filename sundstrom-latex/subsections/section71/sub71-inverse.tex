\subsection*{The Inverse of a Relation}
In Section~\ref{S:inversefunctions}, we introduced the \textbf{inverse of a function}.
\index{inverse of a function}%
\index{function!inverse of}%
 If  $A$ and $B$ are nonempty sets and if $f:A \to B$ is a function, then the inverse of  $f$, denoted by  $f^{ - 1} $, is defined as
\begin{align*}
f^{ - 1}  &= \left\{ { {\left( {b, a} \right) \in B \times A } \mid f\left( a \right) = b} \right\} \\
          &= \left\{ { {\left( {b, a} \right) \in B \times A } \mid \left( {a, b} \right) \in f} \right\}\!.
\end{align*}
%If we use the ordered pair representation for  $f$, we could also write
%\[
%f^{ - 1} = \left\{ { {\left( {b, a} \right) \in B \times A } \mid \left( {a, b} \right) \in f} \right\}.
%\]
Now that we know about relations, we see that  $f^{ - 1} $  is always a relation from  $B$  to 
$A$.  The concept of the inverse of a function is actually a special case of the more general concept of the inverse of a relation, which we now define.

\begin{defbox}{inverseofrelation}{Let  $R$  be a relation from the set  $A$  to the set  $B$.  The \textbf{inverse of}  $\boldsymbol{R}$,
\index{inverse of a relation}%
\index{relation!inverse of}%
 written  $R^{ - 1} $ 
\label{sym:Rinverse} and read  ``$R$  inverse,'' is the relation from  $B$  to  $A$  defined by
\[
\begin{aligned}
  R^{ - 1}  &= \left\{ { {\left( {y, x} \right) \in B \times A } \mid \left( {x, y} \right) \in R} \right\}\text{, or} \hfill \\
  R^{ - 1}  &= \left\{ { {\left( {y, x} \right) \in B \times A } \mid x \mathrel{R} y} \right\}\!. \\ 
\end{aligned} 
\]
That is, $R^{ - 1} $ is the subset of  $B \times A$ consisting of all ordered pairs  
$\left( {y, x} \right)$  such that  $x \mathrel{R} y$.}
\end{defbox}
%
\subsection*{An Example of an Inverse Relation}
Let $D$ be the ``divides'' relation on  $\mathbb{Z}$.  See Progress 
Check~\ref{prog:dividesrelation}.  So
\[
D = \left\{ { {\left( {m, n} \right) \in \mathbb{Z} \times \mathbb{Z} } \mid m\text{  divides  }n} \right\}\!.
\]
This means that we can write
%\begin{center}
$m \mid n$ if and only if $\left( {m, n} \right) \in D$.
%\end{center}
So, in this case,
\[
\begin{aligned}
D^{ - 1}  &= \left\{ { {\left( {n, m} \right) \in \mathbb{Z} \times \mathbb{Z} } \mid \left( {m, n} \right) \in D} \right\} \\ 
          &= \left\{ { {\left( {n, m} \right) \in \mathbb{Z} \times \mathbb{Z} } \mid m\text{  divides  }n} \right\}\!. \\ 
\end{aligned}
\]
Now, if we would like to focus on the first coordinate instead of the second coordinate in  
$D^{ - 1} $, we know that  ``$m$  divides  $n$''  means the same thing as  ``$n$  is a multiple of  $m$.''  Hence,
\[
D^{ - 1}  = \left\{ { {\left( {n, m} \right) \in \mathbb{Z} \times \mathbb{Z} } \mid n\text{  is a multiple of  }m} \right\}\!.
\]
We can say that the inverse of the ``divides'' relation on  $\mathbb{Z}$  is the ``is a multiple of'' relation on  $\mathbb{Z}$.
\hbreak
%
Theorem~\ref{T:inverserelations}, which follows, contains some elementary facts about inverse relations.  The proofs of these results are included in Exercise~(\ref{exer:provinginverse}).
%
\begin{theorem} \label{T:inverserelations}
Let  $R$  be a relation from the set  $A$  to the set  $B$.  Then

\begin{enumerate}
\item The domain of  $R^{ - 1} $ is the range of  $R$.  That is, 
$\text{dom}\!\left( {R^{ - 1} } \right) = \text{range}( R )$.  \label{T:inverserelations1}

\item The range of  $R^{ - 1} $  is the domain of  $R$.   That is, 
$\text{range}\!\left( {R^{ - 1} } \right) = \text{dom}( R )$.  \label{T:inverserelations2}

\item The inverse of  $R^{ - 1} $  is  $R$.  That is, $\left( {R^{ - 1} } \right)^{ - 1}  = R$.  \label{T:inverserelations3}

\end{enumerate}
\end{theorem}


\endinput
