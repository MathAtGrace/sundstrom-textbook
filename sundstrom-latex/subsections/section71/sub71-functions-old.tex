\subsection*{Functions as Relations}
If we have a function  $f\x A \to B$, we can generate a set of ordered pairs  $f$  that is a subset of  $A \times B$ as follows:
\[
  f = \left\{ { {\left( {a, f( a )} \right) } \mid a \in A} \right\} \quad \text{ or}  
  \quad f = \left\{ {( {a, b} ) \in A \times B   \mid b = f( a )} \right\}.  
\]
\setcounter{equation}{0}
This means that  $f$  is a relation from  $A$  to  $B$.  Since,  
$\text{dom}( f ) = A$, we know that
\begin{list}{(1)}
\item For every $a \in A$, there exists a $b \in B$ such that $( a, b ) \in f$.
\end{list}
%\begin{equation}\label{eq:funcasrelation1}
%\text{For every } a \in A, \text{ there exists a } b \in B  \text{ such that }  
%\left( {a,b} \right) \in f.
%\end{equation}
\vskip6pt
\noindent
When $\left( a, b \right) \in f$, we write  $b = f( a )$.  In addition, to be a function, each input can produce only one output.  In terms of ordered pairs, this means that there will never be two ordered pairs   
$( {a, b} )$ and  $( {a, c} )$  in the function  $f$,  where  $a \in A$, $b, c \in B$, and  $b \ne c$.  We can formulate this as a conditional statement as follows:
\begin{list}{(2)}
\item For every $a \in A$ and every $b, c \in B$, if $( a, b ) \in f$ and 
$(a, c ) \in f$, then $b = c$.
\end{list}
%\begin{align} \notag 
%& \text{For every } a \in A \text{ and every } b, c \in B, \text{ if } 
%\left( {a, b} \right) \in f  \text{ and } \left( {a, c} \right) \in f, \\
%\label{eq:funcasrelation2}%
%& \text{ then } b = c. \\ \notag
%\end{align}
\vskip6pt
\noindent
This means that a function  $f$  from  $A$  to  $B$  is a relation from  $A$  to  $B$  that satisfies conditions~(1)
%(\ref{eq:funcasrelation1})  
and~(2).  
%(\ref{eq:funcasrelation2}).  
(See Theorem~\ref{T:functionasordered} in Section~\ref{S:inversefunctions}.)  Not every relation, however, will be a function.  For example, consider the relation $T$ in Progress Check~\ref{A:relationexamples}.
\[
T = \left\{ {\left( {x, y} \right) \in \mathbb{R} \times \mathbb{R}   \mid x^2  + y^2  = 64} \right\}
\]
This relation fails condition~(2) above since a counterexample comes from the facts that  
$(0, 8) \in T$ and $(0, -8) \in T$ and $8 \ne -8$.

\begin{prog}[\textbf{A Set of Ordered Pairs}]\label{prog:setofpairs} \hfill \\
Let $F = \left\{ (x, y) \in \R \times \R \mid y = x^2 \right\}$.  The set $F$ can then be considered to be relation on $\R$ since it is a subset of $\R \times \R$.

\begin{enumerate}
\item List five different ordered pairs that are in the set $F$.

\item Use the roster method to specify the elements of each of the following the sets:
\begin{multicols}{2}
\begin{enumerate}
\item $A = \left\{ x \in \R \mid (x, 4) \in F \right\}$
\item $B = \left\{ x \in \R \mid (x, 10) \in F \right\}$
\item $C = \left\{ y \in \R \mid (5, y) \in F \right\}$
\item $D = \left\{ y \in \R \mid (-3, y) \in F \right\}$
\end{enumerate}
\end{multicols}

\item Can the set  $F$  be used to define a function from the set  $\R$  to the set  $\R^*$?  Explain.
\end{enumerate}
\end{prog}

\endinput
