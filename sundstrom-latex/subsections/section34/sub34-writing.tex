\subsection*{Writing Guidelines for a Proof Using Cases}
When writing a proof that uses cases, we use all the other writing guidelines.  In addition, we make sure that it is clear where each case begins.  This can be done by using a new paragraph with a label such as ``Case 1,'' or it can be done by starting a paragraph with a phrase such as, ``In the case where \ldots.''  %See the solutions for Activities~\ref{PA:logicalequiv} and~\ref{PA:cases} for examples of proofs that use cases.

\begin{prog}[\textbf{Using Cases: $\boldsymbol{n}$ Is Even or $\boldsymbol{n}$ Is Odd}] 
\label{pr:n-even-odd} \hfill \\
Complete the proof of the following proposition:

\vskip6pt
\noindent
\textbf{Proposition}. \emph{For each integer $n$, $n^2 - 5n + 7$ is an odd integer}.

\noindent
\textit{\textbf{Proof}}.  Let $n$ be an integer.  We will prove that $n^2 - 5n + 7$ is an odd integer by examining the case where $n$ is even and the case where $n$ is odd.


\newpar
\textit{Case 1}.  The integer $n$ is even.  In this case, there exists an integer $m$ such that $n = 2m$.  
Therefore, \ldots .
\end{prog}
\hbreak

\newpar
As another example of using cases, consider a situation where we know that $a$ and $b$ are real numbers and 
$ab = 0$.  If we want to make a conclusion about $b$, the temptation might be to divide both sides of the equation by $a$.  However, we can only do this if $a \ne 0$.  So, we consider two cases:  one when $a = 0$ and the other when $a \ne 0$.


%\begin{prog}[\textbf{Using Cases: $\boldsymbol{a = 0}$ or $\boldsymbol{a \ne 0}$}]\label{pr:cases2} \hfill \\
%Complete the proof of the following proposition:
%
%\vskip6pt
%\noindent
%\textbf{Proposition}.  \emph{For all real numbers $a$ and $b$, if $ab = 0$, then $a = 0$ or $b = 0$.}

\begin{proposition}
For all real numbers $a$ and $b$, if $ab = 0$, then $a = 0$ or $b = 0$. \label{prop:zeroproperty}
\end{proposition}

\begin{myproof}
We let $a$ and $b$ be real numbers and assume that $ab = 0$.  We will prove that $a = 0$ or 
$b = 0$ by considering two cases:  (1) $a = 0$, and (2) $a \ne 0$.

In the case where $a = 0$, the conclusion of the proposition is true and so there is nothing to prove.  

In the case where $a \ne 0$, we can multiply both sides of the equation $ab = 0$ by $\dfrac{1}{a}$ and obtain
\[
\begin{aligned}
\frac{1}{a} \cdot ab &= \frac{1}{a} \cdot 0 \\
                    b & = 0.
\end{aligned}
\]
So in both cases, $a = 0$ or $b = 0$, and this proves that for all real numbers $a$ and $b$, if $ab = 0$, then $a = 0$ or $b = 0$.
\end{myproof}
%\vskip6pt
%\noindent
%\textbf{\textit{Proof}}.
%We let $a$ and $b$ be real numbers and assume that $ab = 0$.  We will prove that $a = 0$ or 
%$b = 0$ by considering two cases:  (1) $a = 0$, and (2) $a \ne 0$.
%
%In the case where $a = 0$, \ldots .  
%
%\end{prog}
\hbreak

\endinput
