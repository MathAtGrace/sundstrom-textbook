\subsection*{Further Remarks about Conditional Statements}
\begin{enumerate}
\item The conventions for the truth value of conditional statements may seem a bit strange,especially the fact that the conditional statement is true when the hypothesis of the conditional statement is false.  The following example is meant to show that this makes sense.

Suppose that Ed has exactly \$52 in his wallet.  The following four statements will use the four possible truth combinations for the hypothesis and conclusion of a conditional statement.
\begin{itemize}
  \item If Ed has exactly \$52 in his wallet, then he has \$20 in his wallet.  This is a true statement.  Notice that both the hypothesis and the conclusion are true.
  \item If Ed has exactly \$52 in his wallet, then he has \$100 in his wallet.  This statement is false.  Notice that the hypothesis is true and the conclusion is false.
  \item If Ed has \$100 in his wallet, then he has at least  \$50 in his wallet.  This statement is true regardless of how much money he has in his wallet.  In this case, the hypothesis is false and the conclusion is true.
  \item If Ed has \$100 in his wallet, then he has at least  \$80 in his wallet.  This statement is true regardless of how much money he has in his wallet.  In this case, the hypothesis is false and the conclusion is false.
\end{itemize}
This is admittedly a contrived example but it does illustrate that the conventions for the truth value of a conditional statement make sense.  The message is that in order to be complete in mathematics, we need to have conventions about when a conditional statement is true and when it is false.  %The key thing to remember is that when a conditional statement is true, we may only say that the conclusion is true when the hypothesis is true.

%\item The conventions for conditional statements allow us to tell when a conditional statement is true based on the truth values of the hypothesis and conclusion.  For example, using these conventions, we see that: %statements from Part~(\ref{PA:conditional2}) in Preview Activity~\ref{PA:conditional}, we now say that
%\begin{itemize}
%\item If $3 + 2 = 5$, then $5 <8$ is a true conditional statement since both the hypothesis and the conclusion are true.
%\item If $3 + 2 = 5$, then $8 <5$ is a false conditional statement since the hypothesis is true and the conclusion is false.
%\item If $8 < 5$, then $3 + 2 = 5$ is a true conditional statement since the hypothesis is false and the conclusion is true.
%\item If $8 < 5$, then $3 + 2 = 9$ is a true conditional statement since the hypothesis is false and the conclusion is false.
%\end{itemize}
%
%The last two conclusions in the preceding list may seem a bit strange because they are not statements that we would use in mathematics or everyday life.  However, to be complete in mathematics, we need to have conventions about when a conditional statement is true and when it is false.  The key thing to remember is that when a conditional statement is true, we may only say that the conclusion is true when the hypothesis is true.

\item The fact that there is only one case when a conditional statement is false often provides a method to show that a given conditional statement is false.  In Progress Check~\ref{prog:condition}, you were asked if you thought the following conditional statement was true or false.
\begin{center}
If  $n$  is a positive integer, then  $\left( {n^2  - n + 41} \right)$ is a prime number.
\end{center}
Perhaps for all of the values you tried for  $n$, $\left( {n^2  - n + 41} \right)$ turned out to be a prime number.  However, if we try  $n = 41$, we get
\begin{align}
  n^2  - n + 41 &= 41^2  - 41 + 41 \notag \\ 
  n^2  - n + 41 &= 41^2 . \notag 
\end{align}
So in the case where  $n = 41$, the hypothesis is true  (41 is a positive integer) and the conclusion is false $\left( {41^2 \text{ is not prime}} \right)$.  Therefore, 41 is a counterexample for this conjecture and the conditional statement
\begin{center}
``If  $n$  is a positive integer, then  $\left( n^2  - n + 41 \right)$ is a prime number''
\end{center}
is false.  There are other counterexamples (such as $n = 42$, $n = 45$, and  $n = 50$), but only one counterexample is needed to prove that the statement is false.

\item Although one example can be used to prove that a conditional statement is false, in most cases, we cannot use examples to prove that a conditional statement is true.  For example, in Progress Check~\ref{prog:condition}, we substituted values for $x$ for the conditional statement ``If $x$ is a positive real number, then
\linebreak
$x^2 + 8x$ is a positive real number.''  For every positive real number used for $x$, we saw that $x^2 + 8x$ was positive.  However, this does not prove the conditional statement to be true because it is impossible to substitute every positive real number for $x$.  So, although we may believe this statement is true, to be able to conclude it is true, we need to write a mathematical proof.  Methods of proof will be discussed in Section~\ref{S:direct}  and Chapter~\ref{C:proofs}.
\end{enumerate}
\hbreak


\vskip6pt
%\begin{test} \hfill \label{test:conditional}
\begin{prog}[\textbf{Working with a Conditional Statement}]\label{pr:conditional} \hfill \\
The following statement is a true statement, which is proven in many calculus texts.
\begin{center}
If the function  $f$  is differentiable at  $a$, then  the function   $f$  is continuous at  $a$.
\end{center}
\noindent
Using only this true statement, is it possible to make a conclusion about the function in each of the following cases?

\begin{enumerate}
\item It is known that the function  $f$, where   $f(x) = \sin x$, is differentiable at  0.
\item It is known that the function  $f$, where  $f(x) = \sqrt[3]{x}$, is not differentiable at  0.
\item It is known that the function  $f$, where   $f(x) = \left| x \right|$, is continuous at  0.
\item It is known that the function  $f$, where  $f(x) = \dfrac{{\left| x \right|}}{x}$ is not continuous at  0.
\end{enumerate}
\end{prog}
\index{conditional statement|)}%
\index{statement!conditional|)}%
\hbreak


\endinput
