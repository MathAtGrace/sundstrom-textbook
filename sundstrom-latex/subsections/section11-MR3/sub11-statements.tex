%\subsection*{Statements}
Much of our work in mathematics deals with statements.  In mathematics, a \textbf{statement}
\label{D:prop}%
\index{statement}%
  is a declarative sentence that is either true or false but not both.    
A statement is sometimes called a \textbf{proposition}.
\index{proposition}%
The key is that there must be no ambiguity.  To be a statement, a sentence must be  true or false, and it cannot be both.  So a sentence such as ``The sky is beautiful'' is not a statement since whether the sentence is true or not is a matter of opinion.  A question such as ``Is it raining?'' is not a statement because it is a question and is not declaring or asserting that something is true.

%\begin{prog}[\textbf{Statements}] \label{prog:statements} \hfill \\
%Determine which of the following sentences are statements?  Do not worry about determining whether a statement is true or false; just determine whether each sentence is a statement or not.
%\begin{multicols}{2}
%\begin{enumerate}
%\item $3 \cdot 4 + 7 = 19$.
%\item $3 \cdot 5 + 7 = 19$.
%\item $3x + 7 = 19$.
%\end{enumerate}
%\end{multicols}
%\begin{enumerate} \setcounter{enumi}{3}
%\item The derivative of $f(x) = \sin x$ is $f'(x) = \cos x$.
%\item Does the equation $3x^2 - 5x - 7 = 0$ have two real number solutions?
%\end{enumerate}
%\end{prog}
%\hbreak

Some sentences that are mathematical in nature often are not statements because we may not know precisely what a variable represents.  For example, the equation $2x + 5 = 10$ is not a statement since we do not know what $x$ represents.  If we substitute a specific value for $x$ (such as $x = 3$), then the resulting equation, $2 \cdot 3 + 5 = 10$ is a statement (which is a false statement).  Following are some more examples:
\begin{itemize}
  \item There exists a real number $x$ such that $2x + 5 = 10$. \\
This is a statement because either such a real number exists or such a real number does not exist.  In this case, this is a true statement since such a real number does exist, namely $x = 2.5$.
  \item For each real number $x$, $2x + 5 = 2\left(x + \dfrac{5}{2} \right)$.\\
This is a statement since either the sentence $2x + 5 = 2\left(x + \dfrac{5}{2} \right)$ is true when any real number is substituted for $x$ (in which case, the statement is true) or there is at least one real number that can be substituted for $x$ and produce a false statement (in which case, the statement is false).  In this case, the given statement is true.
  \item Solve the equation $x^2 - 7x + 10 = 0$. \\
This is not a statement since it is a directive.  It does not assert that something is true.
  \item $(a + b)^2 = a^2 + b^2$ is not a statement since it is not known what $a$ and $b$ represent.  However, the sentence, ``There exist real numbers $a$ and $b$ such that $(a + b)^2 = a^2 + b^2$'' is a statement.  In fact, this is a true statement since there are such integers.  For example, if $a = 1$ and $b = 0$, then 
$(a + b)^2 = a^2 + b^2$.
  \item Compare the statement in the previous item to the statement, ``For all real numbers $a$ and $b$, 
$(a + b)^2 = a^2 + b^2$.''  This is a false statement since there are values for $a$ and $b$ for which 
$(a + b)^2 \ne a^2 + b^2$.  For example, if $a = 2$ and $b = 3$, then $(a + b)^2 = 5^2 = 25$ and 
$a^2 + b^2 = 2^2 + 3^2 = 13$.
\end{itemize}


\begin{prog}[\textbf{Statements}] \label{prog:statements2} \hfill \\
Which of the following sentences are statements?  Do not worry about determining whether a statement is true or false; just determine whether each sentence is a statement or not.

\begin{multicols}{2}
\begin{enumerate}
\item $3 + 4 = 8$.

\item $2 \cdot 7 + 8 = 22$.

\item	$\left( {x - 1} \right) = \sqrt {x + 11}$.

\item $2x + 5y = 7$.\label{PA:prop3}
\end{enumerate}
\end{multicols}
\begin{enumerate} \setcounter{enumi}{4}
\item There are integers  $x$  and  $y$  such that $2x + 5y = 7.$\label{PA:prop4}

\item There are integers  $x$  and  $y$  such that $23x + 37y = 52.$

%\item If  $x$  and  $y$  are odd integers, then $x \cdot y$ is an odd integer.

\item Given a line  $L$  and a point  $P$  not on that line, there is a unique line through  $P$  that does not intersect  $L$.

\item $\left( {a + b} \right)^3  = a^3 + 3a^2b + 3ab^2  + b^3.$\label{PA:prop8}

\item $\left( {a + b} \right)^3  = a^3  + 3a^2b + 3ab^2 + b^3$ for all real numbers  $a$  and  $b$.
\label{PA:prop9}

\item The derivative of the sine function is the cosine function.  %$f(x) = \sin x$ is $f'(x) = \cos x$.

\item Does the equation $3x^2 - 5x - 7 = 0$ have two real number solutions?

\item If $ABC$ is a right triangle with right angle at vertex $B$, and if $D$ is the midpoint of the hypotenuse, then the line segment connecting vertex $B$ to $D$ is half the length of the hypotenuse.

%\item If you pick  $N$  distinct points on the circumference of a circle and draw line segments connecting them all with each other, then the interior of the circle will be divided into  
%$2^{N - 1}$ portions.

\item There do not exist three integers  $x$, $y$, and  $z$ such that 
 $x^3  + y^3  = z^3.$
\end{enumerate}
\end{prog}
\hbreak

\endinput
