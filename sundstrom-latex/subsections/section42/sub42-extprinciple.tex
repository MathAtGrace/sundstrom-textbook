\subsection*{Using the Extended Principle of Mathematical Induction}
The primary use of the Principle of Mathematical Induction is to prove statements of the form
\[
\left( {\forall n \in \mathbb{Z},\text{ with } n \geq M} \right)\left( {P( n )} \right)\!,
\]
where  $M$  is an integer and  $P( n )$ is some open sentence. (In most induction proofs, we will use a value of $M$ that is greater than or equal to zero.)  So our goal is to prove that the truth set  $T$  of the predicate   $P( n )$ contains all integers greater than or equal to  $M$\!.  So to verify the hypothesis of the Extended Principle of Mathematical Induction, we must
\begin{enumerate}
\item Prove that  $M \in T$\!.  That is, prove that  $P( M )$ is true.

\item Prove that for every  $k \in \mathbb{Z}$ with $k \geq M$\!, if  $k \in T$\!, then  
$\left( {k + 1} \right) \in T$\!.  That is, prove that if  $P( k )$ is true, then  
$P( {k + 1} )$ is true.

\end{enumerate}

As before, the first step is called the \textbf{basis step}
\index{basis step}%
\index{mathematical induction!basis step}%
 or the \textbf{initial step}, and the second step is called the \textbf{inductive step}.
\index{inductive step}%
\index{mathematical induction!inductive step}%
  This means that a proof using the Extended Principle of Mathematical Induction will have the following form:
%
\begin{center}
\fbox{\parbox{4.68in}{\textbf{Using the Extended Principle of Mathematical Induction} \\
Let  $M$  be an integer.  To prove:  $\left( {\forall n \in \mathbb{Z} \text{ with } n \geq M} \right)\left( {P( n )} \right)$

\begin{tabular}{r l}
                &                              \\
Basis step:    &  Prove  $P( M )$. \\
                &                              \\
Inductive step: &	 Prove that for every  $k \in \mathbb{Z} $ with $k \geq M$\!,  \\
                &  if  $P( k )$ is true, then  $P( {k + 1} )$ is true. \\
\end{tabular}
\vskip10pt
We can then conclude that  $P( n )$ is true for all  $n \in \mathbb{Z}$ with  
$n \geq M$\!.}}

\end{center}
%
This is basically the same procedure as the one for using the Principle of Mathematical Induction.  The only difference is that the basis step uses an integer  $M$  other than  1.  For this reason, when we write a proof that uses the Extended Principle of Mathematical Induction, we often simply say we are going to use a proof by mathematical induction.  We will use the work from \typeu Activity~\ref*{PA:factorials} to illustrate such a proof.
%\hbreak
%
\setcounter{equation}{0}
\begin{proposition} \label{P:factorialinduction}
For each natural number  $n$  with  $n \geq 4$,  $n! > 2^n $.
\end{proposition}
%
\begin{myproof}
We will use a proof by mathematical induction.  For this proof, we let
\begin{center}
  $P( n )$ be ``$n! > 2^n$.''
\end{center}
%
We first prove that $P( 4 )$ is true.  Using  $n = 4$, we see that  $4! = 24$ and 
$2^4  = 16$.  This means that  $4! > 2^4 $ and, hence,  $P\left( 4 \right)$  is true.
%\vskip10pt
%\noindent

For the inductive step, we prove that for all $k \in \mathbb{N}$ with $k \geq 4$, if 
$P( k )$ is true, then $P( k + 1 )$ is true.  So let  $k$  be a natural number greater than or equal to 4, and assume that  $P( k )$  is true.  That is, assume that
\begin{equation} \label{eq:5h}
k! > 2^k. 
\end{equation}
The goal is to prove that  $P( {k + 1} )$ is true or that  $\left( {k + 1} \right)! > 2^{k + 1}$.  Multiplying both sides of inequality~(\ref{eq:5h}) by $k + 1$ gives
\begin{align} \notag
  \left( k + 1 \right) \cdot k! &> \left( k + 1 \right) \cdot 2^k ,\text{  or} \\ \label{eq:5i}
  \left( k + 1 \right)! &> \left( k + 1 \right) \cdot 2^k .  \\ \notag
\end{align}
Now, $k \geq 4$.  Thus, $k + 1 > 2$, and hence $\left( k + 1 \right) \cdot 2^k > 2 \cdot 2^k$.  This means that
\begin{equation} \label{eq:5j}
\left( k + 1 \right) \cdot 2^k > 2^{k + 1}.
\end{equation}
Inequalities~(\ref{eq:5i}) and~(\ref{eq:5j}) show that
\[
(k + 1)! > 2^{k + 1}, 
\]
%\enlargethispage{\baselineskip}
and this proves that if $P(k)$ is true, then  $P( {k + 1} )$ is true.  Thus, the inductive step has been established, and so by the Extended Principle of Mathematical Induction, $n! > 2^n $ for each natural number  
$n$   with  $n \geq 4$.
\end{myproof}
\hbreak
%

\begin{prog}[\textbf{Formulating Conjectures}] \label{prog:extendedind} \hfill \\
Formulate a conjecture (with an appropriate quantifier) that can be used as an answer to each of the following questions.
\begin{enumerate}
\item For which natural numbers $n$ is $3^n$ greater than $1 + 2^n$?
\item For which natural numbers $n$ is $2^n$ greater than $\left(n + 1 \right)^2$?
\item For which natural numbers $n$ is $\left( 1 + \dfrac{1}{n} \right)^n$ greater than 2.5?
\end{enumerate}
\end{prog}
\hbreak


%In Beginning Activity~\ref{PA:subsetsofaset4}, we observed that a set with one element has two subsets, a set with two elements has four subsets, and a set with three elements has eight subsets.  In this Beginning Activity, we also studied a way to use the eight subsets of a set with three elements to create the 16 subsets of a set with four elements.  This work suggests that the following proposition is true.
%
%\begin{proposition} \label{P:inductivestepforsubsets}
%Let $A$ and $B$ be subsets of some universal set.  If  $A = B \cup \left\{ x \right\}$, where  $x \notin B$, then all the subsets of  $A$  are either subsets of  $B$  or of the form   
%$C \cup \left\{ x \right\}$, where  $C$  is a subset of  $B$.
%\end{proposition}
%%
%\begin{myproof}
%Let $A$ and $B$ be subsets of some universal set, and assume that  $A = B \cup \left\{ x \right\}$ where  $x \notin B$.  Let  $Y$  be a subset of  $A$.  We need to show that  $Y$  is a subset of  $B$  or that   $Y = C \cup \left\{ x \right\}$ where  $C$   is some subset of  $B$.  There are two cases to consider:  (1)  $x$  is not an element of  $Y$, and (2)  $x$  is an element of  $Y$.
%\vskip10pt
%\noindent
%\underline{Case 1}:   Assume that  $x \notin Y$.  Let  $y \in Y$.  Then  $y \in A$  and  
%$y \ne x$.  Since  
%\[
%A = B \cup \left\{ x \right\},
%\]
%this means that  $y$  must be in  $B$.  Therefore,  $Y \subseteq B$.
%\vskip10pt
%\noindent
%\underline{Case 2}:  Assume that  $x \in Y$. In this case, let  $C = Y - \left\{ x \right\}$.  Then every element of  $C$  is an element of  $B$. Hence, we can conclude that  $C \subseteq B$  and that  $Y = C \cup \left\{ x \right\}$.
%\vskip10pt
%\noindent
%Cases 1 and 2 show that if  $Y \subseteq A$, then  $Y \subseteq B$  or  
%$Y = C \cup \left\{ x \right\}$,  where  $C \subseteq B$.
%\end{myproof}
%\hbreak
%%
%The power set of  $A$, $\mathcal{P}\left( A \right)$, is the set of all subsets of  $A$.  So, 
%\[
%Y \in \mathcal{P}\left( A \right)\text{ if and only if }Y \subseteq A.
%\]
%Proposition~\ref{P:inductivestepforsubsets} states that  if $A = B \cup \left\{ x \right\}$ and  $x \notin B$, then $Y \subseteq A$ if and only if  $Y \subseteq B$ or that 
%$Y = C \cup \left\{ x \right\}$, where  $C$   is some subset of  $B$.
%
%Using power sets and still assuming that $A = B \cup \left\{ x \right\}$ and  $x \notin B$, this means that  $Y \in \mathcal{P}\left( A \right)$ if and only if  
%$Y \in \mathcal{P}\left( B \right)$ or that   $Y = C \cup \left\{ x \right\}$, where  
%$C \in \mathcal{P}\left( B \right)$.  This gives us the following corollary of Proposition~\ref{P:inductivestepforsubsets}.
%%
%\begin{corollary} \label{C:inductivestepforsubsets}
%Let $A$ and $B$ be subsets of some universal set. If  $x \notin B$ and 
%$A = B \cup \left\{ x \right\}$, then  $\mathcal{P}(A) = \mathcal{P}(B) \cup \left\{ {C \cup \left\{ x \right\} \left| {C \in \mathcal{P}(B)} \right.} \right\}$.
%\end{corollary}
%\hbreak
%
%\begin{activity}[The Cardinality of a Power Set] \label{A:powersetcardinality}
%\index{power set}%
%\index{power set!cardinality}%
% \hfill
%
%In this activity, we will use mathematical induction to prove the following proposition.
%
%\begin{proposition} \label{P:powersetcardinality}
%Let  $n$  be a nonnegative integer and let  $A$  be a subset of some universal set.  If  $A$  is a finite set with  $n$  elements, then  $A$  has  $2^n $ subsets.  That is,  if  $\left| A \right| = n$, then  $\left| {\mathcal{P}\left( A \right)} \right| = 2^n $.
%\end{proposition}
%\begin{enumerate}
%\item Verify that Proposition~\ref{P:powersetcardinality} is true when  $n = 0$.  That is, it is true when  $A = \emptyset $.  (This is the basis step for the induction proof.)
%
%\item Verify that Proposition~\ref{P:powersetcardinality} is true when  $n = 1$  and  when  $n = 2$.
%
%\item Now assume that  $k$  is a nonnegative integer and assume that if a set has  $k$  elements, then that set has  $2^k $  subsets.  (This is the inductive assumption for the induction proof.)
%
%Let  $A$  be a subset of the universal set with  $\left| A \right| = k + 1$, and let  $x \in A$.  Then, the set  $B = A - \left\{ x \right\}$ has  $k$  elements.
%
%Now use the inductive assumption to determine how many subsets  $B$  has.  Then, use  Proposition~\ref{P:inductivestepforsubsets} to prove that  $A$  has twice as many subsets as  $B$.  This should help complete the inductive step for the induction proof.
%
%\end{enumerate}
%\end{activity}
%\hbreak

\endinput
