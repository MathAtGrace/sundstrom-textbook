\subsection*{Inverse Function Notation}
In the situation where  $f\x A \to B$  is a bijection and  $f^{ - 1} $ is a function from  $B$  to  $A$, we can write  $f^{ - 1} \x B \to A$.  In this case, we frequently say that  $f$  is an \textbf{invertible function},
\index{invertible function}%
\index{function!invertible}%
  and we usually do not use the ordered pair representation for either  $f$  or  $f^{ - 1} $.  Instead of writing  
$( {a, b} ) \in f$, we write  $f( a ) = b$, and instead of writing  $( {b, a} ) \in f^{ - 1} $, we write  $f^{ - 1} ( b ) = a$.  Using the fact that  $( {a, b} ) \in f$  if and only if  $( {b, a} ) \in f^{ - 1} $, we can now write  $f( a ) = b$  if and only if  $f^{ - 1} ( b ) = a$.  We summarize this in Theorem~\ref{T:inversenotation}.

\begin{theorem}  \label{T:inversenotation}
Let  $A$  and  $B$  be nonempty sets and let  $f\x A \to B$  be a bijection.  Then 
$f^{ - 1} \x B \to A$ is a function, and for every  $a \in A$ and $b \in B$,
\[
f( a ) = b  \text{ if and only if } f^{ - 1} ( b ) = a.
\]
\end{theorem}
%
%\hbreak
%
\begin{example}[\textbf{Inverse Function Notation}] \hfill \\
\label{exam:inversenotation}%
For an example of the use of the notation in Theorem~\ref{T:inversenotation}, let  $\R^ +   = \left\{ { {x \in \R} \mid x > 0} \right\}$.  Define

\begin{center}
$f\x \R \to \R$  by  $f( x ) = x^3$; and $g\x \R \to \R^ +  $  by  $g( x ) = e^x $.
\end{center}
\vskip10pt
Notice that  $\R^+ $ is the codomain of  $g$.  We can then say that both  $f$  and  $g$  are bijections.  Consequently, the inverses of these functions are also functions.  In fact,
\begin{center}
$f^{ - 1} \x \R \to \R$  by  $f^{ - 1} ( y ) = \sqrt[3]{y}$; and $g^{ - 1} \x \R^ +   \to \R$  by  $g^{ - 1} ( y ) = \ln y$.
\end{center}
%\begin{list}{}
%\item $f^{ - 1} \x \R \to \R$  by  $f^{ - 1} ( y ) = \sqrt[3]{y}$; and
%
%
%\item  
%
%\item $g^{ - 1} \x \R^ +   \to \R$  by  $g^{ - 1} ( y ) = \ln y$.
%
%\end{list}
\vskip10pt
For each function (and its inverse), we can write the result of Theorem~\ref{T:inversenotation} as follows:
\begin{center}
\begin{tabular}{l | l}
\textbf{Theorem~\ref{T:inversenotation}}   &  \textbf{Translates to:}  \\ \hline
For  $x, y \in \R$,  $f( x ) = y$  &  For  $x, y \in \R$,  $x^3  = y$ \\
if and only if  $f^{ - 1} ( y ) = x$.  &  if and only if $\sqrt[3]{y} = x$.  \\
  &  \\
For  $x \in \R, y \in \R^ +  $,  $g( x ) = y$  &  For  
$x \in \R, y \in \R^ +  $,  $e^x  = y$ \\
if and only if  $g^{ - 1} ( y ) = x$.  &  if and only if  $\ln y = x$.  \\
\end{tabular}
\end{center}
\end{example}
%
\hbreak

\endinput
