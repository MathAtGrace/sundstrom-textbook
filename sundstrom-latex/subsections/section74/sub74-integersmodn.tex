\subsection*{The Integers Modulo  $\boldsymbol{n}$}
Let  $n \in \mathbb{N}$.  Since the relation of congruence modulo  $n$  is an equivalence relation on  $\mathbb{Z}$, we can discuss its equivalence classes.  Recall that in this situation, we refer to the equivalence classes as congruence classes. 

\begin{defbox}{integersmodn}{Let  $n \in \mathbb{N}$.  The set of congruence classes for the relation of congruence modulo  $n$  on  $\mathbb{Z}$ is the set of \textbf{integers modulo}  $\boldsymbol{n}$,
\index{integers modulo $n$}%
 or the set of integers mod  $n$.  We will denote this set of congruence classes by  $\mathbb{Z}_{n} $.} \label{sym:integersmodn}
\end{defbox}
%
\noindent
Corollary~\ref{C:propsofcongclasses2} tells us that  
\[
\mathbb{Z} = [ 0 ] \cup [ 1 ] \cup [ 2 ] \cup   \cdots  \cup  [ {n - 1} ].
\]
In addition, we know that each integer is congruent to precisely one of the integers  
$0, 1, 2,  \ldots , n - 1$.  This tells us that one way to represent  $\mathbb{Z}_{n} $ is
\[
\mathbb{Z}_{n}  = \left\{ {[ 0 ], [ 1 ], [ 2 ],  \ldots, [ {n - 1} ]} \right\}.
\]
Consequently, even though each integer has a congruence class, the set  $\mathbb{Z}_{n} $  has  
only $n$  distinct congruence classes.  


The set of integers  $\Z$ is more than a set.  We can add and multiply integers.  That 
is, there are the arithmetic operations of addition and multiplication on the set  
$\Z$, and we know that  $\Z$  is closed with respect to these two operations.  

One of the basic problems dealt with in modern algebra is to determine if the arithmetic operations on one set ``transfer'' to a related set.  In this case, the related set is  $\mathbb{Z}_{n} $.  For example, in the integers modulo  5, $\Z_5$, is it possible to add the congruence classes  $[ 4 ]$  and  $[ 2 ]$ as follows?
\begin{align*}
  [ 4 ] \oplus [ 2 ] &= [ {4 + 2} ] \\ 
                     &= [ 6 ] \\ 
                     &= [ 1 ]. \\ 
\end{align*}
We have used the symbol  $ \oplus $ to denote addition in  $\mathbb{Z}_{5} $ so that we do not confuse it with addition in  $\mathbb{Z}$.  This looks simple enough, but there is a problem.  The congruence classes  $[ 4 ]$  and  $[ 2 ]$ are not numbers, \emph{they are infinite sets}.  We have to make sure that we get the same answer no matter what element of  $[ 4 ]$ we use and no matter what element of  $[ 2 ]$  we use.  
For example, 
\begin{align*}
9 &\equiv 4 \pmod 5 \quad \text{and so} \quad [ 9 ] = [ 4 ].  \text{  Also,} \\ 
7 &\equiv 2 \pmod 5 \quad \text{and so} \quad [ 7 ] = [ 2 ]. 
\end{align*} 
 
\noindent
Do we get the same result if we add  $[ 9 ]$  and  $[ 7 ]$ in the way we did when we added $[ 4 ]$  and  $[ 2 ]$?  The following computation confirms that we do:
\begin{align*}
  [ 9 ] \oplus [ 7 ] &= [ {9 + 7} ] \\ 
                     &= [ {16} ] \\ 
                     &= [ 1 ]. \\ 
\end{align*}
This is one of the ideas that was explored in \typeu Activity~\ref*{PA:congruencemod6}.  The main difference is that in this activity, we used the relation of congruence, and here we are using congruence classes.  All of the examples in \typeu Activity~\ref*{PA:congruencemod6} should have illustrated the properties of congruence modulo 6 in the following table.  The left side shows the properties in terms of the congruence relation and the right side shows the properties in terms of the congruence classes.

\begin{center}
\begin{tabular}{| p{2.7in} | p{1.8in} |} \hline
If  $a \equiv 3 \pmod 6$ and  $b \equiv 4 \pmod 6$, then
\begin{itemize}
  \item $\left( {a + b} \right) \equiv \left( {3 + 4} \right) \pmod 6$;
  \item $\left( {a \cdot b} \right) \equiv \left( {3 \cdot 4} \right) \pmod 6$.
\end{itemize}
&
If  $[ a ] = [ 3 ]$  and  $[ b ] = [ 4 ]$ in $\Z_6$, then
\begin{itemize}
  \item $[ {a + b} ] = [ {3 + 4} ]$;
  \item $[ {a \cdot b} ] = [ {3 \cdot 4} ]$.
\end{itemize}  \\ \hline

\end{tabular}
\end{center}
%
These are illustrations of general properties that we have already proved in Theorem~\ref{T:propsofcong}.  We repeat the statement of the theorem here because it is so important for defining the operations of addition and multiplication in 
$\mathbb{Z}_{n}$.
\hbreak
%\pagebreak

\noindent
\textbf{Theorem~\ref{T:propsofcong}}
\emph{Let  $n$  be a natural number and let  $a, b, c, \text{and }  d$  be integers.  Then}

\begin{enumerate}
  \item \emph{If  $a \equiv b \pmod n$ and  $c \equiv d \pmod n$, then  $\left( {a + c} \right) \equiv \left( {b + d} \right) \pmod n$.}  

  \item \emph{If  $a \equiv b \pmod n$ and  $c \equiv d \pmod n$, then  
$ac \equiv bd \pmod n$.}  

 \item \emph{If  $a \equiv b \pmod n$ and  $m \in \mathbb{N}$, then  
$a^m \equiv b^m \pmod n$.}
\end{enumerate}

\noindent
Since  $x \equiv y \pmod n$ if and only if  
$[ x ] = [ y ]$, we can restate the result of this Theorem~\ref{T:propsofcong} in terms of congruence classes in  $\mathbb{Z}_n $.

\begin{corollary}\label{C:propsofcong-76}
Let  $n$  be a natural number and let $a, b, c,$ and $d$ be integers. Then, in  $\mathbb{Z}_n $,
\begin{enumerate}
\item If  $[ a ] = [ b ]$ and  $[ c ] = [ d ]$, then              $[ {a + c} ] = [ {b + d} ]$.

\item If  $[ a ] = [ b ]$ and  $[ c ] = [ d ]$, then              $[ {a \cdot c} ] = [ {b \cdot d} ]$.

\item If  $[ a ] = [ b ]$ and $m \in \mathbb{N}$, then
      $[ a ^m] = [ b ^m]$.
\end{enumerate}
\end{corollary}
\hbreak
%
Because of Corollary~\ref{C:propsofcong-76}, we know that the following formal definition of addition and multiplication of congruence classes in  $\mathbb{Z}_{n} $  is independent of the choice of the elements we choose from each class.  We say that these definitions of addition and multiplication are \textbf{well defined}.

\begin{defbox}{modulararithmetic}{Let  $n \in \mathbb{N}$.  \textbf{Addition and multiplication}
\index{integers modulo $n$!addition}%
\index{integers modulo $n$!multiplication}%
 in  $\mathbb{Z}_n $ are defined as follows:  For  $[ a ], [ c ] \in \mathbb{Z}_n $, \label{sym:addmodn}
\[
[ a ] \oplus [ c ] = [ {a + c} ] \text{  and  }[ a ] \odot [ c ] = [ {ac} ].
\]
The term \textbf{modular arithmetic}
\index{modular arithmetic}%
 is used to refer to the operations of addition and multiplication of congruence classes in the integers modulo~$n$.}
\end{defbox}

%\begin{activity}[Something that Is Not Well Defined] \label{A:notwelldefined}
%Define the following subsets of  $\mathbb{Z}$:
%\[
%\begin{aligned}
%  A &= \left\{ { \ldots ,  - 9,  - 5,  - 1, 0, 4, 8,  \ldots } \right\}, \hfill \\
%  B &= \left\{ { \ldots ,  - 12,  - 8,  - 4, 1, 5, 9 \ldots } \right\}, \hfill \\
%  C &= \left\{ { \ldots  - 11,  - 7,  - 3, 3, 7, 11,  \ldots } \right\}, \hfill \\
%  D &= \left\{ { \ldots  - 10,  - 6,  - 2, 2, 6, 10,  \ldots } \right\}. \hfill \\ 
%\end{aligned}
%\]
%\begin{enumerate}
%\item Explain why  $\mathcal{U} = \left\{ {A, B, C, D} \right\}$ is a partition of        $\mathbb{Z}$.  We will call  $A$, $B$, $C$, and  $D$  the classes of this partition.
%
%\item Find an example of integers  $a$  and  $b$  in  $A$  and  $c$  and  $d$  in  $B$ such that                  $a + c$  and  $b + d$  are in different classes of the partition  $\mathcal{U}$.        \label{A:notwelldefined2}
%
%\item Use Part~(\ref{A:notwelldefined2}) to explain why it is not possible to define  
%      $A \oplus B$ as the class containing  $a + c$ if  $a \in A$  and  $c \in B$.
%
%\item Find an example of integers  $a$  and  $b$  in  $A$  and  $c$  and  $d$  in  $B$ such that              $a \cdot c$  and  $b \cdot d$ are in different sets contained in the partition                    $\mathcal{U}$.  \label{A:notwelldefined4}
%
%\item Use Part~(\ref{A:notwelldefined4}) to explain why it is not possible to define  
%      $A \odot B$ as the class containing  $a \cdot c$ if  $a \in A$  and  $c \in B$.
%\end{enumerate}
%\hbreak
%\end{activity}
%
So if $n \in \mathbb{N}$, then we have an addition and multiplication defined on  
$\mathbb{Z}_n $, the integers modulo~$n$.  

Always remember that for each of the equations in the definitions, the operations on the left, 
$ \oplus \text{ and } \odot $, are the new operations that are being defined.  The operations on the right side of the equations $\left( + \text{ and } \cdot \right)$ are the known operations of addition and multiplication in  $\mathbb{Z}$.

Since  $\mathbb{Z}_n$ is a finite set, it is possible to construct addition and multiplication tables for $\mathbb{Z}_n$.    In constructing these tables, we follow the convention that all sums and products should be in the form $[ r]$, where 
$0 \leq r < n$.  For example, in $\Z_3$, we see that by the definition, $[1] \oplus [2] = [3]$, but since $\mod{3}{0}{3}$, we see that $[3] = [0]$ and so we write 
\[
[1] \oplus [2] = [3] = [0].
\]
Similarly, by definition, $[2] \odot [2] = [4]$, and in $\Z_3$, $[4] = [1]$.  So we write
\[
[2] \odot [2] = [4] = [1].
\]
The complete addition and multiplication tables for $\Z_3$ are
\begin{center}
\begin{tabular}{ c | c  c  c p{0.5in} c | c  c  c}
$\oplus$ & $[ 0 ]$ & $[ 1 ]$ & $[2]$ & & $\odot$ & $[ 0 ]$ & $[ 1 ]$ & $[2]$  \\ \cline{1-4} \cline{6-9}

$[ 0 ]$ & $[ 0 ]$ & $[ 1 ]$ & $[2]$ &  & $[ 0 ]$ & $[ 0 ]$ & $[ 0 ]$ & $[0]$  \\ 

$[ 1 ]$ & $[ 1 ]$ & $[ 2 ]$ & $[0]$ & & $[ 1 ]$ & $[ 0 ]$ & $[ 1 ]$ & $[2]$ \\

$[ 2 ]$ & $[ 2 ]$ & $[ 0 ]$ & $[1]$ & & $[ 2 ]$ & $[ 0 ]$ & $[ 2 ]$ & $[1]$ \\

\end{tabular}
\end{center}

%\begin{center}
%\begin{tabular}{ c | c  c  p{0.5in} c | c  c }
%$\oplus$ & $[ 0 ]$ & $[ 1 ]$ &  & $\odot$ & $[ 0 ]$ & $[ 1 ]$  \\ \cline{1-3} \cline{5-7}
%
%$[ 0 ]$ & $[ 0 ]$ & $[ 1 ]$ &  & $[ 0 ]$ & $[ 0 ]$ & $[ 0 ]$  \\ 
%
%$[ 1 ]$ & $[ 1 ]$ & $[ 0 ]$ &  & $[ 1 ]$ & $[ 0 ]$ & $[ 1 ]$  \\
%\end{tabular}
%\end{center}
%\begin{prog}[\textbf{The Integers Modulo 3}] \label{prog:Z3} \hfill \\
%Construct addition and multiplication tables for $\Z_3$, the integers modulo 3.
%\end{prog}
%\hbreak

%
\begin{prog}[\textbf{Modular Arithmetic in} $\boldsymbol{\Z_2}$, $\boldsymbol{\Z_5}$, \textbf{and} $\boldsymbol{\Z_6}$]\label{prog:modulararithmetic5-6} \hfill
\begin{enumerate}
  \item Construct addition and multiplication tables for $\Z_2$, the integers modulo 2.
\item Verify that the following addition and multiplication tables for  $\mathbb{Z}_5 $  are correct.
\label{exam:sec74-tables}

\begin{center}
\begin{tabular}{ c | c  c  c  c  c p{0.5in} c | c  c  c  c  c}
$\oplus$ & $[ 0 ]$ & $[ 1 ]$ & $[ 2 ]$ & $[ 3 ]$ & 
$[ 4 ]$ &  & $\odot$ & $[ 0 ]$ & $[ 1 ]$ & $[ 2 ]$ & $[ 3 ]$ & $[ 4 ]$ \\ \cline{1-6} \cline{8-13}

$[ 0 ]$ & $[ 0 ]$ & $[ 1 ]$ & $[ 2 ]$ & 
$[ 3 ]$ & $[ 4 ]$ &  & $[ 0 ]$ & $[ 0 ]$ & 
$[ 0 ]$ & $[ 0 ]$ & $[ 0 ]$ & $[ 0 ]$ 
\\ %\cline{1-6} %\cline{8-13}

$[ 1 ]$ & $[ 1 ]$ & $[ 2 ]$ & $[ 3 ]$ & 
$[ 4 ]$ & $[ 0 ]$ &  & $[ 1 ]$ & $[ 0 ]$ & 
$[ 1 ]$ & $[ 2 ]$ & $[ 3 ]$ & $[ 4 ]$ 
\\ %\cline{1-6} %\cline{8-13}

$[ 2 ]$ & $[ 2 ]$ & $[ 3 ]$ & $[ 4 ]$ & 
$[ 0 ]$ & $[ 1 ]$ &  & $[ 2 ]$ & $[ 0 ]$ & 
$[ 2 ]$ & $[ 4 ]$ & $[ 1 ]$ & $[ 3 ]$ 
\\ %\cline{1-6} %\cline{8-13}

$[ 3 ]$ & $[ 3 ]$ & $[ 4 ]$ & $[ 0 ]$ & 
$[ 1 ]$ & $[ 2 ]$ &  & $[ 3 ]$ & $[ 0 ]$ & 
$[ 3 ]$ & $[ 1 ]$ & $[ 4 ]$ & $[ 2 ]$ 
\\ %\cline{1-6} %\cline{8-13}

$[ 4 ]$ & $[ 4 ]$ & $[ 0 ]$ & $[ 1 ]$ & 
$[ 2 ]$ & $[ 3 ]$ &  & $[ 4 ]$ & $[ 0 ]$ & 
$[ 4 ]$ & $[ 3 ]$ & $[ 2 ]$ & $[ 1 ]$ 
\\ %\cline{1-6} %\cline{8-13}

\end{tabular}
\end{center}

\item Construct complete addition and multiplication tables for  $\Z_6 $.

\item In the integers, the following statement is true.  We sometimes call this the zero product property for the integers.

\begin{list}{}
\item For all $a, b \in \Z$, if  $a \cdot b = 0$, then  $a = 0$ or $b = 0$.
\end{list}

Write the contrapositive of the conditional statement in this property.

\item Are the following statements true or false? Justify your conclusions.
\begin{enumerate}
\item For all  $[ a ], [ b ] \in \Z_5 $, if  
$[ a ] \odot [ b ] = [ 0 ]$, then  
$[ a ] = [ 0 ]$ or  $[ b ] = [ 0 ]$.
\item For all  $[ a ], [ b ] \in \Z_6 $, if  
$[ a ] \odot [ b ] = [ 0 ]$, then 
$[ a ] = [ 0 ]$ or  $[ b ] = [ 0 ]$.
\end{enumerate}
\end{enumerate}
\end{prog}
\hbreak

\endinput
