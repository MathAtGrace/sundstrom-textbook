\subsection*{Divisibility Tests}
Congruence arithmetic can be used to prove certain divisibility tests.  For example, you may have learned that a natural number is divisible by 9 if the sum of its digits is divisible by 9.  As an easy example, note that the sum of the digits of 5823 is equal to $5 + 8 + 2 + 3 = 18$, and we know that 18 is divisible by 9.  It can also be verified that 5823 is divisible by 9. (The quotient is 647.)  We can actually generalize this property by dealing with remainders when a natural number is divided by 9.

Let  $n \in \mathbb{N}$ and let  $s( n )$  denote the sum of the digits of  $n$.  For example,  if  $n = 7319$, then $s( {7319} ) = 7 + 3 + 1 + 9 = 20$.  In \typeu 
Activity~\ref*{PA:remainderdivide9}, we saw that
\begin{center}
$7319 \equiv 2 \pmod 9$ and  $20 \equiv 2 \pmod 9$.
\end{center}
In fact, for every example in \typeu Activity~\ref*{PA:remainderdivide9}, we saw that  $n$  and   $s( n )$ were congruent modulo 9 since they both had the same remainder when divided by  9.  The concepts of congruence and congruence classes can help prove that this is always true.
 
\setcounter{equation}{0}
We will use the case of  $n = 7319$ to illustrate the general process.  We must use our standard place value system.  By this, we mean that we will write $7319$ as follows:
\begin{equation}
7319 = \left( {7 \times 10^3 } \right) + \left( {3 \times 10^2 } \right) + \left( {1 \times 10^1 } \right) + \left( 9 \times 10^0 \right).  \label{eq:divideby9-1}
\end{equation} 
The idea is to now use the definition of addition and multiplication in  $\mathbb{Z}_9 $
to convert equation~(\ref{eq:divideby9-1}) to an equation in $\mathbb{Z}_9 $.  We do this as follows:
%
\begin{align}
[ {7319} ] &= [ {\left( {7 \times 10^3 } \right) + \left( {3 \times 10^2 }                         \right) + \left( {1 \times 10^1 } \right) + \left( 9 \times 10^0 \right                          )} ]  \notag \\ 
            &= [ {7 \times 10^3 } ] \oplus [ {3 \times 10^2 } ]            \oplus [ {1 \times 10^1 } ] \oplus [ {9 \times 10^0 } ] 
\notag \\ 
            &= \left( {[ 7 ] \odot [ {10^3 } ]} \right) \oplus                          \left( {[ 3 ] \odot [ {10}^2 ]} \right) \oplus         \left( {[ 1 ] \odot [ 10^ 1 ]} \right) \oplus \left( [ 9 ] \odot [ 1 ] \right). \label{eq:divideby9-2}  %\\ \notag
\end{align}
%
Since $10^3  \equiv 1 \pmod 9$, $10^2  \equiv 1 \pmod 9$ and  
$10 \equiv 1 \pmod 9$, we can conclude that $[ 10^3 ] = [ 1 ]$,  
$[ {10^2 } ] = [ 1 ]$  and  $[ {10} ] = [ 1 ]$.  Hence, we can use these facts and  equation~(\ref{eq:divideby9-2}) to obtain
\begin{align} \notag
[ {7319} ] &=\left( {[ 7 ] \odot [ {10^3 } ]} \right) \oplus \left( {[ 3 ] \odot [ {10^2} ]} \right) \oplus \left( {[ 1 ] \odot [ 10 ]} \right) \oplus \left( [ 9 ] \odot [ 1 ] \right) \notag \\ 
                     &= \left( {[ 7 ] \odot [ 1 ]} \right) \oplus \left( {[ 3 ] \odot [ 1 ]} \right) \oplus \left( {[ 1 ] \odot [ 1 ]} \right) \oplus \left( [ 9 ] \odot [ 1 ] \right) \notag \\ 
                     &= [ 7 ] \oplus [ 3 ] \oplus [ 1 ] \oplus [ 9 ] \notag  \\ 
                     &= [ {7 + 3 + 1 + 9} ]. \label{eq:divideby9-3}  %\notag
\end{align}
Equation~(\ref{eq:divideby9-3}) tells us that 7319 has the same remainder when divided by 9 as the sum of its digits.  It is easy to check that the sum of the digits is 20 and hence has a remainder of 2.  This means that when 7319 is divided by 9, the remainder is 2.

\vskip6pt
To prove that any natural number has the same remainder when divided by 9 as the sum of its digits, it is helpful to introduce notation for the decimal representation of a natural number.  The notation we will use is similar to the notation for the number 7319 in equation~(\ref{eq:divideby9-1}).

In general, if  $n \in \mathbb{N}$, and  $n = a_k a_{k - 1}  \cdots a_1 a_0 $ is the decimal representation of  $n$, then
\[
n = \left( {a_k  \times 10^k } \right) + \left( {a_{k - 1}  \times 10^{k - 1} } \right) +  \cdots  + \left( {a_1  \times 10^1 } \right) + \left( {a_0  \times 10^0 } \right)\!.
\]
This can also be written using summation notation as follows:
\[
n = \sum\limits_{j = 0}^k {\left( {a_j  \times 10^j } \right)} .
\]
Using congruence classes for congruence modulo 9, we have
\begin{align}
[ n ] &= [ {\left( {a_k  \times 10^k } \right) + \left( {a_{k - 1}  \times 10^{k - 1} } \right) +  \cdots  + \left( {a_1  \times 10^1 } \right) + \left( {a_0  \times 10^0 } \right)} ]  \notag \\ 
                 &= [ {a_k  \times 10^k } ] \oplus [ {a_{k - 1}  \times 10^{k - 1} } ] \oplus  \cdots  \oplus [ {a_1  \times 10^1 } ] \oplus [ {a_0  \times 10^0 } ]  \notag \\ \notag 
%
                &= \left( {[ {a_k } ] \odot [ {10^k } ]} \right) \oplus \left( {[ {a_{k - 1} } ] \odot [ {10^{k - 1} } ]} \right) \oplus 
\cdots \\
& \hspace{5cm} \oplus \left( {[ {a_1 } ] \odot [ {10^1 } ]} \right) \oplus \left( {[ {a_0 } ] \odot [ {10^0 } ]} \right)\!.\label{eq:divideby9-4}
\end{align}  %\notag 
%
\setcounter{equation}{0}
One last detail is needed.  It is given in Proposition~\ref{P:poweroftenmod9}.  The proof by mathematical induction is Exercise~(\ref{exer:poweroftenmod9}).
%\hbreak
%
\begin{proposition} \label{P:poweroftenmod9}
If  $n$  is a nonnegative integer, then  $10^n  \equiv 1 \pmod 9$, and hence for the equivalence relation of congruence modulo  9, $[ {10^n } ] = [ 1 ]$.
\end{proposition}
%\hrule
%
%\vskip10pt
\noindent
If we let  $s( n )$ denote the sum of the digits of  $n$, then  
\[
s( n ) = a_k  + a_{k - 1}  +  \cdots  + a_1  + a_0,
\]
Now using equation~(\ref{eq:divideby9-4}) and Proposition~\ref{P:poweroftenmod9}, we obtain
\begin{align} \notag
[ n ] &= \left( {[ {a_k } ] \odot [ 1 ]} \right) \oplus \left( {[ {a_{k - 1} } ] \odot [ 1 ]} \right) \oplus  \cdots  \oplus \left( {[ {a_1 } ] \odot [ 1 ]} \right) \oplus \left( {[ {a_0 } ] \odot [ 1 ]} \right)  \notag \\ 
                 &= [ {a_k } ] \oplus [ {a_{k - 1} } ] \oplus  \cdots  \oplus [ {a_1 } ] \oplus [ {a_0 } ] \notag \\ 
                 &= [ {a_k  + a_{k - 1}  +  \cdots  + a_1  + a_0 } ]. \notag \\
%\label{eq:divideby9-5} \\
                 &= [ s( n ) ]. \notag
\end{align}
%and equation~(\ref{eq:divideby9-5}) shows that  $[ n ] = [ {S( n \right)} ]$.  
This completes the proof of Theorem~\ref{T:sumofdigitsmod9}.
%
%\hbreak
%
\begin{theorem} \label{T:sumofdigitsmod9}
Let  $n \in \mathbb{N}$ and let  $s( n )$ denote the sum of the digits of $n$.  Then
\begin{enumerate}
\item $[ n ] = [ {s( n )} ]$, using congruence classes modulo 9.

\item $n \equiv s( n ) \pmod 9$.

\item $9 \mid n$  if and only if \, $9 \mid s( n )$. \label{T:sumofdigitsmod9-part}
\end{enumerate}
\end{theorem}
Part~(\ref{T:sumofdigitsmod9-part}) of Theorem~\ref{T:sumofdigitsmod9} is called a \textbf{divisibility test}.
\index{divisibility test}%
\index{divisibility test!for 9}%
  If gives a necessary and sufficient condition for a natural number to be divisible by 9.  Other divisibility tests will be explored in the exercises.  Most of these divisibility tests can be proved in a manner similar to the proof of the divisibility test for 9.
%
\hbreak

\endinput
