\subsection*{Proving that Statements Are Equivalent}
When we have a list of three statements $P$, $Q$,  and  $R$  such that  each statement in the list is equivalent to the other two statements in the list, we say that the three statements are equivalent.  This means that each of the statements in the list implies each of the other statements in the list.  

The purpose of \typeu Activity~\ref*{PA:provingequiv} was to provide one way to prove that three (or more) statements are equivalent.  The basic idea is to prove a sequence of conditional statements so that there is an unbroken chain of conditional statements from each statement to every other statement.  This method of proof will be used in Theorem~\ref{T:subsetequivs}.
%
%\pagebreak
\begin{theorem} \label{T:subsetequivs}
Let  $A$  and  $B$  be subsets of some universal set  $U$.  The following are equivalent:
\begin{multicols}{3}
\begin{enumerate}
  \item $A \subseteq B$ \label{T:subsetitem1}
  \item $A \cap B^c  = \emptyset $  \label{T:subsetitem2}
  \item $A^c  \cup B = U$  \label{T:subsetitem3}
\end{enumerate}
\end{multicols}
\end{theorem}
%
\begin{myproof}
To prove that these are equivalent conditions, we will prove that (\ref{T:subsetitem1})  implies (\ref{T:subsetitem2}), that (\ref{T:subsetitem2})  implies (\ref{T:subsetitem3}), and that (\ref{T:subsetitem3}) implies (\ref{T:subsetitem1}).

Let  $A$  and  $B$  be subsets of some universal set  $U$.  We have proved that (\ref{T:subsetitem1}) implies (\ref{T:subsetitem2}) in Proposition~\ref{P:subsetprop}.
\vskip6pt

To prove that (\ref{T:subsetitem2}) implies (\ref{T:subsetitem3}), we will assume that  
$A \cap B^c  = \emptyset $ and use the fact that $\emptyset^c = U$.  We then see that
\[
\left( {A \cap B^c} \right)^c  = \emptyset^c.
\]
Then, using one of De Morgan's Laws, we obtain
\[
\begin{aligned}
A^c \cup \left(B^c \right)^c &= U \\
           A^c \cup B &= U. \\
\end{aligned}
\]
This completes the proof that (\ref{T:subsetitem2}) implies (\ref{T:subsetitem3}).
%\vskip6pt

We now need to prove that (\ref{T:subsetitem3}) implies (\ref{T:subsetitem1}).  We assume that  $A^c  \cup B = U$ and will prove that  $A \subseteq B$ by proving that every element of  $A$  must be in $B$.

So let  $x \in A$.  Then we know that  $x \notin A^c $.  However,  $x \in U$ and since  $A^c  \cup B = U$, we can conclude that  $x \in A^c  \cup B$.  Since  $x \notin A^c $, we conclude that  $x \in B$.  This proves that  $A \subseteq B$ and hence that (\ref{T:subsetitem3}) implies (\ref{T:subsetitem1}).
\vskip6pt

Since we have now proved that (\ref{T:subsetitem1}) implies (\ref{T:subsetitem2}), that (\ref{T:subsetitem2}) implies (\ref{T:subsetitem3}), and that (\ref{T:subsetitem3}) implies (\ref{T:subsetitem1}), we have proved that the three conditions are equivalent.
\end{myproof}
\hbreak



\endinput
