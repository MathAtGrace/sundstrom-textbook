In this section, we will learn how to prove certain relationships about sets.  Two of the most basic types of relationships between sets are the equality relation and the subset relation.  So if we are asked a question of the form, ``How are the sets  $A$  and  $B$  related?'', we can answer the question if we can prove that the two sets are equal or that one set is a subset of the other set.  There are other ways to answer this, but we will concentrate on these two for now.  This is similar to asking a question about how two real numbers are related.  Two real numbers can be related by the fact that they are equal or by the fact that one number is less than the other number.

\subsection*{The Choose-an-Element Method} \label{SS:choosemethod}
The method of proof we will use in this section can be called the \textbf{choose-an-element method.}
\index{choose-an-element method|(}%
 This method was introduced in \typeu Activity~\ref*{PA:working2sets}.  This method is frequently used when we encounter a universal quantifier in a statement in the backward process.  This statement often has the form

\begin{center}
For each element with a given property, something happens.
\end{center}
Since most statements with a universal quantifier can be expressed in the form of a conditional statement, this statement could have the following equivalent form:

\begin{center}
If an element has a given property, then something happens.
\end{center}
We will illustrate this with the proposition from \typeu Activity~\ref*{PA:working2sets}.  This proposition can be stated as follows:
\begin{list}{}
\item \emph{Let  S  be the set of all integers that are multiples of  6, and let  T  be the set of all even integers.  Then  S  is a subset of  T.}
\end{list}

\newpar
In \typeu Activity~\ref*{PA:working2sets}, we worked on a know-show table for this proposition.  The key was that in the backward process, we encountered the following statement:

\begin{list}{}
\item Each element of  $S$  is an element of   $T$ or, more precisely, if  $x \in S$\!, then  
$x \in T$\!.
\end{list}
\vskip10pt
In this case, the ``element'' is an integer, the ``given property'' is that it is an element of  $S$, and the ``something that happens'' is that  the element is also an element of  $T$.
One way to approach this is to create a list of all elements with the given property and verify that for each one, the ``something happens.''  When the list is short, this may be a reasonable approach. However, as in this case, when the list is infinite (or even just plain long), this approach is not practical.

We overcome this difficulty by using the \textbf{choose-an-element method}, where we choose an arbitrary element with the given property.  So in this case, we choose an integer  $x$  that is a multiple of  6.  We cannot use a specific multiple of 6 (such as 12 or 24), but rather the only thing we can assume is that the integer satisfies the property that it is a multiple of  6.  This is the key part of this method.

\vskip9pt
\begin{center}
\parbox{4in}{\emph{Whenever we choose an arbitrary element with a given property, we are not selecting a specific element.  Rather, the only thing we can assume about the element is the given property.}}
\end{center}
It is important to realize that once we have chosen the arbitrary element, we have added information to the forward process.  So in the know-show table for this proposition, we added the statement, ``Let  $x \in S$''  to the forward process.
Following is a completed proof of this proposition following the outline of the know-show table from \typeu Activity~\ref*{PA:working2sets}.

\begin{proposition} \label{P:SissubsetT}
Let  S  be the set of all integers that are multiples of  6, and let  T  be the set of all even integers.  Then  S  is a subset of  T.
\end{proposition}
%
\begin{myproof}
Let  $S$  be the set of all integers that are multiples of  6, and let  $T$  be the set of all even integers.  We will show that  $S$  is a subset of  $T$ by showing that if an integer $x$ is an element of  $S$, then it is also an element of  $T$.

Let  $x \in S$.  (\note  The use of the word ``let'' is often an indication that the we are choosing an arbitrary element.)  This means that  $x$  is a multiple of  6.  Therefore, there exists an integer  $m$  such that
\[
x = 6m.
\]
Since $6 = 2 \cdot 3$, this equation can be written in the form
\[
x = 2( {3m}).
\]
By closure properties of the integers,  $3m$  is an integer.  Hence, this last equation proves that  $x$  must be even.
Therefore, we have shown that if $x$ is an element of  $S$, then $x$ is an element of  $T$, and hence that  $S \subseteq T$.
\end{myproof}

Having proved that $S$ is a subset of $T$, we can now ask if $S$ is actually equal to $T$.  The work we did in \typeu Activity~\ref*{PA:working2sets} can help us answer this question.  In that activity, we should have found several elements that are in $T$ but not in $S$.  For example, the integer 2 is in $T$ since 2 is even but $2 \notin S$ since 2 is not a multiple of 6.  Therefore, $S \ne T$ and we can also conclude that $S$ is a proper subset of $T$.

One reason we do this in a ``two-step'' process is that it is much easier to work with the subset relation than the proper subset relation.  The subset relation is defined by a conditional statement and most of our work in mathematics deals with proving conditional statements.  In addition, the proper subset relation is a conjunction of two statements ($S \subseteq T$ and $S \ne T$) and so it is natural to deal with the two parts of the conjunction separately.
\hbreak

\begin{prog}[\textbf{Subsets and Set Equality}] \label{prog:setequality} \hfill \\
Let  $A = \left\{ x \in \mathbb{Z} \mid x \text{  is a multiple of  9} \right\}$ and let  
$B = \left\{ x \in \mathbb{Z} \mid  x \text{  is a multiple of  3} \right\}$. 
\begin{enumerate}
  \item Is the set $A$ a subset of $B$?  Justify your conclusion.  
  \item Is the set $A$ equal to the set $B$?  Justify your conclusion.
\end{enumerate}
\end{prog}
%\hbreak


\begin{prog}[\textbf{Using the Choose-an-Element Method}] \label{prog:usingchoose} \hfill \\
The Venn diagram in \typeu Activity~\ref*{PA:workingvenn} suggests that the following proposition is true.
\begin{proposition} \label{P:subsetandcomp}
Let  $A$  and  $B$  be subsets of the universal set  $U$.  If  $A \subseteq B$, then  $B^c  \subseteq A^c $.
\end{proposition}
%
\begin{enumerate}
\item The conclusion of the conditional statement is  $B^c  \subseteq A^c$.  Explain why we should try the choose-an-element method to prove this proposition.

\item Complete the following know-show table for this proposition and explain exactly where the choose-an-element method is used.
$$
\BeginTable
\def\C{\JustCenter}
\BeginFormat
|p(0.4in)|p(2in)|p(1.8in)|
\EndFormat
\_
 | \textbf{Step}  |  \textbf{Know}  |  \textbf{Reason}   |  \\+02 \_
 | $P$     |  $A \subseteq B$    |  Hypothesis  |\\ \_1
 | $P1$    |   Let  $x \in B^c $.        |  Choose an arbitrary element of  $B^c$. | \\ \_1
 | $P2$  |  If  $x \in A$, then  $x \in B$.  |  Definition of ``subset'' | \\  \_1
 | \C $\vdots$  |  \C $\vdots$                         | \C $\vdots$     |  \\ \_1
 | $Q1$    |   If $x \in B^c$, then $x \in A^c $.   |     | \\  \_1  
 | $Q$     |  $B^c \subseteq A^c$                     |  Definition of ``subset'' |    \\ \_
 | \textbf{Step}  |  \textbf{Show}  |  \textbf{Reason} |    \\  \_
\EndTable
$$
\end{enumerate}
\end{prog}
\index{choose-an-element method|)}%

\hbreak

\endinput
