\subsection*{Proving Set Equality}
One way to prove that two sets are equal 
\index{set!proving equality}%
 is to use Theorem~\ref{T:setequality} and prove each of the two sets is a subset of the other set.  In particular,  let  $A$  and  $B$  be subsets of some universal set.  Theorem~\ref{T:setequality} states that  
$A = B$  if and only if  $A \subseteq B\text{  and  }B \subseteq A$.

In \typeu Activity~\ref*{PA:workingvenn}, we created a Venn diagram that indicated that  
\linebreak $A - (A - B) = A \cap B$.  Following is a proof of this result.  Notice where the choose-an-element method is used in each case.

\begin{proposition} \label{P:setdifference}
Let  $A$  and  $B$  be subsets of some universal set.  Then 
\linebreak $A - (A - B) = A \cap B$.
\end{proposition}
%
\begin{myproof}
Let  $A$  and  $B$  be subsets of some universal set.  We will prove that \linebreak
$A - (A - B) = A \cap B$ by proving that  $A - (A - B) \subseteq A \cap B$  and that  
$A \cap B  \subseteq A - (A - B)$.

First, let  $x \in A - (A - B)$.  This means that
\[
x \in A\text{  and  }x \notin (A - B).
\]
We know that an element is in $(A - B)$ if and only if it is in $A$ and not in $B$.  Since $x \notin (A - B)$, we conclude that $x \notin A$ or $x \in B$.  However, we also know that $x \in A$ and so we conclude that $x \in B$.  This proves that 
\[
x \in A \text{  and  }x \in B.
\]
This means that  $x \in A \cap B$, and hence we have proved that  $A - (A - B) \subseteq A \cap B$.

Now choose  $y \in A \cap B$.  This means that
\[
y \in A\text{  and  }y \in B. 
\]
We note that $y \in (A - B)$ if and only if $y \in A$ and $y \notin B$ and hence, $y \notin (A - B)$ if and only if $y \notin A$ or $y \in B$.  Since we have proved that $y \in B$, we conclude that $y \notin (A - B)$, and hence, we have established that $y \in A$ and $y \notin (A - B)$.  This proves that if $y \in A \cap B$, then $y \in A - (A - B)$ and hence, $A \cap B \subseteq A - (A - B)$.

Since we have proved that  $A - (A - B) \subseteq A \cap B$  and  $A \cap B  \subseteq A - (A - B)$, we conclude that  $A - (A - B) = A \cap B$.
\end{myproof}
\hbreak
%
\begin{prog}[\textbf{Set Equality}] \label{prog:setequality2} \hfill \\
Prove the following proposition.  To do so, prove each set is a subset of the other set by using the choose-an-element method.

\begin{proposition} \label{P:setequality}
Let  $A$  and  $B$  be subsets of some universal set.  Then $A - B = A \cap B^c$.
\end{proposition}
\end{prog}
\hbreak

\endinput
