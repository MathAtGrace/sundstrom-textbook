\subsection*{Sequences as Functions}
A sequence can be considered to be an infinite list of objects that are indexed (subscripted) by the natural numbers (or some infinite subset of  $\mathbb{N} \cup \left\{ 0 \right\}$).  Using this idea, we often write a sequence in the following form:
\[
a_1 , a_2 ,  \ldots , a_n ,  \ldots .
\]
In order to shorten our notation, we will often use the notation  
$\left\langle {a_n } \right\rangle $
 to represent this sequence.  Sometimes a formula can be used to represent the terms of a sequence, and we might include this formula as the  $n$th  term in the list for a sequence such as in the following example:
\[
1, \frac{1}{2}, \frac{1}{3},  \ldots , \frac{1}{n},  \ldots .
\]
In this case, the  $n^\text{th}$ term of the sequence is  $\dfrac{1}{n}$.  If we know a formula for the  $n$th term, we often use this formula to represent the sequence.  For example, we might say
\begin{center}
Define the sequence  $\left\langle {a_n } \right\rangle $  by  $a_n  = \dfrac{1}{n}$  for each  $n \in \mathbb{N}$.
\end{center}
This shows that this sequence is a function with domain  $\mathbb{N}$.  If it is understood that the domain is $\N$, we could refer to this as the sequence $\left\langle \dfrac{1}{n} \right \rangle$. Given an element of the domain, we can consider  $a_n $ to be the output.  In this case, we have used subscript notation to indicate the output rather than the usual function notation.  We could just as easily write  
\[
a( n ) = \frac{1}{n} \text{ instead of } a_n  = \frac{1}{n}.  
\]
We make the following formal definition.
%
\begin{defbox}{sequence}{An (infinite) \textbf{sequence}
\index{sequence}%
 is a function whose domain is  $\mathbb{N}$ or some infinite subset of  $\mathbb{N} \cup \left\{ 0 \right\}$.}
\end{defbox}

\begin{prog}[\textbf{Sequences}] \label{pr:sequences} \hfill \\
Find the sixth and tenth terms of the following sequences, each of whose domain is $\N$:
\begin{enumerate}
\item $\dfrac{1}{3}, \dfrac{1}{6}, \dfrac{1}{9}, \dfrac{1}{12}, \ldots$
\item $\langle a_n \rangle$, where $a_n = \dfrac{1}{n^2}$ for each $n \in \N$
\item $\langle (-1)^n \rangle$
\end{enumerate}
\end{prog}
\hbreak

\endinput
