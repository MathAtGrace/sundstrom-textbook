\subsection*{Equality of Functions}
The idea of equality of functions has been in the background of our discussion of functions, and it is now time to discuss it explicitly.  The preliminary work for this discussion was \typeu Activity~\ref*{PA:diagonals}, in which 
$D = \mathbb{N} - \left\{ {1, 2} \right\}$, and there were two functions:
\begin{itemize}
\item $d\x D \to \mathbb{N} \cup \left\{ 0 \right\}$, where  $d( n )$ is the number of diagonals of a convex polygon with  $n$  sides

\item $f\x \mathbb{R} \to \mathbb{R}$, where  
$f( x ) = \dfrac{{x\left( {x - 3} \right)}}{2}$, for each real number $x$.
\end{itemize}

In \typeu Activity~\ref*{PA:diagonals}, we saw that these two functions produced the same outputs for certain values of the input (independent variable).  For example, we can verify that
\begin{align*}
  d( 3 ) &= f( 3 ) = 0, \qquad &d( 4 ) = f( 4 ) = 2,\\ 
  d( 5 ) &= f( 5 ) = 5, \qquad \text{and }  &d( 6 ) = f( 6 ) = 9. \\ 
\end{align*}
Although the functions produce the same outputs for some inputs, these are two different functions.  For example, the outputs of the function  $f$  are determined  by a formula, and the outputs of the function  $d$  are determined by a verbal description.  This is not enough, however, to say that these are two different functions.  Based on the evidence from \typeu Activity~\ref*{PA:diagonals},  we might make the following conjecture:
\begin{center}
For  $n \geq 3$,  $d( n ) = \dfrac{{n\left( {n - 3} \right)}}{2}$.
\end{center}
Although we have not proved this statement, it is a true statement. (See Exercise~\ref{exer:sec62-6}.)  However, we know the function $d$ and the function $f$ are not the same function.  For example,
\begin{itemize}
\item $f( 0 ) = 0$,  but  0  is not in the domain of  $d$;

\item $f( \pi ) = \dfrac{{\pi \left( {\pi  - 3} \right)}}{2}$, but $\pi $ is not in the domain of  $d$.

\end{itemize}

We thus see the importance of considering the domain and codomain of each of the two functions in determining whether the two functions are equal or not.  This motivates the following definition. 
%
\begin{defbox}{functionequality}{Two functions  $f$  and  $g$  are \textbf{equal}
\index{equal functions}%
\index{function!equality}%
 provided that
\begin{itemize}
\item The domain of  $f$  equals the domain of  $g$.  That is, \\ 
$\text{dom}( f ) = \text{dom}( g )$.

\item The codomain of  $f$  equals the codomain of  $g$.  That is,  \\
$\text{codom}( f ) = \text{codom}( g )$.

\item For each  $x$  in the domain of  $f$  (which equals the domain of  $g$),  \\
$f( x ) = g( x )$.

\end{itemize}}
\end{defbox}
%
\begin{prog}[\textbf{Equality of Functions}] \label{pr:equalfunc} \hfill \\
Let  $A$  be a nonempty set.  The \textbf{identity function on the set}
\index{identity function}%
  $\boldsymbol{A}$, denoted by  $I_A $\label{sym:idfunc}, is the function  $I_A\x A \to A$ defined by  $I_A ( x ) = x$ for every  $x$  in  $A$.   That is, for the identity map, the output is always equal to the input.

For this progress check, we will use the functions $f$ and $g$ from Progress 
Check~\ref{pr:congfunctions}. The identity function on the set $R_5$ is 
\begin{center} 
$I_{R_5}:R_5  \to R_5$ by $I_{R_5}(x) = x \pmod 5$, for each $x \in R_5$.
\end{center}
Is the identity function on $R_5$ equal to either of the functions $f$ or $g$ from Progress 
Check~\ref{pr:congfunctions}?  Explain.
\end{prog}
\hbreak

\endinput
