\subsection*{Properties of Equivalence Classes}
As we have seen, in \typeu Activity~\ref*{PA:setswithrelation}, the relation  $R$  was an equivalence relation. For that activity, we used  $R[ y ]$ to denote the equivalence class of   $y \in A$, and we observed that these equivalence classes were either equal or disjoint.

However, in \typeu Activity~\ref*{PA:setswithrelation}, the relation  $S$  was not an equivalence relation, and hence we do not use the term ``equivalence class'' for this relation.  We should note, however, that the sets  $S[ y ]$ were not equal and were not disjoint.  This exhibits one of the main distinctions between equivalence relations and relations that are not equivalence relations.  

In Theorem~\ref{T:propsofequivclasses}, we will prove that if  $\sim$  is an equivalence relation on the set  $A$, then we can ``sort'' the elements of  $A$  into distinct equivalence classes.  The properties of equivalence classes that we will prove are as follows: 
\index{equivalence class!properties of}%
(1) Every element of  $A$  is in its own equivalence class; (2) two elements are equivalent if and only if their equivalence classes are equal; and (3) two equivalence classes are either identical or they are disjoint.  
%\hbreak 
%
\begin{theorem} \label{T:propsofequivclasses}
Let  $A$  be a nonempty set and let $\sim$  be an equivalence relation on the set $A$. Then, 
\begin{enumerate}
\item For each  $a \in A$,  $a \in [ a ]$.
\label{T:propsofequivclasses1}%

\item For each  $a, b \in A$, $a \sim b$  if and only if \,   
$[ a ] = [ b ]$.
\label{T:propsofequivclasses2}%

\item For each  $a, b \in A$,   $[ a ] = [ b ]$  or  
$[ a ] \cap [ b ] = \emptyset $.
\label{T:propsofequivclasses3}%
\end{enumerate}
\end{theorem}
%
\begin{myproof}
Let  $A$  be a nonempty set and assume that  $\sim$  is an equivalence relation on  $A$.  To prove the first part of the theorem, let  $a \in A$.  Since  $\sim$ is an equivalence relation on  $A$, it is reflexive on  $A$.  Thus,  $a \sim a$, and we can conclude that  
$a \in [ a ]$.

%\vskip10pt
The second part of this theorem is a biconditional statement.  We will prove it by proving two conditional statements.  We will first prove that  if   $a \sim b$, then  
$[ a ] = [ b ]$.  So let  $a, b \in A$ and assume that  $a \sim b$.  We will now prove that the two sets  $[ a ]$  and  $[ b ]$ are equal.  We will do this by proving that each is a subset of the other. 

First, assume that  $x \in [ a ]$.  Then, by definition,  $x \sim a$.  Since we have assumed that  $a \sim b$, we can use the transitive property of  $\sim$  to conclude that  
$x \sim b$, and this means that  $x \in [ b ]$.  This proves that  
$[ a ] \subseteq [ b ]$.

We now assume that  $y \in [ b ]$.  This means that  $y \sim b$, and hence by the symmetric property, that  $b \sim y$.  Again, we are assuming that  $a \sim b$.  So we have  
\begin{center}
$a \sim b$ and  $b \sim y$.
\end{center}
We use the transitive property to conclude that  $a \sim y$ and then, using the symmetric property, we conclude that  $y \sim a$.  This proves that  $y \in [ a ]$ and, hence, that  $[ b ] \subseteq [ a ]$.  This means that we can conclude that if  $a \sim b$, then  $[ a ] = [ b ]$.

%\vskip10pt
We must now prove that if  $[ a ] = [ b ]$, then  $a \sim b$. Let  
$a, b \in A$ and assume that  $[ a ] = [ b ]$.  Using the first part of the theorem, we know that  $a \in [ a ]$ and since the two sets are equal, this tells us that  $a \in [ b ]$.  Hence by the definition of  $[ b ]$, we conclude that  $a \sim b$.  This completes the proof of the second part of the theorem.

%\vskip10pt
For the third part of the theorem, let  $a, b \in A$.  Since this part of the theorem is a disjunction, we will consider two cases:  Either  
\[
[ a ] \cap [ b ] = \emptyset  \quad \text{or}  \quad
[ a ] \cap [ b ] \ne \emptyset. 
\]
 
In the case  where  $[ a ] \cap [ b ] = \emptyset $, the first part of the disjunction is true, and hence there is nothing to prove.  So we assume that  
$[ a ] \cap [ b ] \ne \emptyset $ and will show that  
$[ a ] = [ b ]$.  Since  
$[ a ] \cap [ b ] \ne \emptyset $, there is an element  $x$  in  $A$  such that
\[
x \in [ a ] \cap [ b ].
\]
This means that  $x \in [ a ]$ and $x \in [ b ]$.  Consequently,  
$x \sim a$  and  $x \sim b$, and so we can use the second part of the theorem to conclude that   
$[ x ] = [ a ]$  and  $[ x ] = [ b ]$.  Hence, 
$[ a ] = [ b ]$, and we have proven that  
$[ a ] = [ b ]$  or  
$[ a ] \cap [ b ] = \emptyset $.
\end{myproof}
%\hbreak
%

Theorem~\ref{T:propsofequivclasses} gives the primary properties of equivalence classes.  Consequences of these properties will be explored in the exercises.  The following table restates the properties in Theorem~\ref{T:propsofequivclasses} and gives a verbal description of each one.

%\begin{center}
%\begin{tabular}{| p{2.2in} | p{2.2in} |} \hline
%\textbf{Formal Statement from}  &
%\textbf{Verbal Description} \\
%\textbf{Theorem~\ref{T:propsofequivclasses}} &  \\ \hline
%For each  $a \in A$,  $a \in [ a ]$. &
%Every element of $A$ is in its own equivalence class.  
%\\ \hline
%For each  $a, b \in A$, $a \sim b$  if and only if   
%$[ a ] = [ b ]$. &
%Two elements of $A$ are equivalent if and only if their equivalence classes are equal.
%\\ \hline
%For each  $a, b \in A$,   $[ a ] = [ b ]$  or  &  Any two equivalence classes are \\
%$[ a ] \cap [ b ] = \emptyset $.  &   either equal or they are disjoint.  This means that if two equivalence classes are not disjoint then they must be equal.
%\\ \hline
%\end{tabular}
%\end{center}
$$
\BeginTable
\BeginFormat
| p(2.2in) | p(2.2in) |
\EndFormat
\_
| \textbf{Formal Statement from \\Theorem~\ref{T:propsofequivclasses}} | \textbf{Verbal Description} | \\+22 \_
| For each  $a \in A$, $a \in [a]$.  | Every element of $A$ is in its own equivalence class. | \\+24 \_
| For each  $a, b \in A$, $a \sim b$  if and only if   
$[ a ] = [ b ]$. | Two elements of $A$ are equivalent if and only if their equivalence classes are equal. | \\+24 \_
| For each  $a, b \in A$,   $[ a ] = [ b ]$  or $[ a ] \cap [ b ] = \emptyset $. | Any two equivalence classes are either equal or they are disjoint.  This means that if two equivalence classes are not disjoint then they must be equal. | \\+24 \_
\EndTable
$$
\begin{prog}[\textbf{Equivalence Classes}] \label{prog:equivclass} \hfill \\
Let  $f\x \mathbb{R} \to \mathbb{R}$ be defined by  $f( x ) = x^2  - 4$ for each  
$x \in \mathbb{R}$.  Define a relation  $\sim$  on  $\mathbb{R}$  as follows:
\begin{center}
For  $a, b \in \mathbb{R}$,  $a \sim b$ if and only if  
$f( a ) = f( b )$.
\end{center}
In Exercise~(\ref{exer:equivrelwithfunction}) of Section~\ref{S:equivrelations}, we proved that  ~$\sim$ is an equivalence relation on  $\mathbb{R}$.  Consequently, each real number has an equivalence class.  For this equivalence relation,
\begin{enumerate}
\item Determine the equivalence classes of  5, $-5$, 10, $-10$, $\pi$, and $-\pi$.
\item Determine the equivalence class of 0.
\item If $a \in \R$, use the roster method to specify the elements of the equivalence class 
$[ a ]$.
\end{enumerate} 
%For example,
%\[
%\begin{aligned}
%[ 5 ] &= \left\{ { {x \in \mathbb{R} } \mid x \sim 5} \right\} \\ 
%                 &= \left\{ { {x \in \mathbb{R} } \mid f\left( x \right) = f\left( 5                     \right)} \right\} \\ 
%                 &= \left\{ { {x \in \mathbb{R} } \mid x^2  - 4 = 21} \right\} \\ 
%                 &= \left\{ { {x \in \mathbb{R} } \mid x^2  = 25} \right\} \\ 
%                 &= \left\{ { - 5, 5} \right\}. \\ 
%\end{aligned} 
%\]
%Similarly,  $[ { - 5} ] = \left\{ { - 5, 5} \right\}$,  
%$[ {10} ] = \left\{ { - 10, 10} \right\}$, and  
%$[ { - 10} ] = \left\{ { - 10, 10} \right\}$.
%
%For this equivalence relation, there are infinitely many equivalence classes, and so it is not possible to list all of them.  However, if  $a \in \mathbb{R}$, we see that
%\[
%\begin{aligned}
%[ a ] &= \left\{ { {x \in \mathbb{R} } \mid x \sim a} \right\} \\ 
%                 &= \left\{ { {x \in \mathbb{R} } \mid f\left( x \right) = f\left( a                     \right)} \right\} \\ 
%                 &= \left\{ { {x \in \mathbb{R} } \mid x^2  - 4 = a^2  - 4} \right\} \\ 
%                 &= \left\{ { {x \in \mathbb{R} } \mid x^2  = a^2 } \right\} \\ 
%                 &= \left\{ { - a, a} \right\}. \\ 
%\end{aligned} 
%\]
\end{prog}
\hbreak

The results of Theorem~\ref{T:propsofequivclasses} are consistent with all the equivalence relations studied in the \typel activities and in the progress checks.  Since this theorem applies to all equivalence relations, it applies to the relation of congruence modulo  $n$  on the integers.  Because of the importance of this equivalence relation, these results for congruence modulo  $n$  are given in the following corollary.

%\pagebreak
\begin{corollary}\label{C:propsofcongclasses}
Let  $n \in \mathbb{N}$.  For each  $a \in \mathbb{Z}$, let  $[ a ]$ represent the congruence class of  $a$  modulo  $n$.
\begin{enumerate}
\item For each  $a \in \mathbb{Z}$,  $a \in [ a ]$.

\item For each  $a, b \in \mathbb{Z}$, $ a \equiv  b \pmod n$ if and only if \,  $[ a ] = [ b ]$.

\item For each  $a, b \in \mathbb{Z}$, $[ a ] = [ b ]$  or  
$[ a ] \cap [ b ] = \emptyset $.
\end{enumerate}
\end{corollary}

For the equivalence relation of congruence modulo  $n$, Theorem~\ref{T:congtorem} and Corollary~\ref{C:congtorem}  tell us that each integer is congruent to its remainder when divided by  $n$, and that each integer is congruent modulo  $n$  to precisely one of one of the integers  $0, 1, 2,  \ldots , n - 1$.  This means that each integer is in precisely one of the congruence classes  $[ 0 ]$, $[ 1 ]$, $[ 2 ],  \ldots , [ {n - 1} ]$.   Hence, Corollary~\ref{C:propsofcongclasses} gives us the following result.

\begin{corollary}\label{C:propsofcongclasses2}
Let  $n \in \mathbb{N}$.  For each  $a \in \mathbb{Z}$, let  $[ a ]$ represent the congruence class of  $a$  modulo  $n$.
\begin{enumerate}
\item $\mathbb{Z} = [ 0 ] \cup [ 1 ] \cup [ 2 ] \cup   \cdots  \cup  [ {n - 1} ]$

\item For $j, k \in \left\{ {0, 1, 2,  \ldots , n - 1} \right\}$, if  $j \ne k$, then  
$[ j ] \cap [ k ] = \emptyset $.
\end{enumerate}
\end{corollary}
%
\hbreak
%

\endinput
