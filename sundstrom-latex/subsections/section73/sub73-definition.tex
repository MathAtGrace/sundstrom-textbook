\subsection*{The Definition of an Equivalence Class}
We have indicated that an equivalence relation on a set is a relation with a certain combination of properties (reflexive, symmetric, and transitive) that allow us to sort the elements of the set into certain classes.  We saw this happen in the \typel activities.  We can now illustrate specifically what this means.  For example, in \typeu  Activity~\ref*{PA:congruencemodulo3}, we used the equivalence relation of congruence modulo 3 on  $\mathbb{Z}$ to construct the following three sets:
\[
\begin{aligned}
  C[ 0 ] &= \left\{ { {a \in \mathbb{Z} } \mid a \equiv 0 \pmod 3} \right\}\!, \\ 
  C[ 1 ] &= \left\{ { {a \in \mathbb{Z} } \mid a \equiv 1 \pmod 3} \right\}\!,\text{ and} \\ 
  C[ 2 ] &= \left\{ { {a \in \mathbb{Z} } \mid a \equiv 2 \pmod 3} \right\}\!. \\ 
\end{aligned}
\]
The main results that we want to use now are Theorem~\ref{T:congtorem} and Corollary~\ref{C:congtorem} on page~\pageref{T:congtorem}.  This corollary tells us that for any  $a \in \mathbb{Z}$,  $a$  is congruent to precisely one of the integers 0, 1, or 2.   Consequently, the integer  $a$  must be congruent to 0, 1, or  2, and it cannot be congruent to two of these numbers. Thus

\begin{enumerate}
\item For each  $a \in \mathbb{Z}$,  $a \in C[ 0 ]$, $a \in C[ 1 ]$, or
$a \in C[ 2 ]$; and

\item $C[ 0 ] \cap C[ 1 ] = \emptyset$, 
$C[ 0 ] \cap C[ 2 ] = \emptyset$ , and 
$C[ 1 ] \cap C[ 2 ] = \emptyset $.
\end{enumerate}

This means that the relation of congruence modulo 3 sorts the integers into three distinct sets, or classes, and that each pair of these sets have no elements in common.  
%Theorem~\ref{T:congtorem} and the Division Algorithm tell us that each integer is congruent, modulo 3, to its remainder when it is divided by 3, and Theorem~\ref{T:congruence-remainder} tells us that two integers are congruent modulo 3  if and only if they both have the same remainder when divided by 3.  
So if we use a rectangle to represent  $\mathbb{Z}$, we can divide that rectangle into three smaller rectangles, corresponding to  $C[ 0 ]$, 
$C[ 1 ]$, and $C[ 2 ]$, and we might picture this situation as follows:
%\begin{center}
%\textbf{The Integers}
%\begin{tabular}{| p{1.4in} | p{1.4in} | p{1.4in} |} \hline
%$C[ 0 ]$ consisting of all integers with a remainder of 0 when divided by 3  &
%$C[ 1 ]$ consisting of all integers with a remainder of 1 when divided by 3  &
%$C[ 2 ]$ consisting of all integers with a remainder of 2 when divided by 3
%\\ \hline
%\end{tabular}
%\end{center}
$$
\BeginTable
\BeginFormat
|p(1.4in) | p(1.4in) | p(1.4in) |
\EndFormat
" \use3 \JustCenter \textbf{The Integers} " \\+02 \_
|$C[ 0 ]$ consisting of all integers with a remainder of 0 when divided by 3 |
$C[ 1 ]$ consisting of all integers with a remainder of 1 when divided by 3 |
$C[ 2 ]$ consisting of all integers with a remainder of 2 when divided by 3 | \\+33 \_
\EndTable
$$

Each integer is in exactly one of the three sets  $C[ 0 ]$, 
$C[ 1 ]$, or $C[ 2 ]$, and two integers are congruent modulo 3  if and only if they are in the same set.  We will see that, in a similar manner, if  $n$  is any natural number, then the relation of congruence modulo  $n$  can be used to sort the integers into  $n$  classes.   We will also see that in general, if we have an equivalence relation  $R$  on a set  $A$, we can sort the elements of the set  $A$  into classes in a similar manner.

\begin{defbox}{equivalenceclass}{Let  $\sim$  be an equivalence relation on a nonempty set  $A$.  For each  $a \in A$, the \textbf{equivalence class of}
\index{equivalence class}%
  $\boldsymbol{a}$  determined by  $\sim$  is the subset of  $A$, denoted by  $[ a ]$, 
\label{sym:equivclass} 
consisting of all the elements of  $A$  that are equivalent to  $a$.  That is,
\[
[ a ] = \left\{ { {x \in A } \mid x \sim a} \right\}\!.
\]
We read  $[ a ]$ as ``the equivalence class of  $a$'' or as ``bracket $a$.''}
\end{defbox}

\noindent
\textbf{Notes}
\begin{enumerate}
  \item We use the notation  $[ a ]$ when only one equivalence relation is being used.  If there is more than one equivalence relation, then we need to distinguish between the equivalence classes for each relation.  We often use something like  
$[ a ]_{ \sim} $, or if  $R$  is the name of the relation, we can use  
$R[ a ]$ or $[a]_R$ for the equivalence class of  $a$  determined by  $R$.
In any case, always remember that when we are working with any equivalence relation on a set  $A$  if $a \in A$,  then \emph{the equivalence class  $[ a ]$ is a subset of  $A$.}

  \item We know that each integer has an equivalence class for the equivalence relation of congruence modulo 3.  But as we have seen, there are really only three \emph{distinct} equivalence classes.  Using the notation from the definition, they are:
\begin{align*}
[0] &= \left\{ a \in \Z \mid \mod{a}{0}{3} \right\}, & [1] &= \left\{ a \in \Z \mid \mod{a}{1}{3} \right\}, \text{ and} \\
[2] &= \left\{ a \in \Z \mid \mod{a}{2}{3} \right\}.
\end{align*}
\end{enumerate}
\hbreak


\begin{prog}[\textbf{Equivalence Classes from \typeu Activity \ref*{PA:setswithrelation}}] \label{prog:prev73} \hfill \\
Without using the terminology at that time, we actually determined the equivalence classes of the equivalence relation $R$ in \typeu Activity \ref*{PA:setswithrelation}.  What are the distinct equivalence classes for this equivalence relation?
\end{prog}
\hbreak

\endinput
