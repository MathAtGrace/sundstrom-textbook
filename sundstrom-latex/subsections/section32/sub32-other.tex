\subsection*{Using Other Logical Equivalencies}
As was noted in Section \ref{S:logequiv}, there are several different logical equivalencies.  Fortunately, there are only a small number that we often use when trying to write proofs, and many of these are listed in Theorem~\ref{T:logequiv}  at the end of Section \ref{S:logequiv}.  We will illustrate the use of one of these logical equivalencies with the following proposition:
\begin{list}{}
  \item For all real numbers $a$ and $b$, if  $a \ne 0$  and  $b \ne 0$, then  $ab \ne 0$.
\end{list}
\vskip6pt
\noindent
First, notice that the hypothesis and the conclusion of the conditional statement are stated in the form of negations.  This suggests that we consider the contrapositive.  Care must be taken when we negate the hypothesis since it is a conjunction.  We use one of De Morgan's Laws as follows:
\[
\mynot  \left( {a \ne 0\; \wedge \;b \ne 0} \right) \equiv \left( {a = 0} \right) \vee \left( {b = 0} \right).
\]
\hbreak

\begin{prog}[\textbf{Using Another Logical Equivalency}]\label{pr:contrapositive} \hfill
\begin{enumerate}
\item In English, write the contrapositive of, ``For all real numbers $a$ and $b$, if  $a \ne 0$  and  $b \ne 0$, then  $ab \ne 0$.'' 
\label{pr:contrapositive1}%
\end{enumerate}

The contrapositive is a conditional statement in the form   $X \to \left( {Y \vee Z} \right)$.  The difficulty is that there is not much we can do with the hypothesis  $\left( {ab = 0} \right)$ since we know nothing else about the real numbers  $a$  and  $b$.  However, if we knew that  $a$  was not equal to zero, then we could multiply both sides of the equation  $ab = 0$  by  
$\dfrac{1}{a}$.   This suggests that we consider using the following logical equivalency based on a result in  Theorem~\ref{T:logequiv} on page~\pageref{T:logequiv}:
\[
X \to \left( {Y \vee Z} \right) \equiv \left( {X \wedge \mynot  Y} \right) \to Z.
\]

\begin{enumerate} \setcounter{enumi}{1}
\item In English, use this logical equivalency to write a statement that is logically equivalent to the contrapositive from Part~(\ref{pr:contrapositive1}).
\label{pr:contrapositive2}%
\end{enumerate}

%\hbreak

%So the contrapositive is
%\begin{list}{}
%  \item For all real numbers $a$ and $b$, if  $ab = 0$, then  $a = 0\text{  or  }b = 0$.
%\end{list}
%\vskip6pt
%
%The contrapositive is a conditional statement in the form   $P \to \left( {Q \vee R} \right)$.  The difficulty is that there is not much we can do with the hypothesis  $\left( {ab = 0} \right)$ since we know nothing else about the real numbers  $a$  and  $b$.  However, if we knew that  $a$  was not equal to zero, then we could multiply both sides of the equation  $ab = 0$  by  
%$\dfrac{1}{a}$.   This suggests that we consider using the following logical equivalency from Theorem~\ref{T:logequiv} on page~\pageref{T:logequiv}:
%\[
%P \to \left( {Q \vee R} \right) \equiv \left( {P \wedge \mynot  Q} \right) \to R.
%\]
%
The logical equivalency in Part~(\ref{pr:contrapositive2}) makes sense because if we are trying to prove  $Y \vee Z$, we only need to prove that at least one of  $Y$  or  $Z$  is true.  So the idea is to prove that if  $Y$  is false, then  $Z$  must be true.  
%In the following completed proof, we will make use of this logical equivalency in this informal manner rather than explicitly stating the logical equivalency.  (However, it is perfectly acceptable to make explicit use of the logical equivalency.)
%\hbreak
\setcounter{oldenumi}{\theenumi}
\begin{enumerate} \setcounter{enumi}{\theoldenumi}
\item Use the ideas presented in the progress check to complete the proof of the following proposition.

\begin{proposition}\label{P:abnotzero}
For all real numbers $a$ and $b$, if $ a \ne 0$ and $ b \ne 0 $, then $ ab \ne 0 $.
\end{proposition}
\end{enumerate}

\begin{myproof}
We will prove the contrapositive of this proposition, which is
\begin{center}
For all real numbers $a$ and $b$, if  $ab = 0$, then  $a = 0\text{  or  }b = 0$.
\end{center}

\noindent
This contrapositive, however, is logically equivalent to the following:
\begin{center}
For all real numbers $a$ and $b$, if $ab = 0$ and $a \ne 0$, then $b = 0$.
\end{center}
To prove this, we let $a$ and $b$ be real numbers and assume that  $ab = 0$ and $a \ne 0$.  We can then multiply both sides of the equation  $ab = 0$  by  $ \dfrac{1}{a} $.  This gives
\begin{center}
Now complete the proof. \\
$\vdots$
\end{center}
\quarter

%\[
%\frac{1}{a}\left( {ab} \right) = \frac{1}{a} \cdot 0.
%\]
%We now use the associative property on the left side of this equation and simplify both sides of the equation to obtain
%
%\[
%\begin{aligned}
%  \left( {\frac{1}{a} \cdot a} \right)b &= 0 \\ 
%  1 \cdot b &= 0 \\ 
%  b &= 0. \\ 
%\end{aligned} 
%\]
%
Therefore, $b = 0$.  This completes the proof of a statement that is logically equivalent to the contrapositive, and hence, we have proven the proposition.
\end{myproof}
\end{prog}
\hbreak


\endinput
