\subsection*{Writing Guidelines} 
\index{writing guidelines}%
One of the basic rules of writing mathematical proofs is to keep the reader informed.  So when we prove a result using the contrapositive, we indicate this within the first few lines of the proof.  For example, 
%\vskip10pt
%\noindent
%\textbf{Examples:}
\begin{itemize}
  \item We will prove this theorem by proving its contrapositive.

  \item We will prove the contrapositive of this statement.
\end{itemize}

In addition, make sure the reader knows the status of every assertion that you make.  That is, make sure you state whether an assertion is an assumption of the theorem, a previously proven result, a well-known result, or something from the reader's mathematical background. 
Following is a completed proof of a statement from 
\typeu Activity~\ref*{PA:attempt}.
%\hbreak

\begin{theorem}\label{T:n2odd}
For each integer $n$, if $n^2$ is an even integer, then $n$ is an even integer.
\end{theorem}
%
\begin{myproof}
We will prove this result by proving the contrapositive of the statement, which is

\begin{list}{}
  \item For each integer $n$, if  $n$  is an odd integer, then  $n^2 $  is an odd integer.
\end{list}

\newpar
However, in Theorem~\ref{T:xyodd} on page~\pageref{T:xyodd}, we have already proven that if $x$ and $y$ are odd integers, then $x \cdot y$ is an odd integer.  So using $x = y = n$, we can conclude that if $n$ is an odd integer, then $n \cdot n$, or $n^2$, is an odd integer.
%So we assume that  $n$  is an odd integer and will show that  $n^2 $  is an odd integer.   By definition, there exists an integer  $k$  such that $n = 2k + 1$.
%Squaring both sides of this equation yields
%
%\[
%n^2  = 4k^2  + 4k + 1.
%\]
%However, we can rewrite the right side of this equation as follows:
%\[
%n^2  = 2\left( {2k^2  + 2k} \right) + 1.
%\]
%Using the closure properties of the integers, we conclude that  $2k^2  + 2k$  is an integer.  Hence, the last equation proves that  $n^2 $  is an odd integer.   
We have thus proved the contrapositive of the theorem, and consequently, we have proved that if  $n^2 $ is an even integer, then  $n$  is an even integer.
\end{myproof}
\hbreak

\endinput
