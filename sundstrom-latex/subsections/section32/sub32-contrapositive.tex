\subsection*{Proof Using the Contrapositive}
\index{contrapositive}%
\index{proof!contrapositive}%

As we saw in \typeu Activity \ref*{PA:attempt}, it is sometimes difficult to construct a direct proof of a conditional statement.  This is one reason we studied logical equivalencies in Section~\ref{S:logequiv}. Knowing that two expressions are logically equivalent tells us that if we prove one, then we have also proven the other.  In fact, once we know the truth value of a statement, then we know the truth value of any other statement that is logically equivalent to it.

One of the most useful logical equivalencies in this regard is that a conditional statement   $P \to Q$  is logically equivalent to its contrapositive,  
$\mynot  Q \to \mynot  P$.  This means that if we prove the contrapositive of the conditional statement, then we have proven the conditional statement.    The following are some important points to remember.

\begin{itemize}
  \item A conditional statement is logically equivalent to its contrapositive.  %If the conditional statement is, ``If  $P$  then  $Q$,'' then the contrapositive is ``If not $Q$ then not $P$.''  Thus, $ P \to Q $ can be proven to be true by proving that its contrapositive $ \mynot  Q \to \mynot  P$  is true.

  \item Use a direct proof to prove that  $ \mynot  Q \to \mynot  P $ is true.

  \item Caution:  One difficulty with this type of proof is in the formation of correct negations. (We need to be very careful doing this.)

  \item We might consider using a proof by contrapositive when the statements  $P$  and  $Q$  are stated as negations.

\end{itemize}

\endinput
