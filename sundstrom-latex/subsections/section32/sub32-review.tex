\subsection*{Review of Direct Proofs}
In Sections~\ref{S:direct} and~\ref{S:directproof}, we studied direct proofs
\index{direct proof}%
\index{proof!direct}%
 of mathematical statements.  Most of the statements we prove in mathematics are conditional statements that can be written in the form   $P \to Q$.  A direct proof of a statement of the form  $P \to Q$ is based on the definition that a conditional statement can only be false when the hypothesis,  $P$,  is true and the conclusion,  $Q$,  is false.  Thus, if the conclusion is true whenever the hypothesis is true, then the conditional statement must be true.  So, in a direct proof,
\begin{itemize}
  \item We start by assuming that  $P$  is true.
  \item From this assumption, we logically deduce that  $Q$  is true.
\end{itemize}
We have used the so-called forward and backward method to discover how to logically deduce  $Q$  from  the assumption that  $P$  is true.

\endinput
