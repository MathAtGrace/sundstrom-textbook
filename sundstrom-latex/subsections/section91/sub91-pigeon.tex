\subsection*{The Pigeonhole Principle}

The last property of finite sets that we will consider in this section is often called the 
\textbf{Pigeonhole Principle}.
\index{Pigeonhole Principle}%
  The ``pigeonhole'' version of this property says,  ``If $m$ pigeons go into $r$ pigeonholes and $m > r$, then at least one pigeonhole has more than one pigeon.''

In this situation, we can think of the set of pigeons as being equivalent to a set $P$ with cardinality $m$ and the set of pigeonholes as being equivalent to a set $H$ with cardinality 
$r$.  We can then define a function $f\x P \to H$ that maps each pigeon to its pigeonhole.  The Pigeonhole Principle states that this function is not an injection.  (It is not one-to-one since there are at least two pigeons ``mapped'' to the same pigeonhole.)

\begin{theorem} [\textbf{The Pigeonhole Principle}] \label{T:pigeonhole}
Let $A$ and $B$ be finite sets.  If $\text{card} ( A ) > \text{card} ( B )$, then any function $f\x A \to B$ is not an injection.
\end{theorem}
%
\begin{myproof}
Let $A$ and $B$ be finite sets. We will prove the contrapositive of the theorem, which is, 
if there exists a function $f\x A \to B$ that is an injection, then 
$\text{card} ( A ) \leq \text{card} ( B )$.

So assume that $f\x A \to B$ is an injection.  As in Theorem~\ref{T:finitesubsets}, we define a function $g\x A \to f ( A )$ by 
\begin{center}
$g ( x ) = f ( x )$ for each $x \in A$.
\end{center}
As we saw in Theorem~\ref{T:finitesubsets}, the function $g$ is a bijection.  But then 
$A \approx f ( A )$ and $f ( A ) \subseteq B$.  Hence, 
\begin{center}
$\text{card} ( A ) = \text{card} \!\left( f ( A ) \right)$ and 
$\text{card} \!\left( f ( A ) \right) \leq \text{card} ( B )$.
\end{center}
Hence, $\text{card} ( A ) \leq \text{card} ( B )$, and this proves the contrapositive.  Hence, if $\text{card} ( A ) > \text{card} ( B )$, then  any function $f\x A \to B$  is not an injection.
\end{myproof}
The Pigeonhole Principle has many applications in the branch of mathematics called ``combinatorics.''  Some of these will be explored in the exercises.
\hbreak
\index{finite set!properties of|)}%

\endinput
