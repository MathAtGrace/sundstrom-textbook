\subsection*{Some Set Notation}\label{SS:sets}
In \typeu Activity~\ref*{PA:sets}, we indicated that a set is a well-defined collection of objects that can be thought of as an entity itself.  
\begin{itemize}
  \item If  $A$  is a set and  $y$  is one of the objects in the set $A$, we write  $y \in A$ 
\label{sym:elementof}%
 and read this as ``$y$ is an element of  $A$''  or ``$y$ is a member of  $A$.''  For example, if $B$ is the set of all integers greater than 4, then we could write $5 \in B$ and $10 \in B$.
  \item If an object  $z$  is not an element in the set  $A$, we write  $z \notin A$ 
\label{sym:notelement}%
 and read this as ``$z$  is not an element of  $A$.''  For example, if $B$ is the set of all integers greater than 4, then we could write $-2 \notin B$ and $4 \notin B$.
\end{itemize}
%
When working with a mathematical object, such as set, we need to define when two of these objects are equal.  We are also often interested in whether or not one set is contained in another set.

\begin{defbox}{D:setequality}{Two sets, $A$ and $B$,  are \textbf{equal} \label{sym:setequal12}
\index{equal sets}%
\index{set equality}%
\index{set!equality}%
 when they have precisely the same elements.  In this case, we write  $A = B$ \label{sym:setequal2}.  
%More formally, 
%\begin{center}
%$A = B$ provided that for each $x \in U$, $x \in A$ if and only if $x \in B$.
%\end{center}
When the sets  $A$  and  $B$  are not equal, we write  $A \ne B$.

\newpar
The set $A$ is a \textbf{subset} 
\index{subset}%
 of a set $B$ provided that each element of $A$ is an element of $B$.  In this case, we write $A \subseteq B$ and also say that $A$ is \textbf{contained} in $B$.  \label{sym:subset2} 
%More formally,
%\begin{center}
%$A \subseteq B$ provided that for each $x \in U$, if $x \in A$, then $x \in B$.
%\end{center}
When $A$ is not a subset of $B$, we write $A \not \subseteq B$.
}
\end{defbox}  
\noindent
Using these definitions, we see that for any set $A$, $A = A$ and since it is true that for each $x \in U$, if $x \in A$, then $x \in A$, we also see that $A \subseteq A$.  That is, any set is equal to itself and any set is a subset of itself.  For some specific examples, we see that:

\begin{multicols}{2}
\begin{itemize}
\item $\left\{ {1, 3, 5} \right\} = \left\{ {3, 5, 1} \right\}$

\item $\left\{ {4, 8, 12} \right\} = \left\{ {4, 4, 8, 12, 12} \right\}$

\item $\left\{ {5, 10} \right\} = \left\{ {5, 10, 5} \right\}$
\end{itemize}
\end{multicols}
\begin{itemize}
\item $\left\{ {5, 10} \right\} \ne \left\{ {5, 10, 15} \right\}$ but 
$\left\{ {5, 10} \right\} \subseteq \left\{ {5, 10, 15} \right\}$ and 
$\left\{ {5, 10, 15} \right\} \not \subseteq \left\{ {5, 10} \right\}$.
\end{itemize}
In each of the first three examples, the two sets have exactly the same elements even though the elements may be repeated or written in a different order.
\hbreak

\begin{prog}[\textbf{Set Notation}]\label{pr:setnotation} \hfill
\begin{enumerate}
  \item Let $A = \{ -4, -2, 0, 2, 4, 6, 8, \ldots \}$.  Use correct set notation to indicate which of the following integers are in the set $A$ and which are not in the set $A$.  For example, we could write $6 \in A$ and $5 \notin A$.
\[
10 \qquad 22 \qquad 13 \qquad -3  \qquad 0 \qquad -12
\]
  \item Use correct set notation (using $=$ or $\subseteq$) to indicate which of the following sets are equal and which are subsets of one of the other sets.
\begin{align*}
A &= \{3, 6, 9 \} & B &= \{6, 9, 3, 6 \} \\
C &= \{3, 6, 9, \ldots \} & D &= \{ 3, 6, 7, 9 \} \\
E &= \{9, 12, 15, \ldots \} & F &= \{9, 7, 6, 2 \}
\end{align*}
\end{enumerate}
\end{prog}
\hbreak

\endinput
