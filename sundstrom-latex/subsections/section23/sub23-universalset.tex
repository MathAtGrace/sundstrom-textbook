\subsection*{When the Truth Set Is the Universal Set}
The truth set of a predicate can be the universal set.  For example, if the universal set is the set of real numbers $\mathbb{R}$, then the truth set of the predicate ``$x + 0 = x$''
is $\mathbb{R}$.  

Notice that the sentence ``$x + 0 = x$'' has not been quantified and a particular element of the universal set has not been substituted for the variable $x$.  Even though the truth set for this sentence is the universal set, we will adopt the convention that unless the quantifier is stated explicitly, we will consider the sentence to be a predicate or open sentence.  So, with this convention, if the universal set is $\mathbb{R}$, then
\begin{itemize}
\item $x + 0 = x$ is a predicate;

\item For each real number $x$, $\left( x + 0 = x \right)$ is a statement.
%\item $\left( \exists x \in \mathbb{R} \right) \left( x + 0 = x \right)$ is a statement.
\end{itemize}
\hbreak

\endinput
