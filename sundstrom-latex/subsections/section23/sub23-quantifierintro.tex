\subsection*{An Introduction to Quantifiers}
\index{quantifier}%
We have seen that one way to create a statement from a predicate is to substitute a specific element from the universal set for each variable in the predicate.  Another way is to make some claim about the truth set of the predicate.  This is often done by using a quantifier.    For example, if the universal set is  $\mathbb{R}$, then is the following sentence a statement or a predicate?
\begin{center}
For each real number  $x$,  $x^2 > 0$.
\end{center}
The phrase ``For each real number  $x$'' is said to quantify the variable that follows it in the sense that the sentence is claiming that something is true for all real numbers.  This sentence is a statement (which happens to be false).
%
\begin{defbox}{D:every}{The phrase ``for every'' (or its equivalents) is called a \textbf{universal quantifier}.
\index{universal quantifier}%
\index{quantifier!universal}%
  The phrase ``there exists'' (or its equivalents) is called an \textbf{existential quantifier}.
\index{existential quantifier}%
\index{quantifier!existential}%
  The symbol $\forall$ 
\label{sym:forall}%
 is used to denote a universal quantifier, and the symbol  $\exists $ 
\label{sym:exist}%
 is used to denote an existential quantifier.}
\end{defbox}
%
%\begin{flushleft}
%\fbox{\parbox{5in}{\begin{definition} \label{D:quantifier}
%The phrase ``for every'' (or its equivalents) is called a \textbf{universal quantifier}.  The phrase ``there exists'' (or its equivalents) is called an \textbf{existential quantifier}.  In informal writing, the symbol $\forall$ is used to denote a universal quantifier, and the symbol  $\exists $ is used to denote an existential quantifier.
%\end{definition}}}
%\end{flushleft}
For example, the statement ``For each real number  $x$,  $x^2 > 0$'' could be written in symbolic form as: $\left( {\forall x \in \mathbb{R}} \right)\left( {x^2 > 0} \right)$.
%\[
%\left( {\forall x \in \mathbb{R}} \right)\left( {x^2 > 0} \right).
%\]
The following is an example of a statement involving an existential quantifier.
\begin{center}
There exists a rational number $x$ such that  $x^2  - 3x - 7 = 0$.
\end{center}
This could be written in symbolic form as
\[
\left( {\exists x \in \mathbb{Q}} \right)\left( {x^2  - 3x - 7 = 0} \right).
\]
This statement is false because there are no rational numbers that are solutions of the quadratic equation $x^2  - 3x - 7 = 0$.
Table~\ref{T:quantifiers} summarizes the facts about the two types of quantifiers.


%\begin{table}[h]
%\begin{tabular}{| p{1.2in} | p{1.5in} | p{1.5in} |}
%\hline
% \textbf{A statement involving }  &  \textbf{Often has the form}  &  \textbf{The statement is true provided that}  \\ \hline
%  A universal quantifier: $\left( \forall x, P(x) \right)$  &  ``For every $x$, $P(x)$,'' where $P(x)$ is a predicate.  &  Every value of $x$ in the universal set makes $P(x)$ true.  \\ \hline
%  An existential quantifier: $\left( \exists x, P(x) \right)$     &   ``There exists an $x$ such that $P(x)$,'' where $P(x)$ is a predicate.                              &   There is at least one value of $x$ in the universal set that makes $P(x)$ true.          \\  \hline
%\end{tabular}
%\caption{Properties of Quantifiers}
%\label{T:quantifiers}
%\end{table}


\begin{table}[!h]
$$
\BeginTable
\BeginFormat
|p(1.2in)|p(1.5in)|p(1.5in)|
\EndFormat
\_
 | \textbf{A statement involving }  |  \textbf{Often has the form}  |  \textbf{The statement is true provided that} | \\+22 \_
 | A universal quantifier: $\left( \forall x, P(x) \right)$  |  ``For every $x$, $P(x)$,'' where $P(x)$ is a predicate.  |  Every value of $x$ in the universal set makes $P(x)$ true. | \\ \_
 | An existential quantifier: $\left( \exists x, P(x) \right)$     |   ``There exists an $x$ such that $P(x)$,'' where $P(x)$ is a predicate.                              |   There is at least one value of $x$ in the universal set that makes $P(x)$ true.   |       \\ \_
\EndTable
$$
\caption{Properties of Quantifiers}
\label{T:quantifiers}
\end{table}

In effect, the table indicates that the universally quantified statement is true provided that the truth set of the predicate equals the universal set, and the existentially quantified statement is true provided that the truth set of the predicate contains at least one element.  We will study quantifiers more extensively in Section~\ref{S:quantifier}.
\hbreak
%
%\subsection*{The Empty Set}
%When a set contains no elements, we say that the set is the \textbf{empty set}. 
%\label{sym:empty}%
%\index{empty set}%
%  For example, if the universal set is the set of rational numbers $\mathbb{Q}$, then the truth set of the predicate $x^2 - 3x - 7 = 0$ is the empty set.  This means that the statement  
%$\left( {\exists x \in \mathbb{Q}} \right)\left( {x^2  - 3x - 7 = 0} \right)$ is false.  In mathematics, the empty set is usually designated by the symbol  $\emptyset $.  We usually read the symbol  $\emptyset $ as ``the empty set'' or ``the null set.''  (The symbol  $\emptyset $ is actually the last letter in the Danish-Norwegian alphabet.)
%\hbreak
\subsection*{Forms of Quantified Statements in English}
There are many ways to write statements involving quantifiers in English.  In some cases, the quantifiers are not apparent, and this often happens with conditional statements.  The following examples illustrate these points.  Each example contains a quantified statement written in symbolic form followed by several ways to write the statement in English.
\begin{enumerate}
  \item $\left( {\forall x \in \mathbb{R}} \right)\left( {x^2  > 0} \right)$.
  \begin{itemize}
    %\item For any real number  $x$,  $x^2  > 0$.
    \item For each real number  $x$, $x^2  > 0$.
    \item The square of every real number is greater than 0.
    \item The square of a real number is greater than 0.
    \item If  $x \in \mathbb{R}$, then  $x^2  > 0$.
  \end{itemize}
In the second to the last example, the quantifier is not stated explicitly.  Care must be taken when reading this because it really does say the same thing as the previous examples.
The last example illustrates the fact that conditional statements often contain a ``hidden'' universal quantifier.  

If the universal set is  $\R$, then the truth set of the predicate  $x^2  > 0$ is the set of all nonzero real numbers.  That is, the truth set is
\[
\left\{ {x \in \mathbb{R}} \mid x \ne 0 \right\}.
\]
So the preceding statements are false.  For the conditional statement, the example using  
$x = 0$ produces a true hypothesis and a false conclusion.  This is a \textbf{counterexample}
\index{counterexample}%
\label{D:counterexample}%
 that shows that the statement with a universal quantifier is false.

%\pagebreak
\item $\left( {\exists x \in \mathbb{R}} \right)\left( {x^2  = 5} \right)$.
  \begin{itemize}
    \item There exists a real number  $x$  such that  $x^2  = 5$.
    \item $x^2  = 5$ for some real number $x$.
    \item There is a real number whose square equals 5.
  \end{itemize}

The second example is usually not used since it is not considered good writing practice to start a sentence with a mathematical symbol. 

If the universal set is  $\R$, then the truth set of the predicate  ``$x^2  = 5$''  is  
$\left\{ { - \sqrt 5 ,\;\sqrt 5 } \right\}$.  So these are all true statements.
\end{enumerate}
\hbreak


\endinput

